\documentclass[12pt]{article}                   % Začátek dokumentu
\usepackage{../../MFFStyle}                     % Import stylu

\begin{document}

\begin{priklad}[7.7]
	Uvažujme aritmetický afinní prostor $®R^4$ se zaměřením $®R^4$. Definujme jeho podprostory
	$$ D_1 = \text{AO}\{\begin{pmatrix} 4 \\ -4 \\ 6 \\ 2 \end{pmatrix}, \begin{pmatrix} 5 \\ -1 \\ 9 \\ 3 \end{pmatrix}, \begin{pmatrix} 6 \\ -3 \\ 7 \\ -1 \end{pmatrix}\}, $$
	$$ D_2 = \begin{pmatrix} 1 \\ -12 \\ 9 \\ 9 \end{pmatrix} + \LO\{\begin{pmatrix} 1 \\ 3 \\ -2 \\ 0 \end{pmatrix}, \begin{pmatrix} 2 \\ 2 \\ -1 \\ -4 \end{pmatrix}\}. $$
	Pro oba podprostory určete jejich dimenzi a nalezněte jejich rovnicové vyjádření. Určete jejich vzájemnou polohu a v případě různoběžnosti i jejich průsečík.

	\begin{reseni}[Dimenze a rovnice]
		Jako první přepíšeme $D_1$ do „parametrického“ tvaru: $D_1 = $
		$$ \begin{pmatrix} 4 \\ -4 \\ 6 \\ 2 \end{pmatrix} + \LO\{\begin{pmatrix} 5 \\ -1 \\ 9 \\ 3 \end{pmatrix} - \begin{pmatrix} 4 \\ -4 \\ 6 \\ 2 \end{pmatrix}, \begin{pmatrix} 6 \\ -3 \\ 7 \\ -1 \end{pmatrix} - \begin{pmatrix} 4 \\ -4 \\ 6 \\ 2 \end{pmatrix}\} = \begin{pmatrix} 4 \\ -4 \\ 6 \\ 2 \end{pmatrix} + \LO\{\begin{pmatrix} 1 \\ 3 \\ 3 \\ 1 \end{pmatrix}, \begin{pmatrix} 2 \\ 1 \\ 1 \\ -3 \end{pmatrix}\}, $$

		Zřejmě jsou v „$\LO$“ v obou případech nezávislé vektory (buď na první pohled vidíme, že jsou nezávislé, nebo spočítáme determinant prvních dvou souřadnic). Tedy $\dim D_1 = \dim D_2 = 2$. Jsou to tedy ($®R^2$) roviny v $®R^4$, tudíž jsou rovnicově vyjádřeny jako hodnoty skalárního součinu souřadnic s nezávislými „normálovými“ vektory.

		Vektory kolmé na roviny spočítáme jednoduše jako vektory kolmé na „$\LO$“, tedy řešení rovnic
		$$ \(\begin{array}{c c c c | c} 1 & 3 & 3 & 1 & 0 \\ 2 & 1 & 1 & -3 & 0 \end{array}\) \sim \(\begin{array}{c c c c | c} 0 & 1 & 1 & 1 & 0 \\ 2 & 1 & 1 & -3 & 0 \end{array}\), \qquad \(\begin{array}{c c c c | c} 1 & 3 & -2 & 0 & 0 \\ 2 & 2 & -1 & -4 & 0 \end{array}\), $$
		Tedy rovnice budou například (můžu zvolit libovolná dvě nezávislá řešení u obou rovnic)
		$$\begin{array}{c c c c c c c c c c c c}
			D_1:& 0 &+& y &-& z &+& 0 &+& c_1 &=& 0\\
			   &-2x &+& y &+& 0 &-& u &+& c_2 &=& 0\\
			D_2:& 2x &+& 2y &+& 4z &+& 1u &+& c_3 &=& 0\\
			   & 8x &+& 0 &+& 4z &+& 3u &+& c_4 &=& 0
		\end{array}$$
		Když dosadíme z obou roviny do jejich rovnic bod, který máme, vyjde nám $c_1 = 10$, $c_2 = 14$, $c_3 = 23$ a $-71$, tedy:
		$$\begin{array}{c c c c c c c c c c c c}
			D_1:& 0 &+& y &-& z &+& 0 &+& 10 &=& 0\\
			   &-2x &+& y &+& 0 &-& u &+& 14 &=& 0\\
			D_2:& 2x &+& 2y &+& 4z &+& 1u &+& 23 &=& 0\\
			   & 8x &+& 0 &+& 4z &+& 3u &-& 71 &=& 0
		\end{array}$$
	\end{reseni}

	\begin{reseni}[Vzájemná poloha]
		Pro vzájemnou polohu nejdříve najdeme průsečík. Řešíme tedy soustavu rovnic:
		$$ \(\begin{array}{c c c c | c} 0 & 1 & -1 & 0 & -10 \\ -2 & 1 & 0 & -1 & -14 \\ 2 & 2 & 4 & 1 & -23 \\ 8 & 0 & 4 & 3 & 71 \end{array}\) \sim \(\begin{array}{c c c c | c} 0 & 1 & -1 & 0 & -10 \\ 0 & 3 & 4 & 0 & -37 \\ 0 & 8 & 12 & 1 & -163 \\ 8 & 0 & 4 & 3 & 71 \end{array}\). $$
		Průsečík je tedy pouze bod $(33, -11, -1, -63)^T$. Tedy roviny jsou různoběžné.
	\end{reseni}
\end{priklad}

\end{document}
