\documentclass[12pt]{article}                   % Začátek dokumentu
\usepackage{../../MFFStyle}                     % Import stylu

\begin{document}

\begin{priklad}[1.7]
	Nalezněte všechny shodnosti v $®R^2$, které zobrazují přímku $3x + 4y −5 = 0$ na osu „$x$“ a bod $[2, 1]$ na některý bod osy „$y$“.

	\begin{reseni}
		Z přednášky víme, že všechny shodnosti v $®R^2$ lze napsat jako ($¦z \in ®R^2$):
		$$ ¦z \mapsto A¦z + ¦b = \begin{pmatrix} a_{11} & a_{12} \\ a_{21} & a_{22} \end{pmatrix}¦z + \begin{pmatrix} b_1 \\ b_2 \end{pmatrix}. $$
		Z první podmínky (zobrazení bodů $\[x, -\frac{3}{4}x + \frac{5}{4}\]$ na osu „$x$“, tj. na body $[…, 0]$):
		$$ a_{21}x + a_{22}\left(- \frac{3}{4}x + \frac{5}{4}\right) + b_2 = 0. $$
		Tedy (protože rovnice musí platit pro všechna $x \in ®R$, tedy toto víme např. z rovnosti polynomů).
		$$ a_{21} = \frac{3}{4}a_{22}, \qquad b_2 = -\frac{5}{4}a_{22}. $$

		Z druhé podmínky (zobrazení bodu $[2, 1]$ na osu „$y$“, tj. na bod $[0, …]$):
		$$ 2a_{11} + 1a_{12} + b_1 = 0. $$
		Tudíž lze pro libovolné $a_{11}$ a $a_{12}$ vybrat $b_1$ (konkrétně $b_1 = -2a_{11} - a_{12}$), aby byla druhá podmínka splněna.

		Zbývá ještě jedna podmínka, která nám vznikla během řešení – matice musí být ortogonální (aby to bylo shodné zobrazení), tj.
		$$ A\cdot A^T = I \Leftrightarrow a_{11}^2 + a_{12}^2 = 1 = a_{21}^2 + a_{22}^2 \land a_{21}a_{11} + a_{12}a_{22} = 0. $$

		Nyní můžeme dosadit $a_{21} = \frac{3}{4}a_{22}$. Dostaneme $1 = \(\frac{9}{16} + 1\)a_{22}^2$, tj. $a_{22} = ±\frac{4}{5}$ a $a_{21} = ±\frac{3}{5}$. Z poslední rovnice pak máme $±\frac{3}{5} a_{11} = \mp \frac{4}{5}a_{12}$. Tj. $\(1 + \frac{9}{16}\)a_{11}^2 = 1$ a $a_{11} = ±\frac{4}{5}$, $a_{12} = \mp\frac{3}{5}$. Tudíž máme 4 zobrazení:
		$$ ¦z \mapsto \begin{pmatrix} +\frac{4}{5} & -\frac{3}{5} \\ +\frac{3}{5} & +\frac{4}{5} \end{pmatrix} ¦z + \begin{pmatrix} -1 \\ -1 \end{pmatrix}, \qquad ¦z \mapsto \begin{pmatrix} +\frac{4}{5} & -\frac{3}{5} \\ -\frac{3}{5} & -\frac{4}{5} \end{pmatrix} ¦z + \begin{pmatrix} -1 \\ +1 \end{pmatrix}, $$
		$$ ¦z \mapsto \begin{pmatrix} -\frac{4}{5} & +\frac{3}{5} \\ +\frac{3}{5} & +\frac{4}{5} \end{pmatrix} ¦z + \begin{pmatrix} +1 \\ -1 \end{pmatrix}, \qquad ¦z \mapsto \begin{pmatrix} -\frac{4}{5} & +\frac{3}{5} \\ -\frac{3}{5} & -\frac{4}{5} \end{pmatrix} ¦z + \begin{pmatrix} +1 \\ +1 \end{pmatrix}. $$
	\end{reseni}
\end{priklad}

\end{document}
