\documentclass[12pt]{article}					% Začátek dokumentu
\usepackage{../../MFFStyle}					    % Import stylu

\begin{document}

% 29. 09. 2021

\section*{Organizační úvod}
	TODO!!!

\section*{Úvod}
	TODO!!!

% 06. 10. 2021

\section{Shodná zobrazení}
\begin{definice}
	Zobrazení $f: ®R^n \rightarrow ®R^n$ se nazývá shodné (nebo shodnost), jestliže pro každé dva body $¦X, ¦Y \in ®R^n$ platí $||f(¦X) - f(¦Y)|| = ||¦X - ¦Y||$.
\end{definice}

\begin{lemma}
	Přímo z definice plyne, že složení dvou shodností je shodnost, shodnosti jsou prostá zobrazení a inverzní zobrazení ke shodnosti je opět shodnost.

	\begin{dukazin}
		Triviální.
	\end{dukazin}
\end{lemma}

TODO!!!

\begin{definice}[Grupa]
	Množinu s jedinou binární operací $(M, \circ)$ nazveme grupou, jestliže je tato operace asociativní, existuje pro ní neutrální (jednotkový) prvek a ke každému prvku existuje prvek inverzní.
\end{definice}

\begin{dusledek}[Grupa shodností]
	Shodnosti jsou surjektivní a vzhledem ke skládání tvoří grupu, kterou budeme označovat $®E(n)$.

	\begin{dukazin}
		Shodná zobrazení jsou tvaru $f(¦X) = ¦A·¦X + ¦p$, $g(¦X) = ¦B·¦X + ¦q$, kde ¦A a ¦B jsou ortogonální. Potom
		$$ f^{-1} = ¦A^{-1}·¦X - ¦A^{-1}·¦p, (g \circ f)(¦X) = (¦B·¦A)·¦X + ¦B·¦p + ¦q. $$
	\end{dukazin}
\end{dusledek}

\begin{definice}[Přímé zobrazení]
	Zobrazení $f$ nazveme přímé, jestliže $\det ¦A = 1$, a nepřímé, jestliže $\det ¦A = -1$. Přímá zobrazení tvoří podgrupu $®E_+(n)$. Zobrazení, pro která je $¦A$ jednotková matice nazýváme posunutí a tvoří podgrupu označovanou (pokud nehrozí nedorozumění) rovněž $®R^n$. Zobrazení, pro která je ¦p nulový vektor tvoří podgrupu označovanou $®{ON}(n)$ (ortonormální grupa).

	\begin{dukazin}
		To, že jsou to podgrupy se dokáže jednoduše přes uzavřenosti.
	\end{dukazin}

	\begin{poznamka}
		Shodná zobrazení můžeme vyjádřit jako kartézský součin, ale grupové operace by pak nefungovali. Proto je množina shodných zobrazení definovaná tzv. semidirektním součinem: $\{(¦A, ¦p)\} = ®{ON} \ltimes ®R^n$.
	\end{poznamka}
\end{definice}

\begin{veta}
	Pro každou shodnost $f \in ®E(n)$ tvaru $f(¦X) = ¦A·¦X + ¦p$ platí maticová rovnost zapsaná blokově jako
	$$ \binom{f(¦X)}{1} = \begin{pmatrix} ¦A & ¦p \\ 0 & 1 \end{pmatrix}\binom{¦X}{1}. $$

	Navíc zobrazení, které každé shodnosti přiřazuje tuto matici rozměrů $(n+1) \times (n+1)$, tj.
	$$ f \mapsto \begin{pmatrix} ¦A & ¦p \\ 0 & 1 \end{pmatrix},  $$
	je vnoření grupy $®E(n)$ do grupy regulárních matic $®{GL}(n+1)$.

	\begin{dukazin}
		Plyne z maticového násobení.
	\end{dukazin}
\end{veta}

\begin{definice}[Asociovaný homomorfismus, samodružné směry, samodružné body]
	Mějme shodné zobrazení $f(¦X) = ¦A·¦X + ¦p$. Jeho body splňující $f(¦X) = ¦X$ nazýváme samodružné body. Lineární zobrazení $f_{¦A}: ®R^n \rightarrow ®R^n$ dané maticí ¦A nazýváme asociovaným homomorfismem k zobrazení $f$ a vlastní směry tohoto zobrazení nazýváme samodružné směry zobrazení $f$.

	Řekneme, že množina $M$ je samodružná množina zobrazení $f$, jestliže ji zobrazení zachovává (jako celek, ne nutně každý její bod zvlášť). Přesněji jestliže platí
	$$ \forall ¦X \in ®R^n: ¦X \in ®M \implies f(¦X) \in M. $$
\end{definice}

\begin{lemma}
	Přímka $p: C + <¦v>$ je samodružnou množinou shodnosti $f$ právě tehdy, když $<¦v>$ je jeho samodružný směr a $f(C) - C$ je násobkem ¦v.

	\begin{dukazin}
		Ať $¦D = ¦C + ¦v$. Z linearity je $p$ samodružná právě tehdy, když $f(¦C), f(¦D) \in p$. To už dokážeme rozepsáním.
	\end{dukazin}
\end{lemma}

\begin{veta}[Klasifikační věta v $®R^2$]
	Pro každou shodnost $f \in ®E(2)$ nastane právě jedna z těchto možností:

	f je přímá shodnost a
	\begin{itemize}
		\item má všechny body samodružné a všechny směry samodružné s vlastním číslem 1. Pak jde o identitu.
		\item má právě jeden samodružný bod, pak ji nazýváme otočení. Samodružné směry pak nemá buď žádné, nebo všechny s vlastním číslem -1. Tedy jde libovolné otočení nebo o otočení o $\pi$ (= středová souměrnost).
		\item nemá žádný samodružný bod a všechny směry jsou samodružné s vlastním číslem 1. Pak ji nazýváme posunutí.
	\end{itemize}

	f je nepřímá shodnost. Pak má právě dva samodružné směry, jeden s vlastním číslem 1 a jeden s vlastím číslem −1 a
	\begin{itemize}
		\item buď má právě jednu přímku samodružných bodů, pak ji nazýváme osová souměrnost,
		\item nebo nemá žádné samodružné body, pak ji nazýváme posunutá osová souměrnost.
	\end{itemize}
\end{veta}

% 13. 10. 2021

\begin{definice}[Kvaterniony]
	Kvaterniony jsou algebra nad $®R^4$, kde kanonická báze se značí $1, i, j, k$ a $i^2=j^2=k^2 = -1$, $ij=-ji=k$, $jk=-kj=i$, $ki=-ik=j$.
\end{definice}

TODO kvaterniony.

% 13. 10. 2021

TODO!!!
\section{Křivky}
TODO!!!

% 27. 10. 2021

\begin{definice}[Reparametrizace, změna parametru]
	Je-li $c: I \rightarrow ®R^n$ regulární parametrická křivka a $\phi: \tilde{I} \rightarrow I$ hladký difeomorfismus intervalu $\tilde{I}$ na $I$ (tedy hladká bijekce s hladkým inverzním zobrazením), je $\tilde{c} = c \circ \phi: \tilde{I} \rightarrow ®R^n$ regulární parametrická křivka se stejným obrazem jako $c$. Difeomorfismus $\phi$ pak nazýváme změnou parametru a $\tilde{c}$ reparametrizací $c$. Je-li navíc $\phi' > 0$ na $\tilde{I}$, nazveme $\tilde{c}$ reparametrizací $c$ zachovávající orientaci.
\end{definice}

\begin{definice}[Křivka, orientovaná křivka]
	Býti reparametrizací je relace ekvivalence na množině všech regulárních parametrizovaných křivek a každou její třídu nazýváme křivka. Každého zástupce příslušné třídy ekvivalence nazýváme parametrizací této křivky. Býti reparametrizací zachovávající orientaci je rovněž relace ekvivalence na množině všech regulárních parametrizovaných křivek a každou její třídu nazýváme orientovaná křivka.

	\begin{dukazin}
		Reflexivita: $c \circ \id = c$ (difeomorfismus s kladnou derivací). Symetrie: $\tilde{c} = c \circ \phi \Leftrightarrow c = \tilde{c} \circ \phi^{-1}$ (pokud je $\phi' > 0$, pak $\(\phi^{-1}\)' > 0$). Tranzitivita:
		$$ \tilde{\tilde{c}} = \tilde{c} \circ \phi \land \tilde{c} = c \circ \psi \implies \tilde{\tilde{c}} = x \circ \psi \circ \phi. $$
		(Navíc $(\psi \circ \phi)' = \psi'·\phi'$).
	\end{dukazin}
\end{definice}

\begin{poznamka}
	Dále budeme reparametrizaci označovat pouze změnou parametru a tečka bude značit derivaci podle $t$, čárkou pak podle $s$.
\end{poznamka}

\begin{lemma}
	Pro derivace dvou parametrizací $c(t)$ a $c(s) = c(\phi(s))$ téže hladké regulární křivky v každém odpovídajícím bodě platí
	$$ (\dot{c}|\ddot{c}|\dddot{c}) = (c'|c''|c''') \begin{pmatrix} \dot{\phi} & \ddot{\phi} & \dddot{\phi} \\ 0 & \dot{\phi}^2 & 3\dot{\phi}\ddot{\phi} \\ 0 & 0 & \dot{\phi}^3 \end{pmatrix}. $$

	\begin{dukazin}
		$$ \dot c = \frac{dc}{dt} · \frac{d\phi}{ds} = c' · \dot\phi \qquad \(\c{dc}{dt}|_{\phi(s_0)} · \frac{d\phi}{ds}|_{s_0}\). $$
		$$ \ddot c = \frac{d}{ds}\(c' · \dot\phi\) = c'·\ddot\phi + \frac{dc'}{ds}·\dot \phi = c'\ddot\phi + c''(\dot\phi)^2. $$
		$$ \dddot c = \frac{d}{ds}\(c'\ddot\phi + c''(\dot\phi)^2\) = c'·\dddot\phi + c''\dot\phi\ddot\phi + c''·2\dot\phi\ddot\phi + c'''·(\dot\phi)^3. $$
	\end{dukazin}
\end{lemma}

	\subsection{Rovinné křivky}
	\begin{definice}[Tečný vektor, orientovaný jednotkový normálový vektor, znaménková křivost]
		V každém bodě hladké regulární parametrické křivky $c(t)$ v $®R^2$ definujeme jednotkový tečný vektor
		$$ ¦t(t) = \frac{c'(t)}{||c'(t)||}, $$
		dále orientovaný jednotkový normálový vektor
		$$ ¦n_*(t) = \begin{pmatrix} 0 & -1 \\ 1 & 0 \end{pmatrix}¦t(t), $$
		a znaménkovou křivost
		$$ \kappa_z(t) = \frac{\det(c'(t)|c''(t))}{||c'(t)||^3}. $$

		Bod, kde je znaménková křivost nulová, nazýváme inflexní.
	\end{definice}

	\begin{veta}
		Při reparametrizaci křivky v $®R^2$ zachovávající orientaci se v daném bodě tečný vektor, orientovaný normálový vektor a znaménková křivost nemění. Při reparametrizaci, která mění orientaci se tyto vektory mění na opačné a znaménková křivost pouze změní znaménko.

		\begin{dukazin}
			Jen dosadíme z minulého lemmatu.
		\end{dukazin}
	\end{veta}

	\begin{veta}
		Znaménková křivost, tečný a normálový vektor jsou ekvivariantní vůči schodnostem $®R^2$. Přesněji, mějme shodnost ve tvaru $f(X) = AX+p$, parametrickou křivku $c(t)$ a v jejím libovolném bodě veličiny $\kappa_2$, $t$, $n_*$. Pak křivka $\tilde c = f(c(t)) = Ac(t) + p$ má v odpovídajícím bodě znaménkovou křivost $\tilde \kappa_2 = \kappa_2\det A$, normálový vektor $\tilde n_* = An_s\det A$ a tečný vektor $\tilde t = At$.

		\begin{dukazin}
			Rozepsáním.
		\end{dukazin}
	\end{veta}

	\begin{veta}
		Křivka má v každém svém bodě kontakt nejvyššího řádu s tečnou přímkou (ze všech přímek) a v každém neinflexním bodě s oskulační kružnicí (ze všech rovnic).
	\end{veta}

	\begin{veta}
		Pro hladkou regulární parametrickou křivku $c: I \rightarrow ®R^2$ platí
		$$ t'(t) = ||c'(t)||\kappa_2(t)n_*(t). $$

		Dále platí, že existuje hladká funkce $\theta(t): I \rightarrow ®R$ splňující $t(t) = (\cos\theta(t), \sin\theta(t))$ pro $t \in I$ a pro znaménkovou křivost pak platí
		$$ \kappa_2(t) = \frac{\theta'(t)}{||c'(t)||}, t \in I. $$

		Pokud je tedy křivka parametrizována konstantní jednotkovou rychlostí $||c'(t)|| = 1$, pak je tedy znaménková křivost rychlostí změny směru křivky.

		\begin{dukazin}
			$$ t(t)' = \(\frac{c'}{||c'||}\)' = \frac{\sqrt{c'·c'}c'' - \frac{1}{2}\frac{2c'·c''}{\sqrt{c'·c'}}c'}{c'·c'} = \frac{||c'||^2c'' - (c'·c'')c'}{||c'||^3}. $$
			Dokážeme, že $(¦t' \perp ¦t) \Leftrightarrow (¦t' \perp c')$.
			$$ ¦t'·c' = \frac{1}{||c'||^3} \[\(||c'||^2c'' - (c'·c'')c'\)·c'\] = ||c'||^2(c''·c') - (c'·c'')(c'·c') = 0, $$
			tedy existuje $K \in ®R$, že $¦t' = K·¦n_*$.
			$$ \det(¦t, ¦t') = \det(¦t, K¦n_*) = K\det(¦t, ¦n_*) = K. $$
			$$ \det(¦t, ¦t') = \det(\frac{c'}{||c'||}, \frac{||c'||^2·c'' - (c'·c'')c'}{||c'||^3}) = \frac{1}{||c'||^4}\det(c', ||c'||^2·c'') + 0 = $$
			$$ = \frac{\det(c', c'')}{||c'||^3}||c'|| = \kappa_2·||c'||. $$

			K důkazu úhlu bychom potřebovali komplexku. (Pro důkaz existence.)
		\end{dukazin}
	\end{veta}

	\subsection{Křivky v prostoru}
	\begin{definice}[Jednotkový tečný vektor, křivost, binormála, jednotkový normálový vektor, torze]
		Jednotkový tečný vektor bude totožný, křivost (tentokrát není znaménková) je totožná, jen místo determinantu je velikost vektorového součinu. Dále definujeme binormálu (normovaný vektorový součin první a druhé derivace), normála je pak vektorový součin binormály a tečny.

		Torze je $\tau(t) = \frac{\det(c'(t)|c''(t)|c'''(t))}{||c'(t)\times c''(t)||^2}$.
	\end{definice}

	\begin{definice}[Oskulační, retrifikační a normálová rovina]
		Oskulační rovina je množina $¦c(t) + <¦t(t), ¦n(t)>$, retrifikační rovina je množina $¦c(t) + <¦t(t), ¦b(t)>$ a normálová rovina je $¦c(t) + <¦n(t), ¦b(t)>$.
	\end{definice}

% 03. 11. 2021

	\begin{veta}
		Na otevřeném intervalu $I$ budiž zadány dvě hladké reálné funkce $k(t)$, $r(t)$, přičemž $r(t) > 0$ pro všechna $t \in I$. Pak existuje až na přímou podobnost právě jedna hladká parametrická rovinná křivka $c(t)$, $t \in I$, pro kterou platí
		$$ ||c'(t)|| = r(t), \qquad \kappa_z(t) = f(t). $$

		\begin{dukazin}
			Zintegrujeme a vyjde nám jedna funkce až na konstanty.
		\end{dukazin}
	\end{veta}

	\begin{veta}
		Při reparametrizaci křivky v $®R^3$ zachovávající orientaci se v daném bodě křivost, torze a Frenetův repér nemění. Při reparametrizaci, která mění orientaci, se křivost, torze a normálový vektor rovněž nemění, zatímco tečný a binormálový vektor se mění na vektory opačné.

		\begin{dukazin}
			Prostě se spočítá.
		\end{dukazin}
	\end{veta}

	\begin{veta}
		Křivost, tečný a normálový vektor jsou ekvivariantní vůči shodnostem $®R^3$. TODO

		\begin{dukazin}
			Prostě se spočítá.
		\end{dukazin}
	\end{veta}

	\begin{definice}[Tečna]
		Pro hladkou regulární křivku $c(t)$ v $®R^3$ definujeme v každém bodě tečnou přímku jako množinu $c(t) + \<¦t(t)\>$.
	\end{definice}

	\begin{definice}[a lemma o parametrizaci obloukem]
		O hladké parametrizované křivce $c(ť)$ řekneme, že je parametrizovaná obloukem nebo jednotkovou rychlostí, jestliže pro všechna $t \in I$ platí $||c'(t)|| = 1$. Každou hladkou regulární křivku lze parametrizovat obloukem. Je-li $c(t)$ nějaké parametrizace obloukem, pak všechny ostatní parametrizace této křivky obloukem získáme reparametrizací $t = \phi(s)$, $\phi(s) = ±s + s_0$, kde $s_0$ je libovolná konstanta.
		
		\begin{dukazin}
			Zřejmé.
		\end{dukazin}
	\end{definice}

	\begin{lemma}
		Pro hladkou křivku $c(t)$ v $®R^3$ parametrizovanou obloukem v každém bodě platí $¦t(t) = c'(t)$ a v každém neinflexním bodě navíc platí
		$$ n(t) = \frac{c''(t)}{||c''(t)||}, \qquad \kappa(t) = ||c''(t)|| = ||c'(t) \times c''(t)||. $$

		\begin{dukazin}
			Prostě se spočítá.
		\end{dukazin}
	\end{lemma}

	\begin{veta}[Frenetovy vzorce]
		Je-li $c(t)$ hladká křivka v $®R^3$ parametrizovaná obloukem, pak v každém neinflexním bodě platí
		$$ ¦t' = \kappa n, \qquad n' = -\kappa ¦t + \tau b, \qquad b' = -\tau n', $$
		což lze vyjádřit maticově jako
		$$ (t'|n'|b') = (t|n|b)\begin{pmatrix} 0 & -\kappa & 0 \\ \kappa & 0 & -\tau \\ 0 & \tau & 0 \end{pmatrix}, $$
		nebo s využitím takzvaného Darbouxova vektoru $d = \tau¦t + \kappa b$ jako
		$$ t' = d\times t, \qquad n' = d \times n, \qquad b' = d\times b. $$
	\end{veta}

	\begin{veta}
		Nechť $f(t) > 0$, $g(t)$ jsou hladké funkce definované na otevřeném intervalu $I$. Pak existuje až na přímou eukleidovskou shodnost právě jedna hladká křivka $c(t)$ v $®R^3$ parametrizovaná obloukem na intervalu $I$ tak, že $\kappa(t) = f(t)$, $\tau(t) = g(t)$.

		Tyto rovnice se někdy nazývají přirozené rovnice křivky.
	\end{veta}

	\begin{veta}
		Pro regulární hladkou para TODO!
	\end{veta}

% 10. 11. 2021

	TODO

	\begin{veta}
		Křivka je určena křivostí, torzí a počáteční polohou Frenetova repéru.

		\begin{dukazin}
			Technický. Chce to věty o řešení diferenciálních rovnic.
		\end{dukazin}
	\end{veta}

	\begin{veta}
		Pro regulární hladkou parametrizovanou křivku $c: I \rightarrow ®R^3$ bez inflexních bodů platí, že leží v nějaké rovině právě tehdy, když $\tau(t) = 0$, $\forall t \in I$.

		\begin{dukazin}
			$c(t)$ leží v rovině $a·x + b·y + c·z + d = 0$ $\Leftrightarrow$ $c(t)·(a, b, c) = -d$, tj. $c'·(a, b, c) = c''·(a, b, c) = c'''·(a, b, c) = 0$, tedy $c', c'', c''' \in \<(a, b, c)\>^\perp$, který je dimenze dva, tedy determinant daných vektorů (a tedy torze) musí být 0.

			Naopak jestliže $\tau = 0$, pak $b' = \tau·¦n = 0$ $\implies$ $b$ je konstantní. Zvolíme $t_0 \in I$ a definujeme $h(t) = (c(t) - c(t_0))·b$. $h(t_0) = 0$ a navíc $h'(t) = c'(t)·b = ¦t·¦b = 0 \implies h ≡ 0$. $0 = (c(t) - c(t_0))·b = c(t)·b = c(t_0)·b$.
		\end{dukazin}
	\end{veta}

	\begin{veta}
		Pro regulární hladkou parametrizovanou křivku $c: I \rightarrow ®R^2$ vnořenou do $®R^3$ zobrazením $(x, y) \mapsto (x, y, 0)$ platí $\kappa = |\kappa_z|$ a v neinflexních bodech $n = \sgn(\kappa_z)n_s$.

		\begin{dukazin}
			Přímým výpočtem.
		\end{dukazin}
	\end{veta}

\section{Křivkový integrál}
\begin{definice}[Křivkový integrál 1. druhu]
	Mějme hladkou parametrickou křivku $c(t), t \in (\alpha, \beta)$ v $®R^n$ a reálnou funkci $f$ definovanou na $\<c\>$. Pak definujeme křivkový integrál 1. druhu
	$$ \int_c f ds := \int_\alpha^\beta f(c(t))||c'(t)||dt, $$
	pokud integrál napravo existuje jako Lebesgueův integrál.
\end{definice}

\begin{veta}
	Křivkový integrál prvního druhu nezávisí na (re)parametrizaci.

	\begin{dukazin}
		Větou o substituci.
	\end{dukazin}
\end{veta}

\begin{definice}[Délka křivky]
	Délku křivky definujeme jako integrál prvního druhu z konstantní jednotkové funkce
	$$ l(c) = \int_c 1 ds. $$
\end{definice}

\begin{definice}[Uzavřená, jednoduchá, Jordanova křivka]
	Parametrizovaná křivka $c: [\alpha, \beta] \rightarrow ®R^2$ se nazývá uzavřená, jestliže $c(\alpha) = c(\beta)$. Tuto křivku navíc nazveme jednoduchou, je-li $c$ prosté na $[\alpha, \beta)$. Jednoduchá uzavřená rovinná křivka se rovněž nazývá Jordanova.
\end{definice}

\begin{veta}[Umlaufsatz]
	Je-li $c(t), t \in [\alpha, \beta]$ hladká uzavřená křivka, pro kterou navíc $¦t(\alpha) = ¦t(\beta)$, pak existuje $k \in ®Z$ (nazývané index křivky) takové, že $\int_c \kappa_z ds = 2k\pi$.

	Je-li navíc $c$ jednoduchá a kladně orientovaná (proti směru hodinových ručiček), pak $k = 1$.

	\begin{dukazin}
		TODO
	\end{dukazin}
\end{veta}

\begin{definice}
	Mějme hladkou parametrickou křivku $c(t)$, $t \in (\alpha, \beta)$ v $®R^n$ a zobrazení (vektorové pole) $F: \<c\> \rightarrow ®R^n$. Pak definujeme Křivkový integrál 2. druhu
	$$ \int_c FdX := \int_\alpha^\beta F(c(t))·c'(t)dt, $$
	pokud integrál napravo existuje jako Lebesgueův integrál.
\end{definice}

\begin{veta}[Greenova]
	Nechť $c$ je jednoduchá hladká uzavřená kladně orientovaná (proti směru hodinových ručiček) křivka v $®R^2$. Nechť $F(x, y) = (F_1(x, y), F_2(x, y))$ je hladké vektorové pole definované na nějakém okolí uzávěru $\Int c$. Pak
	$$ \int_c FdX = \int_{\Int c} \(\frac{\partial F_2}{\partial x} - \frac{\partial F_1}{\partial y}\)dxdy. $$

	\begin{dukazin}
		
	\end{dukazin}
\end{veta}

\begin{lemma}
	Buď $c(t) = (c_x(t), c_y(t))^T$, $t \in [\alpha, \beta]$ kladně orientovaná hladká jednoduchá uzavřená křivka. Pak plošný obsah obsah oblasti $\Int c$ je roven
	$$ A = \int_\alpha^\beta c_x(t)c'_y(t) dt = - \int_\alpha^\beta c_y(t)c'_x(t) dt = \frac{1}{2} \int_\alpha^\beta (c_xc'_y - c'_xc_y)(t)dt. $$

	\begin{dukazin}
		Dosadíme správná vektorová pole do Greenovy věty a spočítáme.
	\end{dukazin}
\end{lemma}

\begin{veta}[Isoperimetrická nerovnost]
	Buď $c: [a, b] \rightarrow ®R^2$ hladká jednoduchá uzavřená křivka délky $l$ a buď $A$ plošný obsah $\Int c$. Pak
	$$ \frac{l^2}{4\pi} ≥ A, $$
	přičemž rovnost nastane, právě když $c$ je kružnice.
\end{veta}

\begin{lemma}[Wirtinger]
	Nechť $f(t): [0, \pi] \rightarrow ®R$ je hladká funkce, pro kterou platí $f(0) = f(\pi) = 0$. Pak
	$$ \int_0^\pi f'^2(t)dt ≥ \int_0^\pi f^2(t)dt, $$
	TOOD.

	\begin{dukazin}
		Vynechán.
	\end{dukazin}
\end{lemma}

\end{document}
