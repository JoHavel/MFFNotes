\documentclass[12pt]{article}					% Začátek dokumentu
\usepackage{../../MFFStyle}					    % Import stylu



\begin{document}

% 29. 09. 2021

\section*{Organizační úvod}
	TODO!!!
\section*{Úvod}
	TODO!!!

% 06. 10. 2021

\begin{definice}
	Zúplnění míry $\lambda^n_B$ nazveme Lebesgueovou mírou v $®R^n$.
\end{definice}

\begin{poznamka}
	\begin{enumerate}
		\item Lebesgueova míra je $\sigma$-konečná.
		\item Množinu $©B_0\left(®R^n\right) := \sigma(©B(®R^n)\cup©N)$ nazýváme $\sigma$-algebrou lebesgueovsky měřitelných množin. Platí $©B\left(®R^n\right)  \subsetneq ©B_0\left(®R^n\right) \subsetneq ©P\left(®R^n\right)$.
		\item Lebesgueova míra je regulární v následujícím smyslu:
			$$ \forall E \in ©B_0\left(®R^n\right) \forall \epsilon > 0\ \exists \text{otevřená množina } G\ \exists \text{uzavřená množina } F: F \subset E \subset G \land \mu(G \setminus F) < \epsilon. $$
	\end{enumerate}
\end{poznamka}

\begin{definice}[Značení]
	Nechť $X, Y$ jsou množiny a $f: X \rightarrow Y$. Je-li $©S \subset ©P(Y)$, pak $f^{-1}(©S) := \{f^{-1}(S) | S \in ©S\}$.
\end{definice}

\begin{veta}[O zobrazení $f: X \rightarrow Y$]
	Nechť $X, Y$jsou množiny a $f: X \rightarrow Y$.

	\begin{enumerate}
		\item Je-li ©M $\sigma$-algebra na $Y$, pak $f^{-1}(©M)$ je $\sigma$-algebra na $X$.
		\item Je-li $©S \subset ©P(Y)$, pak $f^{-1}(\sigma(©S)) = \sigma\left(f^{-1}(©S)\right)$.
	\end{enumerate}

	\begin{dukazin}
		Později.
	\end{dukazin}
\end{veta}

\section{Měřitelná zobrazení}
\begin{definice}[Měřitelné zobrazení]
	Nechť $(X, ©A)$, $(Y, ©M)$ jsou měřitelné prostory. Zobrazení $f: X \rightarrow Y$ nazveme měřitelným (vzhledem k ©A a ©M), jestliže $f^{-1}(©M) \subset ©A$.

	Jestliže některý z prostorů $X, Y$ je metrický prostor, pak za příslušnou $\sigma$-algebru bereme $\sigma$-algebru borelovských podmnožin (pokud není řečeno jinak).

	Měřitelné zobrazení mezi dvěma metrickými prostory se nazývá borelovsky měřitelné (krátce borelovské).
\end{definice}

\begin{poznamka}
	\begin{enumerate}
		\item Snadno se ověří, e kompozice dvou měřitelných zobrazení je měřitelné zobrazení.
		\item Z věty O zobrazení… plyne, že jsou-li $(X, ©A)$, $(Y, ©M)$ měřitelné prostory, pak zobrazení $f: X \rightarrow Y$ je měřitelné právě tehdy, když $f^{-1}(©S) \subset ©A$, kde $©S \subset ©P(Y)$ je generátor $\sigma$-algebry ©M. Speciálně je-li $(X, ©A)$ a $Y$ metrický prostor, pak zobrazení $f: X \rightarrow Y$ je měřitelné $\Leftrightarrow$ $f^{-1}(G) \in ©A\ \forall \text{otevřenou množinu }G \subset Y$.
	\end{enumerate}
\end{poznamka}

\begin{dusledek}
	Každé spojité zobrazení mezi dvěma metrickými prostory je měřitelné (borelovské).

	\begin{dukazin}
		Z věty O zobrazení… (vzory otevřených množin při spojitém zobrazení jsou otevřené množiny).
	\end{dukazin}
\end{dusledek}

\begin{veta}[Generátory $©B^n := ©B\left(®R^n\right) $]
	Borelovská $\sigma$-algebra $©B^n$ je generována
	
	\begin{enumerate}
		\item otevřenými intervaly $\(a_1, b_1\)\times … \times \(a_n, b_n\)$, kde $-∞ < a_i < b_i < +∞$,
		\item systémem $©S := \{(-∞, a_1) \times … \times (-∞, a_n)\}$, kde $a_i \in ®R$.
	\end{enumerate}
\end{veta}

\begin{veta}[O měřitelných zobrazeních]
	Nechť $(X, ©A)$ je měřitelný prostor.

	\begin{enumerate}
		\item Jsou-li $f: X \rightarrow ®R^n$ a $g: X \rightarrow ®R^n$ měřitelná zobrazení, pak zobrazení $(f, g): X \rightarrow ®R^{n+m}$ je měřitelné.

		\item Jsou-li $f, g: X \rightarrow ®R^n$ měřitelná zobrazení, pak zobrazení $f ± g$ jsou měřitelná zobrazení.

		\item Jsou-li $f, g: X \rightarrow ®R$ měřitelné funkce, pak také $f·g, \max(f, g), \min(f, g)$ jsou měřitelné.
	\end{enumerate}
\end{veta}

\begin{poznamka}
	Prostor $®R^*$ je metrický prostor s metrikou např. $\rho^*$ danou předpisem $\rho^*(x, y) = |\phi(x) - \phi(y)|$, kde $\phi(x) := \frac{x}{1 + |x|}$ pro konečné $x$ a $\phi(±∞) = ±1$ (tzv. redukovaná metrika).

	Redukovaná metrika má následující vlastnosti (viz Jarník – Diferenciální počet 2, str. 245, 246):

	\begin{enumerate}
		\item V množině ®R je ekvivalentní s eukleidovskou metrikou.
		\item Konvergence v prostoru $\(®R^*, \rho^*\)$ splývá s konvergencí zavedenou v $®R^*$ pomocí okolí bodů.
	\end{enumerate}

	Platí $©B^* := ©B\left(®R^*\right) = \sigma\left(\{\<-∞, a\) | a \in ®R\}\right)$. Plyne z:
	
	\begin{enumerate}
		\item $\forall$ otevřenou množinu $G \subset ®R^*$ lze psát jako spočetné sjednocení intervalů typu $\<-∞, a\), (a, b), \(b, ∞\>$.
		\item $\<-∞, a\)$ je stejný jako v $®R^*$.
		\item $\(a, +∞\>$ je $®R^* \setminus \<-∞, a\)$.
		\item $\(a, +∞\> = \bigcup_{n \in ®N}\left<a + \frac{1}{n}, +∞\right>$.
		\item $(a, b) = \<-∞, b\) \cap \(a, +∞\>$.
	\end{enumerate}
\end{poznamka}

\begin{veta}[O měřitelných funkcích]
	Buď $(X, ©A)$ měřitelný prostor. Pak platí

	\begin{enumerate}
		\item $f: (X, ©A) \rightarrow ®R$ je měřitelná funkce právě tehdy, když $f^{-1}\left((-∞, a)\right) \in ©A, \forall a \in ®R$.
		\item $f: (X, ©A) \rightarrow ®R^*$ je měřitelná funkce právě tehdy, když $f^{-1}\left(\<-∞, a\)\right) \in ©A, \forall a \in ®R$.
	\end{enumerate}
\end{veta}

\begin{dusledek}
	Nechť $f, g: (X, ©A) \rightarrow ®R^*$ jsou měřitelné funkce. Pak

	\begin{enumerate}
		\item množiny $\{x \in X | f(x) < g(x)\}, \{f ≤ g\}, \{f = g\}$ jsou měřitelné.
		\item funkce $\max(f, g), \min(f, g)$ jsou měřitelné funkce.
	\end{enumerate}
\end{dusledek}

\begin{veta}[O měřitelných funkcích podruhé]
	Jsou-li funkce $\(f_n\)_{n=1}^∞$ množiny $(X, ©A)$ do $®R^*$ měřitelné funkce, pak funkce $\sup_{n \in ®N} f_n, \inf_{n \in ®N} f_n, \limsup_{n \in ®N} f_n, \liminf_{n \in ®N} f_n$ jsou měřitelné.
\end{veta}



\begin{definice}[Jednoduchá funkce]
	Funkce $S: X \rightarrow \[0, +∞\)$ se nazývá jednoduchá, jestliže množina $S(X)$ je konečná.

	Platí, že $s(x) = \sum_{\alpha \in S(X)} \alpha·\chi_{S = \alpha}$. Součet na pravé straně této rovnosti nazveme kanonickým vyjádřením jednoduché funkce.
\end{definice}

\section{Abstraktní Lebesgueův integrál}

\begin{veta}[O nezáporné měřitelné funkci]
	Nechť $f: (X, ©A) \rightarrow \<0, +∞\>$ je měřitelná funkce. Pak existuje posloupnost jednoduchých (nezáporných) měřitelných funkcí $\{s_n\}_{n \in ®N}$ tak, že $s_n \nearrow f$ (konverguje nahoru).

	Jestliže navíc $f$ je omezená, pak $s_n \rightrightarrows f$.
\end{veta}

\begin{definice}
	Nechť $(X, ©A, \mu)$ je prostor s mírou.

	\begin{enumerate}
		\item Je-li $s: (X, ©A) \rightarrow [0, +∞)$ jednoduchá měřitelná funkce, zapíšeme ji v kanonickém tvaru $s = \sum_{j=1}^k \alpha_j \chi_{E_j}$ a definujeme
			$$ \int_X s d\mu = \int_X s(x) d \mu(x) := \sum_{j=1}^k \alpha_j \mu (E_j). $$
		\item Je-li $f: (X, ©A) \rightarrow \[0, +∞\]$ měřitelná funkce, pak definujeme
			$$ \int_X f d\mu = \sup \{\int_X s d\mu | 0 ≤ s ≤ f \land s \text{ je jednoduchá}\}. $$
		\item Je-li $f: (X, ©A) \rightarrow ®R*$, pak definujeme
			$$ \int_X f d\mu = \int_X f^+ d\mu - \int_X f^- d\mu, \text{ má li pravá strana smysl}.$$
	\end{enumerate}
\end{definice}

% 13. 10. 2021

\begin{poznamka}
	Je-li $(X, ©A, \mu)$ prostor s mírou a $f, g$ jsou nezáporné měřitelné funkce na $X$ splňující $0 ≤ f < g$ na $X$, pak $0 ≤ \int_X fd\mu ≤ \int_X gd\mu$.

	Je-li $(X, ©A, \mu)$ prostor s mírou a $E \in ©A$, pak $©A_E := \{A \cap E, A \in ©A\}$ je $\sigma$-algebra na $E$ a $(E, ©A_E, \mu)$ je prostor s mírou ($\implies \int_E f d\mu$ je definován).

	Je-li $f$ měřitelná funkce na $X$ a $E \in ©A$, pak $\int_X (f\chi_E) d\mu = \int_E f d\mu$.
\end{poznamka}

\begin{veta}[Leviho]
	Je-li $(X, ©A, \mu)$ prostor s mírou a $f_n, n \in ®N$, jsou nezáporné měřitelné funkce na $X$ splňující $f_n\nearrow f$, pak $\int_X f_nd\mu \nearrow \int_X fd\mu$.
	
	\begin{dukazin}
		Později.
	\end{dukazin}
\end{veta}

\begin{veta}[Fatouovo lemma]
	Je-li $(X, ©A, \mu)$ prostor s mírou a $f_n, n \in ®N$, jsou nezáporné měřitelné funkce, pak
	$$ \int_X (\liminf_{n\rightarrow ∞} f_n)d\mu ≤ \liminf_{n \rightarrow ∞}\int_X f_n d\mu. $$
	
	\begin{dukazin}
		Později.
	\end{dukazin}
\end{veta}

\begin{definice}[Skoro všude]
	Buď $(X, ©A, \mu)$ prostor s mírou, $E \in ©A$, $x \in X$. Nechť $V(x)$ je nějaká vlastnost, kterou bod $x$ může, ale nemusí mít. Řekneme, že $V(x)$ platí $\mu$-skoro všude na $E$, jestliže
	$$ \exists N \in ©A, N \subset E, \mu(N) = 0: V(x) \text{ platí } \forall x \in E \setminus N. $$
	Je-li $E = X$, pak místo $\mu$-skoro všude na $E$, píšeme pouze $\mu$-skoro všude. Nehrozí li nedorozumění, o jakou míru se jedná, pak místo $\mu$-skoro všude píšeme skoro všude.
\end{definice}

\begin{lemma}
	Buď $(X, ©A, \mu)$ prostor s mírou a $f, g$ měřitelné funkce na $X$ takové, že $f = g$ skoro všude, pak $\int_X fd\mu = \int_X g d\mu$, jakmile má jedna strana rovnosti smysl.
\end{lemma}

\begin{definice}[Měřitelná funkce (skoro všude)]
	Buď $(X, ©A, \mu)$ prostor s mírou, $D \in ©A$, $\mu(D^c) = 0$ a $f: D \rightarrow ®R^*$. Řekneme, že $f$ je měřitelná, jestliže $\forall$ otevřenou množinu $G \subset ®R$ platí $f^{-1}(G) \cap D \in ©A$.

	Pro měřitelnou funkci $f$ pak definujeme $\int_X f d\mu := \int_X \tilde{f} d\mu$, kde $\tilde{f} = \begin{cases} f \text{ na } D, \\ 0 \text{ na } D^c. \end{cases}$
\end{definice}

\begin{definice}[Prostory ©L]
	Označíme $©L^*(\mu) := \{f:X \rightarrow ®R^* | f \text{ je měřitelná na } X \land \exists \int_X f d\mu\}$.

	Dále $©L^1(\mu) := \{f \in ©L^*(\mu) | \int_X |f| d\mu \in ®R\}$.
\end{definice}

\begin{veta}[Linearita integrálů]
	Buď $(X, ©A, \mu)$ prostor s mírou, $f, g \in ©L^*(\mu)$ a $\lambda \in ®R$. Pak
	$$ \int_X (\lambda f) d\mu = \lambda\int_X f d\mu, $$
	$$ \int_X (f+g) d\mu = \int_X f d\mu + \int_X g d\mu, \text{ pokud má pravá strana smysl.} $$

	\begin{dukazin}
		Později.
	\end{dukazin}

	\begin{poznamkain}
		Má-li pravá strana druhého bodu smysl, pak nemůže nastat případ, kdy by jedna funkcí $f, g$ je rovna $+∞$ a druhá $-∞$ na množině kladné míry. Odtud plyne, že součet $f+g$ je definován skoro všude.
	\end{poznamkain}
\end{veta}

\begin{dusledek}
	Buď $(X, ©A, \mu)$ prostor s mírou a $f_n$, $n \in ®N$, nezáporné měřitelné funkce. Pak
	$$ \int_X\(\sum_{n=1}^∞ f_n\) d\mu = \sum_{n=1}^∞ \int_X f_n d\mu. $$

	\begin{dukazin}
		Z minulé věty pro libovolné $k \in ®N$ platí $\int_X\(\sum_{n=1}^k f_n\) d\mu = \sum_{n=1}^k \int_X f_n d\mu$. Použitím limitního přechodu pro $k \rightarrow ∞$ a Leviho věty dostaneme příslušnou rovnost.
	\end{dukazin}
\end{dusledek}

\begin{veta}[Zobecněná Leviho]
	Buď $(X, ©A, \mu)$ prostor s mírou a $f_n$, $n \in ®N$, měřitelné funkce na $X$ splňující $f_n \nearrow f$ a $\int_X f_1 > - ∞$. Pak
	$$ \lim_{n \rightarrow ∞} \int_X f_n d\mu = \int_X f d\mu. $$

	\begin{dukazin}
		$g_n = f_n - f_1 ≥ 0$. Z Leviho věty pak snadno plyne tato.
	\end{dukazin}
\end{veta}

\begin{dusledek}
	Buď $(X, ©A, \mu)$ prostor s mírou a $f_n, n \in ®N$, měřitelné funkce splňující $f_n\searrow f$ a $\int_X f_1 < +∞$. Pak též můžeme prohodit limitu a integrál.

	\begin{dukazin}
		Aplikace předchozí věty na $-f_n$.
	\end{dukazin}
\end{dusledek}

\begin{veta}[Lebesgue]
	Je-li $(X, ©A, \mu)$ prostor s mírou a $f_n$, $n \in ®N$, jsou měřitelné funkce takové, že $\lim_{n \rightarrow ∞} f_n = f$ na $X$, a existuje $g \in ©L^1(\mu): |f_n| ≤ g$ skoro všude $\forall n \in ®N$. Pak
	$$ lim_{n \rightarrow ∞} \int_X f_n d\mu = \int_X f d\mu. $$

	\begin{dukazin}
		Později.
	\end{dukazin}
\end{veta}

\begin{dusledek}
	Nechť $(X, ©A, \mu)$ je prostor s mírou a $f_n$, $n \in ®N$, jsou měřitelné funkce na $X$ takové, že $\sum_{n=1}^∞ f_n$ konverguje skoro všude. Jestliže existuje $g \in ©L^1(\mu)$ tak, že $|\sum_{n=1}^k f_n| ≤ g$ skoro všude $\forall k \in ®N$, pak $\sum_{n=1}^∞ f_n \in ©L^1(\mu)$ a platí:
	$$ \sum_{n=1}^∞ \int_X f_n d\mu = \int_X \(\sum_{n=1}^∞ f_n\) d\mu. $$

	\begin{dukazin}
		Aplikace předchozí věty na posloupnost částečných součtů řady $\sum_{n=1}^∞ f_n$.
	\end{dukazin}
\end{dusledek}

\begin{veta}[Další vlastnosti měřitelných funkcí a integrálu]
	Buď $(X, ©A, \mu)$ prostor s mírou.

	\begin{itemize}
		\item Je-li $f$ nezáporná měřitelná funkce na $X$ a $\int_X f d\mu = 0$, pak $f = 0$ skoro všude.
		\item Je-li $f \in ©L^1(\mu)$ a $\int_E f d\mu = 0\ \forall E \in ©A$, pak $f = 0$ skoro všude.
		\item Je-li $f$ měřitelná, pak $\int_X f d\mu \in ®R \Leftrightarrow \int_X |f| d\mu$.
		\item Je-li $f \in ©L^1(\mu)$, pak $|\int_X f d\mu| ≤ \int_X |f| d\mu$.
		\item Je-li $f \in ©L^1(\mu)$, pak f je konečná skoro všude.
	\end{itemize}

	\begin{dukazin}
		Později.
	\end{dukazin}
\end{veta}

	\subsection{Lebesqueův integrál v ®R}
	\begin{poznamka}[Značení]
		Restrikci míry $\lambda^1$ na interval $I \subset ®R$ opět značíme $\lambda^1$.

		Je-li $I = (a, b) \subset ®R$, $a < b$, pak
		$$ \int_a^b f d\lambda^1 := \int_{(a, b)} f d\lambda^1. $$
	\end{poznamka}

	\begin{veta}[Vztah Riemannova a Lebesgueova integrálu]
		Je-li $-∞ < a < b < +∞$ a $f:<a, b> \rightarrow ®R$ taková, že $(R) \int_a^b f$ existuje, pak $\int_a^b f d\mu^1 \in ®R$ a platí
		$$ \int_a^b f d\lambda^1 = (R) \int_a^b f. $$
	\end{veta}

	\begin{veta}[Vztah Newtonova a Lebesgueova integrálu]
		Nechť $-∞ ≤ a < b ≤ +∞$ a $f:<a, b> \rightarrow ®R$ je spojitá a nezáporná. Pak následující tvrzení jsou ekvivalentní:
		
		\begin{itemize}
			\item $(N) \int_a^b$ existuje.
			\item $\int_a^b d\lambda^1 \in ®R$.
		\end{itemize}

		Zároveň pokud je jedna (tj. obě) z těchto podmínek splněna, potom
		$$ \int_a^b f d\lambda^1 = (N) \int_a^b f. $$
	\end{veta}

% 20. 10. 2021

TODO!!!

% 27. 10. 2021

\begin{definice}
	Systém $©D \subset ©P(X)$ nazveme d-systém (nebo Dynkinův systém) na $X$, jestliže
	
	\begin{itemize}
		\item $\O \in ©D$,
		\item $D \in ©D \implies D^c \in ©D$,
		\item $D_n \in ©D\ \forall n \in ®N, D_n \cap D_m\ \forall n≠m \implies \bigcup_n D_n \in ©D$.

	\end{itemize}
\end{definice}

\begin{poznamka}
	Každá $\sigma$-algebra je d-systém.

	D-systém je uzavřený na konečné sjednocení disjunktních množin (jelikož $\O \in ©D$).

	Je-li $A, B \in ©D, A \subset B$, pak $B \setminus A \in ©D$, neboť $B \setminus A = X \setminus ((X \setminus B) \cup A)$.

	Jsou-li $\mu$ a $\nu$ dvě míry na $(X, ©A)$, pak $©D := \{A \in ©A | \mu(A) = \nu(A)\}$ je d-systém.
\end{poznamka}

\begin{veta}[O průniku d-systémů]
	Nechť $©D_\alpha$, $\alpha \in I$, jsou d-systémy na $X$ ($I$ je libovolná množina indexů). Pak $\bigcap_{\alpha \in I}$ je d-systém.

	\begin{dukazin}
		Přenechán čtenáři.
	\end{dukazin}
\end{veta}

\begin{dusledek}
	Je-li $©S \subset ©P(X)$, pak existuje nejmenší d-systém $d©S$ obsahující systém $©S$.
\end{dusledek}

\begin{poznamka}
	Je-li $©S \subset ©P(X)$, pak $d©S \subset \sigma©S$.
\end{poznamka}

\begin{definice}
	Systém $©S \subset ©P(X)$ nazveme $\pi$-systém, jestliže systém ©S je uzavřen na konečné průniky množin z ©S.
\end{definice}

\begin{veta}[O rovnosti $d©S = \sigma©S$]
	Je-li $S \subset ©P(X)$ zároveň $\pi$-systémem, pak $d©S = \sigma©S$.

	\begin{dukazin}
		Využijeme následující 2 tvrzení. $d©S$ je d-systém, tedy z druhého tvrzení $d©S$ je $\pi$-systém. Z prvního tvrzení pak $d©S$ je $\sigma$ algebra, tedy $\sigma©S \subset d©S$. Opačná implikace plyne z poznámky výše, $d©S \subset \sigma©S$, tedy $d©S = \sigma ©S$.
	\end{dukazin}
\end{veta}

\begin{tvrzeni}
	Je-li d-systém ©D na $X$ zároveň $\pi$-systémem, pak ©D je $\sigma$-algebra na $X$.

	\begin{dukazin}
		Ověříme body $\sigma$-algebry.
	\end{dukazin}
\end{tvrzeni}

\begin{tvrzeni}
	Je-li $©S \subset ©P(X)$ $\pi$-systém, pak $d©S$ je $\pi$-systém.

	\begin{dukazin}
		Ověříme, že $©D: \{D \in d©S | D \cap S \in d©S \forall S \in ©S\}$ je d-systém. Zřejmě $©D = d©D$. Nyní buď $D \in d©S$ pevné a definujeme $©D_D := \{E \in ©P(X) | E \cap D \in d©S\}$. O tom dokážeme, že je to d-systém. Následně dokážeme $®S \subset ©D_D$, tedy $D = ©D_D$. Vítězství!
	\end{dukazin}
\end{tvrzeni}

TODO?

% 03. 11. 2021

\begin{veta}[O jednoznačnosti míry]
	Nechť $©S \subset ©P(X)$ je $\pi$-systém a $\mu, \gamma$ jsou dvě míry na $\sigma ©S$ splňující $\mu(S) = \gamma(S)$, $\forall S \in ©S$. Jestliže existují množiny $X_n \in ©S$, $X_n \nearrow X$, $\mu(X_n) < +∞$, $\forall n \in ®N$, pak $\mu = \gamma$ na $\sigma ©S$.

	\begin{dukazin}
		Nejprve předpokládejme, že $\mu(X) < +∞$. Pak definujme systém $©D := \{A \in \sigma ©S | \mu(A) = \gamma(A)\}$. Platí $©S \subset ©D$, tedy $d©S \subset d©D = ©D \subset \sigma ©S$, tedy $©D = \sigma©S$.

		Je-li $\mu(X) = +∞$, pak definujeme $©D_n := \{A \in \sigma ©S | \mu(A \cap X_n) = \gamma(A \cap X_n)\}$, $n \in ®N$. Platí $©D_n$ je $d$-systém $\forall n \in ®N$ (ověř!). $©S \subset ©D_n$, $\forall n \in ®N$, neboť $S \in ©S: \mu(S \cap X_n) = \gamma(S\cap X_n)$. $d©S \subset d©D_n = ©D_n \subset \sigma ©S$, tedy $©D_n = \sigma ©S$, $\forall n \in ®N$.

		Nechť $A \in \sigma ©S$. Pak $\mu(A) = \lim_{n \rightarrow ∞} \mu(A \cap X_n) = \lim_{n \rightarrow ∞} \gamma(A \cap X_n) = \gamma(A)$. Tedy $\mu = \gamma$ na $\sigma ©S$.
	\end{dukazin}
\end{veta}

\section{Součin měr a Fubiniova věta}
\begin{poznamka}[Předpoklady pro další 2 přednášky]
	Nechť $(X, ©A, \mu)$, resp. $(Y, ©B, \gamma)$, je prostor se $\sigma$ konečnou mírou $\mu$, resp. $\gamma$.
\end{poznamka}

\begin{definice}[Měřitelný obdélník, ©O]
	Množinu $A \times B \subset X \times Y$, kde $A \in ©A$, $B \in ©B$, nazveme měřitelným obdélníkem.

	Symbolem ©O označíme systém všech měřitelných obdélníků.
\end{definice}

\begin{definice}[Součinová $\sigma$-algebra]
	Definujeme $\sigma$-algebru $©A \otimes ©B$ předpisem $©A \otimes ©B := \sigma ©O$.

	$\forall E \in ©A \otimes ©B\ \forall x \in X\ \forall y \in Y$ definujeme řezy $E_x$, $E^y$ množiny $E$ takto:
	$$ E_x := \{y \in Y | [x, y] \in E\}, \qquad E^y := \{x \in X | [x, y] \in E\}. $$
\end{definice}

\begin{veta}[O součinové $\sigma$-algebře $©A \otimes ©B$]
	Je-li $E \in ©A \otimes ©B$, tak

	\begin{enumerate}
		\item $\forall x \in X: E_x \in ©B$,
		\item $\forall y \in Y: E^y \in ©A$,
		\item funkce $x \mapsto \gamma(E_x)$ je měřitelná na $(X, ®A)$,
		\item funkce $y \mapsto \mu(E^y)$ je měřitelná na $(Y, B)$.
	\end{enumerate}

	Je-li funkce $f: (X\times Y, ©A \otimes ©B) \rightarrow ®R^*$ měřitelná, pak

	\begin{enumerate}
		\item $\forall x \in X$ je funkce $f_x: y \mapsto f(x, y)$ je měřitelná na $(Y, ©B)$,
		\item $\forall y \in Y$ je funkce $f_y: x \mapsto f(x, y)$ je měřitelná na $(X, ©A)$.
	\end{enumerate}

	\begin{dukazin}[Pouze lichá tvrzení, sudá jsou analogická]
		Definujeme $©E = \{E \in ©A \otimes ©B | E_x \in ©B\}$. Ověříme, že $©E$ je $\sigma$-algebra.

		TODO!!!
	\end{dukazin}
\end{veta}

% 10. 11. 2021

\begin{veta}[Existence a jednoznačnost součinové míry]
	Existuje právě jedna míra $\mu \otimes \nu$ (tzv. součinová míra) na $©A \otimes ©B$ splňující $(\mu \otimes \nu)(A \times B) = \mu(A)·\nu(B)$, $\forall A \in ©A$, $\forall B \in ©B$.

	Pro tuto míru platí
	$$ E \in ©A \otimes ©B \implies (\mu \otimes \nu)(E) = \int_X \nu(E_x) d\mu(x) \qquad \(= \int_Y \mu(E^y)d\nu(y)\). $$

	\begin{dukazin}
		1. Existence: Je-li $E \in ©A \otimes ©B$, pak definujeme $(\mu \otimes \nu)(E) = \int_X \nu(E_x) d\mu(x)$. O té dokážeme, že je mírou a že splňuje předpis v definici.

		2. Jednoznačnost: Nechť $\tau$ je míra na $©A \otimes ©B$, která splňuje $\tau(A \times B) = \mu(A)\nu(B)$, $\forall A \in ©A$, $\forall B \in ©B$, tedy $\tau = \mu \otimes \nu$ na ©O to je $\pi$-systém. Prostory $(X, ©A, \mu)$, $(Y, ©B, \nu)$ jsou prostory s $\sigma$-konečnými mírami. Tj.
		$$ \exists X_n \in ©A\ \forall n in ®N, X_n \nearrow X, \mu(X_n) < +∞\ \forall n \in ®N \land $$
		$$ \land \exists Y_n \in ©B\ \forall n in ®N, Y_n \nearrow Y, \nu(Y_n) < +∞\ \forall n \in ®N \Leftrightarrow $$
		$$ TODO. $$
	\end{dukazin}
\end{veta}

\begin{poznamka}
	Jsou-li $(X, ©A, \mu)$ a $(Y, ©B, \nu)$ prostory s úplnými $\sigma$-konečnými mírami, pak $\mu \otimes \nu$ nemusí být úplná.
\end{poznamka}


\begin{veta}[Fubiniova]
	Pro $\forall f \in ©L^*(\mu \otimes \nu)$ platí
	\begin{enumerate}
		\item $x \mapsto \int_Y f(x, y) d\nu(y)$ je měřitelná na $X$,
		\item $y \mapsto \int_X f(x, y) d\nu(x)$ je měřitelná na $Y$,
		\item $\int_{X \times Y} f d(\mu \otimes \nu) = \int_X\(\int_Y f(x, y)d\nu(y)\) d\mu(x) = \int_Y\(\int_X f(x, y)d\mu(x)\) d\nu(y)$.
	\end{enumerate}

	\begin{dukazin}
		1) $f = \chi_E$, $E \in ©A \otimes ©B$: $\nu(E_x) = \int_Y$ TODO!!!(Dokáže se nejprve pro charakteristickou funkci, pak pro jednoduché nezáporné, nakonec pro všechny.)
	\end{dukazin}
\end{veta}

\begin{poznamka}[Značení]
	Místo $(©A \otimes ©B)_0$ značíme $©A \overset{0}{\otimes} ©B$ (budu značit $©A \otimes_0 ©B$). A místo $(\mu \otimes \nu)_0$ píšeme \ldots (já píšu $©A \otimes_0 \nu$).
\end{poznamka}

\begin{veta}[Fubiniova věta pro zúplnění součinové míry]
	Nechť $(X, ©A, \mu)$, $(Y, ©B, \nu)$ jsou prostory s úplnými $\sigma$-konečnými mírami. Je-li $f \in ©L^*(\mu \otimes_0 \nu)$, pak

	\begin{itemize}
		\item funkce $x \mapsto f(x, y)$ je měřitelná na $X$ pro $\mu$-skoro všechna $y \in Y$,
		\item funkce $y \mapsto f(x, y)$ je měřitelná na $Y$ pro $\mu$-skoro všechna $x \in X$,
		\item funkce $x \mapsto \int_Y f(x, y) d\nu(y)$ je měřitelná na $X$,
		\item funkce $y \mapsto \int_X f(x, y) d\nu(x)$ je měřitelná na $Y$,
		\item $\int_{X \times Y} f d(\mu \otimes_0 \nu) = \int_X\(\int_Y f(x, y)d\nu(y)\) d\mu(x) = \int_Y\(\int_X f(x, y)d\mu(x)\) d\nu(y)$.
	\end{itemize}

	\begin{dukazin}
		Z 2 následujících tvrzení a Fubiniovy věty se věta snadno dokáže.
	\end{dukazin}
\end{veta}

\begin{tvrzeni}
	Buď $(Z, ©C, \rho)$ prostor s mírou a $(Z, ©C_0, \rho_0)$ jeho zúplnění. Je-li $f: (Z, ©C_0) \rightarrow ®R^*$ $\rho_0$ měřitelná funkce, pak existuje $\rho$ měřitelná funkce $g: (Z, ©C) \rightarrow ®R^*$ tak, že $f = g$ $\rho$-skoro všude na $Z$.
	
	\begin{dukazin}
		Vynechán.
	\end{dukazin}
\end{tvrzeni}

\begin{tvrzeni}
	Nechť $(X< ©A, \mu)$ a $(Y, ©B, \nu)$ jsou prostory s úplnými $\sigma$-konečnými mírami. Nechť $h: (X \times Y, ©A \otimes_0 ©B) \rightarrow ®R^*$ je $(\mu \otimes_0 \nu)$-měřitelná funkce a $h(x, y) = 0$ $\mu \otimes_0 \nu$-skoro všude na $X \times Y$. Pak pro $\mu$-skoro všechna $x \in X$ je $h(x, y)$ rovno 0 pro $\nu$-skoro všechna $y \in Y$.

	(Tzn, že pro $\mu$-skoro všechna $x \in X$ je funkce $h_x$ rovna 0 $\nu$-skoro všude na $Y$.)

	Speciálně, funkce $h_x$ je měřitelná na $X$ pro $\nu$-skoro všechna $y \in Y$.
	
	\begin{dukazin}
		Vynechán.
	\end{dukazin}
\end{tvrzeni}

\begin{definice}
	$\lambda^n = \(\lambda^*_{©B}\)_0$
\end{definice}

\begin{veta}[O míře $\lambda^p \otimes \lambda^q$]
	Nechť $p, q \in ®N$. Pak
	
	\begin{itemize}
		\item $©B(®R^{p + q}) = ©B(®R^p) \otimes ©B(®R^q)$,
		\item $\lambda^{p+q} = \lambda^p \otimes \lambda^q$.
	\end{itemize}

	\begin{dukazin}
		Neuveden.
	\end{dukazin}
\end{veta}

\begin{veta}[Fubiniova věta pro $\lambda^{p + q}$]
	Nechť $f \in ©L^*(\lambda^{p + q})$, $p, q \in ®N$. Pak
	$$ \int_{®R^{p + q}} f d\lambda^{p + q} = \int_{®R^p} \(\int_{®R^q} f(x, y) d\lambda^q(y)\) \lambda^p(x) = \int_{®R^q} \(\int_{®R^p} f(x, y) d\lambda^p(x)\) \lambda^q(y). $$
\end{veta}

\begin{definice}[Značení]
	$p, q \in ®N$, $x \in ®R^p$, $y \in ®R^q$. Definujeme projekce
	$$ \pi_1(x, y) = x, \qquad \pi_2(x, y) = y. $$
\end{definice}

\begin{dusledek}
	Nechť $p, q \in ®N$, $A \in ©B^{p + q}_0 := ©B(®R^{p + q})_0$. Je-li $f \in ©L^*(\lambda^{p + q})$ a projekce $\pi_1A, \pi_2A$ jsou měřitelné, pak
	$$ \int_A f d\lambda^{p + q} \int_{\pi_1A} \(f(x, y) d \lambda^q(y)\)\lambda^p(x) = \int_{\pi_2A} \(f(x, y) d \lambda^p(x)\)\lambda^q(y). $$
\end{dusledek}

\begin{poznamka}[Značení]
	Místo $d\lambda^p(x)$ píšeme $dx$ a místo $d\lambda^q(y)$ píšeme $dy$.
\end{poznamka}

% 24. 11. 2021

\begin{lemma}
	Lebesgueova míra $\lambda^n$ je translačně invariantní (tzn. $\lambda^n(B + r) = \lambda^n(B)$).

	\begin{dukazin}
		$\lambda^n$ a $\mu(B) := \lambda^n(B + r)$, $\forall B \in ©B_0^n$, $r \in ®R^n$, jsou míry, které se shodují na systémech otevřených intervalů v $®R^n$. Ty spolu s prázdnou množinou tvoří $\pi$-systém, takže se míry shodují i na Borelovských množinách $\implies$ jsou shodné.
	\end{dukazin}
\end{lemma}

\begin{veta}[O obrazu míry]
	Nechť $(X, ©A, \mu)$ je prostor s mírou a $(Y, ©B)$ je měřitelný prostor. Buď $\phi: (X, ©A) \rightarrow (Y, ©B)$ měřitelné zobrazení. Pak množinová funkce $\phi(\mu)$ daná předpisem
	$$ \phi(\mu)(B) = \mu(\phi^{-1}(B)), \forall B \in ©B $$
	je míra na $Y$ (nazýváme ji obraz míry $\mu$ při zobrazení $\phi$) a platí (má-li alespoň jedna strana smysl):
	$$ \int_Y f d\phi(\mu) = \int_X f(\phi(x)) d\mu(x). $$

	\begin{dukazin}
		Ověří se, že je to míra (bod po bodu). Rovnost integrálů pak postupně ověříme na charakteristické funkce, pro jednoduché funkce, pro „jednoznaménkové“ (jako monotónní limity jednoduchých) a potom pro všechny (jako součty kladných a záporných funkcí).

		Pro charakteristické funkce:
		$$ \int_X f(\phi(x)) d\mu(x) = \int)X \chi_B(phi(x)) d\mu = \int_X \chi_{\phi^{-1}(B)}(x) = d\mu(x) = \mu(\phi^{-1}(B)) = $$
		$$ \phi(\mu)(B) = \int_Y \chi_B d\phi(\mu) = \int_Y f d\phi(\mu). $$
	\end{dukazin}
\end{veta}

\begin{veta}
	Nechť $L: ®R^n \rightarrow ®R^n$ je invertibilní lineární zobrazení.

	1. Je-li $\nu(A) := \lambda^n(L(A))$, $\forall A \in ©B^n := ©B(®R^n)$, pak $\nu$ je míra a platí $\nu = |\det L|\lambda_{©B}^n$.

	2. Je-li $\mu := |\det L| \lambda_{©B}^n$, pak $L \mu = \lambda_{©B}^n$ a $\forall f \in ©L^*(\lambda_{©B}^n)$ platí
	$$ \int_{®R^n} f d\lambda^n = \int_{®R^n} (f\circ L) |\det L| d\lambda_{©B}^n. $$

	\begin{dukazin}
		1. $L$ je lineární zobrazení z $®R^n$ do $®R^n$, a tedy $L$ je spojité. $L$ je invertibilní $\implies \exists$inverzní zobrazení $L^{-1}$, které je opět lineární a spojité. Tedy $L$ je měřitelné.
		$$ (L^{-1}\lambda^n)(A) = \lambda^n(L(A)) = \nu(A), \forall A \in ©B^n $$
		$\implies$ $nu$ je míra dle předchozí věty.

		Z lineární algebry je známo, že $L$ lze vyjádřit jako kompozici konečně mnoha „elementárních“ lineárních zobrazení jednoho z následujících typů: $L_1(x_1, …, x_n) = (\alpha x_1, x_2, …, x_n)$, $\forall (x_1, …, x_n) \in ®R^n$, kde $\alpha \in ®R\setminus\{0\}$, $L_2(x_1, …, x_n) = (x_1, …, x_j, …, x_i, …, x_n)$, $\forall (x_1, …, x_n) \in ®R^n$, $j > i \in ®N$, $L_3(x_1, …, x_n) = (x_1 + x_2, x_2, …, x_n)$, $\forall (x_1, …, x_n) \in ®R^n$, $i, j \in ®N$.

		Protože determinant součinu matic se rovná součinu determinantů, stačí tvrzení ověřit pro „elementární“ zobrazení. Ověříme na intervalech, $L_1$ ho jen natáhne o $\alpha$, tedy na determinant násobek, $L_2$ „otočí“ interval, ale $\lambda^n$ se otočením nezmění, $L_3$ posune a zdeformuje interval, ale tím se $\lambda^n$ nezmění (dokážeme přes Fubiniovu větu). Všechny 3 zobrazení stejně operují na prázdné množině, takže i na $\pi$ systému $I \cup \{\O\}$, tedy míry se rovnají všude.

		2.
		$$ (L(\mu))(A) \overset{1.}{=} \mu(L^{-1}(A)) = |\det L| \lambda_{©B}^n(L^{-1}(A)) = |\det L|·|\det L^{-1}| \lambda_{©B}^n(A) = \lambda_{©B}^n(A) \forall A \in ©B^n, $$
		tedy $L(\mu) = \lambda_{©B}^n$. Z předchozí věty pak plyne rovnost integrálů.
	\end{dukazin}

	\begin{lemma}
		Buď $T: ®R^n \rightarrow ®R^n$ zobrazení splňující Li. podmínku (tzn. $\exists C \in <0, +∞): ||T(x) - T(y)|| ≤ C||x - y||, \forall x, y \in ®R^n$). Je-li $A$ $\lambda^n$-měřitelná, pak také $T(A)$ je $\lambda^n$-měřitelná množina.

		\begin{dukazin}
			Bez důkazu (není čas a důkaz je jednoduchý).
		\end{dukazin}
	\end{lemma}
\end{veta}

\begin{veta}
	Je-li $L$ invertibilní zobrazení $®R^n$ do $®R^n$, pak
	$$ \int_{R^n} f d\lambda^n = \int_{®R^n} (f \circ L) |\det L| d\lambda^n, $$
	má-li alespoň jedna strana smysl.
\end{veta}

\begin{tvrzeni}[Opakování]
	Je-li $G \subset ®R^n$ otevřená množina a $T: G \rightarrow ®R^n$ je zobrazení třídy $C^1$ na $G$, pak $Tx - Tx_0$, kde  $x_0 \in G$, lze lokálně aproximovat lineárním zobrazením, jehož matice je $(\frac{\partial T_i}{\partial x_j} (x_0))_{i, j = 1}^n$ a jehož determinant je $\Jac(T)(x_0)$.
\end{tvrzeni}

\begin{lemma}
	$$ \lambda^n(®R^{n-1}) = 0. $$

	\begin{dukazin}
		$®R^{n-1}$ je uzavřená v $®R^n$, tedy je $\lambda$-měřitelná.
		$$ ®R^{n-1} \subset \bigcup_{k=1}^∞ I_{k, \epsilon}, I_{k, \epsilon} = (-k, k)^{n-1} \times \(\frac{-\epsilon}{(2k)^{n-1}}\frac{1}{2^k}, \frac{\epsilon}{(2k)^{n-1}}\frac{1}{2^k}\), $$
		$$ 0 ≤ \lambda^n(®R^{n-1}) ≤ \sum_{k=1}^∞ \lambda^n(I_{k, \epsilon}) = \sum_{k=1}^∞ (2k)^{n-1} \frac{2\epsilon}{(2k)^{n-1}}\frac{1}{2^k} = 2\epsilon \implies \lambda^n(®R^{n-1}) = 0. $$
	\end{dukazin}
\end{lemma}

\veta[O substituci]
	Buď $G \subset ®R^n$ otevřená množina a $\phi: G \rightarrow ®R^n$ difeomorfimsus. Je-li $f: \phi(G) \rightarrow ®R$ $\lambda^n$-měřitelná funkce, pak
	$$ \int_G f(\phi(x)) |\Jac \phi(x)| dx = \int_{\phi(G)} f(y) dy, $$
	má li alespoň jedna strana smysl.

	\begin{dukazin}
		Bez důkazu.
	\end{dukazin}

% 01. 12. 2021
\section{Důkazy}
\begin{dukaz}[Věta 1.4?]
	Označme $\tilde{©A}_0 = \{E \subset X | \exists A, B \in ©A: A \subset E \subset B, \mu(B \setminus A) = 0\}$. Ověříme, že $\tilde{©A}_0$ je $\sigma$-algebra.

	TODO!!!
\end{dukaz}

\begin{lemma}
	Buď $(X, ©A, \mu)$ prostor s mírou a $s$ jednoduchá nezáporná měřitelná funkce na $X$. Definujeme-li
	$$ \phi(A) = \int_A s d\mu, \forall A \in ©A, $$
	pak $\phi$ je míra na ©A.

	\begin{dukazin}
		$s = \sum_{j=1}^k \alpha_j·\chi_{E_j}$ kanonický tvar funkce $s$.
		$$ A \in ©A \implies \phi(A) = \int_A s d\mu = \int_X \chi_A · s d\mu = \int_X \tilde{s} d\mu = \sum_{j=1}^k \alpha_j \mu(A \cap E_j), \forall A \in ©A. $$
		Následně ověříme body definice míry:
		$\phi(\O) = 0$.
		$$ A \in ©A\ \forall i \in ®N, i ≠ j \implies A_i \cap A_j = \O. $$
		$$ \phi(\bigcup_l A_l) = \sum_{j=1}^k \alpha_j \mu((\bigcup_l A_l) \cap E_j) = \sum_{j=1}^k \alpha_l \sum_{i \in ®N} \mu(A_l \cup E_j) = $$
		$$ = \lim_{n \rightarrow ∞} \sum_{j=1}^k \sum_{i=1}^n \alpha_j \mu(A_i \cap E_j) = \lim_{n \rightarrow ∞} \sum_{l=1}^n \sum_{j=1}^k \alpha_l \mu(A_l \cap E_j) = \sum_{l=1}^∞ \phi(A_i). $$
	\end{dukazin}
\end{lemma}

\begin{dukaz}[Leviho věty]
	$$ f_n ≤ f_{n+1} \implies \int_X f_n d\mu ≤ \int_X f_{n+1}d\mu \implies \exists \alpha \in \<0, +∞\>: \int_X f_n d\mu \rightarrow \alpha. $$
	$$ \int_X f_n d\mu ≤ \int_X f d\mu, \forall n \in ®N. $$

	Obrácená nerovnost (pro $s$ jednoduché měřitelné funkce):
	$$ \int_\lambda f d\mu = \sup_{0 ≤ s ≤ f} \int_X s d\mu. $$
	$$ \forall c: c \in (0, 1) \implies f > cs. $$
	
	$E_n = \{x \in X | f_n(x) ≥ c·s(x)\}$, $n \in ®N$:
	$$ E_1 \subset E_2 \subset … \subset X, X = \bigcup_{n \in ®N} E_n, $$
	TODO.
\end{dukaz}

% 08. 12. 2021

\begin{lemma}[O míře s hustotou $f$]
	Buď $f$ nezáporná měřitelná funkce na $(X, ©A, \mu)$ a definujme $\nu(A) := \int_A f d\mu$, $\forall A \in ©A$. Pak $\nu$ je míra na ®A a
	$$ \int_X g d\nu = \int_X g·f d\mu. $$
	Pro každou nezápornou měřitelnou funkci $g$ na $X$.

	\begin{dukazin}
		Je jasné, že $\nu ≥ 0$ a $\nu(\O) = \int_{\O}fd\mu = \int_X \chi_{\O} f d\mu = \int_X 0 d\mu = 0$.

		Buď $A = \bigcup_{j=1}^∞ A_j$, $A_j \in ©A$, $\forall j \in ®N$, $A_i \cap A_j = \O$ ($i ≠ j$).
		$$ \nu(A) = \nu(\bigcup_j A_j) = \int_{\bigcup_j A_j} f d\mu = \int_X \chi_{\bigcup_j A_j} fd\mu = \int_X(\sum_j \chi_{A_j}) f d\mu \overset{\text{Levi}}{=} \sum_j \int_X \chi_{A_j} f d\mu = $$
		$$ = \sum_{j=1}^∞ \nu(A_j). $$

		Rovnost ověříme postupně pro: $g = \chi_E$, $E \in ©A$:
		$$ \int_X g d\nu = \int_X \chi_E d\nu = \int_E d\nu = \nu(E) = \int_E f d\mu = \int_X \chi_E f d\mu = \int_X gf d\mu. $$
		Obdobně pokračujeme i pro další „typy“ funkcí.
	\end{dukazin}
\end{lemma}

\begin{definice}[Hustota míry]
	Funkci $f$ z předchozího lemmatu se říká hustota míry $\nu$ vzhledem k míře $\mu$.
\end{definice}

\begin{definice}[Absolutně spojitá míra]
	Nechť $\mu$, $\nu$ jsou míry na $(X, ©A)$. Řekneme, že míra $\nu$ je absolutně spojitá vzhledem k míře $\mu$ (značíme $\nu \ll \mu$), jestliže platí
	$$ \forall A \in ©A: \mu(A) = 0 \implies \nu(A) = 0. $$
\end{definice}

\begin{veta}
	TODO?
\end{veta}

\begin{veta}[Charakterizace faktu $\nu \ll \mu$ pro konečné míry]
	Nechť $\nu$, $\mu$ jsou konečné míry na $(X, ©A)$. Pak $\nu \ll \mu$ právě tehdy, když
	$$ \forall \epsilon > 0\ \exists \delta\ \forall A \in ©A, \mu(A) < \delta: \nu(A) < \epsilon. $$

	\begin{dukazin}
		„$\impliedby$“: Buď $A \in ©A$, $\mu(A) = 0$. Pro $\epsilon = \frac{1}{k}$ nám podmínka dává $\exists \delta_k > 0$, $\mu(A) < \delta_k \implies \nu(A) = \frac{1}{k}$, tedy $\nu(A) = 0$ (jelikož levá strana předchozí implikace je splněna vždy).

		„$\implies$“: Sporem. Nechť $\nu \ll \mu$ a předpokládejme, že podmínka neplatí, tj.
		$$ \exists \epsilon > 0\ \forall \delta > 0 \exists A \in ©A, \mu(A) < \delta: \nu(A) ≥ \epsilon. $$
		Volíme $\delta = \frac{1}{2^n}$, $n \in ®N$. Tedy $\exists A_n \in ©A$, $\mu(A_n) < \frac{1}{2^n}$ a $\nu(A_n) ≥ \epsilon$. Nechť $B_k := \bigcup_{n = k+1}^∞ A_n$, $k \in ®N$. Tedy $B_1 \supset B_2 \supset …$. Z nějaké věty výše plyne $\mu\(\bigcap_{k=1}^∞ B_k\) = \lim_{k \rightarrow ∞} \mu(B_k)$ a obdobně $\nu\(\bigcap_{k=1}^∞ B_k\) = \lim_{k \rightarrow ∞} \nu(B_k)$.

		$\mu(B_k) \rightarrow 0$, $\nu(B_k) \rightarrow L ≥ \epsilon$. \lightning
	\end{dukazin}
\end{veta}

\begin{lemma}
	Jestliže $\mu, \nu$ jsou konečné míry na $(X, ©A)$ takové, že $\forall A \in ©A: \nu(A) ≤ \mu(A)$. Pak existuje měřitelná funkce $f$ splňující $0 ≤ f ≤ 1$ $\mu$-skoro všude a platí
	$$ \forall A \in ©A: \nu(A) = \int_A f d\mu. $$

	\begin{dukazin}
		Definujeme funkcionál
		$$ Jg := \int_X g^2 d\mu - 2\int_X g d\nu, \qquad \forall g \in ©L^2(\mu). $$
		Definice $J$ je korektní, protože konvergence v $L^2$ je silnější než konvergence v $L^1$ (nebo z Hölderovy nerovnosti), tedy oba integrály jsou pro $g \in L^2$ konečné. Dále definujeme $c := \inf_{g \in L^2(\mu)} Jg$.
		$$ Jg = \int_X g^2 d\mu - 2 \int_X g d\nu ≥ \int_X g^2 d\mu - 2\int_X |g| d\mu = \int_X(|g| - 1)^2 d\mu - \mu(X) ≥ -\mu(X) > -∞, \forall g \in L^2(\mu). $$

		Předpokládejme, že $\exists f\: c = Jf$. Buď $A \in ©A$ pevná množina, definujeme $g(t) := J(f + t\chi_A)$, $\forall t \in ®R$. Tedy $g$ má minimum v bodě $0$. Tudíž $g'(0) = 0$, pokud $g'$ existuje. Ověříme výpočtem z definice existenci a dosadíme 0:
		$$ g'(0) = \lim_{t \rightarrow 0} \frac{g(t) - g(0)}{t} = \lim_{t \rightarrow 0} \frac{J(f + t\chi_A) - J(f)}{t} = \lim_{t \rightarrow 0} \frac{1}{t}\[\int_X (f + t\chi_A)^2 d\mu - 2\int_X(f + t\chi_A) d\nu - \int_X f^2 d\mu + 2\int_X fd\nu\] = $$
		$$ \lim_{t \rightarrow 0} \[\int_X 2f \chi_A d\mu + t\int_X \chi_A d\mu - 2 \int_X \chi_A d\nu\] = 2\[\int_X f \chi_A d\mu - \int_X \chi_A d\nu\] = 0 $$
		Tedy $\forall A \in ©A: \nu(A) = \int_A f d\mu$.

		$$ 0 ≤ \int_{\{f > 1\}}(f - 1)^+ d\mu = \int_{\{f > 1\}} (f - 1) d\mu = \int_{\{f > 1\}} d\mu - \int_{\{f > 1\}} 1d\mu = \nu(\{f > 1\}) - \mu(\{f > 1\}) ≤ 0 \implies f ≤ 1 \mu\text{-skoro všude} $$
		$$ 0 ≤ \int_{\{f < 0\}} f^- d\mu = - \int_{\{f < 0\}} f d\mu = - nu(\{f < 0\}) ≤ 0 \implies f ≥ 0 \mu\text{-skoro všude} $$

% 15. 12. 2021

		$$ J(g) + J(h) - J\(\frac{g + h}{2}\) = \int_X \frac{g^2 - 2gh + h^2}{2} d\mu = \frac{1}{2} \int_X (g - h)^2 d\mu = \frac{1}{2} ||g - h||_{L^2(\mu)}^2. $$
		$\exists \{f_n\} \subset L^2(\mu)$. $J(f_n) \rightarrow c$ pro $n \rightarrow ∞$. $g = f_n$, $h = f_m$:
		$$ J(f_n) + J(f_m) - 2J\(\frac{f_n + f_m}{2}\) = \frac{1}{2} ||f_n - f_m||_{L^2(\mu)}^2, \forall n, m \in ®N $$
		$$ ≤ J(f_n) + J(f_n) - 2c \rightarrow 0 \implies \exists f \in L^2(\mu): f_n \rightarrow f \in L^2(\mu). $$
		$$ \int_X |f_n - f|d\nu ≤ \int_X |f_n - f| d\mu ≤ \(\int_X |f_n - f|^2\)^{\frac{1}{2}}·(\mu(X))^{\frac{1}{2}} \rightarrow 0 \implies $$
		$$ \implies ||f_n - f||_{L^2(M)} \rightarrow 0 \implies J(f_n) \rightarrow J(f). $$
	\end{dukazin}
\end{lemma}

\begin{veta}[Radonova-Nikodymova věta]
	Nechť $\mu$, $\nu$ jsou $\sigma$-konečné míry na $(X, ©A)$ splňující $\nu \ll \mu$. Pak existuje nezáporná měřitelná funkce $f$ na $X$ tak, že
	$$ \nu(A) = \int_A f d\mu, \forall A \in ©A. $$
	Této funkci se říká derivace míry $\nu$ vzhledem k $mu$, nebo také Radonova-Nikodymova hustota…

	\begin{dukazin}
		1. krok: Nejprve předpokládejme, že $\mu, \nu$ jsou konečné míry. Platí $\nu ≤ \mu + \nu$. (Z lemmatu někde výše $\exists h, 0 ≤ h ≤ 1$ $(\mu + \nu)$-skoro všude, že $\nu(A) = \int_A h d(\mu + \nu)$ $\forall A \in ©A$, $ = \int_A h d\mu + \int_A h d\nu = \int_X \xi_A h d\mu + \int_X \xi_A h d\nu$).

		$\int_X \xi_A (1 - h) d\nu = \int_X \xi_A h d\mu$. Z linearity $\int$ + $(h < 1, \text{skoro všude})$
		$$ \int_X g(1 - h) d\nu = \int_X g h d\mu, \qquad \forall g \text{ jednoduchou, nezápornou, měřitelnou funkci na } X. $$

		Volbou $g = \frac{1}{1 - h}\xi_A$, $A \in ©A$, $\nu(A) = \int … = \int \xi_A d\mu$?

		2. krok Nechť $\nu$, $\mu$ jsou $\sigma$-konečné
		$$ X = \bigcup_{i=1}^∞ E_i, \mu(E_i) < ∞, E_i \cap E_j = \O, i ≠ j, $$
		$$ X = \bigcup_{i=1}^∞ F_i, \nu(F_i), F_i \cap F_j = \O, i ≠ j, $$
		$$ D_{ij} = E_i \cap F_j, X = \bigcup_{i, j = 1}^∞ D_{ij}, \nu(D_{ij}) < +∞, \mu(D_{ij}) < +∞, \forall i, j \in ®N $$
		tedy jako v první části důkazu zvolíme $f_{ij}$ měřitelné nezáporné na $D_{ij}$, aby $\nu|_{D_{ij}}(A) = \int_{D_{ij}} f_{ij} d\mu|_{D_{ij}}$. Nyní již $\forall x \in X \exists! i \in ®N\ \exists! j \in ®N: x \in D_{ij} \implies f(x) = f_{ij}(x)$.
	\end{dukazin}
\end{veta}

\begin{definice}[Singulární míra]
	Nechť $\nu, \mu$ jsou míry na měřitelném prostoru $(X, ©A)$. Řekneme, že míra $\nu$ je singulární vzhledem k míře $\mu$ (značení $\nu \perp \mu$), jestliže
	$$ \exists S \in ©AL \mu(S) = 0 \land \nu(X\setminus S) = 0. $$
\end{definice}

\begin{veta}[Lebesgueova dekompozice]
	Buď $(X, ©A)$ měřitelný prostor, $\mu$ míra na $X$, $\nu$ $\sigma$-konečná míra na $X$. Pak existují jednoznačně určené míry $\nu_a$, $\nu_s$ na $X$ tak, že $\nu_a \ll \mu$, $\nu_s \perp \mu$ a $\nu_a + \nu_s = \nu$.

	\begin{dukazin}
		1. krok: Nechť $\nu$ je konečná míra. Pak existence plyne z:
		$$ ©N_\mu := \{B \in ©A | \mu(B) = 0\}, $$
		$$ c := \sup\{\nu(B) | V \in ©N_\mu\} ≤ \nu(X) < ∞ $$
		$$ \exists \{B_j\} \subset ©A, \lim_{j \rightarrow ∞} \nu(B_j) = e $$
		$$ N := \bigcup_{j=1}^∞ B_j \implies \mu(N) ≤ \sum_{j=1}^∞ \mu(B_j) = 0. $$
		$$ \nu_s(A) := \nu(A \cap N) \forall A \in ©A, $$
		$$ \nu(X \setminus N) - \nu(X \setminus N \cap N) = 0. $$
		$$ \nu_a(A) = \nu(A) - \nu_s(A) = \nu(A) - \nu(A \cap N) = \nu(A \setminus N) = \nu(A \cap N) $$.

		$\nu_a \ll \mu$: Buď $A \in ©A$, $\mu(A) = 0$.
		$$ N \cup (A \cap N^c) \implies \mu(N \cup (A \cap N^c)) ≤ \mu(N) + \mu(A \cap N^c) = 0. $$

		Platí $\nu(A \cap N^c) = 0$, neboť (sporem) kdyby $\nu(A \cap N^c) > 0$, pak by
		$$ \nu(N \cup (A \cup N^c)) = \nu(N) + \nu(A \cap N^c) = c + (>0). \text{\lightning}. $$
		$$ \nu(A \cup N^c) = \nu_a(A) $$

		Jednoznačnost: Nechť $\nu = \nu_a + \nu_s$ a $\nu = \tilde \nu_a + \tilde \nu_s$, $\nu_a \ll \mu$, $\tilde \nu_a \ll \mu$, $\nu_s \perp \mu$, $\tilde \nu_s \perp \mu$.
		$$  \implies \exists N \in ©A: \mu(N) = 0 \land \nu_s(N^c) = 0 \land $$
		$$ \land \exists \tilde N \in ©A: \mu(\tilde N) = 0 \land \tilde\nu_s(\tilde N^c) = 0 $$
		$$ \implies \nu_a(N) = 0, \tilde \nu_a(N) = 0, N_0 := N \cup \tilde N \implies \mu(N_0) = 0 $$
		$$ \nu_s(C_0^c) = \nu_s(X \setminus N_0) ≤ \nu_s(X \setminus N) = \nu_s(N^c) = 0 \implies \nu_s(N_0^c). $$

		$$ \nu_s(A) = \nu_s(A \cap N_0) = \nu(A \cap N_0) - \nu_a(A \cap N_0) = \nu(A \cap N_0). $$
		Analogicky
		$$ \tilde \nu_s(A) = \tilde \nu_s(A \cap N_0) = \nu(A \cap N_0) - \tilde \nu_a(A \cap N_0) = \nu(A \cap N_0). $$
		$$ \implies \nu_s = \tilde \nu_s, \nu_a = \tilde \nu_a. $$

		2. krok: Předpokládejme, že $\nu$ je $\sigma$-konečná. $X = \bigcup_{k=1}^∞ D_k$, $D_k \in ©A$, $D_k \cap D_l$, $k ≠ l$, $\nu(D_k) < ∞$. Provedeme první krok na každé množině zvlášť a pak je dáme dohromady.

% 22. 12. 2021

		TODO
	\end{dukazin}
\end{veta}

\begin{definice}[Distribuční funkce]
	Buď $\mu$ konečná borelovská míra na ®R. Pak funkci
	$$ F_\mu := \mu(\(-∞, x\>), \forall x \in ®R, $$
	nazýváme distribuční funkcí míry $\mu$.
\end{definice}

\begin{lemma}[O distribuční funkci]
	Funkce $F_\mu$ splňuje:

	\begin{itemize}
		\item $F_\mu$ je neklesající.
		\item $F_\mu(-∞) = 0$, $F_\mu(+∞) = \mu(®R) < +∞$.
		\item $F_\mu$ je zprava spojitá.
	\end{itemize}

	\begin{dukazin}
		První bod: $x < y$, $x, y \in ®R$. $F_\mu(x) = \mu(\(-∞, x\>) ≤ \mu(\(-∞, y\>) = F_\mu(y)$.

		Druhý bod:
		$$ F_\mu(-∞) = \lim_{n \rightarrow ∞} F_\mu(-n) = \lim_{n \rightarrow ∞} \mu(\(-∞, -n\>) = \mu\(\bigcup_{n \in ®N} \(-∞, n\>\) = \mu(\O) = 0. $$
		$$ F_\mu(+∞) = \lim_{n \rightarrow ∞} F_\mu(n) = \lim_{n \rightarrow ∞} \mu(\(-∞, n\>) = \mu\(\bigcup_{n \in ®N} \(-∞, n\>\) = \mu(®R) < +∞. $$

		Třetí bod: $x \in ®R$
		$$ \lim_{y \rightarrow x_+} F_\mu(y) = \lim_{n \rightarrow +∞} F_\mu\(x + \frac{1}{n}\) = \lim_{n \rightarrow ∞} \mu\(\(-∞, x + \frac{1}{n}\>\) = \mu\(\bigcap_{n \in ®N} \(-∞, x + \frac{1}{n}\>\) = \mu(\(-∞, x\>) = F_\mu(x). $$
	\end{dukazin}
\end{lemma}

\begin{veta}[O Lebesgueově-Stieltjesově míře]
	Nechť $F: ®R \rightarrow ®R$ je funkce splňující
	\begin{itemize}
		\item $F$ je neklesající,
		\item $F_\mu(-∞) = 0$, $F_\mu(+∞) < +∞$,
		\item $F$ je zprava spojitá.
	\end{itemize}
	Pak existuje právě jedna konečná borelovská míra na prostoru ®R, že $F = F_\mu$.

	\begin{dukazin}
		Nedokazovali jsme.
	\end{dukazin}
\end{veta}

\begin{definice}[Lebesgueův-Stieltjesův integrál]
	Je-li $F$ distribuční funkce (konečné borelovské míry $\mu$) na ®R a $A \in ©B(®R)$, pak
	$$ \int_A f dF := \int_A f d\mu,\qquad \text{pokud má pravá strana smysl.} $$
\end{definice}

\begin{veta}[Per partes pro L-S integrál]
	Je-li $F_\mu$ distribuční funkce míry $\mu$ a $G_\nu$ distribuční funkce míry $\nu$, pak platí
	$$ \forall -∞ < a < b < +∞ \in ®R: F(b)G(b) - F(a)G(a) = \int_{\<a, b\>} F(x) dG(x) + \int_{\<a, b\>} G(x) dF(x). $$

	\begin{dukazin}
		$\Omega = \{[x, y] \in ®R^2 | a < x ≤ y ≤ b\}$. Z Fubiniovy věty si spočteme dvěma způsoby:
		$$ (\mu \otimes \nu)(\Omega) = \int_{\(a, b\>} \(\int_{\<x, b\>} dG(y)\)dF(x) = $$
		$$ = \int_{\(a, b\>} (G(b) - G(x)) dF(x) = G(b)(F(b) - F(a)) - \int_{\(a, b\>} G(x) dF(x). $$
		$$ (\mu \otimes \nu)(\Omega) = \int_{\(a, b\>} \(\int_{\<a, y\>} dF(x)\)dG(y) = $$
		$$ = \int_{\(a, b\>} (F(y) - F(a)) dG(y) = F(a)(G(b) - G(a)) - \int_{\(a, b\>} F(y) dG(y) $$
		Odečtením dostáváme dokazovanou rovnost.
	\end{dukazin}
\end{veta}

\begin{lemma}[O míře absolutně spojité k $\lambda^1$]
	Nechť $\mu$ je konečná borelovská míra na ®R. Jestliže $F_\mu \in C^1(®R)$ a $\mu \ll \lambda^1$, pak platí
	$$ \frac{d \mu}{d \lambda^1} = F_\mu'. $$

	\begin{dukazin}
		©S systém množin, který se skládá z $\O$ a všech intervalů typu $\(a, b\>$, kde $-∞ < a < b < +∞$. Pak ©S je $\pi$-systém.

		Buď $\nu$ mír daná předpisem $\nu(A) := \int_A F_\mu' d\lambda^1$, $\forall A \in ©B(®R)$. Pak $\nu = \mu$ na ©S, neboť $\nu(\O) = 0 = \mu(\O)$, $\mu(\(a, b\>) = F_\mu(b) - F_\mu(a) = \int_a^b F'_\mu(x) dx - \int_{\(a, b\>} F'_\mu d\lambda^1 = \mu(\<-u, u\>)$.

		Z nějaké věty dříve plyne $\mu = \nu$ na $\sigma ©S = ©B(®R)$ $\implies$ $\mu(A) = \int_A F'_\mu d\lambda^1$, $\forall A \in ©B(®R)$.
	\end{dukazin}
\end{lemma}

TODO Důkaz Lebesgueovy věty.

\end{document}
