\documentclass[12pt]{article}					% Začátek dokumentu
\usepackage{../../MFFStyle}					    % Import stylu



\begin{document}

% 29. 09. 2021

\section*{Organizační úvod}
	TODO!!!
\section*{Úvod}
	TODO!!!

% 06. 10. 2021

\begin{definice}
	Zúplnění míry $\lambda^n_B$ nazveme Lebesgueovou mírou v $®R^n$.
\end{definice}

\begin{poznamka}
	\begin{enumerate}
		\item Lebesgueova míra je $\sigma$-konečná.
		\item Množinu $©B_0\left(®R^n\right) := \sigma(©B(®R^n)\cup©N)$ nazýváme $\sigma$-algebrou lebesgueovsky měřitelných množin. Platí $©B\left(®R^n\right)  \subsetneq ©B_0\left(®R^n\right) \subsetneq ©P\left(®R^n\right)$.
		\item Lebesgueova míra je regulární v následujícím smyslu:
			$$ \forall E \in ©B_0\left(®R^n\right) \forall \epsilon > 0\ \exists \text{otevřená množina } G\ \exists \text{uzavřená množina } F: F \subset E \subset G \land \mu(G \setminus F) < \epsilon. $$
	\end{enumerate}
\end{poznamka}

\begin{definice}[Značení]
	Nechť $X, Y$ jsou množiny a $f: X \rightarrow Y$. Je-li $©S \subset ©P(Y)$, pak $f^{-1}(©S) := \{f^{-1}(S) | S \in ©S\}$.
\end{definice}

\begin{veta}[O zobrazení $f: X \rightarrow Y$]
	Nechť $X, Y$jsou množiny a $f: X \rightarrow Y$.

	\begin{enumerate}
		\item Je-li ©M $\sigma$-algebra na $Y$, pak $f^{-1}(©M)$ je $\sigma$-algebra na $X$.
		\item Je-li $©S \subset ©P(Y)$, pak $f^{-1}(\sigma(©S)) = \sigma\left(f^{-1}(©S)\right)$.
	\end{enumerate}

	\begin{dukazin}
		Později.
	\end{dukazin}
\end{veta}

\section{Měřitelná zobrazení}
\begin{definice}[Měřitelné zobrazení]
	Nechť $(X, ©A)$, $(Y, ©M)$ jsou měřitelné prostory. Zobrazení $f: X \rightarrow Y$ nazveme měřitelným (vzhledem k ©A a ©M), jestliže $f^{-1}(©M) \subset ©A$.

	Jestliže některý z prostorů $X, Y$ je metrický prostor, pak za příslušnou $\sigma$-algebru bereme $\sigma$-algebru borelovských podmnožin (pokud není řečeno jinak).

	Měřitelné zobrazení mezi dvěma metrickými prostory se nazývá borelovsky měřitelné (krátce borelovské).
\end{definice}

\begin{poznamka}
	\begin{enumerate}
		\item Snadno se ověří, e kompozice dvou měřitelných zobrazení je měřitelné zobrazení.
		\item Z věty O zobrazení… plyne, že jsou-li $(X, ©A)$, $(Y, ©M)$ měřitelné prostory, pak zobrazení $f: X \rightarrow Y$ je měřitelné právě tehdy, když $f^{-1}(©S) \subset ©A$, kde $©S \subset ©P(Y)$ je generátor $\sigma$-algebry ©M. Speciálně je-li $(X, ©A)$ a $Y$ metrický prostor, pak zobrazení $f: X \rightarrow Y$ je měřitelné $\Leftrightarrow$ $f^{-1}(G) \in ©A\ \forall \text{otevřenou množinu }G \subset Y$.
	\end{enumerate}
\end{poznamka}

\begin{dusledek}
	Každé spojité zobrazení mezi dvěma metrickými prostory je měřitelné (borelovské).

	\begin{dukazin}
		Z věty O zobrazení… (vzory otevřených množin při spojitém zobrazení jsou otevřené množiny).
	\end{dukazin}
\end{dusledek}

\begin{veta}[Generátory $©B^n := ©B\left(®R^n\right) $]
	Borelovská $\sigma$-algebra $©B^n$ je generována
	
	\begin{enumerate}
		\item otevřenými intervaly $\(a_1, b_1\)\times … \times \(a_n, b_n\)$, kde $-∞ < a_i < b_i < +∞$,
		\item systémem $©S := \{(-∞, a_1) \times … \times (-∞, a_n)\}$, kde $a_i \in ®R$.
	\end{enumerate}
\end{veta}

\begin{veta}[O měřitelných zobrazeních]
	Nechť $(X, ©A)$ je měřitelný prostor.

	\begin{enumerate}
		\item Jsou-li $f: X \rightarrow ®R^n$ a $g: X \rightarrow ®R^n$ měřitelná zobrazení, pak zobrazení $(f, g): X \rightarrow ®R^{n+m}$ je měřitelné.

		\item Jsou-li $f, g: X \rightarrow ®R^n$ měřitelná zobrazení, pak zobrazení $f ± g$ jsou měřitelná zobrazení.

		\item Jsou-li $f, g: X \rightarrow ®R$ měřitelné funkce, pak také $f·g, \max(f, g), \min(f, g)$ jsou měřitelné.
	\end{enumerate}
\end{veta}

\begin{poznamka}
	Prostor $®R^*$ je metrický prostor s metrikou např. $\rho^*$ danou předpisem $\rho^*(x, y) = |\phi(x) - \phi(y)|$, kde $\phi(x) := \frac{x}{1 + |x|}$ pro konečné $x$ a $\phi(±∞) = ±1$ (tzv. redukovaná metrika).

	Redukovaná metrika má následující vlastnosti (viz Jarník – Diferenciální počet 2, str. 245, 246):

	\begin{enumerate}
		\item V množině ®R je ekvivalentní s eukleidovskou metrikou.
		\item Konvergence v prostoru $\(®R^*, \rho^*\)$ splývá s konvergencí zavedenou v $®R^*$ pomocí okolí bodů.
	\end{enumerate}

	Platí $©B^* := ©B\left(®R^*\right) = \sigma\left(\{\<-∞, a\) | a \in ®R\}\right)$. Plyne z:
	
	\begin{enumerate}
		\item $\forall$ otevřenou množinu $G \subset ®R^*$ lze psát jako spočetné sjednocení intervalů typu $\<-∞, a\), (a, b), \(b, ∞\>$.
		\item $\<-∞, a\)$ je stejný jako v $®R^*$.
		\item $\(a, +∞\>$ je $®R^* \setminus \<-∞, a\)$.
		\item $\(a, +∞\> = \bigcup_{n \in ®N}\left<a + \frac{1}{n}, +∞\right>$.
		\item $(a, b) = \<-∞, b\) \cap \(a, +∞\>$.
	\end{enumerate}
\end{poznamka}

\begin{veta}[O měřitelných funkcích]
	Buď $(X, ©A)$ měřitelný prostor. Pak platí

	\begin{enumerate}
		\item $f: (X, ©A) \rightarrow ®R$ je měřitelná funkce právě tehdy, když $f^{-1}\left((-∞, a)\right) \in ©A, \forall a \in ®R$.
		\item $f: (X, ©A) \rightarrow ®R^*$ je měřitelná funkce právě tehdy, když $f^{-1}\left(\<-∞, a\)\right) \in ©A, \forall a \in ®R$.
	\end{enumerate}
\end{veta}

\begin{dusledek}
	Nechť $f, g: (X, ©A) \rightarrow ®R^*$ jsou měřitelné funkce. Pak

	\begin{enumerate}
		\item množiny $\{x \in X | f(x) < g(x)\}, \{f ≤ g\}, \{f = g\}$ jsou měřitelné.
		\item funkce $\max(f, g), \min(f, g)$ jsou měřitelné funkce.
	\end{enumerate}
\end{dusledek}

\begin{veta}[O měřitelných funkcích podruhé]
	Jsou-li funkce $\(f_n\)_{n=1}^∞$ množiny $(X, ©A)$ do $®R^*$ měřitelné funkce, pak funkce $\sup_{n \in ®N} f_n, \inf_{n \in ®N} f_n, \limsup_{n \in ®N} f_n, \liminf_{n \in ®N} f_n$ jsou měřitelné.
\end{veta}



\begin{definice}[Jednoduchá funkce]
	Funkce $S: X \rightarrow \[0, +∞\)$ se nazývá jednoduchá, jestliže množina $S(X)$ je konečná.

	Platí, že $s(x) = \sum_{\alpha \in S(X)} \alpha·\chi_{S = \alpha}$. Součet na pravé straně této rovnosti nazveme kanonickým vyjádřením jednoduché funkce.
\end{definice}

\section{Abstraktní Lebesgueův integrál}

\begin{veta}[O nezáporné měřitelné funkci]
	Nechť $f: (X, ©A) \rightarrow \<0, +∞\>$ je měřitelná funkce. Pak existuje posloupnost jednoduchých (nezáporných) měřitelných funkcí $\{s_n\}_{n \in ®N}$ tak, že $s_n \nearrow f$ (konverguje nahoru).

	Jestliže navíc $f$ je omezená, pak $s_n \rightrightarrows f$.
\end{veta}

\begin{definice}
	Nechť $(X, ©A, \mu)$ je prostor s mírou.

	\begin{enumerate}
		\item Je-li $s: (X, ©A) \rightarrow [0, +∞)$ jednoduchá měřitelná funkce, zapíšeme ji v kanonickém tvaru $s = \sum_{j=1}^k \alpha_j \chi_{E_j}$ a definujeme
			$$ \int_X s d\mu = \int_X s(x) d \mu(x) := \sum_{j=1}^k \alpha_j \mu (E_j). $$
		\item Je-li $f: (X, ©A) \rightarrow \[0, +∞\]$ měřitelná funkce, pak definujeme
			$$ \int_X f d\mu = \sup \{\int_X s d\mu | 0 ≤ s ≤ f \land s \text{ je jednoduchá}\}. $$
		\item Je-li $f: (X, ©A) \rightarrow ®R*$, pak definujeme
			$$ \int_X f d\mu = \int_X f^+ d\mu - \int_X f^- d\mu, \text{ má li pravá strana smysl}.$$
	\end{enumerate}
\end{definice}

% 13. 10. 2021

\begin{poznamka}
	Je-li $(X, ©A, \mu)$ prostor s mírou a $f, g$ jsou nezáporné měřitelné funkce na $X$ splňující $0 ≤ f < g$ na $X$, pak $0 ≤ \int_X fd\mu ≤ \int_X gd\mu$.

	Je-li $(X, ©A, \mu)$ prostor s mírou a $E \in ©A$, pak $©A_E := \{A \cap E, A \in ©A\}$ je $\sigma$-algebra na $E$ a $(E, ©A_E, \mu)$ je prostor s mírou ($\implies \int_E f d\mu$ je definován).

	Je-li $f$ měřitelná funkce na $X$ a $E \in ©A$, pak $\int_X (f\chi_E) d\mu = \int_E f d\mu$.
\end{poznamka}

\begin{veta}[Leviho]
	Je-li $(X, ©A, \mu)$ prostor s mírou a $f_n, n \in ®N$, jsou nezáporné měřitelné funkce na $X$ splňující $f_n\nearrow f$, pak $\int_X f_nd\mu \nearrow \int_X fd\mu$.
	
	\begin{dukazin}
		Později.
	\end{dukazin}
\end{veta}

\begin{veta}[Fatouovo lemma]
	Je-li $(X, ©A, \mu)$ prostor s mírou a $f_n, n \in ®N$, jsou nezáporné měřitelné funkce, pak
	$$ \int_X (\liminf_{n\rightarrow ∞} f_n)d\mu ≤ \liminf_{n \rightarrow ∞}\int_X f_n d\mu. $$
	
	\begin{dukazin}
		Později.
	\end{dukazin}
\end{veta}

\begin{definice}[Skoro všude]
	Buď $(X, ©A, \mu)$ prostor s mírou, $E \in ©A$, $x \in X$. Nechť $V(x)$ je nějaká vlastnost, kterou bod $x$ může, ale nemusí mít. Řekneme, že $V(x)$ platí $\mu$-skoro všude na $E$, jestliže
	$$ \exists N \in ©A, N \subset E, \mu(N) = 0: V(x) \text{ platí } \forall x \in E \setminus N. $$
	Je-li $E = X$, pak místo $\mu$-skoro všude na $E$, píšeme pouze $\mu$-skoro všude. Nehrozí li nedorozumění, o jakou míru se jedná, pak místo $\mu$-skoro všude píšeme skoro všude.
\end{definice}

\begin{lemma}
	Buď $(X, ©A, \mu)$ prostor s mírou a $f, g$ měřitelné funkce na $X$ takové, že $f = g$ skoro všude, pak $\int_X fd\mu = \int_X g d\mu$, jakmile má jedna strana rovnosti smysl.
\end{lemma}

\begin{definice}[Měřitelná funkce (skoro všude)]
	Buď $(X, ©A, \mu)$ prostor s mírou, $D \in ©A$, $\mu(D^c) = 0$ a $f: D \rightarrow ®R^*$. Řekneme, že $f$ je měřitelná, jestliže $\forall$ otevřenou množinu $G \subset ®R$ platí $f^{-1}(G) \cap D \in ©A$.

	Pro měřitelnou funkci $f$ pak definujeme $\int_X f d\mu := \int_X \tilde{f} d\mu$, kde $\tilde{f} = \begin{cases} f \text{ na } D, \\ 0 \text{ na } D^c. \end{cases}$
\end{definice}

\begin{definice}[Prostory ©L]
	Označíme $©L^*(\mu) := \{f:X \rightarrow ®R^* | f \text{ je měřitelná na } X \land \exists \int_X f d\mu\}$.

	Dále $©L^1(\mu) := \{f \in ©L^*(\mu) | \int_X |f| d\mu \in ®R\}$.
\end{definice}

\begin{veta}[Linearita integrálů]
	Buď $(X, ©A, \mu)$ prostor s mírou, $f, g \in ©L^*(\mu)$ a $\lambda \in ®R$. Pak
	$$ \int_X (\lambda f) d\mu = \lambda\int_X f d\mu, $$
	$$ \int_X (f+g) d\mu = \int_X f d\mu + \int_X g d\mu, \text{ pokud má pravá strana smysl.} $$

	\begin{dukazin}
		Později.
	\end{dukazin}

	\begin{poznamkain}
		Má-li pravá strana druhého bodu smysl, pak nemůže nastat případ, kdy by jedna funkcí $f, g$ je rovna $+∞$ a druhá $-∞$ na množině kladné míry. Odtud plyne, že součet $f+g$ je definován skoro všude.
	\end{poznamkain}
\end{veta}

\begin{dusledek}
	Buď $(X, ©A, \mu)$ prostor s mírou a $f_n$, $n \in ®N$, nezáporné měřitelné funkce. Pak
	$$ \int_X\(\sum_{n=1}^∞ f_n\) d\mu = \sum_{n=1}^∞ \int_X f_n d\mu. $$

	\begin{dukazin}
		Z minulé věty pro libovolné $k \in ®N$ platí $\int_X\(\sum_{n=1}^k f_n\) d\mu = \sum_{n=1}^k \int_X f_n d\mu$. Použitím limitního přechodu pro $k \rightarrow ∞$ a Leviho věty dostaneme příslušnou rovnost.
	\end{dukazin}
\end{dusledek}

\begin{veta}[Zobecněná Leviho]
	Buď $(X, ©A, \mu)$ prostor s mírou a $f_n$, $n \in ®N$, měřitelné funkce na $X$ splňující $f_n \nearrow f$ a $\int_X f_1 > - ∞$. Pak
	$$ \lim_{n \rightarrow ∞} \int_X f_n d\mu = \int_X f d\mu. $$

	\begin{dukazin}
		$g_n = f_n - f_1 ≥ 0$. Z Leviho věty pak snadno plyne tato.
	\end{dukazin}
\end{veta}

\begin{dusledek}
	Buď $(X, ©A, \mu)$ prostor s mírou a $f_n, n \in ®N$, měřitelné funkce splňující $f_n\searrow f$ a $\int_X f_1 < +∞$. Pak též můžeme prohodit limitu a integrál.

	\begin{dukazin}
		Aplikace předchozí věty na $-f_n$.
	\end{dukazin}
\end{dusledek}

\begin{veta}[Lebesgue]
	Je-li $(X, ©A, \mu)$ prostor s mírou a $f_n$, $n \in ®N$, jsou měřitelné funkce takové, že $\lim_{n \rightarrow ∞} f_n = f$ na $X$, a existuje $g \in ©L^1(\mu): |f_n| ≤ g$ skoro všude $\forall n \in ®N$. Pak
	$$ lim_{n \rightarrow ∞} \int_X f_n d\mu = \int_X f d\mu. $$

	\begin{dukazin}
		Později.
	\end{dukazin}
\end{veta}

\begin{dusledek}
	Nechť $(X, ©A, \mu)$ je prostor s mírou a $f_n$, $n \in ®N$, jsou měřitelné funkce na $X$ takové, že $\sum_{n=1}^∞ f_n$ konverguje skoro všude. Jestliže existuje $g \in ©L^1(\mu)$ tak, že $|\sum_{n=1}^k f_n| ≤ g$ skoro všude $\forall k \in ®N$, pak $\sum_{n=1}^∞ f_n \in ©L^1(\mu)$ a platí:
	$$ \sum_{n=1}^∞ \int_X f_n d\mu = \int_X \(\sum_{n=1}^∞ f_n\) d\mu. $$

	\begin{dukazin}
		Aplikace předchozí věty na posloupnost částečných součtů řady $\sum_{n=1}^∞ f_n$.
	\end{dukazin}
\end{dusledek}

\begin{veta}[Další vlastnosti měřitelných funkcí a integrálu]
	Buď $(X, ©A, \mu)$ prostor s mírou.

	\begin{itemize}
		\item Je-li $f$ nezáporná měřitelná funkce na $X$ a $\int_X f d\mu = 0$, pak $f = 0$ skoro všude.
		\item Je-li $f \in ©L^1(\mu)$ a $\int_E f d\mu = 0\ \forall E \in ©A$, pak $f = 0$ skoro všude.
		\item Je-li $f$ měřitelná, pak $\int_X f d\mu \in ®R \Leftrightarrow \int_X |f| d\mu$.
		\item Je-li $f \in ©L^1(\mu)$, pak $|\int_X f d\mu| ≤ \int_X |f| d\mu$.
		\item Je-li $f \in ©L^1(\mu)$, pak f je konečná skoro všude.
	\end{itemize}

	\begin{dukazin}
		Později.
	\end{dukazin}
\end{veta}

	\subsection{Lebesqueův integrál v ®R}
	\begin{poznamka}[Značení]
		Restrikci míry $\lambda^1$ na interval $I \subset ®R$ opět značíme $\lambda^1$.

		Je-li $I = (a, b) \subset ®R$, $a < b$, pak
		$$ \int_a^b f d\lambda^1 := \int_{(a, b)} f d\lambda^1. $$
	\end{poznamka}

	\begin{veta}[Vztah Riemannova a Lebesgueova integrálu]
		Je-li $-∞ < a < b < +∞$ a $f:<a, b> \rightarrow ®R$ taková, že $(R) \int_a^b f$ existuje, pak $\int_a^b f d\mu^1 \in ®R$ a platí
		$$ \int_a^b f d\lambda^1 = (R) \int_a^b f. $$
	\end{veta}

	\begin{veta}[Vztah Newtonova a Lebesgueova integrálu]
		Nechť $-∞ ≤ a < b ≤ +∞$ a $f:<a, b> \rightarrow ®R$ je spojitá a nezáporná. Pak následující tvrzení jsou ekvivalentní:
		
		\begin{itemize}
			\item $(N) \int_a^b$ existuje.
			\item $\int_a^b d\lambda^1 \in ®R$.
		\end{itemize}

		Zároveň pokud je jedna (tj. obě) z těchto podmínek splněna, potom
		$$ \int_a^b f d\lambda^1 = (N) \int_a^b f. $$
	\end{veta}

% 20. 10. 2021

TODO!!!

% 27. 10. 2021

\begin{definice}
	Systém $©D \subset ©P(X)$ nazveme d-systém (nebo Dynkinův systém) na $X$, jestliže
	
	\begin{itemize}
		\item $\O \in ©D$,
		\item $D \in ©D \implies D^c \in ©D$,
		\item $D_n \in ©D\ \forall n \in ®N, D_n \cap D_m\ \forall n≠m \implies \bigcup_n D_n \in ©D$.

	\end{itemize}
\end{definice}

\begin{poznamka}
	Každá $\sigma$-algebra je d-systém.

	D-systém je uzavřený na konečné sjednocení disjunktních množin (jelikož $\O \in ©D$).

	Je-li $A, B \in ©D, A \subset B$, pak $B \setminus A \in ©D$, neboť $B \setminus A = X \setminus ((X \setminus B) \cup A)$.

	Jsou-li $\mu$ a $\nu$ dvě míry na $(X, ©A)$, pak $©D := \{A \in ©A | \mu(A) = \nu(A)\}$ je d-systém.
\end{poznamka}

\begin{veta}[O průniku d-systémů]
	Nechť $©D_\alpha$, $\alpha \in I$, jsou d-systémy na $X$ ($I$ je libovolná množina indexů). Pak $\bigcap_{\alpha \in I}$ je d-systém.

	\begin{dukazin}
		Přenechán čtenáři.
	\end{dukazin}
\end{veta}

\begin{dusledek}
	Je-li $©S \subset ©P(X)$, pak existuje nejmenší d-systém $d©S$ obsahující systém $©S$.
\end{dusledek}

\begin{poznamka}
	Je-li $©S \subset ©P(X)$, pak $d©S \subset \sigma©S$.
\end{poznamka}

\begin{definice}
	Systém $©S \subset ©P(X)$ nazveme $\pi$-systém, jestliže systém ©S je uzavřen na konečné průniky množin z ©S.
\end{definice}

\begin{veta}[O rovnosti $d©S = \sigma©S$]
	Je-li $S \subset ©P(X)$ zároveň $\pi$-systémem, pak $d©S = \sigma©S$.

	\begin{dukazin}
		Využijeme následující 2 tvrzení. $d©S$ je d-systém, tedy z druhého tvrzení $d©S$ je $\pi$-systém. Z prvního tvrzení pak $d©S$ je $\sigma$ algebra, tedy $\sigma©S \subset d©S$. Opačná implikace plyne z poznámky výše, $d©S \subset \sigma©S$, tedy $d©S = \sigma ©S$.
	\end{dukazin}
\end{veta}

\begin{tvrzeni}
	Je-li d-systém ©D na $X$ zároveň $\pi$-systémem, pak ©D je $\sigma$-algebra na $X$.

	\begin{dukazin}
		Ověříme body $\sigma$-algebry.
	\end{dukazin}
\end{tvrzeni}

\begin{tvrzeni}
	Je-li $©S \subset ©P(X)$ $\pi$-systém, pak $d©S$ je $\pi$-systém.

	\begin{dukazin}
		Ověříme, že $©D: \{D \in d©S | D \cap S \in d©S \forall S \in ©S\}$ je d-systém. Zřejmě $©D = d©D$. Nyní buď $D \in d©S$ pevné a definujeme $©D_D := \{E \in ©P(X) | E \cap D \in d©S\}$. O tom dokážeme, že je to d-systém. Následně dokážeme $®S \subset ©D_D$, tedy $D = ©D_D$. Vítězství!
	\end{dukazin}
\end{tvrzeni}

TODO?

% 03. 11. 2021

\begin{veta}[O jednoznačnosti míry]
	Nechť $©S \subset ©P(X)$ je $\pi$-systém a $\mu, \gamma$ jsou dvě míry na $\sigma ©S$ splňující $\mu(S) = \gamma(S)$, $\forall S \in ©S$. Jestliže existují množiny $X_n \in ©S$, $X_n \nearrow X$, $\mu(X_n) < +∞$, $\forall n \in ®N$, pak $\mu = \gamma$ na $\sigma ©S$.

	\begin{dukazin}
		Nejprve předpokládejme, že $\mu(X) < +∞$. Pak definujme systém $©D := \{A \in \sigma ©S | \mu(A) = \gamma(A)\}$. Platí $©S \subset ©D$, tedy $d©S \subset d©D = ©D \subset \sigma ©S$, tedy $©D = \sigma©S$.

		Je-li $\mu(X) = +∞$, pak definujeme $©D_n := \{A \in \sigma ©S | \mu(A \cap X_n) = \gamma(A \cap X_n)\}$, $n \in ®N$. Platí $©D_n$ je $d$-systém $\forall n \in ®N$ (ověř!). $©S \subset ©D_n$, $\forall n \in ®N$, neboť $S \in ©S: \mu(S \cap X_n) = \gamma(S\cap X_n)$. $d©S \subset d©D_n = ©D_n \subset \sigma ©S$, tedy $©D_n = \sigma ©S$, $\forall n \in ®N$.

		Nechť $A \in \sigma ©S$. Pak $\mu(A) = \lim_{n \rightarrow ∞} \mu(A \cap X_n) = \lim_{n \rightarrow ∞} \gamma(A \cap X_n) = \gamma(A)$. Tedy $\mu = \gamma$ na $\sigma ©S$.
	\end{dukazin}
\end{veta}

\section{Součin měr a Fubiniova věta}
\begin{poznamka}[Předpoklady pro další 2 přednášky]
	Nechť $(X, ©A, \mu)$, resp. $(Y, ©B, \gamma)$, je prostor se $\sigma$ konečnou mírou $\mu$, resp. $\gamma$.
\end{poznamka}

\begin{definice}[Měřitelný obdélník, ©O]
	Množinu $A \times B \subset X \times Y$, kde $A \in ©A$, $B \in ©B$, nazveme měřitelným obdélníkem.

	Symbolem ©O označíme systém všech měřitelných obdélníků.
\end{definice}

\begin{definice}[Součinová $\sigma$-algebra]
	Definujeme $\sigma$-algebru $©A \otimes ©B$ předpisem $©A \otimes ©B := \sigma ©O$.

	$\forall E \in ©A \otimes ©B\ \forall x \in X\ \forall y \in Y$ definujeme řezy $E_x$, $E^y$ množiny $E$ takto:
	$$ E_x := \{y \in Y | [x, y] \in E\}, \qquad E^y := \{x \in X | [x, y] \in E\}. $$
\end{definice}

\begin{veta}[O součinové $\sigma$-algebře $©A \otimes ©B$]
	Je-li $E \in ©A \otimes ©B$, tak

	\begin{enumerate}
		\item $\forall x \in X: E_x \in ©B$,
		\item $\forall y \in Y: E^y \in ©A$,
		\item funkce $x \mapsto \gamma(E_x)$ je měřitelná na $(X, ®A)$,
		\item funkce $y \mapsto \mu(E^y)$ je měřitelná na $(Y, B)$.
	\end{enumerate}

	Je-li funkce $f: (X\times Y, ©A \otimes ©B) \rightarrow ®R^*$ měřitelná, pak

	\begin{enumerate}
		\item $\forall x \in X$ je funkce $f_x: y \mapsto f(x, y)$ je měřitelná na $(Y, ©B)$,
		\item $\forall y \in Y$ je funkce $f_y: x \mapsto f(x, y)$ je měřitelná na $(X, ©A)$.
	\end{enumerate}

	\begin{dukazin}[Pouze lichá tvrzení, sudá jsou analogická]
		Definujeme $©E = \{E \in ©A \otimes ©B | E_x \in ©B\}$. Ověříme, že $©E$ je $\sigma$-algebra.

		TODO!!!
	\end{dukazin}
\end{veta}

% 10. 11. 2021

\begin{veta}[Existence a jednoznačnost součinové míry]
	Existuje právě jedna míra $\mu \otimes \nu$ (tzv. součinová míra) na $©A \otimes ©B$ splňující $(\mu \otimes \nu)(A \times B) = \mu(A)·\nu(B)$, $\forall A \in ©A$, $\forall B \in ©B$.

	Pro tuto míru platí
	$$ E \in ©A \otimes ©B \implies (\mu \otimes \nu)(E) = \int_X \nu(E_x) d\mu(x) \qquad \(= \int_Y \mu(E^y)d\nu(y)\). $$

	\begin{dukazin}
		1. Existence: Je-li $E \in ©A \otimes ©B$, pak definujeme $(\mu \otimes \nu)(E) = \int_X \nu(E_x) d\mu(x)$. O té dokážeme, že je mírou a že splňuje předpis v definici.

		2. Jednoznačnost: Nechť $\tau$ je míra na $©A \otimes ©B$, která splňuje $\tau(A \times B) = \mu(A)\nu(B)$, $\forall A \in ©A$, $\forall B \in ©B$, tedy $\tau = \mu \otimes \nu$ na ©O to je $\pi$-systém. Prostory $(X, ©A, \mu)$, $(Y, ©B, \nu)$ jsou prostory s $\sigma$-konečnými mírami. Tj.
		$$ \exists X_n \in ©A\ \forall n in ®N, X_n \nearrow X, \mu(X_n) < +∞\ \forall n \in ®N \land $$
		$$ \land \exists Y_n \in ©B\ \forall n in ®N, Y_n \nearrow Y, \nu(Y_n) < +∞\ \forall n \in ®N \Leftrightarrow $$
		$$ TODO. $$
	\end{dukazin}
\end{veta}

\begin{poznamka}
	Jsou-li $(X, ©A, \mu)$ a $(Y, ©B, \nu)$ prostory s úplnými $\sigma$-konečnými mírami, pak $\mu \otimes \nu$ nemusí být úplná.
\end{poznamka}


\begin{veta}[Fubiniova]
	Pro $\forall f \in ©L^*(\mu \otimes \nu)$ platí
	\begin{enumerate}
		\item $x \mapsto \int_Y f(x, y) d\nu(y)$ je měřitelná na $X$,
		\item $y \mapsto \int_X f(x, y) d\nu(x)$ je měřitelná na $Y$,
		\item $\int_{X \times Y} f d(\mu \otimes \nu) = \int_X\(\int_Y f(x, y)d\nu(y)\) d\mu(x) = \int_Y\(\int_X f(x, y)d\mu(x)\) d\nu(y)$.
	\end{enumerate}

	\begin{dukazin}
		1) $f = \chi_E$, $E \in ©A \otimes ©B$: $\nu(E_x) = \int_Y$ TODO!!!(Dokáže se nejprve pro charakteristickou funkci, pak pro jednoduché nezáporné, nakonec pro všechny.)
	\end{dukazin}
\end{veta}

\begin{poznamka}[Značení]
	Místo $(©A \otimes ©B)_0$ značíme $©A \overset{0}{\otimes} ©B$ (budu značit $©A \otimes_0 ©B$). A místo $(\mu \otimes \nu)_0$ píšeme \ldots (já píšu $©A \otimes_0 \nu$).
\end{poznamka}

\begin{veta}[Fubiniova věta pro zúplnění součinové míry]
	Nechť $(X, ©A, \mu)$, $(Y, ©B, \nu)$ jsou prostory s úplnými $\sigma$-konečnými mírami. Je-li $f \in ©L^*(\mu \otimes_0 \nu)$, pak

	\begin{itemize}
		\item funkce $x \mapsto f(x, y)$ je měřitelná na $X$ pro $\mu$-skoro všechna $y \in Y$,
		\item funkce $y \mapsto f(x, y)$ je měřitelná na $Y$ pro $\mu$-skoro všechna $x \in X$,
		\item funkce $x \mapsto \int_Y f(x, y) d\nu(y)$ je měřitelná na $X$,
		\item funkce $y \mapsto \int_X f(x, y) d\nu(x)$ je měřitelná na $Y$,
		\item $\int_{X \times Y} f d(\mu \otimes_0 \nu) = \int_X\(\int_Y f(x, y)d\nu(y)\) d\mu(x) = \int_Y\(\int_X f(x, y)d\mu(x)\) d\nu(y)$.
	\end{itemize}

	\begin{dukazin}
		Z 2 následujících tvrzení a Fubiniovy věty se věta snadno dokáže.
	\end{dukazin}
\end{veta}

\begin{tvrzeni}
	Buď $(Z, ©C, \rho)$ prostor s mírou a $(Z, ©C_0, \rho_0)$ jeho zúplnění. Je-li $f: (Z, ©C_0) \rightarrow ®R^*$ $\rho_0$ měřitelná funkce, pak existuje $\rho$ měřitelná funkce $g: (Z, ©C) \rightarrow ®R^*$ tak, že $f = g$ $\rho$-skoro všude na $Z$.
	
	\begin{dukazin}
		Vynechán.
	\end{dukazin}
\end{tvrzeni}

\begin{tvrzeni}
	Nechť $(X< ©A, \mu)$ a $(Y, ©B, \nu)$ jsou prostory s úplnými $\sigma$-konečnými mírami. Nechť $h: (X \times Y, ©A \otimes_0 ©B) \rightarrow ®R^*$ je $(\mu \otimes_0 \nu)$-měřitelná funkce a $h(x, y) = 0$ $\mu \otimes_0 \nu$-skoro všude na $X \times Y$. Pak pro $\mu$-skoro všechna $x \in X$ je $h(x, y)$ rovno 0 pro $\nu$-skoro všechna $y \in Y$.

	(Tzn, že pro $\mu$-skoro všechna $x \in X$ je funkce $h_x$ rovna 0 $\nu$-skoro všude na $Y$.)

	Speciálně, funkce $h_x$ je měřitelná na $X$ pro $\nu$-skoro všechna $y \in Y$.
	
	\begin{dukazin}
		Vynechán.
	\end{dukazin}
\end{tvrzeni}

\begin{definice}
	$\lambda^n = \(\lambda^*_{©B}\)_0$
\end{definice}

\begin{veta}[O míře $\lambda^p \otimes \lambda^q$]
	Nechť $p, q \in ®N$. Pak
	
	\begin{itemize}
		\item $©B(®R^{p + q}) = ©B(®R^p) \otimes ©B(®R^q)$,
		\item $\lambda^{p+q} = \lambda^p \otimes \lambda^q$.
	\end{itemize}

	\begin{dukazin}
		Neuveden.
	\end{dukazin}
\end{veta}

\begin{veta}[Fubiniova věta pro $\lambda^{p + q}$]
	Nechť $f \in ©L^*(\lambda^{p + q})$, $p, q \in ®N$. Pak
	$$ \int_{®R^{p + q}} f d\lambda^{p + q} = \int_{®R^p} \(\int_{®R^q} f(x, y) d\lambda^q(y)\) \lambda^p(x) = \int_{®R^q} \(\int_{®R^p} f(x, y) d\lambda^p(x)\) \lambda^q(y). $$
\end{veta}

\begin{definice}[Značení]
	$p, q \in ®N$, $x \in ®R^p$, $y \in ®R^q$. Definujeme projekce
	$$ \pi_1(x, y) = x, \qquad \pi_2(x, y) = y. $$
\end{definice}

\begin{dusledek}
	Nechť $p, q \in ®N$, $A \in ©B^{p + q}_0 := ©B(®R^{p + q})_0$. Je-li $f \in ©L^*(\lambda^{p + q})$ a projekce $\pi_1A, \pi_2A$ jsou měřitelné, pak
	$$ \int_A f d\lambda^{p + q} \int_{\pi_1A} \(f(x, y) d \lambda^q(y)\)\lambda^p(x) = \int_{\pi_2A} \(f(x, y) d \lambda^p(x)\)\lambda^q(y). $$
\end{dusledek}

\begin{poznamka}[Značení]
	Místo $d\lambda^p(x)$ píšeme $dx$ a místo $d\lambda^q(y)$ píšeme $dy$.
\end{poznamka}

\end{document}
