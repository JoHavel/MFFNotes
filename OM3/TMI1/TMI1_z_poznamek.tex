\documentclass[12pt]{article}					% Začátek dokumentu
\usepackage{../../MFFStyle}					    % Import stylu



\begin{document}

% 1. přednáška

\part{Definice}
\begin{definice}[Množinová funkce]
	Buď $X$ množina a $©P(X)$ její potenční množina, tj. $©P(X) := \{A | A \subset X\}$. Nechť $©A \subset ©P(X)$. Pak zobrazení $\tau: ©A \rightarrow ®R^*$ se nazývá množinová funkce.
\end{definice}

\begin{definice}[$\sigma$-algebra a algebra]
	Systém $©A \subset ©P(X)$ nazveme $\sigma$-algebra na $X$, jestliže

	\begin{itemize}
		\item $\O \in ©A$;
		\item $A \in ©A \implies A^c := X \setminus A \in ©A$;
		\item $A_i \in ©A\ \forall i \in ®N \implies \bigcup_{i \in ®N} A_i \in ©A$.
	\end{itemize}

	Jestliže nahradíme třetí podmínku za $A, B \in ©A \implies A \cup B \in ®A$, pak se systém ©A nazývá algebra.
\end{definice}

\begin{definice}[$\sigma ©S$]
	Je-li $©S \subset ©P(X)$ libovolný množinový systém, pak nejmenší $\sigma$-algebru obsahující systém ©S označíme $\sigma ©S$. (Existence vyplývá z věty o průniku $\sigma$-algeber.)
\end{definice}

\begin{definice}[Generátor $\sigma$-algebry]
	Je-li $©S \subset ©P(X)$ a $©A = \sigma ©S$, pak ©S nazveme generátor $\sigma$-algebry ©A (také říkáme, že ©A je generováno systémem ©S).
\end{definice}

\begin{definice}[Borelovská $\sigma$-algebra]
	Je-li $(X, \rho)$ metrický prostor a ©G systém všech otevřených podmnožin $X$, pak $©B(X) := \partial ©G$ se nazývá borelovská $\sigma$-algebra na $X$.
\end{definice}

\begin{definice}[Měřitelný prostor a měřitelná množina]
	Je-li ©A $\sigma$-algebra na $X$, pak dvojice $(X, ©A)$ se nazývá měřitelný prostor. Množiny $A \in ©A$ se nazývají ©A-měřitelné (krátce měřitelné, pokud nehrozí nedorozumění).
\end{definice}

\begin{definice}[Míra, prostor s mírou]
	Buď $(X, ©A)$ měřitelný prostor. Zobrazení $\mu: ©A \rightarrow [0, +∞]$ splňující

	\begin{itemize}
		\item[(M1)] $\mu(\O) = 0$;
		\item[(M2)] jestliže $A_i \in ©A$, $i \in ®N$, jsou po dvou disjunktní, pak $\mu\(\bigcup_{i=1}^∞ A_i\) = \sum_{i=1}^∞ \mu(A_i)$,
	\end{itemize}

	se nazývá míra. (M2 se také nazývá spočetná/sigma aditivita)

	Trojice $(X, ©A, \mu)$ se nazývá prostor s mírou.
\end{definice}

\begin{definice}[Nulová množina, úplný prostor, zúplnění]
	Buď $(X, ©A, \mu)$ prostor s mírou. Řekneme, že množina $N \subset X$ je nulová množina, jestliže existuje $A \in ©A$ tak, že $N \subset ©A$ a $\mu(A) = 0$. Symbolem ©N označíme systém všech nulových množin.

	Řekneme, že prostor $(X, ©A, \mu)$ je úplný, pokud $©N \subset ©A$. $\sigma$-algebru $©A_0 := \sigma(©A \cup ©N)$ nazveme zúplněním $\sigma$-algebry ©A vzhledem k míře $\mu$.
\end{definice}

\begin{definice}[Borelovská, konečná, pravděpodobnostní a $\sigma$-konečná míra]
	Buď $(X, ©A, \mu)$ prostor s mírou. Řekneme, že míra $\mu$ je:

	\begin{itemize}
		\item borelovská, je-li $X$ metrický prostor a $©A = ©B(X)$;
		\item konečná, je-li $\mu(X) < +∞$;
		\item pravděpodobnostní, je-li $\mu(X) = 1$;
		\item $\sigma$-konečná, existují-li množiny $X_i \in ©A$, $i \in ®N$, tak, že $\mu(X_i) < +∞$, $\forall i \in ®N$, a $X = \bigcup_{i \in ®N} X_i$.
	\end{itemize}
\end{definice}

% 2. přednáška

\begin{definice}[Lebesgueova míra]
	Zúplněné míry $\lambda_{©B}^n$ nazveme Lebesgueovou mírou v $®R^n$ a označíme $\lambda^n$.

	($\lambda_{©B}^n$ je borelovská míra na $®R^n$ taková, že $\lambda_{©B}^n([a_1, b_1] \times … \times [a_n, b_n]) = (b_1-a_1)·…·(b_n - a_n)$.)
\end{definice}

\begin{definice}[Vzor systému]
	Ať $X$, $Y$ jsou množiny, $f: X \rightarrow Y$ zobrazení a $©S \subset ©P(Y)$. Pak $f^{-1}(©S) := \{f^{-1}(S) | S \in ©S\}$.
\end{definice}

\begin{definice}[Měřitelné zobrazení, borelovsky měřitelné zobrazení]
	Nechť $(X, ©A)$ a $(Y, ©M)$ jsou měřitelné prostory. Zobrazení $f: X \rightarrow Y$ nazýváme měřitelné (vzhledem k $©A$ a $©M$), jestliže $f^{-1}(©M) \subset ©A$. Pak píšeme $f: (X, ©A) \rightarrow (Y, ©M)$.

	Je-li některý z prostorů $X$, $Y$ metrický prostor, pak za příslušnou $\sigma$-algebru bereme borelovskou $\sigma$-algebru (pokud není řečeno jinak). Měřitelné zobrazení mezi dvěma metrickými prostory se nazývá borelovsky měřitelné (stručně borelovské).
\end{definice}

\begin{definice}[Jednoduchá funkce]
	Funkce $s: X \rightarrow [0, +∞)$ se nazývá jednoduchá, jestliže $s(X)$ je konečná množina.

	Pak platí $s = \sum_{\alpha \in s(x)} \alpha·\chi_{\{s=\alpha\}}$. Součet na pravé straně této rovnosti nazýváme kanonickým tvarem jednoduché funkce $s$.
\end{definice}

\begin{definice}[Lebesgueův integrál]
	Buď $(X, ©A, \mu)$ prostor s mírou.

	\begin{itemize}
		\item Je-li $s: (X, ©A) \rightarrow [0, +∞)$ jednoduchá měřitelná funkce, zapíšeme ji v kanonickém tvaru $s = \sum_{j=1}^k \alpha_j \chi_{E_j}$, pro $E_j := \{x \in X | s(x) = \alpha_j\}$, a definujeme
		$$ \int_X s d\mu = \int_X s(x) d\mu(x) := \sum_{j=1}^k \alpha_j \mu(E_j). $$
	\item Je-li $f: (X, ©A) \rightarrow [0, +∞]$ měřitelná funkce, pak definujeme
		$$ \int_X f d\mu := \sup\{\int_X s d\mu | 0 ≤ s ≤ f, s \text{ jednoduchá, měřitelná}\}. $$
	\item Je-li $f: (X, ©A) \rightarrow ®R^*$, pak definujeme
		$$ \int_X f d\mu := \int_X f^+ d\mu - \int_X f^- d\mu, $$
		má-li rozdíl smysl.
	\end{itemize}
\end{definice}

% 3. přednáška

\begin{definice}[Skoro všude]
	Buď $(X, ©A, \mu)$ prostor s mírou a $V(x)$ vlastnost, kterou bod $x$ může, ale nemusí mít. Je-li $E \in ©A$, pak výrok $V(x)$ platí $\mu$-s. v. na $E$ znamená:
	$$ \exists N \in ©A \cap ©N, N \subset E\ \forall x \in E \setminus N: V(x). $$

	Je-li $E = X$, pak místo $\mu$-s. v. na $E$ píšeme pouze $\mu$-s. v. Pokud nehrozí nedorozumění, o jakou míru se jedná, tak píšeme pouze s. v. místo $\mu$-s. v.
\end{definice}

\begin{definice}[Měřitelná funkce]
	Buď $(X, ©A, \mu)$ prostor s mírou. Řekneme, že funkce $f$ definovaná na množině $D \in ©A$ s hodnotami v $®R^*$ je měřitelná na $X$, jestliže $\mu(D^c) = 0$ a $\forall$ otevřenou množinu $G \subset ®R^*$ platí $f^{-1}(G) \cap D \in ©A$.

	Pro měřitelnou funkci $f$ pak definujeme
	$$ \int_X f d\mu := \int_X \tilde f d\mu, \qquad \tilde f(x) := \begin{cases}f(x), & \forall x \in D, \\ 0, & \forall x \in D^c.\end{cases} $$
\end{definice}

\begin{definice}[$©L^*$ a $©L^1$]
	Je-li $(X, ©A, \mu)$ prostor s mírou, pak označujeme
	$$ ©L^*(\mu) := \{f | (X, ©A) \rightarrow ®R^*, f \text{ je měřitelná na } X, \exists \int_X f d\mu\}, $$
	$$ ©L^1(\mu) := \{f \in ©L^*(\mu) | \int_X |f| d\mu < +∞\}. $$
\end{definice}

% 4. přednáška

% 5. přednáška

\begin{definice}[Dynkinův systém (d-systém)]
	Systém $©D \subset ©P(X)$ se nazývá d-systém (nebo Dynkinův systém) na $X$, jestliže

	\begin{itemize}
		\item $\O \in ©D$;
		\item $D \in ©D \implies D^c \in ©D$;
		\item $D_n \in ©D$, $\forall n \in ®N$, $D_n \cap D_m = \O$ pro $n ≠ m$ $\implies \bigcup_{n \in ®N}D_n \in ©D$.
	\end{itemize}
\end{definice}

\begin{definice}[$\pi$-systém]
	Je-li systém $©S \subset ©P(X)$ uzavřen na konečné průniky (neboli $\forall S, T \in ©S: S \cap T \in ©S$), pak systém $©S$ nazveme $\pi$-systém.
\end{definice}

% 6. přednáška

\begin{definice}[Měřitelný obdélník, součinová $\sigma$-algebra, řezy]
	Je-li $A \in ©A$, $B \in ©B$, pak množinu $A \times B \subset X \times Y$ nazveme měřitelným obdélníkem. Systém všech měřitelných obdélníků označíme symbolem $©O$.

	Součinová $\sigma$-algebra $©A \otimes ©B$ na prostoru $X \times Y$ je dána předpisem
	$$ ©A \otimes B := \sigma ©O. $$
	Pro $E \in ©A \otimes ©B$ a $x \in X$, $y \in Y$ definujeme řezy $E_x$, $E^y$ množiny $E$ předpisy
	$$ E_x := \{y \in Y | [x, y] \in E\}, \qquad E^y := \{x \in X | [x, y] \in E\}. $$
\end{definice}

% 7. přednáška

% 8. přednáška

\begin{definice}[$C^1$-difeomorfismus]
	Buď $G \subset ®R^n$ otevřená množina. Zobrazení $\phi: G \rightarrow ®R^m$ se nazývá difeomorfismus, je-li prosté, třídy $C^1$ na $G$ a $\forall x \in G: J \phi(x) ≠ 0$.
\end{definice}

% 9. přednáška

% 10. přednáška

\begin{definice}[Absolutní spojitost měr]
	Nechť $\mu$, $\nu$ jsou míry na $(X, ©A)$. Řekneme, že míra $\nu$ je absolutně spojitá vzhledem k míře $\mu$ (a značíme $\nu \ll \mu$), jestliže
	$$ \forall A \in ©A: \mu(A) = 0 \implies \nu(A) = 0. $$
\end{definice}

\begin{definice}[(Radonova-Nikodymova) hustota / derivace míry]
	Funkce $f$ z Radonovy-Nikodymovy věty se nazývá (Radonova-Nikodymova) hustota nebo derivace míry $\nu$ vzhledem k míře $\mu$ a vztah
	$$ \nu(A) = \int_A f d\mu \qquad \forall A \in ©A $$
	se někdy zapisuje ve tvaru $d\nu = f d\mu$ nebo také $f = \frac{d\nu}{d\mu}$.
\end{definice}

% 11. přednáška

\begin{definice}[(Vzájemně) singulární míry]
	Řekneme, že míry $\mu$, $\nu$ na měřitelném prostoru $(X, ©A)$ jsou vzájemně singulární (a píšeme $\mu \perp \nu$), jestliže
	$$ \exists S \in ©A: \mu(S) = 0 \land \nu(X \setminus S) = 0. $$
\end{definice}

% 12. přednáška

\begin{definice}[Distribuční funkce]
	Buď $\mu$ konečná borelovská míra na ®R. Pak funkci
	$$ F_\mu (x) := \mu(\(-∞, x\>), \qquad x \in ®R, $$
	nazýváme distribuční funkcí míry $\mu$.
\end{definice}

\begin{definice}[Lebesgueův-Stieltjesův integrál]
	Je-li $F$ distribuční funkce konečné borelovské míry $\mu$ a $A \subset ©R$ borelovská množina, pak
	$$ \int_A f dF := \int_A f d\mu, \qquad (\text{má-li pravá strana smysl}). $$
\end{definice}

% 13. přednáška

\begin{definice}[Konvergence podle míry]
	Buď $(X, ©A, \mu)$ prostor s mírou a $f$, $f_n, n \in ®N$, měřitelné funkce na $X$. Řekneme, že funkce $f_n$ konvergují k funkci $f$ podle míry $\mu$ (značení $f_n \overset{\mu}{\rightarrow} f$), jestliže
	$$ \forall \epsilon > 0: \lim_{n \rightarrow ∞} \mu(\{x \in X\ |\ |f_n(x) - f(x)| ≥ \epsilon\}) = 0. $$
\end{definice}

\part{Tvrzení}

% 1. přednáška

\begin{veta}[O průniku $\sigma$-algeber]
	Nechť $©A_\alpha$, $\alpha \in I$, jsou $\sigma$-algebry na $X$ (kde $I$ je libovolná indexová množina). Pak $\bigcap_{\alpha \in I} ©A_\alpha$ je $\sigma$-algebra na $X$.

	\begin{dukazin}
		Triviální, přenechán čtenáři.
	\end{dukazin}
\end{veta}

\begin{dusledek}
	Je-li $©S \subset ©P(X)$ libovolný množinový systém, pak existuje nejmenší $\sigma$-algebra $\sigma ©S$ obsahující systém ©S.

	\begin{dukazin}
		$$ \sigma ©S := \bigcap \{©A \subset ©P(X) | ©S \subset ©A \lambda ©A \text{ je $\sigma$-algebra}\}. $$
	\end{dukazin}
\end{dusledek}

\begin{veta}[Vlastnosti míry]
	Buď $(X, ©A, \mu)$ prostor s mírou. Pak

	\begin{enumerate}
		\item $A, B \in ©A, A \cap B = \O \implies \mu(A \cup B) = \mu(A) + \mu(B)$;
		\item $A, B \in ©A, A \subset B \implies \mu(A) ≤ \mu(B)$;
		\item $A_i \in ©A, i \in ®N \implies \mu\(\bigcup_i A_i\) ≤ \sum_i \mu(A_i)$, (subaditivita míry);
		\item $A_1 \subset A_2 \subset … \implies \mu(A_i) \nearrow \mu\(\bigcup A_i\)$;
		\item $A_1 \supset A_2 \supset …, \mu(A_1) < +∞ \implies \mu(A_i) \searrow \mu\(\bigcap_i A_i\)$.
	\end{enumerate}

	\begin{dukazin}
		% Ze 4. přednášky
		Ad 1.: $A, B \in ©A$, $A \cap B = \O$, $A, B \in ©A$ $\implies$ $$ \implies A \cup B = A \cup B \cup \O \cup \O \cup … \implies $$
		$$ \implies \mu(A \cup B) = \mu(A) + \mu(B) + \mu(\O) + \mu(\O) + … = \mu(A) + \mu(B) $$

		Ad 2.: $A, B \in ©A$, $A \subset B$ $\implies$
		$$ \implies B = A \cup B \setminus A \implies \mu(B) = \mu(A) + \mu(B \setminus A) ≥ \mu(A). $$

		Ad 3.: $A_i \in ©A \forall i \in ®N$:
		$$ \bigcup_{i=1}^∞ A_i = A_1 \cup (A_2 \setminus A_1) \cup (A_3 \setminus (A_1 \cup A_2)) \cup … \implies $$
		$$ \implies \mu\(\bigcup_{i=1}^∞ A_i\) = \mu(A_1) + \mu(A_2 \setminus A_1) + … ≤ \mu(A_1) + \mu(A_2) + … $$

		Ad 4.: $A_1 \subset A_2 \subset …$, $A_i \in ©A \forall i \in ®N$
		$$ \implies A_k = \bigcup_{i=1}^k A_i = A_1 \cup (A_2 \setminus A_1) \cup (A_3 \setminus A_2) \cup … \cup (A_k \setminus A_{k-1}) \forall k \in ®N, k ≥ 2, $$
		$$ \bigcup_{i=1}^∞ A_i = A_1 \cup (A_2 \setminus A_1) \cup … \implies $$
		$$ \implies \mu(A_k) = \mu(A_1) + \sum_{i=2}^k \mu(A_i \setminus A_{i-1}) \forall k \in ®N, k ≥ 2, $$
		$$ \mu\(\bigcup_i^∞ A_i\) = \mu(A_1) + \sum_{i=2}^∞ \mu(A_i \setminus A_{i-1}). $$
		Z toho plyne $\mu(A_k) \nearrow \mu\(\bigcup_{i=1}^∞ A_i\)$.

		Ad 5. $A_1 \supset A_2 \supset …$, $A_i \in ©A \forall i \in ®N$, $\mu(A_1) < +∞$. Položíme $B_i = A_1 \setminus A_i \forall i \in ®N$. Pak na posloupnost $B_i$ aplikujeme 4.:
		$$ \mu(A_1) - \mu(A_i) \nearrow \mu(A_1) - \mu\(\bigcup_{i=1}^∞ A_i\) \implies -\mu(A_i) \nearrow \mu\(\bigcap_{i=1}^∞ A_i\) \implies $$
		$$ \implies \mu(A_i) \searrow \mu\(\bigcap_{i=1}^∞ A_i\). $$
	\end{dukazin}
\end{veta}

\begin{veta}[Zúplnění míry]
	Buď $(X, ©A, \mu)$ prostor s mírou. Pak platí
	
	\begin{enumerate}
		\item $©A_0 = \{R \subset X | \exists A, B \in ©A \land A \subset E \subset B \land \mu(B \setminus A) = 0\}$.
		\item Míru $\mu$ lze jednoznačně rozšířit na $©A_0$ (rozšířenou míru označíme $\mu_0$).
		\item Prostor $(X, ©A_0, \mu_0)$ je úplný.
	\end{enumerate}

	\begin{dukazin}[1.]
		Označme
		$$ \tilde{©A}_0 := \{E \subset X | \exists A, B \in ©A \land A \subset E \subset B \land \mu(B \setminus A) = 0\}. $$
		Ukážeme, že $\tilde{©A}_0$ je $\sigma$-algebra:

		\begin{itemize}
			\item $E = \O \in \tilde{©A}$, neboť volba $A = \O = B$ dává $A \subset E \subset B$, $\mu(B \setminus A) = \mu(\O \setminus \O) = \mu(\O) = 0$. Tedy $\O \in \tilde{©A}_0$.
			\item Je-li $E \in \tilde{©A}_0 \implies \exists A, B \in ©A: A \subset E \subset B \land \mu(B \setminus A) = 0$. Tedy $A^c, B^c \in ©A$ a $B^c \subset E^c \subset A^c$. Protože $A^c \setminus B^c = B \setminus A$, tak $\mu(A^c \setminus B^c) = \mu(B \setminus A) = 0$, a tedy $E^c \in \tilde{©A}_0$.
			\item Nechť $E_i \in \tilde{©A}_0\ \forall i \in ®N$. $\implies$ $\exists A_i, B_i \in ©A: A_i \subset E_i \subset B_i$, $\mu(B_i \setminus A_i) = 0$. $\implies \bigcup_i A_i \subset \bigcup_i E_i \subset \bigcup_i B_i$ a
				$$ \mu\(\bigcup_i B_i \setminus \bigcup_i A_i\) ≤ \mu\(\bigcup_i (B_i \setminus A_i)\) ≤ \sum_i \mu(B_i \setminus A_i) = 0. $$
				Tedy $\bigcup_i E_i \in \tilde{©A}_0$.
		\end{itemize}
		
		Tudíž $\tilde{©A}_0$ je $\sigma$-algebra.

		Platí $©A \cup ®N \subset \tilde{©A}_0$, neboť \vspace{-1em}
		
		\begin{itemize}
			\item Je-li $N \in ©N$ $\implies$ $\exists A \in ©A: N \subset A$, $\mu(A) = 0$ $\implies$ $N \in \tilde{©A}_0$, tedy $©N \subset \tilde{©A}_0$.
			\item Je-li $A \in ©A$, pak $A \subset A \subset A$ $\land$ $\mu(A \setminus A) = \mu(\O) = 0$ $\implies$ $A \in \tilde{©A}_0$, tedy $©A \subset \tilde{©A}_0$.
		\end{itemize}
		
		Z $©A \cup ©N \subset \tilde{©A}_0$ plyne $\sigma (©A \cup ©N) \subset \sigma(\tilde{©A}_0) = \tilde{©A}_0$, tj. $©A_0 \subset \tilde{©A}_0$.

		Opačnou inkluzi získáme snadno: Je-li $E \subset \tilde{©A}_0$, pak
		$$ \exists A, B \in ©A: A \subset E \subset B \land \mu(B \setminus A) = 0 \implies E = A \cup (E \setminus A) \implies E \in \sigma(©A \cup ©N) = ©A_0. $$

	\end{dukazin}

	\begin{dukazin}[2.]
		Buď $E \in ©A_0$. Tedy $\exists A, B \in ©A: A \subset E \subset B$, $\mu(B \setminus A) = 0$. Definujeme $\mu_0(E) := \mu(A)$. Tato definice je korektní, neboť: Je-li $A_i, B_i \in ©A$, $A_i \subset E \subset B_i$, $\mu(B_i \setminus A_i) = 0$, $i \in \{1, 2\}$, pak $A_1 = (A_1 \cap A_2) \cup (A_1 \setminus A_2)$, a tedy
		$$ \mu(A_1) = \mu(A_1) = \mu(A_1 \cap A_2) + \mu(A_1 \setminus A_2) = \mu(A_1 \cap A_2). $$
		Analogicky dostaneme $\mu(A_2) = \mu(A_1 \cap A_2)$. Tudíž $\mu(A_1) = \mu(A_2)$, což dokazuje korektnost definice.

		Dále ověříme, že $\mu_0$ je míra: Je jasné, že $\mu_0 ≥ 0$. Taktéž zřejmě $\mu_0(\O) = 0$, neboť $\O \subset \O \subset \O$, $\mu(\O \setminus \O) = \mu(\O) = 0$ $\implies$ $\mu_0(\O) = \mu(\O) = 0$.

		Je-li $E_i \in ©A_0 \forall i \in ®N$, $E_i \cap E_j = \O$ pro $i ≠ j$, pak
		$$ \exists A_i, B_i \in ©A: A_i \subset E_i \subset B_i, \mu(B_i \setminus A_i) = 0 \qquad \forall i \in ®N \implies $$
		$$ \implies \bigcup_i A_i \subset \bigcup_i E_i \subset \bigcup_i B_i, \qquad \bigcup_i B_i \setminus \bigcup_i A_i \subset \bigcup(B_i \subset A_i) \implies $$
		$$ \mu(\bigcup_i B_i \setminus \bigcup_i A_i) ≤ \mu\(\bigcup_i (B_i \setminus A_i)\) ≤ \sum_i \mu(B_i \setminus A_i) = 0. $$
		Proto $\mu_0\(\bigcup_i E_i\) = \mu\(\bigcup_i A_i\) = \sum_i \mu(A_i) = \sum_i \mu_0(E_i)$, tj. $\mu_0\(\bigcup_i E_i\) = \sum_i \mu_0(E_i)$. Takže $\mu_0$ je míra na $©A_0$.

		Dále platí, že $\mu_0$ je rozšířením $\mu$, neboť je-li $A \in ©A$, pak $A \subset A \subset A$, $\mu(A \setminus A) = \mu(\O) = 0$, tudíž $\mu_0(A) = \mu(A)$.

		Nakonec jednoznačnost: Buď $E \in ©A_0$. Pak $\exists A, B \in ©A: A \subset E \subset B$, $\mu(B \setminus A) = 0$. Proto $\mu_0(A) ≤ \mu_0(E) ≤ \mu_0(B)$. Ovšem $\mu(B) = \mu(A \cup (B \setminus A)) = \mu(A) + \mu(B \setminus A) = \mu(A)$. Tedy $\mu_0(E) = \mu(A)$.
	\end{dukazin}

	\begin{dukazin}[3.]
		Je-li $M \subset X$ nulová množina v $(X, ©A_0, \mu_0)$, pak
		$$ \exists N \in ©A_0: M \subset N \land \mu_0(N) = 0 \implies $$
		$$ \implies \exists A, B \in ©A: A \subset N \subset B \land \mu(B \setminus A) = 0 \implies \mu(A) = 0, $$
		a proto $\mu(B) = \mu(A) + \mu(B \setminus A) = 0 + 0 = 0$. Tudíž $\O \subset M \subset N \subset B$, tj. $\O \subset M \subset B$ a přitom $\mu(B \setminus \O) = \mu(B) = 0$. Tedy $M \in ©A_0$ a $\mu_0(M) = 0$.
	\end{dukazin}
\end{veta}

\begin{veta}[O míře $\lambda_{©B}^n$]
	Existuje právě jedna borelovská míra $\lambda_{©B}^n$ na $®R^n$ taková, že
	$$ \lambda_{©B}^n([a_1, b_1] \times … \times [a_n, b_n]) = (b_1 - a_1) \times … \times (b_n - a_n), $$
	jestliže $-∞ < a_i < b_i < + ∞$, $\forall i \in [n]$.

	\begin{dukazin}
		V TMI2.
	\end{dukazin}
\end{veta}

% 2. přednáška

\begin{veta}[O zobrazení $f: X \rightarrow Y$]
	Nechť $X$, $Y$ jsou množiny a $f: X \rightarrow Y$ zobrazení. Pak platí:

	\begin{enumerate}
		\item Je-li $©M$ $\sigma$-algebra na $Y$, pak $f^{-1}(©M)$ je $\sigma$-algebra na $X$.
		\item Je-li $©S \subset ©P(Y)$, pak $\sigma(f^{-1}(S)) = f^{-1}(\sigma ©S)$.
	\end{enumerate}

	\begin{dukazin}
		Bez důkazu?
	\end{dukazin}
\end{veta}

\begin{dusledek}
	Jsou-li $(X, ©A)$, $(Y, ©M)$ měřitelné prostory a $©S \subset ©M$ generátor $©M$, pak $f: X \rightarrow Y$ je měřitelné $\Leftrightarrow$ $f^{-1}(©S) \subset ©A$.
\end{dusledek}

\begin{dusledek}
	Je-li $X, ©A$ měřitelný prostor a $Y$ metrický prostor, pak $f: X \rightarrow Y$ je měřitelné právě tehdy, když $f^{-1}(G) \in ©A$ pro všechny otevřené množiny $G \subset Y$.
\end{dusledek}

\begin{dusledek}
	Každé spojité zobrazení $f$ mezi metrickými prostory je borelovsky měřitelné.
\end{dusledek}

\begin{veta}[Generátory $©B^n := ©B(®R^n)$]
	Borelovská $\sigma$-algebra $B^n$ je generována:

	\begin{itemize}
		\item otevřenými intervaly $(a_1, b_1) \times … \times (a_n, b_n)$, kde $-∞ < a_i < b_i < +∞$, $i \in [n]$;
		\item systémem $©S := \{(-∞, a_1) \times … \times (-∞, a_n) | a_i \in ®R\ \forall i \in [n]\}$.
	\end{itemize}

	\begin{dukazin}
		Bez důkazu?
	\end{dukazin}
\end{veta}

\begin{veta}[O měřitelných zobrazeních]
	\ 

	\begin{enumerate}
		\item Jsou-li $f: (X, ©A) \rightarrow ®R^n$ a $g: (X, ©A) \rightarrow ®R^m$ měřitelné zobrazení, pak $(f, g): (X, ©A) \rightarrow ®R^{n + m}$ je měřitelné zobrazení.
		\item Jsou-li $f, g: (X, ©A) \rightarrow ®R^n$ měřitelná zobrazení, pak $f ± g$ je měřitelné zobrazení.
		\item Jsou-li $f, g : (X, ©A) \rightarrow ®R$ měřitelné funkce, pak také funkce $f·g$, $\max\{f, g\}$, $\min\{f, g\}$ jsou měřitelné.
	\end{enumerate}

	\begin{dukazin}
		Bez důkazu?
	\end{dukazin}
\end{veta}

\begin{veta}[O měřitelných funkcích]
	Buď $(X, ©A)$ měřitelný prostor. Pak platí:

	\begin{enumerate}
		\item $f: (X, ©A) \rightarrow ®R$ je měřitelná funkce $\Leftrightarrow$ $f^{-1}((-∞, a)) \in ©A$ $\forall a in ®R$.
		\item $f: (X, ©A) \rightarrow ®R^*$ je měřitelná funkce $\Leftrightarrow$ $f^{-1}(\<-∞, a\)) \in ©A$ $\forall a in ®R$.
	\end{enumerate}

	\begin{dukazin}
		Bez důkazu.
	\end{dukazin}
\end{veta}

\begin{dusledek}
	Nechť $f, g: (X, ©A) \rightarrow ®R^*$ jsou měřitelné funkce. Pak:

	\begin{enumerate}
		\item Množiny $\{x \in X | f(x) < g(x)\}$, $\{x \in X | f(x) ≤ g(x)\}$, $\{x \in X | f(x) = g(x)\}$ jsou měřitelné.
		\item $\max\{f, g\}$, $\min\{f, g\}$ jsou měřitelné funkce. (Speciálně funkce $f^+ = \max\{f, 0\}$ a $f^- = \min\{f, 0\}$ jsou měřitelné.)
	\end{enumerate}
\end{dusledek}

\begin{veta}[O měřitelných funkcích podruhé]
	Jsou-li funkce $f_n: (X, ©A) \rightarrow ®R^*$, $n \in ®N$, měřitelné, pak i funkce $\sup_{n \in ®N} f_n$, $\inf_{n \in ®N} f_n$, $\limsup_{n \rightarrow ∞} f_n$ a $\liminf_{n \rightarrow ∞} f_n$ měřitelné.

	Speciálně bodová limita posloupnosti měřitelných funkcí je měřitelná funkce.

	\begin{dukazin}
		Bez důkazu?
	\end{dukazin}
\end{veta}

\begin{veta}[O nezáporné měřitelné funkci]
	Nechť $f: (X, ©A) \rightarrow \<0, +∞\>$ je měřitelná funkce. Pak existují jednoduché nezáporné měřitelné funkce $s_n$ na $X$, $n \in ®N$, tak, že
	$$ \forall x \in X: s_n(x) \nearrow f(x). $$
	Je-li navíc funkce $f$ omezená, pak $s_n \rightrightarrows f$ na $X$.

	\begin{dukazin}
		% Ze 13. přednášky
		Pro $n \in ®N$ a $i \in [n·2^n]$ definujeme
		$$ E_{n, i} := f^{-1}\(\<\frac{i-1}{2^n}, \frac{i}{2^n}\>\), \qquad F_n := f^{-1}\(\<n, +∞\>\), $$
		$$ s_n := \sum_{i=1}^{n·2^n} \frac{i-1}{2^n} \chi_{E_{n, i}} + n \chi_{F_n}. $$

		Množiny $E_{n, i}$ a $F_n$ jsou měřitelné, tedy $s_n$ jsou měřitelné. Je jasné, že $s_n$ je jednoduchá nezáporná funkce a platí $s_n ≤ s_{n+1}$.

		Je-li $x \in X$ takové, že $f(x) < +∞$, pak pro dostatečně velká $n \in ®N$ platí $f(x) - \frac{1}{2^n} ≤ s_n(x) ≤ f(x)$. Je-li $x \in X$ takové že $f(x) = +∞$, pak $s_n(x) = n \rightarrow +∞$. Tedy $s_n(x) \rightarrow f(x)\ \forall x \in X$.

		Je také jasné, že $s_n \rightrightarrows f$, pokud je funkce $f$ omezená (neboť pak $|s_n(x) - f(x)| ≤ \frac{1}{2^n}\ \forall x \in X\ \forall n \in ®N$).
	\end{dukazin}
\end{veta}

% 3. přednáška

\begin{lemma}[K důkazu Leviho věty]
	% Z 9. přednášky
	Buď $(A, ©A, \mu)$ prostor s mírou a $S$ jednoduchá, nezáporná, měřitelná funkce na $X$. Definujeme-li
	$$ \phi(A) := \int_A s d\mu \qquad \forall A \in ©A, $$
	pak $\phi$ je míra na ©A.

	\begin{dukazin}
		Je jasné, že $\phi ≥ 0$. Nechť $s = \sum_{j=1}^k \alpha_j \chi_{E_j}$ je kanonický tvar funkce $s$. Je-li $A \in ©A$, pak
		$$ \phi(A) = \int_A s d\mu = \int_X \chi_A s d\mu = \int_X \tilde{s} d\mu = \sum_{j=1}^k \alpha_j \mu(E_j \cap A), $$
		kde $\tilde s := \sum_{j=1}^k \alpha_j·\chi_{E_j \cap A}$ je jednoduchá nezáporná měřitelná funkce, tudíž můžeme použít definici integrálu. Z této rovnosti už vyplývá $\phi(\O) = \sum_{j=1}^k \alpha_j \mu(E_j \cap \O) = 0$.

		Nechť $A_i \in A\ \forall i \in ®N$, $A_i \cap A_j = \O$ pro $i ≠ j$. Buď $A = \bigcup_{i=1}^∞ A_i$. Pak
		$$ \phi(A) = \sum_{j=1}^k \alpha_j \mu(E_j \cap A) = \sum_{j=1}^k \alpha_j \sum_{i=1}^∞\mu(E_j \cap A_i) = $$
		$$ = \lim_{n \rightarrow ∞} \sum_{j=1}^k \sum_{i=1}^n \alpha_j \mu(E_j \cap A_i) = \lim_{n \rightarrow ∞} \sum_{i=1}^n \phi(A_i) = \sum_{i=1}^∞ \phi(A_i). $$
	\end{dukazin}
\end{lemma}

\begin{veta}[Levi]
	Je-li $(X, ©A, \mu)$ prostor s mírou a $f_n$, $n \in ®N$ jsou nezáporné měřitelné funkce na $X$ splňující $f_n \nearrow f$, pak $\int_X f_n d\mu \nearrow \int_X f d\mu$.

	\begin{dukazin}
		Protože $f_n ≤ f_{n+1}$ tak $\int_X f_n d\mu ≤ \int_X f_{n+1} d\mu \implies$
		$$ \exists \alpha \in \<0, +∞\>: \int_X f_n d\mu \rightarrow \alpha. $$
		Podle věty o měřitelných zobrazeních podruhé je funkce $f$ měřitelná na $X$. Protože $f_n ≤ f$, tak $\int_X f_n d\mu ≤ \int_X f d\mu$. Odtud plyne $\alpha ≤ \int_X d\mu$.

		Nyní dokážeme opačnou nerovnost: Buď $s$ libovolná jednoduchá měřitelná funkce splňující $0 ≤ s ≤ f$ a nechť $C \in (0, 1)$. Definujeme $\forall n \in ®N$
		$$ E_n := \{x \in X | f_n(x) ≥ C s(x)\}. $$
		Množiny $E_n$ jsou měřitelné
		$$ E_1 \subset E_2 \subset … \qquad \land \qquad X = \bigcup_{n=1}^∞ E_n. $$
		Dále platí
		$$ \int_X f_n d\mu ≥ \int_{E_n} f_n d\mu ≥ C \int_{E_n} s d\mu. $$
		Protože funkce $\phi(A) := \int_A s d\mu$ je z předchozího lemmatu míra na ©A, tak pravá strana konverguje k $c \int_X s d\mu$. Levá strana konverguje k $\alpha$, tedy $\alpha ≥ c \int_X d\mu$,
		a limitním přechodem pro $c \nearrow 1$ máme $\alpha ≥ \int_X s d\mu$. Odtud dostaneme
		$$ \alpha ≥ \sup_{0 ≤ s ≤ f} \int_X s d\mu = \int_X f d\mu. $$
		Tedy $\int_X f_n d\mu \rightarrow \int_X f d\mu$.
	\end{dukazin}
\end{veta}

\begin{veta}[Fatouovo lemma]
	Je-li $(X, ©A, \mu)$ prostor s mírou a $f_n$, $n \in ®N$, jsou nezáporné měřitelné funkce na $X$, pak
	$$ \int_X (\liminf_{n \rightarrow ∞} f_n) d\mu ≤ \liminf_{n \rightarrow ∞} \int_X f_n d\mu. $$

	\begin{dukazin}
		% Ze 4. přednášky
		Buď
		$$ g_n(x) := \inf \{f_k(x) | k ≥ n\}, x \in X, n \in ®N. $$
		Pak $g_n$ jsou měřitelné a platí
		$$ g_n \nearrow g := \lim_{n \rightarrow ∞} g_n := \liminf_{n \rightarrow ∞} f_n. $$
		
		Podle Leviho věty $\int_X g_n d\mu \nearrow \int_X g d\mu$. Protože $g_n ≤ f_n \forall n \in ®N$, tak $\int_X g_n d\mu ≤ \int_X f_n d\mu \forall n \in ®N$. Z uvedeného limitním přechodem dostaneme
		$$ \liminf_{n \rightarrow ∞} \int_X g_n d\mu ≤ \liminf_{n \rightarrow ∞} \int_X f_n d\mu. $$
		Pravá strana je rovna
		$$ \lim_{n \rightarrow ∞} \int_X g_n d\mu = \int_X g d\mu = \int_X \liminf_{n \rightarrow ∞} f_n d\mu. $$
	\end{dukazin}
\end{veta}

\begin{lemma}
	Je-li $(X, ©A, \mu)$ prostor s mírou a $f$, $g$ jsou měřitelné funkce na $X$ splňující $f = g$ skoro všude, pak $\int_X f d\mu = \int_X g d\mu$, má-li jedna strana rovnosti smysl.

	\begin{dukazin}
		Bez důkazu?
	\end{dukazin}
\end{lemma}

\begin{veta}[Linearita integrálu]
	Jestliže $f, g \in ©L^*(\mu)$ a $\lambda \in ®R$, pak
	$$ \int_X(\lambda f) d\mu = \lambda \int_X f d\mu, $$
	$$ \int_X (f + g) d\mu = \int_X f d\mu + \int_x g d\mu, \qquad \text{má-li pravá strana smysl}. $$

	\begin{dukazin}
		Bez důkazu?
	\end{dukazin}
\end{veta}

\begin{dusledek}[Linearity a Leviho]
	Je-li $(X, ©A, \mu)$ prostor s mírou a $f_n$, $n \in ®N$, jsou nezáporné měřitelné funkce na $X$, pak
	$$ \int_X \(\sum_{n=1}^∞ f_n\) d\mu = \sum_{n=1}^∞ \int_X f_n d\mu. $$

	\begin{dukazin}
		Z předchozí věty máme
		$$ \int_X \(\sum_{n=1}^k f_n\) d\mu = \sum_{n=1}^k \int_X f_n d\mu \qquad \forall k \in ®N. $$
		Odtud limitním přechodem pro $k \rightarrow +∞$ pomocí Leviho věty dostaneme dané tvrzení.
	\end{dukazin}
\end{dusledek}

\begin{veta}[Zobecněná Leviho věta]
	Je-li $(X, ©A, \mu)$ prostor s mírou a $f_n$, $n \in ®N$, měřitelné funkce na $X$ splňující $f_n \nearrow f$ a $\int_X f_1 d\mu > -∞$, pak
	$$ \int_X f_n d\mu \nearrow \int_X f d\mu. $$

	\begin{dukazin}
		BÚNO $\int_X f_1 < +∞$, jinak vztah plyne z monotonie integrálu. Buď $g_n := f_n - f_1$, $n \in ®N$, $g:= f - f_1$. Pak $g_n ≥ 0$, $g_n \nearrow g$ a z Leviho věty dostaneme $\int_X g_n d\mu \nearrow \int_X g d\mu$. Odtud pak, s využitím aditivity integrálu z předpředchozí věty, plyne $\int_X f_n d\mu = \int_X f d\mu$.
	\end{dukazin}
\end{veta}

\begin{dusledek}
	Totéž platí pro obrácená znamínka.
\end{dusledek}

\begin{veta}[Lebesgueova]
	Nechť $(X, ©A, \mu)$ je prostor s mírou a $f$, $f_n, n \in ®N$, jsou měřitelné funkce na $X$ splňující $f_n \rightarrow f$ skoro všude. jestliže existuje funkce $g \in ©L^1(\mu)$ tak, že $|f_n| ≤ g$ skoro všude $\forall n \in ®N$, pak $f \in ©L^1(\mu)$ a $\int_X f_n d\mu \implies \int_X f d\mu$.

	\begin{dukazin}
		% Z 12. přednášky
		Předefinujme funkce $f_n$, $f$ na množině $\{x | f_n(x) \nrightarrow f(x)\} \cup \bigcup_{n=1}^∞ \{x\ |\ |f_n(x)| > g(x)\}$
		nulové míry tak, aby předpoklady platily $\forall x \in X$. Je-li
		$$ g_n := \inf \{f_n, f_{n+1}, …\}, h_n := \sup\{f_n, f_{n+1}, …\}, n \in ®N, \text{ pak} \forall n \in ®N: $$
		$$ -g ≤ g_n ≤ f_n ≤ h_n ≤ g \implies g_n, f_n \in ©L^1(\mu), -g ≤ \lim_{n \rightarrow ∞} f_n = f ≤ g \implies f \in ©L^1(\mu). $$
		Protože $g_n \nearrow f$, $h_n \searrow f$ pro $n \rightarrow ∞$, tak podle zobecněné Leviho věty a jejího důsledku platí $\int_X g_n d\mu \rightarrow \int_X f d\mu$ a $\int_X h_n d\mu \rightarrow \int_X f d\mu$. Protože
		$$ \int_X g_n d\mu ≤ \int_X f_n d\mu ≤ \int_X h_n d\mu \implies \int_X f_n d\mu \rightarrow \int_X f d\mu. $$
	\end{dukazin}
\end{veta}

\begin{dusledek}
	Nechť $(X, ©A, \mu)$ je prostor s mírou a $f_n, n \in ®N$, jsou měřitelné funkce na $X$ takové, že $\sum_{n=1}^∞ f_n$ konverguje skoro všude. Jestliže existuje funkce $g \in ©L^1(\mu)$ tak, že $\left|\sum_{n=1}^k f_n\right| ≤ g$ skoro všude $\forall k \in ®N$, pak $\sum_{n=1}^∞ f_n \in ©L^1(\mu)$ a $\int_X\(\sum_{n=1}^∞ f_n\) d\mu = \sum_{n=1}^∞ \int_X f_n d\mu$.

	\begin{dukazin}
		Aplikujeme předchozí větu na posloupnost částečných součtů $\sum_{n=1}^∞ f_n$.
	\end{dukazin}
\end{dusledek}

\begin{veta}[Další vlastnosti integrálů a měřitelných funkcí]
	Buď $(X, ©A, \mu)$ prostor s mírou.

	\begin{itemize}
		\item Jestliže $f$ je nezáporná měřitelná funkce na $X$ a $\int_X f d\mu = 0$, pak $f = 0$ skoro všude.
		\item Je-li $f \in ©L^1(\mu)$ a $\int_E f d\mu = 0\ \forall E \in ©A$, pak $f = 0$ skoro všude.
		\item Je-li $f$ měřitelná funkce na $X$, pak
			$$ \int_X f d\mu \in ®R \Leftrightarrow \int_X |f| d\mu \in ®R. $$
		\item Je-li $f \in ©L^1(\mu)$, pak $\left| \int_X f d\mu \right| ≤ \int_X |f| d\mu$.
		\item Je-li $f \in ©L^1(\mu)$, pak $f$ je konečná skoro všude.
	\end{itemize}

	\begin{dukazin}
		Bez důkazu?
	\end{dukazin}
\end{veta}

\begin{veta}[Vztah Riemannova a Lebesgueova integrálu]
	Nechť $-∞ < a < b < +∞$ a $f: \<a, b\> \rightarrow ®R$. Jestliže $(R) \int_a^b f$ existuje, pak $\int_a^b f d\lambda^1 \in ®R$ a platí
	$$ (R) \int_a^b f = \int_a^b f d\lambda^1. $$

	\begin{dukazin}
		Bez důkazu?
	\end{dukazin}
\end{veta}

\begin{veta}[Vztah Newtonova a Lebesgueova integrálu]
	Nechť $-∞ ≤ a < b ≤ +∞$ a $f: (a, b) \rightarrow ®R$ je spojitá a nezáporná funkce. Pak $(N) \int_a^b f$ existuje právě tehdy, když $\int_a^b f d\lambda^1 \in ®R$.
	
	V takovém případě navíc $(N) \int_a^b = \int_a^b f d\lambda^1$.

	\begin{dukazin}
		Bez důkazu?
	\end{dukazin}
\end{veta}

% 4. přednáška

\begin{veta}[O limitě integrálu závislém na parametru]
	Buď $(X, ©A, \mu)$ prostor s mírou, $(T, \rho)$ metrický prostor, $M \subset T$, $t_0 \in M'$ a $f: X \times T \rightarrow ®R$. Nechť platí:

	\begin{itemize}
		\item Pro $\mu$-skoro všechna $x \in X$ existuje
			$$ \lim_{t \rightarrow t_0, t \in M} f(x, t) =: \phi(x). $$
		\item $\forall t \in M \setminus \{t_0\}$ je funkce $f(·, t)$ $\mu$-měřitelná.
		\item Existuje funkce $g \in ©L^1(\mu)$ tak, že $|f(x, t)| ≤ g(x)$ pro $\mu$-skoro všechna $x \in X$ a $\forall t \in M \setminus \{t_0\}$.
	\end{itemize}

	Pak $\phi \in ©L^1(\mu)$ a $\lim_{t \rightarrow t_0, t \in M} \int_X f(x, t) d\mu = \int_X \phi(x) d\mu$.

	\begin{dukazin}
		K ověření rovnosti integrálů dle Heineho věty stačí dokázat: Je-li $t_n \in M \setminus \{t_0\}$, $n \in ®N$, $t_n \rightarrow t_0$, pak $\int_X f(x, t_n) d\mu \rightarrow \int_X \phi(x) d\mu$:

		Z první podmínky máme $f(x, t_n)$ pro $\mu$-skoro všechna $x \in X$. Dále platí (z druhé podmínky) $|f(x, t_n)| ≤ g(x)$ pro $\mu$-skoro všechna $x \in X$ a $\forall n \in ®N$.

		Tedy rovnost integrálů (a také existence integrálu) plyne z Lebesgueovy věty, položíme-li v ní $f_n(x) := f(x, t_n) \forall n \in ®N$.
	\end{dukazin}
\end{veta}

\begin{veta}[O spojitosti integrálu závislém na parametru]
	Buď $(X, ©A, \mu)$ prostor s mírou, $(T, \rho)$ metrický prostor, $M \subset T$ a $f: X \times T \rightarrow ®R$. Nechť platí:

	\begin{itemize}
		\item Pro $\mu$-skoro všechna $x \in X$ je funkce $f(x, ·)$ spojitá na $M$.
		\item $\forall t \in M$ je funkce $f(·, t)$ $\mu$-měřitelná.
		\item Existuje funkce $g \in ©L^1(\mu)$ tak, že $|f(x, t)| ≤ g(x)$ pro $\mu$-skoro všechna $x \in X$ a $\forall t \in M$.
	\end{itemize}

	Pak funkce $F(t) := \int_X f(x, t) d\mu$, $t \in M$, je spojitá na $M$.

	\begin{dukazin}
		Dle Heineho věty stačí dokázat: Je-li $t_0 \in M \cap M'$, pak $\lim_{t \rightarrow t_0, t \in M} F(t) = F(t_0)$, tj. $\lim_{t \rightarrow t_0, t \in M} \int_X f(x, t) d\mu = \int_X f(x, t_0) d\mu$. To ale plyne z předchozí věty.
	\end{dukazin}
\end{veta}

\begin{veta}[O derivaci integrálu podle parametru]
	Buď $(X, ©A, \mu)$ prostor s mírou, $I \subset ®R$ otevřený interval a $f: X \times I \rightarrow ®R$. Nechť platí:

	\begin{itemize}
		\item $\forall t \in I$ je funkce $f(·, t)$ $\mu$-měřitelná.
		\item $\exists N \in ©A$, $\mu(N) = 0$, tak, že $\forall x \in X \setminus N$ a $\forall t \in I$ existuje konečná derivace $\frac{\partial f}{\partial t}(x, t)$.
		\item Integrál $F(t) := \int_X f(x, t) d\mu$ konverguje alespoň pro jednu hodnotu $t \in I$.
		\item $\exists g \in ©L^1(\mu)$ tak, že $\forall x \in X \setminus ®N$ a $\forall t \in I$ platí $\left| \frac{\partial f}{\partial t}(x, t)\right| ≤ g(x)$.
	\end{itemize}

	Pak $\forall t \in I$ integrál $F(t)$ konverguje a platí
	$$ F'(t) = \int_X \frac{\partial f}{\partial t}(x, t) d\mu \qquad \forall t \in I. $$

	\begin{dukazin}
		Je-li $t$, $t + h \in I$, pak $\forall x \in X \setminus N$ dle Lagrangeovy věty dle druhé a čtvrté podmínky platí
		$$ |f(x, t + h) - f(x, t)| = \left|h · \frac{\partial f}{\partial t}(x, t + \Theta h)\right| ≤ |h| · g(x), $$
		kde $\Theta \in (0, 1)$. Speciálně, je-li $t \in I$ a $t_0$ onen bod, pro který integrál $F(t)$ konverguje, pak
		$$ |f(x, t)| ≤ |f(x, t_0)| + |t - t_0|·g(x) \forall x \in X \setminus N, $$
		odkud plyne, že integrál $F(t)$ konverguje $\forall t \in I$.

		Dále, je-li $t$, $t + h \in I$, $h ≠ 0$, pak
		$$ \frac{1}{h}(F(t + h) - F(t)) = \int_X \frac{f(x, t + h) - f(x, t)}{h} d\mu. $$
		Protože z nerovnosti výše je
		$$ \left|\frac{f(x, t + h) - f(x, t)}{h}\right| = \left|\frac{\partial f}{\partial t}(x, t + \Theta h)\right| ≤ g(x) \forall x \in X \setminus N, \forall t, t + h \in I, h≠0, $$
		tedy
		$$ \lim_{h \rightarrow 0} \int_X \frac{f(x, t + h) - f(x, t)}{h} d\mu = \int_X \(\lim_{h \rightarrow 0} \frac{f(x, t + h) - f(x, t)}{h}\) d\mu = \int_X \frac{\partial f}{\partial t}(x, t) d\mu. $$
	\end{dukazin}
\end{veta}

% 5. přednáška

\begin{veta}[O průniku d-systémů]
	Nechť $©D_\alpha, \alpha \in I$, jsou d-systémy na $X$ ($I$ je libovolná indexová množina). Pak $\bigcap_{\alpha \in I} ©D_\alpha$ je d-systém na $X$.

	\begin{dukazin}
		Je triviální a přenechán čtenáři.
	\end{dukazin}
\end{veta}

\begin{dusledek}
	Je-li $©S \subset ©P(X)$ libovolný množinový systém, pak existuje nejmenší d-systém $d ©S$ na $X$ obsahující systém $©S$.

	\begin{dukazin}
		$$ d©S := \bigcap \{©D \subset ©P(X) | ©S \subset ©D \land ©D \text{ je d-systém}\}. $$
	\end{dukazin}
\end{dusledek}

\begin{veta}[O rovnosti $d©S = \sigma ©S$]
	Je-li $©S \subset ©P(X)$ $\pi$-systém, pak $d©S = \sigma ©S$.

	\begin{dukazin}
		Z následujících dvou tvrzení. Protože $©S \subset ©P(X)$ je $\pi$-systém, tak je $d©S$ $\pi$-systém. Protože $d©S$ je také d-systém, tak $d©S$ je $\sigma$-algebra na $X$, která obsahuje ©S. Proto $\sigma ©S \subset d©S$, neboť $\sigma ©S$ je nejmenší $\sigma$-algebra obsahující ©S. Opačná implikace tj. $d©S \subset \sigma ©S$ platí triviálně. Tedy $d©S = \sigma ©S$.
	\end{dukazin}
\end{veta}

\begin{tvrzeni}
	Je-li d-systém ©D na $X$ $\pi$-systém, pak ©D je $\sigma$-algebra na $X$.

	\begin{dukazin}
		Je třeba ověřit, že platí $A_k \in ©D\ \forall k \in ®N \implies \bigcup_{k=1}^∞ A_k \in ©D$. To provedeme v několika krocích:

		Platí $A \setminus B \in ©D$, je-li $A, B \in ©D$, neboť $A \setminus B = A \setminus (A \cap B)$ a přitom $A \cap B \subset A$, tedy $A \setminus B \in ©D$.

		Platí $A \cup B \in ©D$, je-li $A, B \in ©D$, neboť $A \cup B = (A \setminus B) \cup B$ a přitom $(A \setminus B) \cap B = \O$, tedy $A \cup B \in ©D$.

		Je-li $n \in ®N$ a $A_1, …, A_n \in ©D$, pak $\bigcup_{i=1}^n A_i \in ©D$ (indukcí s využitím předchozího odstavce).

		Nechť tedy $A_k \in ©D \forall k \in ®N$. Položíme-li $A_0 := \O \in ©D$, pak
		$$ \bigcup_{k=1}^∞ A_k = \bigcup_{k=1}^∞\(\(\bigcup_{i=0}^k A_i\) \setminus \(\bigcup_{i=1}^{k-1} A_i\)\) = \bigcup_{k=1}^∞ \tilde A_k, $$
		kde $\tilde A_k := \(\bigcup_{i=0}^k A_i\) \setminus \(\bigcup_{i=0}^{k-1} A_i\) \forall k \in ®N$. Protože $\bigcup_{i=0}^k A_i \in ©D \forall k \in ®N_0$, tak $\tilde A_k \in ©D \forall k \in ®N$. Navíc $\tilde A_k \cap \tilde A_m = \O$ pro $k ≠ m$, $k, m \in ®N$. Tedy $\bigcup_{k=1}^∞ \tilde A_k \in ©D$, tj. $\bigcup_{k=1}^∞ A_k \in ©D$.
	\end{dukazin}
\end{tvrzeni}

\begin{tvrzeni}
	Je-li $©S \subset ©P(X)$ $\pi$-systém, pak $d©S$ je také $\pi$-systém.

	\begin{dukazin}
		Je-li $©D := \{D \in d ©S | D \cap S \in d ©S\ \forall S \in ©S\}$, pak ©D je d-systém, neboť:
		
		\begin{itemize}
			\item $\O \in ©D$, protože $\O \in d©S$ a $\O \cap S = \O \in d ©S\ \forall S \in ©S$;
			\item $D \in ©D \implies D \cap S \in d©S\ \forall S \in ©S \implies$
				$$ \implies (X \setminus D) \cap S = (X \cap S) \setminus (D \cap S) = S \setminus (D \cap S) \in d ©S\ \forall S \in ©S \implies X \setminus D \in ©D; $$
			\item $D_n \in ©D\ \forall n \in ®N, D_n \cap D_m = \O$ pro $n≠m$ $\implies$
				$$ \implies D_n \cap S \in d©S\ \forall S \in ©S \qquad \forall n \in ®N. $$
				a $(D_n \cap S) \cap (D_m \cap S) \subset D_n \cap D_m = \O$ pro $n ≠ m$ $\forall S \in ©S$.

				Pak $\(\bigcup_n D_n\) \cap S = \bigcup_n(D_n \cap S) \in d©S$. Tj. $\bigcup_n D_n \in ©D$.
		\end{itemize}

		Dále platí $©S \subset ©D$, neboť $\forall D \in ©S$ máme $D \in d©S$ a přitom $D \cap S \in ©S\ \forall S \in ©S$. Odtud máme $D \cap S \in d©S\ \forall S \in ©S$ (neboť $©S \subset d©S$), a tedy $D \in ©D$.

		Z inkluze $©S \subset ©D$ plyne $d©S \subset d©D = ©D$. Navíc (dle definice systému ©D) platí $©D \subset d©S$. Celkem tedy platí $©D = d©S$, což znamená
		$$ \forall D \in d©S: D \cap S \in d©S \qquad \forall S \in ©S. $$

		Je-li $D \in d©S$ pevné a $©D_D := \{E \in ©P(X) | E \cap D \in d©S\}$, pak $©D_D$ je d-systém, neboť:

		\begin{itemize}
			\item $\O \in ©D_D$, protože $\O \in ©P(X)$ a $\O \cap D = \O \in d©S$;
			\item $E \in ©D_D \implies E \cap D \in d©S$, a tedy
				$$ (X \setminus E) \cap D = (X \cap D) \setminus (E \cap D) = D \setminus (E \cap D) \in d©S \implies X \setminus E \in ©D_D; $$
			\item $E_n \in ©D_D\ \forall n \in ®N, E_n \cap E_m = \O$ pro $n ≠ m$ $\implies$ $E_n \cap D \in d©S \forall n \in ®N$ a $(E_n \cap D) \cap (E_m \cap D) = \O$ pro $n ≠ m$. Pak $(\bigcup_n E_n) \cap D = \bigcup_n(E_n \cap D) \in d©S$, tj. $\subset_n E_n \in ©D_D$.
		\end{itemize}

		Z $©D = d©S$ plyne $©S \subset ©D_D\ \forall D \in d©S$, tj. $d©S \subset ©D_D\ \forall D \in d©S$, což znamená
		$$ \forall E \in d©S: E \cap D \in d©S\ \forall D \in d©S, $$
		a tedy $d©S$ je uzavřený na průniky dvou množin $\implies$ $d©S$ je uzavřený i na průniky konečného počtu množin $\implies$ $d©S$ je $\pi$-systém.
	\end{dukazin}
\end{tvrzeni}

% 6. přednáška

\begin{veta}[O jednoznačnosti míry]
	Nechť $©S \subset ©P(X)$ je $\pi$-systém a $\mu$, $\nu$ jsou dvě míry na $\sigma ©S$ splňující $\mu(S) = \nu(S) \forall S \in ©S$. Jestliže existují množiny $X_n \in ©S, n \in ®N$, tak, že $X_n \nearrow X$ a $\mu(X_n) < +∞ \forall n \in ®N$, pak $\mu = \nu$ na $\sigma ©S$.

	\begin{dukazin}
		Předpokládejme nejprve, že $\mu(X) < +∞$. Systém
		$$ ©D := \{A \in \sigma ©S | \mu(A) = \nu(A)\} $$
		je d-systém. Dle předpokladu $©S \subset ©D$, odtud plyne
		$$ d ©S \subset d ©D = ©D. $$
		Podle věty o rovnosti $d©S = \sigma ©S$
		$$ \sigma ©S = d©S \subset ©D \subset \partial ©S $$
		odkud dostáváme $d©S = \sigma ©S = ©D$ $\implies$ $\mu(A) = \nu(A)$ na $\sigma ©S$.

		Je-li $\mu(X) = +∞$, pak definujeme
		$$ ©D_n := \{A \in \sigma ©S | \mu(A cap X_n) = \nu(A \cup X_n)\} \qquad \forall n \in ®N. $$
		Analogicky jako v první části důkazu lze ověřit, že $©D_n$ je d-systém ($\forall n \in ®N$), který obsahuje ©S (neboť $S \cap X_n \in ©S\ \forall n \in ®N, \forall S \in ©S$, protože ©S je $\pi$-systém). Tedy
		$$ d©S \subset d©D_n \subset \sigma ©S \qquad \forall n \in ®N \implies $$
		$$ \implies d©S = \sigma ©S = ©D_n \qquad \forall n \in ®N. $$
		Z vlastnosti míry pak $\forall A \in \sigma ©S$ dostaneme
		$$ \mu(A) = \lim_{n \rightarrow ∞} \mu(A \cap X_n) = \lim_{n \rightarrow ∞} \nu(A \cap X_n) = \nu(A). $$
	\end{dukazin}
\end{veta}

\begin{veta}[O součinové $\sigma$-algebře $©A \otimes ©B$]
	Je-li $E \in ©A \otimes ©B$, pak

	\begin{itemize}
		\item $\forall x \in X: E_x \in ©B$, $\forall y \in Y: E^y \in ©A$;
		\item Funkce $x \mapsto \nu(E_x)$ je měřitelná na $(X, ©A)$, funkce $y \mapsto \mu(E^y)$ je měřitelná na $(Y, B)$.
	\end{itemize}

	Je-li funkce $f: (X \times Y, ©A \otimes ©B) \rightarrow ®R^*$ měřitelná, pak $\forall x \in X$ je funkce $f_x: y \mapsto f(x, y)$ měřitelná na $(Y, B)$ a $\forall y \in Y$ je funkce $f^y: x \mapsto f(x, y)$ měřitelná na $(X, ©A)$.

	\begin{dukazin}
		Provedem pouze pro $E_x$, funkci $x \mapsto \nu(E_x)$ a funkci $f_x$ (pro $E^y$, funkci $y \mapsto \mu(E^y)$ a funkci $f^y$ je důkaz analogický).
	\end{dukazin}

	\begin{dukazin}[1.]
		$\forall x \in X$ je množina $©E := \{E \in ©A \otimes ©B | E_x \in ©B\}$ $\sigma$-algebra, neboť:
		
		\begin{itemize}
			\item $\O \in ©E$, protože $\O_x = \O \in ©B$;
			\item $E \in ©E \implies E_x \in ©B \implies (E^c)_x = (X \times Y \setminus E)_x = Y \setminus E_x \in ©B$, a tedy $E^c \in ©E$;
			\item $E_n \in ©E\ \forall n \in ®N \implies (E_n)_x \in ©B\ \forall n \in ®N \implies$
				$$ \implies (\bigcup_n E_n)_x = \bigcup_n(E_n)_x \in ©B, $$
				a tedy $\bigcup_n E_n \in ©E$.
		\end{itemize}

		Dále platí $\O \subset ©E \implies ©A \otimes B = \sigma ©O \subset \sigma ©E = ©E$. Ovšem z definice ©E máme $©E \subset ©A \otimes ©B$. Tedy $©E = \sigma ©O = ©A \otimes ©B$, což znamená, že
		$$ \forall x \in X\ \forall E \in ©A \otimes ©B: E_x \in ©B. $$
	\end{dukazin}

	\begin{dukazin}[3.]
		Buď $Y_0 \in ©B$, $\nu(Y_0) < +∞$ a
		$$ ©D := \{E \in ©A \otimes ©B | x \mapsto \nu(E_x \cap Y_0) \text{ je měřitelná na } (X, ©A)\}. $$
		Dokážeme-li, že a) $©O \subset ©D$, b) $©D$ je d-systém, c) ©O je $\pi$-systém, pak $d©O \subset d©D = ©D \subset ©A \otimes ©B$ $\implies$ $©D = ©A \otimes ©B$.

		a) Je-li $E \subset ©O$, pak $E = A \times B$, kde $A \in ©A$, $B \in ©B$ $\implies$ $E_x = B$ pro $x \in A$ a $\O$ pro $x \notin A$ $\implies$ $E_x \cap Y_0 = B \cap Y_0$ pro $x \in A$ a $\O$ pro $x \notin A$ $\implies$ $\nu(E_x \cap Y_0) = \nu(B \cap Y_0)·\chi_A(x)$ $\implies$ funkce $x \mapsto \nu(E_x \cap Y_0)$ je $(X, ©A)$ měřitelná (neboť $A \in ©A$). Tedy $©O \subset ©D$.

		b) $\O \in ©D$, neboť $\O \in ©A \otimes ©B$ a funkce $x \mapsto \nu(\O_x \cap Y_0) = \nu(\O) = 0\ \forall x \in X$ $\implies$ funkce $x \mapsto \nu(\O_k \cap Y_0)$ je $(X, ©A)$ měřitelná.

		$E \in ©D$ $\implies$ $E \in ©A \otimes ©B$ a funkce $x \mapsto \nu(E_x \cap Y_0)$ je $(X, ©A)$ měřitelná. Protože
		$$ (E^c)_x \cap Y_0 = (X \times Y \setminus E)_x \cap Y_0 = (Y \setminus E_x) \cap Y_0 = Y_0 \setminus E_x \cap Y_0, $$
		tak $\nu((E^c)_x \cap Y_0) = \nu(Y_0) \setminus \nu(E_x \cap Y_0)$ $\implies$ funkce $x \mapsto \nu((E^c)_x \cap Y_0)$ je rozdílem dvou měřitelných funkcí $x \mapsto \nu(Y_0)$ a $x \mapsto \nu(E_x \cap Y_0)$, tedy je to měřitelná funkce (na $(X, ©A)$).

		Buď $E_n \in ©D\ \forall n \in ®N$ a $E_n \cap E_m = \O$ pro $n ≠ m$. Tedy $E_n \in ©A \otimes ©B\ \forall n \in ®N$ a funkce $x \mapsto \nu((E_n)_x \cap Y_0)$ jsou měřitelné na $(X, ©A)$ $\forall n \in ®N$

		Protože $(\bigcup_n E_n)_x \cap Y_0 = \(\bigcup_n(E_n)_x\) \cap Y_0 = \bigcup_n((E_n)_x \cap Y_0)$, tak
		$$ \nu((\bigcup_n E_n)_x \cap Y_0) = \nu(\bigcup_n((E_n)_x \cap Y_0)) = \sum_n \nu((E_n)_x \cap Y_0) = \lim_{k \rightarrow ∞} \sum_{n=1}^k \nu((E_n)_x \cap Y_0) \implies $$
		Funkce $x \mapsto \nu((\bigcup_n E_n)_x \cap Y_0)$ je limita ($k \rightarrow +∞$) měřitelných funkcí $x \mapsto \sum_{n=1}^k \nu((E_n)_x \cap Y_0)$ $\implies$ funkce $x \mapsto \nu((\bigcup_n E_n)_x \cap Y_0)$ je měřitelná funkce (na $(X, ©A)$) $\implies$ $\bigcup_n E_n \in ©D$.

		c) To je jasné, neboť je-li $E_i \in ©O$, pak $E_i = A_i \times B_i$, kde $A_i \in ©A$, $B_i \in ©B$ $\implies$
		$$ \implies E_1 \cap E_2 = (A_1 \cap A_2) \times (B_1 \cap B_2) \in ©O. $$
		Tedy platí $d©O \subset d©D = ©D \subset ©A \otimes ©B$ $\implies$ $©D = ©A \otimes ©B$.

		Protože $\nu$ je $\sigma$-konečná míra, tak existují množiny $Y_n \subset Y$ ($\forall n \in ®N$) takové, že $\nu(Y_n) < +∞$ a $\nu(Y_n) \nearrow \nu(Y)$. Pak $\forall E \in ©A \otimes ©B$ platí $\nu(E_x) = \lim_{n \rightarrow ∞} \nu(E_x \cap Y_n)$, a tedy funkce $x \mapsto \nu(E_x)$ je měřitelná na $(X, ©A)$, neboť limitou funkcí $x \mapsto \nu(E_x \cap Y_n)$, které jsou dle předchozí části důkazu měřitelné na $(X, ©A)$.
	\end{dukazin}

	\begin{dukazin}[5.]
		Buď $a \in ®R^*$ a $E := \{[x, y] \in X \times Y | f(x, y) < a\}$. Protože $f$ je měřitelná, tak $E \in ©A \otimes ©B$. Dále $\forall x \in X$ platí
		$$ \{y \in Y | f_x(y) < a\} = E_x \in ©B. $$
		Tudíž $\forall x \in X$ je funkce $f_x$ měřitelná na $(Y, ©B)$.
	\end{dukazin}
\end{veta}

\begin{veta}[Existence a jednoznačnost součinové míry]
	Existuje právě jedna míra $\mu \otimes \nu$ na $©A \otimes ©B$ (tzv. součinová míra) splňující
	$$ (\mu \otimes \nu)(A \times B) = \mu(A)·\nu(B) \qquad \forall A \in ©A\ \forall B \in ©B. $$

	Pro tuto míru platí
	$$ E \in ©A \otimes ©B \implies (\mu \otimes \nu)(E) = \int_X \nu(E_x) d\mu(x). $$

	\begin{dukazin}
		1. Existence: $\forall E \in ©A \otimes ©B$ definujeme
		$$ (\mu \otimes \nu)(E) := \int_X \nu(E_x) d\mu(x). $$
		Pak $\mu \otimes \nu$ je míra na $(X \times Y, ©A \otimes ©B)$, neboť:
		$$ (\mu \otimes \nu)(\O) = \int_X \nu(\O_x) d\mu(x) = \int_X 0 d\mu(x) = 0; $$
		Je-li $E_n \in ©A \otimes ©B$, $n \in ®N$, $E_n \cap E_m = \O$ pro $n ≠ m$, pak
		$$ (\mu \otimes \nu)\(\bigcup_n E_n\) = \int_X \nu\(\(\bigcup_n E_n\)_x\) d\mu(x) = \int_X \nu\(\bigcup_n\(E_n\)_x\) = $$
		$$ = \int_X \sum_n \nu\((E_n)_x\) d\mu(x) = \sum_n \int_X \nu\((E_n)_x\) d\mu(x) = \sum_n (\mu \otimes \nu)(E_n). $$

		Z definice $\mu \otimes \nu$ na $©A \otimes ©B$ $\forall A \in ©A\ \forall B \in ©B$ dostáváme
		$$ (\mu \otimes \nu)(A \times B) = \int_X \nu\((A \times B)_x\) d\mu(x) = \int_X \nu(B)·\chi_A(x) d\mu(x) = $$
		$$ = \nu(B) \int_A d\mu(x) = \nu(B) ·\mu(A). $$

		2. Jednoznačnost: Předpokládejme, že $\tau$ je míra na $©A \otimes ©B$ splňující $\tau(A \times B) = \mu(A)·\nu(B)\ \forall A \in ©A\ \forall B \in ©B$, tj. $\tau = \mu \otimes \nu$ na $©O$. Systém $©O$ je $\pi$-systém. Protože $\mu$ a $\nu$ jsou $\sigma$-konečné, tak existují množiny $X_n \in ©A$, $\mu(X_n) < +∞\ \forall n \in ®N$, $X_n \nearrow X$ a množiny $Y_n \in ©B$, $\nu(Y_n) < +∞\ \forall n \in ®N$, $Y_n \nearrow Y$. Pak pro množiny $X_n \times Y_n$ platí $X_n \times Y_n \in ©O$, $(\mu \otimes \nu)(X_n \times Y_n) < +∞\ \forall n \in ®N$, $X_n \times Y_n \nearrow X \times Y$. Z věty 6.1? pak plyne $\tau = \mu \otimes \nu$ na $\sigma ©O$, tj. na $©A \otimes ©B$.
	\end{dukazin}
\end{veta}

% 7. přednáška

\begin{veta}[Fubini]
	Pro každou funkci $f \in ©L^*(\mu \otimes \nu)$ platí

	\begin{itemize}
		\item Funkce $x \mapsto \int_Y f(x, y) d\nu(y)$ je měřitelná na $X$;
		\item Funkce $y \mapsto \int_X f(x, y) d\nu(x)$ je měřitelná na $Y$;
		\item $\int_{X \otimes Y} f(x, y) d(\mu \otimes \nu) = \int_X \(\int_Y f(x, y) d\nu(y)\) d\mu(x) = \int_Y \(\int_X f(x, y) d\mu(x)\) d\nu(y)$.
	\end{itemize}

	\begin{dukazin}
		Je-li $f = \chi_E$, kde $E \in ©A \otimes ©B$, pak 3. plyne z věty o existenci a jednoznačnost součinové míry, neboť
		$$ \int_{X \times Y} \chi_E (x, y) d(\mu \otimes \nu) = (\mu \otimes \nu)(E) = \int_X \nu(E_x) d \mu(x) = \int_X \(\int_Y \chi_E(x, y) d\nu(y)\) d\mu(x), $$
		neboť $\nu(E_x) = \int_Y \chi_{E_x}(y) d\nu(y) = \int_Y \chi_E(x, y) d\nu(y)\ \forall x \in X$.

		Podobně dostaneme
		$$ \int_{X \times Y} \chi_E (x, y) d(\mu \otimes \nu) = (\mu \otimes \nu)(E) = \int_Y \mu(E_y) d \nu(y) = \int_Y \(\int_X \chi_E(x, y) d\mu(x)\) d\nu(y), $$
		neboť $\nu(E^y) = \int_X \chi_{E^y}(x) d\mu(x) = \int_X \chi_E(x, y) d\mu(x)\ \forall y \in Y$. Tedy 3. platí pro $f = \chi_E$, kde $E \in ©A \otimes ©B$.

		Pro jednoduchou nezápornou měřitelnou funkci $s = \sum_{i=1}^k \alpha_i \chi_{E_i}$ na $(X \times Y, ©A \otimes ©B)$ máme
		$$ \int_{X \times Y} s(x, y) d(\mu \otimes \nu) = \sum_{i=1}^k \alpha_i(\mu \otimes \nu)(E_i) = \sum_{i=1}^k \alpha_i \int_X \(\int_Y \chi_{E_i}(x, y) d\nu(y)\) d\mu(x) = $$
		$$ = \int_X \(\int_Y s(x, y) d\nu(y)\) d\mu(x), $$
		tj. pro funkci $s$ platí první rovnost v 3.

		Z uvedeného také plyne, že funkce $x \mapsto \int_Y s(c, y) d\nu(y)$ je měřitelná na $X$. Druhá rovnost v 3. pro funkci $s$ se dokáže analogicky.

		Buď $f ≥ 0$ měřitelná na $(X \times Y, ©A \otimes ©B)$. Dle věty o nezáporné měřitelné funkci existuje posloupnost nezáporných jednoduchých měřitelných funkcí $\{s_n\}$ tak, že $s_n \nearrow f$. Pak podle Leviho věty platí
		$$ \int_Y s_n(x, y) d\nu(y) \nearrow \int_Y f(x, y) d\nu(y) \qquad \forall x \in X. $$
		Protože integrály na levé straně této nerovnosti jsou měřitelnými funkcemi, tak i integrál na pravé je měřitelná funkce na $X$ a další aplikací Leviho věty dostaneme
		$$ \int_X\(\int_Y s_n(x, y) d\nu(y)\)d\mu(x) \nearrow \int_X \(\int_Y f(x, y) d\nu(y)\) d\mu(x). $$

		Druhá nerovnost v 3. se zase dokáže analogicky.

		Pro $f = f^+ - f^- \in ©L^*(\mu \otimes \nu)$ dané tvrzení plyne z příslušných tvrzení pro $f^+$ a $f^-$ (a z linearity integrálu).
	\end{dukazin}
\end{veta}

\begin{veta}[Fubiniova věta pro zúplněnou součinovou míru]
	Nechť $(X, ©A, \mu)$ a $(Y, B, \nu)$ jsou prostory s úplnými $\sigma$-konečnými mírami. Je-li $f \in ©L^*(\mu \overset{0}{\times} \nu)$, pak

	\begin{itemize}
		\item Funkce $f_y: x \mapsto f(x, y)$ je měřitelná na $X$ pro $\nu$-skoro všechna $y \in Y$ a funkce $f_x: y \mapsto f(x, y)$ je měřitelná na $Y$ pro $\mu$-skoro všechna $y \in Y$;
		\item Funkce $x \mapsto \int_Y f(x, y) d\nu(y)$ je měřitelná na $X$ a funkce $y \mapsto \int_X f(x, y) d\nu(x)$ je měřitelná na $Y$;
		\item $\int_{X \otimes Y} f(x, y) d(\mu \overset{0}\otimes \nu) = \int_X \(\int_Y f(x, y) d\nu(y)\) d\mu(x) = \int_Y \(\int_X f(x, y) d\mu(x)\) d\nu(y)$.
	\end{itemize}

	\begin{dukazin}
		Důkaz se nestíhal, pouze bylo zmíněno, že se použije předchozí věta a následující 2 Lemmata.
	\end{dukazin}
\end{veta}

\begin{lemma}
	Nechť $(Z, ©C, \rho)$ je prostor s mírou a $(Z, ©C_0, \rho_0)$ jeho zúplnění. Je-li funkce $f: (Z, ©C_0) \rightarrow ®R^*$ $\rho_0$ měřitelná, pak existuje $\rho$-měřitelná funkce $g: (Z, ©C) \rightarrow ®R^*$ tak, že $f = g$ $\rho$-skoro všude na $X$.

	\begin{dukazin}
		Bez důkazu.
	\end{dukazin}
\end{lemma}

\begin{lemma}
	Nechť $(X, ©A, \mu)$ a $(Y, ©B, \nu)$ jsou prostory s úplnými $\sigma$-konečnými mírami. Nechť $h$ je $\mu \overset{0}\otimes \nu$-měřitelná funkce na $X \times U$ a $h = 0$ $\mu \overset{0}\otimes \nu$-skoro všude na $X \times Y$. Potom pro $\mu$-skoro všechna $x \in X$ platí $h(x, y) = 0$ pro $\nu$-skoro všechna $y \in Y$. Speciálně, funkce $h_x$ je měřitelná na $(Y, ©B, \nu)$ pro $\mu$-skoro všechna $x \in X$. (Obdobně pro $h^y$).

	\begin{dukazin}
		Bez důkazu.
	\end{dukazin}
\end{lemma}

\begin{veta}[O míře $\lambda^p \otimes \lambda^q$]
	Je-li $p, q \in ®N$, pak:
	$$ ©B(®R^{p + q}) = ©B(®R^p) \otimes ©B(®R^q), \qquad (\text{tj. } \lambda_{©B}^{p + q} = \lambda_{©B}^p \otimes \lambda_{©B}^q) $$
	$$ \lambda^{p + q} = \lambda^p \overset{0}\otimes \lambda^q. $$

	\begin{dukazin}
		Bez důkazu.
	\end{dukazin}
\end{veta}

\begin{veta}[Fubiniova věta v $®R^{p + q}$]
	Je-li $s, q \in ®N$ a $f \in ©L^*(\lambda^{p + q})$, pak
	$$ \int_{®R^{p + q}} f d\lambda^{p + q} = \int_{®R^p} \(\int_{®R^q} f(x, y) d\lambda^q(y)\) d\lambda^p(x) = \int_{®R^q} \(\int_{®R^p} f(x, y) d\lambda^p(x)\) d\lambda^q(y). $$

	\begin{dukazin}
		Bez důkazu, ale lehký důsledek předchozí věty a Fubiniovy věty.
	\end{dukazin}
\end{veta}

\begin{definice}[Značení]
	Je-li $p, q \in ®N$, $x \in ®R^p$, $y \in ®R^q$, pak definujeme projekce předpisem
	$$ \pi_1(x, y) := x, \qquad \pi_2(x, y) := y. $$
\end{definice}

\begin{dusledek}
	Nechť $p, q \in ®N$, $A \in ©B_0^{p + q} := (©B(®R^{p + q}))_0$. Jestliže $f \in ©L^*(\lambda^{p + q})$ a množiny $\pi_1 A, \pi_2 A$ jsou měřitelná, pak
	$$ \int_A f d\lambda^{p + q} = \int_{\pi_1 A} \(\int_{A_x} f(x, y) d\lambda^q(y)\)d\lambda^p(x) = \int_{\pi_2 A} \(\int_{A^y} f(x, y) d\lambda^p(x)\)d\lambda^q(y). $$
\end{dusledek}

% 8. přednáška

\begin{lemma}
	Lebesgueova míra $\lambda^n$ je translačně invariantní, tzn.
	$$ \lambda^n(B + r) = \lambda^n(B) \qquad \forall B \in ©B_0^n\ \forall r \in ®R^n. $$

	\begin{dukazin}
		Dané tvrzení plyne z věty o jednoznačnosti míry, neboť míry $\lambda^n$ a $\mu(B) := \lambda^n(B + z)\ \forall B \in ©B_0^n$ a pro libovolné pevné $r \in ®R^n$ se shodují na systému $©B_0^n$.
	\end{dukazin}
\end{lemma}

\begin{veta}[O obrazu míry]
	Nechť $(X, ©A, \mu)$ je prostor s mírou a $(Y, ©B)$ je měřitelný prostor. Je-li $\phi: (X, ©A) \rightarrow (Y, ©B)$ měřitelné zobrazení, pak množinová funkce daná předpisem
	$$ (\phi(\mu))(B) := \mu\(\phi^{-1}(B)\) \qquad \forall B \in ©B $$
	je míra na $(Y, ©B)$ (tzn. obraz míry $\mu$ při zobrazení $\phi$) a pro každou měřitelnou funkci $f$ na $Y$ platí
	$$ \int_Y f d\phi(\mu) = \int_X (f \circ \phi) d\mu, $$
	pokud alespoň jedna strana má smysl.

	\begin{dukazin}
		Snadno ověříme, že množinová funkce $\phi(\mu)$ daná prvním předpisem má všechny vlastnosti míry. Ověření druhé rovnosti je také jednoduché: Nejprve se ověří pro případ, že $f = \chi_B$, kde $B \in ©B$.

		V tomto případě je se pravá strana druhé rovnosti rovná levé:
		$$ \int_X (\chi_B \circ \psi) d\mu = \int_X \chi_{\psi^{-1}(B)} d\mu = \mu\(\psi^{-1}(B)\) = $$
		$$ (\psi(\mu))(B) = \int_X \chi_B d\psi(\mu). $$
		
		S použitím tohoto výsledku se druhá rovnost ověří pro jednoduché funkce, pak pro nezáporné měřitelné funkce a nakonec se tento výsledek spolu s rovností $f = f^+ - f^-$ použije k ověření rovnosti pro měřitelnou funkci $f$ na $Y$.
	\end{dukazin}
\end{veta}

\begin{veta}
	Buď $L: ®R^n \rightarrow ®R^n$ invertibilní lineární zobrazení

	\begin{itemize}
		\item Je-li $\nu(A) := \lambda^n(L(A))\ \forall A \in ©B^n := ©B(®R^n)$, pak $\nu$ je mír a platí $\nu = |\det L|·\lambda^n$.
		\item Je-li $\mu(A) := |\det L|\lambda_{©B}^n(A)\ \forall A \in ©B^n$, pak $L(\mu) = \lambda_{©B}^n$ a pro $f \in ©L^*(\lambda_{©B}^n)$ platí
			$$ \int_{®R^n} f d\lambda^n = \int_{®R^n} (f \circ L) |\det L| d\lambda^n. $$
	\end{itemize}

	\begin{dukazin}
        % Z live přednášky (nechtělo se mi opisovat)
		1. $L$ je lineární zobrazení z $®R^n$ do $®R^n$, a tedy $L$ je spojité. $L$ je invertibilní $\implies \exists$inverzní zobrazení $L^{-1}$, které je opět lineární a spojité. Tedy $L$ je měřitelné.
		$$ (L^{-1}\lambda^n)(A) = \lambda^n(L(A)) = \nu(A), \forall A \in ©B^n $$
		$\implies$ $nu$ je míra dle předchozí věty.

		Z lineární algebry je známo, že $L$ lze vyjádřit jako kompozici konečně mnoha „elementárních“ lineárních zobrazení jednoho z následujících typů:
		$$ L_1(x_1, …, x_n) = (\alpha x_1, x_2, …, x_n), \qquad \forall (x_1, …, x_n) \in ®R^n, \text{ kde }\alpha \in ®R\setminus\{0\}, $$
		$$ L_2(x_1, …, x_n) = (x_1, …, x_j, …, x_i, …, x_n), \qquad \forall (x_1, …, x_n) \in ®R^n, j > i \in ®N, $$
		$$ L_3(x_1, …, x_n) = (x_1 + x_2, x_2, …, x_n), \qquad \forall (x_1, …, x_n) \in ®R^n, \qquad i, j \in ®N. $$

		Protože determinant součinu matic se rovná součinu determinantů, stačí tvrzení ověřit pro „elementární“ zobrazení. Ověříme na intervalech, $L_1$ ho jen natáhne o $\alpha$, tedy na determinant násobek, $L_2$ „otočí“ interval, ale $\lambda^n$ se otočením nezmění, $L_3$ posune a zdeformuje interval, ale tím se $\lambda^n$ nezmění (dokážeme přes Fubiniovu větu). Všechny 3 zobrazení stejně operují na prázdné množině, takže i na $\pi$ systému $I \cup \{\O\}$, tedy míry se rovnají všude.

		2.
		$$ (L(\mu))(A) \overset{1.}{=} \mu(L^{-1}(A)) = |\det L| \lambda_{©B}^n(L^{-1}(A)) = |\det L|·|\det L^{-1}| \lambda_{©B}^n(A) = \lambda_{©B}^n(A) \forall A \in ©B^n, $$
		tedy $L(\mu) = \lambda_{©B}^n$. Z předchozí věty pak plyne rovnost integrálů.
	\end{dukazin}
\end{veta}

\begin{lemma}
	Buď $T: ®R^n \rightarrow ®R^n$ Lipschitzovské zobrazení. Je-li $A \subset ®R^n$ lebesgueovsky měřitelná množina, pak také $T(A)$ je lebesgueovsky měřitelná množina.

	\begin{dukazin}
		Vynechán.
	\end{dukazin}
\end{lemma}

\begin{veta}
	Je-li $L: ®R^n \rightarrow ®R^n$ invertibilní lineární zobrazení, pak
	$$ \int_{®R^n} f d\lambda^n = \int_{®R^n}(f \circ L) | \det L| d \lambda^n, $$
	má-li jedna strana smysl.

	\begin{dukazin}
		Bez důkazu, ale jednoduše vyplývá z předchozího lemmatu a věty.
	\end{dukazin}
\end{veta}

\begin{veta}[O substituci]
	Buď $G \subset ®R^n$ otevřená množina a $\phi: G \rightarrow ®R^n$ difeomorfismus. Jestliže $f: \phi(G) \rightarrow ®R$ je lebesgueovsky měřitelná funkce, pak
	$$ \int_G f(\phi(x)) |J \phi(x)| dx = \int_{\phi(G)} f(y) dy, $$
	má-li jedna strana rovnosti smysl.

	\begin{dukazin}
		Bude v TMI2.
	\end{dukazin}
\end{veta}

\begin{dusledek}
	Je-li navíc $M \subset \phi(G)$ lebesgueovsky měřitelná množina, pak
	$$ \int_{\phi^{-1}(M)} f(\phi(x)) |J \phi(x)| dx = \int_M f(y) dy. $$
\end{dusledek}

\begin{lemma}
	$$ \lambda^n (®R^{n-1}) = 0. $$

	\begin{dukazin}
		Množina $®R^{n-1}$ je $\lambda^n$-měřitelná, neboť je uzavřená v $®R^n$. Dále platí $®R^{n-1} \subset \bigcup_{k=1}^∞ I_{k, \epsilon}$, kde $\epsilon > 0$ a
		$$ I_{k, \epsilon} := (-k, k)^{n-1} \times \(\frac{- \epsilon}{(2k)^{n-1}}·\frac{1}{2^k}, \frac{\epsilon}{(2k)^{n-1}}·\frac{1}{2^k}\) \qquad \forall k \in ®N, $$
		a tedy
		$$ \lambda^n(®R^{n-1}) ≤ \sum_{k=1}^∞ \lambda^n(I_{k, \epsilon}) = \sum_{k=1}^∞ (2k)^{n-1}·\frac{2 \epsilon}{(2k)^{n-1}}\frac{1}{2^k} = \sum_{k=1}^∞ \frac{\epsilon}{2^{k-1}} = 2\epsilon. $$
		Protože $\epsilon > 0$ bylo libovolné, tak $\lambda^n(®R^{n-1}) = 0$.
	\end{dukazin}
\end{lemma}

% 9. přednáška

% 10. přednáška

\begin{lemma}[O míře s hustotou $f$]
	Buď $(X, ©A, \mu)$ prostor s mírou a $f$ nezáporná měřitelná funkce na $X$. Definujeme-li
	$$ \nu(A) := \int_A f d\mu \qquad \forall A \in ©A, $$
	pak $\nu$ je míra na $©A$ a pro měřitelnou funkci $g:(X, ©A) \rightarrow \<0, +∞\>$ platí
	$$ \int_X g d\nu = \int_X g f d\mu. $$

	\begin{dukazin}
		Je jasné, že $\nu ≥ 0$. Dále
		$$ \nu(\O) = \int_{\O} f d\mu = \int_X f·\chi_{\O} d\mu = \int_X 0 d\mu = 0. $$
		Je-li $A := \bigcup_{j=1}^∞ A_j$, kde $A_j \in ©A$, $A_j \cap A_i = \O$ pro $i ≠ j$, pak
		$$ \nu(A) = \int_A f d\mu = \int_X f·\chi_A d\mu = \int_X f\(\sum_{j=1}^∞ \chi_{A_j}\) d\mu = $$
		$$ = \sum_{j=1}^∞ \int_X f \chi_{A_j} d\mu = \sum_{j=1}^∞ \int_{A_j} f d\mu = \sum_{j=1}^∞ \nu(A_j). $$
		Tedy $\nu$ je míra na ©A.

		K důkazu rovnosti: Je-li $g := \chi_E$, kde $E \in ©A$, pak
		$$ \int_X g d\nu = \int_X \chi_E d\nu = \nu(E) = \int_E f d\mu = \int_X \chi_E f d\mu = \int_X g f d\mu, $$
		tj. rovnost platí. Z toho a linearity integrálu plyne, že rovnost platí v případě, že $g$ je jednoduchá nezáporná měřitelná funkce na $X$.

		Nakonec je-li $g: (X, ©A) \rightarrow \<0, +∞\>$ měřitelná funkce, pak existují nezáporné jednoduché měřitelné funkce $g_n$ splňující $g_n \nearrow g$. Odtud z předchozího a Leviho věty pak plyne rovnost i pro $g$.
	\end{dukazin}
\end{lemma}

\begin{veta}[Charakterizace faktu $\nu \ll \mu$ pro konečné míry]
	Nechť $\mu$, $\nu$ jsou konečné míry na $(X, ©A)$. Pak $\nu \ll \mu$ právě tehdy, když
	$$ \forall \epsilon > 0\ \exists \delta > 0\ \forall A \in ©A: \mu(A) < \delta \implies \nu(A) < \epsilon. $$

	\begin{dukazin}
		„$\Leftarrow$“: Buď $A \in ©A$, $\mu(A) = 0$. Pak volbou $\epsilon = \frac{1}{k}, k \in ®N$, ze znění dostaneme
		$$ \forall k \in ®N\ \exists \delta_k > 0\ \forall A \in ©A, \mu(A) < \frac{1}{k}: \nu(A) < \frac{1}{k}. $$
		Protože $\mu(A) = 0 < \delta_k\ \forall k \in ®N$, tak $\nu(A) < \frac{1}{k}\ \forall k \in ®N$, tj. $\nu(A) = 0$.

		„$\implies$“: Nechť $\nu \ll \mu$ a předpokládejme, že podmínka ze znění neplatí, tj.
		$$ \exists \epsilon > 0\ \forall \delta > 0\ \exists A \in ©A: \mu(A) < \delta \land \nu(A) ≥ \epsilon. $$
		Volme $\delta = \frac{1}{2^n}, n \in ®N$. Tedy dle předchozího platí
		$$ \exists A_n \in ©A, \mu(A_n) < \frac{1}{2^n} \land \nu(A_n) ≥ \epsilon\ \forall n \in ®N. $$
		Položme $B_k := \bigcup_{n=k+1}^∞ A_n, k \in ®N$. Pak
		$$ B_1 \supset B_2 \supset …, \mu(B_1) ≤ \mu(X) < +∞, \nu(B_1) ≤ \nu(X) < +∞, $$
		a tedy
		$$ \mu\(\bigcap_{k=1}^∞ B_k\) = \lim_{k \rightarrow ∞} \mu(B_k), $$
		$$ \nu\(\bigcap_{k=1}^∞ B_k\) = \lim_{k \rightarrow ∞} \nu(B_k). $$
		Protože
		$$ \mu(B_k) = \mu\(\bigcup_{n = k + 1}^∞ A_n\) ≤ \sum_{n = k + 1}^∞ \frac{1}{2^n} = \frac{1}{2^{k+1}}·\frac{1}{1 - \frac{1}{2}} = \frac{1}{2^k}, $$
		tj. $\mu(B_k) ≤ \frac{1}{2^k}$, tak $\lim_{k \rightarrow ∞} \mu(B_k) = 0$.

		Dále platí $\nu(B_k) = \nu\(\bigcup_{n=k+1}^∞ A_n\) ≥ \nu(A_{k+1}) ≥ \epsilon\ \forall k \in ®N$, a tedy $\lim_{k \rightarrow ∞} \nu(B_k) ≥ \epsilon$. Tedy pro $B := \bigcap_{k=1}^∞ B_k$ dle předchozích výsledků důkazu máme $\mu(B) = 0 \land \nu(B) ≥ \epsilon$, což je spor, neboť $\nu \ll \mu$. Tedy podmínka z věty platí.
	\end{dukazin}
\end{veta}

\begin{veta}[Radonova-Nikodymova]
	Jsou-li $\mu$, $\nu$ $\sigma$-konečné míry na $(X, ©A)$ splňující $\mu \ll \nu$, pak existuje nezáporná měřitelná funkce $f$ na $X$ tak, že
	$$ \nu(A) = \int_A f d\mu \qquad \forall A \in ©A. $$

	\begin{dukazin}
		% Z 11. přednášky
		Nejprve předpokládejme, že $\mu$, $\nu$ jsou konečné míry na $(X, ©A)$ a $\nu \ll \mu$. Dle následujícího lemmatu aplikovaného na míry $\nu$, $\nu + \mu$, splňující $\nu ≤ \mu + \nu$, existuje měřitelná funkce $h$ na $X$, $0 ≤ h ≤ 1$ $(\mu + \nu)$-skoro všude tak, že
		$$ \nu(A) = \int_A h d(\mu + \nu) = \int_A h d\mu + \int_A h d\nu \qquad \forall A \in ©A \implies $$
		$$ \implies \nu(\{h = 1\}) = \int_{\{h=1\}} h d(\mu + \nu) = \mu(\{h = 1\}) + \nu(\{h = 1\}), $$
		a tedy $\mu(\{h = 1\}) = 0$. Protože $\nu \ll \mu$, tak také $\nu(\{h = 1\}) = 0$. Proto $h < 1$ $(\mu + \nu)$-skoro všude.

		Rovnost výše lze psát ve tvaru
		$$ \int_X \chi_A d\nu = \int_X \chi_A h d\mu + \int_X \chi_A h d\nu, $$
		tj.
		$$ \int_X \chi_A (1 - h) d\nu = \int_X \chi_A h d\mu \qquad \forall A \in ©A. $$
		Odtud a z linearity integrálu pak plyne, že platí $\int_X g (1 - h) d\nu = \int_X g h d\mu$ pro všechny nezáporné jednoduché $\mu$-měřitelné funkce $g$ na $X$. Pomocí Leviho věty lze ukázat, že tato rovnost platí pro každou nezápornou $\mu$-měřitelnou funkci $g$ na $X$. Volbou $g := \frac{1}{1 - h} \chi_A$, kde $A \in ©A$, pak dostaneme
		$$ \int_X \chi_A d\nu = \int_X \frac{h}{1 - h} \chi_A d\mu \implies \nu(A) = \int_A f d\mu\ \forall A \in ©A, $$
		kde $f := \frac{h}{1 - h}$ je hledaná hustota $\frac{d\nu}{d\mu}$.

		Jsou-li $\mu, \nu$ $\sigma$-konečné míry, pak nalezneme posloupnosti $\{E_i\}, \{F_j\} \subset ©A$ po dvou disjunktních množin tak, aby $\mu(E_i) < +∞\ \forall i \in ®N$, $\nu(F_j) < +∞\ \forall j \in ®N$, $\bigcup_{i=1}^∞ E_i = X = \bigcup_{j=1}^∞ F_j$.

		Položíme-li $D_{ij} := E_i \cap F_j, i, j \in ®N$, pak $X = \bigcup_{i,j=1}^∞ D_{ij}$ a pro konečné míry $\nu|_{D_{ij}}$, $\mu|_{D_{ij}}$ (tj. restrikce daných měr $\nu$, $\mu$ na $D_{ij}$) splňující $\nu|_{D_{ij}} \ll \mu|_{D_{ij}}$ určíme příslušnou hustotu $f_{ij} = \frac{d\nu|_{D_{ij}}}{d\mu|_{D_{ij}}}$. Hledaná hustota $f = \frac{d\nu}{d\mu}$ je pak definovaná takto:

		Je-li $x \in X$, pak $\exists!i \in ®N$, $\exists!j \in ®N$ tak, že $x \in D_{ij}$ a položíme $f(x) = f_{ij}(x)$.
	\end{dukazin}
\end{veta}

\begin{lemma}[Radonova-Nikodymova věta -- baby verze]
	Jestliže $\mu$, $\nu$ jsou konečné míry na $(X, ©A)$ takové, že $\nu(A) ≤ \mu(A)\ \forall A \in ©A$, pak existuje měřitelná funkce $f$ na $X$ splňující $0 ≤ f ≤ 1$ $\mu$-skoro všude a
	$$ \nu(A) = \int_A f d\mu \qquad \forall A \in ©A. $$

	\begin{dukazin}
        % Z live přednášky (nechtělo se mi opisovat)
		Definujeme funkcionál $Jg := \int_X g^2 d\mu - 2\int_X g d\nu$, $\forall g \in ©L^2(\mu)$.
		Definice $J$ je korektní, protože konvergence v $L^2$ je silnější než konvergence v $L^1$ (nebo z Hölderovy nerovnosti), tedy oba integrály jsou pro $g \in L^2$ konečné. Dále definujeme $c := \inf_{g \in L^2(\mu)} Jg$.
		$$ Jg = \int_X g^2 d\mu - 2 \int_X g d\nu ≥ \int_X g^2 d\mu - 2\int_X |g| d\mu = $$
		$$ = \int_X(|g| - 1)^2 d\mu - \mu(X) ≥ -\mu(X) > -∞, \forall g \in L^2(\mu). $$

		Předpokládejme, že $\exists f\: c = Jf$. Buď $A \in ©A$ pevná množina, definujeme $g(t) := J(f + t\chi_A)$, $\forall t \in ®R$. Tedy $g$ má minimum v bodě $0$. Tudíž $g'(0) = 0$, pokud $g'$ existuje. Ověříme výpočtem z definice existenci a dosadíme 0:
		$$ g'(0) = \lim_{t \rightarrow 0} \frac{g(t) - g(0)}{t} = \lim_{t \rightarrow 0} \frac{J(f + t\chi_A) - J(f)}{t} = $$
		$$ = \lim_{t \rightarrow 0} \frac{1}{t}\[\int_X (f + t\chi_A)^2 d\mu - 2\int_X(f + t\chi_A) d\nu - \int_X f^2 d\mu + 2\int_X fd\nu\] = $$
		$$ \lim_{t \rightarrow 0} \[\int_X 2f \chi_A d\mu + t\int_X \chi_A d\mu - 2 \int_X \chi_A d\nu\] = 2\[\int_X f \chi_A d\mu - \int_X \chi_A d\nu\] = 0 $$
		Tedy $\forall A \in ©A: \nu(A) = \int_A f d\mu$.

		$$ 0 ≤ \int_{\{f > 1\}}(f - 1)^+ d\mu = \int_{\{f > 1\}} (f - 1) d\mu = \int_{\{f > 1\}} d\mu - \int_{\{f > 1\}} 1d\mu = $$
		$$ = \nu(\{f > 1\}) - \mu(\{f > 1\}) ≤ 0 \implies f ≤ 1 \mu\text{-skoro všude} $$
		$$ 0 ≤ \int_{\{f < 0\}} f^- d\mu = - \int_{\{f < 0\}} f d\mu = - nu(\{f < 0\}) ≤ 0 \implies f ≥ 0 \mu\text{-skoro všude} $$

		$$ J(g) + J(h) - J\(\frac{g + h}{2}\) = \int_X \frac{g^2 - 2gh + h^2}{2} d\mu = \frac{1}{2} \int_X (g - h)^2 d\mu = \frac{1}{2} ||g - h||_{L^2(\mu)}^2. $$
		$\exists \{f_n\} \subset L^2(\mu)$. $J(f_n) \rightarrow c$ pro $n \rightarrow ∞$. $g = f_n$, $h = f_m$:
		$$ J(f_n) + J(f_m) - 2J\(\frac{f_n + f_m}{2}\) = \frac{1}{2} ||f_n - f_m||_{L^2(\mu)}^2, \forall n, m \in ®N $$
		$$ ≤ J(f_n) + J(f_n) - 2c \rightarrow 0 \implies \exists f \in L^2(\mu): f_n \rightarrow f \in L^2(\mu). $$
		$$ \int_X |f_n - f|d\nu ≤ \int_X |f_n - f| d\mu ≤ \(\int_X |f_n - f|^2\)^{\frac{1}{2}}·(\mu(X))^{\frac{1}{2}} \rightarrow 0 \implies $$
		$$ \implies ||f_n - f||_{L^2(M)} \rightarrow 0 \implies J(f_n) \rightarrow J(f). $$
	\end{dukazin}
\end{lemma}

% 11. přednáška

\begin{veta}[Lebesgueův rozklad míry]
	Buď $\mu$ míra na $(X, d)$ a $\nu$ $\sigma$-konečná míra na $(X, ©A)$. Pak existuje rozklad $\nu = \nu_a + \nu_s$ na $\sigma$-konečné míry $\nu_a$ a $\nu_s$ takový, že $\nu_a \ll \mu$, $\nu_s \perp \mu$, přičemž míry $\nu_a$ a $\nu_s$ jsou určeny jednoznačně.

	\begin{dukazin}[Konečná míra, existence rozkladu]
		Předpokládejme nejprve, že $\nu$ je konečná míra. Nejprve se zabývejme existencí rozkladu:

		Buď $©N_\mu := \{B \in ©A, \mu(B) = 0\}$. Pak
		$$ c := \sup \{\nu(B) | B \in ©N_\mu\} ≤ \nu(X) < +∞. $$
		Nechť $\{B_j\}_j \subset ©N_\mu$ je taková posloupnost, že $\lim_{j \rightarrow ∞} \nu(B_j) = c$. Označíme-li $N := \bigcup_{j=1}^∞ B_j$, pak $\mu(N) ≤ \sum_j \mu(B_j) = 0$, a tedy $\mu(N) = 0$, tj. $N \in ©N_\mu$.

		Dále platí
		$$ c ≥ \nu(N) = \nu(\bigcup_j B_j) ≥ \nu(i) \qquad \forall i \in ®N \implies \nu(N) = c. $$

		Definujeme
		$$ \nu_s(A) := \nu(A \cap N) \qquad \forall A \in ©A. $$
		Pak
		$$ \nu_s(X \setminus N) = \nu((X \setminus N) \cap N) = \nu(\O) = 0. $$
		Odtud a z $\mu(N) = 0$ plyne $\nu_s \perp \mu$. Následně definujeme $\nu_a := \nu - \nu_s$. Tedy
		$$ \nu_a(A) = \nu(A) - \nu_s(A) = \nu(A) - \nu(A \cap N) = \nu(A \setminus N) = \nu(A \cap N^c), $$
		tj.
		$$ \nu_a(A) = \nu(A \cap N^c) \qquad \forall A \in ©A. $$
		Dokažme, že $\nu_a \ll \mu$: Nechť $\mu(a) = 0$. Pak
		$$ N \cup (A \cap N^c) \in ©N_\mu, $$
		a kdyby $\nu(A \cap N^c) > 0$, pak by
		$$ \nu(N \cup (A \cap N^c)) = \nu(N) + \nu(A \cap N^c) > c, $$
		což je spor s definicí čísla $c$. Tedy $\nu(A \cap N^c) = 0$, tj. $\nu_a(A) = 0$, tudíž $\nu_a \ll \mu$.
	\end{dukazin}

	\begin{dukazin}[Konečná míra, jednoznačnost rozkladu]
		Nechť
		$$ \nu = \nu_a + \nu_s \land \nu = \tilde \nu_a + \tilde \nu_s, $$
		$$ \nu_s \perp \mu \land \tilde \nu_s \perp \mu \land \nu_a \ll \mu \land \tilde \nu_a \ll \mu. $$
		Ze singularit měr plyne
		$$ \exists N \in ©A: \mu(N) = 0 \land \nu_s(N^c) = 0, $$
		$$ \exists \tilde N \in ©A: \mu(\tilde N) = 0 \land \tilde \nu_s(\tilde N^c) = 0. $$
		Buď $N_0 := N \cup \tilde N$. Pak $\mu(N_0) ≤ \mu(N) + \mu(\tilde N) = 0$, a tedy $\mu(N_0) = 0$, odkud plyne $\nu_a(N_0) = 0 \land \tilde \nu_a(N_0) = 0$.

		Dále platí
		$$ \nu_s(N_0^c) = \nu_s (X \setminus N_0) ≤ \nu_s(N^c) = 0, $$
		$$ \tilde \nu_s(N_0^c) = \tilde \nu_s (X \setminus N_0) ≤ \tilde \nu_s(\tilde N^c) = 0, $$
		tj. $\nu_s(N_0^c) = 0 \land \tilde \nu_s(N_0^c) = 0$. Tedy $\forall A \in ©A$ platí
		$$ \nu_s(A) = \nu_s(A \cap N_0) = \nu(A \cap N_0) - \nu_a(A \cap N_0) = \nu(A \cap N_0) $$
		a analogicky
		$$ \tilde \nu_s(A) = \tilde \nu_s(A \cap N_0) = \nu(A \cap N_0) - \tilde \nu_a(A \cap N_0) = \nu(A \cap N_0), $$
		odkud dostáváme $\nu_s = \tilde \nu_s$, což spolu s tím, že je to rozklad dává $\nu_a = \tilde \nu_a$.
	\end{dukazin}

	\begin{dukazin}[Sigma-konečná míra]
		Předpokládejme nyní, že $\nu$ je $\sigma$-konečná míra. Pak existuje posloupnost $\{D_k\}_{k \in ®N} \subset ©A$ po dvou disjunktních množin tak, že $X = \bigcup_k D_k$. Označme $A_k := \{A \cap D_k | A \in ©A\}$ a aplikujeme stejný postup jako pro konečnou míru na měřitelné prostory $(D_k, ©A_k)$ a restrikce měr $\mu$, $\nu$ na $©A_k$, $k \in ®N$.

		Nechť $N_1, N_2, …$ jsou $\mu$-nulové množiny zkonstruovatelné jako množina $N$ v části $I$ a nechť $N = \bigcup_{k=1}^∞ N_k$. Pak míry $\nu_s$, $\nu_a$ definované předpisem
		$$ \nu_s(A) := \nu(A \cap N), \qquad \nu_a(A) = \nu(A \cap N^c) \qquad \forall A \in ©A $$
		tvoří Lebesgueův rozklad míry $\nu$, neboť
		$$ \mu(N) = \mu(\bigcup_k N_k) = \sum_k \mu(N_k) = 0, $$
		$$ \nu_s(X \setminus N) = \nu((X \setminus N) \cap N) = \nu(\O) = 0, $$
		a tedy $\nu_s \perp \mu$.

		Je-li $\mu(A) = 0$ a označíme-li $A_k = A \cap D_k$, $k \in ®N$, pak $\mu(A_k) = 0$, a tedy
		$$ (\nu|_{D_k})_a(A_k) = 0 \qquad \forall k \in ®N \qquad (\text{neboť } (\nu_{D_k})_a \ll \mu|_{D_k}). $$
		Dále platí
		$$ \nu_a(A) = \nu(A \cap N^c) = \sum_k \nu(A \cap D_k \cap N^c), $$
		a protože
		$$ D_k \cap N^c = D_k \cap (X \setminus N) = D_k \cap (X \setminus \bigcup_j N_j) = D_k \cap \bigcup_j (X \setminus N_j) \subset D_k \cap (X \setminus N_k) = D_k \setminus N_k, $$
		tak
		$$ \nu_a(A) ≤ \sum_k \nu(A \cap D_k \cap (D_k \setminus N_k)) = \sum_k \nu(A_k \cap (D_k \setminus N_k)) = \sum_k (\nu|_{D_k})(A_k \cap (D_k \setminus N_k)) = $$
		$$ = \sum_k(\nu|_{D_k})_a(A_k) = 0, $$
		a tedy $\nu_a \ll \mu$.

		Jednoznačnost rozkladu $\nu = \nu_a + \nu_s$ plyne z faktů, že $\forall A \in ©A$ platí
		$$ \nu(A) = \sum_k \nu|_{D_k} (A \cap D_k), \qquad \nu_s(A) = \sum_k(\nu|_{D_k})_s (A \cap D_k), $$
		$$ \nu_a(A) = \sum_k (\nu|_{D_k})_a (A \cap D_k \setminus N_k) $$
		a „lokální rozklady“ $\nu|_{D_k} = (\nu|_{D_k})_s + (\nu|_{D_k})_a$, $k \in N$, jsou určeny jednoznačně.
	\end{dukazin}
\end{veta}

% 12. přednáška

\begin{lemma}[O distribuční funkci]
	Distribuční funkce $F_\mu$ splňuje:

	\begin{itemize}
		\item $F_\mu$ je neklesající;
		\item $F_\mu(-∞) := \lim_{x \rightarrow -∞} F_\mu(x) = 0$, $F_\mu(+∞) := \lim_{x \rightarrow +∞} F_\mu < ∞$;
		\item $F_\mu$ je zprava spojitá.
	\end{itemize}

	\begin{dukazin}
		První bod je jednoduchý: Je-li $x, y \in ®R$, $x < y$, pak
		$$ F_\mu(x) = \mu(\(-∞, x\>) ≤ \mu(\(-∞, y\>) = F_\mu(y). $$

		Druhý bod: Obě uvedené limity existují, neboť dle prvního bodu je funkce $F_\mu$ neklesající. Tedy
		$$ F_{\mu}(-∞) = \lim_{n \rightarrow ∞} F_{\mu}(-n) = \lim_{n \rightarrow ∞} \mu(\(-∞, -n\>) = $$
		$$ = \mu\(\bigcap_{n=1}^∞ \(-∞, -n\>\) = \mu(\O) = 0. $$

		Analogicky dostaneme
		$$ F_{\mu}(+∞) = \lim_{n \rightarrow ∞} F_{\mu}(n) = \lim_{n \rightarrow ∞} \mu(\(-∞, n\>) = $$
		$$ = \mu\(\bigcap_{n=1}^∞ \(∞, n\>\) = \mu(®R) < +∞. $$

		Třetí bod je také jednoduchý: Je-li $x \in ®R$, pak $\(-∞, x\> = \bigcap_{n=1}^∞ \(-∞, x + \frac{1}{n}\>$, a tedy
		$$ F_\mu(x+) = \lim_{n \rightarrow ∞} F_\mu\(x + \frac{1}{n}\) = \lim_{n \rightarrow ∞} \mu\(\(-∞, x + \frac{1}{n}\>\) = $$
		$$ = \mu\(\bigcap_{n=1}^∞ \(-∞, x + \frac{1}{n}\>\) = \mu(\(-∞, x\>) = F_\mu(x). $$
	\end{dukazin}
\end{lemma}

\begin{veta}[O Lebesgueově-Stieltjosově míře]
	Je-li $F: ®R \rightarrow ®R$ funkce splňující

	\begin{itemize}
		\item $F$ je neklesající;
		\item $F_\mu(-∞) := \lim_{x \rightarrow -∞} F_\mu(x) = 0$, $F_\mu(+∞) := \lim_{x \rightarrow +∞} F_\mu < ∞$;
		\item $F_\mu$ je zprava spojitá.
	\end{itemize}

	pak existuje právě jedna konečná borelovská míra na ®R (tzn. Lebesgueova-Stieltjesova míra příslušná funkci $F$) taková, že $F_\mu = F$.

	\begin{dukazin}
		Bude v TMI2.
	\end{dukazin}
\end{veta}

\begin{veta}[Per partes pro L-S integrál]
	Jestliže $F$, $G$ jsou distribuční funkce a $-∞ < a < b < + ∞$, pak
	$$ F(b)G(b) - F(a)G(a) = \int_{\(a, b\>} F(x) dG(x) + \int_{\(a, b\>} G(x) dF(x). $$

	\begin{dukazin}
		Nechť $\Omega := \{[x, y] \in ®R^n | a < x ≤ y ≤ b\}$. Použitím Fubiniovy věty k výpočtu $(\mu_F \otimes \mu_G)(\Omega)$ obdržíme
		$$ (\mu_F \otimes \mu_G)(\Omega) = \int_{\(a, b\>} \(\int_{\<x, b\>} d G(y)\)d F(x) = \int_{\(a, b\>} (G(b) - G(x-)) d F(x) = $$
		$$ = G(b)(F(b) - F(a)) - \int_{\(a, b\>} G(x-) dF(x), $$
		$$ (\mu_F \otimes \mu_G)(\Omega) = \int_{\(a, b\>} \(\int_{\(, y\>} d F(x)\) d G(y) = \int_{\(a, b\>} (F(y) - F(a))d G(y) = $$
		$$ = \int_{\(a, b\>} F(y) d G(y) - F(a)(G(b) - G(a)). $$

		Odečteme-li předchozí od sebe, dostaneme
		$$ 0 = G(b)F(b) - G(b)F(a) - \int_{\(a, b\>} G(x-) d F(x) - \int_{\(a, b\>} F(y) d G(y) + F(a)G(b) - F(a)G(a). $$
	\end{dukazin}
\end{veta}

\begin{lemma}[O $\mu \ll \lambda^1$]
	Nechť $\mu$ je konečná borelovská míra na ®R. Jestliže $F_\mu \in C^1(®R)$, pak $\mu \ll \lambda^1$ a $\frac{d \mu}{d \lambda^1} = F_\mu'$. (Tj. platí $\mu(A) = \int_A F_\mu' d\lambda^1\ \forall A \in ©B(®R)$.)

	\begin{dukazin}
		Nechť ©S je systém, který se skládá z $\O$ a všech intervalů $\(a, b\>$, kde $-∞ < a < b < +∞$. Pak ©S je $\pi$-systém. Buď $\nu$ míra daná předpisem
		$$ \nu(A) := \int_A F_\mu' d\lambda^1 \qquad \forall A \in ©B(®R). $$
		Pak $\mu = \nu$ na ©S, neboť
		$$ \mu(\O) = 0 = \nu(\O), $$
		$$ \mu(\(a, b\>) = F_\mu(b) - F_\mu(a) = \int_a^b F_\mu'(x) dx = \int_{\(a, b\>} F_\mu' d\lambda^1 = \nu(\(a, b\>), $$
		je-li $-∞ < a < b < +∞$.

		Dále platí $X_n := \(-n, n\> \in ©S$, $X_n \nearrow X := ®R$, $\mu(X_n) < +∞ \forall n \in ®N$. Proto, dle věty o jednoznačnosti míry platí $\mu = \nu$ na $\sigma ©S = ©B(®R)$. Tedy
		$$ \mu(A) = \nu(A) = \int_A F_\mu' d\lambda^1\ \forall A \in ©B(®R), $$
		tj. $\frac{d\mu}{d\lambda^1} = F_\mu'$.
	\end{dukazin}
\end{lemma}

% 13. přednáška

\begin{lemma}[Čebyševova nerovnost]
	Je-li $1 ≤ p < +∞$, $f \in L^p(\mu)$ a $c \in (0, +∞)$, pak
	$$ \mu(\{x \in X\ |\ |f(x)| ≥ c\}) ≤ \(\frac{||f||_p}{c}\)^p. $$

	\begin{dukazin}
		$$ \mu(\overbrace{\{x \in X\ |\ |f()x| ≥ c\}}^{M :=}) = \int_M 1 d\mu ≤ \int_M \(\frac{|f|}{c}\)^p d\mu ≤ \int_X \(\frac{|f|}{c}\)^p d\mu = \(\frac{||f||_p}{c}\)^p. $$
	\end{dukazin}
\end{lemma}

\begin{veta}[Vztah mezi konvergencí v $L^p(\mu)$ a konvergencí podle míry]
	Je-li $1 ≤ p ≤ +∞$ a $f, f_n \in L^p(\mu)$ ($\forall n \in ®N$), pak
	$$ f_n \overset{L^p(\mu)}\rightarrow f \implies f_n \overset{\mu} \rightarrow f. $$

	\begin{dukazin}
		Je-li $p \in \<1, +∞\)$, pak implikace plyne z Čebyševovy nerovnosti. Jinak předpokládejme $f_n \overset{L^p(\mu)}\rightarrow f$ a $\epsilon > 0$, pak
		$$ \exists n_0 \in ®N\ \forall n \in ®N, n ≥ n_0: ||f_n - f||_∞ < \epsilon, $$
		a tedy $\mu(\{x \in X\ |\ |f_n(x) - f(x)| ≥ \epsilon\}) = 0\ \forall n \in ®N$, $n ≥ n_0$. Proto $f_n \overset{\mu}\rightarrow f$.
	\end{dukazin}
\end{veta}

\begin{veta}[1. vztah mezi konvergencí podle míry a konvergencí skoro všude]
	Jestliže $(X, ©A, \mu)$ je prostor s mírou a $f_n \overset{\mu}\rightarrow f$, pak existuje vybraná podposloupnost $\{f_{n_k}\}_{k \in ®N}$ tak, že $f_{n_k} \rightarrow f$ $\mu$-skoro všude.

	\begin{dukazin}
		Protože
		$$ \lim_{n \rightarrow ∞} \mu(\{x \in X\ |\ |f_n(x) - f(x)| ≥ \epsilon\}) = 0 \qquad \forall \epsilon > 0, $$
		tak lze konstruovat posloupnost čísel $\{n_k\}_{k \in ®N}$ tak, že
		$$ \mu\(\{x \in X\ |\ |f_{n_1}(x) - f(x)| ≥ 1\}\) ≤ \frac{1}{2} $$
		a zbývající členy posloupnosti $\{n_k\}$ určit induktivně tak, aby $n_k > n_{k-1}$ a
		$$ \mu(\{x \in X\ |\ |f_{n_k}(x) - f(x)|≥ \frac{1}{k}\}) ≤ \frac{1}{2^k}, \qquad k \in ®N \setminus \{1\}. $$

		Definujeme množiny $A_k$, $k \in ®N$, předpisem
		$$ A_k := \{x \in X\ |\ |f_{n_k}(x) - f(x)|≥ \frac{1}{k}\} $$
		a $A := \bigcap_{j=1}^∞ \bigcup_{k ≥ j} A_k$. Jestliže $x \notin A$, pak $\exists j \in ®N$ tak, že $x \notin \bigcup_{k ≥ j}A_k$, tedy
		$$ |f_{n_k}(x) - f(x)| < \frac{1}{k} \qquad \forall k \in \{j, j+1, …\}, $$
		tudíž $\{f_{n_k}\}_k$ konverguje k $f$ pro všechna $x \notin A$.

		Protože $\forall j \in ®N$ platí
		$$ \mu(A) ≤ \mu\(\bigcup_{k ≥ j} A_k\) ≤ \sum_{k=j}^∞ \mu(A_k) ≤ \sum_{k=j}^∞ \frac{1}{2^k} = \frac{1}{2^j}·\frac{1}{1 - \frac{1}{2}} = \frac{1}{2^{j + 1}} \implies \mu(A) = 0. $$
	\end{dukazin}
\end{veta}

\begin{dusledek}
	Je-li $1 ≤ p ≤ +∞$ a $f_n \overset{L^p(\mu)}\rightarrow f$, pak
	$$ \exists \{f_{n_k}\}_{k \in ®N}: f_{n_k} \rightarrow f \mu\text{-skoro všude}. $$

	\begin{dukazin}
		Přímý důsledek předchozích dvou vět.
	\end{dukazin}
\end{dusledek}

\begin{veta}[2. vztah mezi konvergencí podle míry a konvergencí skoro všude]
	Jestliže $(X, ©A, \mu)$ je prostor s konečnou mírou a $f_n \rightarrow f$ $\mu$-skoro všude, pak $f_n \overset{\mu} \rightarrow f$.

	\begin{dukazin}
		Máme dokázat, že $\forall \epsilon > 0$ platí
		$$ \lim_{n \rightarrow ∞} \mu(\{x \in X\ |\ |f_n(x) - f(x)| ≥ \epsilon\}) = 0. $$
		Buď $\epsilon > 0$. Definujeme množiny $A_n$, $B_n$, $n \in ®N$, předpisem
		$$ A_n := \{x \in X\ |\ |f_n(x) - f(x)| ≥ \epsilon\}, \qquad B_n := \bigcup_{k=n}^∞ A_k. $$
		Pak $B_1 \supset B_2 \supset …$ a platí
		$$ \bigcap_{n \in ®N} B_n \subset \{x \in X | f_n(x) \nrightarrow f(x)\}. $$
		Tedy $\mu\(\bigcap_{n \in ®N} B_n\) = 0$ dle předpokladu, což spolu s větou o vlastnostech míry dává $\lim_{n \rightarrow ∞} \mu(B_n) = 0$. Protože $A_n \subset B_n$, tak $\lim_{n \rightarrow ∞} \mu(A_n) = 0$.
	\end{dukazin}
\end{veta}

\begin{veta}[Jegorov]
	Jestliže $(X, ©A, \mu)$ je prostor s konečnou mírou, $\epsilon > 0$ a $f$, $f_n, n \in ®N$, jsou měřitelné funkce splňující $f_n \rightarrow f$ $\mu$-skoro všude, pak
	$$ \exists B \in ©A, \mu(B^c) < \epsilon: f_n \rightrightarrows f \text{ na $B$}. $$

	\begin{dukazin}
		Buď $\epsilon > 0$. Položme
		$$ g_n := \sup_{j ≥ n} |f_j - f| \qquad \forall n \in ®N. $$
		Pak $g_n \rightarrow 0$ $\mu$-skoro všude (neboť $f_n \rightarrow f$ $\mu$-skoro všude), a tedy dle předpředchozí věty $g_n \overset{\mu}\rightarrow 0$. Proto
		$$ \forall k \in ®N\ \exists n_k \in ®N: \mu(\{x \in X | g_{n_k}(x) ≥ \frac{1}{k}\}) < \frac{\epsilon}{2^k}. $$
		Definujeme množiny $B_1, B_2, …$ předpisem
		$$ B_k := \{x \in X | g_{n_k}(x) < \frac{1}{k}\} $$
		a nechť $B := \bigcap_{k \in ®N} B_k$. Pak $\mu(B^c) = \mu\(\bigcup_{k \in ®N} B_k^c\) ≤ \sum_{k \in ®N} \mu(B_k^c) < \sum_{k \in ®N} \frac{\epsilon}{2^k} = \epsilon$. Buď $\delta > 0$ a $k \in ®N$ takové číslo, že $\delta > \frac{1}{k}$. Pak
		$$ \forall x \in B\ \forall n \in ®N, n ≥ n_k: |f_n(x) - f(x)| ≤ g_{n_k}(x) < \frac{1}{k} < \delta, $$
		a tedy $f_n \rightrightarrows f$ na $B$.
	\end{dukazin}
\end{veta}

\end{document}
