\documentclass[12pt]{article}					% Začátek dokumentu
\usepackage{../../MFFStyle}					    % Import stylu



\begin{document}

% 04. 10. 2021

\section*{Organizační úvod}
	%TODO
\section*{Úvod}

\begin{poznamka}[Motivace]
	Hledání řešení diferenciálních rovnic. (Např. nahradíme rovnici definicí operátoru a hledáme, kde je operátor identita. Tedy neřešíme rovnice, ale prostory, na kterých máme funkce.)
\end{poznamka}

\begin{definice}
	$$ ®K = ®C \lor ®K = ®R. $$ 
\end{definice}

\section{Banachovy a Hilbertovy prostory}

\begin{definice}[Normovaný lineární prostor]
	Nechť $X$ je vektorový prostor nad ®K. Funkci $||·||: X \rightarrow [0, ∞)$ nazveme normou na $X$, pokud
	$$ || x || = 0 \Leftrightarrow x = ¦o, $$
	$$ || x + y || ≤ || x || + || y ||, $$ 
	$$ || \alpha·x || = | \alpha | · || x | |. $$ 
\end{definice}

\begin{tvrzeni}
	Nechť $\left( X, || · || \right) $ je normovaný lineární prostor nad ®K.
	
	Funkce $\rho\left( x, y  \right) = || x - y ||$ je translačně invariantní metrika na $X$.

	Norma  je $1$-lipschitzovská (a tedy spojitá) funkce na $x$.

	Zobrazení $+: X \times X \rightarrow X$ a $·: ®K \times X \rightarrow X$ jsou spojitá.

	\begin{dukazin}
		První část byla na MA3. Druhá: Zvol $x, y \in X$. Pak $||y||, ||x|| ≤ ||x|| + ||x - y||$, tudíž $\left|||x|| - ||y|| \right| ≤ ||x - y||$.

		Třetí část: Připomenutí: Součin metrických prostorů s maximovou metrikou je metrický prostor. Důkaz tohoto i třetí části je pak jednoduché cvičení.
	\end{dukazin}
\end{tvrzeni}

\begin{definice}[Uzavřená a otevřená koule]
	$$ B_X\left(x, r\right) = \{ y \in X | ||x - y|| ≤ r \}.  $$ 
	$$ U_X\left(x, r\right) = \{ y \in X | ||x - y|| < r \}.  $$ 
	$$ S_X\left(x, r\right) = \{ y \in X | ||x - y|| = r \}.  $$ 

	$$ B_X = B\left(0, 1\right)  $$ 
	$$ U_X = U\left(0, 1\right)  $$ 
	$$ S_X = S\left(0, 1\right)  $$ 
\end{definice}

\begin{definice}[Banachův prostor]
	Banachův prostor je normovaný lineární prostor, který je úplný v metrice dané normou.
\end{definice}

Dále se opakovaly metrické prostory. Úplnost, kompaktnost a Bairova věta.

\begin{tvrzeni}
	Nechť $X$ je normovaný lineární prostor a $Y$ jeho podprostor. Potom a) Je-li $Y$ Banachův, pak je $Y$ uzavřený v $X$. Pokud je naopak $X$ Banachův, pak $Y$ je Banachův právě tehdy, když je uzavřený.

	\begin{dukazin}
		Je-li $\left(P,\rho\right)$ úplný, pak $M \subseteq P$ je úplný $\Leftrightarrow$ $M$ je uzavřený. To dává speciálně b).

		$\left(P, \rho\right)$ je MP, pak $M \subseteq P$ je úplný $\implies$ $M$ uzavřený. To dává speciálně a).
	\end{dukazin}
\end{tvrzeni}

\begin{priklady}
	$\left(®K, ||·||_p\right)$, $L_p(\Omega, ©A, \mu, ®K)$, kde funkce je $\Omega \rightarrow ®K$ a norma je definována jako $p$-tá odmocnina z integrálu funkce na $p$. $l_p(l)$ resp. $l_p(l, ®K)$ je diskrétní verze předchozího (tj. se sumou). $®C(K)$, kde $K$ je hausdorfův a kompaktní TP.

	$c$ jsou všechny posloupnosti se supremovou normou, $c_0$ jsou všechny posloupnosti konvergující k 0 se supremovou normou. $c_{00}$ sestává z těch posloupností, kde je jen konečně mnoho nenulových prvků (norma je maximová), je to lineární prostor, ale není Banachův. $c_0(I)$ je zobecnění z $c_0(®N)$ na libovolnou diskrétní množinu $I$, tj. obsahuje „posloupnosti“, kde pro každé $\epsilon$ je pouze konečně mnoho členů větších než $\epsilon$ (pak $(c_0(I), ||·||_∞)$ je Banachův).

	$©L^1([0, 1], || · ||_{©L^1})$ (prostor hladkých funkcí na intervalu $[0, 1]$), kde $||f||_{©L^1} = ||f||_∞ + ||f'||_∞$. $©M(K) = \{\mu : Borel(K) \rightarrow ®K | \mu \text{ regulární míra}\}$, $||\mu|| := \sup\{\sum_{i=1}^∞|\mu(B_i)|, \bigcup B_i = K, B_i \text{ Borelovská}\}$.
\end{priklady}

\begin{veta}
	Na konečněrozměrném vektorovém prostoru jsou všechny normy ekvivalentní.

	\begin{dukazin}
		Později.
	\end{dukazin}
\end{veta}

\begin{lemma}
	Nechť $X$ je vektorový prostor, $||·||_1$ a $||·||_2$ jsou normy na $X$, $B_1 = B_{X, ||·||_1}$, $B_2 = B_{X, ||·||_2}$ a $a, b > 0$. Pak $a||x||^2 ≤ ||x||_1 ≤ b||x||_2$ pro každé $x \in X$, právě když $a B_1 \subset B_2 \subset b B_1$. Speciálně $||·||_1 = ||·||_2$ právě tehdy, když $B_1 = B_2$.

	\begin{dukazin}
		$\implies:$ Zvol $x \in a B_1$, pak $||\frac{x}{a}||_1 ≤ 1 \implies x \in B_2$. Opačně: Zvol $x \in B_2$, pak $||x||_2 ≤ 1 \implies x \in B_1$.

		$\Leftarrow:$ Pokud $x = 0$, pak jsou nerovnosti jasné. Zvol $x≠0$. Pak $\frac{x}{||x||_1} \in B_1$. Pak $\frac{ax}{||x||_1} \in B_1 \subseteq B_2 \implies a||x||_2 ≤ ||x||_1$. Analogicky pro druhý směr.  
	\end{dukazin}
\end{lemma}

\begin{tvrzeni}
	Nechť $X$ je vektorový prostor a $||·||_1$ a $||·||_2$ jsou normy na $X$ a $B_1$ a $B_2$ jako minule. Následující tvrzení jsou ekvivalentní:
	\begin{enumerate}
		\item Normy $||·||_1$ a $||·||_2$ jsou ekvivalentní.
		\item Existují $a, b > 0$ taková, že $aB_1 \subset B_2 \subset bB_1$.
		\item Zobrazení $\id: (X, ||·||_1) \rightarrow \left(X, ||·||_2\right)$ je homeomorfismus.
		\item Otevřené množiny v $\left(X, ||·||_1\right)X$ splývají s otevřenými množinami $\left(X, ||·||_2\right)$.
		\item $||x_n - x||_1 \rightarrow 0$, právě když $||x_n - x||_2 \rightarrow 0$ pro $\{x_n\} \subset X$, $x \in X$.
	\end{enumerate}

	\begin{dukazin}
		$1 \Leftrightarrow 2$ plyne z předchozího lemmatu. $3 \Leftrightarrow 4 \Leftrightarrow 5$ je lehké a platí ve všech MP. $1 \implies 5$ jasné.

		$5 \implies 1:$ Sporem posloupností jdoucí k 1. TODO
	\end{dukazin}
\end{tvrzeni}

\begin{definice}
	Nechť $X$ je vektorový prostor. Řekneme, že množina $M \subset X$ je konvexní, pokud pro každé $x, y \in M$ a $\lambda \in [0, 1]$ platí, že $\lambda x + (1-\lambda)y \in M$.
\end{definice}

\begin{poznamka}[Fakt]
	Koule v normovaném lineárním prostoru jsou konvexní množiny. (A naopak každá konvexní množina může být koulí v nějaké normě.)
\end{poznamka}

\begin{definice}[Konvexní obal]
	Nechť $X$ je vektorový prostor a $M \subset X$. Konvexním obalem $M$ nazveme množinu $\conv M = \bigcap \{C \supset M | C \subset X \text{ je konvexní}\}$.
\end{definice}

\begin{tvrzeni}
	Nechť $X$ je vektorový prostor a $M \subset X$. Pak
	$$ \conv M = \{\sum_{i=1}^n \lambda_ix_i | x_i \in M, \lambda_i ≥ 0, \sum \lambda = 1, n \in ®N\}. $$

	\begin{dukazin}
		$\subseteq:$ Stačí dokázat, že množina vpravo je konvexní. Přímočaré.

		$\supseteq:$ Stačí dokázat, že každý prvek vlevo je v konvexním obalu. Indukcí podle $n$, přímočaré.
	\end{dukazin}
\end{tvrzeni}

\begin{definice}
	Nechť $X$ je vektorový prostor. Řekneme, že množina $M \subset X$ je symetrická, pokud $-M = M$.
\end{definice}

\begin{poznamka}[Fakt]
	Nechť $M$ je symetrická konvexní podmnožina normovaného lineárního prostoru $X$, která obsahuje $U(x, r)$ respektive $B(x, r)$ pro nějaké $x \in X$ a $r ≥ 0$. Pak $U(0, r) \subset M$, resp. $B(0, r) \subset M$.

	\begin{dukazin}
		Jednoduchý.
	\end{dukazin}
\end{poznamka}

\begin{definice}
	Nechť $X$ je normovaný lineární prostor a $M \subset X$. Pak definujeme uzavřený lineární obal $M$ jako $$ \overline{\spn}M = \bigcap \{Y \supset M | Y \text{ uzavřený podprostor } X\} $$ a uzavřený konvexní obal jako $\overline{\conv} M = \bigcap \{TODO\}$.
\end{definice}

\begin{poznamka}[Fakt]
	Nechť $X$ je normovaný lineární prostor, $Y$ je podprostor $X$ a $C \subset X$ je konvexní. Pak $\overline{Y}$ je podprostor $X$ a $\overline{C}$ je konvexní množina.
\end{poznamka}

\begin{poznamka}[Fakt]
	Nechť $X$ je normovaný lineární prostor a $M \subset X$. Pak $\overline{\spn} M = \overline{\spn M}$ a $\overline{\conv} M = \overline{\conv M}$.
\end{poznamka}

\begin{veta}
	Nechť $X$ je normovaný lineární prostor, $Y \subset X$ uzavřený podprostor a $Z \subset X$ konečněrozměrný podprostor. Pak $\spn(Y \cup Z)$ je uzavřený.

	\begin{dukazin}
		Stačí dokázat pro $\dim Z = 1$ (pak indukcí). Ať $Z = \spn(e)$, $e \notin Y$. Ověřme, že $\spn(Y \cup \{e\}) = \{y + ke | k \in ®K\}$ je uzavřený: Ať $x_n = y_n + k_n e \rightarrow x \in X$. Chci $x \in \spn Y$.

		1. krok: $(t_n)$ je omezená. (Kdyby ne, pak má limitu nekonečno.) Pak ale $||\frac{y_{n_k}}{t_{n_k}} + e || = \frac{1}{|t_{n_k}|} ||x_{n_k}|| \rightarrow 0$, tedy $\frac{y_{n_k}}{t_{n_k}} \rightarrow -e \notin Y$, tedy $Y$ není uzavřená. \lightning

		Tedy existuje posloupnost $(n_k)$, že $t_{n_k} \rightarrow t \in ®K$. Pak ale $y_{n_k} = x_{n_k} - t_{n_k}e \rightarrow x - t e \in Y$. Tedy $\exists z \in Y: x - t e = z$, tj. $x = z + t e \in \spn(Y \cup \{e\})$.
	\end{dukazin}
\end{veta}

\begin{dusledek}
	Nechť $X$ je normovaný lineární prostor. Každý konečněrozměrný podprostor $X$ je uzavřený v $X$.
\end{dusledek}

% 11. 10. 2021

TODO

\begin{veta}[Test úplnosti]
	Nechť $X$ je normovaný lineární prostor. Pak $X$ je Banachův, právě když každá absolutně konvergentní řada je konvergentní.

	\begin{dukazin}
		$\implies$: Ať $X$ je Borelovský, $\sum_{n=1}^∞ x_n$ je AK řada. $s_N = \sum_{n=1}^N x_n$. Chceme $(s_n)$ je cauchy: Buď $\epsilon > 0$. Ať $n_0 \in ®N$ je takové, že $\sum_{n=N}^M ||x_n|| < \epsilon$, $n_0 ≤ N < M$. Pak ale pro $n_0 ≤ N < M$ je
		$$ ||s_N - s_M|| = ||\sum_{n=N+1}^M x_n|| ≤ \sum_{N+1}^M||x_N|| < \epsilon. $$
		Tedy $(s_n)$ je konvergentní.

		$\Leftarrow$: Ať $(x_n)$ je cauchyovská. Indukcí najdeme podposloupnost, že $||x_{n_k} - x_{n_{k+1}} < 2^{-k}$, $k \in ®N$. Pak
		$$ z = \sum_{k=1}^∞(x_{n_{k+1}} - x_{n_k}) = \lim_{k \rightarrow ∞}(x_{n_{k+1}} - x_{n_1}) $$
		$$ \implies \lim_{k \rightarrow ∞} x_{n_k} = \lim_{k \rightarrow ∞} (x_{n_k} - x_{n_1} + x_{n+1}) = \lim(x_{n_k} - x_{n_1}) + \lim x_{n_1} = z + x_{n_1}. $$
		Celkem $\exists (n_k) \nearrow$, že $\lim(x_{n_k})$ existuje. Značme $x = lim_{k \rightarrow ∞} x_{n_k}$.
		Chceme $\lim_{n \rightarrow ∞} x_n = x$. V metrickém prostoru konverguje Cauchyovská posloupnost právě tehdy, pokud existuje její konvergentní podposloupnost.
	\end{dukazin}
\end{veta}

\begin{definice}[Zobecněná řada]
	Nechť $X$ je normovaný lineární prostor, $\Gamma$ je množina a $\{x_\gamma\}_{\gamma \in \Gamma}$ je kolekce prvků prostoru $X$. Symbol $\sum_{\gamma \in \Gamma} x_\gamma$ nazveme zobecněnou řadou.

	Dále $©F(\Gamma)$ značí systém všech konečných podmnožin $\Gamma$. Řekneme, že zobecněná řada … konverguje (též konverguje bezpodmínečně) k $x \in X$, pokud platí
	$$ \forall \epsilon > 0\ \exists F \in ©F(\Gamma)\ \forall F' \in ©F(\Gamma), F' \subseteq F: ||x - \sum_{\gamma \in F'} x_\gamma|| < \epsilon. $$

	Existuje-li $x \in X$, říkáme, že je zobecněná řada … (bezpodmínečně) konvergentn a $x$ nazýváme jejím součtem. Konverguje-li zobecněná řada reálných čísel $\sum_{\gamma \in \Gamma} ||x_\gamma||$, pak se zobecněná řada $\sum_{\gamma \in \Gamma} x_\gamma$ nazývá absolutně konvergentní.
\end{definice}

\begin{definice}[Bolzanova-Cauchyova podmínka]
	Řekneme, že zobecněná řada TODO
\end{definice}

\begin{veta}[Nutná podmínka konvergence]
	Nechť $\sum_{\gamma \in \Gamma} x_\gamma$ je konvergentní zobecněná řada v normovaném lineárním prostoru $X$. Pak je její součet určen jednoznačně a $(||x_\gamma||)_{\gamma \in \Gamma} \in c_0(\Gamma)$.

	\begin{dukazin}[Jednoznačnost]
		Ať $\sum_{\gamma \in \Gamma} x_\gamma = x ≠ y = \sum_{\gamma \in \Gamma} = \sum_{\gamma \in \Gamma}x_\gamma$. Pak $\forall \epsilon > 0$:
		$$ \exists F_x \in ©F(\Gamma)\ \forall F \supseteq F_x: ||x - \sum_{\gamma \in \Gamma} x_\gamma|| < \frac{\epsilon}{2}, $$
		$$ \exists F_y \in ©F(\Gamma)\ \forall F \supseteq F_y: ||x - \sum_{\gamma \in \Gamma} x_\gamma|| < \frac{\epsilon}{2}. $$
		Pak pro $\epsilon = ||x - y|| ≤ ||x - \sum_{F_x \cup F_y} x_\gamma || + ||\sum_{F_x \cup F_y} x_\gamma - y|| < \epsilon$. \lightning
	\end{dukazin}

	\begin{dukazin}[Existence]
		Chceme $(||x_\gamma||) \in c_0(\Gamma)$: Ať $\epsilon > 0$ libovolné. Najdeme 
		$$ F \in ©F(\Gamma)\ \forall F' \supset F: ||x - \sum_{\gamma \in F'}x_\gamma || < \frac{\epsilon}{2}. $$
		Pak pro $\gamma_0 \notin F$ máme
		$$ ||x_{\gamma_0}|| = ||\sum_{\gamma \in F \cup \{\gamma_0\}} x_\gamma - x + x - \sum_{\gamma \in F}x_\gamma|| ≤ ||…|| + ||…|| < \epsilon. $$
		Tedy $\{\gamma \in \Gamma | ||x_\gamma|| > \epsilon\} \subseteq F \in ©F(\Gamma) \implies (||x_\gamma||) \in c_0(\Gamma)$. (Je tam pouze konečný počet prvků větších než $\epsilon$.)
	\end{dukazin}
\end{veta}

\begin{veta}
	Nechť $X$ je Banachův prostor.
	
	\begin{enumerate}
		\item Zobecněná řada v $X$ je konvergentní právě tehdy, když splňuje Bolzanovu-Cauchyovu podmínku.
		\item Každá absolutně konvergentní zobecněná řada v $X$ je konvergentní.
		\item Je-li zobecněná řada $\sum_{\gamma \in \Gamma} x_\gamma$ v $X$ konvergentní a $\Lambda \subset \Gamma$, pak je i zobecněná řada $\sum_{\gamma \in \Lambda} x_\gamma$ konvergentní.
	\end{enumerate}

	\begin{dukazin}[1.]
		$\implies$: Ať $\sum_{\gamma \in \Gamma} x_\gamma$ je konvergentní. Zvol $\epsilon > 0$. Zvolíme 
		$$ F \in ©F(\Gamma)\ \forall F' \supseteq F: ||\sum_{\gamma \in \Gamma}x_\gamma - \sum_{\gamma \in F'}x_\gamma|| < \frac{\epsilon}{2}. $$
		Pak pro $\tilde{F} \in ©F(\Gamma)$, že $\tilde{F} \cap F = \O$ máme:
		$$ ||\sum_{\gamma \in \tilde{F}} x_\gamma|| = ||\sum_{\gamma \in F \cup \tilde{F}} x_\gamma - \sum_{\gamma \in \Gamma} x_\gamma + \sum_{\gamma \in \Gamma} x_\gamma - \sum_{\gamma \in F} x_\gamma|| ≤ ||…|| + ||…|| < \epsilon. $$

		$\Leftarrow$: Ať $\sum_{\gamma \in \Gamma} x_\gamma$ splňuje B-C podmínku. Pak najdeme posloupnost $(F_n)_{n=1}^∞ \in ©F(\Gamma)^{®N}$, že
		$$ F_1 \subset F_2 \subset … \land \forall F' ©F(\Gamma): F' \cap F_n = \O: ||\sum_{\gamma \in F'} x_{\gamma}|| < \frac{1}{n}. $$
		Označ $y_n = \sum_{\gamma \in F_n} x_\gamma$.
		1. krok: $(y_n)$ je cauchyovská. (Dokáže se snadno.) 2. krok: Tedy existuje $y \in X: \lim y_n = y$. Chceme $y = \sum_{\gamma \in \Gamma} x_\gamma$: Ať $\epsilon > 0$. 
		$$ \forall F' \supset F: ||y  - \sum_{\gamma \in F'} x_\gamma|| ≤ ||y_{n_0} - \sum_{\gamma \in F'} x_\gamma|| + ||y_{n_0} - y|| = \sum_{\gamma \in F' \setminus F_{n_0}} x_\gamma ≤ \frac{1}{n_0} + ||y_{n_0} - y|| < \epsilon. $$
	\end{dukazin}

	\begin{dukazin}[2.]
		Víme, že $\sum_{\gamma \in \Gamma}||x_\gamma||$ je konvergentní. Dle tvrzení níže tedy
		$$ \sum_{\gamma \in \Gamma}||x_\gamma|| = S = \sup\{\sum_{\gamma \in \Gamma} | ||x_\gamma|| | F \in ©F(\Gamma)\}. $$
		Ověříme, že $\sum x_\gamma$ splní B-C podmínku: Ať $\epsilon > 0$. Ať $F \in ©F(\Gamma)$ tak, že $S - \epsilon < \sum_{\gamma \in F}||x_\gamma||$. Pak $\forall F' \in ©F(\Gamma)$, že $F' \cap F = \O$:
		$$ ||\sum_{\gamma \in F'} x_\gamma || ≤ \sum_{\gamma \in F'} || x_\gamma || = \sum_{\gamma \in F' \cup F} ||x_\gamma|| - \sum_{\gamma \in F} ||x_\gamma|| < \epsilon. $$
	\end{dukazin}

	\begin{dukazin}[3.]
		Snadný důsledek 1., protože B-C podmínka se zjevně dědí na podmnožiny.
	\end{dukazin}
\end{veta}

\begin{tvrzeni}
	Nechť $\sum_{\gamma \in \Gamma} a_\gamma$ je zobecněná řada nezáporných čísel. Pak tato řada konverguje, právě když $\sup\{\sum_{\gamma \in F} a_\gamma: F \in ©F(\Gamma)\} < +∞$. A navíc platí $\sum_{\gamma \in \Gamma} a_\gamma = \sup\{\sum_{\gamma \in F} a_\gamma: F \in ©F(\Gamma)\}$.

	\begin{dukazin}
		$\implies$: Ať $\sum_{\gamma \in \Gamma} a_\gamma$ konverguje. Pak zvolíme $F \in ©F(\Gamma)\ \forall F' \supset F: ||\sum_{\gamma \in \Gamma} a_\gamma - \sum_{\gamma \in F} a_\gamma|| < 1$.
		Pak $\forall H \in ©F(\Gamma): \sum_{\gamma \in H} a_\gamma ≤ \sum_{\gamma \in H \cup F} a_\gamma ≤ \sum_{\gamma \in \Gamma}a_\gamma + 1.$ Tedy $\sup… ≤ \sum_{\gamma \in \Gamma} a_p + 1 < ∞$.

		$\Leftarrow$: Ať $S:= \sup… < ∞$. Chceme $\sum_{\gamma \in \Gamma} a_\gamma = S$. Ať $\epsilon > 0$. Ať $H \in ©F(\Gamma)$ (z definice suprema) taková, že $S - \epsilon < \sum_{\gamma \in H} a_\gamma$. Pak pro $F' \supset H$ máme
		$$ |S - \sum_{\gamma \in F'} a_\gamma| = S - \sum_{\gamma \in F'} a_\gamma < S - \sum_{\gamma \in H}a_\gamma < \epsilon. $$
		Tedy $\sum a_\gamma = S$.
	\end{dukazin}
\end{tvrzeni}

% 12. 10. 2021

\begin{tvrzeni}
	Nechť $X$ je normovaný lineární prostor a $\{x_n\} \subset X$. Pak zobecněná řada $\sum_{n \in ®N}x_n$ je absolutně konvergentní, právě když řada $\sum_{n=1}^∞ x_n$ je konvergentní.

	\begin{dukazin}
		$\implies$: Ať $\sum_{n=1}^∞ ||x_n|| =: S < ∞$. Pak
		$$ \sup_{F \in ®F(®N)} \sum_{n \in F} ||x_n|| ≤ \sup_{N \in ®N} \sum_{n=1}^N ||x_n|| = \sum_{n=1}^∞ ||x_n|| = S < ∞, $$
		neboť každá konečná množina v přirozených číslech má maximum (a odebráním kladných prvků sumu zmenšíme).

		$\Leftarrow$: Ať $\sum_{n \in ®N} ||x_n||$ je konvergentní, pak dle předchozího tvrzení $S:=\sup_{F \in ©F(®N)} \sum_{n \in F} ||x_n|| < ∞$. Tedy
		$$ \sum_{n=1}^∞||x_n|| = \sup_{N \in ®N} \sum_{n \in [N]} ||x_n|| ≤ S < ∞. $$
	\end{dukazin}
\end{tvrzeni}

\begin{veta}
	Nechť $\{x_n\}$ je posloupnost v Banachově (pro normovaný lineární prostor je důkaz složitější) prostoru $X$. Pak následující tvrzení jsou konvergentní:
	
	\begin{enumerate}
		\item $\sum_{n \in ®N} x_n$ konverguje (říkáme $\sum_{n=1}^∞ x_n$ konverguje bezpodmínečně).
		\item $\sum_{n=1}^∞ x_{\pi(n)}$ konverguje pro každou permutaci $\pi: ®N \rightarrow ®N$ ke stejnému součtu.
		\item $\sum_{n=1}^∞ x_{\pi(n)}$ konverguje pro každou permutaci $\pi: ®N \rightarrow ®N$.
	\end{enumerate}

	\begin{dukazin}
		$1 \implies 2$: Ať $\epsilon > 0$ a $\pi \in ®S(®N)$. Ať $F \in ©F(®N)$ splňuje, že $\forall F' \supseteq F: ||\sum_{n \in F'} x_n - x|| < \epsilon$, kde $x = \sum_{n \in ®N}x_n$. Zvolme $n_0 \in ®N: F \subseteq \{\pi(1), …, \pi(n_0)\}$. Pak $\forall n ≥ n_0: ||\sum_{i=1}^n x_{\pi(i)} - x|| < \epsilon$. Tedy $\sum_{n=1}^∞ x_{\pi(n)} = x$.

		$2 \implies 3$: okamžitě. $3 \implies 1$: Pro spor předpokládejme, že $\sum_{n=1}^∞ x_{\pi(n)}$ konverguje pro každou $\pi \in ®S(®N)$, ale $\sum_{n \in ®N} x_n$ nesplňuje B-C podmínku. Zvolme $\epsilon > 0$ svědčící o tom, že B-C podmínka není splněna. Pak existuje $\(F_n\)_{n=1}^∞ \in ©F(®N)^{®N}$, že $F_n \cap F_m = \O$ $\forall n ≠ m$, $\max F_n < \min F_{n+1}, n \in ®N$ a $||\sum_{i \in F_n} x_i||$.

	Zvolme $\pi \in ®S(®N)$ splňující, že existuje $\(n_k\)\nearrow$ a $\(p_k\)_{k=1}^∞ \in ®N^{®N}$, že $\pi\(\{n_k, n_k + 1, …, n_k + p_k\}\) = F_k$ $\forall k \in ®N$. Tedy $\forall k \in ®N: ||\sum_{i = n_k}^{n_k + p_k} x_{\pi(i)}|| = ||\sum_{i \in F_k} x_i || ≥ \epsilon$. To však znamená, že $\sum_{i=1}^∞ x_{\pi(i)}$ nesplňuje B-C podmínku, tedy není konvergentní. \lightning
	\end{dukazin}
\end{veta}

\begin{veta}
	Každá absolutně konvergentní řada v Banachově prostoru je bezpodmínečně konvergentní.

	\begin{dukazin}
		Jasný z minulé věty.
	\end{dukazin}

	Navíc v ®R platí ekvivalence.
\end{veta}

\begin{veta}
	Pokud $\dim X = +∞$, pak $\exists(x_n): \sum_{n=1}^∞ ||x_n||$ konverguje, ale $\sum_{n \in ®N}x_n$ není konvergentní.
\end{veta}

\section{Lineární operátory a funkcionály}
\begin{poznamka}[Opakovali jsme]
	Lineární zobrazení (viz lingebra), dále:
	
	\begin{veta}
		Nechť $X, Y$ jsou normované lineární prostory a $T: X \rightarrow Y$ je lineární zobrazení. Pak následující tvrzení jsou ekvivalentní:
		
		\begin{enumerate}
			\item $T$ je spojité.
			\item $T$ je spojité v jednom bodě.
			\item $T$ je spojité v 0.
			\item $\exists C ≥ 0$ tak, že $||T(x)|| ≤ C ||x||$ $\forall x \in X$.
			\item $T$ je Lipschitzovské.
			\item $T$ je stejnoměrně spojité.
			\item $T(A)$ je omezená pro každou omezenou $A \subset X$.
			\item $T(B_X)$ je omezená.
			\item $T(U(0, \delta))$ je omezená pro nějaké $\delta > 0$.
		\end{enumerate}
	\end{veta}

	Prostor $©L(X< Y)$ s normou $||T|| = \sup_{x \in B_x}||T(x)||$ je normovaný lineární prostor.
\end{poznamka}

\begin{lemma}
	Nechť $X, Y$ jsou normované lineární prostory a $T \in ©L(X, Y)$.
	
	\begin{itemize}
		\item $||T(x)|| ≤ ||T||·||x||$ pro každé $x \in X$.
		\item $||T|| = \sup_{x \in S_X}||T(x)|| = \sup_{x \in X\setminus\{¦o\}}\frac{||T(x)||}{||x||} = \sup_{x \in U_X}||T(x)||$.
		\item $||T|| = \inf\{C ≥ 0 | ||T(x)|| ≤ C||x|| \forall x \in X\}$.
	\end{itemize}

	\begin{dukazin}
		Pro $x \in X \setminus\{¦o\}$ platí $||T(x)|| = ||T(\frac{x}{||x||})||·||x|| ≤ ||T||·||x||$.

		$S_X \subseteq B_X$, tedy $||T|| ≥ \sup_{x \in S_X}||T(x)||$. $\forall x \in X \setminus\{¦o\}$:
		$$ \frac{||T(x)||}{||x||} = ||T(\frac{x}{||x||})|| ≤ \sup_{y \in S_X} ||T(y)||, $$
		tedy $\sup_{x \in S_X} ||T(x)|| ≥ \sup_{x \in X\setminus\{¦o\}} \frac{||T(x)||}{||x||} =: S_3$. Pro $x \in U_X\setminus\{¦o\}$ platí $||T(x)|| ≤ \frac{||T(x)||}{||x||} ≤ S_3$, tedy $\sup_{x \in U_x} ||T(x)|| ≤ S_3$. Konečně, pro $x \in B_x$: $||T(x)|| \leftarrow ||T\(\(1 - \frac{1}{n}\)x\)|| ≤ \sup_{x \in U_X} =: S_4$, tedy $||T_x|| = \lim_{n \rightarrow ∞} ||T\(1-\frac{1}{n}\)x|| ≤ S_4$ $\implies$ $\sup_{x \in B(x)} ||T(x)|| ≤ S_4$.

		Dle prvního bodu máme nerovnost „$≥$“. Pro „$≤$“ zvolme $\epsilon > 0$ … ať $\tilde{c} > 0$ je takové, že $\tilde{c} < \inf\{…\} + \epsilon$. Pak $||T|| = \sup_{x \in B_x} \frac{||T_x||}{||x||} ≤ \inf\{…\}$.
	\end{dukazin}
\end{lemma}

\begin{definice}
	Nechť $X$ je normovaný lineární prostor nad ®K. Prostor $©L(X, ®K)$ značíme $X^*$ a nazýváme jej duálním prostorem k prostoru $X$.
\end{definice}

% 18. 10. 2021

TODO!!!

% 19. 10. 2021

TODO!!!

% 25. 10 2021

TODO!!!

\begin{poznamka}[Kvocient]
	Nechť $X$ je vektorový prostor nad ®K a $Y$ jeho podprostor. Definujeme relaci ekvivalence $\sim$ na $X$ jako $x \sim y \Leftrightarrow x-y \in Y$.

	Pro $x \in X$ pak definujeme $[x]$ jako třídu ekvivalence obsahující $x$.

	Na množině $X / Y = \{[x] | x \in X\}$ definujeme operace $[x] + [y] = [x + y]$ a $\alpha [x] = [\alpha x]$.

	\begin{definicein}[Kvocient]
		Nechť $X$ je vektorový prostor a $Y$ jeho podprostor. Pak vektorový prostor $X / Y$ nazýváme faktoprostorem prostoru $X$ podle $Y$ nebo též kvocientem $X$ podle $Y$. Dále definujeme tzv. kanonecké kvocientové zobrazení $q: X \rightarrow X / Y$ předpisem $q(x) = [x]$.
	\end{definicein}

	\begin{definicein}[Norma na kvocientu]
		Buď $X$ normovaný lineární prostor a $Y$ jeho uzavřený podprostor. Pak $(X / Y, ||·||_{X / Y})$ je normovaný lineární prostor s normou
		$$ ||[x]||_{X / Y} = \inf_{y \in [x]}||y|| = \inf_{y \in Y}||x + y|| = \inf_{y \in Y}||x - y|| = \dist(x + Y, 0) = \dist(x, Y). $$
		Tato norma se nazývá kanonická kvocientová norma.

		\begin{dukazin}[Je to norma]
			Triviální.
		\end{dukazin}
	\end{definicein}

	\begin{tvrzeniin}
		Nechť $X$ je normovaný lineární prostor a $Y$ jeho uzavřený podprostor. Pak kanonické kvocientové zobrazení $q: X \rightarrow X / Y$ je spojitý lineární operátor, který je na a splňuje $q(U_x) = U_{X / Y}$. Je-li $Y$ vlastní, pak $||q|| = 1$.

		\begin{dukazin}
			Zřejmý.
		\end{dukazin}
	\end{tvrzeniin}
\end{poznamka}

\begin{veta}
	Nechť $X$ je Banachův prostor. Potom TODO!

	\begin{dukazin}
		Přes test úplnosti ($X$ je Banachův, právě když každá abs. konvergentní řada je konvergentní). Ať $\{[x]_n | n \in ®N\}$ splňuje $\sum_{n=1}^∞ < ∞$. Chceme $\sum_{[x]_n}$. Ať $\{y_n | n \in ®N\} \subseteq Y$ jsou takové, že $\sum_{n = 1}^∞ ||x_n + y_n|| < ∞$. Pak $\sum (x_n + y_n)$ je konvergentní (podle testu úplnosti) a je prvkem $X$, tedy $q\(\sum_{n=1}^∞ (x_n + y_n)\) = \sum_{n=1}^∞ q(x_n + y_n) = \sum_{n=1}^∞ [x_n]$. Tudíž $\sum_{n=1}^∞ [x_n]$ je v prostoru $q(X) = X / Y$.
	\end{dukazin}
\end{veta}

\begin{poznamka}[Zajímavosti]
	$l_∞ / c_0$ je docela zajímavý prostor (Rosemider? + Brech? 2012: Je nerozhodnutelné, zda $l_∞ / c_0$ je izometricky univerzální Banachův prostor hustoty $|®R|$. Dokonce je nerozhodnutelné, zda takový prostor existuje.) ($l_∞ / c_0 ≡ ©C(\beta ®N \setminus ®N)$)
\end{poznamka}

\begin{definice}[Direktní součet]
	Nechť $X$ je vektorový prostor a $A, B$ jsou jeho podprostory. Říkáme, že $X$ je direktním (též algebraickým) součtem $A$ a $B$ (značíme $X = A \oplus B$) pokud $A \cap B = \{¦o\}$ a $X = A + B = \spn\{A \cup B\}$.
\end{definice}

\begin{definice}[Projekce]
	Nechť $X$ je vektorový prostor. Lineární zobrazení $P: X \rightarrow X$ se nazývá (lineární) projekce, pokud $P^2 = P \circ P = P$.
\end{definice}

\begin{tvrzeni}[Fakt]
	Nechť $X$ je vektorový prostor.

	\begin{itemize}
		\item Je-li $P: X \rightarrow X$ lineární projekce, pak $P\lhook{\Rang P} = \id_{\Rang P}$.
		\item Je-li $Y$ podprostor $X$ a $P: X \rightarrow Y$ lineární zobrazení splňující $P\lhook_Y = \id_Y$, pak $P$ je projekce $X$ na $Y$.
	\end{itemize}

	\begin{dukazin}
		Triviální.
	\end{dukazin}
\end{tvrzeni}

\begin{tvrzeni}
	Nechť $X$ je vektorový prostor. Jsou-li $P_A$ a $P_B$ projekce příslušné rozkladu $X = A \oplus B$, pak $P_A + P_B = \id_X$, $\Rang P_A = A$, $\ker P_A = B$, $\Rang P_B = B$ a $\Ker P_B = A$.

	\begin{dukazin}
		Jednoduchý.
	\end{dukazin}

	Na druhou stranu, je-li $P$ lineární projekce v $X$, pak $X = A \oplus B$, kde $A = \Rang P$, $B = \Ker P$ a $P = P_A$.

	\begin{dukazin}
		Jednoduchý.
	\end{dukazin}
\end{tvrzeni}

\begin{veta}
	Nechť $X$ je vektorový prostor a $Y$ jeho podprostor.

	\begin{itemize}
		\item Prostor $Y$ má algebraický doplněk v $X$.
		\item Je-li $A$ algebraický doplněk $Y$ v $X$, je $A$ algebraicky izomorfní s $X / Y$, speciálně $\dim(A) = \dim(X / Y)$.
	\end{itemize}

	\begin{dukazin}
		Díky Zornovu lemmatu existuje algebraická báze $B \subset Y$ prostoru $Y$. Stejně tak existuje $B' \supset B$ báze $X$. Potom $Z = \spn(B' \setminus B)$ je algebraický doplněk $Y$ v $X$, neboli $X = Y \oplus X$.

% 26. 10. 2021

		Ať $X = Y \oplus A$. Pak chceme $q\upharpoonright_A: A \rightarrow X / Y$ je lineární izomorfismus: Víme $q$ je lineární, $q$ je prosté (ať $x \in A, q(x) = 0$, pak $x \in Y$, tedy $x \in A \cap Y = \{¦o\}$, takže $x = ¦o$) a $q$ je na (Ať $x = y + a \in X$, pak $q(x) = q(a)$, tedy $q(x) \in q|_A(A)$). 
	\end{dukazin}
\end{veta}

\begin{definice}[Kodimenze]
	Je-li $X$ vektorový prostor a $Y$ jeho podprostor, pak kodimenzí (značíme $\codim Y$?) $Y$ rozumíme dimenzi libovolného algebraického doplňku $Y$ (což je rovno dimenzi $X / Y$).
\end{definice}

\begin{definice}
	Je-li $X$ normovaný lineární prostor a $X = A \oplus B$, pak říkáme, že $X$ je topologickým součtem $A$ a $B$, pokud jsou příslušné projekce $P_A$ a $P_B$ spojité. Tento fakt značíme $X = A \oplus_t B$. Je-li $A$ podprostor $X$, pak každý podprostor $B \subset X$ splňující $A \oplus_t B = X$ se nazývá topologický doplněk $A$ v $X$. Má-li $A$ topologický doplněk, pak říkáme, že je komplementovaný (v $X$).
\end{definice}

\begin{veta}
	Nechť $X$ je normovaný lineární prostor a $Y, Z$ jsou jeho podprostory splňující $X = Y \oplus Z$. Pak $X = Y \oplus_t Z$, právě když zobrazení $T: X \rightarrow Y \oplus_1 Z$, $T(x) = (P_Y(x), P_Z(x))$ je izomorfismus.

	\begin{dukazin}
		$\implies$: $\forall x \in X$: $||T(x)|| = ||P_Y x|| + ||P_Z x|| ≤ 2\max (||P_Y||, ||P_Z||) ||x|| ≤ ||(P_Y + P_Z)x|| = ||x||$. Tedy $T$ je izomorfismus.

		$\Leftarrow$: $\forall x \in X$: $||P_yx|| ≤ ||P_yx|| + ||P_z x|| = ||T x|| ≤ ||T|| ||x||$, tedy $||P_y|| ≤ ||T||$.
	\end{dukazin}
\end{veta}

\begin{veta}
	Nechť $X$ je Banachův prostor a $Y, Z \subset X$ jeho podprostory splňující $X = Y \oplus Z$. Pak $X = Y \oplus_t Z$, právě když $Y$ a $Z$ jsou uzavřené.

	\begin{dukazin}
		Zatím bez důkazu.
	\end{dukazin}
\end{veta}

\begin{veta}
	Nechť $X, Y$ jsou normované lineární prostory. Pak

	\begin{itemize}
		\item $Y$ je isomorfní komplementovanému podprostoru $X$, právě když existují lineární operátory $S: X \rightarrow Y$ a $T: Y \rightarrow X$ splňující $S \circ T = \id_Y$.
		\item $Y$ je isometrické $1$-komplementovanému podprostoru $X$, právě když existují lineární operátory $S: X \rightarrow Y$ a $T: Y \rightarrow X$ splňující $S \circ T = \id_y$ a $\max\{||S||, ||T||\} ≤ 1$.
	\end{itemize}

	\begin{dukazin}
		$\Leftarrow$: Polož $p := T \circ S: X \rightarrow X$. Pak $p$ je zřejmě lineární a $||p|| ≤ ||T||·||S||$, navíc $p^2 = (T\circ S)\circ(T \circ S) = p$, tedy $p$ je projekce. Zároveň $p(X) = T(S(X))$, jelikož $S \circ T$ je identita, tak $S$ je na a $p(X) = T(Y) = \Rang T$. Zbývá si uvědomit, že $T$ je izomorfismus (izometrie, pokud $||S||, ||T|| ≤ 1$): Máme
		$$ \forall x \in X: ||S x|| = ||STS x|| ≤ ||S||·||TS x||, $$
		tedy (protože $S$ je na):
		$$ \forall y \in Y: ||y||\frac{1}{||S||} ≤ ||Ty||, $$
		tudíž $T$ je izomorfismus.

		$\implies$: Ať $P: X \rightarrow X$ je projekce, $L: P(X) \rightarrow Y$ izomorfismus na. Položíme $S := L \circ P$, $T := L^{-1}$, pak $S \circ T = L \circ P \circ L^{-1} = L \circ L^{-1} = \id$.
	\end{dukazin}
\end{veta}

\begin{poznamka}[Zajímavosti pro všechny (nezkouší se)]
	Ví se ($\dim X = +∞$, $X$ Banach)

	\begin{itemize}
		\item $X$ lze komplementovaně vnořit do $l_p$ $\implies$ $X \cong l_p$, $p \in [1, ∞]$.
		\item $X$ lze komplementovaně vnořit do $c_0$ $\implies$ $X \cong l_0$.
		\item Existuje nespočetně neizomorfních podprostorů $L_p$, $p \in (1, ∞)$.
	\end{itemize}

	Neví se:

	\begin{itemize}
		\item $X$ lze komplementovaně vnořit do $L_1$ $\implies$ $X \in \{l_1, L_1\}$.
		\item $X$ lze komplementovaně vnořit do $©C([0, 1])$ $\implies$ $X \cong ©C(k?)$.
	\end{itemize}

	Ví se:

	\begin{itemize}
		\item $X \cong l_2$ $\Leftrightarrow$ $(\forall Z, \dim Z = +∞, Z Banach, Z \hookrightarrow l_2 \implies Z \cong l_2)$.
	\end{itemize}

	Neví se, zda platí izometrická varianta předchozího.
\end{poznamka}

\section{Hilbertovy prostory}
\begin{lemma}
	$A^{\perp}$ je uzavřený podprostor.

	\begin{dukazin}
		Pro $y \in X$ ať $f_y(x) = <x, y>$. Pak $f_y$ je lineární a spojité (z Cauchy-Swartze). $A^{\perp} = \bigcap_{y \in A}f^{-1}_y(0)$.
	\end{dukazin}
\end{lemma}

\begin{definice}
	Prostor se skalárním součinem $(X, <·, ·>)$ se nazývá Hilbertův prostor, pokud je úplný v metrice indukované skalárním součinem, tj. pokud $(X, ||·||)$ je Banachův prostor, kde $||x|| = \sqrt{<x, x>}$.
\end{definice}

\begin{priklady}
	\begin{itemize}
		\item $l_2$ … $<x, y> := \sum_{n=1}^∞ x_n \overline{y_n}$.
		\item $L_2([0, 1])$ … $<f, g> := \int_0^1 f(x) \overline{g(x)} dx$.
	\end{itemize}
\end{priklady}

\begin{tvrzeni}
	Nechť $(X, <·, ·>)$ je prostor se skalárním součinem nad ®K. Pak funkce $<·, ·>: X \times X \rightarrow ®K$ je lipschitzovská na omezených množinách (a tedy spojitá).

	\begin{dukazin}
		Přímočarý s použitím Cauchy-Swartze.
	\end{dukazin}
\end{tvrzeni}

\begin{tvrzeni}[Polarizační vzorec]
	Nechť $X$ je prostor se skalárním součinem. Pak pro všechna $x, y \in X$ platí
	$$ <x, y> = \frac{1}{4}(||x + y||^2 - ||x - y||^2) $$
	v reálném případě, resp.
	$$ <x, y> = \frac{1}{4}(||x + y||^2 - ||x - y||^2 + i||x + iy||^2 - i||x - iy||^2) $$
	v komplexním.

	\begin{dukazin}[Reálný případ, v $®C$ analogicky]
		$$ 4<x, y> = 2<x, y> - 2<x, -y> = ||x + y||^2 - ||x||^2 - ||y||^2 - ||x - y||^2 + ||x||^2 + ||-y||^2 = ||x + y||^2 - ||x - y||^2. $$	
	\end{dukazin}
\end{tvrzeni}

\begin{dusledek}
	Nechť $X, Y$ jsou prostory se skalárním součinem a $T: X \rightarrow Y$ je lineární izometrie do. Pak $T$ zachovává skalární součin, tj. $<T(x), T(y)> = <x, y>$ pro každé $x, y \in X$.

	\begin{dukazin}
		Izometrie zachovává pravé strany v polarizačním vzorci.
	\end{dukazin}
\end{dusledek}

% 01. 11. 2021

TODO!
\begin{veta}
	$(X, ||·||)$ je NLP. Pak $||x|| = \sqrt{\<x, x\>}$ pro skalární součin $\<·,·\>$ $\Leftrightarrow$ platí:
	$$ \forall x, y \in X: ||x + y||^2 + ||x - y||^2 = 2\(||x||^2 + ||y||^2\). $$

	\begin{dukazin}[Reálný případ, komplexní analogicky]
		$\implies$ z Polarizačního vzorce. Pro $\Leftarrow$ položme $\<x, y\> := \frac{1}{4}\(||x + y||^2 - ||x-y||^2\)$, $x, y \in X$. Následně ověříme podmínky (kromě linearity (speciálně aditivity) je ověření triviální). Aditivita: Chceme
		$$ LS = \forall x, y, z \in X: \<x+y, z\> + \<x - y, z\> = 2\<x, z\> = PS. $$
		$$ LS = \frac{1}{4}\(\underline{||x + y + z||^2} - ||x + y - z||^2 + \underline{||x - y + z||^2} - ||x - y - z||^2\) = $$
		$$ \overset{\text{z předpokladu}}{=} \frac{1}{4}\(\underline{2\(||x + z||^2 + ||y||^2\)} - 2\(||x - y||^2 + ||y||^2\)\) = \frac{1}{2}\(||x + z||^2 - ||x - z||^2\) = PS. $$

		Tuto rovnost aplikujeme na $x = y$: $\<2x, z\> = 2\<x, z\>$, a na $\tilde{x} = \frac{1}{2}(x + y)$, $\tilde{y} = \frac{1}{2}(x - y)$:
		$$ \<x, z\> + \<y, z\> = 2\<\frac{1}{2}(x + y), z\> = \<x + y, z\>. $$
	\end{dukazin}
\end{veta}

\begin{veta}[Frigyes Riesz, 1934]
	Nechť $C$ je uzavřená neprázdná konvexní množina v Hilbertově prostoru $H$. Pak pro každé $x \in H$ existuje právě jedno $y \in C$ tak, že $||x - y|| = \dist(x, C)$.

	\begin{dukazin}
		Zvolme $\(y_n\)_{n=1}^∞$ posloupnost v $C$, že $\lim_{n \rightarrow ∞} ||y_n - x|| = d(x, C)$. Chceme, že $\(y_n\)_{n=1}^∞$ je cauchyovská. Tedy, protože $C$ je uzavřená, existuje $y \in C: y_n \rightarrow y$. Pak ale $d(x, c) = ||x - y||$.

		Zbývá jednoznačnost: Ať $y, z \in C$ taková, že $||x - y|| = ||x - z|| = \dist(x, C)$. Pak $||y - z||^2 ≤ 0$, tedy $y = z$.
	\end{dukazin}
\end{veta}

\begin{veta}[Frigyes Riesz, 1934]
	Nechť $X$ je prostor se skalárním součinem, $Y$ jeho podprostor a $x \in X$. Pak $y \in Y$ splňuje $||x - y|| = \dist(x, Y)$ právě tehdy, když $x - y \in Y^{\perp}$.

	\begin{dukazin}
		Jednoduchý.
	\end{dukazin}
\end{veta}

\begin{veta}[Frigyes Riesz, 1934]
	Nechť $Y$ je uzavřený podprostor Hilbertova prostoru $H$. Pak $H = Y \oplus_t Y^{\perp}$ a projekce $P_y: H \rightarrow Y$ příslušná rozkladu $H = Y \oplus Y^{\perp}$ má následující vlastnosti:
	
	\begin{itemize}
		\item $||P_Y(x) - x|| = \dist(x, Y) ≤ ||x||$ pro každé $x \in H$,
		\item $||P_Y|| ≤ 1$.
	\end{itemize}

	\begin{dukazin}
		$Y \cap Y^{\perp} = \{¦o\}$: Ať $x \in Y \cap Y^{\perp}$. Pak $\<x, x\> = 0 \implies x = 0$.

		$H = Y + Y^{\perp}$: Zvol $x \in H$. Dle vět výše existuje právě jedno $y \in Y: x - y \in Y^{\perp}$. Pak $x = y + x - y \in Y + Y^{\perp}$.

		Tedy, $H = Y \oplus Y^{\perp}$, a zároveň z důkazu víme, že
		$$ P_Y(x) = \text{ „jediný prvek $y \in Y$, že $x - y \in Y^\perp$“ } = \text{ „j. p. $y \in Y$, že $||x - y|| = d(x, Y)$“.} $$
		
		Tedy $||P_Y(x) - x|| = d(x, y) ≤ ||x||$. Zbývá $||P_y|| ≤ 1$: $||P_y x||^2 = $
	\end{dukazin}
\end{veta}

\begin{veta}
	Nechť $H$ je Hilbertův prostor a $\{x_n\}_{n=1}^∞ \subset H$ je podposloupnost navzájem ortogonálních prvků. Pak řada $\sum_{n=1}^∞ x_n$ konverguje bezpodmínečně, právě když konverguje.

	\begin{dukazin}
		$\implies$ už víme. $\Leftrightarrow$: Víme $\sum_{n=1}^∞ x_n$ splňuje B-C podmínku. Tedy pro $\epsilon > 0 \exists n_0 \in ®N$:
		$$ \forall m > n ≥ n_0: ||\sum_{k=n+1}^m x_k|| < \epsilon. $$
		Polož $F = \{1, …, n_0\}$. Zvol $F' \in ©F(®N) : F' \cap F = \O$. Pak
		$$ ||\sum_{k \in F'} x_k||^2 \overset{\text{Pyt. věta}}{=} = \sum_{k \in F'} ||x_k||^2 ≤ \sum_{k \in \min F'}^{\max F'} ||x_k||^2 = ||\sum_{…}^{…} x_k||^2 < \epsilon. $$
	\end{dukazin}
\end{veta}

% 02. 11. 2021

\begin{definice}[Ortogonální, ortonormální, maximální ortonormální, úplný ortonormální, ortonormální báze]
	Je-li $X$ prostor se skalárním součinem a $A \subset X$, řekneme, že množina $A$ je

	\begin{itemize}
		\item ortogonální, pokud $x \perp y$ pro všechna $x, y \in A$, $x ≠ y$.
		\item ortonormální, pokud $A$ je ortogonální a $A \subset S_X$.
		\item maximální ortonormální, pokud $A$ je ortonormální a neexistuje ortonormální množina obsahující $A$ různá od $A$.
		\item úplný ortonormální, pokud $A$ je ortonormální a $\overline{\spn} A = X$.
		\item ortonormální báze, pokud $A = \{e_\gamma | \gamma \in \Gamma\}$ je ortonormální množina a každé $x \in X$ lze vyjádřit jako $x = \sum_{\gamma \in \Gamma}x_\gamma e_\gamma$ pro nějaké skaláry $x_\gamma$.
	\end{itemize}
	TODO
\end{definice}

\begin{tvrzeni}[Fakt]
	Je-li $A$ ortonormáľní množina v prostoru se skalárním součinem, pak $||x - y|| = \sqrt{2}$ pro každé dva prvky $x, y \in A$, $x ≠ y$.

	\begin{dukazin}
		$$ ||x - y||^2 = ||x||^2 + ||y||^2. $$
	\end{dukazin}

	\begin{poznamkain}
		Tedy, pokud $X$ je separabilní se skalárním součinem $\implies$ každý ON-systém je spočetný.
	\end{poznamkain}
\end{tvrzeni}

\begin{veta}
	Každý prostor se skalárním součinem obsahuje maximální ortonormální systém.

	\begin{dukazin}
		$©P = \{A \subset X | A \text{ je ON-systém}\}$ s uspořádáním inkluzí. Zvol $©O \subset ©P$ lineárně uspořádané, pak $\bigcup ©O \in ©P$ je horní závora $©O$ $\implies$ (z Zornova lemmatu) $\exists A \in ©P$ maximální. To je hledaný maximální ON-systém.
	\end{dukazin}
\end{veta}

\begin{veta}[Besselova nerovnost]
	Je-li $\{e_\gamma\}_{\gamma \in \Gamma}$ ortonormální soustava v prostoru $X$ se skalárním součinem, platí $\sum_{\gamma \in \Gamma}|\<x, e_\gamma\>|^2 ≤ ||x||^2$ pro každé $x \in X$.

	\begin{dukazin}
		Ať $F \in ©F(\Gamma)$, $x_F := \sum_{\gamma \in F}\<x, e_\gamma\> e_\gamma$. Pak $||x||^2 = ||x - x_F||^2 + ||x_F||^2$ podle Pythagorovy věty ($x - x_F \perp x_F$: $\forall i \in F: \<x - x_F, e_i\> = \<x, e_i\> - \<x_F, e_i\> = \<x, e_i\> - \<\<x, e_i\>e_i, e_i\> = 0$). Tj. $||x||^2 ≥ ||x_F||^2 = \sum_{\gamma \in F}|\<x, e_\gamma\>|^2$. Tedy máme omezení pro všechny konečné součty, tudíž celý součet bude omezen stejně (celý součet je supremum z konečných podle tvrzení někde výše).
	\end{dukazin}
\end{veta}

\begin{veta}
	Nechť $H$ je Hilbertův prostor a $\{e_\gamma\}_{\gamma \in \Gamma}$ je ortonormální systém v $H$. Pak následující tvrzení jsou ekvivalentní:

	\begin{enumerate}
		\item $||x||^2 = \sum_{\gamma \in \Gamma} |\<x, e_\gamma\>|^2$ pro každé $x \in H$ (tzv. Parsevalova rovnost).
		\item $x = \sum_{\gamma \in \Gamma}\<x, e_\gamma\>e_\gamma$ pro každé $x \in H$.
		\item $\{e_\gamma\}$ je ortonormální báze.
		\item $H = \overline{\spn}\{e_\gamma | \gamma \in \Gamma\}$.
		\item $\{e_\gamma\}$ je maximální ortonormální systém.
	\end{enumerate}

	\begin{dukazin}
		$1 \implies 2$: Nechť $\epsilon > 0$. Zvolíme $F \in ©F(\Gamma)$: $||x||^2 - \epsilon < \sum_{\gamma \in F} |\<x, e_\gamma\>|^2$. Zvolíme $F' \supset F$, $F' \in ©F(\Gamma)$. Pak
		$$ ||x - \sum_{\gamma \in F'}\<x, e_\gamma\> e_\gamma||^2 \overset{\text{cos + Pythagorova věta}}{=} ||x||^2 + \sum_{\gamma \in F'} |\<x, e_\gamma\>|^2 - 2\Re \<x, \sum_{\gamma \in F'\<x, e_\gamma\>e_\gamma}\> = $$
		$$ = … + … - 2\Re\sum_{\gamma \in F'}\overline{\<x, e_\gamma\>}\<x, e_\gamma\> = ||x||^2 - \sum_{\gamma \in F'} |\<x, e_\gamma\>| < \epsilon. $$

		$2 \implies 3$: Triviální. $3 \implies 4$: Triviální. $4 \implies 1$: Nechť $x \in H$ a $F \in ©F(\Gamma)$, že existuje $\sum_{\gamma \in F} a_\gamma e_\gamma$ splňující $||x - \sum_{\gamma \in F}a_\gamma e_\gamma|| < \epsilon$. Položme $y := \spn(e_\gamma, \gamma \in F)$, pak $d(x, y) ≤ ||x - \sum_{\gamma \in F} a_\gamma e_\gamma|| < \epsilon$. (Jelikož $d(x, y) = ||x - \sum_{\gamma \in F}\<x, e_\gamma\>e_\gamma||$, neboť z lemmatu někde výše stačí ověřit $y \perp x - \sum_{\gamma \in F}\<x, e_\gamma\>e_\gamma$, tj. stačí $\forall i \in F: \<x - \sum_{\gamma \in F}\<x, e_\gamma\>e_\gamma, e_i\> = 0$, což je jednoduché.)

		Tedy $||x|| ≤ \epsilon + ||\sum_{\gamma \in F \<x, e_\gamma\>e_\gamma}||$ (z Bsselovy nerovnosti víme, že suma konverguje a navíc víme, že v 1 platí $≥$, tj. stačí dokázat $≤$)
		$$ ||x||^2 ≤ \(\epsilon + ||\sum_{\gamma \in F \<x, e_\gamma\>e_\gamma}||\)^2 = \epsilon^2 + 2\epsilon ||x|| + \sum_{\gamma \in F}||\<x, e_\gamma\>e\gamma|| ≤ \epsilon^2 + 2\epsilon ||x|| + \sum_{\gamma \in \Gamma}|\<x, e_\gamma\>|^2. $$

		$2 \implies 5$: Ať $x \in \{e_\gamma | \gamma \in \Gamma\}^{\perp}$ (chceme, že $x = 0$). Z 2. víme, že $x = \sum_{\gamma \in \Gamma}\<x, e_\gamma\>e_\gamma = \sum 0 = 0$.

		$5 \implies 4$: Ať $Y = \overline{\spn}(e_\gamma, \gamma \in \Gamma)$. Pak $H = Y \oplus_t Y^{\perp}$ (zde se používá úplnost jako předpoklad věty, ze které toto plyne). $H = Y \oplus_t \{e_\gamma | \gamma \in \Gamma\}^\perp \overset{5}{=} Y \oplus_t \{¦o\}$.
	\end{dukazin}

	\begin{poznamka}
		Bez úplnosti jsou ekvivalentní 1, 2, 3 a 4 a vyplývá z nich 5.
	\end{poznamka}
\end{veta}

% 08. 11. 2021

TODO!

\begin{veta}[Ernst Sigismund Fisher (1907), Frigyes Riesz (1907)]
	TODO!!!
\end{veta}

\begin{veta}[?]
	TODO!!!
\end{veta}

\begin{veta}[Heinrich Löwig (1934), F. Riesz (1934)]
	Nechť $H$ je Hilbertův prostor. Pro každé $y \in H$ označme $f_y \in H^*$ funkcionál definovaný jako $f_y(x) = \<x, y\>$ pro $x \in H$. Pak zobrazení $l: H \rightarrow H^*$, $l(y) = f_y$ je sdruženě lineární ($I(\alpha y) = \overline{\alpha} I(y)$) izometrie $H$ na $H^*$.

	\begin{dukazin}
		$\forall y \in H$ máme: $f_y$ je lineární, $\forall x \in H$: $f_y(x) ≤ ||x||·||y||$, tedy $f_y$ je spojité a $||f_y|| ≤ ||y||$, $f_y\(\frac{y}{||y||}\) = \<\frac{y}{||y||}, y\> = ||y|| \implies ||f_y|| = ||y||, y \in H$. $\implies I$ je izometrie, sdruženě lineární. Zbývá „na“. To se dokáže z následujícího lemmatu:

		Zvol $f \in H^*$, pak $H = \Ker f \oplus (\Ker f)^\perp$. Tedy existuje $z \in (\Ker f)^\perp$ splňující $H = \Ker f \oplus_t \spn\{z\}$. Položme $y:= f(z)z$. Pak $I(y) = f$, jelikož:
		$$ \forall x \in H: I(y)(x) = \<x, y\> = \<x_{\Ker f} + \alpha_x z, y\> = \<\alpha_x z, y\> = \alpha_x\<z, \overline{f(z)}z\> = f(\alpha_x z) = f(x). $$
	\end{dukazin}
\end{veta}

\begin{lemma}
	Nechť $X$ je vektorový prostor, $f$ je lineární forma na $X$ a $x \in X \setminus \Ker f$. Pak $X = \Ker f \oplus \spn\{x\}$. Tedy $\codim \Ker f = 1$.

	\begin{dukazin}
		$\Ker f \cap \spn\{x\} = \{¦o\}$: Ať $\alpha \in ®K$, pak pokud $\alpha x \in \Ker f$, pak $\alpha f(x) = f(\alpha x) = 0$, tedy $\alpha = ¦o$.

		Ať $y \in X$. Pak $y = \(y - \frac{f(y)}{f(x)}x\) + \frac{f(y)}{f(x)}x$.
	\end{dukazin}
\end{lemma}

\begin{definice}
	Nechť $X$ je komplexní normovaný lineární prostor. Symbolem $X_R$ označme prostor $X$ uvažovaný jako reálný. Tj. $X_R$ je tatáž množina jako $X$ uvažovaná s operací sčítání jako v $X$, s násobením reálným číslem jako v $X$ a stejně definovanou normou.
\end{definice}

\begin{veta}[Reálná verze komplexního normovaného lineárního prostoru]
	Nechť $Z$ je komplexní normovaný lineární prostor. Pak platí
	
	\begin{enumerate}
		\item $X_R$ je reálný normovaný lineární prostor. (Zřejmé.)
		\item $X_R$ je úplný, právě když $X$ je úplný. (Norma je pořád tatáž.)
		\item $\phi: X \rightarrow ®C$ je lineární, právě když $\Re \phi: X_R \rightarrow ®R$ je lineární a $\Im \phi(x) = -\Re \phi(ix)$ pro každé $x \in X$.
		\item Je-li $\phi \in X^*$, pak funkcionál $\psi(x) = \Re \phi(x)$, $x \in X_R$, patří do $(X_R)^*$ a platí $||\psi|| = ||\phi||$.
		\item Je-li $\psi \in (X_R)^*$, pak existuje právě jeden funkcionál $\phi \in X^*$ takový, že $\psi(x) = \Re \phi(x)$ pro $x \in X_R$. Je dán vzorcem $\phi(x) = \psi(x) - i\psi(ix)$ a splňuje $||\psi|| = ||\phi||$.
		\item Prostory $(X_R)^*$ a $(X^*)_R$ jsou izometrické.
	\end{enumerate}

	\begin{dukazin}
		TODO.
	\end{dukazin}
\end{veta}

\begin{definice}
	Nechť $X$ je reálný normovaný lineární prostor. Na $X \times X$ definujeme:
	$$ (x_1, x_2) + (y_1, y_2) = (x_1 + y_1, x_2 + y_2), \qquad x_1, x_2, y_1, y_2 \in X, $$
	$$ (\alpha_1 + i\alpha_2)·(x_1, x_2) = (\alpha_1x_1 - \alpha_2x_2, \alpha_1x_2 + \alpha_2 x_1), \qquad \alpha_1, \alpha_2 \in ®R, x_1, x_2 \in X, $$
	$$ ||(x_1, x_2)||_{X_C} = \sup\{||(\cos \alpha)x_1 + (\sin \alpha)x_2||_X | \alpha \in [0, 2\pi)\}, \qquad x_1, x_2 \in X. $$

	Symbolem $(X_C, ||·||)$ značíme komplexní normovaný lineární prostor $(X \times X, +, ·, ||·||_{X_C})$.
\end{definice}

\begin{veta}[Komplexifikace]
	Je-li $X$ reálný normovaný lineární prostor, pak je $(X_C, ||·||)$ komplexní normovaný lineární prostor. Je-li navíc $X$ Banachův, pak je $X_C$ Banachův.

	\begin{dukazin}
		Linearitu nebudeme dokazovat (definice je zvolena tak, aby to vycházelo, lehké cvičení). Norma je taktéž jednoduchá, nejtěžší je dokázat, že lze vytýkat konstanty.

		$X_C$ je Banachův plyne z toho, že $X \oplus_∞ X$ je Banach a norma $||·||_{X_C}$ je ekvivalentní (konstanty $1$ a $2$) maximové normě, která je v definici součinu metrických prostorů a součin úplných metrických prostorů je úplný.
	\end{dukazin}
\end{veta}

% 9. 11. 2021

\begin{definice}[Sublineární funkcionál, pseudonorma]
	TODO!
\end{definice}

\begin{veta}[Hans Hanh (1927), Stefan Banach (1929)]
	Nechť $X$ je vektorový prostor a $Y$ je podprostor.

	\begin{itemize}
		\item Je-li $X$ reálný, $p$ je sublineární funkcionál na $X$ a $f$ je lineární forma na $Y$ splňující $f(x) ≤ p(x)$ pro každé $x \in Y$, pak existuje lineární forma $F$ na $X$ taková, že $F|_Y = f$ a $F(x) ≤ p(x)$ pro každé $x \in X$.
		\item Je-li $p$ pseudonorma na $X$ a $t$ je linearní forma na $Y$ splňující $|f(x)| ≤ p(x)$ pro každé $x \in Y$, pak existuje lineární forma $F$ na $X$ taková, že $F|_Y = f$ a $|F(x)| ≤ p(x)$, $x \in X$.
	\end{itemize}

	\begin{dukazin}[1. bod]
		1. krok: rozšíříme $f$ o jednu dimenzi, tj. na $Z = Y \oplus \spn(x)$, kde $x \notin Y$. Položme $F(y + tx) := f(y) + t\alpha$, $y \in Y$, $t \in ®R$, kde $\alpha \in ®R$ je vhodně zvolená: Linearita $f$ vyplývá z definice, tedy stačí $f(y) + t\alpha ≤ p(y + t\alpha)$, $y \in Y$, $t \in R$ $\Leftrightarrow$
		$$ \Leftrightarrow \forall t > 0 : \alpha ≤ \frac{1}{t}(p(y + tx) - f(y)) \land \forall t < 0: \alpha ≥ \frac{1}{2}(p(y+tx) - f(y)), y \in Y \Leftrightarrow $$
		$$ \Leftrightarrow \forall t > 0: \alpha ≤ p(\frac{y}{t} + x) - f(\frac{y}{t}) \land \forall t < 0: \alpha ≥ f(\frac{-y}{t}) - p(\frac{-y}{t} - x), y \in Y \Leftrightarrow $$
		$$ \Leftrightarrow \forall y \in Y: \alpha \in \[f(y) - p(y - x), p(y + x) - f(y)\] \Leftrightarrow $$
		$$ \Leftrightarrow \forall y, z \in Y: f(y) - p(y - x) ≤ p(z + x) - f(z), $$
		tedy máme $f(y) + f(z) = f(y + z) ≤ p(y + z) ≤ p(y - x) + p(z + x)$. Tedy $\alpha$ můžeme volit libovolně z intervalu $[\sup_y f(y) - p(y - x), \inf_y p(y + x) - f(y)]$.

		2. krok: přidáme všechny dimenze (transfinitní) indukcí. (Hanh-Banachova věta je ekvivalentní axiomu výběru.)
	\end{dukazin}

	\begin{dukazin}[2. bod]
		1. krok: Pro $®K = ®R$ aplikujeme první bod: VIme, že existuje $F: X \rightarrow ®R$ lineární, že $F|_Y = f$. Pak ale $F(x) ≤ p(x)$, $x \in X$ $\land -F(x) = F(-x) ≤ p(-x) = p(x), x \in X$ $\implies |F(x) ≤ p(x), x \in X|$.

		2. krok: Pro $®K = ®C$: Polož $g = \Re f$. Pak podle 1. části $\exists G: X_R \rightarrow ®R$ lineární, že $G|_Y = g$ $\land$ $|G(x)| ≤ p(x), x \in X$. Pak máme $f(x) = g(x) - ig(ix)$, $x \in X$ a položíme $F(x) := G(x) - iG(ix)$, $x \in X$. Pak $f|_Y = f$, $F$ je lineární a pro $x \in X$ máme:

		Zvolme $|\lambda| = 1$, $\lambda \in ®C: |F(x)| = \lambda F(x)$, pak $|F(x)| = F(\lambda x) = G(\lambda x) - i G(i \lambda x) = G(\lambda x) ≤ P(\lambda x) ≤ p(x)$.
	\end{dukazin}
\end{veta}

\begin{veta}[Hahnova-Banachova]
	Nechť $X$ je normovaný lineární prostor, $Y$ je podprostor $X$ a $f \in Y^*$. Pak existuje $F \in X^*$ takové, že $F|_Y = f$ a $||F|| = ||f||$.

	\begin{dukazin}
		Aplikujeme předchozí větu na $p(x) := ||f||·||x||, x \in X$. Pak $|f(x)| ≤ ||f||·||x|| = p(x)$, $x \in Y$ $\implies$ $\exists F: X \rightarrow ®K$ lineární, $F|_y = f$, $|F| ≤ p$. Pak $|F(x)| ≤ p(x) = ||f||·||x||$, $x \in X$, tedy $||F|| ≤ ||f||$ (opačná nerovnost triviální).
	\end{dukazin}
\end{veta}

\begin{dusledek}
	Nechť $X$ je netriviální normovaný lineární prostor. Pro každé $x \in X$ existuje $f \in S_{X^*}$ takové, že $f(x) = ||x||$. Odtud plyne, že jsou-li $x, y \in X$ různé body, pak existuje $f \in X^*$ takový, že $f(x) ≠ f(y)$ (říkáme, že $X^*$ odděluje body $X$).

	\begin{dukazin}
		Zvol $x \in X$. BÚNO $x ≠ ¦o$. Polož $Y = \spn(x)$, $g : Y \rightarrow ®K$ definujeme předpisem $g(t x) := t||x||$, $\forall t \in ®K$. Pak $g$ je zřejmě lineární a $||g||=1$, protože
		$$ |g(tx)| = |t|·||x|| = ||tx||, \forall t \in ®K. $$
		Podle H-B $\exists f \in X^*: f|_Y = g$, $||f|| = ||y|| = 1$. Pak $f(x) = ||x||$.

		Ad „speciálně“: Zvol $x + y$. Najdi $f \in S_{X^*}: f(x - y) = ||x - y||$, pak $f(x) ≠ f(t)$, protože $||x - y|| ≠ 0$.
	\end{dukazin}
\end{dusledek}

\begin{dusledek}
	Je-li $X$ normovaný lineární prostor a $x \in X$, pak $||x|| = \max_{t \in B_{X^*}}|f(x)|$.

	\begin{dukazin}
		Triviální.
	\end{dukazin}
\end{dusledek}

\begin{dusledek}[Oddělování bodu a podprosotru]
	 Nechť $X$ je normovnaný lineární prostor, $Y$ je uzavřený podprostor $X$ a $x \notin Y$. Pak existuje $f \in S_{X^*}$ tak, že $f|_Y = 0$ a $f(x) = \dist(x, Y) > 0$.

	 \begin{dukazin}
		 Zvolme $Z := Y \oplus \spn(x) \subset X$. $f(y + \alpha x) := \alpha \dist(x, Y)$, $y \in Y$, $\alpha \in ®K$. Pak $f: Z \rightarrow ®K$ je lineární. $||f|| = 1$: $|f(y + \alpha x)| = |\alpha|\dist(x, Y) ≤ |\alpha|·||x + \frac{y}{\alpha}|| = ||\alpha x + y||$, $y \in Y$, $\alpha \in ®K$. Zvolme $(y_n)_{n=1}^∞$ v $Y$, že $d(x, Y) = \lim_{n \rightarrow ∞} ||x - y_n||$. Pak $\frac{|f(y_n + x)}{||y_n + x||} = \frac{d(x, Y)}{||y_n + x||} \rightarrow 1$.

		 Nyní z H-B věty rozšíříme na celé $Y$: $\exists F \in X^X: F|_z = f \land ||F|| = 1$.
	 \end{dukazin}
\end{dusledek}

\begin{veta}[Oddělování konvexních množín]
 	Nechť $X$ je normovaný lineární prostor a $A, B \subset X$ jsou disjuktní konvexní množiny. Pak platí následující tvrzení
	
	\begin{itemize}
		\item Je-li $A$ otevřená, pak existuje $f \in X^*$ takový, že $\Re f(x) < \inf_B \Re f$ pro každé $x \in A$.
		\item Je-li $A$ uzavřená a $B$ kompaktní, pak existuje $f \in X^*$ takový, že $\sup_A \Re f < \inf_B \Re f$.
	\end{itemize}

	\begin{poznamkain}
		Ekvivalentní H-B větě.
	\end{poznamkain}

	\begin{dukazin}
		BÚNO $X$ je nad ®R. BÚNO $A ≠ \O ≠ B$. První bod: Zvolíme $a \in A$, $b \in B$. Polož $w = b - a$ a $C = w + A - B$. Pak $w \notin C$, $¦o \in C$, $C$ je konvexní ($A$ i $B$ jsou konvexní, takže i jejich posunutý rozdíl je konvexní) a otevřená ($A$ je otevřená, posunutý rozdíl otevřené a libovolné je otevřená). Položme $p_c(x) := \inf\{t > 0 | x \in tC\}$ (lehce se ověří, že $p_c$, tzv. Minkowského funkcionál, je sublineární). $p_c(x) < +∞$ (protože $C$ obsahuje nulu a z otevřenosti i kouli kolem ní a každé $x$ se vejde do dostatečně nafouklé koule). $p_c ≤ 1$ na $C$ a $p_c(w) ≥ 1$.

		Položme $Y := \spn(w)$, $g(\alpha w) := \alpha$, $\alpha \in ®R$, $g: Y \rightarrow ®R$ (pak $g ≤ p_c$). Z H-B tedy plyne:
		$$ \exists G: X \rightarrow ®R \text{ lineární }, G|_Y = g, G ≤ p_c. $$
		Pak $G \in X^*$ protože $G ≤ p_c ≤ 1$ na $C$, ale to obsahuje kouli, takže je $G$ omezené na nějaké kouli $\implies$ je spojité.

		Konečně $\forall x \in A\ \forall y \in B: G(x) = G(y) + G(x - y + w) - G(w) ≤ G(y) + 1 - 1 = G(y)$. Rovnost nemůže nastat, protože $A$ je otevřené.
	\end{dukazin}
\end{veta}

% 15. 11. 2021

TODO

\begin{poznamka}[Nevím, kam patří]
	Nějaký důsledek H-B, viz foto.

	\begin{dukazin}
		Z kompaktnosti máme $\max_B g < \inf_A f$. Zvol $f = -g$ a to je ta hledaná funkce.
	\end{dukazin}
\end{poznamka}

\begin{dusledek}[H-B věty]
	$X$ je NLP, $Y \subset X$ podprostor. Buď $\dim Y < ∞$ nebo $\codim Y < ∞$. Pak $Y \overset{C}{\hookrightarrow} X$. (Tj. $\exists P: X \rightarrow Y$ spojitý, že $P |_Y = \id_Y$.)

	\begin{dukazin}
		$\dim Y < ∞$: Ať $\{e_1, …, e_n\}$ je báze $Y$, $\{f_1, …, f_n\}$ je duální báze $Y$. Pak $f_1, …, f_n: Y \rightarrow ®K$ jsou spojité ($Y$ má konečnou dimenzi). Z H-B $\exists F_1, …, F_n: X \rightarrow ®K$ spojité, $||F_i|| = ||f_i||$, $F_i \supset f_i$. Definujme $P: X \rightarrow Y$ předpisem $P(x):=\sum_{i=1}^n F_i(x) e_i \in Y$. $P$ je lineární,
		$$ ||Px|| ≤ \sum_{i=1}^n ||F_i(x)||·||e_i|| ≤ \sum_{i=1}^n ||F_i||·||x||·||e_i|| ≤ \(n·\max_{i \in [n]} ||F_i||·||e_i||\)·||x||. $$
		$P$ je tedy spojité. Zbývá ověřit $P_y = \id_n$.
		$$ \forall y \in Y: P(y) = P(\sum_{i=1}^n f_i(y)e_i) = \sum_{i=1}^n f_i(Y)P(e_i) = \sum_{i=1}^n f_i(y) \sum_{j=1}^n F_j(e_i) e_j = \sum_{i=1}^n f_i(y) e_i = y. $$

		$\codim Y < ∞$: ($\codim Y = \dim (X/Y)$) ať $\{q(e_1), …, q(e_n)\}$ je báze $X / Y$ ($q: x \mapsto [x]$) a $\{f_1, …, f_n\}$ duální funkcionály. Ty jsou spojité. Polož $F_i = f_i \circ q$ ($i \in [n]$), což je složení dvou spojitých funkcionálů, tedy spojitý funkcionál. Definujeme $P: X \rightarrow \spn(e_1, …, e_n)$, $P(x) := \sum_{i=1}^n F_i(x) e_i$, $x \in X$. „$P$ je lineární“ je jasné, stejně tak spojitost $P$ (podobně jako v první části).

		$P|_{\spn(e_1, …, e_n)} = \id$:
		$$ \forall i \in [n]: P(e_i) = \sum_{j=1}^n F_j(e_i)e_j = \sum_{j = 1}^n f_j(q(e_j))e_j = e_i. $$
		Tedy $P$ je spojitá lineární projekce a navíc $\Ker P = Y$: $Px = 0 \Leftrightarrow F_i(x) = 0 \forall in \in [n]$ $\Leftrightarrow f_i(q(x)) = 0$, $\Leftrightarrow q(x) = 0$. Máme $X = \Rang P \oplus_t \Ker P$. Položíme $Q = \id - P$, pak $\Rang Q = \Ker P = Y$, $Q$ spojitá projekce.
	\end{dukazin}
\end{dusledek}

\begin{definice}
	Nechť $X, Y$ jsou normované lineární prostory a $T \in ©L(x, Y)$. Operátor $T^*: Y^* \rightarrow X^*$ definovaný předpisem $T^*f(x) = f(Tx)$ pro $f \in Y^*$ a $x \in X$ se nazývá duální (nebo též adjungovaný) operátor k $T$.

	Operátor $(T^*)^*$ značíme $T^{**}$.
\end{definice}

\begin{veta}
	Nechť $X, Y, Z$ jsou normované lineární prostory.

	\begin{enumerate}
		\item Je-li $T \in ©L(X, Y)$, je $T*f \in X^*$ pro každé $f \in Y^*$. Dále $T^* \in ©L(Y^*, X^*)$ a $||T^*|| = ||T||$.
		\item Zobrazení $T \mapsto T^*$ je lineární izometrie z $©L(X, Y)$ do $©L(Y^*, X^*)$.
		\item $T \in ©L(X, Y)$ a $S \in ©L(Y, Z)$. Pak $(S \circ T)* = T^* \circ S^*$. Dále $\id_X^* = \id_{X^*}$.
	\end{enumerate}

	\begin{dukazin}
		1. Spojitost $T*f$ je zřejmá z definice (složení dvou lineárních funkcí), stejně tak linearita $T$. Dále
		$$ \forall y^* \in B_{Y^*}: ||T^*y^*|| = \sup_{x \in B_X} |T^* y^* (x) | = \sup_{x \in B_X} |y^*(Tx)| ≤ \sup_{x \in B_X} ||Tx|| = ||T||, $$
		tedy $||T^*|| ≤ ||T||$ a $T$ je spojité. Zbývá $||T|| ≤ ||T^*||$. (Dokazujeme opačnou nerovnost k té výše.) Zvolme $x \in B_X$. Najdi (z jednoho z důsledků H-B) $y^* \in S_{Y^*}$. $||T_x|| = |y^*(Tx)|$. Pak
		$$ ||Tx|| = |y^*(Tx)| = |T^*y^*(x)| ≤ ||T^*||·||y^*||·||x|| ≤ ||T^*||. $$
		Tj. $||T|| ≤ ||T^*||$.

		2. Linearita zobrazení plyne z předpisu a izometrie pak plyne z prvního bodu.

		3. $\forall z^* \in Z^*\ \forall x \in X:$
		$$ \((S \circ T)^* z^*\)(x) = z^*(S(T(x))) = S^*z^*(Tx) = \(T^*S^*z^*\)(x). $$
		A to platí pro všechna $x$ a $z^*$, tedy funkcionály na ně aplikované musí být tytéž. Identita je triviální z definice.
	\end{dukazin}
\end{veta}

\begin{veta}
	Nechť $H_1, H_2$ jsou Hilbertovy prostory a $T \in ©L(H_1, H_2)$. Pak existuje jednoznačně určený operátor $T^\bigstar \in ©L(H_2, H_1)$ takový, že pro každé $y \in H_2$ a $x \in H_1$ platí
	$$ \<Tx, y\>_{H_2} = \<x, T^\bigstar y\>_{H_1}. $$
	Dále platí, že $T^\bigstar = I_1^{-1} \circ T^* \circ I_2$, kde $I_j: H_j \rightarrow H_j^*$, $j = 1, 2$ jsou příslušné sdružené lineární izometrie z věty výše (89 ve skriptech). ($I_i: y \mapsto \<·, y\> \in H_1^*$.)

	\begin{dukazin}
		Zvol $x \in H_1$, $y \in H_2$. Uvažuj $g \in (H_1)^*$ definované předpisem $\<Tx, y\>_{H_1}$. Dle věty 89 ve skriptech, $\exists! z \in H_1: g(x) = \<x, z\>$, $x \in H_1$. Tedy rovnost z věty platí $\Leftrightarrow T^\bigstar y = z$. Celkem $\exists! T^\bigstar: H_2 \rightarrow H_1$, pro které platí rovnost ze znění.

		Zbývá: $T^\bigstar = I_1^{-1} \circ T^* \circ I_2$ (pak operátor $T^\bigstar$ je lineární a spojitý). Stačí jen, že $I_1^{-1} \circ T^* \circ I_2$ splňuje rovnost ze zadání, protože existuje právě jeden takový operátor. Z definice $I_i$ a přelévání písmenek (definice sdruženého operátoru) tedy:
		$$ \forall x \in H_1\ \forall y \in H_2: \<x, \(I_1^{-1} \circ T^* \circ I_2\)(y)\>_{H_1} = $$
		$$ \(I_1\(I_1^{-1} \circ T^* \circ I_2\)\)(x) = \(T^* \circ I_2\)(x) = (I_2y)(Tx) = \<Tx, y\>. $$
	\end{dukazin}
\end{veta}

\begin{definice}[Hilbertovsky adjungovaný operátor]
	Operátor $T^\bigstar$ z předcházející věty nazýváme hilbertovsky adjungovaným operátorem k $T$.
\end{definice}

\begin{veta}
	Nechť $H_1, H_2, H_3$ jsou Hilbertovy prostory.
	
	\begin{enumerate}
		\item Je-li $T \in ©L(H_1, H_2)$, je $T^{\bigstar\bigstar} = (T^\bigstar)^\bigstar = T$.
% 16. 11. 2021
		\item Zobrazení $T \mapsto T^\bigstar$ je sdruženě lineární izometrie $©L(H_1, H_2)$ na $©L(H_2, H_1)$.
		\item Nechť $T \in ©L(H_1, H_2)$ a $S \in ©L(H_2, H_1)$. Pak $(S \circ T)^\bigstar = T^\bigstar \circ S^\bigstar$. Dále $(\id_{H_1})^\bigstar = \id_{H_1}$.
	\end{enumerate}

	\begin{dukazin}
		1. Máme
		$$ \forall x \in H_1\ \forall y \in H_2: \<T^{\bigstar\bigstar}x, y\>_{H_2} = \<x, T^\bigstar y\>_{H_1} = \<Tx, y\>_{H_2}. $$
		Tedy pro každé $x, y$ jsou tyto operátory stejné, tedy $T^{\bigstar\bigstar} = T$.

		2. Sdružená linearita: Zachování „$+$“ plyne ze vzorce, „zachování“ „$·$“:
		$$ \forall x, y\ \forall \alpha \in ®K: \<x, T^\bigstar \alpha y\> = \<Tx, \alpha y\> = \overline{\alpha} \<Tx, y\> = \overline{\alpha}\<x, T^\bigstar y\> $$.

		Izometrie plyne z toho, že $T^\bigstar$ je složení izometrií. To že je na plyne z 1.

		$$ 3. \forall x, y: \<x, (S \circ T)^\bigstar y\> = \<S(Tx), y\> - \<Tx, S^\bigstar y\> = \<x, T^\bigstar S^\bigstar y\>. $$
	\end{dukazin}
\end{veta}

\begin{definice}[Sdružený exponent]
	Nechť $p \in ®R$, $p ≥ 1$, nebo $p = ∞$. Číslo $q \in ®R$, $q ≥ 1$, nebo $q = ∞$ nazýváme sdruženým exponentem k $p$, pokud platí $\frac{1}{p} + \frac{1}{q} = 1$.
\end{definice}

\begin{veta}[Reprezentace duálů ke klasickým prostorům]
	Nechť $I ≠ \O$.

	\begin{enumerate}
		\item Prostor $c_0(I)*$ je lineárně izometrický s prostorem $l_1(I)$ pomocí zobrazení $I: l_1(I) \rightarrow c_0(I)^*$, $I(y) = f_y$, kde
			$$ f_y(x) = \sum_{i \in I} x_iy_i. $$
		\item Je-li $1 ≤ p < ∞$ a $q$ je sdružený exponent k $p$, pak prostor $l_p(I)^*$ je lineárně izometrický s prostorem $l_q(I)$ pomocí zobrazení $I: l_q(I) \rightarrow l_p(I)^*$, $I(y) = f_y$, kde
			$$ f_y(x) = \sum_{i \in I}x_iy_i. $$
		\item Je-li $(\Omega, S, \mu)$ libovolný prostor s mírou $1 < p < ∞$ a $q$ je sdružený exponent k $p$, pak prostor $L_p(\mu)^*$ je lineárně izometrický s prostorem $L_q(\mu)$ pomocí zobrazení $I: L_q(\mu) \rightarrow L_p(\mu)^*$, $I(g) = \phi_g$, kde
			$$ \phi_g(f) = \int_\Omega f·g d\mu. $$
		\item Je-li $(\Omega, S, \mu)$ prostor se $\sigma$-konečnou mírou, pak prostor $L_1(\mu)^*$ je lineárně izometrický s prostorem $L_∞(\mu)$ pomocí zobrazení $I: L_∞(\mu)$ pomocí zobrazení $I: L_∞(\mu) \rightarrow L_1(\mu)^*$, $l(g) = \phi_g$, kde
			$$ \phi_g(f) = \int_\Omega f·g d \mu. $$
	\end{enumerate}

	\begin{dukazin}[1.]
		$||I|| ≤ 1$:
		$$ \forall y \in l_1(I)\ \forall x \in c_0(I)\ \forall F \in ©F(I): |\sum_{i \in F} y_ix_i| ≤ \sum_{i \in F}|y_ix_i| ≤ ||x||_{∞}·\sum_{i \in F}|y_i| ≤ ||x||_∞·||y||_1 \implies $$
		$$ \implies |I(y)(x)| ≤ ||x||_∞·||y||_1, $$
		takže opravdu $I(y) \in c_0(I)^*$ a navíc $||I(y)|| ≤ ||y||_1$, tedy $I$ je lineární, dobře definované, $||I|| ≤ 1$.

		Izometrie: Zvol $y \in l_1(I)$, zvol $F \in ©F(I)$. Polož $x_F := \sum_{i \in F, y(i)≠0} \frac{|y(i)|}{y(i)}e_i \in B_{c_0(I)}$. Pak
		$$ ||I(Y)|| ≥ |I(y)(x_F)| = |\sum_{i \in F, y(i) ≠ 0} y(i)·\frac{|y(i)|}{y(i)}| = \sum_{i \in F}|y(i)|. $$
		Tedy, protože $||y||_1 = \sup_{F \in ©F(I)} \sum_{i \in F}y(i)$, dostáváme $||I(y)|| ≥ ||y||$.

		Zbývá už jen „na“: Zvol $f \in c_0(I)^*$. Polož $y(i) := f(e_i)$, $i \in I$. Pak $y \in l_1(I)$: Zvol $F \in ©F(I)$. Pak
		$$ \sum_{i \in F}|y(i)| = \sum_{i \in F, y(i) ≠ 0} y(i)·\frac{|y(i)|}{y(i)} = \sum_{i \in F, y(i) ≠ 0} f(e_i)·\frac{|y(i)|}{y(i)} = f\(\sum_{i \in F, y(i) ≠ 0} \frac{|y(i)|}{y(i)}·e_i\) ≤ ||f||. $$
		Tudíž $y \in l_1(I)$ (a $||y||_1 ≤ ||f||$).

		Chceme $I(y) = f$: Máme $\forall i \in I: I(y)(e_i) = y(i) = f(e_i)$. Tedy $I(y) = f$ na $e_i$, takže z linearity a spojitosti na $\overline{\spn}(e_i, i \in I) = c_0(I)$.
	\end{dukazin}

	\begin{dukazin}[2.]
		Případ $p=1$: $||I|| ≤ 1$ se dokáže jako v důkazu 1:
		$$ \forall y \in l_∞(I)\ \forall x \in l_1(I)\ \forall F \in ©F(I): \sum_{i \in F}|y_ix_i| ≤ ||y||_∞·||x||_1. $$

		$I$ izometrie: Ať $y \in l_∞(I)$, pak
		$$ \forall i \in I: ||I(y)|| ≥ |I(y)(e_i)| = |y(i)| \implies ||I(y)|| ≥ \sup_i |y(i)| = ||y||_∞. $$

		$I$ je na: Ať $f \in l_1(I)^*$. Polož $y(i) := f(e_i)$, $i \in I$. Pak $y \in l_∞(I):$
		$$ \forall i \in I: |y(i)| = |f(e_i)| ≤ ||f|| \implies ||y||_∞ ≤ ||f||. $$

		$I(y) = f$ je totožné jako v důkazu 1.
		
		2. Případ $p > 1$: $||I|| ≤ 1$ se dokáže podobně jen se použije Hölder:
		$$ \forall y \in l_q(I)\ \forall x \in l_p(I)\ \forall F \in ©F(I): \sum_{i \in F}|y_ix_i| ≤ ||y||_q·||x||_p. $$

		$I$ izometrie: Ať $y \in l_q(I)$. Polož $x_F = \frac{\sum_{i \in F, y(i) ≠ 0} \frac{|y(i)|}{y(i)}e_i}{||---||---||_p} \in S_{l_p(I)}$ (BÚNO $\exists i \in F: y(i) ≠ 0$).
		$$ x_F = \(\sum |y(i)|^{p(q - 1)}\)^{-\frac{1}{p}} · \sum_{i \in F, y(i) ≠ 0}\frac{|y(i)|^q}{y(i)}e_i $$
		a zároveň
		$$ ||I(y)|| ≥ I(y)(x_F) = \(\sum |y(i)|^{p(q - 1)}\)^{-\frac{1}{p}} · \sum_{i \in F} |y(i)|^q = ||y(i)||_q. $$

		$I$ je na: Ať $f \in l_p(I)^*$. Polož $y(i) := f(e_i)$, $i \in I$. Pak $y \in l_q(I)$: Zvol $F \in ©F(I)$. Pak polož $x_F = \sum_{i \in F, y(i) ≠ 0 \frac{|y(i)|^q}{y(i)}} e_i$.
		$$ \sum_{i \in F}|y(i)|^q = \sum_{i \in F} f(e_i) x_F(i) = f(\sum_{i \in F} x_F(i)·e_i) ≤ ||f||·||x_F||_p = ||f||\(\sum_{i \in F} |y(i)|^q\)^{\frac{1}{p}} $$.
		Celkem
		$$ \inf_{F \in ©F(i)} \(\sum_{i \in F} |y(i)|^q\)^{1 - \frac{1}{p}} ≤ ||f||, $$
		tedy $y \in l_q(I)$ a $||y||_q ≤ ||f||$.
	\end{dukazin}

% 22. 11. 2021

	\begin{dukazin}[3, 4]
		1. krok $\mu$ konečná: $I$ je spojitý, lineární a $||I|| ≤ 1$: $p = 1$:
		$$ |I(f)(g)| ≤ \int_{\Omega} |f g| d\mu ≤ ||f||_∞ \int_{\Omega}|g| d\mu = ||f||_∞·||g||_1. $$
		Tedy $I$ je dobře definované, lineární a $||I|| ≤ 1$. $p > 1$:
		$$ |I(f)(g)| ≤ \int_{\Omega} |f g| d\mu ≤ ||f||_q·||g||_p. $$
		Tedy $I$ je dobře definované, lineární a $||I|| ≤ 1$.

		$I$ je izometrie: $p > 1$: Ať $f \in L_q(\Omega)$, BÚNO $f ≠ 0$. Zvol
		$$ g(x) := \frac{\frac{|f(x)|^q}{f(x)} \chi_{\{x|f(x) ≠ 0\}}}{||--||--||} \in S_{L_p(\mu)} = \(\int_\Omega |f(x)|^{p(q-1)} dx\)^{\frac{1}{p}}·\frac{|f(x)|^q}{f(x)} \chi_{\{x | f(x) ≠ 0\}}, $$
		$$ ||f|| ≥ ||I(f)|| ≥ I(f)(g) = \(\int_\Omega |f(x)|^q d\mu(x) \)^{-\frac{1}{p}}·\int_\Omega |f(x)|^q d\mu(x) = ||f||_q. $$
		Tedy $||I(f)|| = ||f||$ a $I$ je izometrie.

		$p=1$: Ať $f \in L_∞(r)$, BÚNO $f ≠ 0$. Zvol $||f||_∞ > \epsilon > 0$ je libovolné, ať
		$$ A = \{x | f(x) > ||f||_∞ - \epsilon\}. $$
		Pak $\mu(A) > 0$. Ať $\mu(A) < ∞$ (v případě $\sigma$-konečné míry můžeme $A$ aproximovat). Polož $g(x) := \frac{|f(x)|}{f(x)} \frac{\chi_A}{\mu(A)} \in B_{L_{1, \mu}}$. Pak
		$$ ||f|| ≥ ||I(f)|| ≥ I(f)(g) = \int_\Omega |f(x)| \chi_A(x)·\frac{1}{\mu(A)} d\mu(x) > \frac{||f||_∞ - \epsilon}{\mu(A)} \int_A 1 d\mu(x) = ||f||_∞ - \epsilon. $$

		$I$ je na: Ať $x^* \in (L_p)^*$. Položme $\nu(A) := x^*(\chi_A)$, $A \in ©A$. Pak $\nu$ je ®K-hodnotová míra: $\nu(\O) = x^*(0) = 0$. Ať $(A_j)_{j=1}^∞$ posloupnost množin z ©A, po 2 disjunktní. Pak
		$$ ||\chi_{\bigcup_{j=1}^∞ A_j} - \chi_{\bigcup_{j=1}^n A_j}||_p = \mu(TODO) $$
		Tedy
		$$ \nu(\bigcup_{j=1}^∞ A_j) = x^*(\chi_{\bigcup_{j=1}^∞ A_j}) = \lim_{n \rightarrow ∞} x^*(\chi_{\bigcup_{j=1}^∞ A_j}) = \lim_{n \rightarrow ∞} x^*(\chi_{A_1}) + … + x^*(\chi_{A_n}) = $$
		$$ = \lim_{n \rightarrow ∞} \sum_{i=1}^n \nu(A_i) = \sum_{i=1}^∞ \nu(A_i). $$
		Zároveň $\nu << \mu$:
		$$ \mu(A) = 0 \implies \chi_A = 0 \text{ skoro všude } \implies x^*(\chi_A) = 0. $$
	\end{dukazin}

	\begin{dukazin}[Pokračování 3, 4]
		Tedy z R-M věty $\exists g \in L_1(\mu)$: $\nu(A) = \int_A g d\mu$, $A \in ©A$. Pak $x^*(s) = \int_{\Omega} g·s d\mu$, pro $s$ jednoduchou funkci. Tedy pro $f \in (L_p(\mu))$ najdu $s_k \rightarrow f$, $|s_k| ≤ 4f$, $s_k$ jednoduché. Pak ale $s_k \stackrel{L_p}{\rightarrow} f$ (z Lebesgueovy věty, jelikož $5f$ je integrovatelná majoranta). Tedy
		$$ x^*(f) = \lim_{k} x^*(s_k) = \lim_k \int_\Omega g·s_k d\mu = \int_\Omega g·f d\mu. $$

		Poslední věc, co zbývá je $g \in L_q(\mu)$: $p = 1$: Chceme $|g(x)| ≤ ||x^*||$ skoro všude. Pokud ne, pak $A = \{x | |g(x)| > ||x^*||\}$ má kladnou míru. Ať $A_+ = \{x | g(x) > ||x^*||\}$ má kladnou míru. Pak
		$$ ||x^*|| \mu(A_+) ≤ | \int_{A_+} g d\mu | = |x^*(\chi_{A_+})| ≤ ||x^*||\mu(A_+). \text{\lightning}. $$
		Podobně pro $A_- := \{x | g(x) < -||x^*||\}$. $p > 1$ vynecháme.

		Další kroky byly vynechány.
	\end{dukazin}
\end{veta}

\begin{veta}
	Nechť $X, Y$ jsou normované lineární prostory a $1 ≤ p ≤ ∞$. Nechť $q$ je sdružený exponent k $p$. Pak zobrazení $I: X^* \oplus_q Y^* \rightarrow (X \oplus_p Y)^*$ dané předpisem
	$$ I(f, g)(x, y) = f(x) + g(y) $$
	je lineární izometrie $X^* \oplus_q Y^*$ na $(X \oplus_p Y)^*$.

	\begin{dukazin}
		$I(f, g)$ lineární pro $(f, g) \in X^* \oplus_q Y^*$ lehké. Zvol $(f, g) \in X^* \oplus_q Y^*$. Pak
		$$ ||I(f, g)|| = \sup_{(x, y) \in B_{X \oplus_p Y}} |f(x) + g(y)| ≤ \sup_{(x, y) \in B_{X \oplus_p Y}} (||f||·||x|| + ||g||·||y||) = $$
		$$ = \sup_{(\alpha, \beta) \in B_{(®K^2, ||·||_p)}} \tilde{I}(||f||, ||g||)(\alpha, \beta) = ||\tilde{I}(||f||, ||g||)|| = ||(||f||, ||g||)||_q = \sqrt[q]{||f||^q + ||g||^q} = $$
		$$ = ||(f, g)||_{X^* \oplus_q Y^*}. $$
		Tedy $||I|| ≤ 1$.

		$I$ je izometrie: Ať $(f, g) \in X^* \oplus_q Y^*$, BÚNO $(f, g) ≠ 0$. Zvol $\epsilon > 0$ libovolné. Ať $\eta > 0$ je „dost malé“: Zvolme
		$$ x \in B_x: |f(x)| > ||f|| - \eta, |\alpha| = 1, |f(x)| = \alpha f(x), $$
		$$ y \in B_y: |f(y)| > ||f|| - \eta, |\beta| = 1, |f(x)| = \beta f(y). $$
		Položme
		$$ u = \frac{(||f||^{q-1} \alpha x, ||g||^{q-1} \beta y)}{(||f||^q + ||g||^q)^{\frac{1}{p}}} = \frac{…}{C}. $$
		$$ ||u|| = \(\frac{1}{C} ||f||^{p(q-1)}||\alpha x||^p + ||g||^{p(q - 1) ||\beta y||^p}\)^{\frac{1}{p}} ≤ \frac{1}{C}(||f||^q + ||g||^q)^{\frac{1}{p}} = 1, $$
		tedy $u \in B_{…}$.
		Pak ale
		$$ ||I(f, g)|| ≥ I(f, g)(u) = \frac{1}{c} (||f||^{q-1} f(\alpha x) + ||g||^{q-1} g(\beta y)) ≥ $$
		$$ ≥ \frac{1}{C}(||f||^{q-1} (||f|| - \eta) + ||g||^{q-1} (||g|| - \eta)) \rightarrow \frac{1}{C}· (||f||^q + ||g||^q) = ||(f, g)||. $$

% 23. 11. 2021

		$I$ je na: Ať $\phi \in (X \oplus_p Y)^*$. Polož $f(x) := \phi(x, 0)$, $x \in X$ a $g(x) := \phi(0, y)$, $y \in Y$. Pak $f \in X^*$, $g \in Y^*$ a $I(f, g) = \phi$.
	\end{dukazin}
\end{veta}

\begin{definice}
	Nechť $K$ je kompaktní prostor. Řekneme, že lineární funkcionál $\Lambda$ na $C(K)$ je nezáporný, jestliže $\Lambda(f) ≥ 0$ pro každou nezápornou funkci $f \in C(K)$.
\end{definice}

\begin{veta}[O reprezentaci nezáporných lineárních funkcionálů na $C(K)$]
	Nechť $K$ je kompaktní prostor a $\Lambda$ je nezáporný lineární funkcionál na $C(K)$. Pak existuje jednoznačně určená regulární borelovská nezáporná míra $\mu$ na $K$ splňující $\Lambda(f) = \int_K f d\mu$ pro každé $f \in C(K)$.

	\begin{dukazin}
		Bez důkazu.
	\end{dukazin}
\end{veta}

\begin{veta}[Rieszova věta o reprezentaci $C(K)^*$]
	Je-li $K$ kompaktní prostor, pak prostor $C(K)^*$ je lineárně izometrický s prostorem $M(K)$ všech regulárních borelovských komplexních (resp. znaménkových) měr na $K$ pomocí zobrazení $I: M(K) \rightarrow C(K)^*$, $I(\mu)_k = \phi_k$, kde
	$$ \phi_{\mu}(f) = \int_K f d\mu. $$

	\begin{dukazin}
		Bez důkazu.
	\end{dukazin}
\end{veta}

\section{Anihilátory, dualita kvocientů a podprostorů}
\begin{definice}[Horní a dolní anihilátor]
	Je-li $X$ normovaný lineární prostor a $A \subset X$, pak definujeme tzv. anihilátor množiny $A$ jako
	$$ A^\perp = \{f \in X^* | f(x) = 0\ \forall x \in A\}. $$

	\begin{poznamka}
		Vlastně je to zobecnění kolmého prostoru (v Hilbertových prostorech je to „totéž“).
	\end{poznamka}

	Pro množinu $B \subset X^*$ pak definujeme tzv. zpětný anihilátor jako
	$$ B_\perp = \{x \in X | f(x) = 0\ \forall f \in B\}. $$
\end{definice}

\begin{lemma}
	Nechť $X$ je normovaný lineární prostor a $A \subset X$, $B \subset X^*$. Pak

	\begin{itemize}
		\item $A^\perp$ je uzavřený podprostor $X^*$,
		\item $B_\perp$ je uzavřený podprostor,
		\item $(A^\perp)_\perp = \overline{\spn}A$,
		\item $(B_\perp)^\perp \supset \overline{\spn}B$.
	\end{itemize}

	\begin{dukazin}
		První dva body triviální cvičení. Další dva body jsou lehké.
	\end{dukazin}
\end{lemma}

\begin{veta}
	Nechť $X$ je normovaný lineární prostor a $Y$ jeho podprostor.

	\begin{enumerate}
		\item Nechť $Y$ je uzavřený. Zobrazení $I: Y^\perp \rightarrow (X / Y)^*$ dané předpisem
			$$ I(f)(\hat{x}) = f(x) $$
			je lineární izometrie $Y^\perp$ na $(X / Y)^*$.

		\item Zobrazení $I: X^* / Y^\perp \rightarrow Y^*$ dané předpisem
			$$ I(\hat{f}) = f|_Y $$
			je lineární izometrie $X^* / Y^\perp$ na $Y^*$.
	\end{enumerate}

	\begin{dukazin}
		1. a) $I(f)$ je dobře definované: Ať $\hat{x} = \hat{y}$, pak $x - y \in Y$ a $f \in Y^\perp$ (tj. $f(x - y) = 0$), tedy $f(x) = f(y)$.

		b) Zároveň $||I(f)|| = \sup_{\hat{x} U_{X / Y}} |I(f)(\hat{x})| = \sup_{x \in U_x} |I(f)(\hat{x})| = \sup_{x \in U_x} |f(x)| = ||f||$, tedy $I$ je spojité a izometrie (linearita je triviální).

		c) Ať $\phi \in (X / Y)^*$. Pak $\phi \circ q \in X^*$ a $I(\phi \circ q) = \phi$ $\land$ $\phi \circ q \in Y^\perp$: $\forall y \in Y: \phi(q(y)) = \phi(\hat{0}) = 0$. Tedy $\phi \circ q \in Y^\perp$. $\forall \hat{x} \in X / Y: I(\phi\circ q)(\hat{x}) = (\phi \circ q)(x) = \phi(\hat{x})$, tedy $I(\phi \circ q) = \phi$.

		2. a) $I(\hat{f})$ je dobře definované: Ať $\hat{f} = \hat{g}$, pak $f - g \in Y^\perp$, tedy $f|_Y = g|_Y$.

		b) $I$ zřejmě lineární. Zároveň $||I(\hat{f})|| = \sup_{y \in B_y} ||f(y)|| = ||f|_Y|| ≤ \inf_{h \in \hat{f}} ||h|| = ||\hat{f}||$.

		c) $I$ je izometrie: Ať $\hat{f} \in X^* / Y^\perp$. Zvol $g \in X^*: g|_Y = f|_Y \land ||g|| = ||f|_Y||$ z H-B věty. Pak $\hat{g} = \hat{f}$ a $||I(\hat{f})|| = ||I(\hat{g})|| = ||g|_Y|| = ||g|| ≥ \inf_{h \in \hat{f}} ||h|| = ||f_|Y||$.

		d) $I$ je na: Ať $\phi \in Y^*$. Z H-B věty existuje $f \in X^*: f|_Y = \phi$. Pak $I(\hat{f}) = f|_Y = \phi$.
	\end{dukazin}
\end{veta}

\begin{veta}
	Jsou-li $X, Y$ normované lineární prostory a $T \in ©L(X, Y)$, pak platí

	\begin{enumerate}
		\item $\Ker T^* = (\Rang T)^\perp$,
		\item $\Ker T = (\Rang T^*)_\perp$,
		\item $\overline{\Rang T} = (\Ker T^*)_\perp$,
		\item $T^*$ je prostý, právě když $\Rang T$ je hustý.
	\end{enumerate}

	\begin{dukazin}
		1. $y^* \in \Ker T^* \Leftrightarrow T^*y^* = 0 \Leftrightarrow y^*\circ T = 0 \Leftrightarrow y^*|_{\Rang T} = 0$.

		2. $x \in \Ker T \Leftrightarrow Tx = 0 \Leftrightarrow \forall y^* \in Y^*: y^*(Tx) = 0 \Leftrightarrow \forall y^* \in Y^*: T^*y^*(x) = 0 \Leftrightarrow x \in (\Rang T^*)_\perp$.

		3. $\overline{\Rang T} = ((\Rang T)^\perp)_\perp = (\Ker T^*)_\perp$.

		4. $T^*$ prostý $\Leftrightarrow$ $\Ker T = \{¦o\}$, ale $\{¦o\}\perp = Y$, tedy dle 3. $\overline{\Rang R} = Y$. Naopak sporem: Ať $\exists x \in Y / \overline{\Rang T}$. Potom dle H-B věty $\exists f \in Y^*: f(x) ≠ 0 \land f|_{\Rang T} = 0$. Pak ale
		$$ T^* f(x_0) = f(Tx_0) = 0, \forall x_0 \in X \implies T^*f = 0 \implies f \in \Ker T^*. \text{\lightning}. $$
	\end{dukazin}
\end{veta}

\begin{definice}[Druhý duál, evaluační funkcionál]
	Nechť $X$ je normovaný lineární prostor. Symbolem $X^{**}$ značíme $(X^*)^*$, tj. duál k prostoru $X^*$. Tento prostor nazýváme druhým duálem.

	Je-li $x \in X$, pak definujeme tzv. evaluační funkcionál $\epsilon_x \in X^{**}$ předpisem $\epsilon_x(f) = f(x)$ pro každé $f \in X^*$. Zobrazení $\epsilon: X \rightarrow X^{**}$ dané předpisem $\epsilon(x) = \epsilon_x$ se nazývá kanonické vnoření $X$ do $X^{**}$.
\end{definice}

\begin{tvrzeni}
	Nechť $X$ je normovaný lineární prostor. Pak kanonické vnoření $\epsilon: X \rightarrow X^{**}$ je lineární izometrie do. Je-li tedy navíc $X$ Bansachův, pak $\epsilon(X)$ je uzavřený podprostor $X^{**}$.

	\begin{dukazin}
		Linearita zřejmá. Izometrie
		$$ ||\epsilon_x|| = \sup_{x^* \in B_{X^*}} |\epsilon_* (x^*)| = \sup_{x^* \in B_{X^*}} |x^*(x)| = ||x||. $$
	\end{dukazin}
\end{tvrzeni}

\begin{dusledek}
	TODO.
\end{dusledek}

\begin{tvrzeni}[J. P. Schauder, 1930]
	Nechť $X$, $Y$ jsou normované lineární prostory, $\epsilon_X: X \rightarrow X^{**}$ a $\epsilon_Y: $
\end{tvrzeni}

% 29. 11. 2021

\begin{tvrzeni}
	Komutování kanonických vnoření do duálů? TODO

	\begin{dukazin}
		TODO
	\end{dukazin}
\end{tvrzeni}

\begin{veta}
	Pro každý normovaný lineární prostor $X$ existuje jeho zúplnění, tj. Banachův prostor takový, že $X$ je jeho hustý podprostor. Pro každý prostor se skalárním součinem $X$ existuje jeho zúplnění, tj. Hilbertův prostor takový, že $X$ je jeho hustý podprostor.

	Tato rozšíření jsou určena jednoznačně až na izometrii, tj. jsou-li $X_1$, $X_2$ dvě zúplnění $X$, pak existuje lineární izometrie $X_1$ na $X_2$, která je na $X$ identitou.

	\begin{dukazin}
		Položme $\hat{X} = \overline{©\epsilon(X)} \subseteq X^{**}$. Ztoho plyne existunce.

		Pokud $X$ má skalární součin, pak platí rovnoběžníkové pravidlo. To platí i v $\hat{X}$, tedy $\hat{X}$ je Hilbertův.

		Ať $I_1: X \rightarrow X_1$ je izometrie, $\overline{I_1(X)} = X_1$, $I_2: X \rightarrow X_2$ je izometrie, $\overline{I_2(X)} = X_2$. Pak $I_1 \circ I_2^{-1}|_{I_2(X)}: I_2(X) \rightarrow X_1$ je spojitý lineární operátor, tedy $\exists! S_1: X_2 \rightarrow X_1$ spojitý lineární, že $S_1 \supset I_1 \circ I_2^{-1}|_{I_2(X)}$. Obdobně existuje $S_2: X_1 \rightarrow X_2$. Pak se snadno ověří, že $(S_2 \circ S_1)|_{I_2(x)} = \id|_{I_2(x)}$, tedy $S_2 \circ S_1 = \id$. Analogicky $S_1 \circ S_2 = \id$.

		Následně se ukáže, že $S_1$ je izometrie: Zvol $x \in X_2$, ať pro $(x_n)_{n=1}^∞$, posloupnost v $X$, je $I_2(x_n) \rightarrow x$. Pak
		$$ ||S_1 x|| = \lim_{n \rightarrow ∞}||S_1(I_2(x_n))|| = \lim_{n \rightarrow ∞} ||I_1(x_n)|| = \lim_{n \rightarrow ∞}||x_n|| = ||x||. $$
		Analogicky $S_2$ je izometrie, tedy $X_1$, $X_2$ jsou izometrické.
	\end{dukazin}

\end{veta}
	
\begin{veta}
	Nechť $X$, $Y$ jsou Banachovy prostory a $T \in ©L(X, Y)$.

	\begin{enumerate}
		\item $T$ je izomorfismus na, právě když $T^*$ je izomorfismus na. V tomto případě navíc platí $(T^*)^{-1} = (T^{-1})^*$.
		\item $T$ je izometrie na, právě když $T^*$ je izometrie na.
	\end{enumerate}

	Speciálně, jsou-li $X$ a $Y$ lineárně izometrické, pak jsou také $X^*$ a $Y^*$ lineárně izometrické.

	\begin{dukazin}
		$\implies$ (1.):
		$$ \forall y^* \in Y^*: ((T^{-1})^* T^*(y^*))(y) = T^*y^*(T^{-1}y) = y^*(T T^{-1} y) = y^{*}(y). $$
		Analogicky $T^*\circ (T^{-1})^* = \id x^{*}$.

		$\Leftarrow$ (1.): Dle první části: $T^*$ je izomorfismus $\implies$ $T^{**}$ je izomorfismus $\implies$ $\epsilon_Y\circ T$ je izomorfismus $\implies$ $T$ je izomorfismus.

		$\implies$ (2.): Dle 1. stačí: $T^*$ je izometrie:
		$$ \forall y^* \in Y^*: ||T^*y^*|| = \sup_{x \in B_X} |y^*(Tx)| = \sup_{y \in B_y} |y^*(y)| = ||y^*||. $$
		Opačná implikace analogicky jako v 1.
	\end{dukazin}
\end{veta}

\begin{definice}[Reflexivní prostor]
	Banachův prostor $X$ se nazývá reflexivní, pokud $X^{**} = \epsilon(X)$.

	\begin{upozorneni}
		Existují i prostory, pro které je $X$ izometrické $X^{**}$, ale ne pomocí $\epsilon$.
	\end{upozorneni}
\end{definice}

\begin{veta}
	Nechť $X, Y$ jsou Banachovy prostory.

	\begin{itemize}
		\item Je-li $X$ izomorfní s reflexivním prostorem, pak je i $X$ reflexivní.
		\item Je-li $Y$ uzavřený podprostor $X$, $X$ reflexivní $\implies$ $Y$ reflexivní.
		\item Prostor $X$ je reflexivní právě tehdy, když jeho duál $X^*$ je reflexivní.
		\item Je-li $X$ reflexivní a $Y$ jeho uzavřený podprostor, pak je $X / Y$ reflexivní.
	\end{itemize}

	\begin{dukazin}
		1. Zvol $y^{**} \in Y^{**}$. Ať $T: Y \rightarrow X$ je izomorfismus. Pak $T^{**}y^{**} \in X^{**}$ $\implies$ $\exists x \in X: \epsilon_X(x) = T^{**}y^{**}$. Polož $y = T^{-1}x \in Y$. Následně dokážeme, že $\epsilon_Y(y) = y^{**}$:
		$$ \forall y^* \in Y^*: \epsilon_Y(y)(y^*) = y^*(y) = y^*(T^{-1}x) = (T^{-1})^* y^*(x) = T^{**}y^{**} ((T^{-1})^*y^*) = $$
		$$ = y^{**}(T^*(T^{-1})^* y^*) = y^{**}(y^*). $$

		2. Zvol $y^{**} \in Y^{**}$ a uvažujme
		$$ F(X^*) = y^{**}(x^*|_Y), x^* \in X^*. $$
		Pak $F \in X^{**}$ (lehké ověřit) $\implies \exists x \in X: F = \epsilon_X(x)$. $x \in Y$, jelikož: Ať ne, pak (dle H-B) $\exists f \in X^*: 0 ≠ f(x)\land f|_Y ≡ 0$. Pak $F(f) = y^{**}(0) = 0$, \lightning.

		Teď už jen ověříme, že $\epsilon_Y(x) = y^{**}$: Zvol $y^* \in Y^*$. Dle H-B existuje $x^* \in X^*$, že $x^*|_Y = y^*$. Pak
		$$ y^{**}(y^*) = y^{**}(x|_Y) = F(x^*) = x^*(x) = \epsilon_Y (x) (x^*) $$

		3. $\implies$: Zvol $x^{***} \in X^{***}$. Uvažuj $x^* = x^{***}\circ \epsilon_X \in X^*$. Pak
		$$ \forall x \in X: x^{***}(\epsilon_X(x)) = x^*(x) = \epsilon_X(x)(x^*) = \epsilon_{X^*}(x^*)(\epsilon_X(x)) $$
		$$ \implies x^{***} = \epsilon_{X^*}(x^*), \text{ na } \epsilon_X(x) = x^{**}. $$
		Ať $\phi \in (X / Y)^{**}$, pak
		$$ I^*(\phi) = (Y^\perp)^* \implies \exists F \in X^{**}: I^*(\phi) \subset F \implies \exists x \in X: F = \epsilon_X(x). $$
		Potom už jen chceme $\epsilon_{X / Y}(q(x)) = \phi$:
		$$ \forall f \in Y^\perp: \epsilon_{X / Y}(q(x))(I(f)) = I(f)(q(x)) = f(x) = F(f) = (I^*(\phi))(f) = \phi(I f) \implies $$
		$$ \implies \epsilon_{X / Y}(q(x)) = \phi. $$
	\end{dukazin}
\end{veta}

\begin{veta}
	Banachův prostor $X$ je reflexivní, právě když pro každé $x^* \in X^*$ existuje $x \in B_X$ splňující $||x^*|| = x^*(x)$.

	\begin{dukazin}
		Bez důkazu.
	\end{dukazin}
\end{veta}

% 30. 11. 2021

\section{Slabá konvergence}
\begin{definice}[Slabá konvergence, s. konvergence s hvězdičkou]
	Nechť $X$ je normovaný lineární prostor.

	\begin{itemize}
		\item Řekneme, že posloupnost $\{x_n\}$ v prostoru $X$ slabě konverguje k $x \in X$ (značíme $x_n \overset{w}{\rightarrow} x$) pokud pro každé $x^* \in X^*$ platí $x^*(x_n) \rightarrow x^*(x)$.
		\item Řekneme, že posloupnost $\{x_n^*\}$ v prostoru $X^*$ slabě s hvězdičkou konverguje k $x^* \in X^*$ (značíme $x_n^* \overset{w^*}{\rightarrow} x^*$) pokud pro každé $x \in X$ platí $x_n^*(x) \rightarrow x^*(x)$.
	\end{itemize}
\end{definice}

\begin{lemma}
	Nechť $X$ je normovaný lineární prostor, $\{x_n\}$ posloupnost v $X$ a $\{x_n^*\}$ posloupnost v $X^*$.

	\begin{enumerate}
		\item Existuje nejvýše jedno $x \in X$ splňující $x_n \overset{w}{\rightarrow} w$.
		\item Existuje nejvýše jedno $x^* \in X^*$ splňující $x_n^* \overset{w^*}{\rightarrow} x^*$.
		\item Pokud $x \in X$ a $x_n \rightarrow x$, pak $x_n \overset{w}{\rightarrow} x$.
		\item Pokud $x^* \in X^*$ a $x_n^* \overset{w}{\rightarrow} x^*$, pak $x_n^* \overset{w^*}{\rightarrow} x^*$.
	\end{enumerate}

	\begin{dukazin}
		1.–4. triviální.
	\end{dukazin}
\end{lemma}

\begin{veta}
	Nechť $X$ je separabilní normovaný lineární prostor a $\{x_n^*\}$ omezená posloupnost v $X^*$. Pak $\{x_n^*\}$ má $w^*$-konvergentní podposloupnost.

	\begin{dukazin}
		Ať $\{x_n | n \in ®N\} \subseteq B_x$ hustá v $B_x$. 1. krok: Najdeme $(x^*_{n_k})$, že $x_{n_k}^*(x_n)$ je konvergentní pro $n \in ®N$: Ať $A_1 \subset ®N$ nekonečná. K $((x_k^*)(x_1))_{k \in A_1}$ je konvergentní. Totéž pro $A_2$ a $x_2$, $A_3$ a $x_3$, … Potom vybereme prvky na diagonále.

		2. krok: Pak $x_{n_k}^*(x)$ konverguje pro $x \in B_x$: $\epsilon > 0$ dáno. Ať $n \in ®N$, že $||x_n - x|| < \epsilon$.
		$$ k_0 \in ®N \forall k, l ≥ k_0: |x_{n_k}^*(x_n) - x_{n_k}^*(x_n)| < \epsilon. $$
		Pak
		$$ \forall k, l ≥ k_0: |x_{n_k}^*(x) - x_{n_k}^*(x)| ≤ |x_{n_k}^*(x - x_n)| + |x_{n_k}^*(x_n) - x_{n_l}^*(x_n)| + |x_{n_l}^*(x_n) - x_{n_l}^*(x)| < \epsilon(||x_{n_k}^*|| + 1 + ||x_{n_l}^*||). $$

		3. kror: Tedy z linearity $x_{n_k}^*(x)$ konverguje pro $x \in X$: Polož $x^*(x) = \lim_{k \rightarrow ∞} x_{n_k}^*(x)$. Pak $x* \in X^*$
	\end{dukazin}
\end{veta}

\begin{veta}
	Banachův prostor $X$ je reflexivní, právě když každá omezená posloupnost $\{x_n\}$ v $X$ má slabě konvergentní podposloupnost.

	\begin{dukazin}
		$\Leftarrow$ nebude (teď je těžký, bude v funkcionální analýze). $\implies$ plyne z následující věty: $X^*$ separabilní $\implies$ $X$ separabilní. Polož $Y = \overline{\spn}(x_n) \subset X$, pak $Y$ je separabilní a reflexivní $\implies$ $Y^*$ je (reflexivní +) separabilní, dle následující věty. $\implies \exists (x_{n_k})$, $w^*$-konvergentní podposloupnost v $Y^{**} ≡ \epsilon(Y)$ $\implies$ $\exists y \in Y: \epsilon(x_{n_k}) \overset{w^*}{\rightarrow} \epsilon(y)$ $\Leftrightarrow$ $x_{n_k} \overset{w}{\rightarrow} y$.
	\end{dukazin}
\end{veta}

\begin{veta}
	Nechť $X$ je normovaný lineární prostor a $X^*$ je separabilní. Pak $X$ je separabilní.

	\begin{dukazin}
		Zvol $\{x_n^* | n \in ®N\} \subset S_{X^*}$ TODO. Pro $n \in ®N$ najdi $x_n \in B_x: x_n^*(x_n) > \frac{1}{2}$. Pak $\overline{\spn}\{x_n | n \in ®N\} = X$ (a tím bude hotovo, protože $\overline{\spn}(x_n) = \overline{\spn_*}(x_n)$): Ať ne, pak existuje $f \in S_{X^*}: f |_{\overline{\spn}} = 0$, $f ≠ 0$. Zvol $n \in ®N$, že $||x_n^* - f|| < \frac{1}{4}$. Pak
		$$ 0 = |f(x_n)| ≥ |x_n^*(x_n)| - |(x_n^* - f)(x_n)| > \frac{1}{2} - \frac{1}{4} > 0.\text{ \lightning} $$
	\end{dukazin}
\end{veta}

\section{Omezené operátory v Banachových prostorech}
\begin{definice}[Kompaktní operátor, konečněrozměrný operátor]
	Nechť $X, Y$ jsou normované lineární prostory a $T: X \rightarrow Y$ je lineární zobrazení. Pak $T$ se nazývá kompaktní operátor, pokud pro každou omezenou $A \subset X$ je množina $T(A)$ relativně kompaktní (tj. její uzávěr je kompaktní) v $Y$.

	Množinu všech kompaktních lineárních operátorů z $X$ do $Y$ značíme $©K(X, Y)$.

	Lineární operátor $T$ se nazývá konečněrozměrný, pokud $\Rang T$ má konečnou dimenzi.

	$©F(X, Y)$ značí množinu všech konečněrozměrných spojitých lineárních operátorů z $X$ do $Y$.
\end{definice}

\begin{poznamka}
	$X$ je MP, $A \subset X$. Pak

	\begin{itemize}
		\item $A$ je relativně kompaktní $\leftrightarrow$ z každé posloupnosti v $A$ lze vybrat konvergentní posloupnost v $X$.
		\item Pokud $X$ je úplný, pak $A$ je relativně kompaktní $\leftrightarrow$ $A$ je totálně omezená.
	\end{itemize}
\end{poznamka}

\begin{tvrzeni}
	Nechť $X$, $Y$ jsou normované lineární prostory. Každý kompaktní lineární operátor z $X$ do $Y$ je automaticky spojitý. Dále, je-li $T: X \rightarrow Y$ lineární, pak následující tvrzení jsou ekvivalentní:

	\begin{enumerate}
		\item $T$ je kompaktní.
		\item $T(B_X)$ je relativně kompaktní.
		\item Je-li $\{x_n\}$ omezená posloupnost v $X$, pak posloupnost $\{T(x_n)\}$ má konvergentní podposloupnost.
	\end{enumerate}

	\begin{dukazin}
		1. $\implies$ 2: triviální. 2. $\implies$ 3: Ať $(x_n)$ je posloupnost v $B(O, r)$ (kde $r > 0$). Pak $(\frac{x_n}{r})$ je posloupnost v $B_x$ $\implies$ dle 2. $\exists (n_k)$, že $T(\frac{x_{n_k}}{r})$ je konvergentní, tedy $T(x_{n_k})$ je konvergentní.

		3. $\implies$ 1.: Ať $A \subset X$ omezená, ať $(y_n)$ je posloupnost v $T(a)$. Pak $\exists x_n \in A: T x_n = y_n$, $n \in ®N$. $\implies \exists (n_k): T_{x_{n_k}}$ je konvergentní v $Y$.
	\end{dukazin}
\end{tvrzeni}

\begin{veta}
	Nechť $X$, $Y$ jsou Banachovy prostory.

	\begin{enumerate}
		\item Operátor $T \in ©L(X, Y)$ je konečněrozměrný právě tehdy, když existují $f_1, …, f_n \in X^*$ a $y_1, …, y_n \in Y$ takové, že $T(x) = \sum_{i = 1}^n f_i(x)y_i$ pro každé $x \in X$.
		\item $©K(X, Y)$ je podprostor $©L(X, Y)$ a $©F(X, Y)$ podprostor $©K(X, Y)$.
		\item $©K(X, Y)$ je uzavřený podprostor $©L(x, Y)$.
		\item Složíme-li kompaktní lineární operátor se spojitým lineárním operátorem (zleva či zprava), dostaneme opět kompaktní operátor.
	\end{enumerate}

	\begin{dukazin}
		1. $\Leftarrow$: Jasné protože pak $\Rang T \subset \spn\{y_1, …, y_n\}$. $\implies$: Ať $y_1, …, y_n$ je báze $\Rang T$. Uvažujme $g_i \in (\Rang T)^*$, $g_i(y_j) = \delta_{ij}$. Polož $f_i = g_i \circ T \in X^*$, $i \in [n]$. Pak $T(x) = \sum_{i=1}^n g_i(Tx)y_i = \sum_{i=1}^n f_i(x)y_i$.

		2. Ať $S, T \in ©K(X, Y)$. Pak $(S + T)(B_x) = S(B_X) + T(B_X) \subseteq \overline{S(B_X)} + \overline{T(B_X)}$, což jsou kompaktní prostory. Protože součet kompaktů je kompakt (a uzavřený podprostor kompaktu také), $\overline{(S + T)(B_x)}$ je kompaktní. Násobení triviálně.

		$T \in ©F(x, y)$ $\implies$ $\Rang T$ je konečnědimenzionální, tedy uzavřená $\implies$ $\overline{T(B_x)} \subseteq \Rang T \cong ®K^n$.
	\end{dukazin}
\end{veta}

\end{document}
