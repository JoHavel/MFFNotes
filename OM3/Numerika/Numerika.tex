\documentclass[12pt]{article}					% Začátek dokumentu
\usepackage{../../MFFStyle}					    % Import stylu



\begin{document}

% 2. přednáška (z minulého roku)

\section*{Organizační úvod}
\begin{poznamka}[Organizační úvod]
	Přednáška rozdělena na 2 části. Materiály na stránkách / studentském úložišti.
\end{poznamka}

\section*{Úvod}
\begin{poznamka}[O čem vlastně Numerická matematika je]
	Nejprve musíme problém diskretizovat (většinou bývá nekonečně dimenzionální/spojitý a my chceme konečnou dimenzi a diskrétní problém). Inverze k diskretizaci je relevance.
\end{poznamka}

\section{Chování algoritmů v počítačové aritmetice}
\begin{definice}[Double precision]
	Obsahuje 64 bitů – znaménko (1b) + exponent (11b) + mantisa (52b). Reprezentuje číslo v normalizovaném tvaru ($±1,\text{mantisa}·2^{e = \text{exponent}}$), je-li $-1022 = e_{min} ≤ e ≤ e_{max} = 1023$.

	Speciální hodnoty jsou $e = e_min - 1$, kde pokud je mantisa $0$, tak reprezentujeme $0$, jinak máme $±0,\text{mantisa}·2^e$. Pro $e = e_{max} + 1$ je při mantise nula hodnota $±∞$, jinak tzv. NaN (not-a-number).

	Toto „těleso“ označujeme ®F.

	\begin{poznamkain}
		V intervalu mezi dvěma exponenty jsou čísla rozloženy rovnoměrně.
	\end{poznamkain}

	\begin{poznamkain}
		Samozřejmě existují i jiné přesnosti, standardně binary32. binary128 je málo podporované, binary16 je spíše pro grafické karty.
	\end{poznamkain}
\end{definice}	

\begin{definice}[Operace s ®F]
	První, co potřebujeme je schopnost zaokrouhlit. V standardu je zaokrouhlení na nejbližší číslo (tj. $round(x) = x(1 + \delta)$, kde $|\delta| ≤ u$ je minimální a záleží na reprezentaci).

	Standardní model aritmetiky s konečnou přesností předpokládá, že výsledek každé operace je rozmazán nějakým $(1 + \delta)$, kde $|\delta| ≤ u$.
\end{definice}

\begin{definice}[Škálování, kancelace (vyrušení, ztráta informace)]
	Může se nám stát, že přičítáme tak malá čísla, že se výsledek neliší od původní hodnoty. Tomu se předchází škálováním dat.

	Také se může stát, že odčítáme dvě téměř stejná čísla ($|a - b| \ll |a| + |b|$), tak se odečtou úplně, protože drobná změna se ztratila v jejich reprezentaci. Takové výrazy je třeba vyjádřit jinak, aby se předešlo odčítání blízkých čísel.
\end{definice}

	\subsection{Analýza chyb}
	\begin{poznamka}
		Na chyby při výpočtech se můžeme dívat 2 způsoby: jako na chybu výsledku, nebo jako na chybu zadání.
	\end{poznamka}

	\begin{lemma}
		Nechť $|\alpha_i| ≤ u$, $i \in [n]$ a nechť $n·u < 0.01$. Potom platí
		$$ \prod_{i=1}^n (1 + \alpha_i) = 1 + \beta_n, $$
		kde $|\beta_n| ≤ 1.01·n·u$.

		\begin{dukazin}
			Vyjádříme si $\beta_n$, omezíme $|\beta_n| ≤ (1 + u)^n - 1$. Jelikož $1 + x ≤ e^x$ pro $x ≥ 0$, pak tuto hodnotu dále omezíme $e^{n·u} - 1 < n·u(1 + \frac{n·u}{2} + \(\frac{n·u}{2}\)^2 + …) = \frac{n·u}{1 - n·u / 2} < 1.01·n·u$.
		\end{dukazin}
	\end{lemma}

% 3. přednáška

	\begin{definice}[Přímá analýza chyb]
		Popis šíření zaokrouhlovacích chyb v algoritmu, odhad přímé chyby je
		$$ ||a(x) - fl(a(x))||, $$
		kde $fl$ je strojové zaokrouhlování. Nezávislý (tzv. efektivní) odhad možný zřídka.
	\end{definice}

	\begin{definice}[Zpětná analýza chyb]
		Snaha interpretovat zaokrouhlovací chyby pomocí změn vstupních dat, tj. odhad zpětné chyby, která je
		$$ ||x - \hat{x}||, $$
		kde $\hat{x}$ je vstupní dato, se kterým by se dospělo stejného výsledku, ale přesnou cestou.
	\end{definice}

	\subsection{Stabilita algoritmu}
	\begin{definice}[$O(u)$]
		$O(u)$ je neurčité číslo,
		$$ |O(u)| ≤ K·u, $$
		kde $K$ nezávisí na datech (jedině na dimenzi, na které může závisel libovolně).
	\end{definice}

	\begin{definice}[Zpětně stabilní algoritmus]
		Algoritmus $a(x)$ je zpětně stabilní, pokud
		$$ \forall x\ \exists \hat{x}: fl(a(x)) = a(\hat{x}) \land \frac{||x - \hat{x}||}{||x||} = O(u). $$

		(Tj. jestliže se chyby výpočtu způsobené zaokrouhlováním promítnou do dostatečně malých změn vstupních dat.)
	\end{definice}

	\begin{definice}[Numerická stabilita algoritmu]
		Algoritmus $a(x)$ je numericky stabilní, pokud
		$$\forall x\ \exists \hat{x}: \frac{||x - \hat{x}||}{||x||} = O(u) \land \frac{||a(\hat{x} - fl(a(x)))}{a(\hat{x})} = O(u). $$
	\end{definice}

	\subsection{Podmíněnost problému}
	\begin{poznamka}
		Mějme matematický problém $f: X \rightarrow Y$, kde $X$, $Y$ jsou normované lineární prostory, $f$ je spojité zobrazení, mající vlastnosti, že zkoumáme existenci, jednoznačnost řešení a že problém je citlivý na změny dat. Co to ale je?
	\end{poznamka}

	\begin{definice}[Číslo podmíněnosti]
		Označme $\Delta x$ malou změnu dat (perturbace), $\Delta f(x) = f(x + \Delta x) - f(x)$ změnu řešení.

		Číslo podmíněnosti problému $f$ v $x$ ke
		$$ \kappa_f(x) ≡ \lim_{\delta \rightarrow 0} \sup_{||\Delta x|| ≤ \delta} \(\frac{||\Delta f(x)||}{||f(x)||} / \frac{||\Delta x||}{||x||}\). $$

		Špatně či dobře podmíněný problém je, že malé změny $x$ vedou na velké či malé změny v $f(x)$.
	\end{definice}

	\begin{definice}[Generovaná maticová norma]
		Mějme vektorovou normu $||·||$ na $®C^n$ a $®C^m$. Generovanou maticovou normou nazveme funkcionál $g: ®C^{n \times m} \rightarrow ®R$,
		$$ g(A) ≡ \max_{x ≠ ¦o} \frac{||Ax||}{||x||} = \max_{||x|| = 1} ||Ax||. $$
		$g(A)$ je norma, značíme ji $||A||$. Je navíc „skoro multiplikativní“: $||AB|| ≤ ||A||·||B||$. A z definice $||Ax|| ≤ ||A||·||x||$. Je-li $A$ čtvercová, pak $||I|| = 1$.
	\end{definice}

	\begin{definice}[Frobeniova norma]
		$||A||_F = \sqrt{\sum_{i, j} a_{ij}^2}$.
	\end{definice}

	\begin{poznamka}[Podmíněnost $Ax$]
		Pokud je problém $f: x \mapsto Ax$, $\kappa_f(x) = \frac{||x||}{||Ax||}||A||$.

		Pokud $A$ je regulární, pak z $||x|| = ||A^{-1}Ax|| ≤ ||A^{-1}||·||Ax||$ plyne $\kappa_f(x) ≤ ||A||·||A^{-1}||$.
	\end{poznamka}

	\begin{definice}
		Číslo z předchozí poznámky nazveme číslo podmíněnosti matice $A$ a značíme ho $\kappa(A) ≡ ||A||·||A^{-1}||$.

		\begin{poznamkain}
			Zřejmě $1 = ||A^{-1}A|| ≤ \kappa(A)$.
		\end{poznamkain}
	\end{definice}

	\begin{definice}[Přesnost výpočtu]
		Nechť $a(x)$ je algoritmus. Pak (relativní) přesnost výpočtu je
		$$ \frac{||a(x) - fl(a(x))||}{||a(x)||}. $$
	\end{definice}

	\begin{poznamka}
		Je-li algoritmus $a$ řešící problém $f$ v ®F zpětně stabilní, potom $fl(a(x)) = a(\hat{x})$ pro nějaké $\hat{x}$ splňující
		$$ \frac{||x - \hat{x}||}{||x||} = O(u) ≡ \delta. $$

		Algoritmus řeší daný problém znamená $f(x) = a(x)$, proto
		$$ \frac{||a(x) - a(\hat{x})||}{||a(x)||} = \frac{||f(x) - f(\hat{x})||}{||f(x)||}. $$

		Potom
		$$ \frac{\frac{||f(x) - f(\hat{x})||}{||f(x)||}}{\frac{||x - \hat{x}||}{||x||}} ≤ \sup_{\frac{||x - \tilde x||}{||x||} ≤ \delta} \frac{\frac{||f(x) - f(\tilde{x})||}{||f(x)||}}{\frac{||x - \tilde{x}||}{||x||}} \approx \kappa_f(x) $$
		pro „rozumně“ spojitý problém $f$. Tj.
		$$ \frac{||a(x) - fl(a(x))||}{||a(x)||} \lesssim \kappa_f(x) O(u). $$

		Tedy přesnost zpětně stabilního algoritmu je (může být) ovlivněna podmíněností problému a strojovou přesností.
	\end{poznamka}
\end{document}
