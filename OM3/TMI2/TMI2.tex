\documentclass[12pt]{article}					% Začátek dokumentu
\usepackage{../../MFFStyle}					    % Import stylu



\begin{document}

% 04. 10. 2021

\section*{Organizační úvod}
	\begin{poznamka}
		Zkouška bude snad ústní.
	\end{poznamka}

\section*{Úvod}
\begin{veta}[Lebesgueova míra]
	Existuje právě jedna borelovská míra $\lambda^n$ v $®R^n$ taková, že
	$$ \lambda^n\left(\bigtimes_{i=1}^n[a_i, b_i]\right) = \prod_{i=1}^n \left(b_i - a_i\right), -∞ < a_i ≤ b_i < ∞, 1 ≤ i ≤ n. $$

	\begin{poznamkain}
		Zúplněnou $\sigma$-algebru značíme $B_0^n$ a platí $B^n \subsetneq B_0^n$ (pro $n ≥ 2$ jednoduché, pro $n = 1$ možná někdy příště).

		$\lambda^n$ je translačně a rotačně invariantní (posunutím a otočením se nezmění).

		$\lambda^n$ je $\sigma$-konečná.

		$\lambda^n$ je regulární (můžeme její hodnotu na množině aproximovat jejími hodnotami na otevřené nadmnožině a uzavřené podmnožině\footnote{$$ \forall E \in B_0^n\ \forall \epsilon > 0\ \exists F \subset E \subset G, F\text{ uzavřená}, G\text{ otevřená}, \lambda^n\left(G \setminus F \right) < \epsilon.  $$}).
	\end{poznamkain}
\end{veta}

\begin{definice}
	$\tilde{\mu}: ©A \rightarrow [0, ∞]$ je pramíra (premeasure) na algebře ©A podmnožin $X$, jestliže:
	$$ \tilde{\mu}\left(\O\right) = 0, $$
	$$ A_i \in ©A, \bigcup_i A_i \in ©A, A_i \text{ po dvou disjunktní} \implies \tilde{mu}\left(\bigcup_iA_i\right) = \sum_i \tilde{\mu}\left(A_i\right). $$
\end{definice}

\begin{veta}[Hahn-Kolmogorov]
	Buď $\tilde{\mu}$ pramíra na algebře ©A. Pak existuje míra $\mu$ na $\sigma ©A$ taková, že $\mu = \tilde{\mu}$ na ©A. Je-li $\tilde{\mu}$ $\sigma$-konečná, je $\mu$ určená jednoznačně.
\end{veta}

\section{Konstrukce Lebesgueovy míry z vnější míry}
\begin{definice}[Vnější míra (outer measure)]
	Nechť $X ≠ \O$. Funkce $\mu^*: ©P\left(X\right) \rightarrow [0, ∞]$ je vnější míra na $X$, jestliže:
	$$ \mu^*\left(\O\right) = 0, $$
	$$ A \subset B \implies \mu^*(A) ≤ \mu^*(B),\ \text{(monotonie)}  $$
	$$ A_i \subset X \left(i \in ®N\right) \implies \mu^*\left(\bigcup_iA_i\right) ≤ \sum_i\mu^*(A_i).\ \text{(spočetná subadivita)}  $$

	\begin{prikladyin}
		$$ \mu^* ≡ 0, $$
		$$ \mu^* = \delta_x, x \in X, $$
		$$ \mu^*(A) = \card A, $$
		$$ \mu*(A) := 0, A=\O, \mu*(A) := 1, A≠\O, $$
		$$ X = ®R, \lambda^*(A) := \inf\{\sum_i |I_i|, A \subset \bigcup_i I_i, I_i \text{ otevřené intervaly}\} $$
	\end{prikladyin}
\end{definice}

\begin{definice}[Měřitelnost vůči vnější míře]
	Řekneme, že množina $A \subset X$ je $\mu^*$-měřitelná, jestliže
	$$ \forall T \subset X: \mu^*(T) = \mu*(T\cap A) + \mu^*(T\setminus A). (*)  $$

	Značíme $©A_{\mu^*} := \{A\subset X | A\text{ je $\mu^*$-měřitelná}\}$.

	\begin{poznamkain}
		Ať $\mu^*$ je vnější míra na $X$, $Y \subset X$. Pak restrikce $\mu^*|_Y: A \mapsto \mu^*(A \cap Y)$ je vnější míra a platí:
		$$ ©A_{\mu^*} \subset ©A_{\mu^*}|_Y $$
	
		\begin{dukazin}
			$$ A \in ©A_{\mu^*}: \mu^*|_Y (T) = \mu^*(T\cap Y) = \mu^*(T\cap Y \cap A) + \mu^*((T \cap Y) \setminus A) = $$
			$$ = \mu^*|_Y (T \cap A) + \mu^*|_Y(T \setminus A). $$
		\end{dukazin}
	\end{poznamkain}
\end{definice}

\begin{veta}[Caratheodory]
	$©A_{\mu^*}$ je $\sigma$-algebra na $X$ a $\mu := \mu^*|_{©A_{\mu^*}}$ je míra. Prostor $(X, ©A_{\mu^*}, \mu)$ je úplný.

	\begin{dukazin}
		$\O \in ©A_\mu^*$ je zřejmé. Uzavřenost na komplement je také snadná, z definice $©A_{\mu^*}$. Místo sjednocení ukážeme uzavřenost na konečný průnik: Víme $T \subset X: \mu^*(T) = \mu^*(T\cap A) + \mu^*(T \setminus A)$, $\mu^*(T \cap A) = \mu^*(T \cup A \cup B) + \mu^*((T \cap A) \setminus B)$ a $\mu^*(T\setminus (A \cap B)) = \mu^*((T \setminus (A \cap B)) \cap A) + \mu^*((T \setminus (A \cap B)) \setminus A) = \mu^*((T \cap A) \setminus B) + \mu^*(T \setminus A)$.

		Tedy $\mu^*(T) = \mu^*(T \cap A \cap B) + \mu^*((T \cap A) \setminus B) + \mu^*(T \setminus A) = \mu^*(T \cap A \cap B) + \mu^*(T \setminus (A \cap B))$. Tudíž $©A_{\mu^*}$ je algebra.

		Nyní chceme ukázat, že $\mu^*$ je $\sigma$-aditivní na $©A_{\mu^*}$: Buďte $A_i \in ©A_{\mu^*}$ po dvou disjunktní. Volbou $T = A_1 \cup A_2$ dostaneme $\mu^*(A_1 \cup A_2) = \mu^*(A_1) + \mu^*(A_2)$ $\implies$ $\mu^*$ je konečně aditivní na $©A_{\mu^*}$.
		$$ \forall n \in ®N: \sum_{i=1}^∞ \mu^*(A_i) = \lim_{n \rightarrow ∞} \sum_{i=1}^n \mu^*(A_i) = \lim_{n \rightarrow ∞} \mu^*\left(\bigcup_{i=1}^n A_i\right) ≤ \mu^*\left(\bigcup_{i=1}^∞ A_i\right). $$
		Opačná nerovnost plyne ze spočetné subaditivity. To znamená, že $\mu^*\left(\bigcup_iA_i\right) = \sum_i\mu^*\left(A_i\right), A_i \in ©A_{\mu^*}$, po dvou disjunktní.

		$©A_{\mu^*}$ je uzavřená na disjunktní spočetné sjednocení: $A_i \in ©A_{\mu^*}$, po dvou disjunktní, $T \subset X$.
		$$ \mu^*(T) = \mu^*\left(T \setminus \bigcup_{i=1}^n A_i\right) + \mu^*\left(T \cap \bigcup_{i=1}^n A_i\right) ≥ \mu^*\left(T\setminus \bigcup_{i=1}^∞ A_i\right) + \left(\mu^*|_T\right)\left(\bigcup_{i=1}^∞ A_i\right) = TODO $$

		Limitním přechodem $n \rightarrow ∞$ dostaneme
		$$ \mu^*(T) ≥ \mu^*(T \setminus \bigcup_{i=1}^∞ A_i) + \sum_{i=1}^∞\left(\mu^*|_T\right)\left(A_i\right) = \mu^*\left(T \setminus \bigcup_{i=1}^∞\right) \left(\mu^*|_T\right)\left(\bigcup_{i=1}^∞ A_i\right) \implies \bigcup_{i=1}^∞ A_i \in ©A_{\mu^*}. $$

		Z tohoto všeho plyne, že $\mu = \mu^*|_{A_{\mu^*}}$ je míra na $\sigma$-algebře $©A_{\mu^*}$. Zbývá už jen úplnost:
		$$ \mu^*(A) = 0, T \subset X \implies \mu^*(T) ≥ \mu^*(T\setminus A) = \underbrace{\mu^*(T \cap A)}_{=0} + \mu^*(T \setminus A) \implies A \in ©A_{\mu^*}. $$
	\end{dukazin}
\end{veta}

% 19. 10. 2021

TODO!!!

% 26. 10. 2021

\begin{veta}[Regularita Lebesgueovy míry]
	Nechť $E \subset ®R^n$. Je ekvivalentní:
	
	\begin{enumerate}
		\item $E \in ©A_{\lambda^{n*}}$,
		\item $\forall \epsilon > 0 \ \exists F \subset E \subset G$, $F$ uzavřená, $G$ otevřená, $\lambda^n(G \setminus F) < \epsilon$,
		\item $\exists A \subset E \subset B$, $A, B \in ©B^n$, $\lambda^n(B \setminus A) = 0$,
		\item $E \in ©B_0^n$.
	\end{enumerate}

	\begin{dukazin}
		$1 \implies 2$: Mějme $E \in ©A_{\lambda^{n*}}$, $\epsilon > 0$. Nechť nejprve $\lambda^{n*}(E) < ∞$. Pak $\exists I_i \in O_n$, $E \subset \bigcup_i I_i$, $\sum_i v(I_i) < \lambda^{n*}(E) + \frac{\epsilon}{2}$. Položme $G := \bigcup_i I_i$ (otevřená), $E \subset G$, $\lambda^n(G \setminus E) < \frac{\epsilon}{2}$. Je-li $\lambda^{n*}(E) = ∞$, pak ze $\sigma$-konečnosti je $E = \bigcup_m E_m$, $E_m := E \cap [-m, m]^n$. $\lambda^{n*}(E_m) < ∞ \implies \exists G_m$ otevřená, $E_m \subset G_m$, $\lambda^n(G_m \setminus E_m) < \frac{\epsilon}{2^{m+1}}$. $G := \bigcup_m G_m$ otevřené, $E \subset G$, $\lambda^n(G \setminus E) ≤ \sum_m \lambda^n(G_m \setminus E_m) < \frac{\epsilon}{2}$.

		$E^c \in ©A_{\lambda^{m*}} \implies \exists H$ otevřená, $E^c \subset H$, $\lambda^n(H \setminus E^c) < \frac{\epsilon}{2}$. $F:= H^c$ uzavřená, $F \subset E$, $\lambda^n(E \setminus F) = \lambda^n(E \setminus H^c) = \lambda^n(H \setminus E^c) < \frac{\epsilon}{2}$. TODO

		$2 \implies 3$: Nechť $E \subset ®R^n$ splňuje 2.
		$$ \forall j \in ®N \ \exists F_j \subset E \subset G_j, F_j \text{ uzavřená}, G_j \text{ uzavřená}, \lambda^n(G_j \setminus F_j) < \frac{1}{j}. $$
		Položme $A := \bigcup_j F_j$, $B := \bigcap_j G_j$, $A, B \in ©B^n$, $A \subset E \subset B$. $\lambda^n(B \setminus A)  ≤ \lambda^n(G_j \setminus F_j) < \frac{1}{j}$ pro libovolné $j \in ®N$, tedy $\lambda^n(B \setminus A) = 0$.

		$3 \implies 4$: Jsou-li $A \subset E \subset B$ jako v 3, pak $B \setminus A$ je $\lambda^n$-nulová množina, a tedy $E \in ©B_0^n$.

		$4 \implies 1$: $©A_{\lambda^{n*}}$ obsahuje $©B^n$ a nulové množiny, tedy obsahuje $©B_0^n$.
	\end{dukazin}
\end{veta}

\begin{veta}[Luzinova (běžně bývá obecnější)]
	Buď $f: ®R^n \rightarrow ®R$ lebesgueovsky měřitelná. Buď $\epsilon > 0$. Pak existuje $G \subset ®R^n$ otevřená taková, že $\lambda^n(G) < \epsilon$ a restrikce $f|_{G^c}$ je spojitá.

	\begin{dukazin}
		Buď $U_1, U_2, …$ posloupnost všech otevřených intervalů s racionálními koncovými body. $f$ je lebesgueovsky měřitelná, tedy $\forall j, f^{-1}(U_1) \in ©B_0^n$. Podle regularity pak $\exists F_j \subset f^{-1}(U_j) \subset G_j$, $F_j$ uzavřená, $G_j$ otevřená, $\lambda^n(G_j \setminus F_j) < \frac{\epsilon}{2^j}$. Položme $G := \bigcup_j (G_j \setminus F_j)$. Zřejmě $G$ je otevřená, $\lambda^n(G) ≤ \sum_j \lambda^n (G_j \setminus F_j) < \epsilon$.

		Pro restrikci $g := F|_{G^c}$ platí:
		$$ g^{-1}(U_j) = \{x \in G^c: f(x) \in U_j\} = f^{-1}(U_j) \cap G^c = G_j \cap G^c, j \in ®N. $$
		Zřejmě $U \subset ®R$ otevřená $\implies$ $U = \cup_{U_j \subset U} U_j$, tedy $g^{-1}(U) = \bigcup_{U_j \in U}g^{-1}(U_j)$ otevřená množina v $G^c$, tedy $g$ je spojitá na $G^c$.
	\end{dukazin}

	\begin{poznamkain}
		Obecně nelze požadovat $\lambda^n(G) = 0$. Např. charakteristická funkce diskontinua kladné míry (podobně jako Cantorovo diskontinuum, ale nenulové míry), které dostaneme tak, že z prostředků intervalů v $i$-tém kroku vždy odebereme intervaly délky $a_i$ tak, aby $a_1 + 2a_2 + 4a_3 + … < 1$. ($G$ z minulé věty pak bude sjednocení malých okolíček krajních bodů odebíraných intervalů.)
	\end{poznamkain}
\end{veta}

\section{Regularita borelovských měr}
\begin{definice}[Regulární borelovská míra]
	Borelovská míra $\mu$ na topologickém (metrickém) prostoru $X$ je regulární, jestliže $\forall B \in ©B(X): \mu(B) = \inf\{\mu(G) | B \subset G, G \text{ otevřená}\}$.
\end{definice}

\begin{poznamka}
	1) Často se hovoří o vnější regularitě (outer regular measure). 2) Pro konečné míry: $\mu$ je regulární $\implies \forall B \in ©B: \mu(B) = \sup\{\mu(F) | F \subset B, F \text{ uzavřená}\}$.
\end{poznamka}

\begin{veta}
	Každá konečná borelovská míra na metrickém prostoru je regulární.

	\begin{dukazin}
		$(X, \rho)$ metrický prostor, $\mu$ borelovská míra na $X$, $\mu(X) < ∞$. Označme
		$$ ©D := \{B \in ©B(X) | \epsilon > 0 \ \exists F \subset B \subset G, F \text{ uzavřená}, G \text{ otevřená}, \mu(G \setminus F) < \epsilon\}. $$
		Ukážeme $©D := ©B(X)$. $©D$ obsahuje všechny množiny: $F \subset X$ uzavřená, $F_{<\epsilon} := \{x \in X | \rho(x, F) < \epsilon\}$ (otevřená). Zřejmě $F_{<\frac{1}{j}} \searrow F$, $j \rightarrow ∞$ z uzavřenosti $F$. $\mu$ konečna $\implies$ (spojitost míry) $\mu F_{<\frac{1}{j}} \rightarrow \mu(F)$.

		$©D$ je $\sigma$-algebra: $\O \in ©D$, $D \in ©D \implies D^c \in ©D$:
		$$ F \subset D \subset G, \mu(G \setminus F) < \epsilon \implies G^c \subset D^c \subset F^c, \mu(F^c \setminus G^c) < \epsilon. $$

		$D_i \in ©D \implies \bigcup_i D_i \in ©D$:
		$$ \exists F_i \subset D_i \subset G_i, \mu(G_i \setminus F_i) < \frac{\epsilon}{2^i}. $$
		$$ TODO! $$
		$$ \bigcup_{i=1}^N F_i \subset \cup_{i=1}^∞ D_i \subset \bigcup_{i = 1}^∞ G_i, N \in ®N. $$
		$$ \bigcup_{i=1}^∞ G_i \setminus \bigcup_{i=1}^F_i $$
	\end{dukazin}
\end{veta}

\end{document}
