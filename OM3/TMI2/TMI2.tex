\documentclass[12pt]{article}					% Začátek dokumentu
\usepackage{../../MFFStyle}					    % Import stylu



\begin{document}

% 04. 10. 2021

\section*{Organizační úvod}
	\begin{poznamka}
		Zkouška bude snad ústní.
	\end{poznamka}

\section*{Úvod}
\begin{veta}[Lebesgueova míra]
	Existuje právě jedna borelovská míra $\lambda^n$ v $®R^n$ taková, že
	$$ \lambda^n\left(\bigtimes_{i=1}^n[a_i, b_i]\right) = \prod_{i=1}^n \left(b_i - a_i\right), -∞ < a_i ≤ b_i < ∞, 1 ≤ i ≤ n. $$

	\begin{poznamkain}
		Zúplněnou $\sigma$-algebru značíme $B_0^n$ a platí $B^n \subsetneq B_0^n$ (pro $n ≥ 2$ jednoduché, pro $n = 1$ možná někdy příště).

		$\lambda^n$ je translačně a rotačně invariantní (posunutím a otočením se nezmění).

		$\lambda^n$ je $\sigma$-konečná.

		$\lambda^n$ je regulární (můžeme její hodnotu na množině aproximovat jejími hodnotami na otevřené nadmnožině a uzavřené podmnožině\footnote{$$ \forall E \in B_0^n\ \forall \epsilon > 0\ \exists F \subset E \subset G, F\text{ uzavřená}, G\text{ otevřená}, \lambda^n\left(G \setminus F \right) < \epsilon.  $$}).
	\end{poznamkain}
\end{veta}

\begin{definice}
	$\tilde{\mu}: ©A \rightarrow [0, ∞]$ je pramíra (premeasure) na algebře ©A podmnožin $X$, jestliže:
	$$ \tilde{\mu}\left(\O\right) = 0, $$
	$$ A_i \in ©A, \bigcup_i A_i \in ©A, A_i \text{ po dvou disjunktní} \implies \tilde{mu}\left(\bigcup_iA_i\right) = \sum_i \tilde{\mu}\left(A_i\right). $$
\end{definice}

\begin{veta}[Hahn-Kolmogorov]
	Buď $\tilde{\mu}$ pramíra na algebře ©A. Pak existuje míra $\mu$ na $\sigma ©A$ taková, že $\mu = \tilde{\mu}$ na ©A. Je-li $\tilde{\mu}$ $\sigma$-konečná, je $\mu$ určená jednoznačně.
\end{veta}

\section{Konstrukce Lebesgueovy míry z vnější míry}
\begin{definice}[Vnější míra (outer measure)]
	Nechť $X ≠ \O$. Funkce $\mu^*: ©P\left(X\right) \rightarrow [0, ∞]$ je vnější míra na $X$, jestliže:
	$$ \mu^*\left(\O\right) = 0, $$
	$$ A \subet B \implies \mu^*(A) ≤ \mu^*(B),\ \text{(monotonie)}  $$
	$$ A_i \subet X \left(i \in ®N\right) \implies \mu^*\left(\bigcup_iA_i\right) ≤ \sum_i\mu^*(A_i).\ \text{(spočetná subadivita)}  $$

	\begin{prikladyin}
		$$ \mu^* ≡ 0, $$
		$$ \mu^* = \delta_x, x \in X, $$
		$$ \mu^*(A) = \card A, $$
		$$ \mu*(A) := 0, A=\O, \mu*(A) := 1, A≠\O, $$
		$$ X = ®R, \lambda^*(A) := \inf\{\sum_i |I_i|, A \subset \bigcup_i I_i, I_i \text{ otevřené intervaly}\} $$
	\end{prikladyin}
\end{definice}

\begin{definice}[Měřitelnost vůči vnější míře]
	Řekneme, že množina $A \subset X$ je $\mu^*$-měřitelná, jestliže
	$$ \forall T \subset X: \mu^*(T) = \mu*(T\cap A) + \mu^*(T\setminus A). (*)  $$

	Značíme $©A_{\mu^*} := \{A\subset X | A\text{ je $\mu^*$-měřitelná}\}$.

	\begin{poznamkain}
		Ať $\mu^*$ je vnější míra na $X$, $Y \subset X$. Pak restrikce $\mu^*|_Y: A \mapsto \mu^*(A \cap Y)$ je vnější míra a platí:
		$$ ©A_{\mu^*} \subset ©A_{\mu^*}|_Y $$
	
		\begin{dukazin}
			$$ A \in ©A_{\mu^*}: \mu^*|_Y (T) = \mu^*(T\cap Y) = \mu^*(T\cap Y \cap A) + \mu^*((T \cap Y) \setminus A) = $$
			$$ = \mu^*|_Y (T \cap A) + \mu^*|_Y(T \setminus A). $$
		\end{dukazin}
	\end{poznamkain}
\end{definice}

\begin{veta}[Caratheodory]
	$©A_{\mu^*}$ je $\sigma$-algebra na $X$ a $\mu := \mu^*|_{©A_{\mu^*}}$ je míra. Prostor $(X, ©A_{\mu^*}, \mu)$ je úplný.

	\begin{dukazin}
		$\O \in ©A_\mu^*$ je zřejmé. Uzavřenost na komplement je také snadná, z definice $©A_{\mu^*}$. Místo sjednocení ukážeme uzavřenost na konečný průnik: Víme $T \subset X: \mu^*(T) = \mu^*(T\cap A) + \mu^*(T \setminus A)$, $\mu^*(T \cap A) = \mu^*(T \cup A \cup B) + \mu^*((T \cap A) \setminus B)$ a $\mu^*(T\setminus (A \cap B)) = \mu^*((T \setminus (A \cap B)) \cap A) + \mu^*((T \setminus (A \cap B)) \setminus A) = \mu^*((T \cap A) \setminus B) + \mu^*(T \setminus A)$.

		Tedy $\mu^*(T) = \mu^*(T \cap A \cap B) + \mu^*((T \cap A) \setminus B) + \mu^*(T \setminus A) = \mu^*(T \cap A \cap B) + \mu^*(T \setminus (A \cap B))$. Tudíž $©A_{\mu^*}$ je algebra.

		Nyní chceme ukázat, že $\mu^*$ je $\sigma$-aditivní na $©A_{\mu^*}$: Buďte $A_i \in ©A_{\mu^*}$ po dvou disjunktní. Volbou $T = A_1 \cup A_2$ dostaneme $\mu^*(A_1 \cup A_2) = \mu^*(A_1) + \mu^*(A_2)$ $\implies$ $\mu^*$ je konečně aditivní na $©A_{\mu^*}$.
		$$ \forall n \in ®N: \sum_{i=1}^∞ \mu^*(A_i) = \lim_{n \rightarrow ∞} \sum_{i=1}^n \mu^*(A_i) = \lim_{n \rightarrow ∞} \mu^*\left(\bigcup_{i=1}^n A_i\right) ≤ \mu^*\left(\bigcup_{i=1}^∞ A_i\right). $$
		Opačná nerovnost plyne ze spočetné subaditivity. To znamená, že $\mu^*\left(\bigcup_iA_i\right) = \sum_i\mu^*\left(A_i\right), A_i \in ©A_{\mu^*}$, po dvou disjunktní.

		$©A_{\mu^*}$ je uzavřená na disjunktní spočetné sjednocení: $A_i \in ©A_{\mu^*}$, po dvou disjunktní, $T \subset X$.
		$$ \mu^*(T) = \mu^*\left(T \setminus \bigcup_{i=1}^n A_i\right) + \mu^*\left(T \cap \bigcup_{i=1}^n A_i\right) ≥ \mu^*\left(T\setminus \bigcup_{i=1}^∞ A_i\right) + \left(\mu^*|_T\right)\left(\bigcup_{i=1}^∞ A_i\right) = TODO $$

		Limitním přechodem $n \rightarrow ∞$ dostaneme
		$$ \mu^*(T) ≥ \mu^*(T \setminus \bigcup_{i=1}^∞ A_i) + \sum_{i=1}^∞\left(\mu^*|_T\right)\left(A_i\right) = \mu^*\left(T \setminus \bigcup_{i=1}^∞\right) \left(\mu^*|_T\right)\left(\bigcup_{i=1}^∞ A_i\right) \implies \bigcup_{i=1}^∞ A_i \in ©A_{\mu^*}. $$

		Z tohoto všeho plyne, že $\mu = \mu^*|_{A_{\mu^*}}$ je míra na $\sigma$-algebře $©A_{\mu^*}$. Zbývá už jen úplnost:
		$$ \mu^*(A) = 0, T \subset X \implies \mu^*(T) ≥ \mu^*(T\setminus A) = \underbrace{\mu^*(T \cap A)}_{=0} + \mu^*(T \setminus A) \implies A \in ©A_{\mu^*}. $$
	\end{dukazin}
\end{veta}




\end{document}
