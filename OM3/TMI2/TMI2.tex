\documentclass[12pt]{article}					% Začátek dokumentu
\usepackage{../../MFFStyle}					    % Import stylu



\begin{document}

% 05. 10. 2021

\section*{Organizační úvod}
	\begin{poznamka}
		Zkouška bude snad ústní.
	\end{poznamka}

\section*{Úvod}
\begin{veta}[Lebesgueova míra]
	Existuje právě jedna borelovská míra $\lambda^n$ v $®R^n$ taková, že
	$$ \lambda^n\left(\bigtimes_{i=1}^n[a_i, b_i]\right) = \prod_{i=1}^n \left(b_i - a_i\right), -∞ < a_i ≤ b_i < ∞, 1 ≤ i ≤ n. $$

	\begin{poznamkain}
		Zúplnění $B^n$ značíme $B_0^n$ a platí $B^n \subsetneq B_0^n \subsetneq ©P(®R^n)$ (pro $n ≥ 2$ jednoduché, pro $n = 1$ možná někdy příště).

		$\lambda^n$ je translačně a rotačně invariantní (posunutím a otočením se nezmění).

		$\lambda^n$ je $\sigma$-konečná.

		$\lambda^n$ je regulární (můžeme její hodnotu na množině aproximovat jejími hodnotami na otevřené nadmnožině a uzavřené podmnožině\footnote{$$ \forall E \in B_0^n\ \forall \epsilon > 0\ \exists F \subset E \subset G, F\text{ uzavřená}, G\text{ otevřená}, \lambda^n\left(G \setminus F \right) < \epsilon.  $$}).
	\end{poznamkain}
\end{veta}

\begin{definice}[Pramíra]
	$\tilde{\mu}: ©A \rightarrow [0, ∞]$ je pramíra (premeasure) na algebře ©A podmnožin $X$, jestliže:
	$$ \tilde{\mu}\left(\O\right) = 0, $$
	$$ A_i \in ©A, \bigcup_i A_i \in ©A, A_i \text{ po dvou disjunktní} \implies \tilde{mu}\left(\bigcup_iA_i\right) = \sum_i \tilde{\mu}\left(A_i\right). $$
\end{definice}

\begin{veta}[Hahn-Kolmogorov]
	Buď $\tilde{\mu}$ pramíra na algebře ©A. Pak existuje míra $\mu$ na $\sigma ©A$ taková, že $\mu = \tilde{\mu}$ na ©A. Je-li $\tilde{\mu}$ $\sigma$-konečná, je $\mu$ určená jednoznačně.
\end{veta}

\section{Konstrukce Lebesgueovy míry z vnější míry}
\begin{definice}[Vnější míra (outer measure)]
	Nechť $X ≠ \O$. Funkce $\mu^*: ©P\left(X\right) \rightarrow [0, ∞]$ je vnější míra na $X$, jestliže:
	$$ \mu^*\left(\O\right) = 0, $$
	$$ A \subset B \implies \mu^*(A) ≤ \mu^*(B),\ \text{(monotonie)}  $$
	$$ A_i \subset X \left(i \in ®N\right) \implies \mu^*\left(\bigcup_iA_i\right) ≤ \sum_i\mu^*(A_i).\ \text{(spočetná subadivita)}  $$

	\begin{prikladyin}
		$$ \mu^* ≡ 0, $$
		$$ \mu^* = \delta_x, x \in X, $$
		$$ \mu^*(A) = \card A, $$
		$$ \mu*(A) := 0, A=\O, \mu*(A) := 1, A≠\O, $$
		$$ X = ®R, \lambda^*(A) := \inf\{\sum_i |I_i|, A \subset \bigcup_i I_i, I_i \text{ otevřené intervaly}\} $$
	\end{prikladyin}
\end{definice}

\begin{definice}[Měřitelnost vůči vnější míře]
	Řekneme, že množina $A \subset X$ je $\mu^*$-měřitelná, jestliže
	$$ \forall T \subset X: \mu^*(T) = \mu*(T\cap A) + \mu^*(T\setminus A). (*)  $$

	Značíme $©A_{\mu^*} := \{A\subset X | A\text{ je $\mu^*$-měřitelná}\}$.

	\begin{poznamkain}
		Ať $\mu^*$ je vnější míra na $X$, $Y \subset X$. Pak restrikce $\mu^*|_Y: A \mapsto \mu^*(A \cap Y)$ je vnější míra a platí:
		$$ ©A_{\mu^*} \subset ©A_{\mu^*}|_Y $$
	
		\begin{dukazin}
			$$ A \in ©A_{\mu^*}: \mu^*|_Y (T) = \mu^*(T\cap Y) = \mu^*(T\cap Y \cap A) + \mu^*((T \cap Y) \setminus A) = $$
			$$ = \mu^*|_Y (T \cap A) + \mu^*|_Y(T \setminus A). $$
		\end{dukazin}
	\end{poznamkain}
\end{definice}

\begin{veta}[Caratheodory]
	$©A_{\mu^*}$ je $\sigma$-algebra na $X$ a $\mu := \mu^*|_{©A_{\mu^*}}$ je míra. Prostor $(X, ©A_{\mu^*}, \mu)$ je úplný.

	\begin{dukazin}
		$\O \in ©A_\mu^*$ je zřejmé. Uzavřenost na komplement je také snadná, z definice $©A_{\mu^*}$. Místo sjednocení ukážeme uzavřenost na konečný průnik: Víme $T \subset X: \mu^*(T) = \mu^*(T\cap A) + \mu^*(T \setminus A)$, $\mu^*(T \cap A) = \mu^*(T \cup A \cup B) + \mu^*((T \cap A) \setminus B)$ a $\mu^*(T\setminus (A \cap B)) = \mu^*((T \setminus (A \cap B)) \cap A) + \mu^*((T \setminus (A \cap B)) \setminus A) = \mu^*((T \cap A) \setminus B) + \mu^*(T \setminus A)$.

		Tedy $\mu^*(T) = \mu^*(T \cap A \cap B) + \mu^*((T \cap A) \setminus B) + \mu^*(T \setminus A) = \mu^*(T \cap A \cap B) + \mu^*(T \setminus (A \cap B))$. Tudíž $©A_{\mu^*}$ je algebra.

		Nyní chceme ukázat, že $\mu^*$ je $\sigma$-aditivní na $©A_{\mu^*}$: Buďte $A_i \in ©A_{\mu^*}$ po dvou disjunktní. Volbou $T = A_1 \cup A_2$ dostaneme $\mu^*(A_1 \cup A_2) = \mu^*(A_1) + \mu^*(A_2)$ $\implies$ $\mu^*$ je konečně aditivní na $©A_{\mu^*}$.
		$$ \forall n \in ®N: \sum_{i=1}^∞ \mu^*(A_i) = \lim_{n \rightarrow ∞} \sum_{i=1}^n \mu^*(A_i) = \lim_{n \rightarrow ∞} \mu^*\left(\bigcup_{i=1}^n A_i\right) ≤ \mu^*\left(\bigcup_{i=1}^∞ A_i\right). $$
		Opačná nerovnost plyne ze spočetné subaditivity. To znamená, že
		$$ \mu^*\left(\bigcup_iA_i\right) = \sum_i\mu^*\left(A_i\right), A_i \in ©A_{\mu^*}, $$
		po dvou disjunktní.

		$©A_{\mu^*}$ je uzavřená na disjunktní spočetné sjednocení: $A_i \in ©A_{\mu^*}$, po dvou disjunktní, $T \subset X$.
		$$ \mu^*(T) = \mu^*\left(T \setminus \bigcup_{i=1}^n A_i\right) + \mu^*\left(T \cap \bigcup_{i=1}^n A_i\right) ≥ \mu^*\left(T\setminus \bigcup_{i=1}^∞ A_i\right) + \left(\mu^*|_T\right)\left(\bigcup_{i=1}^∞ A_i\right) = $$
		$$ = \mu^*\(T \setminus \bigcup_{i=1}^∞ A_i\) + \sum_{i=1}^n(\mu^*|T)(A_i). $$

		Limitním přechodem $n \rightarrow ∞$ dostaneme
		$$ \mu^*(T) ≥ \mu^*(T \setminus \bigcup_{i=1}^∞ A_i) + \sum_{i=1}^∞\left(\mu^*|_T\right)\left(A_i\right) = \mu^*\left(T \setminus \bigcup_{i=1}^∞\right) \left(\mu^*|_T\right)\left(\bigcup_{i=1}^∞ A_i\right) \implies \bigcup_{i=1}^∞ A_i \in ©A_{\mu^*}. $$

		Z tohoto všeho plyne, že $\mu = \mu^*|_{A_{\mu^*}}$ je míra na $\sigma$-algebře $©A_{\mu^*}$. Zbývá už jen úplnost:
		$$ \mu^*(A) = 0, T \subset X \implies \mu^*(T) ≥ \mu^*(T\setminus A) = \underbrace{\mu^*(T \cap A)}_{=0} + \mu^*(T \setminus A) \implies A \in ©A_{\mu^*}. $$
	\end{dukazin}
\end{veta}

% 12. 10. 2021 (z materiálů)

\begin{definice}[Metrická vnější míra]
	Buď $(X, \rho)$ metrický prostor. Řekneme, že vnější míra $\mu^*$ na $X$ je metrická, jestliže pro dvě množiny $A, B \subset X$ splňující $\dist(A, B) > 0$ platí
	$$ \mu^*(A \cup B) = \mu^*(A) + \mu^*(B). $$
\end{definice}

\begin{veta}
	Nechť $\mu^*$ je metrická vnější míra na metrickém prostoru $(X, \rho)$. Pak $B(X) \subset ©A_{\mu^*}$.

	\begin{dukazin}
		Buď $F \subset X$ uzavřená. Ukážeme, že $F \in ©A_{\mu^*}$. Označme
		$$ F_{\epsilon} := \{x \in X | \rho(x, F) ≤ \epsilon\}, \qquad \epsilon > 0. $$
		Nechť je dána $T \subset X$. Ověříme, že
		$$ \mu^*(T) ≥ \mu^*(T \cap F) + \mu^*(T \setminus F). $$
		BÚNO $\mu^*(T) < ∞$. Protože $\dist(T \cap F, T \setminus F_\epsilon) ≥ \epsilon > 0$, tak
		$$ \mu^*(T) ≥ \mu^*(T \cap F) + \mu^*(T \setminus F_\epsilon,) $$
		protože $\mu^* $ je metrická. Nyní stačí $\mu^*(T \setminus F_{\frac{1}{j}}) \overset{j \rightarrow ∞}{\rightarrow} \mu^*(T \setminus F)$. Označme $D_i := (F_{\frac{1}{i} \setminus F_{\frac{1}{(i+1)}}}) \cap T, i \in ®N$. Platí $T \setminus F = (T \setminus F_{\frac{1}{j}}) \cup \bigcup_{i=j}^∞ D_i$, a tedy ze spočetné subadivity $\mu^*$ plyne
		$$ \mu^*(T \setminus F) ≤ \mu^*(T \setminus F_{\frac{1}{j}}) + \sum_{i=j}^∞ \mu^*(D_i). $$
		Je-li $|i - j| > 2$ je $\dist(D_i, D_j) > 0$ a tedy
		$$ \sum_{i=1}^n \mu^*(D_{2i}) = \mu^* \(\bigcup_{i=1}^n D_{2i}\) ≤ \mu^*(T) < ∞, $$
		$$ \sum_{i=1}^n \mu^*(D_{2i - 1}) = \mu^* \(\bigcup_{i=1}^n D_{2i - 1}\) ≤ \mu^*(T) < ∞. $$
		Z toho už plyne, že $\sum_{i=j}^∞ \mu{D_{i}} \overset{j \rightarrow ∞}{\rightarrow} 0$.
	\end{dukazin}
\end{veta}

\begin{definice}[Kvádry v $®R^n$, objem kvádru]
	Symbolem $©O_n$ budeme značit množinu všech otevřených omezených kvádrů v $®R^n$ (včetně prázdné množiny). Objemem kvádru $I = (a_1, b_1) \times … \times (a_n, b_n) \in ©O_n$ budeme myslet
	$$ v(I) := (b_1 - a_1)·…·(b_n - a_n). $$
\end{definice}

\begin{tvrzeni}
	Buďte $I, I_1, …, I_k \in ©O_n$

	\begin{enumerate}
		\item Je-li $I \subset \overline{I_1} \cup … \cup \overline{I_k}$, platí $v(I) ≤ v(I_1) + … + v(I_k)$.
		\item Je-li $\overline{I} = \overline{I_1} \cup … \cup \overline{I_k}$ a jsou-li kvádry $I_{[k]}$ po dvou disjunktní, platí $v(I) = v(I_1) + … + v(I_k)$.
	\end{enumerate}

	\begin{dukazin}
		1. krok: Nechť $I = (a_1, b_1) \times … \times (a_n, b_n)$, $©D_i$ je dělení intervalu $(a_i, b_i)$, $i \in [n]$ a označme symbolem ©J systém všech otevřených kvádrů $J_1 \times … \times J_n$, kde $J_i$ je otevřený interval z dělení $©D_i$. Pak zřejmě
		$$ \overline{I} = \bigcup_{J \in ©J}, \qquad v(I) = \sum_{J \in ©J}v(J). $$

		2. krok: Jsou-li $I_1, …, I_k$ jako v druhém bodě, převedeme situaci snadno na případ uvažovaný v 1. kroku. Tím je dokázán druhý bod.

		3. krok: První bod plyne z druhého, jelikož z libovolného pokrytí kvádru $I$ kvádry $I_1, …, I_k$ snadno vyrobíme disjunktní pokrytí.
	\end{dukazin}
\end{tvrzeni}

\begin{definice}
	Pro množinu $E \subset ®R^n$ klademe
	$$ \lambda^{n*}(E) := \inf\{\sum_{i=1}^∞ v(I_i) | E \subset \bigcup_{i=1}^∞ I_i \land I_i \in ©O_n\}. $$
	Pro $\delta > 0$ definujeme
	$$ \lambda_\delta^{n*}(E) = \inf\{\sum_{i=1}^∞ v(I_i) | E \subset \bigcup_{i=1}^∞ I_i \land I_i \in ©O_n \land \diam(I_i) < \delta\}. $$
\end{definice}

\begin{tvrzeni}
	Pro $E \subset ®R^n$ a $\delta > 0$ platí $\lambda^*(E) = \lambda_\delta^{n*}(E)$.

	\begin{dukazin}
		Nerovnost $\lambda^{n*}(E) ≤ \lambda_\delta^{n*}$ je z definice. „$≥$“ BÚNO $\lambda^{n*}(E) < ∞$. Zvolme $\epsilon > 0$. Z definice $\lambda^{n*}(E)$ existují $I_1, … \in ©O_n$ takové, že $E \subset \bigcup_i I_i$ a
		$$ \sum_i v(I_i) < \lambda^{n*}(E) + \epsilon. $$
		Každý z kvádrů $I_i$ můžeme rozdělit na konečný počet disjunktních kvádrů $J_s^{[k(i)]}$ s diametry menšími než $\delta$, přitom $\overline{I_i} = \bigcup \overline{J_i^{[k(i)]}}$. Podle předchozího tvrzení platí $v(I_i) = \sum v(J_i^{[k(i)]})$. Zřejmě existují $I_i^j \in ©O_n$ takové že $\overline{J_i^j} \subset I_i^j$, $\diam I_i^j < \delta$ a $v(I_i^j) < v(J_i^j) + \frac{\epsilon}{k(i)2^i}$. Pak $E \subset \bigcup_{i=1}^∞ \bigcup_{j \in [k(i)]} I_i^j$, a tedy
		$$ \lambda_\delta^{n*}(E) ≤ \sum_{i=1}^∞ \sum_{j \in [k(i)]} v(I_i) + \epsilon < \lambda^{n*}(E) + 2\epsilon. $$
		Nyní už limitním přechodem $\epsilon \rightarrow 0$ dostaneme $\lambda_\delta^{n*}(E) ≤ \lambda^{n*}(E)$.
	\end{dukazin}
\end{tvrzeni}

\begin{veta}
	$\lambda^{n*}$ je metrická vnější míra na $®R^n$ a $\forall I \in ©O_n$ platí $\lambda^{n*}(I) = v(I)$.

	\begin{dukazin}
		Množinová funkce $\lambda^{n*}$ je zřejmě monotónní a platí $\lambda^{n*}(\O) = 0$. Ukážeme spočetnou subaditivitu. Buďte $E_i \subset ®R^n$ a předpokládejme, že $\lambda^{n*}(E_i) < ∞$. Zvolme $\epsilon > 0$. Podle definice $\lambda^{n*}$ existují $I_i^j \in ©O_n$ takové, že $E_i \subset \bigcup_j I_i^j$ a $\sum_j v(I_i^j) < \lambda^{n*}(E_i) + \frac{\epsilon}{2^i}$. Pak ale platí $\bigcup_i E_i \subset \bigcup_{i, j} I_i^j$, a tedy
		$$ \lambda^{n*}\(\bigcup_i E_i\) ≤ \sum_{i, j} v(I_i^j) < \sum_i \lambda^{n*}(E_i) + \epsilon. $$
		Limitním přechodem $\epsilon \rightarrow 0$ dostáváme spočetnou subaditivitu. $\lambda^{n*}$ je tedy vnější míra.

		Dále ukážeme, že $\lambda^{n*}$ je metrická, tedy pro $A, B \subset ®R^n$ takové, že $\dist(A, B) > 0$ platí
		$$ \lambda^{n*}(A \cup B) ≥ \lambda^{n*}(A)  + \lambda^{n*}(B). $$
		Je-li $\lambda^{n*}(A \cup B) = ∞$, nerovnost zřejmě platí. BÚNO ted $\lambda^{n*}(A \cup B) < ∞$. Zvolme $\epsilon > 0$. Podle předchozího tvrzení existují $I_i \in ©O_n$ takové že $\diam(I_i) < \dist\frac{A, B}{2}$ a $\sum_i v(I_i) < \lambda^{n*}(A \cup B) + \epsilon$. Označme
		$$ ©I_A := \{i \in ®N | I_i \cap A ≠ \O\}, ©I_B := \{i \in ®N | I_i \cap B ≠ \O\}. $$
		Žádný z kvádrů $I_i$ nemůže zasáhnout obě množiny $A, B$, proto jsou tyto množiny disjunktní. Navíc zřejmě $A \subset \bigcup_{i \in ©I_A} I_i$ a $B \subset \bigcup_{i \in ©I_B} I_i$. Proto
		$$ \lambda^{n*}(A) + \lambda^{n*}(B) ≤ \sum_{i \in ©I_A \cup ©I_B}v(I_i) ≤ \sum_{i=1}^∞ v(I_i) < \lambda^{n*} (A \cup B) + \epsilon. $$
		Limitním přechodem $\epsilon \rightarrow 0$ dostaneme požadovanou nerovnost.

		Zbývá „$\forall I \in ©O_n: \lambda^{n*}(I) = v(I)$“. Nerovnost $\lambda^{n*}(I) ≤ v(I)$ je zřejmá, stačí zvolit pokrytí $I_1 = I$.

		Předpokládejme pro spor, že $\lambda^{n*}(I) < v(I)$. Pak existují $I_i \in ©O_i$ takové, že $I \subset \bigcup_i I_i$ a $\sum_i v(I_i) < v(I)$. Zřejmě existuje $J \in ©O_n$ takový, že $\overline{J} \subset I$ a $\sum_i v(I_i) < v(J)$. Protože $\overline{J}$ je kompaktní, existuje $k \in ®N$ takové, že $\overline{J} \subset \bigcup I_{[k]}$. Pak ale $v(J) ≤ \sum v(I_{[k]})$ podle tvrzení výše, což je spor.
	\end{dukazin}
\end{veta}

% 19. 10. 2021

\section{Znaménkové míry}
\begin{definice}[Znaménková míra, náboj]
	Řekneme, že funkce $\sigma: ©A \rightarrow ®R^*$ je znaménková míra na měřitelném prostoru $(X, ©A)$, jestliže

	\begin{itemize}
		\item $\sigma(\O) = 0$,
		\item $\sigma$ nabývá nejvýše jedné z hodnot $±∞$,
		\item ($\sigma$-aditivita) pro libovolnou posloupnost po dvou disjunktních množin $A_n \in ©A$ platí
			$$ \sigma\(\bigcup_{n=1}^∞ A_n\) = \sum_{n=1}^∞ \sigma(A_n). $$
	\end{itemize}

	Konečná znaménková míra se též nazývá náboj.
\end{definice}

\begin{definice}[Kladná a záporná množina]
	Buď $\sigma$ znaménková míra na $(X, ©A)$ Řekneme, že množina $A \in ©A$ je kladná pro $\sigma$, jestliže pro každou měřitelnou množinu $E \subset A$ platí $\sigma(E) ≥ 0$. Množina $A \in ©A$ je záporná pro $\sigma$, jestliže pro každou měřitelnou množinu $E \subset A$ platí $\sigma(E) ≤ 0$.
\end{definice}

\begin{tvrzeni}
	Buď $\sigma$ znaménková míra na $(X, ©A)$ a $E \in ©A$ množina taková, že $0 < \sigma(E) < ∞$. Pak existuje kladná množina $A \subset E$ taková, že $\sigma(A) > 0$.

	\begin{dukazin}
		Kdyby sama $E$ byla kladná, položíme $A := E$. Pokud ne, definujeme
		$$ t_1 := \inf \{\sigma(B) | B \subset E \land B \in ©A\} < 0 $$
		a vybereme $E_1 \subset E$ takovou, že $\sigma(E_1) < \max \{\frac{t_1}{2}, -1\}$. Platí $\sigma(E \setminus E_1) = \sigma(E) - \sigma(E_1) > \sigma(E) > 0$, a pokud je množina $E \setminus E_1$ již kladná, vybereme ji za $A$ a jsme hotovi.

		Pokud ne, pokračujeme stejnou konstrukcí, tedy položíme
		$$ t_2 := \inf\{\sigma(B) | B \subset E \setminus E_1 \land B\in ©A\} < 0 $$
		a zvolíme $E_2 \subset E \setminus E_1$ takovou, že $\sigma(E_2) < \max\{\frac{t_2}{2}, -1\}$. Tímto způsobem buď po konečném počtu kroků najdeme kladnou množinu $A \subset E$ kladné míry, nebo sestrojíme posloupnost disjunktních měřitelných množin $E_1, E_2, …, \subset E$ a posloupnost záporných čísel $t_1, t_2, …$ takové, že $\sigma(E_i) < \max\{t_\frac{i}{2}, -1\} < 0$, $i \in ®N$.

		Položme $A := E \setminus \bigcup_{i=1}^∞ E_i$. Ze spočetné aditivity dostaneme
		$$ \sigma(A) + \sum_{i=1}^∞ \sigma(E_i) = \sigma(E) > 0, $$
		tedy $\sigma(A) > \sigma(E) > 0$ a řada $\sum_{i=1}^∞ \sigma(E_i)$ konverguje, tedy nutně $\sigma(E_i) \rightarrow 0$. Pak ale i $t_i \rightarrow 0$. Ukážeme, že $A$ je kladná: Pro libovolnou $B \subset A$ měřitelnou platí $B \cap E_i = \O$, $i \in ®N$, tedy $\sigma(B) ≥ t_i$, $i \in ®N$ a protože $t_i \rightarrow 0$ je $\sigma(B) ≥ 0$.
	\end{dukazin}
\end{tvrzeni}

\begin{veta}[Hahn-Banachův rozklad]
	Buď $\sigma$ znaménková míra na $(X, ©A)$. Pak existuje rozklad $X = P \cup N$ takový, že $P$ je kladná a $N$ záporná množina pro $\sigma$.

	\begin{dukazin}
		BÚNO $\sigma(E) < ∞$ pro každou $E \in ©A$. (Kdyby ne, pracovali bychom s mírou $-\sigma$.) Položme $\lambda := \sup\{\sigma(E) | E \in ©A \land E \text{ kladná pro } \sigma\}$. Zřejmě $\lambda ≥ 0$ ($\O$ je kladná).

		Buďte $A_i \in ©A$ kladné takové, že $\sigma(A_i) \rightarrow \lambda$ (existence plyne z definice suprema). Pak množina $P := \bigcup_{i=1}^∞ A_i$ je kladná a ze vztahu
		$$ \sigma(P) = \sigma(A_i) + \sigma(P \setminus A_i) ≥ \sigma(A_i), \qquad i \in ®N, $$
		plyne $\sigma(P) = \lambda$. Ukážeme dále, že množina $N := X \setminus P$ je záporná. Nechť $B \subset N$ je měřitelná. Kdyby $\sigma(B) > 0$, pak by podle předchozího tvrzení existovala měřitelná kladná množina $B' \subset B$, pro niž $\sigma(B') > 0$. Pak by ale $P \cup B'$ byla rovněž kladná množina s mírou
		$$ \sigma(P \cup B') = \sigma(P) + \sigma(B') > \sigma(B) = \lambda, $$
		což by byl spor s definicí $\lambda$.
	\end{dukazin}
\end{veta}

\begin{definice}[Jordanův rozklad]
	Buď $\sigma$ znaménková míra na $(X, ©A)$. Pak (nezáporné míry) $\sigma_+(·) := \sigma(· \cap P)$ a $\sigma_-(·) := - \sigma(· \cap N)$ nazýváme kladnou a zápornou částí $\sigma$ a platí $\sigma = \sigma_+ - \sigma_-$.

	Míru $|\sigma|:= \sigma_+ + \sigma_-$ nazýváme totální variací znaménkové míry $\sigma$.
\end{definice}

\begin{tvrzeni}
	Je-li $\sigma = \sigma_+' - \sigma_-'$ jiný rozklad znaménkové míry $\sigma$ na rozdíl dvou nezáporných měr, pak $\sigma_+' ≥ \sigma_+$ a $\sigma_-\ ≥ \sigma_-$.

	\begin{dukazin}
		Pro libovolnou $E \in ©A$ platí
		$$ \sigma_+(E) = \sigma(E \cap P) = \sigma_+'(E \cap P) - \sigma_-'(E \cap P) ≤ \sigma_+'(E \cap P) ≤ \sigma_+'(E). $$
		Podobné se ukáže, že $\sigma_-'(E) ≥ \sigma_-(E)$.
	\end{dukazin}
\end{tvrzeni}

% 26. 10. 2021

\begin{veta}[Regularita Lebesgueovy míry]
	Nechť $E \subset ®R^n$. Je ekvivalentní:
	
	\begin{enumerate}
		\item $E \in ©A_{\lambda^{n*}}$,
		\item $\forall \epsilon > 0 \ \exists F \subset E \subset G$, $F$ uzavřená, $G$ otevřená, $\lambda^n(G \setminus F) < \epsilon$,
		\item $\exists A \subset E \subset B$, $A, B \in ©B^n$, $\lambda^n(B \setminus A) = 0$,
		\item $E \in ©B_0^n$.
	\end{enumerate}

	\begin{dukazin}
		$1 \implies 2$: Mějme $E \in ©A_{\lambda^{n*}}$, $\epsilon > 0$. Nechť nejprve $\lambda^{n*}(E) < ∞$. Pak $\exists I_i \in O_n$, $E \subset \bigcup_i I_i$, $\sum_i v(I_i) < \lambda^{n*}(E) + \frac{\epsilon}{2}$. Položme $G := \bigcup_i I_i$ (otevřená), $E \subset G$, $\lambda^n(G \setminus E) < \frac{\epsilon}{2}$. Je-li $\lambda^{n*}(E) = ∞$, pak ze $\sigma$-konečnosti je $E = \bigcup_m E_m$, $E_m := E \cap [-m, m]^n$. $\lambda^{n*}(E_m) < ∞ \implies \exists G_m$ otevřená, $E_m \subset G_m$, $\lambda^n(G_m \setminus E_m) < \frac{\epsilon}{2^{m+1}}$. $G := \bigcup_m G_m$ otevřené, $E \subset G$, $\lambda^n(G \setminus E) ≤ \sum_m \lambda^n(G_m \setminus E_m) < \frac{\epsilon}{2}$.

		$E^c \in ©A_{\lambda^{m*}} \implies \exists H$ otevřená, $E^c \subset H$, $\lambda^n(H \setminus E^c) < \frac{\epsilon}{2}$. $F:= H^c$ uzavřená, $F \subset E \subset G$, $\lambda^n(G \setminus F) = \lambda^n(G \setminus E) = \lambda^n(E \setminus F) < \frac{\epsilon}{2} + \frac{\epsilon}{2} = \epsilon$.

		$2 \implies 3$: Nechť $E \subset ®R^n$ splňuje 2.
		$$ \forall j \in ®N \ \exists F_j \subset E \subset G_j, F_j \text{ uzavřená}, G_j \text{ uzavřená}, \lambda^n(G_j \setminus F_j) < \frac{1}{j}. $$
		Položme $A := \bigcup_j F_j$, $B := \bigcap_j G_j$, $A, B \in ©B^n$, $A \subset E \subset B$. $\lambda^n(B \setminus A)  ≤ \lambda^n(G_j \setminus F_j) < \frac{1}{j}$ pro libovolné $j \in ®N$, tedy $\lambda^n(B \setminus A) = 0$.

		$3 \implies 4$: Jsou-li $A \subset E \subset B$ jako v 3, pak $B \setminus A$ je $\lambda^n$-nulová množina, a tedy $E \in ©B_0^n$.

		$4 \implies 1$: $©A_{\lambda^{n*}}$ obsahuje $©B^n$ a nulové množiny, tedy obsahuje $©B_0^n$.
	\end{dukazin}
\end{veta}

\begin{veta}[Luzinova (běžně bývá obecnější)]
	Buď $f: ®R^n \rightarrow ®R$ lebesgueovsky měřitelná. Buď $\epsilon > 0$. Pak existuje $G \subset ®R^n$ otevřená taková, že $\lambda^n(G) < \epsilon$ a restrikce $f|_{G^c}$ je spojitá.

	\begin{dukazin}
		Buď $U_1, U_2, …$ posloupnost všech otevřených intervalů s racionálními koncovými body. $f$ je lebesgueovsky měřitelná, tedy $\forall j, f^{-1}(U_1) \in ©B_0^n$. Podle regularity pak $\exists F_j \subset f^{-1}(U_j) \subset G_j$, $F_j$ uzavřená, $G_j$ otevřená, $\lambda^n(G_j \setminus F_j) < \frac{\epsilon}{2^j}$. Položme $G := \bigcup_j (G_j \setminus F_j)$. Zřejmě $G$ je otevřená, $\lambda^n(G) ≤ \sum_j \lambda^n (G_j \setminus F_j) < \epsilon$.

		Pro restrikci $g := F|_{G^c}$ platí:
		$$ g^{-1}(U_j) = \{x \in G^c: f(x) \in U_j\} = f^{-1}(U_j) \cap G^c = G_j \cap G^c, j \in ®N. $$
		Zřejmě $U \subset ®R$ otevřená $\implies$ $U = \cup_{U_j \subset U} U_j$, tedy $g^{-1}(U) = \bigcup_{U_j \in U}g^{-1}(U_j)$ otevřená množina v $G^c$, tedy $g$ je spojitá na $G^c$.
	\end{dukazin}

	\begin{poznamkain}
		Obecně nelze požadovat $\lambda^n(G) = 0$. Např. charakteristická funkce diskontinua kladné míry (podobně jako Cantorovo diskontinuum, ale nenulové míry), které dostaneme tak, že z prostředků intervalů v $i$-tém kroku vždy odebereme intervaly délky $a_i$ tak, aby $a_1 + 2a_2 + 4a_3 + … < 1$. ($G$ z minulé věty pak bude sjednocení malých okolíček krajních bodů odebíraných intervalů.)
	\end{poznamkain}
\end{veta}

\section{Regularita borelovských měr}
\begin{definice}[Regulární borelovská míra]
	Borelovská míra $\mu$ na topologickém (metrickém) prostoru $X$ je regulární, jestliže $\forall B \in ©B(X): \mu(B) = \inf\{\mu(G) | B \subset G, G \text{ otevřená}\}$.
\end{definice}

\begin{poznamka}
	1) Často se hovoří o vnější regularitě (outer regular measure). 2) Pro konečné míry: $\mu$ je regulární $\implies \forall B \in ©B: \mu(B) = \sup\{\mu(F) | F \subset B, F \text{ uzavřená}\}$.
\end{poznamka}

\begin{veta}
	Každá konečná borelovská míra na metrickém prostoru je regulární.

	\begin{dukazin}
		$(X, \rho)$ metrický prostor, $\mu$ borelovská míra na $X$, $\mu(X) < ∞$. Označme
		$$ ©D := \{B \in ©B(X) | \epsilon > 0 \ \exists F \subset B \subset G, F \text{ uzavřená}, G \text{ otevřená}, \mu(G \setminus F) < \epsilon\}. $$
		Ukážeme, že $©D := ©B(X)$. Nejprve $©D$ obsahuje všechny množiny: $F \subset X$ uzavřená, $F_{<\epsilon} := \{x \in X | \rho(x, F) < \epsilon\}$ (otevřená). Zřejmě $F_{<\frac{1}{j}} \searrow F$, $j \rightarrow ∞$ z uzavřenosti $F$. $\mu$ konečna $\implies$ (spojitost míry) $\mu F_{<\frac{1}{j}} \rightarrow \mu(F)$.

		$©D$ je $\sigma$-algebra: $\O \in ©D$, $D \in ©D \implies D^c \in ©D$:
		$$ F \subset D \subset G, \mu(G \setminus F) < \epsilon \implies G^c \subset D^c \subset F^c, \mu(F^c \setminus G^c) < \epsilon. $$

		$D_i \in ©D \implies \bigcup_i D_i \in ©D$:
		$$ \exists F_i \subset D_i \subset G_i, \mu(G_i \setminus F_i) < \frac{\epsilon}{2^i}. $$
		$$ \bigcup_{i=1}^N F_i \subset \cup_{i=1}^∞ D_i \subset \bigcup_{i = 1}^∞ G_i, N \in ®N. $$
		$$ \bigcup_{i=1}^∞ G_i \setminus \bigcup_{i=1}^?F_i $$
		FIXME?
	\end{dukazin}

% 02. 11. 2021

	\begin{poznamkain}
		$\sigma$-konečné míry nemusí být regulární, viz prostor spočetně přímek procházejících počátkem v $®R^2$.
	\end{poznamkain}
\end{veta}

\begin{definice}[Těsnost (= vnitřní regularita)]
	Borelovská míra $\mu$ na metrickém (topologickém) prostoru $X$ je těsná (= tight), jestliže $\forall B \in ©B(X): \mu(B) = \sup\{\mu(K) | K \subset B \text{ kompaktní}\}$.
\end{definice}

\begin{poznamka}
	$\mu$ je Radonova míra, jestliže je těsná a konečná na kompaktech.

	Pokud $\mu$ je konečná a těsná, pak už je $mu$ regulární.

	Jestliže $\mu$ je konečná a regulární a $\mu(X) = \sup\{\mu(K) | K \subset X \text{ kompaktní}\}$, pak $\mu$ je těsná.
\end{poznamka}

\begin{veta}
	Pokud $\mu$ je konečná borelovská míra na úplném separabilním metrickém prostoru, potom už je těsná.

	\begin{dukazin}
		Stačí ukázat $\mu(X) = \sup\{\mu(K) | K \subset X \text{ kompaktní}\}$: $S = \{x_1, x_2, …\} \subset X$ hustá spočetná (ze separability). $\forall n \in ®N: \bigcup_i ©U_{\frac{1}{n}}(x_i) = X$. Nechť je dáno $\epsilon > 0$. Pak $\forall n\ \exists k_n: \mu\(X \setminus \bigcup_{i=1}^{k_n} ©U_{\frac{1}{n}}(x_i)\) < \frac{\epsilon}{2^n}$ (ze spojitosti míry).

		Definujeme $A:= \bigcap_{n=1}^∞ \bigcup_{i=1}^{k_n} ©U_{\frac{1}{n}} \(x_i\)$ ($\in ©B(X)$). $A$ je totálně omezená (tzn. $\forall \epsilon > 0\ \exists F \subset A$ kompaktní tak, že $A \subset \bigcup_{x \in F}B_{\epsilon}(x)$). $\overline{A}$ je totálně omezená a uzavřená $\implies$ $\overline{A}$ je úplný MP (+ totálně omezený), tedy $\overline{A}$ je kompaktní.

		$$ \mu(X \setminus \overline{A}) ≤ \mu(X \setminus A) = \mu\(\bigcup_{n=1}^∞ \mu\(X \setminus \bigcup_{i=1}^{k_n} ©U_{\frac{1}{n}} \(x_i\)\)\). $$
	\end{dukazin}
\end{veta}

\section{Věta o rozšíření míry}
\begin{veta}[Hahn-Komogorov]
	Buď $\tilde{\mu}$ pramíra na algebře $©A \subset ©P(X)$ $\implies$ existuje míra $\mu$ na $\sigma©A$ taková, že $\mu = \tilde{\mu}$ ne $©A$. Je-li $\tilde{\mu}$ $\sigma$-konečná, je $\mu$ určena jednoznačně.

	\begin{dukazin}
		Pro $E \subset X$ položme $\mu^*(E) := \inf\{\sum_{i=1}^∞ \tilde{\mu}(A_i) | A_i \in ©A, E \subset \bigcup_{i=1}^∞A_i\}$. Ověříme, že $\mu^*$ je vnější míra.

		$\forall A \in ©A: \mu^*(A) = \tilde{\mu}(A)$. Zřejmě $\mu^*(A) ≤ \tilde{\mu}(A)$, jelikož můžeme pokrýt $A$ množinami $A, \O, \O, …$. Pro $≥$ mějme $A \subset \bigcup_i A_i$, $A_i \in ©A$. $B_1:= A_1 \cap A$, $B_2 := (A_2 \cap A) \setminus B_1$, … O nich víme, že $A = \bigcup_i B_i$, $B_i$ po dvou disjunktní, $B_i \in ©A$. $\tilde{\mu}(A) = \sum_i\tilde{\mu}(B_i) ≤ \sum_i\tilde{\mu}(A_i)$, tedy z definice infima $\tilde{\mu}(A) ≤ \inf_{A_i}\sum_i \tilde{\mu}(A_i) = \mu^*(A)$.

		Zbývá ukázat, že $©A \subset ©A_{\mu^*}$. Nechť $A \in ©A$, $T \subset X$, $\mu^*(T) < ∞$. Stačí ukázat, že $\mu^*(T) ≥ \mu(T \cap A) + \mu^*(T \setminus A)$. K danému $\epsilon > 0$ existuje pokrytí $T \subset \bigcup_i A_i$ množinami $A_i \in ©A$ takové že $\sum_i \tilde \mu(A_i) < \mu^*(T) + \epsilon$. Protože $T \cap A \subset \bigcup_i(A_i \cap A)$, $T \setminus A \subset \bigcup_i(A_i \setminus A)$ a množiny $A_i \cap A$ i $A_i \setminus A$ patří do ©A, platí
		$$ \mu^*(T \cap A) ≤ \sum_i \tilde \mu(A_i \cap A), \qquad \mu^*(T \setminus A) ≤ \sum_i \tilde \mu(A_i \setminus A). $$
		Sečtením dostaneme
		$$ \mu^*(T \cap A) + \mu^*(T \setminus A) ≤ \sum_i \tilde \mu(A_i \cap A) + \sum_i \tilde (A_i \setminus A) = \sum_i \tilde \mu(A_i) < \mu^*(T) + \epsilon, $$
		z čehož plyne dokazovaná nerovnost. Podle C. věty je $\mu := \mu^*|A_{\mu^*}$ míra, která navíc podle druhé části důkazu rozšiřuje pramíru $\tilde \mu$ a podle třetí je definovaná na $\sigma ©A$.

		Jednoznačnost: ©A uzavřená na konečné průniky, $\tilde{\mu}$ je $\sigma$-konečná $\implies$ $\exists A_n \in ©A$, $A_n \nearrow X, \tilde{\mu}(A_n) < ∞$ $\implies$ $\mu$ je jednoznačně určena (věta o jednoznačnosti míry, TMI1).
	\end{dukazin}
\end{veta}

\begin{poznamka}[Zobecnění příkladu z TMI1]
	$E = \bigtimes_{i=1}^∞ E_i$, $E_i$ úplné separabilní metrické prostory (např. $E_i = ®R$), $\O ≠ I \subset ®N$ … $E_{I} = E_i$, $E, E_I$ metrické prostory. $\pi_I: E \rightarrow E_I$ kanonická projekce. A následující věta:
\end{poznamka}

\begin{veta}[Daniell-Kolmogorov]
	$E_i$ úplné separabilní metrické prostory, $i \in ®N$. Nechť pro každou $\O ≠ I \subset ®N$ existuje borelovská pravděpodobnostní míra $\mu_I$ na $E_I$. A nechť je splněna projektivní vlastnost:
	$$ \O ≠ I \subset J \subset ®N \text{ konečná}, \forall B \in ©B(E_I): \mu_I(B) = \mu_J\(\(\pi_I^J\)^{-1}(B)\), $$
	pak $\exists!$ borelovská míra $\mu$ na $E = \bigtimes_{i=1}^∞ E_i$ taková, že $\forall \O ≠ I \subset ®N$ konečná, $\forall B \in ©B\(E_I\): \mu(\pi_I^{-1}(B) = \mu_I(B))$.
\end{veta}

% 09. 11. 2021

\begin{lemma}
	$$ 1) x_n, x \in E: x_n \rightarrow x \Leftrightarrow x_n(i) \rightarrow x(i), i \in ®N, $$
	$$ x_n, x \in E_I: x_n \rightarrow x \Leftrightarrow x_n(i) \rightarrow x(i), i \in I $$

	2) $\pi_I, \pi_I^J$ jsou spojitá zobrazení.

	3) $\forall I \in ?_f$: $E_I$ je úplný separabilní MP.

	4) $©B(E_I) = \otimes_{i \in I}©B(E_i)$.

	\begin{dukazin}
		1 jsme nedokazovali, 2 a 3 jsou triviální.
		$$ 4) \otimes_{i \in I}©B(E_i) = \sigma \{\bigtimes_{i \in I} B_i | B_i \in ©B(E_i)\} = \sigma \{\bigtimes_{i \in I} | G_i \subset E_i \text{ otevřené}\}, $$
		tedy $\bigtimes_{i \in I} G_i$ je otevřená v $E_I$ $\implies$ $\otimes_{i \in I} ©B(E_i) \subset ©B(E_I)$. Naopak $U \subset E_I$ otevřená $\implies$ $U = \bigcup_{n=1}^∞ U_n$, kde $U_n = \bigtimes G_i^n$, $G_i^n \subset E_i$ otevřená $\implies$ $©B(E_I) \subset \otimes_{i \in I} ©B(E_i)$.
	\end{dukazin}
\end{lemma}

\begin{veta}[Daniell-Kolmogorov]
	$E_i$ úplné separabilní metrické prostory $i \in ®N$. Nechť pro každou $I \in ©I_f$ existuje borelovská pravděpodobnostní míra $\mu_I$ na $E_I$. Nechť $I \subset J \land I, J \in ©I_f \implies \mu_I = \mu_J(\pi_I^J)^{-1}$. Pak existuje právě jedna borelovská pravděpodobnostní míra $\mu$ na $E$ taková, že
	$$ \forall I \in ©I_f: \mu_I = \mu(\pi_I)^{-1}. $$

	\begin{dukazin}
		Položme $©A := \{\pi_I^{-1}(B) | B \in ©B(E_i), I \in ©I_f\}$. Ukážeme nejprve, že systém $©A$ je algebra. (Prostě se ověří podmínky.)

		Definujeme množinovou funkci $\tilde \mu$ na ©A předpisem
		$$ \tilde \mu(A) = \mu_I(B), A = \pi_I^{-1}(B), B \in ©B_I. $$
		Ukážeme nejprve konzistenci této definice (dvě vyjádření podle předpokladů dávají stejný výsledek, když se rozepíší).

		Dále se ukáže, že $\tilde \mu$ je konečně aditivní množinová funkce. (Jednoduché.) Dále dokážeme, že je to pramíra (že splňuje podmínku spojitosti v prázdné množině).

		Podle Hahn-Kolmogorovovy věty lze tedy pramíru $\tilde \mu$ jednoznačně rozšířit na pravděpodobnostní míru $\mu$ na $\sigma ©A$. Zbývá ukázat, že $\sigma ©A = ©B(E)$. Protože projekce jsou spojité, je vzor každé otevřené množiny otevřená, tedy borelovská množina. Platí tedy $\sigma ©A \subset ©B(E)$. Pro opačnou inkluzi si uvědomme, že borelovská $\sigma$-algebra separabilního prostoru $E$ je generována uzavřenými okolími $\overline{U}_\epsilon(x)$ bodů $x \in E$, $\epsilon >0$. Z definice metriky v $E$ a $E_{[n]}$ snadno dostaneme vztah
		$$ \overline{U}_\epsilon(x) = \bigcup_{n=1}^∞ \pi_{[n]}^{-1}(\overline{U}_\epsilon(\pi_{[n](x)})), $$
		tedy $\overline{U}_\epsilon(x) \in \sigma ©A$. Platí tedy i $©B(E) \subset \sigma ©A$ a důkaz je ukončen.

% 16. 11. 2021

	\end{dukazin}
\end{veta}

\section{Charakterizace Riemannovsky integrovatelných funkcí}
\begin{veta}
	Buď $f: [a, b] \rightarrow ®R$ omezená. Pak
	$$ f \in R[a, b] \Leftrightarrow f \text{je spojitá v $\lambda^1$-skoro všude na $(a, b)$}. $$

	\begin{dukazin}
		$(©D_n)$ posloupnost zjemňujících se dělení intervalu $[a, b]$.
		$$ ©D_n = \{a = x_0^{(n)} < x_1^{(n)} < … < x_{k_n}^{(n)} = b\}, n \in ®N, ||©D_n|| = \max_{1 ≤ i ≤ k_n} (x_i^{(n)} - x_{i-1}^{(n)}) \stackrel{n \rightarrow ∞}{\rightarrow} 0. $$
		Označme $s_n(x) := \inf_{[x^{(n)_{i-1}, x_i^{(n)}}} f$, $S_n(x) := \sup_{[x^{(n)_{i-1}, x_i^{(n)}}} f$, $x \in (x_{i-1}^{(n)}, x_i^{(n)}]$, $n \in ®N$ a $S_n(x) := 0$, $S_n(x):= 0$ pro ostatní $x \in ®R$. Toto jsou jednoduché měřitelné funkce.

		Horní a dolní Riemannův součet splňuje
		$$ \underline{\int_a^b f} \stackrel{n \rightarrow ∞}{\leftarrow} s(f, ©D_n) = \int_a^b s_n d\lambda^1, \overline{\int_a^b f} \stackrel{n \rightarrow ∞}{\leftarrow} S(f, ©D_n) = \int_a^b S_n d\lambda^1. $$
		$|f| ≤ M$, tedy $-M ≤ s_1 ≤ s_2 ≤ … ≤ f ≤ … ≤ S_2 ≤ S_1 ≤ M$. Označme $f_1 := \lim_{n \rightarrow ∞} s_n$, $f_2 := \lim_{n \rightarrow ∞} S_n$ (bodové limity funkcí).
		$$ -M ≤ s_n \searrow f_1 ≤ f ≤ f_2 \nwarrow S_n ≤ M, qquad f_1, f_2 \text{ měřitelné}. $$
		Ze zobecněné Leviho věty $\int_a^b s_n d\lambda^1 \rightarrow \int_a^b f_1 d\lambda^1$, $\int_a^b S_n d\lambda^1 \rightarrow \int_a^b f_2 d \lambda^1$.

		$\implies$: Nechť $f \in R[a, b]$, tedy $\underline{\int_a^b f} = \overline{\int_a^b f}$.
		$$ \implies \int_a^b f_1 d\lambda^1 = \int_a^b f_2 d \lambda^1 \implies \int_a^b (f_2 - f_1) d\lambda^1 = 0 \implies f_1 = f_2 \lambda^1\text{-s.v.} $$
		$$ N := \{x \in [a, b] | f_1(x) ≠ f_2(x)\} \cup \{x_i^{(n)} | 0 ≤ i ≤ k_n, n \in ®N\}, \qquad \lambda^1(N) = 0. $$
		Ukážeme, že $f$ je spojitá ve všech bodech množiny $(a, b) \setminus N$: Buď $x \in (a, b) \setminus N$, $\epsilon > 0$. Potom $f_1(x) = f_2(x)$ $\implies$ $\exists n \in ®N$, $S_n(x) - s_n(x) < \epsilon$. $I_n$ nechť je otevřený interval dělení $©D_n$, pro nějž $x \in I_n$. Pak
		$$ s_n(x) ≤ f(y) ≤ S_n(x), y \in I_n \implies |f(y) - f(x)| ≤ 2\epsilon, y \in I_n \implies f \text{ je spojitá v bodě $x$.} $$

		$\Leftarrow$: Nechť $\lambda^1(©D) = 0$, kde $D := \{x \in (a, b): f\text{ není spojitá v }x\}$. Ukážeme, že $S_n(x) - s_n(x) \stackrel{n \rightarrow ∞}{\rightarrow} 0$
		$$ \implies S(f, ©D_n) - s(f, ©D_n) \rightarrow 0 \implies f \in R[a, b]. $$
		Nechť $x \in (a, b) \setminus ©D$, $\epsilon > 0$. Pak $f$ je spojitá v bodě $x$ $\implies \exists \delta > 0$, $|y - x| < \delta$ $\implies$ $|f(y) - f(x)| < \epsilon$.

		Zvolme $n_0$ tak velké, aby $||©D_n|| < \delta$, $n ≥ n_0$. Pak 
		$$ S_n(x) - s_n(x) ≤ 2 \sup\{|f(y) - f(x)|: |y - x| < \delta\} < 2\epsilon. $$
	\end{dukazin}
\end{veta}

\section{Pokrývací věty}
\begin{poznamka}[Úmluva]
	Koulí se myslí uzavřená koule, $B(x, r) = \{y \in ®R^n | ||y-x|| ≤ r\}$, $r > 0$, $\rad B = r$, $t>0 … tB = B(x, t·r)$.
\end{poznamka}

\begin{lemma}[„$5r$“ covering]
	Nechť ©F je systém koulí v $®R^n$ (uzavřené, nedegenerované), $\sup_{B \in ©F} (\rad B) < ∞$. Pak existuje disjunktní podsystém $©F' \subset ©F$ takový, že
	$$ \forall B \in ©F\ \exists B' \in ©F': B \cap B' ≠ \O \land B \subset 5B. $$
\end{lemma}

\begin{dusledek}
	$$ \bigcup ©F \subset \bigcup_{B' \in ©F'} 5B' $$
\end{dusledek}

% 23. 11. 2021

\begin{dukaz}[„$5r$“ covering]
	Označme $R := \sup_{B \in ©F} \rad B$. $©F_k := \{B \in ©F | \rad B \in \(\frac{R}{2^{k+1}}, \frac{R}{2^k}\]\}$, $k = 0, 1, …$ Dále definujeme indukcí systémy $©B_k$, $k = 0, 1, …$: $©B_0$ libovolný maximální disjunktní podsystém $©F_0$. Máme-li $©B_0, …, ©B_k$: $©B_{k+1}$ libovolný maximální disjunktní podsystém
	$$ \{B \in ©F_{k+1} | \forall B' \in ©B_0 \cup … \cup ©B_k: B \bigcup B: = \O\}, $$
	$©F' := \bigcup_{k=0}^∞ ©B_k$ disjunktní podsystém $©F$.

	Nyní už jen ověříme vztah ze znění: Nechť $B \in ©F$, pak $B \in ©F_k$ $\implies$ $\exists B' \in ©B_0 \cup … \cup ©B_k$, $B \cap B' ≠ \O$ (z maximality). Dále víme, že $\frac{R}{2^{k+1}} < \rad B ≤ \frac{R}{2^k}$ a $\frac{R}{2^{k+1}} < \rad B'$, tedy $\rad B < 2\rad B'$. Navíc $B = B(x, r)$ a $B' = B(x', r')$, $r < 2r'$, $B \cap B' ≠ \O$, tedy $||x - x'|| ≤ r + r'$, tj. $\forall y \in B: ||y - x'|| ≤ ||y - x|| + ||x - x'|| ≤ r + r + r' < 5r'$.
\end{dukaz}

\begin{definice}[Vitaliovo pokrytí]
	Nechť $A \subset ®R^n$. Řekneme, že systém uzavřených koulí ©F je Vitaliovým pokrytím (Vitaly Cover) množiny $A$, jestliže
	$$ \forall a \in A\ \forall \epsilon > 0\ \exists B \in ©F: a \in B, \rad B < \epsilon. $$
\end{definice}

\begin{veta}[Vitaly Covering Theorem]
	Nechť $A \subset ®R^n$ a ©F je Vitaliovo pokrytí $A$. Pak existuje disjuktní $©F' \subset ©F$ takový, že $\lambda^n(A \setminus \bigcup ©F') = 0$.

	\begin{dukazin}
		BÚNO nechť $\sup_{B \in ©F}(\rad B) ≤ 1$. „$5r$“ covering lemma nám pak říká, že $\exists ©F' \subset ©F$ disjuktní takový, že platí
		$$ \forall B \in ©F\ \exists B' \in ©F': B \cap B' ≠ \O \land B \subset 5B. $$
		Ukážeme, že $\lambda^n(A \setminus \bigcup ©F') = 0$. Označme $Z_r := (A \setminus \bigcup ©F') \cap U_r(¦o)$, $\forall r > 0$. Ukážeme, že $\lambda^n(Z_r) = 0$.
		
		Označme $©F'' := \{B' \in ©F' | ©B' \cap U_r(¦o) ≠ \O\}$ a $©F''_k := \{B' \in ©F'' | \rad B' \in \(\frac{1}{2^{k+1}}, \frac{1}{2^k}\]\}$, $k = 0, 1, 2, …$ $©F'$ je disjuktní, tudíž
		$$ \sum_{B' \in ©F''} \lambda^n(B') = \sum_{k=0}^∞ \sum_{B' \in ©F''_k} \lambda^n(B') ≤ \lambda^n(B(0, r+2)) < ∞ $$
		$\implies$ $©F_k''$ je konečný $\forall k$. Nechť je dáno $\epsilon > 0$. Pak
		$$ \exists k_0 \in ®N: \sum_{k > k_0} \sum_{B' \in ©F''_k}\lambda^n(B') < \epsilon. $$
		Zvolme pevně $z \in Z_r$. Zřejmě $z \notin \bigcup_{k=0}^{k_0} \bigcup_{B' \in ©F''_k} B' =: K$ (kompakt). Z vlastnosti Vitaliova pokrytí pak:
		$$ \exists B \in ©F: B \cap K = \O, z \in B, B \subset U_r(0). $$
		Z vlastnosti pokrytí $F'$ zřejmě $B' \in ©F''$, $B' \notin \bigcup_{k=0}^{k_0} \bigcup ©F''_k$, tj. $z \in 5B' \implies Z_r \subset \bigcup_{k > k_0} \bigcup_{B' \in ©F''_k} 5B'$ $\implies$ $\lambda^{n*}(Z_r) ≤ \sum_{k > k_0}\sum_{B' \in ©F''_k} \lambda^n(5B') < 5^n \epsilon$. $\epsilon \rightarrow 0$ nám dá $\lambda^n(Z_r) = 0$.
	\end{dukazin}
\end{veta}

\begin{definice}[Lebesgueova hustota]
	Pro $A \subset ®R^n$, $a \in ®R^n$ definujeme $\Theta^{n*}(A, a) = \limsup_{\epsilon \rightarrow 0_+} \frac{\lambda^{n*}(A \cap B(a, \epsilon)}{\lambda^n(B(a, \epsilon))}$ ($≤ 1$) a $\Theta^n_*(A, a) = \liminf_{\epsilon \rightarrow 0_+} \frac{\lambda^{n*}(A \cap B(a, \epsilon)}{\lambda^n(B(a, \epsilon))}$, tzv. horní a dolní hustota množiny $A$ v $a$. Pokud $\Theta^{n*}(A, a) = \Theta^n_* (A, a)$, pak definujeme Lebesgueovu hustotu $A$ v $a$ vztahem $\Theta^n(A, a) = \Theta^{n*}(A, a)$.
\end{definice}

\begin{veta}[Lebesgueova o hustotě (Lebesgue Density Theorem)]
	Pokud $A \subset ®R^n$ je lebesgueovsky měřitelná, potom $\Theta^n(A, ·) = \chi_A(·)$ $\lambda^n$-skoro všude.

% 30. 11. 2021
	
	\begin{dukazin}
		Stačí ukázat, že $\Theta^n(A, a) = 1$ pro $\lambda^n$-skoro všechna $a \in A$. BÚNO nechť $A$ je omezená (obecně: $A \cap B(0, n)$, $n \rightarrow ∞$). Pro číslo $0 < \delta < 1$ označme
		$$ A_{\delta}:= \{a \in A | \liminf_{r \rightarrow 0+} \frac{\lambda^n(A \cap B(a, r))}{\lambda^n(B(a, r))} < \delta\}. $$

		Ukážeme, že $\lambda^n(A_\delta) = 0$. Z toho pak bude plynout, že $\Theta_*^n(A, a) = 1$, a tedy $\Theta^n(A, a) = 1$, pro skoro všechna $a \in A$.

		Nechť pro spor $\lambda^{n*}(A_\delta) > 0$ pro nějaké $\delta < 1$. Z regularity Lebesgueovy míry (nebo z definice vnější míry $\lambda^{n*}$) víme, že existuje otevřená množina $G \supseteq A_\delta$ taková, že $\lambda^n(G) < \delta^{-1} \lambda^{n*}(A_\delta)$. Položme
		$$ ©F := \{B(a, r) | a \in A_\delta, B(a, r) \subset G, \lambda^n(A \cap B(a, r)) < \delta \lambda^n(B(a, r))\}. $$
		Z definice množiny $A_\delta$, je vidět, že ©F je Vitaliovým pokrytím množiny $A_\delta$. Podle Vitaliovy věty tedy existují po dvou disjunktní koule $B_1, B_2, … \in ©F$ takové, že $\lambda^n(A_\delta \setminus \bigcup_i B_i) = 0$. Pak ale
		$$ \lambda^{n*}(A_\delta) = \lambda^{n*}(A_\delta \cap \bigcup_i B_i) ≤ \sum_i \lambda^{n*}(A_\delta \cap B_i) ≤ \sum_i\lambda^n(A \cap B_i) < $$
		$$ < \delta \sum_i \lambda^n(B_i) ≤ \delta \lambda^n(G) < \lambda^{n*}(A_\delta). \text{ \lightning.} $$
	\end{dukazin}
\end{veta}

\section{Důkaz věty o substituci}
\begin{veta}
	Je-li $A \subset ®R^n$ lebesgueovsky měřitelná a $f: A \rightarrow ®R^n$ $L$-lipschitzovské, platí $\lambda^{n*}(f(A)) ≤ L^n\lambda^n(A)$.

	\begin{dukazin}
		Je-li $A \subset B = B(x, r)$, pak $f(A) \subset f(b) \subset B(f(x), L·r)$. $\implies \lambda^{n*}(f(A)) ≤ L^n \lambda^n(B)$.

		Ukážeme, že pro $N \subset ®R^n$ nulovou (tj. $\lambda^n(N) = 0$) je $\lambda^n(f(N)) = 0$: $N$ nulová $\implies$ $\forall \epsilon > 0\ \exists I_i$ otevřené kvádry, $N \subset \bigcup_i I_i$, $\sum_i \lambda^n(I_i) < \epsilon$.

		Můžeme zařídit, aby $\frac{r(I_i)}{R(I_i)} ≥ \eta > 0$, $i \in ®N$, kde $R(I)$ a $r(I)$ jsou poloměry opsané a vepsané koule $I$: Rozdělíme intervaly vůči delší straně.

		Když $B_i$ jsou koule opsané $\overline{I_i}$, pak $\lambda^n(B_i) ≤ \eta^{-n} \lambda^n(I_i)$ ($B_i' \subset I_i \subset B_i … \lambda^n(I_i) > \lambda^n(B_i') ≥ \eta^n \lambda^n (B_i)$).

		$$ \lambda^{n*}(f(N)) ≤ \lambda^{n*} (\bigcup_i f(I_i)) ≤ \lambda^{n*}(\bigcup_if(B_i)) ≤ \sum_i \lambda^{n*}(f B_i) ≤ L^n \sum_i \lambda^n(B_i) ≤ $$
		$$ ≤ \(\frac{L}{\eta}\)^n \sum_i \lambda^n(I_i) < \(\frac{L}{\eta}\)^n\epsilon. $$
		$$ \epsilon \rightarrow 0 … \lambda^{n*}(f(N)) = 0. $$

		$A \subset ®R^n$ měřitelná, $\epsilon > 0$, BÚNO nechť $\lambda^n(A) < ∞$ (jinak je nerovnost triviálni). $\lambda^n$ regulární $\implies$ $\exists G \supset A$ otevřená, že $\lambda^n(G) < \lambda^n(A) + \epsilon$. $©F := \{B \text{ uzavřená koule}| B \subset G\}$ Vitaliovo pokrytí $G$ $\implies$ $B_1, B_2, … \in ©F$ disjunktní, $\lambda^n(G \setminus \bigcup_i B_i) = 0$.

		$$ \lambda^{n*}(f(A)) ≤ \lambda^{n*}(f(G)) ≤ \lambda^{n*}(\bigcup_i f(B_i) \cup f(N)) ≤ \sum_i \lambda^{n*}(f(B_i)) + \lambda^{n*}(f(N)) ≤ $$
		$$ ≤ L^n \sum_i \lambda^n(B_i) = L^n \lambda^n(G) < L^n \lambda^n(G) < L^n \lambda^n(A) + L^n \epsilon \rightarrow L^n \lambda^n(A). $$
	\end{dukazin}
\end{veta}

\begin{definice}[Funkcionální norma]
	Pro $L: ®R^n \rightarrow ®R^n$ lineární zobrazení, definujeme $||L||:=\sup_{||u|| ≤ 1} ||Lu||$.
\end{definice}

\begin{poznamka}
	Označme $\delta(L) := \inf_{||u|| = 1} ||L u||$, $L$ regulární $\Leftrightarrow \delta(L) > 0$. Tedy platí
	$$ \delta(L) ||u|| ≤ ||L u|| ≤ ||L||·||u||, u \in ®R^n. $$
\end{poznamka}

\begin{tvrzeni}
	$L, M: ®R^n \rightarrow ®R^n$ dvě regulární lineární zobrazení. Nechť existuje $\gamma > 0$ takové, že $\forall u \in ®R^n: ||Lu|| ≤ \gamma ||Mu||$. Pak $|\det L| ≤ \gamma^n |\det M|$.

	\begin{dukazin}
		a) Nechť $M = \id$. Z předpokladů plyne, že pro každou kouli $B = B(O, R)$ je $L(B) \subset \gamma B$, tedy
		$$ |\det L|\lambda^n(B) = \lambda^n(L(B)) ≤ \gamma^n \lambda^n(B) \implies |\det L| ≤ \gamma^n. $$

		b) Pro $M$ obecné: $(v = Mu)$,
		$$ ||LM^{-1} v|| ≤ \gamma ||v||, v \in ®R^n \implies |\det LM^{-1}| ≤ \gamma^n \implies |\det L| ≤ \gamma^n|\det M|. $$
	\end{dukazin}
\end{tvrzeni}

% 07. 12. 2021

\begin{dukaz}[Věty o substituci]
	Ať je dáno $\epsilon > 0$. $\forall x \in ©U\ \exists r_x > 0\ \forall y \in B(x, r_x)$:
	$$ 1. ||Dg(y) - Dg(x)|| < \epsilon·\delta(Dg(x)) \qquad \text{(ze spojitosti diferenciálu $(Dg(·))$)}, $$
	$$ 2. ||g(y) - g(x) - Dg(x)(y - x)|| < \epsilon·\delta(Dg(x))||y - x|| \qquad \text{(ze spojitosti diferenciálu $(Dg(x))$)}. $$

	Z $\delta(L)||u|| ≤ ||Lu|| ≤ ||L||·||u||$ je
	$$ 1.' ||Dg(y)u - Dg(x)u|| < \epsilon·||Dg(x)u||, \qquad u \in ®R^n, $$
	$$ 2.' ||g(y) - g(x) - Dg(x)(y - x)|| < \epsilon·||Dg(x))(y - x)||. $$

	$\exists \{x_1, x_2, …\} \subset ©U$ (spočetná) taková, že $©U = \bigcup_i B(x_i, r_{x_i})$. (Neboť existují $K_j$ kompaktní, které $K_j \nearrow ©U$.) $B_i := B(x_i, r_{x_i})$, $L_i = Dg(x_i)$, $i \in ®N$.
	$$ 1.' \implies 1.'' (1 - \epsilon)||L_iu|| ≤ ||Dg(x)u|| ≤ (1 + \epsilon)||L_iu||, u \in ®R^n, x \in B_i, i \in ®N. $$
	Existuje měřitelný rozklad $U = \bigcup_{i, j = 1}^∞ E_{i, j}$ tak, že:
	$$ (a) E_{i, j} \subset B_i, \qquad (b) \diam E_{i, j} < \frac{1}{j}, \qquad (c) \forall x \in E_{i, j}: r_x > \frac{1}{j}. $$

	$$ \implies \forall x, y \in E_{i, j}: ||g(y) - g(x)|| \overset{2.'}{≤} (1 + \epsilon)||Dg(x)(y-x)|| \overset{1''}{≤} (1 + \epsilon)^2 ||L_i(y - x)||, $$
	$$ ||g(y) - g(x)|| ≥ (1 - \epsilon)||Dg(x)(y - x) ≥ (1 - \epsilon)^2 ||L_i(y - x)||. $$
	$\implies$ zobrazení $g \circ L_i^{-1}: L_i(©U) \rightarrow g(©U)$ je $(1 + \epsilon)^2$-lipschitzovské, stejně jako zobrazení $L_i \circ g^{-1}: g(©U) \rightarrow L_i(©U)$ je $(1 - \epsilon)^{-2}$-lipschitzovské. Označme $\eta := \max\{(1 + \epsilon)^2, (1 - \epsilon)^{-2}\}$.

	$$ \lambda^n(g(A)) = \lambda^n(g(\bigcup_{i, j} E_{i, j})) = \lambda^n(\bigcup_{i, j} g(E_{i, j})) = \sum_{i, j} \lambda^n(g(E_{i, j})) ≤ \eta^n \sum_{i, j}\lambda^n(L_i(E_{i, j})) \overset{\text{TMI1}}{=} $$
	$$ = \eta^n \sum_{i, j} |\det L_i| \lambda^n(E_{i, j}) = \eta^n \sum_{i, j}\int_{E_{i, j}} |\det L_i| dx ≤ \eta^{2n} \sum_{i, j} \int_{E_{i, j}} |J g(x)|dx = $$
	$$ = \eta^{2n} \int_A |Jg(x)|dx. $$
	Podobně
	$$ \lambda^n(g(A)) ≥ \eta^{-n}\sum_{i, j} \lambda^n(L(E_{i, j})) ≥ \eta^{-n} \sum_{i, j} \int_{E_{i, j}}\eta^{-n}|Jg(x)| dx = \eta^{-2n}\int_A |Jg(x)| dx. $$
	Následně pro $\epsilon \rightarrow 0$ je $\eta \rightarrow 1$ a $\lambda^n(g(A)) = \int_A|J(g(x))|dx$.
\end{dukaz}

\section{Konvergence posloupnosti funkcí}
\begin{poznamka}[Přípomenutí TMI1]
	$f_n, f: (X, ©A, \mu) \rightarrow ®R$ nebo $®C$ jsou měřitelné.

	$$ f_n \overset{\text{s. v.}}{\rightarrow} f ≡ \mu\{x | f_n(x) \not\rightarrow f(x)\} = 0. $$
	$$ f_n \overset{L^p}{\rightarrow} f ≡ ||f_n - f||_p \rightarrow 0. $$
	$$ f_n \overset{\mu}{\rightarrow} f ≡ \forall \epsilon > 0: \mu\{x | ||f_n(x) - f(x)|| ≥ \epsilon\} \rightarrow 0. $$

	\begin{tvrzeniin}
		$$ f_n, f \in L^p(\mu), f_n \overset{L^p}{\rightarrow} f \implies f_n \overset{\mu}{\rightarrow} f. $$
		$$ \mu(X) < ∞: f_n \overset{\text{s. v.}}{\rightarrow} \implies f_n \overset{\mu}{\rightarrow} f. $$
		$$ \mu(X) < ∞: f_n \overset{\mu}{\rightarrow} f \implies \exists f_{n_k}, f_{n_k} \overset{\text{s. v.}}{\rightarrow} f. $$
		$$ \mu(X) < ∞: 1 ≤ p < q ≤ ∞ \implies L^p(\mu) \supset L^q(\mu), f_n \overset{L^q}{\rightarrow} f \implies f_n \overset{L^p}{\rightarrow} f. $$
	\end{tvrzeniin}
\end{poznamka}

\begin{veta}[Lebesgueova věta + upgrade]
	$f_n \overset{\text{s. v.}}{\rightarrow} f$, $\exists g \in L^1(\mu), |f_n| ≤ g\ \forall n \implies \int f_n d\mu \rightarrow \int fd\mu$.

	Dokonce $f_n \overset{L^1}{\rightarrow} f$.

	\begin{dukazin}
		BÚNO $f_n(x) \rightarrow f(x)$, $x \in X$ (například v těch bodech předefinujeme všechny funkce na 0).
		$$ g_n := \inf\{f_n, f_{n+1}, …\}, h_n := \sup\{f_n, f_{n+1}\}. $$
		$$ -g ≤ g_n ≤ f_n ≤ h_n ≤ g, \qquad g_n \nearrow f \swarrow h_n. $$
		$$ |f_n - f| ≤ h_n - g_n ≤ 2g \in L^1(\mu), \qquad h_n - g_n \searrow \overset{\text{Levi}}{\implies} \int(h_n - g_n) d\mu \rightarrow 0 \implies $$
		$$ \implies \int |f_n - f| d\mu \rightarrow 0 \Leftrightarrow f_n \overset{L^1}{\rightarrow} f. $$
	\end{dukazin}
\end{veta}

\begin{poznamka}
	$f \in L^1(\mu) \implies \lim_{c \rightarrow ∞} \int_{x: |f(x)| ≤ c} |f(x)| d\mu(x) = 0$.
\end{poznamka}

% 14. 12. 2021

\begin{definice}[Stejnoměrně integrovatelná posloupnost]
	Řekneme, že posloupnost $(f_n)$ měřitelných funkcí na $(X, ©A, \mu)$ je stejnoměrně integrovatelná (uniformly integrable), jestliže
	$$ \lim_{c \rightarrow ∞} \sup_n \int_{|f_n| ≥ C} |f_n| d\mu = 0. $$
\end{definice}

\begin{tvrzeni}
	$\mu$ konečná, $(f_n)$ stejnoměrně integrovatelná $\implies f_n \in L^1(\mu)$, $\sup_n ||f_n||_1 < ∞$.

	\begin{dukazin}
		$\int |f_n| = \underbrace{\int_{|f_n| < c} |f_n| d\mu}_{≤ c · \mu(X)} + \underbrace{\int_{|f_n| > c} |f_n| d\mu}_{<1\text{ pro $c$ dostatečně velké}} ≤ c \mu(X) + 1$ pro dostatečně velká $c$.
	\end{dukazin}
\end{tvrzeni}

\begin{veta}
	Nechť $\mu(X) < ∞$ a $f_n \overset{\mu}{\rightarrow} f$. Pak $f_n \overset{L_1}{\rightarrow} f$ $\Leftrightarrow$ $(f_n)$ je stejnoměrně integrovatelná.

	\begin{dukazin}
		„$\Leftarrow$“: Nechť $f_n \overset{\mu}{\rightarrow} f$, $(f_n)$ je stejnoměrně integrovatelná. Pak $f_n \in L^1(\mu)$ a existuje vybraná podposloupnost $(f_{n_j})$, $f_{n_j} \overset{s. v.}{\rightarrow} f$.
		$$ \int |f| d\mu = \int(\lim_{j \rightarrow ∞} |f_{n_j}|)d \mu ≤ \liminf_{j \rightarrow ∞} \int |f_{n_j}| d\mu < ∞ \implies f \in L^1(\mu). $$
		Předpokládejme nejprve, že $\exists c \in ®R$, $|f_n| < c$, $|f| ≤ c$ skoro všude. Buď $\epsilon > 0$, položme $\delta := \frac{\epsilon}{2 \mu(X)}$.
		$$ \int |f_n - f| d\mu = \int_{\{|f_n - f| ≤ \delta\}} |f_n - f| d\mu + \int_{\{|f_n - f| > \delta\}} |f_n - f| ≤ $$
		$$ ≤ \delta \mu(X) + 2c \mu(\{x | |f_n(x) - f(x)| > \delta\}) \overset{n ≥ n_0}{<} \frac{\epsilon}{2} + 2c\frac{\epsilon}{4c} < \epsilon \implies f_n \overset{L_1}{\rightarrow} f. $$
		Nyní $f_n, f \in L^1$ libovolné, $f_n \overset{\mu}{\rightarrow} f$, $(f_n)$ stejnoměrně integrovatelná, $\epsilon > 0$.
		$$ \int |f_n - f| d\mu ≤ $$
		$$ ≤ \int_{\{|f_n| ≤ c \land |f| ≤ c\}} |f_n - f| d\mu + \int_{\{|f_n| > c\}} |f_n - f| d\mu + \int_{\{|f| > c\}} |f_n - f| d\mu=: I_n^1(c) + I_n^2(c) + I_n^3(c), $$
		$$ I_n^2(c) ≤ $$
		$$ ≤ \int_{\{|f_n| > c\}} |f_n| d\mu + \int_{\{|f_n| > c \land |f| ≤ c\}} |f| d\mu + \int_{\{|f_n| > c \land |f| > c\}} |f| d\mu ≤ 2 \int_{\{|f_n| > c\}} |f_n| + \int_{\{|f| > c\}} |f|, $$
		$$ I_n^3(c) ≤ $$
		$$ ≤ \int_{\{|f| > c\}} |f| d\mu + \int_{\{|f_n| > c \land |f| > c\}} |f_n| d\mu + \int_{\{|f_n| ≤ c \land |f| > c\}} |f_n| d\mu ≤ \int_{\{|f_n| > c\}} |f_n| + 2\int_{\{|f| > c\}} |f|, $$
		$$ I_n^2(c) + I_n^3(c) < \frac{\epsilon}{3}, \qquad \forall c ≥ c_0, $$
		z první části navíc $I_n^1(c) < \frac{\epsilon}{2}$, $\forall n ≥ n_0$. Tedy $\int |f_n - f| d\mu < \epsilon$.

		„$\implies$“ Nechť $f_n \overset{L_1}{\rightarrow f}$, $\epsilon >0$.
		$$ \forall c: \int_{\{|f_n| > c\}} |f_n| d\mu ≤ \int_{\{|f_n| > c\}} |f_n - f|d\mu + \int_{\{|f_n| > c \land |f| ≤ \frac{c}{2}\}} |f| + \int_{\{|f_n| > c \land |f| > \frac{c}{2}\}} |f| ≤ $$
		$$ ≤ \int |f_n - f|d\mu + \frac{c}{2} \mu \{x | |f_n(x) - f(x)| ≥ \frac{c}{2}\} + \int_{\{|f| > \frac{c}{2}\}} |f| d\mu ≤ 2||f_n - f||_1 + \frac{\epsilon}{2} < \epsilon. $$

		Protože $f \in L^1$, existuje $c_0>0$ takové, že $\int_{|f|>\frac{c}{2}} |f| < \frac{\epsilon}{2}$ pro $c > c_0$. Rovněž pro každou funkci $f_1, …, f_{n_0}$ existuje $c_i > 0$ takové, že $\int_{|f_i| > c} |f_i| < \epsilon$, $c > c_i$, $i \in [n_0]$. Pro $c > \max\{c_{[n_0]_0}\}$ pak platí $\int_{|f_n| > c} |f_n| < \epsilon$ pro všechna $n \in ®N$. Tím je dokázána stejnoměrná integrovatelnost $f_n$.
	\end{dukazin}
\end{veta}


\end{document}
