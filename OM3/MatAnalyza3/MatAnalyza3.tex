\documentclass[12pt]{article}					% Začátek dokumentu
\usepackage{../../MFFStyle}					    % Import stylu



\begin{document}

% 29. 09. 2021

\section*{Organizační úvod}
TODO!!!

\section*{Úvod}
TODO!!!

% 06. 10. 2021

\begin{veta}[Spojitý obraz kompaktu]
	Nechť $(P, \rho)$ a $(Q, \tau)$ jsou metrické prostory a $f: P \rightarrow Q$ je spojité zobrazení. Nechť $K \subset P$ je kompaktní množina. Potom $f(K)$ je kompaktní.

	\begin{dukazin}
		Nechť $y_n \in f(K)$. Pak $\exists x_n \in K$, $f\(x_n\) = y_n$. Z definice kompaktnosti $\exists x \in K, x_{n_k} \rightarrow x \in K$. Podle Heineho věty $f\(x_{n_k}\) = f\(y_{n_k}\) \rightarrow f(x) \in f(K)$.
	\end{dukazin}
\end{veta}

\begin{definice}
	Nechť $(®P, \rho)$ a $(®Q, \tau)$ jsou metrické prostory, $K \subset ®P$ a $f: K \rightarrow ®Q$. Řekneme, že $f$ je na $K$ stejnoměrně spojitá, pokud
	$$ \forall \epsilon > 0\ \exists \delta > 0\ \forall x, y \in K: \left(\rho(x, y) < \delta \implies \tau\left(f(x), f(y)\right) \right). $$
\end{definice}

\begin{veta}[O vztahu spojitosti a stejnoměrné spojitosti na MP]
	Nechť $(®P, \rho)$ a $(®Q, \tau)$ jsou MP, $K \subset ®P$ je kompaktní a nechť $f: K \rightarrow ®Q$ je spojitá. Pak $f$ je stejnoměrně spojitá na $K$.

	\begin{dukazin}
		Nechť $f$ je spojitá, ale ne stejnoměrně. Potom
		$$ \exists \epsilon > 0\ \forall \delta > 0\ \exists x, y \in K: \rho(x, y) < \delta \land \tau\(f(x), f(y)\) ≥ \epsilon. $$
		Zvolíme $\delta_n = \frac{1}{n}$ a pro každé si najdeme $x_n, y_n$. $K$ je kompaktní, tedy existuje podposloupnost $x_{n_k} \rightarrow x_0 \in K$.
		$$ \rho(y_{n_k}, x_{0}) ≤ \rho(x_{n_k}, y_{n_k}) + \rho(x_n, x_0) ≤ \frac{1}{n_k} + \rho(x_n, x_0) \rightarrow 0 \implies y_{n_k} \rightarrow x_0 $$
		Z Heineho věty $f(x_{n_k}) \rightarrow f\(x_0\)$ a $f\(y_{n_k}\) \rightarrow f\(x_0\)$. Ale my máme, že jsou od sebe vzdáleny o $\epsilon$. \lightning
	\end{dukazin}
\end{veta}

\section{Úplné metrické prostory}
\begin{definice}[Cauchyovská posloupnost]
	Nechť $(®P, \rho)$ je metrický prostor a $\{x_n\}_{n=1}^∞$ je posloupnost bodů z ®P. Řekneme, že $x_n$ splňuje Bolzano-Cauchyovu podmínku (případně, že je cauchyovská), jestliže platí:
	$$ \forall \epsilon > 0\ \exists n_0 \in ®N\ \forall m, n \in ®N, m,n ≥ n_0: \rho\left(x_n, x_m\right) < \epsilon. $$
\end{definice}

\begin{dusledek}
	Každá konvergentní posloupnost je cauchyovská.
\end{dusledek}

\begin{definice}[Úplný prostor]
	Řekneme, že metrický prostor $(®P, \rho)$ je úplný, jestliže každá cauchyovská posloupnost je konvergentní.
\end{definice}

\begin{veta}[Vztah kompaktnosti a úplnosti]
	Nechť $(®P, \rho)$ je MP a ®P je kompaktní. Pak ®P je úplný metrický prostor.

	\begin{dukazin}
		Nechť $\{x_n\}_{n=1}^∞$ je cauchyovská posloupnost. ®P kompaktní $\implies$ $\exists x_{n_k} \rightarrow x \in ®P$. Nechť $\epsilon > 0$. Najdu $n_0$ z BC podmínky. Z $x_n \rightarrow x \exists k_0 \forall k ≥ k_0: \rho\left(x_{n_k}, x\right) < \epsilon$. Nalezneme $n_k$, $k ≥ k_0$, $n_k ≥ n_0$. Pak
		$$ \forall n ≥ n_0: \rho\(x_n, x\) ≤ \rho\(x_n, x_{n_k}\) + \rho\left(x_{n_k}, x\right) < 2\epsilon. $$
	\end{dukazin}
\end{veta}

\begin{veta}[Úplnost a prostor spojitých funkcí]
	Metrický prostor $C([0, 1])$ se supremovou metrikou je úplný.

	\begin{dukazin}
		Nechť $\{f_n\}_{n=1}^∞$ je cauchyovská posloupnost. Tedy
		$$ \forall \epsilon > 0\ \exists n_0\ \forall m, n ≥ n_0: \rho\(f_n, f_m\) = \sup_{x \in [0, 1]} |f_n(x) - f_m(x)| < \epsilon. \qquad (*) $$
		Zvolme $x \in [0, 1]$ pevné. Potom máme posloupnost reálných čísel místo funkcí, tedy z BC podmínky v ®R je $f_n(x)$ cauchyovská, tedy existuje $\lim_{n \rightarrow ∞} f_n(x) = f(x) \in ®R$. Takto jsme si zadefinovali novou funkci $f$.

		$f_n \rightarrow f$. Provedeme limitu $n \rightarrow ∞$ na $(*)$.
		$$ \forall \epsilon > 0\ \exists n_0\ \forall m, n ≥ n_0: \sup_{x \in [0, 1]}|f(x) - f_n(x)| ≤ \epsilon. $$
		Tedy $\rho(f, f_n) ≤ \epsilon \implies f_n \rightarrow f$.

		$f$ je spojitá: Nechť $y \in [0, 1]$. Chceme dokázat, že $f$ je spojitá v $y$. Nechť $\epsilon > 0$. Z BC $\exists n_0\ \forall x \in [0, 1]: |f_n(x) - f_m(x)| < \epsilon$. Zafixujeme $n_0$. $f_{n_0}$ je spojitá v $y$, tedy $\exists \delta > 0\ \forall x \in [0, 1], |x - y| < \delta: |f_{n_0}(x) - f_{n_0}(y)| < \epsilon$. Nyní $\forall x \in [0, 1], |x - y| < \delta$:
		$$ |f(x) - f(y)| ≤ |f(x) - f_{n_0}(x)| + |f_{n_0}(x) - f_n(y)| + |f_{n_0}(y) - f(y)| ≤ 3\epsilon. $$
		(Třetí člen dostaneme tak, že zafixujeme $m = n_0$ a $n$ pošleme do nekonečna v BC podmínce výše.)
	\end{dukazin}
\end{veta}

% 07. 10. 2021

\begin{veta}[Banachova, o kontrakci]
	Nechť $(®P, \rho)$ je úplný MP a $T: ®P \rightarrow ®P$ je kontrakce (tedy $\exists \gamma \in (0, 1)\ \forall x, y \in P: \rho\(T(x), T(y)\) ≤ \gamma · \rho (x, y)$). Pak existuje právě jedno $x \in ®P$ tak, že $T(x) = x$.

	\begin{dukazin}
		Zvolme $x_1 \in P$ libovolně. Definujeme indukcí $x_{n+1} = T\left(x_n\right)$. Tvrdíme, že $x_n$ je cauchyovská, $\forall n \in ®N$:
		$$ \rho(x_{n+1}, x_n) = \rho(T(x_n), T(x_{n+1})) ≤ \gamma \rho\left(x_n, x_{n+1}\right) ≤ \gamma^2 \rho\left(x_{n-1}, x_n\right) ≤ … ≤ \gamma^n \rho\left(x_1, x_2\right). $$
		Nechť $\epsilon > 0$, zvolme $n_0$, aby $\rho\(x_2, x_1\) \gamma^{n_0 - 1} \frac{1}{1 - \gamma} < \epsilon$. Nyní $\forall m$, $n ≥ n_0$, $m < n$:
		$$ \rho\(x_m, x_n\) ≤ \rho\(x_{m+1}, x_m\) + … + \rho\(x_n, x_{n-1}\) ≤ \rho(x_1, x_2)·\(\gamma^{m-1} + … + \gamma^{n-2}\) ≤ $$
		$$ ≤ \rho\(x_2, x_1\) \gamma^{n_0 - 1} \frac{1}{1 - \gamma}. $$
		Tedy $x_n$ je cauchyovská a má limitu.

		Tvrdíme, že $T\(x_n\) \rightarrow T(x)$: $T$ je spojité v $x$. K $\epsilon > 0$ volme $\delta = \epsilon$. Pak 
		$$ \forall y \in B(x, \delta): \rho(x, y) < \delta \implies \rho(T(x), T(y)) ≤ \gamma · \rho(x, y) ≤ \gamma \delta < \epsilon. $$
		Podle Heineho věty $x_n \rightarrow x \implies T\left(x_n\right) \rightarrow T(x)$. Víme, že $x_{n+1} = T(x_n)$, tj.
		$$ \lim_{n \rightarrow ∞} x_{n+1} = \lim_{n \rightarrow ∞} T\left(x_n\right). $$

		Jednoxznačnost: Nechť $\exists x, y$, $T(x) = x$ a $T(y) = y$. Pak
		$$ \rho(x, y) = \rho(T(x), T(y)) ≤ \gamma·\rho(x, y) \implies \rho(x, y) = 0 \implies x = y. $$
	\end{dukazin}
\end{veta}

\begin{veta}[O převedení na integrální tvar]
	Nechť $I \subset ®R$ je otevřený interval, $x_0 \in I$, $f: I \times ®R \rightarrow ®R$ spojité a $y: I \rightarrow ®R$ je spojitá. Pak $y$ je řešení ODR $y' = f(x, y(x))$ na $I$ s počáteční podmínkou $y(x_0) = y_0$ právě tehdy, když $y(x) = y_0 + \int_{x_0}^x f(s, y(s)) ds$, $\forall x z \in I$.

	\begin{dukazin}
		$\implies:$ víme $y'(s) = f(s, y(s))$ je spojité, tj. lze integrovat:
		$$ y(x) - y_0 = y(x) - y(x_0) = \int_{x_0}^x y'(s) ds = \int_{x_0}^x f(s, y(s))ds. $$

		$\Leftarrow:$ zderivujeme (integrant je spojitý $\implies$ integrál lze zderivovat) $y'(x) = f(x, y(x))$. Zřejmě také $f\left(x_0\right) = y_0$.
	\end{dukazin}
\end{veta}

\begin{veta}[Picard]
	Nechť $I \subset ®R^2$ je otevřený interval a $(x_0, y_0) \in I$.

	\begin{poznamkain}
		Stačí libovolná otevřená množina.
	\end{poznamkain}

	\begin{dukazin}
		Nechť $f: I \rightarrow ®R$ je spojitá a lokálně lipschitzovská vůči $Y$. Pak existuje $\left(x_0 - \delta, x_0 + \delta\right)$ okolí $x_0$ a funkce $y(x)$ definovaná na $(x_0 - \delta, x_0 + \delta)$ tak, že $y(x)$ splňuje ODR $y'(x, y(x))$ na $(x_0 - \delta, x_0 + \delta)$ s počáteční podmínkou $y(x_0) = y_0$. Navíc $y$ je jediné řešení na $(y_0 - \delta, y_0 + \delta)$.
	\end{dukazin}

	\begin{dukazin}
		Zvolme $\delta, \Delta > 0$, aby $\[x_0 - \delta, x_0 + \delta\] \times \[y_0 - \Delta, y_0 + \Delta\] \subset I$. Definujeme
		$$ X = \{y \in C([x_0 - \delta, x_0 + \delta]) | y(x) \in [y_0 - \Delta, y_0 + \Delta] \}. $$
		Definujeme operátor $T: C(\[x_0 - \delta, x_0 + \delta\]) \rightarrow C(\[x_0 - \delta, x_0 + \delta\])$ tak, že $T[y](x) = y_0 + \int_{x_0}^x f(s, y(s)) ds$.

		Klíčové pozorování: $y$ řeší naši ODR $\Leftrightarrow$ $T[y] = y$. (Z předchozí věty.)

		$X$ je úplný: Nejprve dokážeme, že $X$ je uzavřená podmnožina $C([x_0 - \delta, x_0 + \delta])$: $X$ lze zapsat (dokáže se velmi přímočaře) jako $\overline{B(y_0, \Delta)}$: Tj. $X$ je uzavřená a úplnost se dědí na uzavřené podmnožiny.

		Máme pevné $\delta, \Delta > 0$, že $A := \[x_0 - \delta, x_0 + \delta\]\times \[y_0 - \Delta, y_0 + \Delta\] \subset I$. $f$ spojitá na tomto kompaktu $\implies \exists M > 0$, $|f(x, y)| ≤ M$ na $A$. Z lipschitzovskosti $\exists x > 0: \forall [x, y] \in A, \forall [x, \tilde{y}] |f(x, y) - f(x, \tilde{y})| ≤ K·|y - \tilde{y}|$. Případným zmenšením $\delta > 0$ dosáhneme
		$$ \delta ≤ \min\{\frac{\Delta}{M}, \frac{1}{2K}\}. $$

		Ukážeme $T: X \rightarrow X$: $y \in X$, $y(x) \in \[y_0 - \Delta, y_0 + \Delta\]$.
		$$ |T[y](x) - y_0| = |\int_{x_0}^x f(s, y(x)) ds| ≤ |x - x_0| M ≤ \delta·M ≤ \Delta. $$
		$$ \implies T[y](x) \in \[y_0 - \Delta, y_0 + \Delta\] \implies T[y] \in X. $$

		Dokážeme, že je toto zobrazení kontrakce a pak už máme hotovo z věty výše. Kontrakce: Nechť $y, \tilde{y} \in X$ a $x \in [x_0 - \delta, x_0 + \delta]$.
		$$ T[y](x) - T[\tilde{y}](x)| = |\int_{x_0}^x (f(s, y(s)) - f(s, \tilde{y}(s))) ds| ≤ \int || ≤ $$
		$$ ≤ \int_{x_0}^x |K·(y(s) - \tilde{y}(s))|ds < |x_0 - x|·K·\sup_{s \in \[x_0 - \delta, x_0 + \delta\]}(y(s) - \tilde{y}(s)) ≤ \delta·K·\rho(y, \tilde{y}) ≤ \frac{1}{2}\rho(y, \tilde{y}). $$
		Supremum dá $\rho(T[y], T[\tilde{y}]) ≤ \frac{1}{2} \rho(y, \tilde{y})$.
	\end{dukazin}
\end{veta}

% 13. 10. 2021

\section{Funkce více proměnných}
	\subsection{Úvodní definice a spojitost}
	\begin{poznamka}
		Většina definice je jen „opakování“ z letního semestru, nebo z definice spojitých funkcí na metrických prostorech.
	\end{poznamka}

	\begin{definice}[Funkce více reálných proměnných, vektorová funkce]
		Nechť $M \subset ®R^n$. Funkcí více reálných proměnných rozumíme zobrazení $f: M \rightarrow ®R$.

		Vektorovou funkcí více reálných proměnných rozumíme zobrazení $f: M \rightarrow ®R^m$, kde $m \in ®N$.
	\end{definice}

	\begin{definice}[Eukleidovská vzdálenost]
		Pro $[x_1, …, x_n], [y_1, …, y_n] \in ®R^n$ definujeme eukleidovskou vzdálenost (metriku) jako
		$$ |x - y| = \sqrt{\sum_{i=1}^n \(x_i - y_i\)^2}. $$
	\end{definice}

	\begin{definice}[Koule, prstencové okolí]
		$B(c, r) = \{x \in ®R^n | |x-c| < r\}$. $P(c, r) = B(c, r)\setminus \{c\}$.
	\end{definice}

	\begin{definice}[Limita funkce]
		Nechť $F: G \rightarrow ®R$, kde $G \subseteq ®R^n$ je otevřená. Řekneme, že $f$ má v bodě $a \in G$ limitu rovnou $A \in ®R^*$, jestliže platí
		$$ \forall \epsilon > 0\ \exists\delta > 0\ \forall x \in P(a, \delta): f(x) \in B(A, \epsilon). $$
		Značíme $\lim_{x \rightarrow ∞} f(x) = A$.
	\end{definice}

	\begin{definice}[Spojitost]
		Řekneme, že $f$ je spojitá v $a$, jestliže $\lim_{x \rightarrow a} f(x) = f(a)$.
	\end{definice}

	\begin{definice}[Spojitost a limita vektorové funkce]
		Spojitost a limitu vektorové funkce definujeme po složkách.
	\end{definice}

	\begin{poznamka}
		Zřejmě platí aritmetika limit, věta o dvou policajtech a věta o spojitosti složené funkce.
	\end{poznamka}

	\begin{definice}[Limita posloupnosti bodů]
		$$ x_j \in ®R^n, \lim_{j \rightarrow ∞} x_j = a \in ®R^n \Leftrightarrow \forall \epsilon > 0\ \exists j_0 \forall j ≥ j_0: |x_j - a|<\epsilon. $$
	\end{definice}

	\begin{poznamka}
		Následující větu lze dokázat analogicky věty výše.
	\end{poznamka}

	\begin{veta}[Heine]
		Nechť $G \subset ®R^n$ otevřená, $a \in G$, $A \in ®R^*$ a $f: G \rightarrow ®R$. Pak je ekvivalentní

		\begin{itemize}
			\item $\lim_{x \rightarrow a} f(x) = A$.
			\item $\forall$ posloupnost $\{x_j\}_{j=1}^∞$ splňující $x_j \in G \setminus \{a\}$, $\lim_{j \rightarrow ∞} x_j = a$ platí $\lim_{j \rightarrow ∞} f(x_j) = A$.
		\end{itemize}
	\end{veta}

	\subsection{Parciální derivace a totální diferenciál}
	\begin{definice}[Parciální derivace]
		Nechť $G \subset ®R^n$ je otevřená, $i \in [n]$, $f: G \rightarrow ®R$ a $x \in ®G$. Parciální derivací funkce $f$ v bodě $x$ podle $i$-té proměnné nazveme
		$$ \frac{\partial f}{\partial x_i}(x) = \lim_{t \rightarrow 0} \frac{f(x_1, …, x_i + t, …, x_n) - f(x_1, …, x_n)}{t} = \lim_{t \rightarrow 0} \frac{f(x+t·e_i) - f(x)}{t}, $$
		pokud tato limita existuje.
	\end{definice}

	\begin{definice}[Extrémy]
		Nechť $M \subset ®R^n$, $f: M \rightarrow ®R$ a $x_0 \in M$. Řekneme, že $f$ nabývá v bodě $x_0$ svého minima (resp. lokálního minima, resp. maxima, lokálního maxima) vzhledem k $M$, jestliže $\forall x \in M: f(x) ≥ f(x_0)$ (resp. $\exists \delta > 0\ \forall x \in B(x_0, \delta)$, resp. $f(x) ≤ f(x_0)$).
	\end{definice}

% 14. 10. 2021

	\begin{veta}[Nutná podmínka existence extrému]
		Nechť $G \subset ®R^n$ je otevřená, $i \in [n]$, $a \in G$ a $f: G \rightarrow ®R$. Má-li $f$ v bodě $a$ lokální minimum (maximum) a existuje-li $\frac{\partial f}{\partial x_i}(a)$, pak $\frac{\partial f}{\partial x_i}(a) = 0$.

		\begin{dukazin}
			Položme $h(t) = f(a + t·e_i)$. Pak $h$ je definováno na okolí 0. $f$ má v $a$ extrém, tedy $h$ má v 0 extrém. Dále
			$$ h'(0) = \lim_{t \rightarrow 0} \frac{h(t) - h(0)}{t} = \lim_{t \rightarrow ∞} \frac{f(a + t·e_i) - f(a)}{t} = \frac{\partial f}{\partial x_i}(a). $$
			Podle Fermatovy věty je $h'(0) = 0$.
		\end{dukazin}
	\end{veta}

	\begin{definice}[Derivace ve směru]
		Nechť $G \subset ®R^n$ je otevřená, $f: G \rightarrow ®R$, $x \in G$ a $0 ≠ v \in ®R^n$. Derivací funkce $f$ v bodě $x \in G$ ve směru $v$ nazveme
		$$ \frac{\partial f}{\partial v}(x) = \lim_{t \rightarrow 0} \frac{f(x + t·v) - f(x)}{t}, $$
		pokud limita existuje.
	\end{definice}

	\begin{definice}[Totální diferenciál]
		Nechť $G$ je otevřená, $f: G \rightarrow ®R$ a $a \in G$. Řekneme, že lineární zobrazení $L: ®R^n \rightarrow ®R$ je totální diferenciál funkce $f$ v bodě $a$, pokud $\lim_{h \rightarrow 0} \frac{f(a + h) - f(n) - L(h)}{|h|} = 0$.

		Značíme $D_f(a)$ a hodnotu v bodě $h \in ®R^n$ značíme $D_f(a)(h)$.
	\end{definice}

	\begin{poznamka}
		Lineární zobrazení $L: ®R^n \rightarrow ®R$ lze reprezentovat jako $L(h) = A_ih_1 + … + A_nh_n$.

		Ekvivalentně lze definovat jako $\lim_{x \rightarrow a} \frac{f(x) - f(a) - L(x-a)}{|x-a|} = 0$.

		Geometrický význam je, že lineární funkce $f(a) + L(x - a)$ je velmi blízko původní funkce $f(x)$ na okolí $a$.
	\end{poznamka}

	\begin{veta}[O tvaru totálního diferenciálu]
		Nechť $G$ je otevřené, $a \in G$ a $f: G \rightarrow ®R$. Nechť existuje totální diferenciál $f$ v bodě $a$. Pak existují parciální derivace $\frac{\partial f}{\partial x_i}(a)$ a pro všechna $h \in ®R^n$ platí $D_f(a)(h) = \frac{\partial f}{\partial x_1}h_1 + … + \frac{\partial f}{\partial x_n}h_n$. Navíc pro $¦o ≠ v \in ®R^n$ platí $\frac{\partial f}{\partial v}(a) = D_f(a)(v)$.

		\begin{dukazin}
			Víme $\lim_{h \rightarrow 0} \frac{f(a + h) - f(a) - L(h)}{|h|} = 0$. Speciálně pro $h = t·e_i$:
			$$ 0 = \lim_{t \rightarrow 0} \frac{f(a + t·e_i) - f(a) - L(t·e_i)}{t} = \lim_{t \rightarrow 0} \frac{f(a + t·e_i) - f(a) - A_i·t}{t} = \frac{\partial f}{x_i}(a) - A_i. $$
			Tj. $\frac{\partial f}{x_i}(a) = A_i$. Obdobně pro $v$.
		\end{dukazin}
	\end{veta}

% 20. 10. 2021

TODO!!!

% 21. 10. 2021

TODO!!!

% 27. 10. 2021

	\begin{veta}[O přírůstku funkce]
		Nechť $G \subset ®R^n$ je otevřená a $f: G \rightarrow ®R$ má totální diferenciál v každém bodě $G$. Nechť $a, b \in G$ a nechť úsečka $L$ spojující $a, b$ je obsažena v $G$, tj. $L = \{(1 - t)·a + t·b | t \in [0, 1]\} \subset G$. Pak existuje $\zeta \in L$ tak, že $f(b) - f(a) = Df(\zeta)·(b - a)$.

		\begin{dukazin}
			Položme $F(t) = f(a + t(b - a))$. Podle Lagrangeovy věty $\exists \zeta_2 \in (0, 1)$ tak, že $f(b) - f(a) = F(1) - F(0) = F'(\zeta_2)$. Položme $\zeta = a + \zeta_2(b - a)$.

			Podle řetízkového pravidla $\frac{\partial F}{\partial t}(\zeta) = \sum_{j=1}^n \frac{\partial f}{\partial y_j}(\zeta) (b_j - a_j) = Df(\zeta)(b - a)$.
		\end{dukazin}
	\end{veta}

	\subsection{Parciální derivace vyšších řádů}
	\begin{definice}
		Nechť $f$ má na otevřené množině $G \subset ®R^n$ parciální derivaci
		$$ \frac{\partial f}{\partial x_i}, i \in [n], $$
		pak definujeme pro $a \in G$ a $j \in [n]$ druhou parciální derivaci
		$$ \frac{\partial^2 f}{\partial x_j \partial x_i}(a) = \frac{\partial}{\partial x_j}\(\frac{\partial f}{\partial x_i}\)(a), i ≠ j, $$
		$$ \frac{\partial^2 f}{\partial x_i^2}(a) = \frac{\partial}{\partial x_i}\(\frac{\partial f}{\partial x_i}\)(a), i = j. $$
		
		Obdobně definujeme derivace vyšších řádů.
	\end{definice}

	\begin{definice}[$C^k(®R)$]
		Nechť $G \subset ®R^n$ je otevřená a $f: G \rightarrow ®R$. Řekneme, že $f \in C^1(G) = C^1(G, ®R)$, pokud existují parciální derivace $\frac{\partial f}{\partial x_i}$, $i \in [n]$, a jsou to spojité funkce.

		Řekneme, že $f \in C^k(G) = C^k(G, ®R)$, $k \in ®N$, pokud existují všechny parciální derivace $f$ až do řádu $k$ včetně a jsou to spojité funkce.
	\end{definice}

	\begin{dusledek}
		Nechť $G \subset ®R^n$ je otevřená. Z věty dříve dostáváme, že je-li $f \in C^1(G)$, pak existuje totální diferenciál $f$ na $G$.
	\end{dusledek}

	\begin{veta}[Záměnnost parciálních derivací]
		Nechť $G \subset ®R^n$ je otevřená, $a \in G$ a $f \in C^2(G, ®R)$. Pak
		$$ \frac{\partial^2 f}{\partial x_j \partial x_i}(a) = \frac{\partial^2 f}{\partial x_i \partial x_j}(a). $$

		\begin{dukazin}
			SLÚNO $n = 2$. Vezměme $t$ dost malé, aby $B_{\max}([a_1, a_2], t) \subset G$. Položme $W(t) = \frac{f(a_1 + t, a_2 + t) - f(a_1, a_2 + t) - f(a_1 + t, a_2) + f(a_1, a_2)}{t^2}$. Položme $\phi(x) = f(x, a_2 + t) - f(x, a_2)$. Pak $W(t) = \frac{1}{t^2}(\phi(a_1 + t) - \phi(a_1))$.

			$\phi$ je spojitá a $\exists \phi'$. Lagrange: $\exists c_1 \in (a_1, a_1 + t)$:
			$$ \frac{1}{t^2}·\phi'(c_1)·(a_1 + t - a_1) = \frac{1}{t} \(\frac{\partial f}{\partial x}(c_1, a_2 + t) - \frac{\partial f}{\partial x}(c_1, a_2)\) = \frac{1}{t}(h(a_2 + t) - h(a_2)), $$
			$h(a) = \frac{\partial f}{\partial x}(c_1, z)$ je spojitá a derivovatelná, tedy použijeme Lagrange:
			$$ = \frac{1}{t}·h'(c_2)·(a_2 + t - a_2) = \frac{\partial^2 f}{\partial y\partial x}(c_1, c_2) \leftarrow \frac{\partial^2 f}{\partial y\partial x}(a_1, a_2). $$
			($f$ má spojité druhé derivace, tedy můžeme prohodit $f$ a limitu.) Totéž provedeme pro zaměněné souřadnice.
		\end{dukazin}
	\end{veta}

	\begin{definice}[Hessova matice]
		Nechť $G \subset ®R^n$ je otevřená a $a \in G$. Nechť $f \in C^2(G)$. Definujeme Hessovu matici $f$ jako
		$$ D^2f(a) = \begin{pmatrix} \frac{partial^2 f}{\partial x_1^2}(a) & … & \frac{partial^2 f}{\partial x_1 \partial x_n}(a) \\ \vdots & \ddots & \vdots \\ \frac{partial^2 f}{\partial x_n \partial x_1}(a) & … & \frac{partial^2 f}{\partial x_n^2}(a) \end{pmatrix}. $$

		Podle předchozí věty je matice symetrická, a proto můžeme pracovat s následující kvadratickou formou
		$$ D^2f(a)(¦u, ¦v) = u^TD^2f(a)·v, \forall ¦u, ¦v \in ®R^2. $$
	\end{definice}

	\begin{definice}
		Nechť $G \subset ®R^n$ je otevřená a $a \in G$. Nechť $f \in C^2(G)$. Pak definujeme Taylorův polynom stupně 2 jako
		$$ T_2^{f, a}(x) := f(a) + Df(a)(x - a) + \frac{1}{2}D^2f(a)(x - a, x - a). $$
	\end{definice}

	\begin{veta}[Taylorova věta pro druhý řád]
		Nechť $f: ®R^n \rightarrow ®R$ je třídy $C^2$ na okolí bodu $a \in ®R^n$. Pak
		$$ \lim_{x \rightarrow 0} \frac{f(x) - T_2^{f, a}(x)}{|x - a|^2} = 0. $$

		\begin{dukazin}
			Bez důkazu.
		\end{dukazin}
	\end{veta}

% 03. 11. 2021

	\begin{poznamka}
		Lze definovat i Taylorovy polynomy řádu $k$ pomocí $k$-tých parciálních derivací.
	\end{poznamka}

	\begin{veta}[O pozitivně definitní kvadratické formě]
		Nechť $Q: ®R^n \rightarrow ®R$ je pozitivně definitní kvadratická forma. Potom
		$$ \exists \epsilon > 0\ \forall h \in ®R^n: Q(h, h) ≥ \epsilon ||h||^2. $$

		\begin{dukazin}
			Funkce $A(h) = Q(h, h) = \sum_{i, j = 1}^n a_{ij}h_jh_i$ je spojitá. Množina $M \{h \in ®R^n | ||h|| = 1\}$ je omezená a uzavřená, tedy kompaktní. Funkce $A(h)$ tedy nabývá na $M$ svého minima v bodě $h_0$. Označme $\epsilon = Q(h_0, h_0)>0$.

			Nyní
			$$ \forall h \in ®R^n: Q(h, h) = Q\(\frac{h}{||h||}||h||, \frac{h}{||h||}||h||\) = ||h||^2 Q\(\frac{h}{||h||}, \frac{h}{||h||}\) ≥ ||h||^2 Q\(h_0, h_0\) = ||h||^2 \epsilon. $$
		\end{dukazin}
	\end{veta}

	\begin{veta}[Postačující podmínky pro lokální extrém]
		Nechť $G \subset ®R^n$ je otevřená množina, $a \in G$ a nechť $f \in ©C^2(G)$. Nechť $Df(a) = 0$ (tedy je bod podezřelý na lokální extrém).

		\begin{enumerate}
			\item Je-li $D^2f(a)$ pozitivně definitní, pak $a$ je bod lokálního minima.
			\item Je-li $D^2f(a)$ negativně definitní, pak $a$ je bod lokálního maxima.
			\item Je-li $D^2f(a)$ nedefinitní, pak v $a$ nemá extrém.
		\end{enumerate}

		\begin{dukazin}
			1) Z předchozí věty víme, že
			$$ \exists \epsilon > 0\ \forall h \in ®R^n: D^2f(a)(h, h) ≥ \epsilon ||h||^2. $$
			Z té ještě předchozejší víme, že
			$$ \lim_{x \rightarrow a} \frac{f(x) - f(a) - Df(a)(x - a) - \frac{1}{2}D^2f(a)(x - a, x - a)}{||x - a||^2} = 0. $$
			K zadanému $\epsilon > 0$ nalezneme $\delta > 0$:
			$$ \forall x \in P(a, \delta): \frac{f(x) - f(a) - Df(a)(x - a) - \frac{1}{2}D^2 f(a)(x - a, x - a)}{||x - a||^2} > -\frac{1}{4}\epsilon. $$

			Odtud $f(x) - f(a) - \frac{1}{2}D^2f(a)(x - a, x - a) > -\frac{1}{4} \epsilon ||x - a||^2 \implies$
			$$ f(x) > f(a) + \frac{1}{2}D^2f(a)(x - a, x - a) - \frac{1}{4}\epsilon ||x - a||^2 ≥ $$
			$$ ≥ f(a) + \frac{1}{2}\epsilon ||x - a||^2 - \frac{1}{4}\epsilon ||x - a||^2 > f(a). $$

			2) obdobně.

			3) Tedy existují $h_1, h_2 \in ®R^n$ tak, že $D^2f(a)(h_1, h_1)>0$ a $D^2f(a)(h_2, h_2) < 0$. Uvažme funkci $\phi(t) = f(a + t·h_1)$, pak $\phi'(t) = \sum_{i=1}^n \frac{\partial f}{\partial x_1}\(a + t·h_1\)·\(h_1\)_i = Df(a + t·h_1)·h_1$. $\phi'(0) = Df(a)h_1 = 0$.

			Dále $\phi''(t) = D^2f(a + t·h_1)(h_1, h_1)$, tedy $\phi''(0) = D^2f(a)(h_1, h_1) > 0$. Tedy $\phi$ má v $t = 0$ lokální minimum, tj. $f$ nemá v bodě $a$ lokální maximum. Analogicky pro $h_2$, z čehož dostaneme, že $f$ nemá v $a$ ani lokální minimum.
		\end{dukazin}
	\end{veta}

	\subsection{Implicitní funkce a vázané extrémy}
	\begin{veta}[O implicitní funkci]
		Nechť $p \in ®N$, $G \subset ®R^{n+1}$ je otevřená množina, $F: G \rightarrow ®R$, $\tilde{x} \in ®R^n$, $\tilde{y} \in ®R$, $[\tilde{x}, \tilde{y}] \in G$ a nechť platí

		\begin{enumerate}
			\item $F \in ©C^p(G)$,
			\item $F(\tilde{x}, \tilde{y}) = 0$,
			\item $\frac{\partial F}{\partial y}(\tilde{x}, \tilde{y}) ≠ 0$,
		\end{enumerate}

		pak existuje okolí $U \subset ®R^n$ bodu $\tilde{x}$ a okolí $V \subset ®R$ bodu $\tilde{y}$ tak, že
		$$ \forall x \in U\ \exists! y \in V: F(x, y) = 0. $$
		Píšeme-li $y = \phi(x)$, pak $\phi \in ©C^p(U)$ a platí
		$$ \frac{\partial \phi}{\partial x_j}(x) = - \frac{\frac{\partial F}{\partial x_j}(x, \phi(x))}{\frac{\partial F}{\partial y}(x, \phi(x))}, \forall x \in U\ \forall j \in [n]. $$

% 04. 11. 2021

		\begin{dukazin}
			4 kroky: a) $\exists \phi$, b) $\phi$ je spojitá, c) $\phi \in ©C^1$, d) $\phi \in ©C^p$.

			a) BÚNO $\frac{\partial F}{\partial y}(\tilde{x}, \tilde{y}) = 0$. $F$ je $©C^2$, a tedy $\exists \delta_1 > 0\ \exists \zeta_1 > 0$, tak $\forall x \in B(\tilde{x}, \delta_1)\ \forall y \in B(\tilde{y}, \zeta_1)$, $\frac{\partial F}{\partial y}(x, y) > 0$. $\forall B(\tilde{y}, \zeta_1): \frac{\partial F}{\partial y}(\tilde{x}, y) > 0$ $\implies$ funkce $y \mapsto F(\tilde{x}, y)$ je rostoucí, tj. $F(\tilde{x}, \tilde{y} + \zeta_1) > 0$, $F(\tilde{x}, \tilde{y} - \zeta_1) < 0$. Nalezneme $\delta_2 < \delta_1$ tak, že $F(x, \tilde{y} + \zeta_1) > 0$, $F(x, \tilde{y} - \zeta_1) < 0$, $\forall x \in B(\tilde{x}, \delta_2)$. Položme $U = B(\tilde{x}, \delta_2)$ a $V = B(\tilde{y}, \zeta_1)$.

			Nechť $x \in B(\tilde{x}, \delta_2)$ je libovolné pevné. Víme, že $\frac{\partial F}{\partial y}(x, y) > 0$, tedy $y \mapsto F(x, y)$ je rostoucí a spojitá, tedy podle Darbouxovy věty (o nabývání mezihodnot) $\exists! y \in (\tilde{y} - \zeta_1, \tilde{y} + \zeta 1)$ tak, že $F(x, y) = 0$. Označme $y = \phi(x)$.

			b) Nechť $\epsilon > 0$, $\epsilon < \zeta_1$. Mohu použít větu část $a)$ na $F$ a $G^* = U \times (\tilde{y} - \epsilon, \tilde{y} + \epsilon)$. Dostaneme $\exists U^*$ okolí $\tilde{x}$ a $V^*$ okolí $\tilde{y}$, že $\forall x \in U^* \exists! y \in V^* F(x, y) = 0$. Speciálně $\phi(U^*) \subset V^* \subset (\tilde{y} - \epsilon, \tilde{y} + \epsilon)$. Tedy $\phi$ je spojité.

			c) Chceme ukázat, že $\phi$ má totální diferenciál v $\tilde{x}$, tedy
			$$ \forall\epsilon > 0\ \exists \delta > 0\ \forall x \in B(\tilde{x}, \delta) \subset ®R^n: |\phi(\tilde{x} + h) - \phi(\tilde{x}) - \sum_{i=1}^n - \frac{\frac{\partial F}{\partial x_j}(x, \phi(x))}{\frac{\partial F}{\partial y}(x, \phi(x))}h_i| < \epsilon \sum_{i=1}^n |h_i|. $$
			Zvolme $\epsilon > 0$, $\epsilon \frac{1}{\frac{\partial F}{\partial y}(\tilde{x}, \tilde{y})} < \frac{1}{2}$. Víme, že $F$ má totální diferenciál v $[\tilde{x}, \tilde{y}]$, tedy
			$$ \exists \delta > 0\ \forall h \in B(0, \delta) |F(\tilde{x} - \tilde{y}, \tilde{y} + h_{n+1}) - F(\tilde{x}, \tilde{y}) - \sum_{i=1}^n \frac{\partial F}{\partial x_i}(\tilde{x}, \tilde{y})h_i - TODO $$
			$$ |F(\tilde{x} + \tilde{h}, \phi(\tilde{x} + \tilde{h})) - F(\tilde{x}, \tilde{y}) - \sum_{i = 1}^n \frac{\partial F}{\partial x_i}(\tilde{x}, \tilde{y})·h_i - \frac{\partial F}{\partial y}(\tilde{x}, \tilde{y})·(\phi(\tilde{x} + \tilde{h}) - \phi(\tilde{x}))| ≤ \epsilon·(\sum_{i=1}^n |h_i| + |\phi(\tilde{x} + \tilde{h}) - \phi(\tilde{x})|). $$
			$$ |(\phi(\tilde{x} + \tilde{h}) - \phi(\tilde{x})) - \frac{\frac{\partial F}{\partial x_i}(\tilde{x}, \tilde{y})}{\frac{\partial F}{\partial y}(\tilde{x}, \tilde{y})}·h_i| ≤ \frac{\epsilon}{\frac{\partial F}{\partial y}(\tilde{x}, \tilde{y})}·(\sum_{i=1}^n |h_i| + |\phi(\tilde{x} + \tilde{h}) - \phi(\tilde{x})|). $$

			Tudíž stačí jen odhadnout $|\phi(\tilde{x} + \tilde{h}) - \phi'(\tilde{x})|$.
			$$ |\phi(\tilde{x} + \tilde{h}) - \phi'(\tilde{x})| ≤ |\phi(\tilde{x} + \tilde{h}) - \phi'(\tilde{x}) - \sum_{i=1}^n \frac{-\frac{\partial F}{\partial x_i}(\tilde{x}, \tilde{y})}{\frac{\partial F}{\partial y}(\tilde{x}, \tilde{y})}·h_i| + |\sum_{i=1}^n \frac{-\frac{\partial F}{\partial x_i}(\tilde{x}, \tilde{y})}{\frac{\partial F}{\partial y}(\tilde{x}, \tilde{y})}·h_i| ≤ $$
			$$ ≤ \frac{\epsilon}{\frac{\partial F}{\partial y}(\tilde{x}, \tilde{y})}\(\sum_{i=1}^n |h_i| + |\phi(\tilde{x} + \tilde{h}) - \phi'(\tilde{x})|\) + c_i \sum_{i=1}^n |h_i| ≤ $$
			$$ ≤ \frac{1}{2}\(\sum_{i=1}^n |h_i| + |\phi(\tilde{x} + \tilde{h}) - \phi'(\tilde{x})|\) + c_i \sum_{i=1}^n |h_i| \implies $$
			$$ |\phi(\tilde{x} + \tilde{h}) - \phi'(\tilde{x})| ≤ c_2 \sum_{i=1}^n |h_i|. $$
			Kombinací dosažených nerovností už dostaneme chtěnou nerovnost. Tedy $\frac{\partial \phi}{\partial x_j}(x) = - \frac{\frac{\partial F}{\partial x_i}(x, \phi(x))}{\frac{\partial F}{\partial y}(x, \phi(x))}$.

			d) $F \subset ©C^p \implies \phi \in ©C^p$. Indukcí: $p = 1$ víme. Dále nechť víme $\phi \in ©C^{p-1}$ a $F \in ©C^p$. Víme, že
			$$ \frac{\partial \phi}{\partial x_j}(x) = - \frac{\frac{\partial F}{\partial x_i}(x, \phi(x))}{\frac{\partial F}{\partial y}(x, \phi(x))}. $$
			Toto $(p-1)$krát zderivujeme. Tím na pravé straně dostaneme výraz, kde bude $F$ nanejvýš v $((1 + p-1) = p)$-té derivaci a $\phi$ bude nanejvýš v $(p-1)$-té derivaci (podle vzorce pro derivaci složené funkce).
		\end{dukazin}
	\end{veta}

% 10. 11. 2021

	\begin{veta}[O implicitních funkcích]
		Nechť $n, m \in ®N$, $p \in ®N$, $G \subset ®R^{n + m}$ otevřená, $F_j: G \rightarrow ®R$, $j \in [m]$, $\tilde{x} \in ®R^n$, $\tilde{y} \in ®R^m$, $[\tilde x, \tilde y] \in G$ a nechť platí
		
		\begin{itemize}
			\item $F_j \in ©C^p(G)$ pro $j \in [m]$,
			\item $F_j(\tilde x, \tilde y) = 0$, $\forall j \in [m]$,
			\item determinant $m \times m$ matic parciálních derivací $F_j$ je nenulový.
		\end{itemize}

		Pak existuje $U \subset ®R^n$ okolí $\tilde x$ a $V \subset ®R^m$ okolí $\tilde y$ tak, že
		$$ \forall x \in U\ \exists! y \in V, F_j(x, y) = 0\ \forall j \in [m], (y_j = \phi_j(x)) \implies \phi_j \in ©C^p(U), j \in [m]. $$
	\end{veta}

	\begin{veta}[Lagrangeova věta o vázaných extrémech]
		Nechť $G \subset ®R^n$ je otevřená množina, $m < n$, $f$, $g_1, …, g_m \in ©C^1(G)$ a mějme množinu $M = \{z \in ®R^n | g_1(z) = … = g_m(z) = 0\}$. Je-li $a \in M$ bodem lokálního extrému $f$ vzhledem k $M$ a vektory $(\frac{\partial g_1}{\partial z_1}(a), …, \frac{\partial g_1}{\partial z_n}(a))$, …, $(\frac{\partial g_m}{\partial z_1}(a), …, \frac{\partial g_m}{\partial z_n}(a))$ jsou lineárně nezávislé, pak existují čísla $x_1, …, x_m$ tak, že\
		$$ Df(a) + \lambda_1Dg_1(a) + … + \lambda_m·Dg_m(a) = 0. $$

		\begin{dukazin}
			Položme $k = n - m$, $®R^n = ®R^k \times ®R^m$. Víme, že $Dg_1(a), …, Dg_m(a)$ jsou LN $\implies$ $\exists$ $m$ lineárně nezávislých sloupců, BÚNO jsou to poslední sloupce. Podle věty o implicitních funkcích $\exists U$ okolí $\tilde x$ a $V$ okolí $\tilde y$ tak, že $\forall x \in U\ \exists!y \in V: g_j(x, y) = 0$, $j \in [m]$. Píšeme $y_j = \phi_j(x) \in ©C^1$, $j \in [m]$.

			Položme $\psi(x) = f(x_1, …, x_k, \phi_1(x_1, …, x_k), …) \in ©C^1$. Víme $f$ má extrém vzhledem k $M$ $\implies$ $\psi$ má extrém $\implies$ $\frac{\partial\psi}{\partial x_j}(x) = 0$, $j \in [k]$.
			$$ 0 = \frac{\partial \psi}{\partial x_j}(a) = \sum_{i=1}^k \frac{\partial f}{\partial z_i}a \frac{\partial(x_i)}{x_j} + \sum_{i=1}^m \frac{\partial f}{\partial z_{k+i}} (a) \frac{\partial \phi_i(\tilde x)}{\partial x_j} = 0 + …. $$

			Zderivováním $g_l(x_1, …, \phi…, …) = 0$, $l \in [m]$, dostaneme
			$$ \frac{\partial}{\partial z_j}g(x) = \frac{\partial g_l}{\partial z_j}(a)  + \sum_{i=1}^m \frac{\partial g_l}{\partial z_{k+i}}(a)·\frac{\partial \phi_i(\tilde x)}{\partial x_j} = 0. $$
			Označme si vektory $v_j = (0, …, 1, …, 0, \frac{\partial \phi_1}{\partial x_j}(\tilde x), …, \frac{\partial \phi_m}{\partial x_j}(\tilde x))$ (1 je na $j$-tém místě), $j \in [k]$. Označme $A = \LO\{v_1, …, v_k\}$. $\dim A = k$. Z toho plyne $A^\perp = n - k = m$. Derivace $\psi$ říká, že $Df(a) \in A^\perp$. Derivace $g$ říká, že $Dg_l(a) \in A^\perp$, $\forall l \in [m]$. Z předpokladu tvoří $Dg_l(a)$ bázi $A^\perp$ (jelikož jsou LN). Tj. $Df(a)$ lze napsat jako L kombinaci prvků báze, tj. $Dg_l(a)$.
		\end{dukazin}
	\end{veta}
	
% 11. 11. 2021

	\subsection{Regulární zobrazení}
	\begin{definice}[Difeomorfismus]
		Nechť $G \subset ®R^n$ je otevřené a $f: G \rightarrow ®R^n$. Řekneme, že $f$ je difeomorfismus na $G$, jestliže je $f$ prostá na $G$, $U = f(G)$ je otevřená v $®R^n$, $f \in C^1(G)$ a $f^{-1} \in C^1(U)$/
	\end{definice}

	\begin{definice}[Regulární zobrazení]
		Nechť $G \subset ®R^n$ je otevřená a $f: G \rightarrow ®R^n$. Řekneme, že $f$ je regulární zobrazení, jestliže $f \in ©C^1(G)$ a pro každé $a \in G$ a pro každé $a \in G$ platí $J_f(a) ≠ 0$.
	\end{definice}

	\begin{veta}[O lokálním difeomorfismu]
		Nechť $G \subset ®R^n$ je otevřená a $f: G \rightarrow ®R^n$ je třídy $©C^1$. Nechť pro $a \in G$ platí $J_f(a) ≠ 0$. Pak existuje $V \subset G$ okolí $a$ takové, že $f|_V$ je difeomorfismus na $V$.

		\begin{dukazin}
			Definujeme $\Omega = ®R^n \times G \subset ®R^{2n}$ a $F: \Omega \rightarrow ®R^n$, $F(y, x) = f(x) - y \in ®C^1$. Označme $b = f(a)$, pak $F(b, a) = f(a) - b = 0$. Dále Jacobián $F$ podle druhých $n$ souřadnic je roven $J_f(a) ≠ 0$. Podle věty o implicitních funkcích existuje $U_1$ okolí bodu $b$ a $V_1$ okolí bodu $a$ v $®R^n$, že $\forall y \in U_1\ \exists! x \in V_1: F(y, x) = 0$. Tj. při označení $x = \phi(y)$ je $\phi \in ©C^1\(U_1\)$ a $0 = f(x) - y = f(\phi(y)) - y$ $\implies$ $\phi = b^{-1} \in ©C^1$. (Na $A = V_1 \cap f^{-1}(U_1)$, což je otevřená množina jako průnik otevřené a vzoru otevřené při spojitém zobrazení. Nyní $f|_A$ je difeomorfismus a zobrazení $A$ na otevřenou $U_1$.)
		\end{dukazin}
	\end{veta}

\section{Metrické prostory vol. II}
	\subsection{Více o kompaktních a úplných prostorech}
		\begin{definice}[Kompaktní prostor podruhé]
			$(P, \rho)$ MP a $K \subset P$. Řekneme, že $K$ je kompaktní, jestliže z každé posloupnosti bodů z $K$ lze vybrat konvergentní podposloupnost s limitou v $K$.
		\end{definice}

		\begin{definice}[$\epsilon$-síť, totálně omezený]
			Nechť $(P, \rho)$ je metrický prostor. Nechť $\epsilon > 0$ a $H \subset P$. Řekneme, že $H$ je $\epsilon$-síť prostoru $P$, pokud $P \subset \bigcup_{x \in H}B(x, \epsilon)$.

			Řekneme, že $P$ je totálně omezený, pokud $\forall \epsilon > 0\ \exists$ konečná $\epsilon$-síť prostoru $P$.
		\end{definice}

		\begin{veta}[Omezenost a totální omezenost]
			Nechť $(P, \rho)$ je totálně omezený metrický prostor. Potom je $P$ omezený.

			\begin{dukazin}
				$P$ je TO, a tedy existuje konečná $1$-síť $x_1, …, x_n$. Označme $d = \max\{\rho(x_i, x_j) | i, j \in [n]\}$. Nechť $x, y \in P$, pak $\exists i, j \in [n]: x \in B(x_i, 1), y \in B(x_j, 1)$. Nyní
				$$ \rho(x, y) ≤ \rho(x, x_i) + \rho(x_i, x_j) + \rho(x_j, y) < 1 + d + 1. $$
				Volme $x_0$ libovolně, pak $P \subset B(x_0, d + 2)$.
			\end{dukazin}
		\end{veta}

% 18. 11. 2021

		\begin{veta}[Kompaktnost a totální omezenost]
			Nechť $(P, \rho)$ je kompaktní metrický prostor. Potom je $P$ totálně omezený.

			\begin{dukazin}
				Sporem: Nechť $\exists \epsilon > 0\ \forall x_1, …, x_n \in P: P \not\subset \bigcup_{i=1}^n B(x_i, \epsilon)$. Zvolme $x_1 \in P$ libovolně. Víme $P \not\subset B(x_1, \epsilon)$, tedy $\exists x_2: \rho(x_2, x_1) ≥ \epsilon$. Indukcí: Mějme $x_1, …, x_{n-1}$ tak, že $\rho(x_i, x_j) ≥ \epsilon\ \forall i ≠ j$. Víme $P \not\subset \bigcup_{i=1}^n B(x_i, \epsilon)$, tedy $\exists x_n \rho(x_n, x_i) ≥ \epsilon \forall i \in [n-1]$. Nakonec máme posloupnost $\{x_n\}_{n=1}^∞$.

				Z definice kompaktnosti $\exists x_{n_k} \rightarrow x \in P$. Ale toto není možné, protože $\rho(x_i, x_j) ≥ \epsilon \forall i ≠ j$.
			\end{dukazin}
		\end{veta}
		
		\begin{veta}[Kompaktnost a otevřené pokrytí]
			Metrický prostor $(P, \rho)$ je kompaktní právě tehdy, když z každého otevřeného pokrytí lze vybrat konečné podpokrytí. Tedy platí (pro libovolnou indexovou množinu a $G_\alpha$ otevřené)
			$$ P \subset \bigcup_{\alpha \in A} G_\alpha \implies \exists \text{konečná } A_0 \subset A: P \subset \bigcup_{\alpha A_0} G_\alpha. $$

			\begin{dukazin}
				„$\implies$“: Tvrdím, že $\exists m \in ®N$ tak, že $\forall x \in P\ \alpha \in A: x \in B(x, \frac{1}{m}) \subset G_\alpha$. To dokážeme sporem: Nechť $\exists x_m \in P\ \forall \alpha \in A: B(x_m, \frac{1}{m}) \not\subset G_\alpha$. $P$ je kompaktní, tedy $\exists x_{m_k} \rightarrow x$. Z otevřeného pokrytí $\exists \alpha \in A: x \in G_\alpha$, $G_\alpha$.

				$G_\alpha$ je otevřená, tedy $\exists \delta > 0: B(x, \delta) \subset G_\alpha$. Zvolme $k$, aby $\frac{1}{m_k} \in B(x, \frac{\delta}{2})$. Nyní $\forall y \in B(x_{m_k}, \frac{1}{m_k})$ platí
				$$ \rho(x, y) ≤ \rho(x, x_{m_k}) + \rho(x_{m_k}, y) < \frac{\delta}{2} + \frac{1}{m_k} < \delta \implies y \in B(x, \delta) \implies B(x_{m_k}, \frac{1}{m_k}) \subset G_\alpha, \text{\lightning}. $$

				Takže tvrzení výše platí. $(P, \rho)$ je kompaktní, tedy podle věty 11.2 (ve výuce) je totálně omezený. Tedy pro naše $m \in ®N\ \exists$ konečná $\frac{1}{n}$ síť $x_1, …, x_n$. Nyní z toho tvrzení výše $\forall j \in [n]\ \exists G_{\alpha_j}: B(x_j, \frac{1}{m}) \subset G_{\alpha_j}$. Nyní $P \subset \bigcup_{j=1}^n B(x_j, \frac{1}{n}) \subset \bigcup_{j=1}^n G_{\alpha_j}$.

				„$\Leftarrow$“: Nechť $\{x_n\} \in P$. Chceme $x_{n_k} \rightarrow x \in P$. Označme $D = \{x_n, n \in ®N\}$. Je-li $D$ konečná, pak se nějaké $x_n$ opakuje a je snadné vybrat konvergentní ($=$ konstantní) podposloupnost.

				Dále nechť $D$ je nekonečná. Máme 2 možnosti:
				$$ A \exists y \in P\ \forall r > 0: B(y, r) \cap D \text{ je nekonečná, nebo} $$
				$$ D \forall y \in P\ \exists r_y > 0: B(y, r_y) \cap D \text{ konečná}. $$
				$A$: $r = 1$: $\exists x_{n_1} \in B(y, 1) \cap D$, $r = \frac{1}{2}$, protože prvků v libovolné kouli je nekonečno, tak $\exists n_2 > n_1, x_{n_2} \in B(y, \frac{1}{2}) \cap D$. Dále pokračujeme indukcí a dostaneme $\{x_{n_k}\}_{k=1}^∞, x_{n_k} \stackrel{k \rightarrow ∞}{\rightarrow} y$.

				$B$: Víme $P \subset \bigcup_{y \in P} B(y, r_y)$ je otevřené pokrytí, tedy $\exists y_1, …, y_n: P \subset \bigcup_{i=1}^n B(y_1, r_{y_i})$. $D = D \cap P \subset \bigcup_{i=1}^n (B(y_i, r_{y_i}) \cap F)$. $D$ je nekonečné, ale podle $B$ je vpravo konečné sjednocení konečných množin, tedy konečná množina. \lightning.
			\end{dukazin}
		\end{veta}

		\begin{dusledek}[Borelova věta]
			Nechť $a, b \in ®K$, $a < b$ a $\{I_\alpha\}$ je systém otevřených intervalů. Pak
			$$ [a, b] \subset \bigcup_{\alpha \in A}I_\alpha \implies \exists A_0 \subset \text{ konečná } [a, b] \subset_{\alpha \in A_0} I_\alpha. $$
		\end{dusledek}

		\begin{dusledek}
			Spojitá funkce na $[a, b]$ je omezená.
		\end{dusledek}

		\begin{dusledek}
			$f$ je spojitá na $[a, b]$ $\implies$ $f$ je stejnoměrně spojitá na $[a, b]$.
		\end{dusledek}

% 24. 11. 2021

		\begin{veta}[Cantorova, o uzavřených množinách]
			Nechť $(P, \rho)$ je úplný metrický prostor a $F_n$ je posloupnost uzavřených neprázdných množin v $P$ tak, že $F_{n+1} \subset F_n$ a $\lim_{n \rightarrow ∞} \diam F_n = 0$. Pak $|\bigcap_{n=1}^∞ F_n| = 1$.

			\begin{dukazin}[Viz OM2/MetPro/MetPro.pdf Věta 6.1]
				Zvolme $x_n \in F_n$ libovolně. Tvrdíme, že $\{x_n\}_{n=1}^∞$ je cauchyovská. Nechť $\epsilon > 0$. $\exists n_0 \diam F_{n_0} < \epsilon$. Nyní
				$$ \forall m, n ≥ n_0: x_n \in F_n \subset F_{n_0}, x_m \in F_m \subset F_{n_0}: \rho(x_n, x_m) ≤ \diam F_{n_0} < \epsilon. $$
				$P$ je úplný, a tedy $x_n \rightarrow x \in P$. Nechť $j \in ®N$ je pevné a $n ≥ j$, pak $x_n \in F_n \subset F_j$ a $x_n \rightarrow x$, tedy ($F_j$ je uzavřené) $x \in F_j \forall j$, tedy $x \in \bigcap_{j=1}^∞ F_j$. Naopak pokud $x, y \in \bigcap_{j=1}^∞ F_j$, pak zvolíme $j$ tak, aby $\diam F_j < \rho(x, y)$, tedy buď $x \notin F_j$ nebo $y \notin F_j$.
			\end{dukazin}
		\end{veta}

		\begin{veta}[O totální omezenosti a úplnosti]
			Metrický prostor $(P, \rho)$ je kompaktní právě tehdy, když je totálně omezený a úplný.

			\begin{dukazin}
				$\implies$ už máme hotové z věty výše a věty Kompaktnost a totální omezenost. $\Leftarrow$: Nechť $\{x_n\}_{n=1}^∞ \subset P$, chceme $\exists x_{n_k} \rightarrow x$. $P$ je totálně omezený, tedy existuje konečná $1$-síť $P \subset \bigcup_{i=1}^{h_1} B(s_i, 1)$. $\{x_n\}$ je nekonečná $\implies$ $\exists$ kulička $B_1 = B(s_i, 1)$ tak, že $|\{x_n | x_n \in B_1\}| = +∞$. Zvolme $x_{n_1} \in B_1$.

				Dále indukcí: Mějme $B_1, …, B_{k-1}$ kuličky o poloměrech $1, …, \frac{1}{k-1}$ tak, že pro $A_{k-1} = B_1 \cap … \cap B_{k-1}$ platí $|\{x_n | x_n \in A_{k-1}\}| = +∞$, a mějme $n_1 < … < n_{k-1}$ tak, že $x_{n_j} \in A_j$, $\forall j \in [k-1]$. $P$ je totálně omezený $\implies$ $\exists$ konečná $\frac{1}{k}$-síť $A_{k-1} \subset P \subset \bigcup_{i=1}^{h_k} B(s_i, \frac{1}{k})$. V $A_{k-1}$ je nekonečně mnoho $x_n$, tedy $\exists B_k = B(s_i, \frac{1}{k})$, že pro $A_k = A_{k-1} \cap B_k$ platí $|\{x_n | x_n \in A_k\}| = +∞$. Dále zvolme $n_k$ tak, aby $n_k > n_{k-1}$ a $x_{n_k} \in A_k$.

				$x_{n_k}$ je cauchyovská, neboť pro $\epsilon > 0$ $\exists n_0: \frac{1}{n_0} < \epsilon$, nechť $k, l ≥ n_0$, pak $x_{n_k} \in A_k \subset A_{n_0}$ a $x_{n_l} \in A_l \subset A_{n_0}$, tedy jelikož $A_{n_0} \subset B_{n_0}$, $\rho(x_{n_k}, x_{n_l}) < \frac{2}{n_0} < 2\epsilon$. $P$ je úplný, tedy existuje $x$ tak, že $x_{n_k} \rightarrow x$.
			\end{dukazin}
		\end{veta}

		\begin{veta}[O zúplnění metrickéůo prostoru]
			Nechť $(P, \rho)$ je metrický prostor. Pak existuje úplný metrický prostor $(\tilde P, \tilde \rho)$ tak, že $P \subset \tilde P$ a $\forall x, y \in P$ platí $\rho(x, y) = \tilde \rho(x, y)$.

			\begin{dukazin}
				Bez důkazu.
			\end{dukazin}
		\end{veta}

		\begin{veta}[Arzela-Ascoli]
			Nechť $A \subset C([0, 1])$. Pak $\overline{A}$ je kompaktní právě tehdy, když jsou funkce z $A$ stejně omezené a stejně stejnoměrně spojité. Tedy pokud $\exists K > 0$:
			$$ \forall f \in A\ \forall x \in [0, 1]: |f(x)| ≤ K, $$
			$$ \forall \epsilon > 0\ \exists \delta > 0\ \forall x, y \in [0, 1]\ \forall f \in A: |x - y| < \delta \implies |f(x) - f(y)| < \epsilon. $$

			\begin{dukazin}
				$\implies$: $\overline{A}$ je kompaktní $\implies$ $\overline{A}$ je omezená $\implies$ $A$ je omezená $\subset B(0, K)$. Tedy $\forall f \in A\ \forall x \in [0, 1]: |f(x)| ≤ K$.

				$\overline{A}$ je kompaktní $\implies$ $\overline{A}$ je totálně omezená. Nechť $\epsilon > 0$ $\implies \exists$konečná $\epsilon$-síť $\overline{A} \subset \bigcup_{i = 1}^k B(f_i, \epsilon)$. Funkce $f_i$ je spojitá na $[0, 1]$ $\implies$ $f$ je stejnoměrně spojitá na $[0, 1]$. Tedy
				$$ \exists \delta_i>0\ \forall x, y: |x - y| < \delta_i \implies |f_i(x) - f_i(y)| < \epsilon. $$
				Položme $\delta = \min\{\delta_1, …, \delta_k\}$. Nechť $f \in A$, $x, y \in [0, 1]$: $|x - y| < \delta$. K tomuto $f \in A$ najdu $f_i$, aby $f \in B(f_i, \epsilon)$. Potom
				$$ |f(x) - f(y)| ≤ |f(x) - f_i(x)| + |f_i(x) - f_i(y)| + |f_i(y) - f(y)| < 2\epsilon + \epsilon = 3\epsilon $$
			\end{dukazin}
		\end{veta}


\end{document}
