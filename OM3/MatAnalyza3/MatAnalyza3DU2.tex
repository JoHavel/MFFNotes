\documentclass[12pt]{article}                   % Začátek dokumentu
\usepackage{../../MFFStyle}                     % Import stylu

\begin{document}

\begin{priklad}[2.1]
	Ukažte, že následující rovnice určuje v jistém okolí bodu $[0, 0]$ jednoznačně funkci $f$ proměnné $f(0) = 0$. Spočtěte $f'(0)$ a $f''(0)$. Napište rovnici tečny ke grafu funkce $f$ v bodě $[0, 0]$ a rozhodněte, zda je $f$ na okolí bodu $0$ konvexní nebo konkávní.
	$$ F(x, y) = \arcsin(x + y) + \arctan(x + y) + xy = 0. $$

	\begin{reseni}
		Ověříme podmínky věty o implicitní funkci (na otevřeném\footnote{Je otevřené, protože je vzorem $(-1, 1)$ při spojitém zobrazení $(x, y) \mapsto (x + y)^2$.} okolí $G$ bodu $[0, 0]$ daného $(x + y)^2 < 1$, aby měla $F$ derivace):
		$$ F(0, 0) = \arcsin(0) + \arctan(0) + 0 = 0 + 0 + 0 = 0.\ (\text{Tj.} f(0) = 0, \text{ až dokážeme, že existuje.}) $$
		$$ \frac{\partial F(x, y)}{\partial x} = \frac{1}{\sqrt{1 - (x + y)^2}} + \frac{1}{1 + (x + y)^2} + y \in C(G), $$
		$$ \frac{\partial F(x, y)}{\partial y} = \frac{1}{\sqrt{1 - (x + y)^2}} + \frac{1}{1 + (x + y)^2} + x \in C(G). $$
		$$ \frac{\partial F(x, y)}{\partial y}(0, 0) = \frac{1}{\sqrt{1 - 0}} + \frac{1}{1 + 0} + 0 = 1 + 1 = 2 ≠ 0. $$

		Podle věty o implicitní funkci tedy na nějakém okolí $U$ funkce $f$ existuje (svojí definicí je určena jednoznačně) a její derivace je ($f(0) = 0$)
		$$ \frac{\partial f}{\partial x}(0) = -\frac{\frac{\partial F(x, y)}{\partial x}}{\frac{\partial F(x, y)}{\partial y}}(x, f(x))(0) = \(-\frac{\frac{1}{\sqrt{1 - (x + f(x))^2}} + \frac{1}{1 + (x + f(x))^2} + f(x)}{\frac{1}{\sqrt{1 - (x + f(x))^2}} + \frac{1}{1 + (x + f(x))^2} + x}\)(0) = $$
		$$ = \(-1 - \frac{f(x) - x}{\frac{1}{\sqrt{1 - (x + f(x))^2}} + \frac{1}{1 + (x + f(x))^2} + f(x)}\)(0) = -1 - \frac{0}{2} = -1. $$

		Druhou derivaci spočítáme jako počítáme derivace běžně ($f(0) = 0$, $f'(0) = -1$):
		$$ \hspace{-1em}\frac{\partial^2 f}{\partial x^2}(0) = \frac{\frac{\partial f}{\partial x}}{\partial x}(0) = \frac{f'(x) - 1}{\frac{1}{\sqrt{1 - (x + f(x))^2}} + \frac{1}{1 + (x + f(x))^2} + f(x)} + \frac{f(x) - x}{\(\frac{1}{\sqrt{1 - (x + f(x))^2}} + \frac{1}{1 + (x + f(x))^2} + f(x)\)^2} · $$
		$$ · \(-\frac{1}{2}·\frac{2(x + f(x))(1 + f'(x))}{\sqrt{1 - (x + f(x))^2}^3} - \frac{2(x + f(x))(1 + f'(x))}{\(1 - (x + f(x))^2\)^3} + f'(x)\) = \frac{-2}{2} + \frac{0}{4}·(-1) = -1 $$

		Tečna má směrnici $f'(0) = -1$ a prochází $[0, f(0)] = [0, 0]$, tedy její rovnice je $y = x$. A jelikož je druhá derivace $f$ v $0$ záporná, tak musí existovat okolí bodu $0$, na kterém je $f$ konkávní.
	\end{reseni}
\end{priklad}

\begin{priklad}[2.2]
	Vyšetřete lokální extrémy (na $®R^3$) funkce
	$$ f(x, y, z) = 3y^4 + 3yx^2 - x^3 + z^2 - z. $$

	\begin{reseni}
		Jelikož je funkce všude na $®R^3$ definována a je zřejmě $C^∞(®R^3)$ (každý polynom je $C^∞$), tak lokální extrémy mohou být pouze tam, kde jsou všechny parciální derivace nulové.
		$$ \frac{\partial f}{\partial x} = 6yx - 3x^2 = 3x(2y - x), $$
		$$ \frac{\partial f}{\partial y} = 12y^3 + 3x^2, $$
		$$ \frac{\partial f}{\partial z} = 2z - 1. $$

		Tedy lokální extrémy mohou být jen tam, kde $z = \frac{1}{2}$. A tam, kde buď $x = 0$ (a tedy $y = 0$) nebo $x = 2y$ (a tedy $12y^3 = -12y^2$, tj. $y = 0$, $x = 0$ nebo $y = -1$, $x = -2$).

		V bodě $\[0, 0, \frac{1}{2}\]$ lokální extrém není, protože funkce $x \mapsto f\(x, 0, \frac{1}{2}\) = x^3 - \frac{1}{4}$ má v $0$ inflexní bod. (Pro $x < 0$ je menší než $\frac{1}{4}$ a pro $x > 0$ je větší.)

		V bodě $[-2, -1, \frac{1}{2}]$ spočítáme Hessovu matici:
		$$ \begin{pmatrix} 6y - 6x & 6x & 0 \\ 6x & 36y^2 & 0 \\ 0 & 0 & 2 \end{pmatrix}\(-2, -1, \frac{1}{2}\) = \begin{pmatrix} 6 & -12 & 0 \\ -12 & 36 & 0 \\ 0 & 0 & 2 \end{pmatrix}. $$
		Ta je zde pozitivně definitní, neboť hlavní minory jsou $6$, $72$, $144$ $> 0$. Tedy funkce má jediný lokální extrém a to minimum v bodě $\[-2, -1, \frac{1}{2}\]$.
	\end{reseni}
\end{priklad}

\pagebreak
\begin{priklad}[2.3]
	Najděte globální maximum a minimum funkce $f$ (pokud existují) na množině $M$. (Načrtněte $M$.)
	$$ f(x, y) = -y^2 + x^2 + \frac{4}{3} x^3, \qquad M = \{[x, y] \in ®R^2 | x^2 + y^2 ≤ 4 \land x ≤ 0\}. $$

	\begin{reseni}[Načrtněte $M$]
		$M$ je zřejmě („levý“) polokruh o poloměru $2$.
	\end{reseni}

	\begin{reseni}
		Jelikož je $M$ omezená ($||(x, y)|| ≤ 4$) a je průnikem dvou uzavřených množin,\footnote{Množina daná $x^2 + y^2 ≤ 4$ je vzor $[0, 4]$ při spojitém zobrazení $x^2 + y^2$, tedy je uzavřená. Množina $x ≤ 0$ je vzor $(-∞, 0]$ při spojitém zobrazení $(x, y) \mapsto x$, tedy také uzavřená.} tak je $M$ kompaktní, tedy $f$ na ní nabývá minima i maxima. Tedy nalezneme všechny body podezřelé z extrému a porovnáme funkční hodnoty v nich. Na podezřelé body máme 2 možnosti, buď jsou ve vnitřku $M$, nebo na hranici.

		Pokud je ve vnitřku, tak musí mít všechny derivace nulové (bod extrému na otevřené množině je bod lokálního extrému), tedy
		$$ \frac{\partial f}{\partial x} = 2x + 4x^2 = 0 = -2y = \frac{\partial f}{\partial y} \implies y = 0 \land x \in \{0, -\frac{1}{2}\}. $$
		Ale $[0, 0]$ je na hranici, ne ve vnitřku. Tedy z vnitřních bodů je podezřelý pouze $\[-\frac{1}{2}, 0\]$.

		Pro body na hranici platí $g(x, y) := (x^2 + y^2 - 4) · x = 0$. Zřejmě $f, g \in C^1(®R^2)$, jelikož jsou to polynomy. Jestliže je v $(x, y)$ maximum, potom podle Lagrangeových multiplikátorů je buď $\nabla g = (3x^2 + y^2 - 4, 2yx) = (0, 0)$, nebo $\nabla f = (2x + 4x^2, -2y)$ je násobkem $\nabla g$.

		V prvním případu je buď $y = 0$ (a $x = ±\frac{2}{\sqrt{3}}$) nebo $x = 0$ (a $y = ±2$). Z těchto bodů jsou na $\partial M$ pouze $[0, 2]$ a $[0, -2]$.

		V druhém případě je buď $\nabla f = (0, 0)$, tedy $y = 0$ a $x \in \{0, -\frac{1}{2}\}$, nebo $2yx = -2\lambda y$ a $3x^2 + y^2 - 4 = 2\lambda x + 4\lambda x^2$, tedy z prvního buď $y = 0$, ale to jsou jen $2$ body, tak je podezírejme oba, $[-2, 0]$ i $[0, 0]$, nebo $\lambda = -x$ a potom $3x^2 + y^2 - 4 = -2x^2 - 4x^3$, tj. $5x^2 + 4x^3 + y^2 - 4 = 0$. Hledáme body z hranice, tedy buď $x = 0$, pak dostaneme body $[0, 2]$ a $[0, -2]$, nebo $x^2 + y^2 - 4 = 0$, tudíž $4x^2 + 4x^3 = 0$, tedy body $[0, ±2]$ nebo $[-1, \sqrt{3}]$ a $[-1, -\sqrt{3}]$.

		Celkově podezřelé body jsou: $[-2, 0]$, $[0, 0]$, $[0, 2]$, $[0, -2]$, $[-1, \sqrt{3}]$, $[-1, -\sqrt{3}]$, $\[-\frac{1}{2}, 0\]$. V těch jsou funkční hodnoty po řadě $-\frac{20}{3}$, $0$, $-4$, $-4$, $-\frac{10}{3}$, $-\frac{10}{3}$, $\frac{1}{12}$. Tedy maximum je $\frac{1}{12}$ (v bodě $\[-\frac{1}{2}, 0\]$) a minimum je $-\frac{20}{3}$ (v bodě $[-2, 0]$).
	\end{reseni}
\end{priklad}

\end{document}
