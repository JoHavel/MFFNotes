\documentclass[12pt]{article}					% Začátek dokumentu
\usepackage{../../MFFStyle}							% Import stylu

\begin{document}
\begin{priklad}
	Each of two urns contains $m$ balls. Out of the total of the $2m$ balls, $m$ are white and $m$ are black. A ball is simultaneously selected from each urn and moved to the other urn, and the process is indefinitely repeated. What is the steady-state distribution of the number of white balls in each urn?

	Solve in two ways:
	\begin{itemize}
		\item by solving a list of equations, as we did in the tutorial for other Markov chains.
		\item by sampling: Put $m=5$, start with one urn completely white, the other completely black. Repeat many times the update step, observe the number of black balls in the first urn, repeat.
	\end{itemize}

	\begin{reseni}
		Označme si stavy podle počtu bílých kuliček v první urně: 0, 1, …, m. Do stavu $i$ se dostaneme s pravděpodobností $\frac{(i+1)^2}{m^2}$ ze stavu $i+1$ (musíme vybrat bílou z první urny, tj. $\frac{i+1}{m}$, a černou z druhé urny, tj. $\frac{i+1}{m}$), s pravděpodobností $\frac{(m-i+1)^2}{m^2}$ ze stavu $i-1$ (zase musíme vybrat bílou z druhé urny, tj. $\frac{m-(i-1)}{m}$, a černou z první, tj. $\frac{m-(i-1)}{m}$). Poslední možností, jak se do stavu $i$ dostat, je ze samotného stavu $i$, a to vytáhnutím černé a bílé zároveň (z libovolné urny), tedy s pravděpodobností $\frac{2i(m - i)}{m^2}$. Tedy pro steady-state distribuci máme rovnost
		$$ \frac{(m-i+1)^2}{m^2} p_{i-1} + \frac{2i(m - i)}{m^2} p_i + \frac{(i+1)^2}{m^2} p_{i+1} = p_i, $$
		$$ (m-i+1)^2 p_{i-1} + 2i(m - i)p_i + (i+1)^2 p_{i+1} = m^2p_i. $$
		$$ \begin{pmatrix}
			-m^2 & 1^2 & 0 & 0 & 0 & … \\
			m^2 & 2·1·(m - 1) - m^2 & 2^2 & 0 & 0 & … \\
			0 & (m - 1)^2 & 2·2·(m - 2) - m^2 & 3^2 & 0 & … \\
			\\
			0 & … & … & i^2 & … & …\\
			0 & … & (m - i + 1)^2 & 2i(m - i) - m^2 & (i+1)^2 & … \\
			0 & … & … & (m - i)^2 & … & …
		\end{pmatrix} $$

		Můžeme si všimnout, že když sečteme prvních $i+1$ řádků, tak se nám prvních $i$ sloupců sečte na nulu a vyjde nám $(i^2 + 2i(m - i) - m^2)p_i + (i+1)^2 p_{i+1} = 0$, tedy $p_{i+1} = \frac{(m-i)^2}{(i+1)^2}p_i$. Dosazením a upravením dostaneme $p_{i+1} = \(\frac{m! / (m-i-1)!}{(i+1)!}\)^2p_0$, tj. $p_i = \binom{m}{i}^2p_0$.

		Nyní už stačí sečíst pravděpodobnosti jednotlivých stavů, protože víme, že součet těchto pravděpodobností musí být 1, tedy $p_0·\sum_{i=0}^m \binom{m}{i}^2 = 1$. Číslo $\binom{m}{i}^2$ si můžeme rozdělit na $\binom{m}{i}·\binom{m}{m-i}$, tedy v předchozím součtu vybíráme $m$ prvků z $2m$ (jeden sčítanec je výběr přesně $i$ kuliček z první poloviny a zbylých $m-i$ kuliček z druhé), tedy $p_0 = \frac{1}{\binom{2m}{m}}$ a
		$$ p_i = \frac{\binom{m}{i}^2}{\binom{2m}{m}} $$
	\end{reseni}

	\begin{reseni}
		Naprogramoval jsem přesně danou situaci (se značením stavů jako v prvním řešení):
\begin{verbatim}
import random

class Urny:
    def __init__(self, m):
        self.first = 0
        self.second = m
        self.m = m

    def step(self):
        first_white = random.randint(1, self.m) <= self.first
        second_white = random.randint(1, self.m) <= self.second
        self.first -= first_white
        self.second += first_white
        self.second -= second_white
        self.first += second_white


if __name__ == '__main__':
    random.seed(42)
    m = 5
    stat = [0 for i in range(m + 1)]
    n = 100000
    for _ in range(n):
        urny = Urny(m)
        for _ in range(1000):
            urny.step()
        stat[urny.first] += 1
    print([it/n for it in stat])	
\end{verbatim}
		A vyšlo mi \verb|[0.00387, 0.09854, 0.39656, 0.39836, 0.09905, 0.00362]|, což je blízko exaktnímu řešení
		$$ \[\frac{1}{252}, \frac{25}{252}, \frac{100}{252}, \frac{100}{252}, \frac{25}{252}, \frac{1}{252}\]. $$
	\end{reseni}
\end{priklad}



\end{document}
