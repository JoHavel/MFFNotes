\documentclass[12pt]{article}					% Začátek dokumentu
\usepackage{../../MFFStyle}							% Import stylu

\begin{document}
\begin{priklad}[3.1]
	Nechť je $G = \(P = \{1, 2\}, A, u\)$ hra v normálním tvaru pro dva hráče, kde $A_1 = \{a, b, c\}$ a $A_2 = \{d, e, f\}$ s výplatní funkcí $u$ určenou Tabulkou 1.
	\begin{center}
		\begin{tabular}{c|c|c|c}
			    & $d$        & $e$        & $f$            \\ \hline
			$a$ & $(1, 1)$   & $(-1, -1)$ & $(0, 0)$       \\
			$b$ & $(-1, -1)$ & $(1, 1)$   & $(0, 0)$       \\
			$c$ & $(0, 0)$   & $(0, 0)$   & $(-1.1, -1.1)$
		\end{tabular}

		Tabulka 1: Hra z Příkladu 3.1.
	\end{center}

	Ukažte, že zde pravděpodobnostní rozdělení $p$ na $A$ s $p(a, d) = p(b, e) = p(c, f) = 1/3$ je hrubým korelovaným ekvilibriem v $G$ (CCE), ale není korelovaným ekvilibriem v $G$ (CE).

	\begin{dukazin}[CCE]
		Předpokládám definici CCE: $p$ je CCE, když $\sum_{a \in A} u_i(a) p(a) ≥ \sum_{a \in A} u_i(a_i'; a_{-i})p(a)$, $\forall i$, $\forall a_i' \in A_i$. Protože mi přijde jednodušší používat $u$ místo $C$, když $u$ už máme.

		CCE nám vychází (pro oba hráče, protože hra je symetrická):
		$$ \sum_{a \in A} u_i(a) p(a) = \frac{1}{3}·1 + \frac{1}{3}·1 + \frac{1}{3}·1.1 = \frac{1}{3}·0.9 = 0.3 $$
		To je levá strana nerovnosti. Pro pravou stranu si označme $\alpha_1, \alpha_2, \alpha_3$ pravděpodobnosti, že strategie $a_i' \in A_i$ hraje stav $a$, $b$ a $c$ (resp. $d$, $e$ a $f$). Pak pravou stranu spočítáme
		$$ \sum_{a \in A} u_i(a_i', a_{-i})p(a) = \frac{1}{3} \(\alpha_1·1 + \alpha_2·(-1) + \alpha_3·0\) + $$
		$$ + \frac{1}{3} \(\alpha_1·(-1) + \alpha_2·1 + \alpha_3 · 0\) + \frac{1}{3}\(\alpha_1·0 + \alpha_2·0 + \alpha_3·(-1.1)\) = -\frac{1.1}{3}p_3. $$
		To znamená, že ať zvolíme jakákoliv $0 ≤ \alpha_1, \alpha_2, \alpha_3 ≤ 1$, na pravé straně dostaneme nejvýše $0$, tedy méně než $0.3$, což splňuje definici CCE.
	\end{dukazin}

	\begin{dukazin}[$\neg$ CE]
		Zvolíme ($i = 1$, ale kromě značení je vše symetrické) $a_i = c$ a $a_i' = a$. Potom kdyby $p$ bylo CE, pak
		$$ 0·0 + 0·0 + (-1.1)·\frac{1}{3} = \sum_{a_{-i} \in A_{-1}} u_i(a_i; a_{i-1})p(a_i; a_{i-1}) \overset{\text{CE}}≥ $$
		$$ ≥ \sum_{a_{-i} \in A_{-1}} u_i(a_i'; a_{i-1})p(a_i; a_{i-1}) = 1·0 + (-1)·0 + 0·\frac{1}{3}, $$
		ale $-\frac{1.1}{3} \not ≥ 0$.
	\end{dukazin}
\end{priklad}

\begin{priklad}[3.2]
	Nechť je $G = \(P = \{1, 2\}, A, u\)$ hra v normálním tvaru pro dva hráče, kde $A_1 = \{U, D\}$ a $A_2 = \{L, R\}$ s výplatní funkcí $u$ určenou Tabulkou 2. Určete množinu všech korelovaných ekvilibrií v $G$.
	
	\begin{center}
		\begin{tabular}{c|c|c}
			    & $L$        & $R$        \\ \hline
			$U$ & $(4, 4)$   & $(1, 5)$   \\
			$D$ & $(5, 1)$   & $(0, 0)$
		\end{tabular}

		Tabulka 2: Hra z Příkladu 3.2.
	\end{center}

	\begin{reseni}
		Hledáme CE, tedy pravděpodobnostní rozložení na $A = A_1 \times A_2$, tedy si pravděpodobnosti označíme (odpovídá Tabulce 2):
		$$ \begin{pmatrix} p_{11} & p_{12} \\ p_{21} & p_{22} \end{pmatrix}. $$
		Naše $p$ musí splňovat podmínku CE, která je pro

		\begin{itemize}
			\item $a_i = U$, $a_i' = D$:
				$$ p_{11}·4 + p_{12}·1 ≥ p_{11}·5 + p_{12}·0 \implies p_{12} ≥ p_{11}; $$
			\item $a_i = D$, $a_i' = U$:
				$$ p_{21}·5 + p_{22}·0 ≥ p_{21}·4 + p_{22}·1 \implies p_{21} ≥ p_{22}; $$
			\item $a_i = L$, $a_i' = R$:
				$$ p_{11}·4 + p_{21}·1 ≥ p_{11}·5 + p_{21}·0 \implies p_{21} ≥ p_{11}; $$
			\item $a_i = R$, $a_i' = L$:
				$$ p_{12}·5 + p_{22}·0 ≥ p_{12}·4 + p_{22}·1 \implies p_{12} ≥ p_{22}. $$
		\end{itemize}

		Žádné další podmínky na CE nejsou, tedy každá\footnotemark{} $\max(p_{22}, p_{11}) ≤ \min(p_{12}, p_{21})$ jsou CE.
	\end{reseni}
	\footnotetext{1. Ještě jsou $p_{ij}$ pravděpodobnostní rozdělení, tedy $0 ≤ p_{ij}$ a $\sum_{ij} p_{ij} = 1$.}
\end{priklad}
	
\end{document}
