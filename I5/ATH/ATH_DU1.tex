\documentclass[12pt]{article}					% Začátek dokumentu
\usepackage{../../MFFStyle}							% Import stylu

\begin{document}
\begin{priklad}[1.1]
	Spočítejte Nashova ekvilibria ve Hře na kuře a formálně dokažte, že žádná jiná Nashova ekvilibria v této hře neexistují.

	\begin{reseni}
		Označme si $T_i = s_i(\text{Straight})$ a $W_i = s_i(\text{Swerve}) = 1 - T_i$ (součet pravděpodobností v rozdělení musí být 1). Užitek prvního hráče je pak:
		$$ u_1(s) = W_1·W_2·0 + W_1·T_2·(-1) + T_1·W_2·1 + T_1·T_2·(-10) = $$
		$$ = (1 - T_1)·T_2·(-1) + T_1·(1 - T_2) + T_1·T_2·(-10) = $$
		$$ = T_1(T_2 + 1 - T_2 - 10 T_2) - T_2 = T_1(1 - 10T_2) - T_2. $$
		To je přímka se směrnicí $1 - 10T_2$, tedy když $T_2 < 0.1$, tak se prvnímu hráči vyplatí nejvíc vždy jet rovně (tj. $T_1 = 1$), a když $T_2 > 0.1$, tak se prvnímu hráči nejvíce vyplatí vždy uhnout (tj. $T_1 = 0$). Pokud $T_2 = 0.1$, tak nezáleží na tom, jakou strategii volí první hráč, vždy dostane ve střední hodnotě $-0.1$.

		Můžeme si všimnout, že hra je symetrická, tedy pro Nashovo ekvilibrium: pokud $T_1~<~0.1$, pak $T_2 = 1$, pokud $T_1 > 0.1$, pak $T_2 = 0$ a pokud $T_1 = 0.1$, tak $T_2 \in [0, 1]$.

		Takže všechna Nashova ekvilibria jsou
		$$ \begin{pmatrix} s_1(\text{Swe}) & s_1(\text{Str}) \\ s_2(\text{Swe}) & s_2(\text{Str}) \end{pmatrix} = \begin{pmatrix} W_1 & T_1 \\ W_2 & T_2 \end{pmatrix} = \begin{pmatrix} 1 - T_1 & T_1 \\ 1 - T_2 & T_2 \end{pmatrix} \in \{\begin{pmatrix} 1 & 0 \\ 0 & 1 \end{pmatrix}; \begin{pmatrix} 0 & 1 \\ 1 & 0 \end{pmatrix}; \begin{pmatrix} 0.9 & 0.1 \\ 0.9 & 0.1 \end{pmatrix}\}. $$
		Neboť jakmile jeden z hráčů hraje něco jiného než $T=0.1$, pak ten druhý už má nejlepší strategii $T = 1$ nebo $T = 0$.
	\end{reseni}
\end{priklad}

\begin{priklad}[1.2]
	Zhlédněte scénu z filmu Čistá duše (A Beautiful Mind), kde John Nash hraný Russellem Crowem vysvětluje pojem Nashova ekvilibria. Předpokládejme že daná situace je modelována hrou čtyř hráčů (muži), kde každý má pět možných akcí (ženy). Vysvětlete, proč řešení navržené ve filmu není Nashovým ekvilibriem.

	\begin{reseni}
		Nevím, zda jsem našel správnou scénu. A záleží na tom, které řešení.

		Řešení které zde připisuje Nash Adamu Smithovi, tedy to, kdy všichni půjdou za jednou „nejlepší“ ženou není Nashovo ekvilibrium, protože pak nezíská nikdo nic (resp. získají velmi málo), takže každý kdo se odchýlí od této strategie (tj. půjde za jinou), získá.

		Naopak v Nashem nabízeném řešení, tedy pokud nepůjde za tou „nejlepší“ nikdo, každý (jednotlivě) získá tím, že svou volbu změní na „nejlepší“.

		Navíc jednokolové přiblížení je velmi zkreslující a vlastně všichni ztrácí tím, že poslouchají Nashe a nejdou se zadat k tanci jako první.
	\end{reseni}
\end{priklad}

\begin{priklad}[1.3]
	Na ulici se nachází $n ≥ 2$ lidí, kteří si všichni všimnou zraněného muže. Každý z těchto lidí má k dispozici dvě akce: buď muži pomůže či ne. Pokud nikdo nepomůže, pak všichni dostanou výplatu $0$. Pokud někdo pomůže, pak všichni dostanou výplatu $1$ kromě jedinců, kteří se rozhodli pomoci, ti získají výplatu $1 - c$ pro nějaké $c \in (0, 1)$. Nalezněte symetrické Nashovo ekvilibrium této hry, neboli Nashovo ekvilibrium, ve kterém všichni hráči používají tutéž strategii. Jaká je potom pravděpodobnost, že muži někdo pomůže? Je pro muže lepší mít okolo více svědků?

	\begin{reseni}
		Předpokládejme, že máme už nějaké takové Nashovo ekvilibrium, tedy pravděpodobnost $p$, se kterou libovolný hráč pomůže. Aby to bylo Nashovo ekvilibrium, nemůže si jeden konkrétní hráč pomoci tím, že změní svou pravděpodobnost z $p$ na $\tilde p$. Jeho užitek potom bude\footnote{Jeho užitek se skládá z toho, když pomůže on ($\tilde p$), když nepomůže on ($1 - \tilde p$) a pomůže někdo jiný -- tj. nenastane situace, že by nikdo nepomohl -- ($1 - (1 - p)^{n-1}$. Když nepomůže nikdo, užitek je 0.)}
		$$ \tilde p (1 - c) + (1 - \tilde p)·\(1 - (1 - p)^{n-1}\) = \tilde p \((1 - p)^{n-1} - c\) + 1 - (1 - p)^{n-1}. $$

		Takže jsme (nepřekvapivě) dostali zase přímku, tentokrát se směrnicí $(1 - p)^{n-1} - c$. Tedy když $(1 - p)^{n-1} > c$, tak nejlepší $\tilde p$ je $1$, když $(1 - p)^{n-1} < c$, tak $0$, a nakonec pokud $(1 - p)^{n-1} = c$, tak na $\tilde p$ nezáleží.

		Strategie $(1 - p)^{n-1} > c$ a $(1 - p)^{n-1} < c$ tudíž nejsou možné (v prvním případě musí být $1 - p > 0$, ale $p = 1$, aby byla nejlepší, v druhém zase $1 - p < 1$, ale $p = 0$), tedy symetrické Nashovo ekvilibrium je pouze strategie $p$ splňující $(1 - p)^{n-1} = c$, tj.
		$$ p = 1 - \sqrt[n - 1]{c}. $$

		Pravděpodobnost, že pak někdo pomůže, (doplněk k pravděpodobnosti, že nikdo nepomůže) je $\(1 - (1 - p)^n\) = 1 - c·(1 - p) = 1 - c·\sqrt[n - 1]{c}$, což se rostoucím $n$ zmenšuje (odmocnina roste k jedné).
	\end{reseni}
\end{priklad}

\break

\begin{priklad}[1.4]
	Uvažme hru dvou hráčů, ve které si každý hráč vybírá nezáporné číslo velikosti nanejvýš 1000. Hráč 1 si vybírá sudá čísla, hráč 2 si vybírá lichá čísla. Poté, co hráči oznámí své číslo, hráč, který si vybral nižší číslo, vyhrává tolik korun, kolik je hodnota jím vybraného čísla. Nalezněte všechna čistá Nashova ekvilibria této hry.

	\begin{reseni}
		Jakmile si první hráč vybere číslo $≥ 2$, tak pro druhého hráče je nejlepší vybrat si číslo o jedna menší. Naopak pokud si druhý hráč vybere číslo $≥ 3$, tak první hráč si pro svůj maximální užitek vybere číslo o jedna menší.

		Tudíž druhý hráč si pro Nashovo ekvilibrium musí vybrat číslo $1$, protože jinak by si první hráč z maximalizace svého užitku musel vybrat číslo o jedna menší, a druhý hráč ještě o jedna menší. A $n ≠ n - 2$.

		Když si druhý hráč vybere číslo $1$, tak druhý hráč už nic nezíská, takže je mu jedno, co zvolí, ale aby to bylo Nashovo ekvilibrium, tak nesmí dát možnost druhému hráči se zlepšit, tudíž může vybrat buď číslo $0$ nebo číslo $2$. Všechna Nashova ekvilibria jsou tedy dvojice $(0, 1)$ a $(2, 1)$.
	\end{reseni}
\end{priklad}



\end{document}
