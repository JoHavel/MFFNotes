\documentclass[12pt]{article}					% Začátek dokumentu
\usepackage{../../MFFStyle}					    % Import stylu



\begin{document}

\begin{priklad}[4.1 -- Střední hodnota versus pravděpodobnost]
	Bob navrhne Alici následující hru: „Tady mám minci, která není spravedlivá -- pravděpodobnost, že na ní padne hlava je $p \in \(\frac{1}{3}, \frac{1}{2}\)$. Tvůj počáteční vklad je 100 Kč a pokaždé, když na minci padne hlava, tvůj kapitál zdvojnásobím. Pokud padne orel, tak mi naopak dáš polovinu svého kapitálu. Označme $X_n$ hodnotu tvého kapitálu po $n$-tém hodu mincí. Je zřejmé, že $\lim_{n \rightarrow ∞} ®E X_n = ∞$, takže očekávaná hodnota tvého kapitálu poroste nade všechny meze.“

	Je pro Alici výhodné takovou hru hrát? Ověřte Bobovo tvrzení a ukažte, že $\displaystyle \lim_{n \rightarrow ∞} X_n = 0$ skoro jistě.

	\begin{reseni}
		Označme si $Y_n$ jako indikátor, že v $n$-tém hodu padla hlava. Potom můžeme vyjádřit $X_{n+1} = \frac{X_n}{2} + Y_{n + 1} · X_n · \frac{3}{2}$. Střední hodnotu $X_n$ pak spočítáme z linearity střední hodnoty:
		$$ ®E X_n = \frac{®E X_{n - 1}}{2} + \frac{3}{2}·®E(Y_n · X_{n-1}) = \frac{®E X_{n - 1}}{2} + \frac{3}{2}·®EY_n·®EX_{n-1} = \(\frac{1}{2} + \frac{3}{2}p\)®E X_{n - 1} $$
		Neboť $Y_n$ a $X_{n - 1}$ jsou zřejmě nezávislé. To, co nám vyšlo, je ale geometrická posloupnost s koeficientem $q = \frac{1}{2} + \frac{3}{2}p > \frac{1}{2} + \frac{1}{2} = 1$, tedy $\lim_{n \rightarrow ∞} ®E X_n = 100 · q^n = ∞$. Tím jsme dokázali, že Bob říká pravdu.

		Jelikož $X_n$ jsou nezáporná, tak $\lim_{n \rightarrow ∞} X_n = 0$ je ekvivalentní $\limsup_{n \rightarrow ∞} X_n = 0$ („$0 ≤ \liminf ≤ \limsup$“, tedy „$\liminf = \limsup$“, tj. „$\limsup = \lim$“). Ze spojitosti pravděpodobnosti je „$\limsup_{n \rightarrow ∞} X_n = 0$ skoro jistě“ totéž co
		$$ \forall \epsilon > 0: P(\limsup_{n \rightarrow ∞} X_n ≤ \epsilon) = 1. $$
		A protože $X_n ≥ 0$, tak můžeme místo $X_n$ psát $|X_n|$. Navíc se můžeme podívat na jevy $\{|X_n| > \epsilon\}$. Jejich doplněk je $\{|X_n| ≤ \epsilon\}$ a průnik toho od $k$ do $∞$ je $\{\sup_{n ≥ k} |X_n| ≤ \epsilon\}$. Sjednocení přes $n$ je pak $\{\limsup_{n \rightarrow ∞} |X_n| ≤ \epsilon\}$, tedy
		$$ P(\limsup_{n \rightarrow ∞} X_n ≤ \epsilon) = P(\liminf_{n \rightarrow ∞}\{|X_n| > \epsilon\}^C). $$
		Z důsledku Cantelliho věty nám tedy stačí dokázat $\sum_{n=1}^∞ P(|X_n| > \epsilon) < ∞$, pak už bude výraz výše opravdu roven 1.

		$P(|X_n| > \epsilon)$ můžeme podle Markovovy nerovnosti odhadnout jako $P(|X_n| > \epsilon) ≤ \frac{®E |X_n|}{\epsilon} = \frac{®E X_n}{\epsilon}$. A to s použitím výše spočítaného $®E X_n = 100 · q^n$ sečíst na:
		$$ \sum_{n=1}^∞ P(|X_n| > \epsilon) ≤ \sum_{n=1}^∞ \frac{100·q^n}{\epsilon} = \frac{100}{\epsilon}·\frac{1}{1 - q} < ∞. $$

		%můžeme přepsat jako $P(|X_0 · 2^{-n + 2·\sum_1^n Y_n}| > \epsilon)$, to jako $P(-n + 2· \sum_1^n Y_n > \log_2(\frac{\epsilon}{X_0}))$ a to přepíšeme a odhadneme pomocí Markovovy nerovnosti takto:
		%$$ \lim_{n \rightarrow ∞} P\(\overline{Y_n} > \frac{1}{2n}·\log_2\(\frac{\epsilon}{X_0}\) + \frac{1}{2}\) ≤ \lim_{n \rightarrow ∞} \frac{®E \overline{Y_n}}{\frac{1}{2n}·\log_2\(\frac{\epsilon}{X_0}\) + \frac{1}{2}} = \lim_{n \rightarrow ∞} \frac{p}{\frac{1}{n}·const + \frac{1}{2}} = 2p $$
	\end{reseni}
\end{priklad}

\pagebreak
\begin{priklad}[4.2 -- Konvergence v distribuci pro diskrétní náhodné veličiny]
	Buďte $X_n$, $n \in ®N$, a $X$ náhodné veličiny na stejném pravděpodobnostním prostoru, které nabývají skoro jistě hodnot ze $®Z$. Dokažte, že následující tvrzení jsou ekvivalentní:
	\vspace{-1em}

	\begin{enumerate}
		\item $X_n \overset{D} \rightarrow X$;
		\item $\lim_{n \rightarrow ∞} P(X_n = k) = P(X = k)$, $\forall k \in ®Z$,
		\item $\lim_{n \rightarrow ∞} \sum_{k \in ®Z} |P(X_n = k) - P(X = k)| = 0$.
	\end{enumerate}

	\begin{dukazin}[„1. $\implies$ 2.“]
		Jelikož $X_n$ a $X$ nabývají skoro jistě celých čísel, mění distribuční funkce těchto veličin hodnotu pouze v celých číslech a mezi nimi je konstantní, tedy také spojitá. Navíc je distribuční funkce zprava spojitá, tedy na celých číslech je rovna svým hodnotám na jejich pravém okolí. Tedy $\sum_{k = -∞}^z P(X_n = z) = F_n(z) = F_n(y) \forall y \in [z, z+1), z \in ®Z$ a obdobně pro $F$, kde $F_n$ je distribuční funkce $X_n$ a $F$ distribuční funkce $X$.

		Tedy z konvergence v distribuci víme, že $\lim_{n \rightarrow ∞} F_n(x) = F(x)$ pro $x \in ®R \setminus ®Z$ (jelikož tam jsou $F_n$ a $F$ spojité). A protože víme $P(X_n = k) = F_n(k + 0.5) - F_n(k - 0.5)$ a totéž pro $P(X = k)$, tak $\lim_{n \rightarrow ∞} P(X_n = k) = P(X = k)$.
	\end{dukazin}

	\begin{dukazin}[„2. $\implies$ 3.“]
		Využijeme toho, že $\sum_{i=0}^∞ P(X = i + 1) + P(X = i) = 1$. To totiž znamená, že pro každé $\epsilon > 0$ můžeme vybrat $I$ tak, že $\sum_{i=0}^I P(X = i + 1) + P(X = i) > 1 - \epsilon$. Dále z 2. můžeme pro každé $j \in [-I, I+1]$ vybrat $n_0$ tak, že $\forall n > n_0: |P(X_n = j) - P(X = j)| < \frac{\epsilon}{2I + 2}$. Potom
		$$ (\forall \epsilon > 0\ \exists n_0)\ \forall n > n_0: \sum_{k \in ®Z}|P(X_n = k) - P(X = k)| = $$
		$$ = \underbrace{\sum_{k \in ®Z \cap ((-∞, -I) \cup (I + 1, +∞))} |P(X_n = k) - P(X = k)|}_A + \underbrace{\sum_{k = -I}^{I + 1} |P(X_n = k) - P(X = k)|}_B, $$
		kde v $B$ máme každý z $2I + 2$ členů odhadnutý shora $\frac{\epsilon}{2I + 2}$, tedy $A < (2I + 2)·\frac{\epsilon}{2I + 2} = \epsilon$. $B$ můžeme nejprve díky trojúhelníkové nerovnosti rozdělit na
		$$ \sum_{k \in …} |P(X = k)| = 1 - \sum_{i=0}^I P(X = i + 1) + P(X = i) < 1 - (1 - \epsilon) = \epsilon $$
		$$ \text{a } \sum_{k \in …} |P(X_n = k)| = 1 - \sum_{i=-I}^{I + 1} P(X_n = i) ≤ 1 - \sum_{k = -I}^{I + 1} P(X_n = k) - P(X = k) + P(X = k) ≤ $$
		$$ 1 + \sum_{k = -I}^{I + 1} |-P(X_n = k) + P(X = k)| - \sum_{k = -I}^{I+1} |P(X = k)| < 1 + \epsilon - (1 - \epsilon) = 2\epsilon. $$
		Tedy $A + B < 3 \epsilon$ a tak dostáváme výše definici limity z 3.
	\end{dukazin}

	\begin{dukazin}[„3. $\implies$ 1.“]
		V prvním důkazu jsme si řekli, že $F_n(y) = \sum_{i = -∞}^{\left\lfloor y \right\rfloor} P(X_n = i)$ a obdobně pro $F$. A jelikož víme, že $\sum_{i = -∞}^{\left\lfloor y \right\rfloor} P(X_n = i)$ konvergují k $\sum_{i = -∞}^{\left\lfloor y \right\rfloor} P(X = i)$, neboť
		$$ \sum_{i=-∞}^{\left\lfloor y \right\rfloor} |P(X_n = i) - P(X = i)| ≤ \sum_{i=-∞}^{+∞} |P(X_n = i) - P(X = i)| $$
		ze 3., tak i $F_n(y)$ konvergují k $F(y)$, tedy $X_n$ konvergují v distribuci k $X$.
	\end{dukazin}
\end{priklad}

\end{document}
