\documentclass[12pt]{article}					% Začátek dokumentu
\usepackage{../../MFFStyle}					    % Import stylu



\begin{document}

% 15. 02. 2022
\section*{Organizační úvod}
\begin{poznamka}
	Jako vždycky, jen v implikacích u zkoušky budou i pojmy z předchozích semestrů.
\end{poznamka}

\section{Stejnoměrná konvergence posloupností a řad funkcí}
	\subsection{Bodová a stejnoměrná konvergence posloupnosti funkcí}
	\begin{definice}
		Nechť $J \subset ®R$ je interval a nechť máme funkce $f: J \rightarrow ®R$ a $f_n: J \rightarrow ®R$ pro $n \in ®N$. Řekneme, že posloupnost funkcí $\{f_n\}$

		\begin{itemize}
			\item konverguje bodově k $f$ na $J$, pokud $\forall x \in J: \lim_{n \rightarrow ∞} f_n(x) = f(x)$, neboli:
				$$ \forall x \in J\ \forall \epsilon > 0\ \exists n_0\ \forall n ≥ n_0: |f_n(x) - f(x)| < \epsilon; $$
			\item konverguje stejnoměrně k $f$ na $J$ (značíme $f_n \rightrightarrows f$ na $J$), pokud
				$$ \forall \epsilon > 0\ \exists n_0\ \forall n ≥ n_0\ \forall x \in J: |f_n(x) - f(x)| < \epsilon; $$
			\item konverguje lokálně stejnoměrně, pokud pro každý omezený uzavřený $[a, b] \subset J$ platí $f_n \rightrightarrows f$ na $[a, b]$ (značíme $f_n \overset{\text{Loc}}{\rightrightarrows} f$ na $J$).
		\end{itemize}
	\end{definice}

	\begin{veta}[Kritérium stejnoměrné konvergence]
		Nechť $f, f_n: J \rightarrow ®R$, pak
		$$ f_n \rightrightarrows f \text{ na } J \Leftrightarrow \lim_{n \rightarrow ∞} \sup_{x \in J} |f_n(x) - f(x)| = 0. $$

		\begin{dukazin}
			$$ \forall \epsilon > 0\ \exists n_0\ \forall n ≥ n_0\ \forall x \in J: |f_n(x) - f(x)| ≤ \epsilon \Leftrightarrow $$
			$$ \Leftrightarrow \forall \epsilon > 0\ \exists n_0\ \forall n ≥ n_0: \sup_{x \in J} |f_n(x) - f(x)| ≤ \epsilon \Leftrightarrow $$
			$$ \Leftrightarrow \lim_{n \rightarrow ∞} \sup_{x \in J} |f_n(x) - f(x)| = 0. $$
		\end{dukazin}

		\begin{poznamkain}[Pro spojité funkce]
			$$ \Leftrightarrow ||f_n - f||_{©C(J)} \rightarrow 0 \Leftrightarrow f_n \overset{©C(J)} \rightarrow f. $$
		\end{poznamkain}
	\end{veta}

	\begin{veta}[Bolzano-Cauchyova podmínka pro stejnoměrnou konvergenci]
		Nechť $f_n: J \rightarrow ®R$, pak
		$$ (\exists f: f_n \rightrightarrows f \text{ na } J) \Leftrightarrow (\forall \epsilon > 0\ \exists n_0 \in ®N\ \forall m, n ≥ n_0\ \forall x \in J: |f_n(x) - f_m(x)| < \epsilon). $$

		\begin{dukazin}
			„$\implies$“: Víme $\forall \epsilon > 0\ \exists n_0\ \forall n ≥ n_0\ \forall x \in J: |f_n(x) - f(x)| < \epsilon$. Tedy
			$$ \forall m, n ≥ n_0\ \forall x \in J: |f_n(x) - f_m(x)| ≤ |f_n(x) - f(x)| + |f(x) - f_m(x)| < 2\epsilon. $$

			„$\impliedby$“: Víme $\forall \epsilon > 0\ \exists n_0 \in ®N\ \forall m, n ≥ n_0\ \forall x \in J: |f_n(x) - f_m(x)| < \epsilon$. Toto použijeme pro pevné $x \in J$. Pro posloupnost $a_n = f_n(x)$ máme splněnou BC podmínku pro posloupnost reálných čísel, tj. $a_n \rightarrow a \in ®R$.

			Označíme si $f(x) = \lim_{n \rightarrow ∞} f_n(x)$. Nyní v BC podmínce provedeme limitu $n \rightarrow ∞$. Tím dostaneme přesně definici stejnoměrné konvergence.
		\end{dukazin}
	\end{veta}

	\begin{veta}[Moore-Osgood]
		Nechť $x_0 \in ®R^*$ je krajní bod intervalu $J$. Nechť $f_n, f: J \rightarrow ®R$ splňují

		\begin{itemize}
			\item $f_n \rightrightarrows f$ na $J$,
			\item existuje $\lim_{x \rightarrow x_0} f_n(x) = a_n \in ®R\ \forall n \in ®N$.
		\end{itemize}

		Pak existují $\lim_{n \rightarrow ∞} a_n$ a $\lim_{x \rightarrow x_0} f(x)$ a jsou si rovny.

		\begin{dukazin}
			Příště.
		\end{dukazin}
	\end{veta}

	\begin{dusledek}
		Nechť $f_n \rightrightarrows f$ na $I$ a nechť $f_n$ jsou spojité na $I$. Pak $f$ je spojitá na $I$.
	\end{dusledek}

	\begin{poznamka}
		Obdobně lze definovat stejnoměrnou spojitost i pro libovolnou množinu $A \subset ®R^n$ a $f_n: A \rightarrow ®R$ a platí, že stejnoměrná limita je spojitá (stejnoměrná limita spojitých funkcí je spojitá).
	\end{poznamka}

% 22. 02. 2022

	\begin{dukaz}[Moore-Osgood]
		Z BC podmínky
		$$ \forall \epsilon > 0\ \exists n_0\ \forall m, n ≥ n_0\ \forall x \in J: |f_n(x) - f_m(x)| < \epsilon. $$
		Provedeme $\lim_{x \rightarrow x_0}$ a dostaneme
		$$ \forall \epsilon > 0\ \exists n_0\ \forall m, n ≥ n_0: |a_n - a_m| ≤ \epsilon. $$
		Tedy $a_n$ splňuje BC podmínku, a tudíž $\exists \lim_{n \rightarrow ∞} a_n = a \in ®R$.

		Nechť $\epsilon ≥ 0$. Z definice $f_n \rightrightarrows f$
		$$ \exists n_0\ \forall x \in J: |f_{n_0}(x) - f(x)| < \epsilon. $$
		Zároveň předpokládejme $|a_{n_0} - a| < \epsilon$ (zvolíme si $n_0$ jako maximum). Máme pevnou funkci $f_{n_0}$ a $\lim_{x \rightarrow x_0} f_{n_0}(x) = a_{n_0}$. Tedy
		$$ \exists \delta > 0\ \forall x \in P(x_0, \delta) \cap J: |f_{n_0}(x) - a_{n_0}| < \epsilon. $$
		Nyní $\forall x \in P(x_0, \delta) \cap J$ platí
		$$ |f(x) - a| ≤ |f(x) - f_{n_0}(x)| + |f_{n_0}(x) - a_{n_0}| + |a_{n_0} - a| ≤ \epsilon + \epsilon + \epsilon = 3 \epsilon. $$
	\end{dukaz}

	\begin{veta}[O záměně limity a derivace]
		Nechť funkce $f_n$, $n \in ®N$, mají vlastní derivaci na intervalu $(a, b)$ a nechť

		\begin{itemize}
			\item $\exists x_0 \in (a, b): \{f_n(x_0)\}_{n=1}^∞$ konverguje,
			\item pro derivace $f_n'$ platí $f_n' \overset{\text{Loc}}{\rightrightarrows}$ na $(a, b)$.
		\end{itemize}

		Potom existuje funkce $f$ tak, že $f_n \overset{\text{Loc}}{\rightrightarrows} f$ na $(a, b)$, $f$ má vlastní derivaci a platí $f_n' \overset{\text{Loc}}{\rightrightarrows} f'$ na $(a, b)$.

		\begin{dukazin}
			Nechť $x_0 \in [c, d] \subset (a, b)$. Víme $f_n' \rightrightarrows$ na $[c, d]$. Chceme ukázat $f_n \rightrightarrows f$ na $[c, d]$ ($\implies f_n \overset{\text{Loc}}{\rightrightarrows} f$ na $(a, b)$). Nechť $\epsilon > 0$. Z BC podmínky pro $f_n' \rightrightarrows$
			$$ \exists n_0\ \forall m, n ≥ n_0\ \forall x \in [c, d]: |f_n'(x) - f_m'(x)| < \epsilon $$
			a zároveň $\forall m, n ≥ n_0: |f_n(x_0) - f_m(x_0)| < \epsilon$. Nyní $\forall x \in [c, d]:$
			$$ |f_n(x) - f_m(x)| ≤ |f_n(x) - f_m(x) - (f_n(x_0) - f_m(x_0))| + |f_n(x_0) - f_m(x_0)| ≤ $$
			$$ ≤ |h(x) - h(x_0)| + \epsilon ≤ |x - x_0|·|h'(\xi)| + \epsilon ≤ (d - c)·\epsilon + \epsilon, $$
			kde $h = f_n - f_m$ a $\xi \in (x_0, x)$ resp. $(x, x_0)$ z Lagrangeovy věty (cvičení: předpoklady jsou splněné).

			Zbývá dokázat „$f_n' \rightrightarrows f'$ na $[c, d]$“: Zvolme $z \in [c, d]$ a položme $\phi_n(x) = \frac{f_n(x) - f_n(z)}{x - z}$ pro $x \in [c, d] \setminus \{z\}$. Nechť $\epsilon > 0$. Z BC podmínky pro $f_n' \rightrightarrows$
			$$ \exists n_0\ \forall n, m ≥ n_0\ \forall x \in [c, d]: |f_n'(x) - f_m'(x)| < \epsilon. $$
			Podobně jako v první části důkazu
			$$ |f_n(x) - f_m(x) - (f_n(z) - f_m(z))| = |f_n'(\xi) - f_m'(\xi)|·|x - z| < \epsilon·|x - z|. $$
			Nyní $\forall m, n ≥ n_0\ \forall x \in [c, d]\setminus\{z\}$:
			$$ |\phi_n(x) - \phi_m(x)| = \left| \frac{f_n(x) - f_m(z) - (f_m(x) - f_m(z)}{x - z} \right| < \frac{\epsilon·|x - z|}{|x - z|} = \epsilon. $$
			Podle BC $\phi_n \rightrightarrows$ na $[c, d] \setminus \{z\}$. Tedy $\phi_n$ splňuje předpoklady Moore-Osgoodovy věty ($\lim_{x \rightarrow z} \phi_n(x) = \lim_{x \rightarrow z} \frac{f_n(x) - f_n(z)}{x - z} = f'(z)$). Tedy
			$$ \lim_{n \rightarrow ∞} \lim_{x \rightarrow z} \frac{f_n(x) - f_n(z)}{x - z} = \lim_{x \rightarrow z} \lim_{n \rightarrow ∞} \frac{f_n(x) - f_n(z)}{x - z} \Leftrightarrow $$
			$$ \Leftrightarrow \lim_{n \rightarrow ∞} f_n'(z) = \lim_{x \rightarrow z} \frac{f(x) - f(z)}{x - z} = f'(z). $$
			A jelikož víme, že $f_n' \rightrightarrows$, tak $f_n' \rightarrow f' \implies f_n' \rightrightarrows f'$.
		\end{dukazin}
	\end{veta}

	\subsection{Stejnoměrná konvergence řady funkcí}
	\begin{definice}
		Řekněme, že řada funkcí $\sum_{k=1}^∞ u_k(x)$ konverguje stejnoměrně (popřípadě lokálně stejnoměrně) na intervalu $J$, pokud posloupnost částečných součtů $s_n(x) = \sum_{k=1}^n u_k(x)$ konverguje stejnoměrně (popřípadě lokálně stejnoměrně) na $J$.
	\end{definice}

	\begin{veta}[Nutná podmínka stejnoměrné konvergence řady]
		Nechť $\sum_{n=1}^∞ u_n(x)$ je řada funkcí definovaných na intervalu $J$. Pokud $\sum_{n=1}^∞ u_n(x) \rightrightarrows$ na $J$, pak posloupnost funkcí $u_n(x) \rightrightarrows 0$ na $J$.

		\begin{dukazin}
			$$ \forall \epsilon > 0\ \exists n_0\ \forall m, n ≥ n_0\ \forall x \in J: |s_n(x) - s_m(x)| < \epsilon $$
			speciálně pro $m = n + 1$:
			$$ \forall \epsilon > 0\ \exists n_0\ \forall n ≥ n_0\ \forall x \in J: |\sum_{i=1}^{n+1} u_i - \sum_{i=1}^n u_i| = |u_{n+1}(x)| < \epsilon \implies u_n \rightrightarrows 0. $$
		\end{dukazin}
	\end{veta}

	\begin{veta}[Weierstrassovo kritérium]
		Nechť $\sum_{n=1}^∞ u_n(x)$ je řada funkcí definovaných na intervalu $J$. Pokud pro
		$$ \sigma_n = \sup\{|u_n|(x) : x \in J\} $$
		platí, že číselná řada $\sum_{n=1}^∞ \sigma_n$ konverguje, pak $\sum_{n=1}^∞ u_n(x) \rightrightarrows$ na $J$.

		\begin{dukazin}
%
% 01. 03. 2022
%
			Nechť $\epsilon > 0$. Z BC podmínky pro konečnou $\sum_{k=1}^∞ \sigma_k$
			$$ \exists n_0\ \forall m, n ≥ n_0, m > n: |\sum_{k=n+1}^m \sigma_k| < \epsilon. $$
			Chceme ověřit BC podmínku pro $s_n(x) = \sum_{k=1}^n u_k(x)$:
			$$ \forall m, n ≥ n_0, m > n\ \forall x \in J: |s_m(x) - s_n(x)| = \left|\sum_{k=1}^m u_k(x) - \sum_{k=1}^n u_k(x)\right| = $$
			$$ = \left|\sum_{k=n+1}^m u_k(x)\right| ≤ \sum_{k=n+1}^m |u_k(x)| ≤ \sum_{k=n+1}^m \sigma_k < \epsilon. $$
			Tedy podle BC podmínky $\sum u_k \rightrightarrows$.
		\end{dukazin}
	\end{veta}
\end{document}
