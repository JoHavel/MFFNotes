\documentclass[12pt]{article}					% Začátek dokumentu
\usepackage{../../MFFStyle}					    % Import stylu



\begin{document}

% 15. 02. 2022
\section*{Organizační úvod}
\begin{poznamka}
	Jako vždycky, jen v implikacích u zkoušky budou i pojmy z předchozích semestrů.
\end{poznamka}

\section{Stejnoměrná konvergence posloupností a řad funkcí}
	\subsection{Bodová a stejnoměrná konvergence posloupnosti funkcí}
	\begin{definice}
		Nechť $J \subset ®R$ je interval a nechť máme funkce $f: J \rightarrow ®R$ a $f_n: J \rightarrow ®R$ pro $n \in ®N$. Řekneme, že posloupnost funkcí $\{f_n\}$

		\begin{itemize}
			\item konverguje bodově k $f$ na $J$, pokud $\forall x \in J: \lim_{n \rightarrow ∞} f_n(x) = f(x)$, neboli:
				$$ \forall x \in J\ \forall \epsilon > 0\ \exists n_0\ \forall n ≥ n_0: |f_n(x) - f(x)| < \epsilon; $$
			\item konverguje stejnoměrně k $f$ na $J$ (značíme $f_n \rightrightarrows f$ na $J$), pokud
				$$ \forall \epsilon > 0\ \exists n_0\ \forall n ≥ n_0\ \forall x \in J: |f_n(x) - f(x)| < \epsilon; $$
			\item konverguje lokálně stejnoměrně, pokud pro každý omezený uzavřený $[a, b] \subset J$ platí $f_n \rightrightarrows f$ na $[a, b]$ (značíme $f_n \overset{\text{Loc}}{\rightrightarrows} f$ na $J$).
		\end{itemize}
	\end{definice}

	\begin{veta}[Kritérium stejnoměrné konvergence]
		Nechť $f, f_n: J \rightarrow ®R$, pak
		$$ f_n \rightrightarrows f \text{ na } J \Leftrightarrow \lim_{n \rightarrow ∞} \sup_{x \in J} |f_n(x) - f(x)| = 0. $$

		\begin{dukazin}
			$$ \forall \epsilon > 0\ \exists n_0\ \forall n ≥ n_0\ \forall x \in J: |f_n(x) - f(x)| ≤ \epsilon \Leftrightarrow $$
			$$ \Leftrightarrow \forall \epsilon > 0\ \exists n_0\ \forall n ≥ n_0: \sup_{x \in J} |f_n(x) - f(x)| ≤ \epsilon \Leftrightarrow $$
			$$ \Leftrightarrow \lim_{n \rightarrow ∞} \sup_{x \in J} |f_n(x) - f(x)| = 0. $$
		\end{dukazin}

		\begin{poznamkain}[Pro spojité funkce]
			$$ \Leftrightarrow ||f_n - f||_{©C(J)} \rightarrow 0 \Leftrightarrow f_n \overset{©C(J)} \rightarrow f. $$
		\end{poznamkain}
	\end{veta}

	\begin{veta}[Bolzano-Cauchyova podmínka pro stejnoměrnou konvergenci]
		Nechť $f_n: J \rightarrow ®R$, pak
		$$ (\exists f: f_n \rightrightarrows f \text{ na } J) \Leftrightarrow (\forall \epsilon > 0\ \exists n_0 \in ®N\ \forall m, n ≥ n_0\ \forall x \in J: |f_n(x) - f_m(x)| < \epsilon). $$

		\begin{dukazin}
			„$\implies$“: Víme $\forall \epsilon > 0\ \exists n_0\ \forall n ≥ n_0\ \forall x \in J: |f_n(x) - f(x)| < \epsilon$. Tedy
			$$ \forall m, n ≥ n_0\ \forall x \in J: |f_n(x) - f_m(x)| ≤ |f_n(x) - f(x)| + |f(x) - f_m(x)| < 2\epsilon. $$

			„$\impliedby$“: Víme $\forall \epsilon > 0\ \exists n_0 \in ®N\ \forall m, n ≥ n_0\ \forall x \in J: |f_n(x) - f_m(x)| < \epsilon$. Toto použijeme pro pevné $x \in J$. Pro posloupnost $a_n = f_n(x)$ máme splněnou BC podmínku pro posloupnost reálných čísel, tj. $a_n \rightarrow a \in ®R$.

			Označíme si $f(x) = \lim_{n \rightarrow ∞} f_n(x)$. Nyní v BC podmínce provedeme limitu $n \rightarrow ∞$. Tím dostaneme přesně definici stejnoměrné konvergence.
		\end{dukazin}
	\end{veta}

	\begin{veta}[Moore-Osgood]
		Nechť $x_0 \in ®R^*$ je krajní bod intervalu $J$. Nechť $f_n, f: J \rightarrow ®R$ splňují

		\begin{itemize}
			\item $f_n \rightrightarrows f$ na $J$,
			\item existuje $\lim_{x \rightarrow x_0} f_n(x) = a_n \in ®R\ \forall n \in ®N$.
		\end{itemize}

		Pak existují $\lim_{n \rightarrow ∞} a_n$ a $\lim_{x \rightarrow x_0} f(x)$ a jsou si rovny.

		\begin{dukazin}
			Příště.
		\end{dukazin}
	\end{veta}

	\begin{dusledek}
		Nechť $f_n \rightrightarrows f$ na $I$ a nechť $f_n$ jsou spojité na $I$. Pak $f$ je spojitá na $I$.
	\end{dusledek}

	\begin{poznamka}
		Obdobně lze definovat stejnoměrnou spojitost i pro libovolnou množinu $A \subset ®R^n$ a $f_n: A \rightarrow ®R$ a platí, že stejnoměrná limita je spojitá (stejnoměrná limita spojitých funkcí je spojitá).
	\end{poznamka}

% 22. 02. 2022

	\begin{dukaz}[Moore-Osgood]
		Z BC podmínky
		$$ \forall \epsilon > 0\ \exists n_0\ \forall m, n ≥ n_0\ \forall x \in J: |f_n(x) - f_m(x)| < \epsilon. $$
		Provedeme $\lim_{x \rightarrow x_0}$ a dostaneme
		$$ \forall \epsilon > 0\ \exists n_0\ \forall m, n ≥ n_0: |a_n - a_m| ≤ \epsilon. $$
		Tedy $a_n$ splňuje BC podmínku, a tudíž $\exists \lim_{n \rightarrow ∞} a_n = a \in ®R$.

		Nechť $\epsilon ≥ 0$. Z definice $f_n \rightrightarrows f$
		$$ \exists n_0\ \forall x \in J: |f_{n_0}(x) - f(x)| < \epsilon. $$
		Zároveň předpokládejme $|a_{n_0} - a| < \epsilon$ (zvolíme si $n_0$ jako maximum). Máme pevnou funkci $f_{n_0}$ a $\lim_{x \rightarrow x_0} f_{n_0}(x) = a_{n_0}$. Tedy
		$$ \exists \delta > 0\ \forall x \in P(x_0, \delta) \cap J: |f_{n_0}(x) - a_{n_0}| < \epsilon. $$
		Nyní $\forall x \in P(x_0, \delta) \cap J$ platí
		$$ |f(x) - a| ≤ |f(x) - f_{n_0}(x)| + |f_{n_0}(x) - a_{n_0}| + |a_{n_0} - a| ≤ \epsilon + \epsilon + \epsilon = 3 \epsilon. $$
	\end{dukaz}

	\begin{veta}[O záměně limity a derivace]
		Nechť funkce $f_n$, $n \in ®N$, mají vlastní derivaci na intervalu $(a, b)$ a nechť

		\begin{itemize}
			\item $\exists x_0 \in (a, b): \{f_n(x_0)\}_{n=1}^∞$ konverguje,
			\item pro derivace $f_n'$ platí $f_n' \overset{\text{Loc}}{\rightrightarrows}$ na $(a, b)$.
		\end{itemize}

		Potom existuje funkce $f$ tak, že $f_n \overset{\text{Loc}}{\rightrightarrows} f$ na $(a, b)$, $f$ má vlastní derivaci a platí $f_n' \overset{\text{Loc}}{\rightrightarrows} f'$ na $(a, b)$.

		\begin{dukazin}
			Nechť $x_0 \in [c, d] \subset (a, b)$. Víme $f_n' \rightrightarrows$ na $[c, d]$. Chceme ukázat $f_n \rightrightarrows f$ na $[c, d]$ ($\implies f_n \overset{\text{Loc}}{\rightrightarrows} f$ na $(a, b)$). Nechť $\epsilon > 0$. Z BC podmínky pro $f_n' \rightrightarrows$
			$$ \exists n_0\ \forall m, n ≥ n_0\ \forall x \in [c, d]: |f_n'(x) - f_m'(x)| < \epsilon $$
			a zároveň $\forall m, n ≥ n_0: |f_n(x_0) - f_m(x_0)| < \epsilon$. Nyní $\forall x \in [c, d]:$
			$$ |f_n(x) - f_m(x)| ≤ |f_n(x) - f_m(x) - (f_n(x_0) - f_m(x_0))| + |f_n(x_0) - f_m(x_0)| ≤ $$
			$$ ≤ |h(x) - h(x_0)| + \epsilon ≤ |x - x_0|·|h'(\xi)| + \epsilon ≤ (d - c)·\epsilon + \epsilon, $$
			kde $h = f_n - f_m$ a $\xi \in (x_0, x)$ resp. $(x, x_0)$ z Lagrangeovy věty (cvičení: předpoklady jsou splněné).

			Zbývá dokázat „$f_n' \rightrightarrows f'$ na $[c, d]$“: Zvolme $z \in [c, d]$ a položme $\phi_n(x) = \frac{f_n(x) - f_n(z)}{x - z}$ pro $x \in [c, d] \setminus \{z\}$. Nechť $\epsilon > 0$. Z BC podmínky pro $f_n' \rightrightarrows$
			$$ \exists n_0\ \forall n, m ≥ n_0\ \forall x \in [c, d]: |f_n'(x) - f_m'(x)| < \epsilon. $$
			Podobně jako v první části důkazu
			$$ |f_n(x) - f_m(x) - (f_n(z) - f_m(z))| = |f_n'(\xi) - f_m'(\xi)|·|x - z| < \epsilon·|x - z|. $$
			Nyní $\forall m, n ≥ n_0\ \forall x \in [c, d]\setminus\{z\}$:
			$$ |\phi_n(x) - \phi_m(x)| = \left| \frac{f_n(x) - f_m(z) - (f_m(x) - f_m(z)}{x - z} \right| < \frac{\epsilon·|x - z|}{|x - z|} = \epsilon. $$
			Podle BC $\phi_n \rightrightarrows$ na $[c, d] \setminus \{z\}$. Tedy $\phi_n$ splňuje předpoklady Moore-Osgoodovy věty ($\lim_{x \rightarrow z} \phi_n(x) = \lim_{x \rightarrow z} \frac{f_n(x) - f_n(z)}{x - z} = f'(z)$). Tedy
			$$ \lim_{n \rightarrow ∞} \lim_{x \rightarrow z} \frac{f_n(x) - f_n(z)}{x - z} = \lim_{x \rightarrow z} \lim_{n \rightarrow ∞} \frac{f_n(x) - f_n(z)}{x - z} \Leftrightarrow $$
			$$ \Leftrightarrow \lim_{n \rightarrow ∞} f_n'(z) = \lim_{x \rightarrow z} \frac{f(x) - f(z)}{x - z} = f'(z). $$
			A jelikož víme, že $f_n' \rightrightarrows$, tak $f_n' \rightarrow f' \implies f_n' \rightrightarrows f'$.
		\end{dukazin}
	\end{veta}

	\subsection{Stejnoměrná konvergence řady funkcí}
	\begin{definice}
		Řekněme, že řada funkcí $\sum_{k=1}^∞ u_k(x)$ konverguje stejnoměrně (popřípadě lokálně stejnoměrně) na intervalu $J$, pokud posloupnost částečných součtů $s_n(x) = \sum_{k=1}^n u_k(x)$ konverguje stejnoměrně (popřípadě lokálně stejnoměrně) na $J$.
	\end{definice}

	\begin{veta}[Nutná podmínka stejnoměrné konvergence řady]
		Nechť $\sum_{n=1}^∞ u_n(x)$ je řada funkcí definovaných na intervalu $J$. Pokud $\sum_{n=1}^∞ u_n(x) \rightrightarrows$ na $J$, pak posloupnost funkcí $u_n(x) \rightrightarrows 0$ na $J$.

		\begin{dukazin}
			$$ \forall \epsilon > 0\ \exists n_0\ \forall m, n ≥ n_0\ \forall x \in J: |s_n(x) - s_m(x)| < \epsilon $$
			speciálně pro $m = n + 1$:
			$$ \forall \epsilon > 0\ \exists n_0\ \forall n ≥ n_0\ \forall x \in J: |\sum_{i=1}^{n+1} u_i - \sum_{i=1}^n u_i| = |u_{n+1}(x)| < \epsilon \implies u_n \rightrightarrows 0. $$
		\end{dukazin}
	\end{veta}

	\begin{veta}[Weierstrassovo kritérium]
		Nechť $\sum_{n=1}^∞ u_n(x)$ je řada funkcí definovaných na intervalu $J$. Pokud pro
		$$ \sigma_n = \sup\{|u_n(x)| : x \in J\} $$
		platí, že číselná řada $\sum_{n=1}^∞ \sigma_n$ konverguje, pak $\sum_{n=1}^∞ u_n(x) \rightrightarrows$ na $J$.

		\begin{dukazin}
%
% 01. 03. 2022
%
			Nechť $\epsilon > 0$. Z BC podmínky pro konečnou $\sum_{k=1}^∞ \sigma_k$
			$$ \exists n_0\ \forall m, n ≥ n_0, m > n: |\sum_{k=n+1}^m \sigma_k| < \epsilon. $$
			Chceme ověřit BC podmínku pro $s_n(x) = \sum_{k=1}^n u_k(x)$:
			$$ \forall m, n ≥ n_0, m > n\ \forall x \in J: |s_m(x) - s_n(x)| = \left|\sum_{k=1}^m u_k(x) - \sum_{k=1}^n u_k(x)\right| = $$
			$$ = \left|\sum_{k=n+1}^m u_k(x)\right| ≤ \sum_{k=n+1}^m |u_k(x)| ≤ \sum_{k=n+1}^m \sigma_k < \epsilon. $$
			Tedy podle BC podmínky $\sum u_k \rightrightarrows$.
		\end{dukazin}
	\end{veta}

	\begin{veta}[O spojitosti a derivování řady funkcí]
		Nechť $\sum_{n=1}^∞ u_n(x)$ je řada funkcí definovaná na intervalu $(a, b)$.

		\begin{itemize}
			\item Nechť $u_n$ jsou spojité na $(a, b)$ a $\sum_{n=1}^∞ u_n(x) \overset{\text{Loc}}{\rightrightarrows}$ na $(a, b)$. Pak $F(x) = \sum_{n=1}^∞ u_n(x)$ je spojitá na $(a, b)$.
			\item Nechť funkce $u_n$, $n \in ®N$, mají vlastní derivaci na $(a, b)$ a nechť $\exists x_0 \in (a, b): \sum_{n=1}^∞ u_n(x_0)$ konverguje a $\sum_{n=1}^∞ u_n'(x) \overset{\text{Loc}}{\rightrightarrows}$ na $(a, b)$. Pak je funkce $F(x) = \sum_{n=1}^∞ u_n(x)$ dobře definovaná a diferencovatelná a navíc $\sum_{n=1}^∞ u_n(x) \overset{\text{Loc}}{\rightrightarrows} F(x)$ a $\sum_{n=1}^∞ u_n'(x) \overset{\text{Loc}}{\rightrightarrows} F'(x)$ na $(a, b)$.
		\end{itemize}

		\begin{dukazin}
			„První bod“: Funkce $s_n(x) = \sum_{n=1}^k u_n(x)$ jsou spojité a $s_k \overset{\text{Loc}}{\rightrightarrows}$ na $(a, b)$. Tedy podle důsledku věty z dřívějška (stejnoměrná limita spojitých funkcí je spojitá) je jejich limita lokálně spojitá, tedy spojitá.

			„Druhý bod“: Na $s_k$ použijeme větu z dřívějška (pokud mají derivace stejnoměrnou limitu, pak i funkce ji mají a shoduje se až na derivaci). Ověříme podmínky, tedy že $s_k(x) = \sum_{n=1}^k u_n(x)$ konverguje a $s_k' = \sum_{n=1}^k u_k' \overset{\text{Loc}}{\rightrightarrows}$ na $(a, b)$. Podle tamté věty tedy $\exists F(x) = \lim_{k \rightarrow ∞} s_k(x) = \sum_{n=1}^∞ u_n(x)$ a tato funkce je diferencovatelná a
			$$ \sum_{n=1}^∞ u_n(x) \overset{\text{Loc}}{\rightrightarrows} F(x) \quad \land \quad \sum_{n=1}^∞ u_n'(x) \overset{\text{Loc}}{\rightrightarrows} F'(x) \qquad \text{na } (a, b). $$
		\end{dukazin}
	\end{veta}

	\begin{veta}[Abel-Dirichletovo kritérium pro stejnoměrnou konvergenci]
		Nechť $\{a_n(x)\}_{n=1}^∞$ je posloupnost funkcí definovaných na intervalu $J$ a nechť $\{b_n(x)\}_{n=1}^∞$ je posloupnost funkcí na $J$ taková, že $b_1(x) ≥ b_2(x) ≥ … ≥ 0$. Jestliže je splněna některá z následujících podmínek, pak $\sum_{n=1}^∞ a_n(x)·b_n(x) \rightrightarrows$ na $J$.

		\begin{itemize}
			\item[(A)] $\sum_{n=1}^∞ a_n(x) \rightrightarrows$ na $J$ a $b_1$ je omezená.
			\item [(D)] $b_n \rightrightarrows 0$ na $J$ a $\sum_{n=1}^∞ a_n(x)$ má omezené částečné součty, tedy
				$$ \exists K > 0\ \forall m \in ®N\ \forall x \in J: |s_n(x)| = \left|\sum_{i=1}^m a_i(x)\right| < K. $$
		\end{itemize}
		
		\begin{dukazin}
			„Dirichlet“: Nechť $\epsilon > 0$. Nalezneme $n_0$ $\forall n ≥ n_0\ \forall x \in J: |f_n(x)| < \epsilon$. Nechť $m, n ≥ n_0$. Označme $\sigma_i(x) := \sum_{j=m}^i a_j(x)$. Pak
			$$ |\sigma_i(x)| ≤ \left|\sum_{j=1}^i a_j(x) - \sum_{j=1}^{n_0-1} a_j(x)\right| ≤ K + K. $$
			$$ \forall m, n ≥ n_0\ \forall x \in J: \left| \sum_{j=n}^m a_j(x)·b_j(x) \right| = $$
			$$ = \left|a_n·b_n + (\sigma_{n+1} - \sigma_n)b_{n+1} + … + (\sigma_m - \sigma_{m-1})b_m\right| ≤ $$
			$$ ≤ \sup_{j = n, …, m} |\sigma_j(k)| \(b_n - b_{n-1} + b_{n-1} - b_{n-2} + … + b_{m-1} - b_m + b_m\) = $$
			$$ = \sup_{j=n, …, m} |\sigma_j(x)|·|b_n(x)| ≤ 2K·\epsilon. $$
			A z BC podmínky už $\sum a_i(x)b_i(x) \rightrightarrows$ na $J$.

			„Abel“: Nechť $\epsilon > 0$. Z BC podmínky pro $\rightrightarrows$
			$$ \exists n_0\ \forall m, n ≥ n_0: \left|\sum_{j=n}^m a_j(x)\right| < \epsilon. $$
			Tedy pro $\sigma_1(x) = \sum_{j=n}^m a_j(x)$ platí $|\sigma_i(x)| < \epsilon$. Analogicky odhadu výše
			$$ \left|\sum_{j=n}^m a_j(x)·b_j(x)\right| ≤ \sup_{j=n, …, m} |\sigma_j(x)|·|b_n(x)| ≤ \epsilon \sup_{x \in J}(b_1(x)) ≤ \epsilon · K. $$
			Tedy $\sum a_i(x)·b_i(x)$ splňuje BX podmínku.
		\end{dukazin}
	\end{veta}

% 08. 03. 2022

\section{Mocninné řady}
	\begin{definice}[Mocninná řada]
		Nechť $x_0 \in ®R$ a $a_n \in ®R$ pro $n \in ®N_0$. Řadu funkcí $\sum_{n=0}^∞ a_n(x - x_0)^n$ nazýváme mocninnou řadou s koeficienty $a_n$ a středem $x_0$.
	\end{definice}

	\begin{definice}[Poloměr konvergence]
		Poloměrem konvergence mocninné řady $\sum_{n=0}^∞ a_n·(x - x_0)^n$ nazveme
		$$ R = \sup\{r \in [0, ∞) | \sum_{n=0}^∞ a_n (x - x_0)^n \text{ konverguje } \forall x \in [x_0 - r, x_0 + r]\}. $$
	\end{definice}

	\begin{veta}[O poloměru konvergence mocninné řady]
		Nechť $\sum_{n=0}^∞ a_n·(x - x_0)^n$ je mocninná řada a $R \in [0, ∞]$ je její poloměr konvergence. Pak řada konverguje absolutně $\forall x \in (x_0 - R, x_0 + R)$ a diverguje $\forall x \in (-∞, x_0 - R) \cup (x_0 + r, +∞)$. Navíc platí $R = \frac{1}{\limsup \sqrt[n]{|a_n|}}$. Pokud existuje $Q = \lim_{n \rightarrow ∞} \frac{|a_n|}{|a_{n+1}|}$, potom $R = Q$.

		\begin{dukazin}
			Položme $R = \frac{1}{\limsup \sqrt[n]{|a_n|}}$. Pak pro $x: |x - x_0| < R$ platí
			$$ \limsup_{n \rightarrow ∞} \sqrt[n]{|a_n·(x - x_0)^n|} = |x - x_0| \limsup_{n \rightarrow ∞} \sqrt[n]{|a_n|} = \frac{|x - x_0|}{R} < 1 \implies \text{řada k. absolutně.} $$
			Pro $|x - x_0| > R$ dostaneme úplně stejně $> 1$, tedy řada diverguje.

			Nechť existuje $Q = \lim_{n \rightarrow ∞} \frac{|a_n|}{|a_{n+1}|}$, pak
			$$ \lim_{n \rightarrow ∞} \frac{|a_{n+1}·(x - x_0)^{n+1}|}{|a_n·(x - x_0)^n|} = |x - x_0|·\lim_{n \rightarrow ∞} \frac{|a_{n+1}|}{a_n} = |x - x_0|·\frac{1}{Q}. $$
			Pro $|x - x_0| < \frac{1}{Q}$ řada konverguje, pro $|x - x_0| > \frac{1}{Q}$ řada diverguje, tedy $\frac{1}{Q}$ je poloměr konvergence.
		\end{dukazin}
	\end{veta}

	\begin{veta}[O stejnoměrné konvergenci mocninné řady]
		Nechť $\sum_{n=0}^∞ a_n·(x - x_0)^n$ je mocninná řada a $R \in (0, ∞]$ je její poloměr konvergence. Pak řada konverguje lokálně stejnoměrně na $(x_0 - R, x_0 + R)$.

		\begin{dukazin}
			Nechť $0 < r < R$. Podle předchozí věty $\sum a_n·r^n$ konverguje absolutně. Nyní
			$$ \forall x \in [x_0 - r, x_0 + r]: |a_n(x - x_0)^n| ≤ |a_n|·r^n. $$
			Víme, že $\sum |a_n| r^n$ konverguje, tedy podle Weierstrassova kritéria $\sum a_n(x - x_0)^n \rightrightarrows$ na $[x_0 - r, x_0 + r]$. Tedy konverguje lokálně stejnoměrně na $(x_0 - R, x_0 + R)$.
		\end{dukazin}
	\end{veta}

	\begin{veta}[O derivaci mocninné řady]
		Nechť $\sum_{n=0}^∞ a_n·(x - x_0)^n$ je mocninná řada s poloměrem konvergence $R > 0$. Pak $\sum_{n=1}^∞ a_n·n·(x - x_0)^{n-1}$ je také mocninná řada se stejným středem a s poloměrem konvergence $R$.

		Navíc pro $x \in (x_0 - R, x_0 + R)$ platí $\(\sum_{n=0}^∞ a_n·(x - x_0)^n\)' = \sum_{n=1}^∞ a_n·n·(x - x_0)^{n-1}$.

		\begin{dukazin}
			Položme $R = \frac{1}{\limsup \sqrt[n]{|a_n|}}$. Nyní poloměr konvergence $\sum_{n=1}^∞ a_i·n·(x - x_0)^{n-1} \overset{x ≠ x_0}{=} \frac{\sum_{n=1}^∞ a_i·n·(x - x_0)^n}{x - x_0}$ je podle věty výše
			$$ \frac{1}{\limsup \sqrt[n]{|a_n|·n}} = R·\frac{1}{\limsup \sqrt[n]{n}} = R. $$
			Následně použijeme větu o derivaci a stejnoměrné konvergenci (v bodě $x = x_0$ řada jistě konverguje a z předchozí věty řada derivací konverguje lokálně stejnoměrně)
		\end{dukazin}
	\end{veta}

	\begin{dusledek}[O integrování mocninné řady]
		Nechť $\sum_{n=0}^∞ a_n·(x - x_0)^n$ je mocninná řada s poloměrem konvergence $R > 0$. Pak
		$$ \sum_{n=0}^∞ \frac{a_n}{n+1}·(x - x_0)^{n+1} $$
		je mocninná řada se stejným poloměrem konvergence. Navíc platí
		$$ \int \sum_{n=0}^∞ a_n(x - x_0)^n dx = \sum_{n=0}^∞ \frac{a_n}{n+1}·(x - x_0)^{n+1} + C \text{ na } (x_0 - R, x_0 + R). $$

		\begin{dukazin}[Hint k důkazu]
			Mocninou řadu vpravo zderivujeme.
		\end{dukazin}
	\end{dusledek}

% 15. 03. 2022

	\begin{veta}[Abelova]
		Nechť $\sum_{n=0}^∞ a_n (x - x_0)^n$ je mocninná řada s poloměrem konvergence $R > 0$. Nechť navíc $\sum_{n=0}^∞ a_n · R^n$ konverguje. Pak řada $\sum_{n=0}^∞ a_n·(x - x_0)^n$ konverguje stejnoměrně na $[x_0, x_0 + R]$ a platí $\sum_{n=0}^∞ a_n·R^n = \lim_{r \rightarrow R_-} \sum_{n=0}^∞ a_n · r^n$.

		\begin{dukazin}
			$$ \sum_{n=0}^∞ a_n · (x - x_0)^n = \sum_{n=0}^∞ \underbrace{a_n·R^n}_{a_n(x)}·\underbrace{\frac{(x - x_0)^n}{R^n}}_{b_n(x)}. $$
			Víme, že $b_n ≥ b_{n+1} ≥ 0$, jelikož $\frac{(x - x_0)^n}{R^n} ≥ \frac{(x - x_0)^{n+1}}{R^{n+1}} \Leftrightarrow 1 ≥ \frac{x - x_0}{R}$. Navíc $b_0 = 1$. Víme, že $\sum a_n·R^n$ konverguje, tedy podle BC podmínky pro konvergenci reálné řady:
			$$ \exists \epsilon > 0\ \exists n_0\ \forall m, n ≥ n_0: |\sum_{k=n}^m a_k·R^k| < \epsilon. $$
			Z toho ale jednoduše (jelikož $a_n(x)$ na $x$ nezávisí)
			$$ \forall \epsilon > 0\ \exists n_0\ \forall m, n ≥ n_0\ \forall x \in [x_0, x_0 + R]: |\sum_{k=n}^m a_k(x)| = |\sum_{k=n}^m a_k·R^k| < \epsilon. $$
			Tedy podle Abel-Dirichletova kritéria (části Abel) $\sum a_n(x - x_0)^n \leftleftarrows$ na $[x_0, x_0 + R]$.

			Funkce $a_n (x - x_0)^n$ jsou spojité a $\sum \leftleftarrows \implies F(x) = \sum_{n=0}^∞ a_n·(x - x_0)^n$ je spojitá na $[x_0, x_0 + R]$. Tedy
			$$ \lim_{x \rightarrow x_0 + R} F(x) = F(x_0 + R). $$
		\end{dukazin}
	\end{veta}

\section{Absolutně spojité funkce a funkce s konečnou variací}
	\begin{poznamka}
		Všechny integrály v této kapitole jsou Lebesgueovy.
	\end{poznamka}

	\subsection{Derivace monotónní funkce}
	\begin{definice}[Limes superior a limes inferior pro funkce]
		Nechť $x \in (a, b)$ a $f: (a, b) \rightarrow ®R$. Definujeme limes superior a limes inferior jako
		$$ \limsup_{h \rightarrow 0} f(x + h) := \lim_{h \rightarrow 0} \sup_{y \in (x - h, x) \cup (x, x + h)} f(y), \qquad \liminf_{h \rightarrow 0} f(x + h) := \lim_{h \rightarrow 0} \sup_{y \in (x - h, x) \cup (x, x + h)} f(y).$$
	\end{definice}

	\begin{poznamka}
		Analogicky jako u posloupnosti platí věta:
		$$ \exists \lim_{h \rightarrow 0} f(x + h) \in ®R \Leftrightarrow \limsup_{h \rightarrow 0} f(x + h) = \liminf_{h \rightarrow 0} f(x + h). $$
	\end{poznamka}

	\begin{definice}
		Nechť $I$ je interval, $x$ je vnitřní bod $I$ a $f: I \rightarrow ®R$ je funkce. Definujeme horní a dolní derivaci funkce $f$ v bodě $x$ jako
		$$ \overline{D} f(x) = \limsup_{h \rightarrow 0} \frac{f(x + h) - f(x)}{h}, $$
		$$ \underline{D} f(x) = \liminf_{h \rightarrow 0} \frac{f(x + h) - f(x)}{h}. $$
	\end{definice}

	\begin{veta}[Míra vzoru a obrazu]
		Nechť $I \subset ®R$ je interval. Nechť $f: I \rightarrow ®R$ je neklesající funkce, $M \subset I$ je měřitelná a $c > 0$.

		\begin{itemize}
			\item Je-li $\overline{D} f(x) > c$ na $M$, potom $©L^*(f(M)) \supseteq c·©L(M)$.
			\item Je-li $\underline{D} f(x) < c$ na $M$, potom $©L^*(f(M)) \subseteq c·©L(M)$.
		\end{itemize}

		\begin{dukazin}
			Bez důkazu.
		\end{dukazin}
	\end{veta}

	\begin{veta}[Derivace monotónní funkce]
		Nechť $I \subset ®R$ je interval a $f: I \rightarrow ®R$ je monotónní funkce. Potom ve skoro každém bodě $x \in I$ existuje $f'(x)$.

		\begin{dukazin}
			Nechť $M_{p, q} = \{x \in I | \underline{D}f(x) < p < q < \overline{D}f(x)\}$. Podle předchozí věty
			$$ q·©L(M_{p, q}) \subseteq ©L^*(f(M_{p, q})) \subseteq p·©L(M_{p, q}). $$
			Tedy, protože $p < q$, tak $©L(M_{p, q}) = 0$.

			Tvrdíme, že pro množinu $M$ bodů nediferencovatelnosti platí $M = \bigcup_{p, q \in Q, p < q} M_{p, q}$ $\implies$, tedy spočetné sjednocení nulových množin, tudíž $M$ je nulová: „$\supseteq$“: $x \in M_{p, q}, p < q \implies \underline{D}f(x) < \overline{D}f(x) \implies \nexists D f(x)$. „$\subseteq$“: Nechť $x \in M \implies \nexists D f(x) \implies \underline{D} f(x) < \overline{D}f(x)$.
		\end{dukazin}
	\end{veta}

% 22. 03. 2022

	\begin{veta}[Integrál derivace monotónní funkce]
		Nechť $a, b \in ®R$, $a < b$, a $f: [a, b] \rightarrow ®R$ je neklesající funkce. Potom $f'$ je lebesgueovsky integrovatelná na $[a, b]$ a platí
		$$ \int_a^b f'(x) dx ≤ f(b) - f(a). $$

		\begin{dukazin}
			$f$ je neklesající, tedy je měřitelná. Dodefinujeme $f(x) = f(b)$ pro $x > b$. Z předchozí věty víme, že pro skoro všechna $x$ $\exists \lim_{h \rightarrow 0} \frac{f(x + h) - f(x)}{h} = f'(x)$. Definujeme funkce $g_n(x) = \frac{f\(x + \frac{1}{n}\) - f(x)}{\frac{1}{n}}$. Tyto funkce jsou měřitelné a pro skoro všechna $x$ platí $\exists \lim_{n \rightarrow 0} g_n(x) = f'(x)$. Dále $f$ je neklesající, tedy $g_n(x) ≥ 0$ a $f'(x) ≥ 0$.

			Podle Fatouova lemmatu
			$$ \int_a^b f'(x) dx = \int_a^b \liminf_{n \rightarrow ∞} g_n(x) dx ≤ \liminf_{n \rightarrow ∞} \int_a^b g_n(x) dx = $$
			$$ = \liminf_{n \rightarrow ∞} \int_a^b n·(f\(x + \frac{1}{n}\) - f(x))dx = \liminf_{n \rightarrow ∞}\(n·\int_{a + \frac{1}{n}}^{b + \frac{1}{n}} f(t) dt - n·\int_a^b f(x) dx\) = $$
			$$ = \liminf_{n \rightarrow ∞}\(n·\int_b^{b + \frac{1}{n}} f(t) dt - n·\int_a^{a + \frac{1}{n}} f(x) dx\) ≤ \liminf_{n \rightarrow ∞} \(f(b) -  n·\int_a^{a + \frac{1}{n}} f(a) dx\) = $$
			$$ = f(b) - f(a). $$

			Tedy $f'$ je integrovatelná.
		\end{dukazin}
	\end{veta}

	\subsection{Funkce s konečnou variací}
	\begin{definice}[Kladná, záporná a totální variace]
		Nechť $[a, b] \subset ®R$ je uzavřený interval, $f: [a, b] \rightarrow ®R$ a $D$ dělení $[a, b]$. Definujeme kladnou variaci, zápornou variaci a (totální) variaci jako:
		$$ V^+(f, a, b) = \sup_D \{\sum_{i=1}^n (f(x_i) - f(x_{i-1}))^+\}, $$
		$$ V^-(f, a, b) = \sup_D \{\sum_{i=1}^n (f(x_i) - f(x_{i-1}))^-\}, $$
		$$ V(f, a, b) = \sup_D \{\sum_{i=1}^n |f(x_i) - f(x_{i-1})|\}. $$

		Dále zavedeme značení $V_f^+(x) = V^+(f, a, x)$, atd.
	\end{definice}

	\begin{definice}[Konečná variace]
		Řekneme, že funkce $f$ ma na intervalu $[a, b] \subset ®R$ konečnou variaci, jestliže $V(f, a, b) < ∞$. Množinu všech funkcí s konečnou variací značíme $BV([a, b])$.
	\end{definice}

	\begin{poznamka}
		Nechť $[a, b] \in ®R$ a $f: [a, b] \rightarrow ®R$. Pak

		\begin{itemize}
			\item je-li $f$ neklesající na $[a, b]$, pak $V(f, a, b) = V^+(f, a, b) = f(b) - f(a)$;
			\item $|V(f, a, b)| ≥ |f(a) - f(b)|$.
		\end{itemize}
	\end{poznamka}

	\begin{veta}[Vztah omezené variace a monotonie]
		Nechť $[a, b] \subset ®R$ a nechť $f: [a, b] \rightarrow ®R$.

		\begin{itemize}
			\item Má-li $f$ konečnou variaci na $[a, b]$, pak $V_f(x) = V_k^+ + V_f^-(x)$ a $f(x) - f(a) = V_f^+(x) - V_f^-(x)$.
			\item $f \subset BV(a, b)$ právě tehdy, když existují neklesající funkce $u, v: [a, b] \rightarrow ®R$ tak, že $f = v - u$.
		\end{itemize}

		\begin{dukazin}
			„První bod“: Nechť $D = \{a = x_0 < x_1 < … < x_n = b\}$. Búno stačí pro $x = b$.
			$$ V(f, a, b) ≥ \sum_{i=1}^n |f(x_i) - f(x_{i-1})| = $$
			$$ = \sum_{i=1}^n (f(x_i) - f(x_{i-1}))^+ + \sum_{i=1}^n (f(x_i) - f(x_{i-1}))^- ≤ V^+(f, a, b) + V^-(f, a, b). $$
			Z těchto nerovností vezmeme supremum přes všechna dělení $D$ a dostaneme $V(f, a, b) = V^+(f, a, b) + V^-(f, a, b)$ ($≥$ z nerovnosti mezi prvním a třetím výrazem, $≤$ z nerovnosti mezi druhým a čtvrtým).

			$$ f(b) - f(a) = \sum_{i=1}^n f(x_i) - f(x_{i-1}) = \sum_{i=1}^n (f(x_i) - f(x_{i-1}))^+ - \sum_{i=1}^n(f(x_i) - f(x_{i-1}))^- ≤ $$
			$$ ≤ V^+(f, a, b) - \sum_{i=1}^n(f(x_i) - f(x_{i-1}))^-. $$
			Infimum přes dělení $D$ dá $f(b) - f(a) ≤ V^+(f, a, b) - \sup \sum_{i=1}^n(f(x_i) - f(x_{i-1}))^- = V^+(f, a, b) - V^-(f, a, b)$.
			$$ f(b) - f(a) = \sum_{i=1}^n f(x_i) - f(x_{i-1}) = \sum_{i=1}^n (f(x_i) - f(x_{i-1}))^+ - \sum_{i=1}^n(f(x_i) - f(x_{i-1}))^- ≥ $$
			$$ ≥ \sum_{i=1}^n(f(x_i) - f(x_{i-1}))^+ - V^-(f, a, b). $$
			Supremum přes dělení $D$ dá $f(b) - f(a) ≥ \sup \sum_{i=1}^n(f(x_i) - f(x_{i-1}))^+ - V^-(f, a, b) = V^+(f, a, b) - V^-(f, a, b)$.

			„Druhý bod“: „$\implies$“: Z prvního bodu víme, že $f(x) = (f(a) + V^+(f, a, x)) - V^-(f, a, x) = v(x) - u(x)$.

			„$\impliedby$“: Mějme tedy $f(x) = v(x) - u(x)$
			$$ \sum_{i=1}^n (f(x_i) - f(x_{i - 1})) ≤ \sum_{i=1}^n |v(x_i) - v(x_{i-1})| + \sum_{i=1}^n |u(x_i) - u(x_{i-1})| = v(b) - v(a) + u(b) - u(a), $$
			což dá $f \in BV(a, b)$.
		\end{dukazin}
	\end{veta}

	\begin{dusledek}
		$f \in BV \implies f$ má derivaci skoro všude.

		\begin{dukazin}
			Z předchozí věty $f = v - u$ a věty před ní mají $u, v$ derivace skoro všude.
		\end{dukazin}
	\end{dusledek}

% 29. 03. 2022

	\subsection{Absolutně spojitá funkce}
	\begin{definice}[Absolutně spojitá funkce]
		Nechť $[a, b] \subset ®R$ a $f: [a, b] \rightarrow ®R$. Řekneme, že $f$ je absolutně spojitá na $[a, b]$, jestliže $\forall \epsilon > 0\ \exists \delta > 0:$ pro každý systém po dvou disjunktních intervalů $\{(a_j, b_j)\}_{j=1}^n$, $(a_j, b_j \subset [a, b])$ splňující $\sum_{j=1}^n (b_j - a_j) < \delta \implies \sum_{j=1}^n |f(b_j) - f(a_j)| < \epsilon$. Množinu absolutně spojitých funkcí na intervalu $[a, b]$ budeme značit $AC([a, b])$.
	\end{definice}

	\begin{poznamka}
		$f \in Lip([a, b]) \implies f \in AC([a, b]) \implies f \in BC([a, b]) \cap C([a, b])$ a žádnou implikaci nelze obrátit.

		$AC([a, b])$ je lineární prostor.

		$f \in AC \implies V^+ \in AC \land V^- \in AC$.
	\end{poznamka}

	\begin{veta}[Integrál derivace absolutně spojité funkce]
		Nechť $f \in AC([a, b])$. Potom $f' \in L^1([a, b])$ a $f(b) - f(a) = \int_a^b f'(x) dx$.

		\begin{dukazin}
			BÚNO $f$ je rostoucí ($f \in AC \implies f \in BV \implies f(x) = (v(x) + x) - (u(x) + x)$, $v(x) + x, u(x) + x$ jsou rostoucí ($u, v$ neklesající, ale my potřebujeme rostoucí, proto $+x$) a $AC$). Podle věty výše víme, že $\exists f'(x)$ skoro všude. Tedy $[a, b]$ si můžeme rozdělit na tři množiny – s kladnou $f'$ ($D$), s nulovou $f'$ ($Z$) a bez $f'$ ($N$).

			Dokážeme, že $|f(N)| = 0$, $|f(Z)| = 0$, $\int_D f'(x) = |f(D)|$. Pak $\int_a^b f'(x) = \int_D f'(x) dx = |f(D)| = |f(D) \cup f(Z) \cup f(N)| = f(b) - f(a)$, neboť $f$ je rostoucí, tedy prostá.

			„$|f(N)| = 0$“: Víme $|N| = 0$. K $\epsilon > 0$ nalezneme $\delta > 0$ z definice $AC$. K $N$ nalezneme otevřenou množinu $G$ tak, že $N \subset G$ a $|G| < \delta$. Také $G = \bigcup_{i=1}^∞ (a_i, b_i)$ (resp. $\bigcup_{i =1}^N (a_i, b_i)$), kde $(a_i, b_i)$ jsou po dvou disjunktní. Nyní $\forall m \in ®N: \sum_{i=1}^m b_i - a_i ≤ |G| < \delta \implies \sum_{i=1}^m f(b_i) - f(a_i) < \epsilon$. Nakonec pro $m \rightarrow ∞$ je $\sum_{i=1}^∞ |f(b_i) - f(a_i)| ≤ \epsilon$ a $f(N) \subset \bigcup_{i=1}^∞ f((a_i, b_i))$, tudíž $\sum_{i=1}^∞ f(b_i) - f(a_i) ≤ \epsilon$.

			„$|f(Z)| = 0$“: Nechť $\epsilon > 0$ a víme, že $f'(x) < \epsilon$ na $Z$. Podle věty výše $|f(Z)| ≤ \epsilon · |Z| ≤ \epsilon · (b - a)$, tudíž $|f(Z)| = 0$.

			„$\int_D f'(x) = |f(D)|$“: Nechť $\tau > 1$. Označme $D_k = \{x|T^k ≤ f'(x) < \tau^{k+1}\}$. Pak $D = \bigcup_{k \in ®Z} D_k$. Podle věty výše
			$$ \frac{1}{\tau} \int_{D_k} f'(x) dx ≤ \tau^k |D_k| ≤ |f(D_k)| ≤ \tau^{k + 1} |D_k| ≤ \tau \int_{D_k} f' dx $$
			Posčítáme a získáme $\frac{1}{\tau} \int_D f'(x) dx ≤ |f(D)| ≤ \tau \int_D f'(x) dx$. Limita přes $\tau$ nám dá požadovanou rovnost.
		\end{dukazin}
	\end{veta}

	\begin{veta}[Neurčitý Lebesgueův integrál]
		Nechť $\Theta \in L^1(a, b)$ a $f$ je neurčitý Lebesgueův integrál $\Theta$, tj. existuje konstanta $C$, že
		$$ f(x) = \int_a^x \Theta(t) dt + C, \qquad \forall x \in [a, b]. $$
		Pak $f \in AC$ a $f' = \Theta$ skoro všude.

		\begin{tvrzeni}[Pozorování z TMI]
			$$ \Theta \in L^1\ \forall \epsilon > 0\ \exists \delta\ \forall M, |M| < \delta \implies \int_M |\Theta| < \epsilon. $$
		\end{tvrzeni}

		\begin{dukazin}
			„$f \in AC$“: K $\epsilon > 0$ zvolme $\delta > 0$ z pozorování z TMI. Nechť $(a_j, b_j)$ jsou po dvou disjunktní a $\sum_{j=1}^n (b_j - a_j) < \delta$. Nyní
			$$ \sum_{j=1}^n |f(b_j) - f(a_j)| = \sum_{j=1}^n |\int_{a_j}^{b_j} \Theta(t) dt| ≤ \int_{\bigcup (a_j, b_j)} |\Theta(t)| dt < \epsilon \implies f \in AC. $$

			„$f' = \Theta$ skoro všude“: Víme z předchozí věty, že $f(x) - f(a) = \int_a^x f'(t) dt$. Také víme $f(x) - C = \int_a^x \Theta(t) dt$. Tedy $C - f(a) = \int_a^x (f'(t) - \Theta(t)) dt$. Při $x = a$ máme $C = f(a)$, tedy $\forall x \in [a, b]: \int_a^x (f'(t) - \Theta(t)) dt$. Z TMI víme, že $f'(t) = \Theta(t)$ skoro všude.
		\end{dukazin}
	\end{veta}

	\begin{dusledek}
		$$ f \in AC ([a, b]) \Leftrightarrow \exists \Theta \in L^1(a, b), f(x) = \int_a^x \Theta(t) dt + C. $$
	\end{dusledek}

% 05. 04. 2022

TODO (Chyběl jsem)

% 12. 04. 2022

\section{Fourierovy řady}
	\subsection{Bodová konvergence Fourierových řad}
	\begin{definice}[Dirichletovo jádro]
		Nechť $n \in ®N \setminus \{0\}$. Potom funkce $D_n(x) = \frac{1}{2} + \cos x + … + \cos n · x$ nazýváme Dirichletovým jádrem.
	\end{definice}

	\begin{veta}[O částečných součtech Fourierovy řady]
		Nechť $f \in P_{2\pi}$ a $n \in ®N \setminus \{0\}$. Potom pro každé $x \in ®R$ platí
		$$ S_n f(x) = \frac{1}{\pi} \int_{-\pi}^\pi f(x + y) D_n(y) dy = \frac{1}{\pi} · \int_0^\pi (f(x + y) + f(x - y)) D_n(y) dy. $$

		\begin{dukazin}
			$$ S_n f(x) = \frac{a_0}{2} + \sum_{k=1}^n a_k · \cos(k·x) + b_k · \sin(k·x) = $$
			$$ = \frac{1}{2 \pi} \int_{-\pi}^\pi f(x) + \sum_{k=1}^n \frac{1}{\pi} · \int_{-\pi}^\pi f(t)·\cos(k·t) dt · \cos(k·x) + \frac{1}{\pi} \int_{-\pi}^\pi f(t) \sin(k·t) dt · \sin(k·t) = $$
			$$ = \frac{1}{\pi} \int_{-\pi}^\pi f(t)·\(\frac{1}{2} + \sum_{k=1}^n (\cos(k·t) \cos(k·x) + \sin(k·t)\sin(k·x))\) dt = $$
			$$ = \frac{1}{\pi} \int_{-\pi}^\pi f(t)·\(\frac{1}{2} + \sum_{k=1}^n \cos(k·t - k·x)\)dt = \frac{1}{\pi} \int_{-\pi}^\pi f(t) · D_n(t - x) dt = $$
			$$ = \frac{1}{\pi} · \int_{-\pi - x}^{\pi - x} f(y + x)·D_n(y) dy = \frac{1}{\pi} \int_{-\pi}^\pi f(y + x) D_n(y) dy = $$
			$$ = \frac{1}{\pi} \(\int_0^\pi + \int_{-\pi}^0\) = \frac{1}{\pi}·\int_0^\pi (f(x + y) + f(x - y))·D_n(y) dy. $$
		\end{dukazin}
	\end{veta}

	\begin{veta}[Riemannovo-Lebesgueovo lemma]
		Nechť $(a, b) \subset ®R$ je omezený interval a nechť $f \in L^1(a, b)$. Potom $\lim_{t \rightarrow ∞} \int_a^b f(x)·\cos(t·x) dx = 0$ a $\lim_{t \rightarrow ∞} \int_a^b f(x) \sin(t·x) dx = 0$.

		\begin{dukazin}
			Pokud se nepletu, tak elegantnější důkaz je v \verb|OM3/Funkcionalka/Funkcionalka.pdf str 69|.

			$$ 1. \lim_{t \rightarrow ∞} \int_a^b \chi_{(c, d)}(x)·\cos(t·x) dx = \lim_{t \rightarrow ∞} \int_c^d \cos(t·x) dx = $$
			$$ = \lim_{t\rightarrow ∞} \[\frac{\sin(t·x)}{t}\]_c^d = \lim_{t \rightarrow ∞} \frac{\sin(t·d) - \sin(t·c)}{t} = 0. $$

			$$ 2. G = \bigcup_{j=1}^∞(c_j, d_j) \implies \forall \epsilon > 0\ \exists j_0 |\bigcup_{j=j_0}^∞ (c_j, d_j)| < \epsilon. $$
			$$ |\int_a^b \chi_G(x) \cos(t·x) dx| ≤ 0 + \int_a^b \sum_{j=j_0 + 1}^∞ \chi_{(c_j, d_j)}(x) dx < \epsilon. $$

			$$ 3. f = \chi_E, E \in ©B \implies \forall \epsilon > 0\ \exists G \text{ otevřená}: E \subset G \land |G \setminus E| < \epsilon $$
			$$ |\int_a^b \chi_E \cos(t·x) dx| ≤ |\int_a^b \chi_G(x) \cos(t·x) dx| + |\int_a^b \chi_{G \setminus E}(x) · \cos(t·x) dx| ≤ 0 + \epsilon. $$

			4. Pro jednoduchou funkci je to triviální, neboť je to konečný součet 3.

			5. K $\epsilon > 0$ $\exists s$ jednoduchá, že $\int_a^b |f(x) - s(x)| dx < \epsilon$.
			$$ |\int_a^b f(x)·\cos(t·x) dx| ≤ |\int_a^b s(x) \cos(t·x) dx| + |\int_a^b(s(x) - f(x))·\cos(t·x) dx| \rightarrow 0 + \epsilon \rightarrow 0. $$
		\end{dukazin}
	\end{veta}

	\begin{veta}[Riemannova věta o lokalizaci]
		Nechť $f \in P_{2\pi}$, $x \in ®R$ a $s \in ®R$. Potom $S f(x) = s \Leftrightarrow \exists \delta \in (0, \pi)$ tak, že
		$$ \lim_{n \rightarrow ∞} \int_0^\delta(f(x + t) + f(x - t) - 2s)·D_n(t) dt = 0. $$

		\begin{dukazin}
			Z věty výše víme, že $S_n f(x) = \frac{1}{\pi} \int_0^\pi (f(x + y) + f(x - y))·D_n(y) dy$. Dále z vlastností Dirichletova jádra
			$$ s = \frac{1}{\pi} \int_0^\pi D_n(y)·2s dy. $$
			Chceme $0 = \lim_{n \rightarrow ∞} (S_n f(x) - s) = \lim_{n \rightarrow ∞} \frac{1}{\pi}· \int_0^\pi (f(x + y) + f(x - y) - 2s) D_n(y) dy =$
			$$ = \lim_{n \rightarrow ∞} \frac{1}{\pi}· \int_0^\delta (f(x + y) + f(x - y) - 2s)·D_n(y) dy + \lim_{n \rightarrow ∞} \frac{1}{\pi} \int_\delta^\pi … dy. $$
			Stačí ukázat, že druhý integrál je roven nule. Z vlastností $D_n$ je
			$$ \lim_{n \rightarrow ∞} \int_\delta^\pi (f(x + y) + f(x - y) - 2s) \frac{\sin((n + \frac{1}{2})·y)}{2·\sin \frac{y}{2}}dy = $$
			$$ = \lim_{n \rightarrow ∞} \int_\delta^\pi \frac{f(x + y) + f(x - y) - 2s}{2 \sin \frac{y}{2}} \sin((n + \frac{1}{2})y) dy $$
			Pokud je první činitel $L_1$, pak je toto rovno 0 z Riemannova-Lebesgueova lemmatu. Ale první činitel je integrovatelný, neboť $f(x + y)$, $f(x - y)$ i $2s$ jsou integrovatelné a $\sin \frac{y}{2} > \sin \frac{\delta}{2}$.
		\end{dukazin}
	\end{veta}

% 19. 04. 2022

	\begin{definice}[Značení]
		Nechť $x \in ®R$ a $f$ je reálná funkce na okolí $x$. Značíme $f(x+) = \lim_{t \rightarrow x_+} f(t)$ a $f(x-) = \lim_{t \rightarrow x_-} f(t)$
	\end{definice}

	\begin{veta}[Diniovo kritérium]
		Nechť $f \in P_{2\pi}$ a $x \in ®R$. Nechť existují vlastní limity $f(x+)$ a $f(x-)$ a nechť existují limity
		$$ \lim_{t \rightarrow 0+} \frac{f(x + t) - f(x+)}{t}, \qquad \lim_{t \rightarrow 0+} \frac{f(x - t) - f(x-)}{t}. $$
		Potom řada $S_f$ konverguje v $x$ a platí $S_f(x) = \frac{f(x+) + f(x-)}{2}$. Speciálně má-li $f$ konečné jednostranné derivace v $x$, pak $S_f(x) = f(x)$.

		\begin{dukazin}
			Podle předchozí věty stačí ukázat
			$$ \exists \delta > 0: 0 = \lim_{n \rightarrow ∞} \int_0^\delta (f(x + t) + f(x - t) - (f(x+) + f(x-)))·D_n(t)dt = $$
			$$ = \lim_{n \rightarrow ∞} \int_0^\delta \(\frac{f(x + t) - f(x+)}{t} + \frac{f(x - t) - f(x-)}{t}\) \frac{t}{2\sin \frac{t}{2}}·\sin((n + \frac{1}{2})·t) dt $$
			Z definice limity jistě existuje $\delta > 0$ tak, že $\frac{f(x + t) - f(x+)}{t} + \frac{f(x - t) - f(x-)}{t}$ je omezená. Dále $\frac{t}{2 \sin \frac{t}{2}}$ je omezená na $[0, \pi]$. Takže součin (značme ho $F$) těchto dvou funkcí je v $©L^1$. Podle Riemann-Lebesgueova lemmatu je $\lim_{n \rightarrow ∞} \int_0^\delta F(t)·\sin((n + \frac{1}{2})·t) dt = 0$.
		\end{dukazin}
	\end{veta}

	\begin{veta}[Jordan-Dirichletovo kritérium]
		Nechť $f \in P_{2\pi}$ a nechť $f \in BV([0, 2\pi])$. Potom
		
		\begin{itemize}
			\item pro každé $x \in [0, 2\pi]$ Fourierova řada konverguje a $S_f(x) = \frac{f(x+) + f(x-)}{2}$,
			\item je-li $f$ navíc spojitá na $(a, b) \subset [0, 2\pi]$, pak $S_n f \overset{\text{Loc}}\rightrightarrows f$ na $[a, b]$.
		\end{itemize}
	\end{veta}

	\subsection{Stejnoměrná konvergence -- Fejérova věta}
	\begin{definice}[Konvergence v Cesarově smyslu]
		Nechť $\{a_n\}_{n=1}^∞$ je posloupnost reálných čísel. Řekneme, že $a_n$ konverguje k $a \in ®R$ v Cesarově [Čézarově] smyslu, pokud $\sigma_n = \frac{a_0 + … + a_n}{n + 1} \rightarrow a$.
	\end{definice}

	\begin{definice}[Fejérovo jádro]
		Nechť $n \in ®N_0$. Potom funkci $K_n(x) = \frac{1}{n+1}·\sum_{k=0}^n D_k(x)$ nazýváme Fejérovým jádrem.

		\begin{poznamkain}
			$K_n$ je sudá, $2\pi$-periodická a $K_n(0) = \frac{n+1}{2}$. Platí $\int_{-\pi}^\pi K_n(x) = \pi$.
			$$ K_n (x) = \frac{1}{2(n+1)} \(\frac{\sin((n + 1)\frac{x}{2})}{\sin \frac{x}{2}}\)^2, \qquad x \in ®R, \lambda ≠ 2k\pi, k \in ®Z. (\text{To se ukáže indukcí.}) $$
		\end{poznamkain}
	\end{definice}

	\begin{definice}[Částečné Fejérovy součty]
		Nechť $f \in P_{2\pi}$, $x \in ®R$ a $n \in ®N_0$. Pak výraz
		$$ \sigma_n f(x) = \frac{1}{n+1} · \sum_{k_0}^n S_k f(x) $$
		nazýváme $n$-tý částečný Fejérův součet $f$.

		\begin{poznamkain}
			To se z věty výše rovná $\frac{1}{n+1} · \sum_{k=0}^n \frac{1}{\pi} \int_{-\pi}^\pi f(x + y)·D_k(y) dy$.
		\end{poznamkain}
	\end{definice}

	\begin{veta}[Fejérova]
		Nechť $f \in P_{2\pi}$.
		\begin{itemize}
			\item Jestliže pro nějaké $x \in ®R$ existují vlastní limity $f(x+)$ a $f(x-)$, pak
				$$ \lim_{n \rightarrow ∞} \sigma_n f(x) = \frac{f(x+) + f(x-)}{2}. $$
			\item Je-li $f$ spojitá na $(a, b)$, pak $\sigma_n f \overset{\text{Loc}}\rightrightarrows f$ na $(a, b)$.
		\end{itemize}

		\begin{dukazin}[1. bod]
			Podle poznámky
			$$ \sigma_n f(x) = \frac{1}{\pi} \int_0^\pi (f(x + t) + f(x - t))·K_n(t) dt $$
			a $s = \frac{1}{\pi} · \int_0^\pi 2s · K_n(t) dt$. Tedy
			$$ s_n f(x) - s = \frac{1}{\pi} \int_0^\pi (f(x + t) + f(x - t) - 2s) K_n(t) dt = \frac{1}{\pi} \(\int_0^\delta + \int_\delta^\pi\). $$

			K $\epsilon > 0\ \exists \delta > 0\ \forall t \in (0, \delta): |f(x + t) + f(x - t) - (f(x+) + f(x-))| < \epsilon$.
			$$ |\frac{1}{\pi} \int_0^\pi| ≤ \frac{1}{\pi} \int_0^\delta \epsilon · |K_n(t)| dt ≤ \frac{1}{\pi} \int_0^\pi \epsilon · K_n(t) dt = \frac{\epsilon}{2}. $$
			$$ |\frac{1}{\pi} \int_\delta^\pi| ≤ \frac{1}{\pi} \int_0^\pi | f(x + t) + f(x - t) - (f(x+) + f(x-))| dt · \frac{1}{2(n+1)}·\frac{1}{\sin^2 \frac{\delta}{2}} \rightarrow 0. $$
		\end{dukazin}

		\begin{dukazin}[2. bod]
			Nechť $[c, d] \subset (a, b)$. Chceme $\sigma_n f(x) \rightrightarrows f(x)$ na $[c, d]$, tedy $|\sigma_n f(x) - f(x)| ≤$ něco malého, co nezávisí na $x \in [c, d]$. Nalezneme $\gamma > 0$, aby $[c - \gamma, d + \gamma] \subset (a, b)$. Ze stejnoměrné spojitosti $f$ na $[c - \gamma, d + \gamma]$ k $\epsilon > 0$ nalezneme $\delta > 0$ tak, že
			$$ \forall s, t \in [c - \gamma, d + \gamma]: |s - t| < \delta \implies |f(s) - f(t)| < \epsilon. $$
			Tedy $\forall x \in [c, d]$ $\forall t \in (0, \gamma)$
			$$ |f(x + t) + f(x - t) - 2f(x)| ≤ |f(x + t) - f(x)| + |f(x - t) - f(x)| < 2\epsilon. $$

			Analogicky prvnímu bodu:
			$$ |\sigma_n f(x) - f(x)| = |\frac{1}{\pi} \(\int_0^\delta + \int_\delta^\pi\)(f(x + t) + f(x - t) - 2f(x))·K_n(t)dt|. $$
			$$ |\frac{1}{\pi} \int_0^\delta| ≤ \frac{1}{\pi} \int_0^\delta 2\epsilon · K_n(t) dt ≤ \epsilon, $$
			$$ |\frac{1}{\pi} \int_\delta^\pi| ≤ \frac{1}{\pi} \int_\delta^\pi |f(x + t) + f(x - t) + 2f(x)|dt \frac{1}{2(n+1)} \frac{1}{\sin^2 \frac{\delta}{2}} ≤ $$
			$$ ≤ \frac{1}{\pi} \int_0^{2\pi} |f(x + t)| + |f(x - t)| + 2M dt · \frac{1}{2(n + 1)} \frac{1}{\sin \frac{\delta}{2}} ≤ $$
			$$ ≤ \frac{1}{\pi} (2\int_0^{2\pi} |f(t)|dt + 2\pi 2M)· \frac{1}{2(n + 1)} \frac{1}{\sin \frac{\delta}{2}} \rightarrow 0 $$
			($f$ je spojitá na $[c, d] \implies |f(x)| ≤ M$.)
		\end{dukazin}
	\end{veta}

% 26. 04. 2022

	\begin{veta}[Weierstrassova -- trigonometrická verze]
		Nechť $f \in P_{2\pi}$ je spojitá na ®R a nechť $\epsilon > 0$. Pak existuje trigonometrický polynom $T$ splňující $||f - T||_{C(®R)} < \epsilon$.

		\begin{dukazin}
			Z Fejerovy věty (2. bod).
		\end{dukazin}
	\end{veta}

	\begin{dusledek}[Weierstrass]
		Nechť $f \in C([a, b])$ a $\epsilon > 0$. Potom existuje polynom $P$ tak, že $||f - P||_{C([a, b])} < \epsilon$.

		\begin{dukazin}[Šel by udělat přes komplexní čísla, ale zde je důkaz bez nich.]
			Z Taylorovy věty víme, že $\forall [c, d]\ \forall \epsilon > 0\ \exists$ polynomy $P, Q: |\sin x - P(x)| < \epsilon$ a $|\cos x - Q(x)| < \epsilon$ na $[c, d]$. BÚNO $[a, b] = [0, 2\pi]$ a $f$ je periodická.

			Z předchozí věty nalezneme trigonometrický polynom, že $||f - T|| < \epsilon$. Nyní $T = a_0 + \sum_{k=1}^n a_k \cos kx + b_k \sin k x$. Nalezneme polynomy $P_k$ a $Q_k$, aby $|\cos k x - P_k| < \frac{\epsilon}{2^k}·\frac{1}{|a_k| + 1}$, $|\sin k x - Q_k| < \frac{\epsilon}{2^k}·\frac{1}{|b_k| + 1}$. Položme $P = a_0 + \sum_{k=1}^n a_k·P_k + b_k·Q_k$. $P$ je polynom.
			$$ ||P - f|| ≤ ||f - T|| + ||T - P|| < \epsilon + ||\sum_{k=1}^n a_k·(\cos k x - P_k) + b_k(\sin k x - Q_k)|| < $$
			$$ < \epsilon + \sum_{k=1}^n |a_k|·\frac{\epsilon}{2^k}·\frac{1}{|a_k| + 1} + |b_k|·\frac{\epsilon}{2^k}·\frac{1}{|b_k| + 1} < 3·\epsilon $$
		\end{dukazin}
	\end{dusledek}

	\begin{veta}[Fourierovy koeficienty určují funkci]
		Nechť $f, g \in P_{2\pi}$ mají stejné Fourierovy koeficienty. Potom $f = g$ skoro všude.

		\begin{dukazin}
			Funkce $f - g$ má nulové Fourierovy koeficienty. Tedy BÚNO $f ≠ 0$, $g = 0$ a $\int_0^{2 \pi} f(x) \cos n x dx = 0$ a $\int_0^{2\pi} f(x) \sin n x dx = 0$ $\forall n \in ®N_0$. Označme $©T := \{\phi \in L^∞ (0, 2\pi) | \int_0^{2\pi} f(x) · \phi(x) dx = 0\}$.

			1. ©T je zřejmě lineární prostor. 2. Nechť $\phi_n \in ©T$, $\phi_n \rightarrow \phi$ skoro všude a $||\phi_n||_∞ ≤ K\ \forall n$. Pak $\phi \in ©T$ z Lebesgueovy věty. 3. Trigonometrické polynomy $\subset ©T$, neboť $\cos n x \in ©T$, $\sin n x \in ©T$ a použijeme jedničku. 4. Nechť $\phi$ je spojitá, $\phi(0) = \phi(2\pi) = 0$. Podle předchozí věty existuje trigonometrický polynom $T_n \rightrightarrows \phi$. Tedy $\phi \in ©T$ z druhého bodu. 5. $[a, b] \subset (0, 2\pi)$, pak $\chi_{(a, b)} \in ©T$, jelikož k ní konvergují „lichoběžníky“. 6. $G \subset [0, 2\pi]$ otevřená, potom $\chi_G \in ©T$, neboť $G = \bigcup_{i=1}^∞ (a_i, b_i)$ a $\sum_{i=1}^k \chi_{(a_i, b_i)} \rightarrow \chi_G$ skoro všude a sumy jsou stejně omezené $K = 1$. 7. $E \subset (0, 2\pi)$ měřitelná, pak $\chi_E \in ©T$, neboť pro každé $n$ existuje $G_n$ otevřená, že $E \subset G_n \land |G_n \setminus E| < \frac{1}{2^n}$ (z TMI).

			8. $E_1 := \{f > 0\}$ a $E_2 := \{f < 0\}$:
			$$ \int_0^{2 \pi} |f| = \int_{E_1} f - \int_{E_2} f = \int_0^{2\pi} \chi_{E_1} f - \int_0^{2\pi} \chi_{E_2} f = 0 \implies f = 0 \text{ skoro všude}. $$
		\end{dukazin}
	\end{veta}

	\begin{dusledek}
		Trigonometrické funkce tvoří bázi prostoru $L^2(0, 2\pi)$.

		\begin{dukazin}
			Nechť pro spor BÚNO $f ≠ 0$ a $\<f, \sin n x\> = 0$ a $\<f, \cos n x\> = 0$ $\forall n$. Pak je ale $f ≡ 0$ z předchozí věty.
		\end{dukazin}
	\end{dusledek}

	\begin{poznamka}[Komplexní zápis]
		$$ S f(x) = c_0 + \sum_{k=1}^∞ (c_k ·e^{i k x} + c_{-k}·e^{-i k x}), $$
		$$ c_k = \frac{1}{2\pi} \int_0^{2\pi} f(x) e^{-i k x} dx, \qquad k \in ®Z. $$
	\end{poznamka}

% 03. 05. 2022

\section{Fourierova transformace}
\begin{definice}[Fourierova transformace, inverzní Fourierova transformace]
	Nechť $f \in ©L^1(®R)$. Pak Fourierova transformace je definována jako
	$$ \hat{f}(\omega) = \frac{1}{\sqrt{2 \omega}}·\int_{-∞}^{+∞} f(x) e^{-i \omega x}dx. $$
	A inverzní Fourierova transformace $f$ je definována jako
	$$ \check f (\omega) = \frac{1}{\sqrt{2\pi}} · \int_{-∞}^{-∞} f(x) e^{i \omega x} dx. $$
\end{definice}

\begin{poznamka}
	Pro $f \in ©L^2$ platí $\check{\hat{f}} = f$ ve smyslu rovnosti $©L^2$ funkcí.

	Existuje-li vlastní derivace $f$, pak $\check{\hat{f}} = f$.

	Je-li $f \in BV$, pak $\check{\hat{f}}(x) = \frac{f(x+) + f(x-)}{2}$, kde $\check{f}$ je limita pro meze jdoucí do nekonečna.
\end{poznamka}

\begin{definice}[Konvoluce, nezkouší se]
	Nechť $f, g \in L^1(®R)$. Pak konvoluce funkcí $f$ a $g$ je definována jako
	$$ (f * g)(x) = \int_{-∞}^{+∞} f(x - y) g(y) dy. $$
\end{definice}

\begin{veta}[Fourireova transformace konvoluce]
	Nechť $f, g \in L^1(®R)$. Pak $\widehat{(f*g)}(u) = \sqrt{2 \pi}·\hat{f}(\omega)·\hat{g}(\omega)$.

	\begin{poznamkain}
		Analogicky lze dokázat
		$$ \sqrt{2\pi} \widecheck{(f·g)}(x) = (\check f * \check g)(x). $$
	\end{poznamkain}

	\begin{poznamkain}
		$f, g \in ©L^1 \implies f * g \in ©L^1$. (Důkaz rozepsáním a prohozením integrálů.)
	\end{poznamkain}

	\begin{dukazin}
		$$ \widehat{(f * g)} (\omega) = \frac{1}{\sqrt{2 \pi}} \int_{-∞}^{+∞} \(\int_{-∞}^{+∞} f(x - y) g(y) dy\) e^{-i\omega x} dx = $$
		(Pomocí Fubiniovy věty, integrovatelnost máme z poznámky.)
		$$ \frac{1}{\sqrt{2 \pi}} \int_{-∞}^{+∞} g(y) \(\int_{-∞}^{+∞} f(x - y) · e^{-i \omega (x - y)}\)e^{- i \omega y} dy = $$
		$$ = \int_{-∞}^{+∞} g(y) · e^{-i\omega y}·\hat{f}(\omega) dy = \sqrt{2 \pi}·\hat{f}(\omega)·\hat{g}(\omega). $$
	\end{dukazin}
\end{veta}

\begin{veta}[Fourierova transformace derivace]
	Nechť $f \in ©L^1(®R)$, $f$ a $f'$ spojité na ®R, $\lim_{|x| \rightarrow ∞} f(x) = 0$ a $f' \in ©L^{1}(®R)$. Pak
	$$ \widehat{f'}(\omega) = i·\omega·\hat{f}(\omega). $$

	\begin{dukazin}
		$$ \widehat{f'}(\omega) = \frac{1}{\sqrt{2 \pi}}\int_{-∞}^∞ f'(x)·e^{-i\omega x} \overset{per partes}= 0 - \frac{1}{\sqrt{2\pi}} \int_{-∞}^{∞} -i\omega e^{-i \omega x} · f(x) dx = i\omega \hat{f}(\omega). $$
	\end{dukazin}
\end{veta}

\end{document}

