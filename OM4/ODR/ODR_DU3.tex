\documentclass[12pt]{article}					% Začátek dokumentu
\usepackage{../../MFFStyle}					    % Import stylu

\begin{document}

\begin{priklad}
	Ukažte, že matice
	$$ A(t) = \frac{1}{4} \begin{pmatrix} 3 \cos 2t - 1 & 4 - 3 \sin 2t \\ -4 - 3 \sin 2t & -1 -3\cos 2t \end{pmatrix}  $$
	má vlastní čísla $\frac{1}{4}(-1 ± \sqrt{7} i)$ pro každé $t$.

	\begin{dukazin}
		Chceme aby determinant $(A - \lambda I)$ byl nulový. Ten je roven
		$$ 0 = \(\frac{1}{4}(3\cos 2t - 1) - \lambda\)\(\frac{1}{4}(-1 - 3\cos 2t) - \lambda\) - \frac{1}{16}(4 - 3 \sin 2t)(-4 -3 \sin 2t) = $$
		$$ = \(\lambda + \frac{1}{4}\)^2 - \frac{9}{16} \cos^2 2t - \frac{9}{16} \sin^2 2t + 1 = \lambda^2 + \frac{1}{2}\lambda + \frac{1}{16} + \frac{7}{16}. $$
		To má řešení $\frac{-\frac{1}{2} ± i\sqrt{2 - \frac{1}{4}}}{2} = \frac{1}{4}(-1 ± i\sqrt{7})$
	\end{dukazin}
\end{priklad}

\begin{priklad}
	Ukažte, že $e^{\frac{t}{2}} \binom{-cos t}{\sin t}$ je řešením rovnice $x' = A(t) x$.

	\begin{dukazin}
		Prostě dosadíme:
		$$ x' = \frac{e^{\frac{t}{2}}}{2}\binom{-\cos t}{\sin t} + e^{\frac{t}{2}} \binom{\sin t}{\cos t} = \frac{e^{\frac{t}{2}}}{4} \binom{-2 \cos t + 4\sin t}{2\sin t + 4 \cos t}. $$
		$$ A(t)x = \frac{e^{\frac{t}{2}}}{4} \binom{-3 \cos t \cos 2t + \cos t + 4 \sin t - 3 \sin 2t \sin t}{+4\cos t + 3\cos t \sin 2t - \sin t - 3\sin t \cos 2t} = $$
		$$ = \frac{e^{\frac{t}{2}}}{4} \binom{-3 \cos^3 t + 3 \cos t \sin^2 t + \cos t + 4\sin t - 6 \sin^2 t \cos t}{4 \cos t + 6 \sin t \cos^2 t - \sin t + 3\sin^3 t - 3 \sin t \cos^2 t} = $$
		$$ = \frac{e^{\frac{t}{2}}}{4} \binom{-2 \cos t + 4\sin t}{2\sin t + 4 \cos t} $$
	\end{dukazin}
\end{priklad}

\pagebreak
\begin{priklad}
	Najděte řešení rovnice $x' = A(0)x$ (pro nějakou počáteční podmínku) a ukažte, že se vzdaluje od počátku v místech poblíž osy $x$.

	\begin{reseni}
		Řešením je například (spočítal jsem rozklad matice, maticovou exponenciálu a tak jsem došel k tomuto řešení)
		$$ e^{-\frac{t}{4}} \binom{\cos \frac{\sqrt{7}}{4}t + \sqrt{7} \sin \frac{\sqrt{7}}{4}t}{\cos \frac{\sqrt{7}}{4}t - \sqrt{7} \sin \frac{\sqrt{7}}{4}t}. $$

		Pokud je řešení v ose $x$, pak je $e^{-\frac{t}{4}}\(\cos \frac{\sqrt{7}}{4}t - \sqrt{7} \sin \frac{\sqrt{7}}{4}t\) = 0$. Navíc pokud bude znamínko derivace v první souřadnici odpovídat znamínku první souřadnice, tak se řešení vzdaluje od počátku (druhá souřadnice v ose $x$ nemá vliv). Tato derivace je
		$$ -\frac{1}{4}e^{-\frac{t}{4}}·\(\cos \frac{\sqrt{7}}{4}t + \sqrt{7} \sin \frac{\sqrt{7}}{4}t\) + e^{-\frac{t}{4}}·\(-\frac{\sqrt{7}}{4} \sin \frac{\sqrt{7}}{4}t + \frac{7}{4} \cos \frac{\sqrt{7}}{4}t\) = $$
		$$ = e^{-\frac{t}{4}}\(\frac{3}{2} \cos \frac{\sqrt{7}}{4}t - \frac{\sqrt{7}}{2} \sin \frac{\sqrt{7}}{4}t\) \overset{*}= e^{-\frac{t}{4}}\cos\(\frac{\sqrt{7}}{4}t\). $$
		Kde $*$ je podle toho, že druhá souřadnice je rovna $0$. Stejně tak první souřadnice je rovna
		$$ e^{-\frac{t}{4}} \(\cos \frac{\sqrt{7}}{4}t + \sqrt{7} \sin \frac{\sqrt{7}}{4}t\) \overset{*}= \(1 + \sqrt{7}\) e^{-\frac{t}{4}} \cos \frac{\sqrt{7}}{4}t. $$

		Tedy (v ose $x$) znamínko derivace první souřadnice řešení sedí s jeho znamínkem, tedy se zde řešení opravdu vzdaluje.
	\end{reseni}
\end{priklad}

\end{document}
