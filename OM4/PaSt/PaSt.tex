\documentclass[12pt]{article}					% Začátek dokumentu
\usepackage{../../MFFStyle}					    % Import stylu



\begin{document}

% 14. 02. 2022
\section*{Organizační úvod}
\begin{poznamka}
	Podmínkou zápočtu je splnění 1 domácí práce a 1 písemného testu. Není potřeba docházka.

	Bude moodle (přístup dají cvičící). Budou tam poznámky k přednášce, cvičebnice a bude se tam odevzdávat domácí práce.

	Je dobré umět míru.
\end{poznamka}

\section{Úvod}
\begin{poznamka}
	Pravděpodobnost popisuje modely popisující náhodné jevy.

	Statistika se pak snaží popsat reálné věci za pomocí těchto modelů.
\end{poznamka}

\begin{poznamka}[Historie]
	Klasická pravděpodobnost navazuje na dílo Kolmogorova, který popisoval axiomatickou pravděpodobnost.
\end{poznamka}

\section{Pravděpodobnostní prostor}
\begin{definice}[Pravděpodobnostní prostor, pravděpodobnost]
	Pravděpodobnostní prostor je trojice $(\Omega, ©A, P)$, kde $\Omega$ je neprázdná množina, ©A je $\sigma$-algebra a $P$ je pravděpodobnost.

	Pravděpodobnost $P$ je množinová funkce $©A \rightarrow [0, 1]$ splňující:
	\begin{itemize}
		\item $P(A) ≥ 0\ \forall A \in ©A$, (nezápornost)
		\item $P(\Omega) = 1$, (normovanost)
		\item jsou-li $A_i \in ©A$ po dvou disjunktní, pak $P\(\bigcup_{i=1}^∞ A_i\) = \sum_{i=1}^∞ P(A_i)$. ($\sigma$-aditivita)
	\end{itemize}
	
	\begin{poznamkain}[Interpretace]
		$\Omega$ se často nazývá stavový prostor a obsahuje všechny „realizace náhody“ neboli elementární jevy, tj. všechny možnosti, o kterých uvažuji.

		©A je $\sigma$-algebra náhodných jevů. $P$ pak obsahuje veškerou informaci o té dané náhodné situaci.

		Pokud nastal $\omega \in A \in ©A$ ($\omega \in \Omega$), pak nastal jev $A$.
	\end{poznamkain}

	\begin{definicein}[Klasický pravděpodobnostní prostor, diskrétní pravděpodobnostní prostor, spojitý pravděpodobnostní prostor, indikátor]
		$\Omega$ konečná, $©A = 2^\Omega$, $P(\{a\}) = \frac{1}{n}\ \forall a \in \Omega$ je klasický pravděpodobnostní prostor.

		$\Omega$ spočetná (včetně konečná), $©A = 2^\Omega$, $p: \Omega \rightarrow [0, 1]$ je taková, že $p(\omega) ≥ 0\ \forall \omega \in \Omega$ a $\sum_{\omega \in \Omega} = 1$. Položíme $P(A) = \sum_{\omega \in A} p(\omega)\ \forall A \in ©A$ nazýváme diskrétní pravděpodobnostní prostor.

		$\Omega = ®R$, $©A = ©B(®R)$ (resp. $©B_0(®R)$) a $g: ®R \rightarrow [0, ∞)$ měřitelná, že $\int_{®R} g(x) dx = 1$, pak definujeme $P(B) = \int_B g(x) dx$, $b \in ©B(®R) = ©A$ je spojitý pravděpodobnostní prostor. Speciálním případem $g(x) = 1_{[0, 1]}(x)$ je pak tzv. indikátor.
	\end{definicein}
\end{definice}

\begin{definice}[Jev jistý, jev nemožný, podjev, zároveň, alespoň jeden, jev opačný, neslučitelné jevy]
	$\Omega$ je jev jistý, $\O$ je jev nemožný, $A \subset B$ znamená „$A$ je podjev $B$“, $A \cap B$ znamená „nastal $A$ a zároveň $B$“, $A \cup B$ znamená „nastal $A$ nebo $B$“, $A^C$ je jev opačný, $A \cap B = \O$ jsou neslučitelné jevy.
\end{definice}

\begin{veta}
	Buďte $(\Omega, ©A, P)$ pravděpodobnostní prostor a $A, B, A_i \in ©A$ ($i \in ®N$) náhodné jevy. Pak platí:
	\begin{itemize}
		\item $P(\O) = 0$;
		\item $P$ je konečně aditivní;
		\item $P(A^C) = 1 - P(A)$;
		\item $P(A \cup B) = P(A) + P(B) - P(A \cap B)$;
		\item $A \subset B \implies P(A) ≤ P(B)$; (monotonie)
		\item $A_1 \subseteq A_2 \subseteq … \implies P\(\bigcup_{i=1}^∞ A_i\) = \lim_{n \rightarrow ∞} P(A_i)$; (spojitost)
		\item $A_1 \supseteq A_2 \supseteq … \implies P\(\bigcap_{i=1}^∞ A_i\) = \lim_{n \rightarrow ∞} P(A_i)$; (spojitost)
		\item $A_1 \supseteq A_2 \supseteq … \land \bigcap_{i=1}^∞ A_i = \O \implies \lim_{n \rightarrow ∞} P(A_i) = 0$; (spojitost v nule)
		\item $B \subset A \implies P(A \setminus B) = P(A) - P(B)$.
	\end{itemize}

	\begin{dukazin}
		Vše z míry. Pravdědobnost je konečná, předposlední bod vyplývá z předchozího.
	\end{dukazin}
\end{veta}

% 16. 02. 2022

\begin{poznamka}
	28. února bude v 17:20 náhradní přednáška za poslední přednášku.
\end{poznamka}

\begin{veta}[Princip inkluze a exkluze]
	Buď $(\Omega, ©A, P)$ pravděpodobnostní prostor. Pak pro každé $n \in ®N$ a každá $A_i \in ©A$, $i \in ®N$, platí:
	$$ P(\bigcup_{i=1}^n A_i) = \sum_{i=1}^n P(A_i) - \sum_{1 ≤ i ≤ j ≤ n} P(A_i \cap A_j) + … + (-1)^{n-1} P(\bigcap_{i=1}^n A_i). $$

	\begin{dukazin}
		Nebude, v podstatě byl v diskrétce.
	\end{dukazin}
\end{veta}

\section{Podmíněná pravděpodobnost}
\begin{definice}[Podmíněná pravděpodobnost]
	Buďte $A, B \in ©A$ takové, že $P(B) > 0$. Definujeme $P(A|B) = \frac{P(A \cap B)}{P(B)}$ a nazýváme ji podmíněnou pravděpodobností jevu $A$ za podmínky (jevu) $B$.
\end{definice}

\begin{veta}
	Buď $B \in ©A$ takové, že $P(B) > 0$. Pak zobrazení $P(.|B): ©A \rightarrow [0, 1]$ splňuje definici pravděpodobnosti.

	\begin{dukazin}
		Ověříme po bodech: zřejmě $P(A|B) ≥ 0$ $\forall A \in ©A$, $P(\Omega|B) = \frac{P(\Omega \cap B)}{P(B)} = \frac{P(B)}{P(B)} = 1$ a $\sigma$-aditivita plyne ze $\sigma$-aditivity $P(. \cap B)$ a deMorganových pravidel ($B \cap \bigcup_{i=1}^∞ A_i = \bigcup_{i=1}^∞ A_i \cap B$), $P(B)^{-1}$ se prostě z obou stran vytkne.
	\end{dukazin}
\end{veta}

\begin{upozorneni}
	Podmíněná pravděpodobnost nám neříká nic o příčinné souvislosti.
\end{upozorneni}

\begin{pozorovani}[O podmíněné pravděpodobnosti]
	Buďte $A, B, C \in ©A$ a pravděpodobnost „správných“ jevů nenulová. Pak:
	
	\begin{itemize}
		\item $P(A \cup B | C) = P(A|C) + P(B|C) - P(A \cap B|C)$,
		\item $B \subset A \implies P(A|B) = 1$,
		\item $A \cap B = \O \implies P(A|B) = 0$,
		\item $P(A | \Omega) = P(A)$,
		\item pokud $P(\{\omega\}) > 0$, pak $\forall A \in ©A$ platí $P(A|\{\omega\}) = \delta_{\omega}(A)$.
	\end{itemize}

	\begin{dukazin}
		Triviální (buď z definice, nebo z toho, že je to pravděpodobnost).
	\end{dukazin}
\end{pozorovani}

\begin{upozorneni}[Neplatí!]
	$P(A | B \cup C) = P(A|B) + P(A|C)$, ani v případě, že $A \cap B = \O$.
\end{upozorneni}

\begin{veta}[O násobení pravděpodobností]
	Buďte $A_1, A_2, …, A_n \in ©A$ takové, že $P(A_1 \cap A_2 \cap … \cap A_{n-1}) > 0$. Pak
	$$ \hspace{-0.3em}P(A_1 \cap A_2 \cap … \cap A_n) = P(A_n | A_1 \cap … \cap A_{n-1}) · P(A_{n-1} | A_1 \cap A_2 \cap … \cap A_{n-2}) · … · P(A_2 | A_1) · P(A_1).\hspace{-0.3em} $$

	\begin{dukazin}
		Z $P(A_1 \cap … \cap A_{n-1}) > 0$ plyne, že $P(A_1 \cap … \cap A_k) > 0$ pro $k \in [n-1]$, pomocí monotonie pravděpodobnosti. Tedy výraz je dobře definován.

		Dokážeme indukcí: Pro $n = 2$ platí $P(A_1 \cap A_2) = P(A_2|A_1)·P(A_1)$ z definice. Z $n-1$ na $n$: ($B := A_1 \cap … \cap A_{n-1}$)
		$$ P(A_1 \cap A_2 \cap … \cap A_n) = P(B \cap A_n) \overset{\text{def}}= P(A|B)·P(B) \overset{\text{IP}}= $$
		$$ = P(A_n | A_1 \cap … \cap A_{n-1}) · P(A_{n-1} | A_1 \cap A_2 \cap … \cap A_{n-2}) · … · P(A_2 | A_1) · P(A_1). $$
	\end{dukazin}
\end{veta}

\begin{veta}[O celkové pravděpodobnosti]
	Buďte $A, B_1, B_2, …$ náhodné jevy takové, že $P(\bigcup_n B_n) = 1$ a $B_i \cap B_j = \O$ $\forall i ≠ j$ a $P(B_i) > 0\ \forall i$. Potom $P(A) = \sum_n P(A | B_n)·P(B_n)$.
	
	\begin{dukazin}
		Víme $P\(\(\bigcup_n B_n\)^c\) = 0$, a tedy $P(A) = P\(A \cap \bigcup_n B_n\) + P\(A \cap \(\bigcup_n B_n\)\) = P\(A \cap \bigcup_n B_n\)$, protože $P$ je konečně-aditivní a platí monotonie. Dle de Morganových pravidel (a toho, že průnik s další množinou zachovává disjunktnost):
		$$ P(A) = P\(\bigcup_n(A \cap B_n)\) = \sum_n P(A \cap B_n) = \sum_n P(A|B_n) · P(B_n). $$
	\end{dukazin}
\end{veta}

\begin{veta}[Bayesova]
	Za předpokladů věty o celkové pravděpodobnosti a $P(A) > 0$, platí $P(B_i|A) = \frac{P(A|B_i)P(B_i)}{\sum_n P(A|B_n)P(B_n)}$.

	\begin{dukazin}
		Snadný z definice podmíněné pravděpodobnosti a věty o celkové pravděpodobnosti.
	\end{dukazin}
\end{veta}

\begin{priklad}[Pólyovo urnové schéma]
	Máme v urně $n$ koulí $k$ různých barev. Náhodně taháme z urny. Po vytažení koule do urny vytaženou kouli vrátíme a s ní i $\Delta$ (pevný parametr) koulí stejné barvy.

	Podle volby $\Delta$ máme 2 základní schémata: $\Delta = -1$ (tahání bez vracení) a $\Delta = 0$ (tahání s vracením).
\end{priklad}

% 21. 02. 2022

\begin{definice}[Nezávislé jevy]
	Náhodné jevy $A$ a $B$ jsou nezávislé, pokud platí $P(A \cap B) = P(A)·P(B)$.
\end{definice}

\begin{upozorneni}
	Zase to nemá nic do činění s kauzalitou.
\end{upozorneni}

\begin{veta}
	Jsou-li dva jevy $A$ a $B$ nezávislé, pak jsou i jevy $A$ a $B^c$ nezávislé.

	Je-li navíc $P(B) > 0$, pak $P(A|B) = P(A)$.

	\begin{dukazin}
		$$ P(A \cap B^c) = P(A) - P(A \cap B) = P(A) - P(A)·P(B) = P(A)·(1 - P(B)) = P(A)·P(B). $$

		$$ P(A|B) = \frac{P(A \cap B)}{P(B)} = \frac{P(A)·P(B)}{P(B)} = P(A). $$
	\end{dukazin}
\end{veta}

\begin{definice}[Vzájemná nezávislost]
	Buď $\{A_\lambda\}_{\lambda \in \Lambda}$ systém náhodných jevů. Pak říkáme, že tyto jevy jsou (vzájemně) nezávislé, pokud pro každou konečnou množinu $I \subset \Lambda$ (dále $I \in ©F(\Lambda)$) platí $P(\bigcap_{i \in I} A_i) = \prod_{i \in I} P(A_i)$.
\end{definice}

\begin{veta}
	Buď $C = \{B_1, …, B_k\}$, $k \in ®N$, systém nezávislých jevů. Nahradíme-li libovolnou podmnožinu těchto jevů jejich doplňky, dostaneme opět systém nezávislých jevů

	\begin{dukazin}
		Indukcí podle velikosti nahrazované množiny. (Použije se předchozí věta.)
	\end{dukazin}
\end{veta}

\begin{veta}
	Jsou-li jevy $A_1, …, A_n, B_1, …, B_m$ vzájemně nezávislé a $P(B_1 \cap … \cap B_m) > 0$, pak
	$$ P(A_1 \cap … \cap A_n|B_1 \cap … \cap B_m) = P(A_1 \cap … \cap A_n) = P(A_1)·…·P(A_n). $$

	\begin{dukazin}
		Snadný.
	\end{dukazin}
\end{veta}

\section{Náhodné veličiny}
\begin{definice}[Náhodný element]
	Buďte $(\Omega, ©A)$ a $(\Omega', ©A')$ stavové prostory. Pak každé měřitelné zobrazení $X: \Omega \rightarrow \Omega'$ nazveme náhodný element z $\Omega'$.
\end{definice}

\begin{definice}[Náhodná veličina]
	Měřitelné zobrazení $X: (\Omega, ©A) \rightarrow (®R, ©B(®R))$ nazveme (reálnou) náhodnou veličinou.
\end{definice}

\begin{definice}[Značení]
	Místo $\{\omega \in \Omega | X(\omega) ≤ a\}$ píšeme $\{X ≤ a\}$, místo $P(\{X ≤ a\})$ píšeme $P(X ≤ a)$.
\end{definice}

\begin{definice}
	Buď $X$ náhodná veličina. $X^{-1}(©B(®R))$ značíme $\sigma(X)$ a nazýváme $\sigma$-algebrou náhodných jevů generovaných náhodnou veličinou $X$ ($\sigma$-algebra indukovaná $X$).
\end{definice}

\end{document}
