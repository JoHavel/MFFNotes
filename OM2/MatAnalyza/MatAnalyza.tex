\documentclass[12pt]{article}                   % Začátek dokumentu
\usepackage{../../MFFStyle}                     % Import stylu

\begin{document}
\section{Řady}
    \subsection{Úvod}
        \begin{definice}
            Nechť $\{a_n\}_{n \in ®N}$ je posloupnost. Číslo $s_m  = a_1 + a_2 + … + a_m$ nazveme $m$-tým částečným součtem řady $\sum a_n$. Součtem nekonečné řady $\sum_{n=1}^∞ a_n$ nazveme limitu posloupnosti $\{s_m\}_{m \in ®N}$, pokud tato limita existuje. Je-li tato limita konečná, pak řekneme, že řada je konvergentní. Je-li tato limita nekonečná nebo neexistuje, pak řekneme, že řada je divergentní. Tuto limitu budeme značit $\sum_{n=1}^∞ a_n$.
        \end{definice}

        \begin{veta}[Nutná podmínka konvergence]
            Jestliže je $\sum_{n=1}^∞ a_n$ konvergentní, pak $\lim_{n \rightarrow ∞} a_n = 0$.

            \begin{dukazin}
                    $\sum_{n=1}^∞ a_n$ konverguje $\implies \exists \lim_{m \rightarrow ∞} s_m = s \in ®R$. $a_n = s_n - s_{n-1}$. $\lim_{n \rightarrow ∞} a_n = \lim_{n \rightarrow ∞} s_n - s_{n-1} = \lim_{n \rightarrow ∞} s_n - \lim_{n \rightarrow ∞} s_{n-1} = s - s = 0$
            \end{dukazin}
        \end{veta}

        \begin{upozorneni}
            Tato věta je pouze a jen implikace.
        \end{upozorneni}

        \begin{veta}[konvergence součtu řad]
            Nechť $\alpha \in ®R \setminus \{0\}$, pak $\sum_{n=1}^∞ a_n$ konverguje $\Leftrightarrow$ $\sum_{n=1}^∞ \alpha·a_n$ konverguje.

            Nechť $\sum_{n=1}^∞ a_n$ konverguje a $\sum_{n=1}^∞ b_n$ konverguje, pak $\sum_{n=1}^∞ \(a_n + b_n\) = \sum_{n=1}^∞ a_n + \sum_{n=1}^∞ b_n$.

            \begin{dukazin}
                $\sum_{n=1}^∞ a_n$ konverguje $\exists$ limita z $s_m \rightarrow s \in ®R$ a to je z AL právě tehdy, když konverguje $\alpha s_m \rightarrow \alpha·s \in ®R$, tedy $\sum_{n=1}^∞ \alpha·a_n$ konverguje.

                $\sum_{n=1}^∞ a_n = s \in ®R$ i $\sum_{n=1}^∞ b_n = \sigma \in ®R$ konvergují, tedy konverguje i $s_m + \sigma_m \rightarrow s + \sigma \in ®R$.
            \end{dukazin}
        \end{veta}

    \subsection{Řady s nezápornými členy}
        \begin{pozorovani}
            Nechť $\{a_n\}_{n = 1}^∞$ je řada s nezápornými členy. Pak $\sum_{n=1}^∞ a_n$ konverguje, nebo má součet $+∞$.

            \begin{dukazin}
                $s_m = a_1 + … + a_m ≤ a_1 + … + a_{m+1} = s_{m+1}$. $s_m≥0$ neklesající $\implies \exists \lim_{m \rightarrow ∞} s_m \in \[0, ∞\]$.
            \end{dukazin}
        \end{pozorovani}

        \begin{veta}[Srovnávací kritérium]
            Nechť $\sum_{n=1}^∞ a_n$ a $\sum_{n=1}^∞ b_n$ jsou řady s nezápornými členy a nechť $\exists n_0 \in ®N$ tak, že $\forall n \in ®N$, $n ≥ n_0$ platí $a_n ≤ b_n$. Pak a) $\sum_{n=1}^∞ b_n$ konverguje $\implies$ $\sum_{n=1}^∞ a_n$ konverguje b) $\sum_{n=1}^∞ a_n$ diverguje $\implies$ $\sum_{n=1}^∞ b_n$ diverguje.

            \begin{dukazin}
                a) Označme $s_n = a_1 + … + a_n$ a $\sigma_n = b_1 + … + b_n$. Pro každé $n \in ®N$, $n ≥ n_0$ platí
                $$ s_n = a_1 + … + a_{n_0} + a_{n_0 + 1} + … + a_n ≤ a_1 + … + a_{n_0} + b_{n_0 + 1} + … + b_n ≤ a_1 + … + a_{n_0} + \sigma_n ≤ a_1 + … + a_{n_0} + \sigma $$
                A to je konečné, neboť $\sum_{n=1}^∞ b_n$ konverguje, tedy $\sigma \in ®R$. $s_n$ neklesající a omezená $\implies \exists \lim_{n \rightarrow ∞} s_n \in ®R$.

                b) Nepřímím důkazem z a).
            \end{dukazin}
        \end{veta}

        \begin{veta}[Limitní srovnávací kritérium]
            Nechť $\sum_{n=1}^∞ a_n$ a $\sum_{n=1}^∞ b_n$ jsou řady s nezápornými členy a nechť $\lim_{n \rightarrow ∞} \frac{a_n}{b_n} = A \in ®R^*$. Jestliže $A \in (0, ∞)$, pak $\sum_{n=1}^∞ b_n$ konverguje $\Leftrightarrow$ $\sum_{n=1}^∞ a_n$ konverguje. Jestliže $A = 0$, pak $\sum_{n=1}^∞ b_n$ konverguje $\implies$ $\sum_{n=1}^∞ a_n$ konverguje. Jestliže $A = ∞$, pak $\sum_{n=1}^∞ a_n$ konverguje $\implies$ $\sum_{n=1}^∞ b_n$ konverguje.

            \begin{dukazin}
                (i) Z $\lim_{n \rightarrow ∞} \frac{a_n}{b_n} = K \in (0, ∞)$ plyne, k $\epsilon = \frac{K}{2} \exists n_0$ $\forall n ≥ n_0: \left| \frac{a_n}{b_n} - K \right| < \epsilon = \frac{K}{2}$, tedy $\frac{K}{2} ≤ \frac{a_n}{b_n} ≤ \frac{3}{2}K$.

                $\sum_{n=1}^∞ b_n$ konverguje $\overset{\text{konvergence součtu řad}}{\implies}$ $\sum_{n=1}^∞ \frac{3}{2}K·b_n$ konverguje $\land a_n ≤ \frac{3}{2}K·b_n$ $\overset{\text{Srov. kritérium}}{\implies}$ $\sum_{n=1}^∞ a_n$ konverguje.

                $\sum_{n=1}^∞ a_n$ konverguje $\land \frac{K}{2}·b_n ≤ a_n \implies$ $\sum_{n=1}^∞ \frac{K}{2}·b_n$ konverguje $\implies$ $\sum_{n=1}^∞ b_n$ konverguje.

                (ii) Z $\lim_{n \rightarrow ∞} \frac{a_n}{b_n} = 0$ plyne, k $\epsilon = 1 \exists n_0$ $\forall n≥n_0: \left| \frac{a_n}{b_n} - K \right| < \epsilon = 1$, tedy $a_n < b_n$, a pokud $\sum_{n=1}^∞ b_n$ konverguje, tak$\sum_{n=1}^∞ a_n$ konverguje podle srovnávacího kritéria.

                (iii) Úplně stejně jako (ii).
            \end{dukazin}
        \end{veta}



\end{document}
