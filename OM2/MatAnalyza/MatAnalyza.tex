\documentclass[12pt]{article}                   % Začátek dokumentu
\usepackage{../../MFFStyle}                     % Import stylu

\begin{document}

% 3. 3. 2021

\section{Řady}
    \subsection{Úvod}
        \begin{definice}
            Nechť $\{a_n\}_{n \in ®N}$ je posloupnost. Číslo $s_m  = a_1 + a_2 + … + a_m$ nazveme $m$-tým částečným součtem řady $\sum a_n$. Součtem nekonečné řady $\sum_{n=1}^∞ a_n$ nazveme limitu posloupnosti $\{s_m\}_{m \in ®N}$, pokud tato limita existuje. Je-li tato limita konečná, pak řekneme, že řada je konvergentní. Je-li tato limita nekonečná nebo neexistuje, pak řekneme, že řada je divergentní. Tuto limitu budeme značit $\sum_{n=1}^∞ a_n$.
        \end{definice}

        \begin{veta}[Nutná podmínka konvergence]
            Jestliže je $\sum_{n=1}^∞ a_n$ konvergentní, pak $\lim_{n \rightarrow ∞} a_n = 0$.

            \begin{dukazin}
                    $\sum_{n=1}^∞ a_n$ konverguje $\implies \exists \lim_{m \rightarrow ∞} s_m = s \in ®R$. $a_n = s_n - s_{n-1}$. $\lim_{n \rightarrow ∞} a_n = \lim_{n \rightarrow ∞} s_n - s_{n-1} = \lim_{n \rightarrow ∞} s_n - \lim_{n \rightarrow ∞} s_{n-1} = s - s = 0$
            \end{dukazin}
        \end{veta}

        \begin{upozorneni}
            Tato věta je pouze a jen implikace.
        \end{upozorneni}

        \begin{veta}[konvergence součtu řad]
            Nechť $\alpha \in ®R \setminus \{0\}$, pak $\sum_{n=1}^∞ a_n$ konverguje $\Leftrightarrow$ $\sum_{n=1}^∞ \alpha·a_n$ konverguje.

            Nechť $\sum_{n=1}^∞ a_n$ konverguje a $\sum_{n=1}^∞ b_n$ konverguje, pak $\sum_{n=1}^∞ \(a_n + b_n\) = \sum_{n=1}^∞ a_n + \sum_{n=1}^∞ b_n$.

            \begin{dukazin}
                $\sum_{n=1}^∞ a_n$ konverguje $\exists$ limita z $s_m \rightarrow s \in ®R$ a to je z AL právě tehdy, když konverguje $\alpha s_m \rightarrow \alpha·s \in ®R$, tedy $\sum_{n=1}^∞ \alpha·a_n$ konverguje.

                $\sum_{n=1}^∞ a_n = s \in ®R$ i $\sum_{n=1}^∞ b_n = \sigma \in ®R$ konvergují, tedy konverguje i $s_m + \sigma_m \rightarrow s + \sigma \in ®R$.
            \end{dukazin}
        \end{veta}

    \subsection{Řady s nezápornými členy}
        \begin{pozorovani}
            Nechť $\{a_n\}_{n = 1}^∞$ je řada s nezápornými členy. Pak $\sum_{n=1}^∞ a_n$ konverguje, nebo má součet $+∞$.

            \begin{dukazin}
                $s_m = a_1 + … + a_m ≤ a_1 + … + a_{m+1} = s_{m+1}$. $s_m≥0$ neklesající $\implies \exists \lim_{m \rightarrow ∞} s_m \in \[0, ∞\]$.
            \end{dukazin}
        \end{pozorovani}

        \begin{veta}[Srovnávací kritérium]
            Nechť $\sum_{n=1}^∞ a_n$ a $\sum_{n=1}^∞ b_n$ jsou řady s nezápornými členy a nechť $\exists n_0 \in ®N$ tak, že $\forall n \in ®N$, $n ≥ n_0$ platí $a_n ≤ b_n$. Pak a) $\sum_{n=1}^∞ b_n$ konverguje $\implies$ $\sum_{n=1}^∞ a_n$ konverguje b) $\sum_{n=1}^∞ a_n$ diverguje $\implies$ $\sum_{n=1}^∞ b_n$ diverguje.

            \begin{dukazin}
                a) Označme $s_n = a_1 + … + a_n$ a $\sigma_n = b_1 + … + b_n$. Pro každé $n \in ®N$, $n ≥ n_0$ platí
                $$ s_n = a_1 + … + a_{n_0} + a_{n_0 + 1} + … + a_n ≤ a_1 + … + a_{n_0} + b_{n_0 + 1} + … + b_n ≤ a_1 + … + a_{n_0} + \sigma_n ≤ a_1 + … + a_{n_0} + \sigma $$
                A to je konečné, neboť $\sum_{n=1}^∞ b_n$ konverguje, tedy $\sigma \in ®R$. $s_n$ neklesající a omezená $\implies \exists \lim_{n \rightarrow ∞} s_n \in ®R$.

                b) Nepřímím důkazem z a).
            \end{dukazin}
        \end{veta}

        \begin{veta}[Limitní srovnávací kritérium]
            Nechť $\sum_{n=1}^∞ a_n$ a $\sum_{n=1}^∞ b_n$ jsou řady s nezápornými členy a nechť $\lim_{n \rightarrow ∞} \frac{a_n}{b_n} = A \in ®R^*$. Jestliže $A \in (0, ∞)$, pak $\sum_{n=1}^∞ b_n$ konverguje $\Leftrightarrow$ $\sum_{n=1}^∞ a_n$ konverguje. Jestliže $A = 0$, pak $\sum_{n=1}^∞ b_n$ konverguje $\implies$ $\sum_{n=1}^∞ a_n$ konverguje. Jestliže $A = ∞$, pak $\sum_{n=1}^∞ a_n$ konverguje $\implies$ $\sum_{n=1}^∞ b_n$ konverguje.

            \begin{dukazin}
                (i) Z $\lim_{n \rightarrow ∞} \frac{a_n}{b_n} = K \in (0, ∞)$ plyne, k $\epsilon = \frac{K}{2} \exists n_0$ $\forall n ≥ n_0: \left| \frac{a_n}{b_n} - K \right| < \epsilon = \frac{K}{2}$, tedy $\frac{K}{2} ≤ \frac{a_n}{b_n} ≤ \frac{3}{2}K$.

                $\sum_{n=1}^∞ b_n$ konverguje $\overset{\text{konvergence součtu řad}}{\implies}$ $\sum_{n=1}^∞ \frac{3}{2}K·b_n$ konverguje $\land a_n ≤ \frac{3}{2}K·b_n$ $\overset{\text{Srov. kritérium}}{\implies}$ $\sum_{n=1}^∞ a_n$ konverguje.

                $\sum_{n=1}^∞ a_n$ konverguje $\land \frac{K}{2}·b_n ≤ a_n \implies$ $\sum_{n=1}^∞ \frac{K}{2}·b_n$ konverguje $\implies$ $\sum_{n=1}^∞ b_n$ konverguje.

                (ii) Z $\lim_{n \rightarrow ∞} \frac{a_n}{b_n} = 0$ plyne, k $\epsilon = 1 \exists n_0$ $\forall n≥n_0: \left| \frac{a_n}{b_n} - K \right| < \epsilon = 1$, tedy $a_n < b_n$, a pokud $\sum_{n=1}^∞ b_n$ konverguje, tak$\sum_{n=1}^∞ a_n$ konverguje podle srovnávacího kritéria.

                (iii) Úplně stejně jako (ii).
            \end{dukazin}
        \end{veta}

% 5. 3. 2021

        \begin{veta}[Cauchyovo odmocninové kritérium]
            Nechť $\sum_{n=1}^∞ a_n$ je řada s nezápornými členy, potom
            $$ (i) \exists q \in (0, 1)\ \exists n_0 \in ®N\ \forall n ≥ n_0: \sqrt[n]{a_n} < q \implies \sum_{n=1}^∞ a_n \text{ konverguje}, $$ 
            $$ (ii) \limsup\limits_{n \rightarrow ∞} \sqrt[n]{a_n} < 1 \implies \sum_{n=1}^∞ a_n \text{ konverguje}, $$
            $$ (iii) \lim_{n \rightarrow ∞} \sqrt[n]{a_n} < 1 \implies \sum_{n=1}^∞ a_n \text{ konverguje}, $$ 
            $$ (iv) \limsup\limits_{n \rightarrow ∞} \sqrt[n]{a_n} > 1 \implies \sum_{n=1}^∞ a_n \text{ diverguje}, $$ 
            $$ (v) \lim_{n \rightarrow ∞} \sqrt[n]{a_n} > 1 \implies \sum_{n=1}^∞ a_n \text{ diverguje}. $$

            \begin{dukazin}
                $(i)$ $b_n = q^n$. Víme, že $a_n < b_n\ \forall n ≥ n_0$, tedy použijeme srovnávací kritérium.

                $(i) \implies (ii): b_n = \{\sqrt[n]{a_n}, \sqrt[n+1]{a_n}, …\}$. $\lim_{n \rightarrow ∞} b_n = \limsup\limits_{n \rightarrow ∞} \sqrt[n]a_n < 1$. Nalezneme $q \in \(\limsup\limits_{n \rightarrow ∞} \sqrt[n]{a_n}, 1\)$. Z definice $\lim_{n \rightarrow ∞} b_n$ pro $\epsilon = q - \limsup\limits_{n \rightarrow ∞} \sqrt[n]{a_n}$ je $\exists n_0\ \forall n ≥ n_0: b_n < q$, tedy $\forall n ≥ n_0: \sqrt[n]{a_n} < q$, tedy podle $(i)$ $\sum_{n=1}^∞ a_n$ konverguje.

                $(ii) \implies (iii):$ $\exists \lim_{n \rightarrow ∞} \sqrt[n]{a_n} \implies $ $\limsup\limits_{n \rightarrow ∞} \sqrt[n]{a_n} = \lim_{n \rightarrow ∞} \sqrt[n]{a_n} < 1$, tedy podle $(ii)$ $\sum_{n=1}^∞ a_n$ konverguje.

                $(iv):$ podobně jako v $(i) \implies (ii)$ dostaneme $\forall n_0 > n_k: b_{n_0} > q > 1$, tedy $\forall n_0\ \exists n > n_0: \sqrt[n]{a_n} > q > 1 \implies a_n > 1$ $\implies \lim_{n \rightarrow ∞} a_n ≠ 0$, tedy podle nutné podmínky konvergence $\sum_{n=1}^∞ a_n$ diverguje.

                $(iv) \implies (v):$ $\lim_{n \rightarrow ∞} \sqrt[n]{a_n} = \limsup\limits_{n \rightarrow ∞} \sqrt[n]{a_n}$.
            \end{dukazin}
        \end{veta}

        \begin{veta}[d'Alambertovo podílové kritérium]
            Nechť $\sum_{n=1}^∞ a_n$ je řada s kladnými členy. Potom:
            $$ (i) \exists q \in (0, 1)\ \exists n_0 \in ®N\ \forall n ≥ n_0: \frac{a_{n+1}}{a_n} < q \implies \sum_{n=1}^∞ a_n \text{ konverguje}, $$ 
            $$ (ii) \limsup\limits_{n \rightarrow ∞} \frac{a_{n+1}}{a_n} < 1 \implies \sum_{n=1}^∞ a_n \text{ konverguje}, $$
            $$ (iii) \lim_{n \rightarrow ∞} \frac{a_{n+1}}{a_n} < 1 \implies \sum_{n=1}^∞ a_n \text{ konverguje}, $$ 
            $$ (iv) \lim_{n \rightarrow ∞} \frac{a_{n+1}}{a_n} > 1 \implies \sum_{n=1}^∞ a_n \text{ diverguje}, $$ 

            \begin{dukazin}
                $(i)$ Víme indukcí $a_{n_0 + k} < q^k a_{n_0}$ a z konvergence geometrické řady $\sum_{k=1}^∞ q^k a_n$ konverguje $\implies$ $\sum_{k=1}^∞ a_{n_0 + k}$ konverguje $\implies \sum_{n=1}^∞ a_n$ konverguje.

                $(i) \implies (ii)$: $b_n = \sup\{\frac{a_{n+1}}{a_n}, \frac{a_{n+2}}{a_{n+1}}, …\}$. $\lim_{n \rightarrow ∞} b_n = \limsup\limits_{n \rightarrow ∞} \frac{a_{n+1}}{a_n} < 1$. Zvolíme $q \in \(\lim_{n \rightarrow ∞} b_n, 1\)$. Tedy $\exists n_0\ \forall n ≥ n_0: b_n < q \implies$ $\forall n ≥ n_0: \frac{a_{n+1}}{a_n} < q$, tudíž podle $(i)$ $\sum_{n=1}^∞ a_n$ konverguje.

                $(ii) \implies (iii)$ $\lim_{n \rightarrow ∞} \frac{a_{n+1}}{a_n} = \limsup\limits_{n \rightarrow ∞} \frac{a_{n+1}}{a_n} < 1$, tedy podle $(ii)$ $\sum_{n=1}^∞ a_n$ konverguje.

                $(iv)$: Z $\lim_{n \rightarrow ∞} \frac{a_{n+1}}{a_n} > 1$ definicí limity pro $\epsilon < \lim_{n \rightarrow ∞} \frac{a_{n+1}}{a_n} - 1$ vyplývá $\exists n_0\ \forall n ≥ n_0: \frac{a_{n+1}}{a_n} > 1 \implies a_{n+1} > a_n$. Máme rostoucí posloupnost kladných čísel $\implies$ $\lim_{n \rightarrow ∞} a_n ≠ 0$, tedy podle nutné podmínky konvergence $\sum_{n=1}^∞ a_n$ diverguje.
            \end{dukazin}
        \end{veta}

        \begin{veta}[Kondenzační kritérium]
            Nechť $\sum_{n=1}^∞ a_n$ je řada s nezápornými členy splňující $a_n ≥ a_{n+1}$, $\forall n \in ®N$. Pak $\sum_{n=1}^∞ a_n$ konverguje $\Leftrightarrow$ $\sum_{n=1}^∞ 2^n·a_{2^n}$ konverguje.

            \begin{dukazin}
                Pro $k \in ®N:$ $s_k = \sum_{j=1}^k a_j$ $t_k = \sum_{j=0}^k 2^j·a_{2^j}$.

                $\Leftarrow$: Označme $A = \sum_{j=0}^k 2^j·a_{2^j}$, pak $A \in ®R$. Nechť $m \in ®N$ a nalezneme $k in ®N$, $m < 2^k$. Pak $t_k ≤ A$ a:
                $$ s_m ≤ a_1 + \(a_2 + a_3\) + \(a_4 + a_5 + a_6 + a_7\) + … + \(a_{2^{k-1}} + … + a_{2^k-1}\) ≤ t_{k-1} ≤ A. $$ 
                Tedy $s_m$ je shora omezená a rostoucí $\implies \exists \lim_{m \rightarrow ∞} s_m \in ®R$ $\implies \sum_{n=1}^∞ a_n$ konverguje.

                $\implies$: Označme $B = \sum_{n=1}^∞ a_n \in ®R$. Zvolme $k \in ®N$ a nalezneme $m \in ®N$, aby $2^k ≤ m$. Pak $s_m ≤ B$ a platí:
                $$ s_m ≥ a_1 + a_2 + \(a_3 + a_4\) + \(a_5 + a_6 + a_7 + a_8\) + … +\(a_{2^{k-1}+1} + … + a_{2^k}\) ≥ a_1 + \frac{1}{2} \(t_k - 1·a_1\) ≤ \frac{1}{2} t_k \implies t_k ≤ 2·B. $$ 
                $t_k$ je shora omezená rostoucí posloupnost $\implies \sum_{n=1}^∞ 2^na_{2^n}$ konverguje.
            \end{dukazin}
        \end{veta}

% 10. 3. 2021

    \subsection{Neabsolutní konvergence řad}
        \begin{definice}
            Nechť pro řadu $\sum_{n=1}^∞ a_n$ platí, že $\sum_{n=1}^∞ |a_n|$ konverguje. Pak říkáme, že $\sum_{n=1}^∞ a_n$ konverguje absolutně.
        \end{definice}

        \begin{veta}[Bolzano-Cauchyova podmínka pro konvergenci řad]
            Řada $\sum_{n=1}^∞ a_n$ konverguje právě tehdy, když je splněna následující podmínka:
            $$ \forall\epsilon > 0 \ \exists n_0 \in ®N\ \forall m, n \in ®N, m ≥ n_0, n ≥ n_0: \left|\sum_{n=j}^m a_n\right| < \epsilon. $$

            \begin{dukazin}
                    $\sum_{n=1}^∞ a_n$ konverguje $\Leftrightarrow \exists \lim_{n \rightarrow ∞} s_n \in ®R$ $\overset{\text{BC}}{\Leftrightarrow} \forall\epsilon > 0 \ \exists n_0 \in ®N\ \forall m, n \in ®N, m ≥ n_0, n ≥ n_0: \left|s_m - s_{n-1}\right| < \epsilon$. Což je přesně výraz (po odečtení $s_m - s_{n-1}$) ve větě.
            \end{dukazin}
        \end{veta}

        \begin{veta}[Vztah konvergence a absolutní konvergence]
            Nechť řada $\sum_{n=1}^∞ a_n$ konverguje absolutně, pak řada $\sum_{n=1}^∞ a_n$ konverguje.

            \begin{dukazin}
                Z BC podmínky: $\sum_{n=1}^∞ a_n$ konverguje $\implies \forall\epsilon > 0 \ \exists n_0 \in ®N\ \forall m, n \in ®N, m ≥ n_0, n ≥ n_0: \sum_{j=n}^m \left| a_j \right| < \epsilon$. Chceme dokázat, že $\sum_{n=1}^∞ a_n$ konverguje. Stačí ověřit BC podmínku. K $\epsilon > 0$ volme $n_0$ jako výše, pak $\forall m, n ≥ n_0: \left| \sum_{j=n}^m a_j \right| ≤ \sum_{j=n}^m \left| a_j \right| ≤ \epsilon$ $\implies \sum_{n=1}^∞ a_n$ konverguje.
            \end{dukazin}
        \end{veta}

        \begin{veta}[Leibnitzovo kritérium (T5.10)]
            Nechť $\{a_n\}_{n = 1}^∞$ je nerostoucí posloupnost nezáporných čísel, pak $\sum_{n=1}^∞ (-1)^n a_n$ konverguje $\Leftrightarrow \lim_{n \rightarrow ∞} a_n = 0$.

            \begin{dukazin}
                $\implies$: z nutné podmínky (V5.1) $\lim_{n \rightarrow ∞} (-1)^n·a_n = 0 \implies \lim_{n \rightarrow ∞} a_n = 0$.

                $\Leftarrow$: $s_{2k + 2} - s_{2k} = (-1)^{2k+2}·a_{2k+2} + (-1)^{2k+1}·a_{2k+1} = a_{2k+2} - a_{2k+1} ≤ 0 \implies s_{2k}$ je nerostoucí. Obdobně $s_{2k+1} - s_{2k-1} = a_{2k+1} - a_{2k} ≥ 0 \implies s_{2k+1}$ je neklesající. Navíc $s_2k = (-a_1 + a_2) + … + (-a_{2k-1} + a_{2k}) ≤ 0 + … + 0 = 0$. Analogicky $s_{2k+1} ≥ -a_1$.

                Nyní $0 ≥ s_{2k} = s_{2k+1} + a_{2k+1} ≥ -a_1 + a_{2k+1} ≥ -a_1$. Analogicky $-a_1 ≤ s_{2k+1} ≤ 0$. Tedy obě vybrané podposloupnosti jsou omezené a monotónní, tedy konvergují. $\lim_{n \rightarrow ∞} s_{2k} = S_1 \in ®R$ a $\lim_{n \rightarrow ∞} s_{2n+1} = S_2 \in ®R$. Navíc
                $$ S_2 = \lim_{n \rightarrow ∞} s_{2k+1} = \lim_{n \rightarrow ∞} s_{2k} - a_{2k+1} \overset{\text{AL}}{=} S_1 - 0 = S_1. $$
                Tedy jelikož existuje limita sudých i lichých členů a rovnají se, existuje i limita $s_n$.
            \end{dukazin}
        \end{veta}

        \begin{lemma}[Abelova parciální sumace]
            Nechť $m, n \in ®N$ a $m ≤ n$ a nechť $a_m, …„, a_n, b_m, …, b_n \in ®R$. Označme $s_k = \sum_{i=m}^k a_i$. Pak platí
            $$ \sum_{i=m}^n a_i·b_i = \sum_{i=m}^n s_i·(b_i - b_{i+1}) + s_n·b_n. $$
            
            \begin{dukazin}
                $$ = a_m · b_m + a_{m+1} · b_{m+1} + … + a_n·b_n = s_m·b_m + (s_{m+1} - s_m)· b_{m+1} + … + (s_n - s_{n-1})·b_n = \sum_{i=m}^n s_i·(b_i - b_{i+1}) + s_n·b_n. $$ 
            \end{dukazin}
        \end{lemma}

% 12. 3. 2021

        \begin{veta}[Abel-Dirichletovo kritérium]
            Nechť $\{a_n\}_{n = 1}^∞$ je posloupnost reálných čísel a $\{b_n\}_{n = 1}^∞$ je nerostoucí posloupnost nezáporných čísel. Nechť je splněna alespoň jedna z následujících podmínek:
            
            (A) $\sum_{n=1}^∞ a_n$ je konvergentní. (D) $\lim_{n \rightarrow ∞} b_n = 0$ a $\sum_{n=1}^∞ a_n$ má omezené částečné součty (tj. $\exists K > 0\ \forall m \in ®N: |s_m| = |\sum_{n=1}^m a_n| < K$).

            Pak je $\sum_{n=1}^∞ a_n·b_n$ konvergentní.

            \begin{dukazin}
                Podle V 5.8 budeme ověřovat BC podmínku pro $\sum_{n=1}^∞ a_n·b_n$. Označme $s_k = \sum_{n=m}^k a_n$. $b_n$ je nerostoucí a $b_n > 0$ $\implies$ $\forall i: b_i - b_{i+1} ≥ 0$ a $\exists K\ \forall n: |b_n| ≤ K$.

                (A): $\sum_{n=1}^∞ a_n$ konverguje
                $$ \implies \forall \epsilon > 0\ \exists n_0\ \forall i ≥ m ≥ n_0 |\sum_{n=m}^i a_n| = |s_i| < \epsilon. $$
                Nyní k $\epsilon > 0$ volme $n_0$ jako výše a nechť $n ≥ m ≥ n_0$:
                $$ |\sum_{i=m}^n a_i · b_i| \overset{\text{Abel PS}}{≤} \sum_{i=m}^{n-1} |s_i·(b_i - b_{i+1})| + |s_n|·|b_n| ≤ \epsilon · \sum_{n=1}^∞ (b_i - b_{i+1}) + \epsilon·b_n = \epsilon·(b_m - b_n) + \epsilon · b_n ≤ \epsilon·K. $$
                A podle BC podmínky máme $\sum_{n=1}^∞ a_n·b_n$ konverguje.

                (D) Z předpokladů víme, že $\exists M > 0\ \forall i ≥ m: |s_i| = |\sum_{n=1}^i  a_n - \sum_{n=1}^{m-1} a_n| ≤ M$ (volme $M = 2K$). Z $\lim_{n \rightarrow ∞} b_n = 0$ k $\epsilon > 0\ \exists n_0\ \forall n ≥ n_0: |b_n|<\epsilon$. Nyní 
                $$ \forall n ≥ m ≥ n_0: |\sum_{i=m}^n a_i·b_i| ≤ \sum_{i=m}^{n-1} |s_i(b_i - b_{i+1})| + |s_n|·|b_n| ≤ \sum_{i=m}^{n-1} M·(b_i - b_{i+1}) + M·b_n = M·(b_m - b_n) + M·b_n ≤ M·\epsilon. $$
                A podle BC podmínky máme $\sum_{n=1}^∞ a_n·b_n$ konverguje.
            \end{dukazin}
        \end{veta}

        \begin{priklad}
            $\sin n$ a $\cos n$ má omezené částečné součty.

            \begin{dukazin}
                Buď sečtením $\sin 1 + \sin 2 + … + \sin n = $ vzoreček.

                Nebo dokážeme dokonce $\forall x ≠ 2k\pi$ $\sin nx$ a $\cos nx$ má omezené částečné součty.
                $$ e^ix = \cos x + i·\sin x \implies \sum_{k=0}^n e^{i·k·x} = \sum_{k=0}^n \cos k·x + i·\sum_{k=0}^n \sin k·x. $$
                Z geometrické řady ale víme, že
                $$ \sum_{k=0}^n e^{i·k·x} = \frac{1 - \(e^{ix}\)^{n+1}}{1 - e^{ix}} = \frac{1 - \cos x·(n+1) - i·\sin x·(n+1)}{1 - \cos x - i\sin x} · \frac{1 - \cos x + i·\sin x}{1 - \cos x + i·\sin x} = \frac{A_n·B}{(1 - \cos x)^2 + (\sin x)^2}. $$
                Zřejmě $|A_n| ≤ 3$ a $|B|≤3$, jmenovatel je nenulový a není závislý na $n$, tedy pro všechna $n$ je výraz omezen konstantou.
            \end{dukazin}
        \end{priklad}

    \subsection{Přerovnání a součin řad}
        \begin{definice}[Přerovnání řady]
            Nechť $\sum_{n=1}^∞ a_n$ je řada a $p: ®N \rightarrow ®N$ bijekce. Řadu $\sum_{n=1}^∞ a_{p(n)}$ nazýváme přerovnáním řady $\sum_{n=1}^∞ a_n$.
        \end{definice}

        \begin{veta}[O přerovnání absolutně konvergentní řady]
            Nechť $\sum_{n=1}^∞ a_n$ je absolutně konvergentní řada a $\sum_{n=1}^∞ a_{p(n)}$ je její přerovnání. Pak $\sum_{n=1}^∞ a_{p(n)}$ je absolutně konvergentní a má stejný součet.

            \begin{dukazin}
                $\sum_{n=1}^∞ |a_n|$ konverguje $\implies$ splňuje BC podmínku. Tedy 
                $$ \forall \epsilon > 0\ \exists n_0\ \forall n ≥ m ≥ n_0 |\sum_{i=n}^m a_i| < \epsilon \implies \sum_{i=n_0}^∞ |a_i| ≤ \epsilon. $$
                Zvolme $n_0' = \max\{p(1), p(2), …, p(n_0)\}$. Pak $\forall n'≥ n_0': p^{-1}(n') ≥ n_0$. Tedy
                $$ \forall n' ≥ m' ≥ n_0': \sum_{i=m'}^{n'} |a_{p(i)}| ≤ \sum_{i=n_0}^∞ |a_i| < \epsilon. $$
                Tedy podle BC podmínky $\sum_{n=1}^∞ |a_{p(n)}|$ konverguje, tedy i $\sum_{n=1}^∞ a_{p(n)}$ konverguje.

                Konverguje k tomu samému? $\sum_{n=1}^∞ a_n = A$, $\sum_{n=1}^∞ a_{p(n)} = A'$. Víme, že k $\epsilon > 0\ \exists n_0 \sum_{i=n_0}^∞ |a_i| ≤ \epsilon$. Zvolme $n_0' ≥ \max_{i ≤ n_0} p(i)$, aby $\sum_{i=n_0'}^∞ |a_{p(i)}| ≤ \epsilon$. Pak $|\sum_{i=1}^{n_0} a_i - A| ≤ \epsilon$ a $|\sum_{i=1}^{n_0'} a_{p(i)} - A'| ≤ \epsilon$. Nyní
                $$ |A - A'| ≤ |\sum_{i=1}^{n_0} a_i - A| + |\sum_{i=1}^{n_0'} a_{p(i)} - A'| + |\sum_{i=1}^{n_0} a_i - \sum_{i=1}^{n_0'} a_{p(i)}| ≤ \epsilon + \epsilon + \sum_{i=n_0}^∞ |a_i| ≤ 3\epsilon $$ 
            \end{dukazin}
        \end{veta}

% 17. 3. 2021

        \begin{veta}[Rieman]
            Neabsolutně konvergentní řadu lze přerovnat k libovolnému součtu $s \in ®R^*$.

            \begin{dukazin}
                Bez důkazu (idea: rozdělíme na kladné a záporné členy (mají součty $+∞$ a $-∞$) a jdeme nahoru dolu nahoru dolu (vždy alespoň o 1 prvek), abychom se co nejvíce blížili $s$).
            \end{dukazin}
        \end{veta}

        \begin{definice}[Cauchyovský součin]
            Nechť $\sum_{n=1}^∞ a_n$ a $\sum_{n=1}^∞ b_n$ jsou řady. Cauchyovským součinem těchto řad nazveme řadu $\sum_{k=2}^∞ \sum_{i=1}^{k-1} (a_{k-i}·b_i)$.
        \end{definice}

        \begin{veta}[O součinu řad]
            Nechť $\sum_{n=1}^∞ a_n$ a $\sum_{n=1}^∞ b_n$ konvergují absolutně. Pak $\(\sum_{n=1}^∞ a_n\)·\(\sum_{n=1}^∞ b_n\) = \sum_{k=2}^∞ \sum_{i=1}^{k-1} (a_{k-i}·b_i)$.

            \begin{dukazin}
                Označme $s_n = \sum_{i=1}^n a_i \rightarrow A \in ®R$, $\sigma_n = \sum_{i=1}^n b_i \rightarrow B \in ®R$ a $\rho_n = \sum_{k=2}^n\(\sum_{i=1}^{k-1} a_{k-i}b_i\) \overset{\text{Chceme}}{\rightarrow} A·B \in ®R$. Nechť $\epsilon > 0$. Pak $\exists n_0: \sum_{i=n_0}^∞ |a_i| < \epsilon$ a $\sum_{j=n_0}^∞ |b_j| < \epsilon$ (z BC podmínky) a zároveň $|s_{n_0}·\sigma_{n_0} - A·B| < \epsilon$. Nechť $n ≥ 2n_0$, pak
                $$ |\rho_n - A·B| ≤ |\rho_n - s_{n_0}·\sigma_{n_0}| + |s_{n_0}·\sigma_{n_0} - A·B| ≤ $$
                $$ ≤ |(a_1b_1) + (a_1b_2 + a_2b_1) + … + (a_{n-1}·b_1 + … + a_1·b_{n-1}) - (a_1 + … +a_{n_0})·(b_1+…+b_{n_0})| + \epsilon ≤ $$
                $$ ≤ \sum_{i ≥ n_0 \lor j ≥ n_0} |a_ib_j| + \epsilon ≤ \sum_{i=1}^∞ |a_i| · \sum_{j=n_0}^∞ |b_j| + \sum_{i=n_0}^∞ |a_i| · \sum_{j=1}^∞ |b_j| + \epsilon ≤ A\epsilon + B\epsilon + \epsilon = \epsilon·\text{konst}. $$
            \end{dukazin}
        \end{veta}

    \subsection{Limita posloupnosti a součet řady v ®C}
        \begin{definice}
            Nechť $\{a_n\}_{n = 1}^∞$ a $\{b_n\}_{n = 1}^∞$ jsou dvě reálné posloupnosti. Pak $c_n = a_n + ib_n$ je komplexní posloupnost.

            Řekneme, že $\lim_{n \rightarrow ∞} c_n = A+iB$, pokud existují $\lim_{n \rightarrow ∞} a_n = A \in ®R$ a $\lim_{n \rightarrow ∞} b_n = B \in ®R$.
        \end{definice}

        \begin{definice}
            Nechť $\{a_n\}_{n = 1}^∞$ a $\{b_n\}_{n = 1}^∞$ jsou dvě reálné posloupnosti a $c_n = a_n + i b_n$. Řekneme, že komplexní řada $\sum_{n=1}^∞ c_n$ konverguje k $A + iB$, pokud konvergují řady $\sum_{n=1}^∞ a_n = A$ a $\sum_{n=1}^∞ b_n = B$.
        \end{definice}

        \begin{veta}[Vztah konvergence a absolutní konvergence pro komplexní řady]
            Nechť $\{c_n\}_{n = 1}^∞$ je komplexní posloupnost a řada $\sum_{n=1}^∞ |c_n|$ konverguje. Pak řada $\sum_{n=1}^∞ c_n$ konverguje.

            \begin{dukazin}
                Z BC podmínky pro konvergenci $\sum_{n=1}^∞ |c_n|$ dostaneme
                $$ \forall \epsilon > 0\ \exists n_0\ \forall m ≥ n ≥ n_0: \sum_{j=n}^m |c_j| < \epsilon. $$ 
                Víme $c_n = a_n + ib_n$. Nyní $\forall m ≥ n ≥ n_0:$
                $$ \sum_{j=n}^m |a_j| ≤ \sum_{j=n}^m |c_j| < \epsilon \land \sum_{j=n}^m |b_j| ≤ \sum_{j=n}^m |c_j| < \epsilon. $$ 
                Tedy $\sum_{n=1}^∞ |a_n|$ a $\sum_{n=1}^∞ |b_n|$ splňují BC podmínku, tedy konvergují. Podle V5.9 (vztah KaAK), tedy $\sum_{n=1}^∞ a_n$ a $\sum_{n=1}^∞ b_n$ konvergují, tedy konverguje i $\sum_{n=1}^∞ c_n$.
            \end{dukazin}
        \end{veta}

% 19. 3. 2021

\section{Primitivní funkce}
    \subsection{Základní vlastnosti}
        \begin{definice}[Primitivní funkce, integrál]
            Nechť je funkce $f$ definována na otevřeném intervalu $I$. Řekneme, že funkce $F$ je primitivní funkce k funkci $f$, pokud pro každé $x\in I$ existuje $F'(x)$ a $F'(x) = f(x)$.

            Množinu všech primitivních funkcí k $f$ na $I$ značíme $\int f(x)\, dx$
        \end{definice}

        \begin{veta}[O jednoznačnosti primitivní funkce až na konstantu]
            Nechť $F$ a $G$ jsou primitivní funkce k $f$ na otevřeném intervalu $I$. Pak existuje $c \in ®R$ tak, že $F(x) = G(x)+c$ pro všechna $x \in I$.

            \begin{dukazin}
                Označme $H(x) = F(x) - G(x)$. Pak $(H(x))' = (F(x) - G(x))' = f(x) - f(x) = 0$. Tedy (např. z Lagrangeovy věty) $\exists c \in ®R: H(x) = c$ na $I$.
            \end{dukazin}
        \end{veta}

        \begin{poznamka}
            Značíme $\int f(x)\,dx = F(x) + C$. Nechť $F$ je primitivní funkce k $f$. Pak $F$ je spojitá (protože má všude vlastní derivaci).
        \end{poznamka}

        \begin{veta}[O vztahu spojitosti a existence primitivní funkce]
            Nechť $I$ je otevřený interval a $f$ je spojitá funkce na $I$. Pak $f$ má na $I$ primitivní funkci.

            \begin{dukazin}
                Později.
            \end{dukazin}
        \end{veta}

        \begin{veta}[Linearita primitivní funkce]
            Nechť $f$ má primitivní funkci $F$ a $g$ má primitivní funkci $G$ na otevřeném intervalu $I$ a nechť $\alpha, \beta \in ®R$. Pak $\alpha·f + \beta·g$ má primitivní funkci $\alpha F + \beta G$.

            \begin{dukazin}
                $$ (\alpha·F(x) + \beta·G(x))' \overset{\text{AD}}{=} \alpha·F'(x) + \beta·G'(x) = \alpha·f + \beta·g. $$
            \end{dukazin}
        \end{veta}

        \begin{poznamka}[Tabulkové integrály]
            \ 
            \begin{itemize}
                \item $ \int x^n\,dx = \frac{x^n}{n+1} + C, ((x \in ®R \land n \in ®N) \lor (x \in ®R \setminus\{0\} \land n \in ®Z \setminus \{1\}))$.
                \item $\int \frac{1}{x}\, dx = \log|x| + C, (x \in ®R \setminus \{0\})$.
                \item $\int e^x\, dx = e^x + C, (x \in ®R)$.
                \item TODO
            \end{itemize}
        \end{poznamka}

        \begin{veta}[Nutná podmínka existence primitivní funkce]
            Nechť $f$ má na otevřeném intervalu $I$ primitivní funkci. Pak $f$ má na $I$ Darbouxovu vlastnost, tedy pro každý interval $J \subseteq I$ je $f(J)$ interval.

            \begin{dukazin}
                Nechť $J \in I$ je interval. Nechť $y_1, y_2 \in f(J)$ a $y_1 < z < y_2$. Chceme ukázat $z \in f(J)$. Nechť $F$ je primitivní funkce k funkci $f$ na intervalu $I$. Definujeme $H(x) = F(x) - z·x$ pro $x \in I$. Pak $H$ je spojitá na $I$ a $\forall x \in I: (H(x))' = f(x) - z$. Nalezneme $x_1, x_2 \in J$ tak, že $f(x_1) = y_1$ a $f(x_2) = y_2$. Nechť $x_1 < x_2$, v opačném případě je důkaz analogický. Funkce $H$ je spojitá na $[x_1, x_2]$, a tedy tam nabývá minima.

                Víme $H'(x_1) = f(x_1) - z < f(x_1) - y_1 = 0$, tedy $\exists \delta > 0$, že $\forall x \in [x_1, x_1+\delta], H(x) < H(x_1)$, tedy v $x_1$ není minimum. Obdobně v $x_2$ není minimum. Tedy minimum je v $x_0 \in (x_1, x_2) \overset{\text{Fermat}}{\implies} 0 = H'(x_0) = f(x_0) - z$, tj. $f(x_0) = z$.
            \end{dukazin}
        \end{veta}

% 24. 3. 2021

        \begin{veta}[Integrace per partes]
            Nechť $I$ je otevřený interval a funkce $f$ a $g$ jsou spojité na $I$. Nechť $F$ je primitivní k $f$ a $G$ je primitivní k $g$ na $I$. Pak platí $\int g(x)·F(x)\,dx = G(x)·F(x) - \int G(x)·f(x)\,dx$ ne $I$.

            \begin{dukazin}
                $G$ je spojitá, tedy $G(x)·f(x)$ je spojitá (tedy integrál vpravo existuje). Mějme funkci $G·F - H$, kde $H$ je primitivní k $G·f$, pak
                $$ (G(x)·F(x) - H(x))' = g(x)·F(x) + G(x)·f(x) - G(x)·f(x) = g(x)·F(x), $$
                neboli $\int g(x)·F(x)\,dx = G(x)·F(x) - H(x)$.
            \end{dukazin}
        \end{veta}

        \begin{veta}[1. o substituci]
            Nechť $F$ je primitivní funkce k $f$ na $a, b$. Nechť $\phi$ je funkce definovaná na $(\alpha, \beta)$ s hodnotami v intervalu $(a, b)$, která má v každém bodě $(\alpha, \beta)$ vlastní derivaci. Pak $\int f(\phi(t))·\phi'(t)\,dt = F(\phi(t))$ na $(\alpha, \beta)$.

            \begin{dukazin}
                Podle věty o derivaci složené funkce
                $$ (F(\phi(t)))' = F'(\phi(t))·\phi'(t) = f(\phi(t))·\phi'(t)\ \forall t \in (\alpha, \beta). $$ 
            \end{dukazin}
        \end{veta}

        \begin{veta}[2. o substituci]
            Nechť funkce $\phi$ má v každém bodě intervalu $\alpha, \beta$ vlastní nenulovou derivaci a $\phi((\alpha, \beta)) = (a, b)$. Nechť funkce $f$ je definována na intervalu $(a, b)$ a platí $\int f(\phi(t))·\phi'(t)\,dt = G(t)$ ne $(\alpha, \beta)$. Pak $\int f(x)\,dx = G(\phi^{-1}(x))$ na $(a, b)$.

            \begin{dukazin}
                Podle V6.4 $\phi'$ nabývá mezihodnot (a je všude nenulová), tudíž $\phi'$ je na $(\alpha, \beta)$ buď kladná nebo záporná a $\phi$ je tím pádem ryze monotónní a spojitá. Tedy lze použít větu o derivaci inverzní funkce a dostaneme $(\phi^{-1}(x)) = \frac{1}{\phi'(\phi(x))}$. Nyní na $(a, b)$
                $$ (G(\phi^{-1}(x)))' = G'(\phi^{-1}(x))·(\phi^{-1}(x))' = f(\phi(\phi^{-1}(x)))·\phi'(\phi^{-1}(x))·\frac{1}{\phi'(\phi^{-1}(x))} = f(x). $$ 
            \end{dukazin}
        \end{veta}

% 26. 3. 2021
    \subsection{Integrace racionálních funkcí}
        \begin{definice}[Racionální funkce]
            Racionální funkcí rozumíme podíl dvou polynomů $\frac{P}{Q}$, kde $Q$ není nulový polynom.
        \end{definice}

        \begin{veta}[Základní věta algebry]
            Nechť $P(x) = a_nx^n + … + a_0x^0$, $a_i \in ®R$, $a_n ≠ 0$. Pak existují $x_1, …, x_n \in ®C$ tak, že $P(x) = a_n·(x - x_1)·…·(x - x_n)$, $x \in ®R$.
        \end{veta}

        \begin{lemma}[O komplexních kořenech polynomu]
            Nechť $P$ je polynom s reálnými koeficienty a $z \in ®C$ je kořen $P$ násobnosti $k \in ®N$. Pak i $\overline{z}$ je kořen násobnosti $k$.

            \begin{dukazin}
                Nejprve pozorování: $(\overline{z})^k = \overline{z^k}$ (dokážeme přes goniometrický tvar).

                Důkaz provedeme matematickou indukcí podle $k$. $k = 1$: $z$ je kořen, tj. $P(z) = 0 = \overline{P(z)} = \overline{a_n·z^n + … + a_0z^0} = a_n\overline{z^n} + … + a_0\overline{z^0} = P(\overline{z})$ $\implies \overline{z}$ je kořen. Dále předpokládejme, že $z \notin ®R$ (jinak je důkaz triviální.)

                Nyní nechť tvrzení platí pro $k-1$ a $z$ je kořen násobnosti alespoň $k$, potom z IP víme, že $\overline{z}$ je $k-1$násobný kořen. Tedy $P(x) = (x - z)^{k-1}·(x - \overline{z})^{k-1}·Q(x) = (x^2 - (z + \overline{z})·x + z·\overline{z})^{k-1}·Q(x)$, tedy $Q$ má reálně koeficienty a $Q(z) = 0$. Podle 1. kroku indukce je tudíž $\overline{z}$ kořenem $Q$, tedy $k$násobným kořenem $P$.
            \end{dukazin}
        \end{lemma}

        \begin{veta}[O rozkladu na parciální zlomky]
                Nechť $P$ a $Q$ jsou polynomy s reálnými koeficienty takové, že stupeň $P$ je ostře menší než stupeň $Q$ a $Q(x) = a_n·(x - x_1)^{p_1}·…·(x - x_k)^{p_k}·(x^2 + \alpha_1x + \beta_1)^{q_1}·…·(x^2 + \alpha_lx + \beta_l)^{q_l}$, kde $a_n, x_1, …, x_k, \alpha_1, …, \alpha_l, \beta_1, …, \beta_l \in ®R, a_n ≠ 0$, $p_1, …, p_k, q_1, q_l \in ®N$, žádné dva z mnohočlenů nemají společný kořen a mnohočleny $x^2+\alpha_ix+\beta_i$ nemají reálný kořen.

                Pak existují jednoznačně určená čísla $A_j^i \in ®R$, $i \in [k]$, $j \in [p_i]$ a $B^i_j, C^i_j \in ®R$, $i \in [l]$, $j \in [q_i]$ tak, že platí:
                $$ \frac{P(x)}{Q(x)} = \frac{A^1_1}{x - x_1} + … + \frac{A^1_{p_1}}{(x - x_1)^{p_1}} + … + \frac{A^k_1}{x - x_k} + … + \frac{A^k_{p_k}}{(x - x_k)^{p_k}} + \frac{B^1_1x + C^1_1}{(x^2 + \alpha_1 x + \beta_1)^1} + …. $$ 

            \begin{dukazin}
                Bez důkazu (velmi obtížný a docela zbytečný).
            \end{dukazin}
        \end{veta}

        \begin{poznamka}[Postup při integraci racionální funkce]
            \ 
            \begin{enumerate}
                \item Vydělit polynomy.
                \item Rozklad na parciální zlomky podle předchozí věty.
                \item Integrace parciálních zlomků.
            \end{enumerate}
        \end{poznamka}

% 31. 3. 2021

    \subsection{Substituce, převádějící na racionální funkce}
        Viz přednáška. ($R(e^{ax}) \rightarrow t = e^{ax}, R(\log x)·\frac{1}{x} \rightarrow t = \log(x)$).

    \subsection{Integrace trigonometrických funkcí}
        \begin{definice}[Racionálni funkce 2 proměnných]
            Racionální funkcí dvou proměnných rozumíme podíl dvou polynomů $R(a, b) = \frac{P(a, b)}{Q(a, b)}$, kde $P(a, b)$ a $Q(a, b)$ jsou polynomy dvou proměnných a $Q$ není identicky nulový.
        \end{definice}

        \begin{poznamka}
            Při integraci funkcí $R(\sin x, \cos x)$ používáme substituce:

            \begin{itemize}
                \item Pokud $R(-\sin x, \cos x) = -R(\sin x, \cos x)$, pak používáme $t = \cos x$.
                \item Pokud $R(\sin x, -\cos x) = -R(\sin x, \cos x)$, pak používáme $t = \sin x$.
                \item Pokud $R(-\sin x, -\cos x) = R(\sin x, \cos x)$, pak používáme $t = \tan x$.
                \item Vždy funguje $t = \tan \frac{x}{2}$. (Nepoužívat není-li nutné, těžký výpočet!)
            \end{itemize}
        \end{poznamka}

    \subsection{Integrace funkcí obsahujících odmocniny}
        Viz přednáška. ($q \in ®N, ad ≠ bc, R(x, \(\frac{ax+b}{cx+d}\)^{\frac{1}{q}}) \rightarrow t = \(\frac{ax+b}{cx+d}\)^{\frac{1}{q}}$).

        \begin{poznamka}[Eulerovy substituce]
            Nechť $a ≠ 0$. Při integraci funkcí typu $R(x, \sqrt{ax^2 + bx + c})$ používáme substituce:

            \begin{itemize}
                \item polynom $ax^2 + bx + c$ má dvojnásobný kořen a $a > 0$, pak $\sqrt{ax^2 + bx + c} = \sqrt{a}|x-\alpha|$ a řešíme na $x>\alpha$ a $x<\alpha$ jako racionální funkce.
                \item polynom $ax^2 + bx + c$ má dva reálné kořeny $\alpha_1$ a $\alpha_2$. Pak úpravou převedeme na tvar $\sqrt{a\frac{x-\alpha_1}{x-\alpha_2}}$ nebo $\sqrt{a·\frac{\alpha_1-x}{x-\alpha_2}}$.
                \item polynom $ax^2 + bx + c$ nemá reálný kořen a $a > 0$. Pak používáme substituci $\sqrt{ax^2 + bx + c} = \sqrt{a}·x + t$.
            \end{itemize}

            \begin{upozorneni}
                Substituce $\tan x$, $\tan \frac{x}{2}$ a poslední předchozí jsou substituce 2. druhu a je vždy potřeba ověřit, že vnitřní funkce je monotónní a na.
            \end{upozorneni}
        \end{poznamka}

% 7. 4. 2021

\section{Určitý integrál}
    \subsection{Riemannův integrál}
        \begin{definice}[Dělení, zjemnění dělení]
            Konečnou posloupnost $\{x_j\}_{j=0}^n$ nazýváme dělením intervalu $[a, b]$, jestliže $a = x_0 < x_1 < … < x_{n-1} < x_n = b$.

            Řekneme, že dělení $D'$ intervalu $[a, b]$ zjemňuje dělení $D$ intervalu $[a, b]$, jestliže každý bod dělení $D$ je i bodem dělení $D'$.
        \end{definice}

        \begin{definice}[Horní a dolní součty, Riemanovy integrály]
            Nechť $f$ je omezená funkce definovaná na intervalu $[a, b]$ a $D$ je dělení $[a, b]$, definujme horní a dolní součty
            $$ S(f, d) = \sum_{i=1}^n \sup\{f(x)| x \in [x_{i-1}, x_i]\}·(x_i - x_{i-i}), $$
            $$ s(f, d) = \sum_{i=1}^n \inf\{f(x)| x \in [x_{i-1}, x_i]\}·(x_i - x_{i-i}). $$

            Horní a dolní Riemannův integrál definujeme jako
            $$ (R)\overline{\int_a^b} f(x)\,dx = \inf\{S(f, D) | D \text{ je dělení } [a, b]\}, $$ 
            $$ (R)\underline{\int_a^b} f(x)\,dx = \sup\{s(f, D) | D \text{ je dělení } [a, b]\}. $$ 
        \end{definice}

        \begin{definice}
            Řekneme, že $f$ je Riemanovsky integrovatelná, jestliže $(R)\underline{\int_a^b} f(x)\,dx = (R)\overline{\int_a^b} f(x)\,dx$. Tuto hodnotu pak označujeme $(R)\int_a^b f(x)\,dx$.

            Množinu funkcí mající Riemannův integrál značíme $R([a, b])$.
        \end{definice}

        \begin{poznamka}
            Omezenost $f$ je nutnou podmínkou.
        \end{poznamka}

        \begin{veta}[O zjemnění dělení]
            Nechť $f$ je omezená funkce na $[a, b]$, $D$ a $D'$ jsou dělení intervalu $[a, b]$ a $D'$ zjemňuje $D$. Pak $s(f, D) ≤ s(f, D') ≤ S(f, D') ≤ (f, D)$.

            \begin{dukazin}
                Prostřední nerovnost je triviální z $\sup ≥ \inf$.

                Předpokládejme, že $D = \{x_0, x_1, …, x_n\}$ a $D' = \{x_0, x_1, …, x_{j-1}, z, x_{x_j}, …, x_n\}$. Pak $\inf\{f(x), x \in [x_{j - 1}, x_j]\} ≤ \inf\{f(x), x \in [x_{j - 1}, z]\}$ a $\inf\{f(x), x \in [x_{j - 1}, x_j]\} ≤ \inf\{f(x), x \in [z, x_j]\}$. Vynásobením $(z - x_{j-1})$ a $(x_j - z)$ dostaneme
                $$ \inf\{f(x), x \in [x_{j - 1}, x_j]\}·(x_j - x_{j-1}) ≤ \inf\{f(x), x \in [x_{j - 1}, z]\}·(z - x_{j-1}) + \inf\{f(x), x \in [z, x_j]\}·(x_j - z) \implies s(f, D) ≤ s(f, D'). $$

                Pokud se $D$ a $D'$ liší o více bodů, pak postupujeme indukcí. Analogicky pro horní součty.
            \end{dukazin}
        \end{veta}

        \begin{veta}[O dvou děleních]
            Nechť $f$ je omezená funkce na $[a, b]$ a $D_1, D_2$ jsou dělení intervalu $[a, b]$. Pak $s(f, D_1) ≤ S(f, D_2)$.

            \begin{dukazin}
                Nechť $D$ zjemňuje $D_1$ i $D_2$ ($D = D_1 \cup D_2$). Potom $D$ je jemnější než $D_1$ i $D_2$ a podle předchozí věty:
                $$ s(f, D_1) ≤ s(f, D) ≤ S(f, D) ≤ S(f, D_2). $$ 
            \end{dukazin}
        \end{veta}

        \begin{dusledek}
            Nechť $f$ je omezená na $[a, b]$, $D_1$ a $D_2$ jsou dělení $[a, b]$, $m = \inf\{f(x) | x \in [a, b]\}$ a $M = \sup\{f(x) | x \in [a, b]\}$. Pak:
            $$ m·(b - a) ≤ s(f, D_1) ≤ \underline{\int_a^b} f(x)\,dx ≤ \overline{\int_a^b} f(x)\,dx ≤ S(f, D_2) ≤ M·(b - a). $$ 
        \end{dusledek}

        \begin{definice}[Norma dělení]
            Nechť $D$ je dělení $[a, b]$. Číslo $\ni(D) = \max_{j = 1, …, n} |x_j - x_{j-1}|$ nazveme normou dělení $D$.
        \end{definice}

        \begin{veta}[Aproximace R. integrálu pomocí součtů]
            Nechť $f$ je omezená funkce na $[a, b]$ a $\{D_n\}_{n=1}^∞$ je posloupnost dělení $[a, b]$ taková, že $\lim_{n \rightarrow ∞} \ni(D_n) = 0$. Potom $(R) \overline{\int_a^b} f(x)\,dx = \inf_{n \in ®N} S(f, D_n)$ a $(R) \underline{\int_a^b} f(x)\,dx = \sup_{n \in ®N} s(f, D_n)$.

            \begin{dukazin}
                BÚNO $f ≥ 0$ (jinak přičteme k $f$ konstantu). Stačí dokázat druhá rovnost, první je analogická. Nechť $D$ je libovolné dělení a $\epsilon > 0$. Stačí dokázat, že $\exists n_0: s(f, D_{n_0}) ≥ s(f, D) - \epsilon$. Pak
                $$ (R) \underline{\int_a^b} f(x)\,dx = \sup_D S(f, D) ≥ \sup_{D_n} s(f, D_n) ≥ \sup_D(s(f, D) - \epsilon) = (R) \underline{\int_a^b} f(x)\,dx - \epsilon. $$
                Nechť $0≤ f ≤ K$ a zvolme $n_0$, aby $\ni(D_{n_0}) ≤ \frac{\epsilon}{K·4·\#\text{intervalů } D}$. Označme $H = $ intervaly vzniklé dělením $P = D \cup D_{n_0}$ a $\gamma = $ intervaly z $P$, v kterých není žádný bod dělení $D$. $P$ je jemnější než $D$, a proto z věty výše dostáváme
                $$ s(f, D) ≤ s(f, P) = \sum_{L \in H} \inf_{L}f·\text{délka } L = \sum_{L \in \gamma} \inf_{L}f·\text{délka } L + \sum_{L \in H \setminus \gamma} \inf_{L}f·\text{délka } L ≤ s(f, D_{n_0}) + 2·\#\text{intervalů }D·(K·\ni(D_{n_0})) < s(f, D_{n_0}) + \epsilon. $$
            \end{dukazin}
        \end{veta}


% 9. 4. 2021

        \begin{veta}[Kritérium existence R integrálu]
            Nechť $f$ je omezená funkce na $[a, b]$. Pak $f \in R([a, b]) \Leftrightarrow \forall \epsilon > 0\ \exists$ dělení $D$ intervalu $[a, b]$, že $S(f, D) - s(f, D) < \epsilon$.

            \begin{dukazin}
                $\implies$: Zvolme libovolnou posloupnost dělení, že $\ni|D_n| \rightarrow 0$ ($D_{n+1}$ je jemnější než $D_n$). Pak
                $$\lim_{n \rightarrow ∞} S(f, D_n) = (R) \overline{\int_a^b}f(x)\,dx = (R) \int_a^bf(x)\,dx, $$
                $$\lim_{n \rightarrow ∞} s(f, D_n) = (R) \underline{\int_a^b}f(x)\,dx = (R) \int_a^bf(x)\,dx. $$
                Tedy $\exists n_0\ \forall n ≥ n_0: S(f, D_n) - s(f, D_n) < \epsilon$.

                $\Rightarrow$: Zvolme $\epsilon > 0$ a k němu nalezneme $D$ z předpokladu. 
                $$ 0 ≤ (R) \overline{\int_a^b}f(x)\,dx - (R) \underline{\int_a^b}f(x)\,dx ≤ S(f, D) - s(f, D) < \epsilon \implies (R) \overline{\int_a^b}f(x)\,dx = (R) \underline{\int_a^b}f(x)\,dx. $$
            \end{dukazin}
        \end{veta}

        \begin{definice}[Stejnoměrná spojitost]
            Řekneme, že funkce $f$ je stejnoměrně spojitá na intervalu $I$, jestliže
            $$ \forall \epsilon > 0\ \exists \delta > 0\ \forall x, y \in I: |x - y| < \delta \implies |f(x) - f(y)| < \epsilon. $$
        \end{definice}

        \begin{veta}[O vztahu spojitosti a stejnoměrné spojitosti]
            Nechť $f$ je spojitá na omezeném uzavřeném intervalu $[a, b]$, pak $f$ je stejnoměrně spojitá na $[a, b]$.

            \begin{dukazin}
                Sporem. Nechť $f$ je spojitá na $[a, b]$, ale
                $$ \exists \epsilon > 0\ \forall \delta = \frac{1}{n}\ \exists x_n, y_n \in I: |x_n - y_n| < \frac{1}{n} \land |f(x_n) - f(y_n)| ≥ \epsilon. $$
                Interval $a, b$ je omezený, tedy z $x_n$ lze vybrat konvergentní posloupnost podle Weirstrassovy věty. Tedy $\lim_{k \rightarrow ∞} x_{n_k} = x_0$. Dále $\lim_{k \rightarrow ∞} y_{n_k} = x_0$, neboť 
                $$ |y_{n_k} - x_0| ≤ |y_{n_k} - x_{n_k}| + |x_{n_k} - x_0| < \frac{1}{n_k} + |x_{n_k} - x_0| \rightarrow 0. $$
                Víme, že $f$ je spojitá v $x_0$ (vzhledem k $[a, b]$). Tedy k našemu $\epsilon > 0\ \exists \delta > 0$ tak, že $\forall z \in (x_0 - \delta, x_0 + \delta) \cap [a, b]: |f(z) - f(x_0)| < \frac{\epsilon}{3}$. Nalezneme $j \in ®N$, aby $x_{n_k}, y_{n_k} \in (x_0 - \delta, x_0 + \delta)$. Nyní
                $$ \epsilon ≤ |f(x_{n_k} - f(y_{n_k}))| ≤ |f(x_{n_k} - f(x_0))| + |f(x_0) - f(y_{n_k})| < \frac{\epsilon}{3} + \frac{\epsilon}{3} < \epsilon.\text{\lightning}. $$ 
            \end{dukazin}
        \end{veta}

        \begin{veta}[O vztahu spojitosti a Riemannovské integrovatelnosti]
            Nechť $f$ je spojitá na omezeném intervalu $[a, b]$, pak $f \in R([a, b])$.

            \begin{dukazin}
                Podle věty ze zimy je spojitá funkce na omezeném intervalu spojitá. Z předchozí věty víme, že $f$ je dokonce stejnoměrně spojitá na $[a, b]$. Pak
                $$ \exists \delta > 0: \forall x, y \in [a, b]: |x - y| < \delta \implies |f(x) - f(y)| < \epsilon. $$
                Zvolme dělení $D$ intervalu $[a, b]$ tak, že $\ni(D) < \delta$. Nechť $D = \{x_j\}_{j = 0}^n$. Označme $M_j = \sup_{x_j, x_{j + 1}}f$, $m_j = \inf_{x_j, x_{j + 1}}f$. Pak platí $M_j ≤ m_j + \epsilon \forall j \in [n]$.
                $$ S(f, D) - s(f, D) = \sum_{j=1}^n M_j(x_j - x_{j-1}) - \sum_{j=1}^n m_j(x_j - x_{j-1}) = \sum_{j=1}^n (M_j - m_j)·(x_j - x_{j-1}) ≤ \epsilon·\sum_{j=1}^n (x_j - x_{j-1}) = \epsilon·(b - a). $$
                Podle věty výše tedy $f \in R([a, b])$.
            \end{dukazin}
        \end{veta}

        \begin{veta}[Vztah monotonie a Riemanovské integrovatelnosti]
            Nechť $f$ je (omezená) monotonní funkce na intervalu $[a, b]$. Pak $f \in R([a, b])$.

            \begin{dukazin}
                BÚNO $f$ je neklesající. Budeme kritérium existence R integrálu. Nechť $\epsilon > 0$. Zvolme ekvidistantní dělení $D = \{a + (b - a)\frac{j}{n}\}_{j=0}^n$ a volíme $n$, aby $n > \frac{1}{\epsilon}(b-a)·(f(b) - f(a))$. Nyní
                $$ S(f, D) = \sum_{j = 1}^n \sup_{[x_{j-1}, x_j]}f·(x_j - x_{j-1}) = \sum_{j=1}^n f(x_j)·(x_j - x_{j-1}) = \frac{b - a}{n} \sum_{j=1}^n f(x_j), $$ 
                $$ s(f, D) = \sum_{j = 1}^n \inf_{[x_{j-1}, x_j]}f·(x_j - x_{j-1}) = \sum_{j=1}^n f(x_{j-1})·(x_j - x_{j-1}) = \frac{b - a}{n} \sum_{j=1}^n f(x_{j-1}). $$ 
                Odtud
                $$ S(f, D) - s(f, D) = \frac{b-a}{n} \sum_{j=1}^n f(x_j) - f(x_{j-1}) ≤ \frac{b-a}{n}(f(b) - f(a)) < \epsilon. $$ 
            \end{dukazin}
        \end{veta}

% 14. 4. 2021

        \begin{veta}[Vlastnosti R integrálu]
            a) Linearita: $f, g \in R([a, b]), \alpha \in ®R \implies f+g \in R([a, b]) \land \alpha f \in R([a, b])$ a 
            $$ (R) \int_a^b f + g = (R) \int_a^b f + (R) \int_a^b g \land (R) \int_a^b \alpha·f = \alpha·(R) \int_a^b g. $$
            b) Monotonie: $f, g \in R([a, b]), f ≤ g$, pak $(R) \int_a^b f ≤ (R) \int_a^b g$.

            c) Aditivita vzhledem k intervalům: Nechť $a < c < b$. Pak $f\in R([a, b]) \Leftrightarrow f \in R([a, c]) \land f \in R([c, b])$ a platí $(R) \int_a^b f(x)\,dx = (R) \int_a^c f(x)\,dx + (R) \int_c^b f(x)\,dx$.

            \begin{dukazin}[a]
                $f, g \in R([a, b]) \implies f$ a $g$ jsou omezené na $[a, b]$ $\implies f + g$ je omezená a $\alpha f$ je omezená na $[a, b]$. Je-li $I \subseteq [a, b]$ interval, pak $\sup_I(f+g) ≤ \sup_I f + \sup_I g$, $\inf_I(f+g) ≤ \inf_I f + \inf_I g$. Proto pro libovolné dělení $D$ intervalu $[a, b]$ platí
                $$ s(f, D) + s(g, D) ≤ s(f + g, D) ≤ S(f + g, D) ≤ S(f, D) + S(g, D). $$
                Zvolme posloupnost dělení $\{D_n\}$ intervalu $[a, b]$ tak, že $\ni(D_n) \rightarrow 0$ (a $D_{n+1}$ jemnější než $D_n$). Podle věty výše
                $$ \lim_{n \rightarrow ∞} S(f, D_n) + S(g, D_n) = (R) \int_a^b f(x)\,dx + (R) \int_a^b g(x)\,dx, $$
                $$ \lim_{n \rightarrow ∞} s(f, D_n) + s(g, D_n) = (R) \int_a^b f(x)\,dx + (R) \int_a^b g(x)\,dx. $$
                Spolu s nerovností výše je to
                $$ \lim_{n \rightarrow ∞} s(f+g, D_n) = \lim_{n \rightarrow ∞} S(f + g, D_n) \overset{\text{POLICIE}}{=} (R) \int_a^b f(x)\,dx + (R) \int_a^b g(x)\,dx. $$
                Tedy podle věty výše $f+g \in R([a, b])$ a $(R) \int_a^b f + g = (R) \int_a^b f + (R) \int_a^b g$.

                Je-li $f \in R([a, b])$, $\alpha ≥ 0$, je $\alpha·f$ omezená na $[a, b]$. Pro každý interval $I \subseteq [a, b]$
                $$ \sup_I \alpha·f = \alpha·\sup_I f, \qquad \inf_I \alpha·f = \alpha·\inf_I f \implies $$
                $$ S(\alpha f, D) = \alpha·S(f, D), \qquad s(\alpha·f, D) = \alpha·s(f, D). $$
                Nechť $\{D_n\}$ je posloupnost dělení $[a, b]$, že $\nu|D_n| \rightarrow 0$ a $D_{n+1}$ je jemnější než $D_n$. Pak
                $$ \lim_{n \rightarrow ∞} S(\alpha f, D_n) = \lim_{n \rightarrow ∞} \alpha·S(f, D_n) = \alpha·(R) \int_a^b f(x)\,dx, $$
                $$ \lim_{n \rightarrow ∞} s(\alpha f, D_n) = \lim_{n \rightarrow ∞} \alpha·s(f, D_n) = \alpha·(R) \int_a^b f(x)\,dx. $$
                Podle věty výše je pro $\alpha f: \alpha f \in R([a, b])$ a $(R)\int_a^b \alpha f = \alpha (R)\int_a^b f$.

                Zbývá $\alpha < 0$. Stačí $\alpha = -1$ (jelikož pak můžeme násobit kladným). Pak $\forall$ interval $I$ $\sup_I (-f) = -\inf_I f$ a $\inf_I(-f) = -\sup_I f$. Tedy $\forall$ posloupnost dělení $\{D_n\}$, kde $\nu(D_n) \rightarrow 0$ a $D_{n+1}$ je jemnější než $D_n$:
                $$ \lim_{n \rightarrow ∞} S(-f, D_n) = \lim_{n \rightarrow ∞} - s(f, D_n) = - (R)\int_a^b f(x)\,dx, $$ 
                $$ \lim_{n \rightarrow ∞} s(-f, D_n) = \lim_{n \rightarrow ∞} - S(f, D_n) = - (R)\int_a^b f(x)\,dx, $$
                tudíž $-f \in R([a, b])$ a $(R)\int_a^b (-f) = -(R)\int_a^b f$.
            \end{dukazin}

            \begin{dukazin}[b]
                Nechť $D_n$ je posloupnost dělení, $\nu(D_n) \rightarrow 0$ a $D_{n+1}$ je jemnější než $D_n$. Pak $\sup_I f ≤ \sup_I g$. Tedy víme, že
                $$ \int_a^b f(x)\,dx = \lim_{n \rightarrow ∞} S(f, D_n) ≤ \lim_{n \rightarrow ∞} S(g, D_n) = (R)\int_a^b g(x)\,dx. $$
            \end{dukazin}

            \begin{dukazin}[c]
                Nechť $\{D_n^1\}$ a $\{D_n^2\}$ jsou posloupnosti dělení $[a, c]$ respektive $[c, b]$ splňující $\nu(D_n^1) \rightarrow 0$ a $\nu(D_n^2) \rightarrow 0$ a $D^1_{n+1}$ je jemnější než $D^1_n$ a $D^2_{n+1}$ je jemnější než $D^2_n$. Nechť $D_n = D_n^1 \cup D_n^2$. Pak $D_n$ je dělení $[a, b]$ a $\nu(D_n) \rightarrow 0$ a $D_{n+1}$ je jemnější než $D_n$.

                Nechť $f\in R([a, c])$ a $f \in R([c, b])$. Pak podle věty výše
                $$ \lim_{n \rightarrow ∞} S(f, D_n^1) = \lim_{n \rightarrow ∞} s(f, D_n^1) = (R)\int_a^c f(x)\,dx, $$
                $$ \lim_{n \rightarrow ∞} S(f, D_n^2) = \lim_{n \rightarrow ∞} s(f, D_n^2) = (R)\int_c^b f(x)\,dx. $$
                Tedy
                $$ \lim_{n \rightarrow ∞} S(f, D_n) = \lim_{n \rightarrow ∞} S(f, D_n^1) + S(f, D_n^2) = (R) \int_a^c f(x)\,dx + (R)\int_c^b f(x)\,dx, $$
                $$ \lim_{n \rightarrow ∞} s(f, D_n) = \lim_{n \rightarrow ∞} s(f, D_n^1) + s(f, D_n^2) = (R) \int_a^c f(x)\,dx + (R)\int_c^b f(x)\,dx. $$
                Podle věty výše je $f \in R([a, b])$ a $(R) \int_a^b f(x)\,dx = (R) \int_a^c f(x)\,dx + (R)\int_c^b f(x)\,dx$.

                Nechť $f \in R([a, b])$. Pak
                $$ 0 ≤ S(f, D_n^1) - s(f, D_n^1) ≤ S(f, D_n^1) - s(f, D_n^1) + S(f, D_n^2) - s(f, D_n^2) = S(f, D_n) - s(f, D_n) \rightarrow 0 \implies $$ 
                $$ \implies \lim_{n \rightarrow ∞} S(f, D_n^1) - s(f, D_n^1) = 0 \implies f \in R([a, c]). $$
                Analogicky $f \in R([c, b])$. Rovnost $(R) \int_a^b f(x)\,dx = (R) \int_a^c f(x)\,dx + (R) \int_c^b f(x)\,dx$ plyne z předchozí části důkazu.
            \end{dukazin}
        \end{veta}

        \begin{poznamka}[Úmluva]
            1. Nechť $b < a$, pak definujeme $\int_a^b f(x)\,dx = - \int_b^a f(x)\,dx$.
        \end{poznamka}

\end{document}
