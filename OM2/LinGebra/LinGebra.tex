\documentclass[12pt]{article}                   % Začátek dokumentu
\usepackage{../../MFFStyle}                     % Import stylu

\begin{document}

% 3. 3. 2021
\section{Skalární součin}
    \begin{definice}[Standardní skalární součin]
        Buďte $¦u, ¦v \in ®C^n$. Pak standardní skalární součin ¦u a ¦v definujeme jako $¦u·¦v = \overline{u_1}·v_1 + … + \overline{u_n}·v_n$.
    \end{definice}

    \begin{definice}[Euklidovská norma]
        Nechť $·$ je standardní skalární součin na ¦V. Potom $\forall ¦v \in ¦V$ definujeme euklidovskou normu jako $||¦v|| = \sqrt{¦v·¦v}$.
    \end{definice}

    \begin{definice}[Skalární součin]
        Nechť ¦V je vektorový prostor nad ®C. Skalární součin je zobrazení $·: ¦V \times ¦V \rightarrow ®C$, které ($\forall ¦u, ¦v, ¦w \in ¦V$ a $\forall t \in ®C$) splňuje:
        $$ ¦u·¦v = \overline{¦v·¦u}, \text{ (Symetričnost)} $$
        $$ ¦u·(t¦v) = t(¦u·¦v),\ ¦u·(¦v + ¦w) = ¦u·¦v + ¦u·¦w. \text{ (Linearita)}$$ 
    \end{definice}

    \begin{definice}[Hermitovsky sdružená matice]
        Nechť $A = \(a_{ij}\) \in ®C^{m \times n}$, potom hermitovsky sdružená matice je $A^* = \(\overline{a_{ji}}\)$.
    \end{definice}

\end{document}
