\documentclass[12pt]{article}                   % Začátek dokumentu
\usepackage{../../MFFStyle}                     % Import stylu

\begin{document}

% 3. 3. 2021
\section{Skalární součin}
    \begin{definice}[Standardní skalární součin]
        Buďte $¦u, ¦v \in ®C^n$. Pak standardní skalární součin ¦u a ¦v definujeme jako $¦u·¦v = \overline{u_1}·v_1 + … + \overline{u_n}·v_n$.
    \end{definice}

    \begin{definice}[Euklidovská norma]
        Nechť $·$ je standardní skalární součin na ¦V. Potom $\forall ¦v \in ¦V$ definujeme euklidovskou normu jako $||¦v|| = \sqrt{¦v·¦v}$.
    \end{definice}

    \begin{definice}[Skalární součin]
        Nechť ¦V je vektorový prostor nad ®C. Skalární součin je zobrazení $·: ¦V \times ¦V \rightarrow ®C$, které ($\forall ¦u, ¦v, ¦w \in ¦V$ a $\forall t \in ®C$) splňuje:
        $$ ¦u·¦v = \overline{¦v·¦u}, \text{ (Symetričnost)} $$
        $$ ¦u·(t¦v) = t(¦u·¦v),\ ¦u·(¦v + ¦w) = ¦u·¦v + ¦u·¦w, \text{ (Linearita)}$$
        $$ \<¦u, ¦u\> ≥ 0 \land (\<¦u, ¦u\> = 0 \Leftrightarrow ¦u = ¦o). $$ 
    \end{definice}

    \begin{definice}[Hermitovsky sdružená matice]
        Nechť $A = \(a_{ij}\) \in ®C^{m \times n}$, potom hermitovsky sdružená matice je $A^* = \(\overline{a_{ji}}\)$.

        Čtvercová matice je Hermitovská, pokud je rovna své hermitovsky sdružené matici.
    \end{definice}

% 5. 3. 2021

    \begin{definice}
        Buď $®T = ®R$ nebo ®C a buď $A = T^{n \times n}$. Pak $A$ je pozitivně definitní, pokud je hermitovská a platí
        $$ ¦u*Au ≥ 0 \land (¦u*Au = 0 \Leftrightarrow ¦u=¦o). $$ 
    \end{definice}

    \begin{dusledek}
        $\<·, ·\>_A = ·^*A·$ je skalární součin, právě když $A$ je pozitivně definitní.
    \end{dusledek}

    \begin{definice}[Norma]
        Buď ¦V VP nad ®R nebo ®C se skalárním součinem $\<·, ·\>$. Pak normou vektoru $¦u \in ¦V$ rozumíme $||¦u|| = \sqrt{\<¦u, ¦u\>}$ .
    \end{definice}

    \begin{tvrzeni}[Vlastnosti normy]
        $$ ||¦u||≥0 \land (||¦u|| = 0 \Leftrightarrow ¦u = ¦o). $$
        $$ \forall t \in ®T: ||t¦u|| = |t|·||¦u||. $$
        $$ ||¦u+¦v||^2 + ||¦u-¦v||^2 = 2||¦u||^2 + 2||¦v||^2. $$
        $$ \Re(\<¦u, ¦v\>) = \frac{1}{2}||¦u+¦v||^2 - ||¦u||^2 - ||¦v||^2. $$ 
    \end{tvrzeni}

    \begin{veta}[Cauchy-Schwarzova nerovnost]
        Buď $®T = ®R$ nebo ®C, ¦V VP nad ®T se skalárním součinem $\<·, ·\>$. Pak platí $\forall ¦u, ¦v \in ¦V: |\<¦u, ¦v\>| ≤ ||¦u||·||¦v||$. Rovnost platí právě tehdy, pokud $(¦u, ¦v)$ je lineárně závislá.

        \begin{dukazin}
            Případ 1): $(¦u, ¦v)$ je LZ: Buď $¦u = t¦v$: $|\<¦u, ¦v\>| = |\<¦u, t·¦u\>| = |t|·|\<¦u, ¦u\>| = |t|·||¦u||·||¦u|| = ||¦u||·||¦v||.$

            Případ 2): $(¦u, ¦v)$ je LN: Víme, že $||¦u - t¦v||^2 > 0$. Zvolme $t$ tak, aby $\<¦v, ¦u - t¦v\> = 0$. (To lze, protože $\<¦v, ¦u - t¦v\> = \<¦v, ¦u\> - t\<¦v, ¦v\> = \<¦v, ¦u\> - t||¦v||^2 \implies t = \frac{\<¦v, ¦u\>}{||¦v||^2}$.) $0 < ||¦u - t¦v||^2 = \<¦u - t¦v, ¦u - t¦v\> = \<¦u, ¦u - t¦v\> - \overline{t}·\<¦v, ¦u - t¦v\> = \<¦u, ¦u - t¦v\> =  \<¦u, ¦u\> - t\<¦u, ¦v\> = ||¦u||^2 - \frac{\<¦v, ¦u\>·\<¦u, ¦v\>}{||¦v||^2} = ||¦u||^2 - \frac{\overline{\<¦u, ¦v\>}·\<¦u, ¦v\>}{||¦v||^2} = ||¦u||^2 - \frac{|\<¦u, ¦v\>|^2}{||¦v||^2}$.

            Tj. $0 < ||¦u||^2·||¦v||^2 - |\<¦u, ¦v\>|^2$, tedy $|\<¦u, ¦v\>| ≤ ||¦u||·||¦v||$.
        \end{dukazin}
    \end{veta}

    \begin{dusledek}[Trojúhelníková nerovnost]
        Buď $®T = ®R$ nebo ®C, ¦V VP nad ®T se skalárním součinem $\<·, ·\>$. Pak platí $\forall ¦u, ¦v \in ¦V: ||¦u + ¦v|| ≤ ||¦u|| + ||¦v||$. Rovnost platí právě tehdy, pokud $(¦u, ¦v)$ je lineárně závislá.

        \begin{dukazin}
            $||¦u + ¦v||^2 = \<¦u + ¦v, ¦u + ¦v\> = \<¦u, ¦u\> + \<¦v, ¦u\> + \<¦u, ¦v\> + \<¦v, ¦v\> = ||¦u||^2 + \<¦u, ¦v\> + \overline{\<¦u, ¦v\>} + ||¦v||^2 = ||¦u||^2 + 2\Re(\<¦u, ¦v\>) + ||¦v||^2 ≤ ||¦u||^2 + 2|\<¦u, ¦v\>| + ||¦v||^2 ≤ ||¦u||^2 + 2·||¦u||·||¦v|| + ||¦v||^2 = \(||¦u|| + ||¦v||\)^2.$
        \end{dukazin}
    \end{dusledek}

% 10. 3. 2021

    \begin{definice}[Kolmost]
        Buď ¦V VP se skalárním součinem $\<·, ·\>$ a $¦u, ¦v \in ¦V$. Řekneme, že ¦u a ¦v jsou kolmé, značíme $¦u \perp ¦v$, pokud $\<¦u, ¦v\>$.
    \end{definice}

    \begin{poznamka}
        Ze skoro symetrie (SSS) plyne, že relace jsou kolmé je symetrická.
    \end{poznamka}

    \begin{definice}[Kolmost množin]
        Množina nebo posloupnost $M$ vektorů VP ¦V s $\<·, ·\>$ se nazývá ortogonální, pokud každá dvojice různých prvků $M$ je kolmá. Nazývá se ortonormální, pokud je ortogonální a každý prvek má normu 1.
    \end{definice}

    \begin{dusledek}
        Kanonická báze je ortonormální. Normovaná (tj. každý prvek vydělíme normou) ortogonální množina / posloupnost je ortonormální.
    \end{dusledek}

    \begin{tvrzeni}[Pythagorova věta]
        ¦V vektorový prostor se $\<·, ·\>$, buďte $¦u, ¦v \in ¦V$ kolmé vektory. Pak
        $$ ||¦u + ¦v||^2 = ||¦u||^2 + ||¦v||^2. $$

        \begin{dukazin}
            $$ ||¦u + ¦v||^2 = \<¦u + ¦v, ¦u + ¦v\> = \<¦u, ¦u\> + \<¦v, ¦u\> + \<¦u, ¦v\> + \<¦v, ¦v\> = \<¦u, ¦u\> + 0 + 0 + \<¦v, ¦v\> = ||¦u||^2 + ||¦v||^2. $$
        \end{dukazin}
    \end{tvrzeni}

    \begin{dusledek}
        Je-li $(¦v_1, …, ¦v_k)$ ortogonální posloupnost, pak $||¦v_1 + … + ¦v_k||^2 = ||¦v_1||^2 +… + ||¦v_k||^2$.

        \begin{dukazin}
            Indukcí triviálně.
        \end{dukazin}
    \end{dusledek}

    \begin{tvrzeni}
        Buď ¦V vektorový prostor s $\<·, ·\>$ a $(¦v_1, …, ¦v_k)$ ortogonální posloupnost nenulových vektorů. Pak je $(¦v_1, …, ¦v_k)$ LN.

        \begin{dukazin}
            Předpokládejme, že $0 = a_1¦v_1 + … + a_k¦v_k$, kde $a_1, …, a_k \in ®T$ ($®T = ®R \lor ®T = ®C$). Chceme ukázat, že $a_1 = … = a_k = 0$.
            $$ \forall i \in [k]: 0 = \<v_i, ¦o\> = \<¦v_i, a_1¦v_1 + … + a_k¦v_k\> = a_1 \<¦v_i, ¦v_1\> + … + a_k \<¦v_i, ¦v_k\> = a_i·||¦v_i||^2 \implies a_i = 0. $$ 
        \end{dukazin}
    \end{tvrzeni}

    \subsection{Ortonormální báze a vyjádření vektorů vzhledem k nim}
        \begin{tvrzeni}
            ¦V VP s $\<·, ·\>$, $B = (¦v_1, …, ¦v_n)$ ortonormální báze. Pak pro každý $¦u \in ¦V$ platí:
            $$ ¦u = \<¦v_1, ¦u\>·¦v_1 + … + \<¦v_n, ¦u\>·¦v_n. $$
            To jest
            $$ [¦u]_B = (\<¦v_1, ¦u\>, …, \<¦v_k, ¦u\>)^T. $$

            \begin{dukazin}
                Vezmeme $a_1, …, a_n \in ®T$ tak, aby $¦u = a_1¦v_1 + … + a_n¦v_n$. Máme $\<¦v_i, ¦u\> = \<¦v_i,  a_1¦v_1 + … + a_n¦v_n\> = a_1·0 + … + a_i + … + a_n·0 = a_i$.
            \end{dukazin}
        \end{tvrzeni}

        \begin{poznamka}
            Kdyby $B$ byla jen ortogonální, pak $ [¦u]_B = (\frac{\<¦v_1, ¦u\>}{||¦v_1||}, …, \frac{\<¦v_k, ¦u\>}{||¦v_k||})^T$.
        \end{poznamka}

        \begin{poznamka}
            $a_1, …, a_n$ se někdy nazývají Fourierovy koeficienty.
        \end{poznamka}

        \begin{tvrzeni}
            ¦V VP s $\<¦·, ·\>$, $B = (¦v_1, …, ¦v_n)$ ortonormální báze, $¦u, ¦w \in ¦V$. Pak $\<¦u, ¦w\> = [¦u]_B·[¦w]_B = [¦u]^*_B[¦w]_B$.

            \begin{dukazin}
                    Buď $[¦u]_B = (a_1, …, a_n)^T$, $[¦w]_B = (b_1, …, b_n)^T$. Pak $\<¦u, ¦v\> = \<a_1¦v_1 + … + a_n¦v_n, b_1¦v_1 + … + b_n¦v_n\> = \sum_{i=1}^n\sum_{j=1}^n \overline{a_i}·b_j·\<¦v_i, ¦v_j\> = \sum_{i=1}^n\overline{a_i}b_i·\<¦v_i, ¦v_i\> = [¦u]^*_B    [¦w]_B$.
            \end{dukazin}
        \end{tvrzeni}

    \subsection{Kolmost množin}
        \begin{definice}[Kolmost množin]
            ¦V VP s $\<¦·, ·\>$, $¦v \in ¦V$, $M, N \subseteq ¦V$. Pak řekneme, že ¦v je kolmý k $M$, značíme $¦v\perp M$, pokud $¦v \perp ¦w\ \forall ¦w \in M$, a řekneme, že $M$ je kolmá k $N$, značíme $M \perp N$, pokud $¦v \perp ¦w\ \forall ¦v \in M\ \forall ¦w \in N$.
        \end{definice}

% 12. 3. 2021

        \begin{definice}
            Buď ¦V VP s $\<¦·, ·\>$ a buď $¦W ≤ ¦v$. Je-li $¦v \in ¦V$, ortogonální projekcí vektoru ¦v na podprostor ¦W rozumíme vektor ¦w takový, že $¦w \in ¦W$ a $¦v - ¦w \perp ¦w$.
        \end{definice}

        \begin{veta}
        Buď ¦V VP s $\<¦·, ·\>$, buď $¦W≤¦V$, $¦v \in ¦V$ a $¦w \in ¦W$ ortogonální projekce ¦v na ¦W. Potom pro každý vektor $¦u \in ¦W$ různý od ¦w platí: $||¦v - ¦w|| < ||¦v - ¦u||$.

        Speciálně existuje-li ortogonální projekce ¦v na ¦W, pak je určena jednoznačně.

            \begin{dukazin}
                Z předpokladu $¦w, ¦u$, a tedy i $¦w - ¦u$ jsou vektory $¦W$. Tudíž $¦v - ¦w \perp ¦w - ¦u$ ($¦v - ¦w \perp ¦W$). Z Pythagorovy věty: $||¦v - ¦u||^2 = ||¦v - ¦w||^2 + ||¦w - ¦u||^2 > ||¦v - ¦w||^2$.
            \end{dukazin}
        \end{veta}

        \begin{tvrzeni}
            Buď ¦V VP s $\<¦·, ·\>$ a buďte $M, N \subseteq ¦V$. Pak $M \perp N$ $\Leftrightarrow$ $M \perp \LO (N)$ ($\Leftrightarrow$ $\LO(M) \perp \LO(N)$).

            \begin{dukazin}
                $\Leftarrow$: Triviální (protože $N \subseteq \LO(N)$).

                $\implies$: Předpokládejme, že $M \perp N$. Vezměme $¦v \in M$ a $¦w = a_1¦w_1 + … + a_n¦w_n \in \LO(N)$. Pak $\<¦v, ¦w\> = a_1\<¦v_1, ¦w_1\> + … + a_n\<¦v_n, ¦w_n\> = 0$.
            \end{dukazin}
        \end{tvrzeni}

        \begin{tvrzeni}
            Buď ¦V VP s $\<¦·, ·\>$ a $¦W≤¦V$, který má ortonormální bázi $B = (¦u_1, …, ¦u_n)$. Pak pro libovolné $¦v \in ¦V$ je
            $$ ¦w := \<¦u_1, ¦v\>·¦u_1 + \<¦u_2, ¦v\>¦u_2 + … + \<¦u_k, ¦v\>·¦u_k $$
            ortogonální projekcí do ¦W.

            \begin{dukazin}
                Zjevně $¦w \in \LO(B) = ¦W$. Chceme ukázat, že $¦v - ¦w \perp ¦W$. Podle tvrzení výše stačí ukázat, že $¦v - ¦w \perp ¦u_i, \forall i \in [k]$. Označme $a_i := \<¦u_i, ¦v\>$.
                $$ \<¦u_i, ¦v - ¦w\> = \<¦u_i, ¦v - a_1¦u_1 - a_2¦u_2 - … - a_k¦u_k\> = a_i - a_1·0 - … - a_i·1 - … - a_k·0 = ¦o. $$ 
            \end{dukazin}
        \end{tvrzeni}

        \begin{definice}[Gramova-Schmidtova ortogonalizace]
            Postup, který vezme LN posloupnost $(¦v_1, …, ¦v_k)$ z VP s $\<¦·, ·\>$ a vytvoří ortonormální posloupnost $(¦u_1, …, ¦u_k)$ taková, že $\forall i \in [k]: \LO\{¦v_1, …, ¦v_i\} = \LO\{¦u_1, …, ¦u_i\}$.

            1) $¦u_1 := \frac{¦v_1}{||¦v_1||}$. 2) Pro každé $i = 2, …, k$ spočítáme $¦w_i = \<¦u_1, ¦v_i\>·¦u_1 + … + \<¦u_{i-1}, ¦v_i\>·¦u_{i-1}$ a položíme $¦u_i = \frac{¦v_i - ¦w_i}{||¦v_i - ¦w_i||}$.

            \begin{dukazin}
                To, že $(¦u_1, …, ¦u_i)$ je ortonormální $\forall i \in [k]$ dokážeme triviálně indukcí.

                Stejně tak, že $\LO\{¦v_1, …, ¦v_i\} = \LO\{¦u_1, …, ¦u_i\}$.
                $$ ¦u_i = \frac{¦v_i}{||…||} - \frac{¦w_i}{||…||}, \frac{¦w_i}{||…||} \in \LO\{¦u_1, …, ¦u_{i-1}\} \overset{\text{IP}}{=} \LO\{¦v_1, …, ¦v_{i-1}\}. $$
                $¦v_i$ také z definice.

                Nakonec musíme ukázat, že nikdy nedělíme nulou (naopak, pokud dostaneme špatný (= LZ) vstup, tak dělíme). Ukáže se, že kdybychom dělili, tak nějaké $¦u_i \in \LO\{¦u_1, …, ¦u_{i-1}\}$.
            \end{dukazin}
        \end{definice}

        \begin{veta}
            Máme-li ¦V VP s $\<¦·, ·\>$ a aplikujeme-li GS ortogonalizaci na LN posloupnost vektorů z ¦V, pak dostaneme ortonormální posloupnost, že se jejich LO rovnají.

            \begin{dukazin}
                Viz předchozí důkaz.
            \end{dukazin}
        \end{veta}

        \begin{dusledek}
            Každý konečně generovaný VP se skalárním součinem má ortonormální bázi.
        \end{dusledek}

        \begin{dusledek}
            Máme-li ¦V konečně generovaný VP s $\<¦·, ·\>$ a ortonormální posloupnost $(¦u_1, …, ¦u_k)$, můžeme ji doplnit na ortonormální bázi.

            \begin{dukazin}
                Doplníme na bázi a aplikujeme GS ortogonalizaci, kde si rozmyslíme, že nám nezmění původní posloupnost.
            \end{dukazin}
        \end{dusledek}

        \begin{dusledek}
            Je-li ¦V konečně generovaný VP s ortonormální bází $B = (¦u_1, …, ¦u_n)$ a s $\<¦·, ·\>$, pak existuje isomorfismus $¦V \rightarrow T^n$ takový, že $\forall ¦v, ¦w: \<¦v, ¦w\> = f(¦v)·f(¦w)$.
        \end{dusledek}

        \begin{poznamka}
            Aplikováním GS ortogonalizace na $®T^n$ dostaneme tzv. QR - rozklad matice, kde $A = Q·R$ a $A$ má za sloupce původní vektory, $Q$ má ortonormální posloupnost sloupců a $R$ je horní trojúhelníková s nezápornými reálnými čísly na diagonále.
        \end{poznamka}

% 17. 3. 2021

    \subsection{Ortogonální doplněk, Gramova matice}
        \begin{definice}
            Buď ¦V VP s $\<¦·, ·\>$ nad $®T = ®R$ nebo ®C. Je-li $M \subseteq ¦V$ množina vektorů, pak ortogonálním doplňkem k $M$ ve ¦V, rozumíme
            $$ M^{\perp} = \{¦v \in ¦v|¦v \perp M\} = \{¦v \in ¦V | (\forall ¦u \in M: \<¦v, ¦u\> = 0)\}. $$
        \end{definice}

        \begin{dusledek}
            $M \perp M^{\perp}$ a $M^\perp$ je největší taková množina vzhledem k inkluzi.
        \end{dusledek}

        \begin{tvrzeni}
            ¦V VP s $\<¦·, ·\>$, $M \subseteq ¦V$. Pak $M^\perp = (\LO M)^\perp$, $M^\perp$ je podprostor ¦V, $M \subseteq N \implies N^\perp \subseteq M^\perp$.

            \begin{dukazin}
                $$ ¦v \in M^\perp \Leftrightarrow ¦v \perp M \Leftrightarrow ¦v \perp \LO M \Leftrightarrow ¦v \in (\LO M)^\perp. $$

                Vezmeme $¦w_1, ¦w_2 \in M^\perp$ a $t \in ®T$, pak $\forall ¦v \in M: \<¦v, ¦w_1 + ¦w_2\> = \<¦v, ¦w_1\> + \<¦v, ¦u_2\> = 0 + 0 = 0$ a $\<¦v, t·¦w_1\> = t·\<¦v, ¦w_1\> = t·0 = 0$ $\implies ¦w_1 + ¦w_2, t·¦w_1 \in M^\perp$.

                Ať $M \subseteq N$. Pak $¦v \in N^\perp \Leftrightarrow ¦v \perp N \implies ¦v \perp M \Leftrightarrow ¦v \in M^\perp$.
            \end{dukazin}
        \end{tvrzeni}

        \begin{veta}
            Buď ¦V VP s $\<¦·, ·\>$ nad ®R nebo ®C. Buď $¦W ≤ ¦V$ konečně generovaný. Pak platí:
            $$ 1) ¦V = ¦W \oplus ¦W^\perp. $$ 
            $$ 2) (¦W^\perp)^\perp = ¦W. $$ 
            3) Každý vektor $¦v \in ¦V$  má (jednoznačnou) ortogonální projekci jak na ¦W, tak ne $¦W^\perp$.\\
            4) Je-li ¦V konečně generovaný dimenze $n$, pak $n = \dim ¦W + \dim ¦W^\perp$.

            \begin{dukazin}
                1) Triviálně $¦W \cap ¦W^\perp = \{¦o\}$. Navíc použitím toho, že existuje ortogonální projekce (a toho, že je kolmá) na ¦W máme, že $¦W + ¦W^\perp = ¦V$.

                2) $¦W \subseteq (¦W^\perp)^\perp$: je-li $¦w \in ¦W$, pak $w \perp (¦W^\perp)$, tj. $w \in (¦w^\perp)^\perp$. Naopak $(¦W^\perp)^\perp \subseteq ¦W$: vezměme $¦v \in (¦W^\perp)^\perp$. Uvažujme ortogonální projekci $¦v$ na ¦W:
                $$ (¦W^\perp)^\perp \ni ¦v = ¦w + (¦v - ¦w) \land ¦v - ¦w \in (¦W^\perp)^\perp \implies (¦v-¦w)\perp(¦v - ¦w) \implies ¦v - ¦w = ¦o \implies ¦v = ¦w \in ¦W. $$

                Víme, že ortogonální projekce na ¦W existuje. Je-li tedy $¦v \in ¦V$, pak můžeme psát $¦v = ¦w + (¦v - ¦w) = (¦v - ¦w) + ¦w$, potom $(¦v - ¦w) \in ¦W^\perp$ je podle definice ortogonální projekce na $¦W^\perp$. ($¦w \in (¦W^\perp)^\perp$.)

                Použijeme 1) a větu o dimenzi součtu a průniku podprostorů.
            \end{dukazin}
        \end{veta}

        \begin{definice}[Gramova matice]
            Buď ¦V VP s $\<¦·, ·\>$ nad ®R nebo ®C. Buď $(¦u_1, …, ¦u_k)$ posloupnost vektorů. Pak Gramovu matici posloupnosti $(¦u_1, …, ¦u_k)$ definujeme jako:
            $$ (\<¦u_i, ¦u_j\>)_{k \times k}. $$ 
        \end{definice}

        \begin{tvrzeni}
            Buď ¦V VP s $\<¦·, ·\>$, $(¦u_1, …, ¦u_k)$ posloupnost vektorů ¦V, $B$ Gramova matice. Vezměme $¦v \in ¦V$ a $¦w = a_1¦u_1 + … + a_k¦u_k \in ¦W := \LO\{¦u_1, …, ¦u_k\}$. Pak následující je ekvivalentní:\\
            1) ¦w je ortogonální projekce ¦v na ¦W.\\
            2) $B·\(a_1, …, a_k\)^T = (\<¦u_1, ¦v\>, …, \<¦u_k, ¦v\>)$.

            \begin{dukazin}
                1) $\Leftrightarrow ¦v - ¦w \perp ¦W \Leftrightarrow \forall i \in [k]: \Leftrightarrow ¦u_i \perp ¦v - ¦w \Leftrightarrow \forall i \in [k]: \<¦u_i, ¦v - ¦w\> = 0 \Leftrightarrow \forall [k]: \<¦u_i, ¦w\> = \<¦u_i, ¦v\> \Leftrightarrow \forall i \in [k]: \<¦u_i, ¦u_1\>·a_1 + \<¦u_i, ¦u_k\>·a_k =\<¦u_i, ¦v\> \Leftrightarrow 2).$
            \end{dukazin}
        \end{tvrzeni}

        \begin{dusledek}
            Buď $A$ matice typu $n \times k$ nad ®R nebo ®C. Buď $¦v \in ®R^n$ nebo $®C^n$ a $x \in ®C^k$ nebo $®R^k$. Pak následující je ekvivalentní:\\
            1) $Ax$ je ortogonální projekce ¦v na $\Im A$.\\
            2) $A^*A·x = A^*·¦v.$
        \end{dusledek}

        \begin{tvrzeni}[8.80]
            Buď $(¦u_1, …, ¦u_k)$ posloupnost vektorů ve VP ¦V s $\<·, ·\>$ a buď $B \in T^{k \times k}$ gramova matice. Pak platí:\\
            1) $B$ je regulární $\Leftrightarrow$ $(¦u_1, …, ¦u_k)$ LN.\\
            2) $B$ je hermitovská (v reálném případě symetrická).
            3) Je-li $(¦u_1, …, ¦u_k)$ LN, pak $B$ je pozitivně definitní.

% 19. 3. 2021

            \begin{dukazin}
                1) aplikujeme tvrzení výše na $¦v = ¦o$. První podmínka se přepíše na $0 = a_1¦u_1 + … + a_k¦u_k \Leftrightarrow B·(a_1, …, a_k)^T = ¦o \Leftrightarrow (a_1, …, a_k)^T \in \Ker B$. Ale jádro je $\{¦o\}$ $\Leftrightarrow$ $B$ je regulární.

                2) Plyne z rovnosti: $\<¦u_i, ¦u_j\> = \overline{\<¦u_j, ¦u_i\>}$.

                3) Vezměme ortonormální bázi $C$ prostoru $\LO\{¦u_1, …, ¦u_k\}$ a položme $A = \([¦u_1]_C|…|[¦u_k]_C\)$. Pak $A$ je regulární, tj. $A^*A$ je pozitivně definitní.
            \end{dukazin}
        \end{tvrzeni}

    \subsection{Unitární a ortogonální matice}
        \begin{definice}[Unitární a ortogonální matice]
            Čtvercová matice nad ®R se nazývá ortogonální, pokud má ortonormální posloupnost sloupců vzhledem ke standardnímu skalárnímu součinu.

            Čtvercová matice nad ®C se nazývá unitární, pokud má ortonormální posloupnost sloupců vzhledem ke standardnímu skalárnímu součinu.
        \end{definice}

        \begin{tvrzeni}
            Buď $Q$ čtvercová komplexní matice řádu $n$. Pak následující je ekvivalentní: 1) $Q$ je unitární, 2) $Q^*·Q = I_n$, 3) $Q^*$ je unitární, 4) $Q·Q^*=I_n$, 5) $Q^T$ je unitární, 6) $f_Q$ zachovává standardní skalární součin, tj. $\forall ¦u, ¦v \in ®C^n: f(¦u)·f(¦v) = ¦u·¦v$.

            Speciálně je každá unitární matice regulární a $Q^{-1} = Q^*$.

            \begin{dukazin}
                $1) \Leftrightarrow 2)$, $3) \Leftrightarrow 4)$: z definice. $2) \implies 4)$: $2) \implies Q$ má levou inverzi $Q^*$ $\implies Q$ regulární a $Q^{-1} = Q^*$ $\implies$ $Q·Q^{-1} = Q·Q^* = I_n$. $4) \implies 2)$: analogicky.

                $3) \Leftrightarrow 5)$: 5) říká, že $Q$ má ortonormální posloupnost řádků, 3) říká, že když komplexně sdružíme všechny prvky $Q$, pak dostaneme ortonormální posloupnost řádků. Z toho to už jednoduše dostaneme.

                $2) \implies 6)$: Předpokládejme 2), uvažujme $¦u, ¦v \in ®C^n$. Pak $f_Q(¦u)·f_Q(¦v) = (Q¦u)^*(Q¦v) = ¦u^*(Q^*Q)¦v = ¦u^*¦v$.

                $6) \implies 1)$ $Q = \(f_Q(e_1)| …| f_Q(e_n)\) \implies \(f_Q(e_1), …, f_Q(e_n)\)$ ortonormální $ \implies Q$ unitární.
            \end{dukazin}
        \end{tvrzeni}

        \begin{dusledek}
            Součin unitárních matic stejného řádu je unitární matice.

            \begin{dukazin}
                $(AB)^*(AB) = B^*A^*AB = B^*B = I_n$.
            \end{dukazin}
        \end{dusledek}

        \begin{tvrzeni}
            Je-li $A$ regulární komplexní matice a $Q_1R_1 = A = Q_2R_2$ jsou 2 QR rozklady, pak nutně $Q_1 = Q_2$ a $R_1 = R_2$.

            \begin{dukazin}
                Z regularity $Q_1R_1 = Q_2R_2 \implies Q_2^*Q_1 = Q_2^{-1}Q_1 = R_2R^{-1} =: (¦c_1|…|¦c_n)$. Chceme ukázat, ze $¦c_i = e_i \forall i$. To ukážeme indukcí podle $i$. Víme, že $R_2R_1^{-1}$ je horní trojúhelníková, tedy každé $¦c_i$ musí mít kladný prvek na $i$-té pozici a zároveň všude výše musí mít nulu, aby byl kolmý ke všem předchozím (o kterých z IP víme, že jsou to jednotkové vektory).
            \end{dukazin}
        \end{tvrzeni}

        \begin{definice}
            Buď ¦V komplexní VP s $\<·, ·\>_{¦V}$ a ¦W komplexní VP s $\<·, ·\>_{¦W}$. Pak lineární zobrazení $f: ¦V \rightarrow ¦W$ se nazývá unitární, pokud $\forall ¦u, ¦v \in ¦V: \<f(¦v), f(¦w)\>_{¦W} = \<¦u, ¦v\>_{¦V}$.
        \end{definice}

        \begin{tvrzeni}
            Buď $f: ¦V \rightarrow ¦W$ lineární zobrazení, ¦V, ¦W komplexní VP se skalárním součinem, pak následující je ekvivalentní: 1) $f$ je unitární, 2) $\forall ¦u \in ¦V: ||f(¦u)_{¦W} = ||¦u||_{¦V}$ ($f$ zachovává normu), 3) $f$ zobrazí každou ortonormální posloupnost $(¦u_1, …, ¦u_k)$ na ortonormální posloupnost $(f(¦u_1, …, f(¦u_k)))$, 4) $f$ zobrazuje jednotkové vektory na jednotkové vektory.

            Speciálně: každé unitární zobrazení je prosté.

            \begin{dukazin}
                Ve skriptech. Dodatek plyne z 2) a $f$ prosté $\Leftrightarrow \Ker f = \{¦o\}$. $1) \implies 2) \implies 4)$, $1) \implies 3) \implies 4)$ jednoduché. $4) \implies 2)$: $¦o≠¦v \in ¦V$, pak $¦v = t¦u$ pro $t = ||¦v||_{¦V}$, ¦u jednokový. $||f(¦v)||_{¦W} = t·||f(¦u)||_{¦W} = t = ||¦v||_{¦V}$.

                $2) \implies 1)$: Polarizační identity: $\Re\<¦u, ¦v\>, \Im\<¦u, ¦v\> = \frac{1}{2}(…)$.
            \end{dukazin}
        \end{tvrzeni}

        \begin{poznamka}
            Unitární zobrazení může zobrazovat i do prostoru větší dimenze.
        \end{poznamka}

    \subsection{Přibližné řešení SLR metodou nejmenších čtverců}
        \begin{definice}
            Vektor $¦c \in ®C^n$ je přibližné řešení SLR $A¦x = ¦b$ metodou nejmenších čtverců, pokud
            $$ ||A¦c - ¦b|| = \min_{¦x \in ®C^n} ||A¦x - ¦b||. $$ 
        \end{definice}

        \begin{dusledek}
            ¦c je ortogonální projekce ¦b do $\im A$.
        \end{dusledek}

        \begin{poznamka}
            Používá se například, když chybou měření soustava nemá řešení, ale my víme, že řešení mít má.

            Jmenuje se podle čtverců ve výpočtu normy.
        \end{poznamka}

        \begin{tvrzeni}
            ¦c je přibližné řešení $A¦x = ¦b$ metodou nejmenších čtverců, právě když $A^*A¦x = A^*¦b$.
        \end{tvrzeni}

% 24. 3. 2021

\section{Lineární dynamické systémy, vlastní čísla a vlastní vektory}

% 26. 3. 2021

    \begin{definice}[Vlastní čísla a vlastní vektory]
        Buď ®T těleso, $A$ čtvercová matice řádu $n$ (tj. máme $f_a: ®T^n \rightarrow ®T^n$). $\lambda \in ®T$ se nazývá vlastní číslo matice $A$, pokud $\exists ¦v \in T^n, ¦v ≠ ¦o$ takový, že $A·¦V = \lambda·¦v$. Je-li $\lambda \in ®T$ vlastní číslo matice $A$, pak $¦w \in ®T^n$ je vlastním vektorem příslušným k $\lambda$, pokud $A·¦w = \lambda·¦w$.
    \end{definice}

    \begin{definice}[Vlastní čísla a vlastní vektory]
        Buď ®T těleso, ¦V VP nad ®T a $f: ¦V \rightarrow ¦V$ lineární operátor. $\lambda \in ®T$ se nazývá vlastní číslo operátoru $f$, pokud $\exists ¦v \in ¦V, ¦v ≠ ¦o$ takový, že $f(¦V) = \lambda·¦v$. Je-li $\lambda \in ®T$ vlastní číslo operátoru $f$, pak $¦w \in ¦V$ je vlastním vektorem příslušným k $\lambda$, pokud $f(¦w) = \lambda·¦w$.
    \end{definice}

    \begin{pozorovani}
        $A$ má vlastní číslo 0 $\Leftrightarrow$ $\Ker A ≠ \{¦o\}$ $\Leftrightarrow$ (pro čtvercové) $A$ je singulární $\Leftrightarrow$ $\det A = 0$.

        $f$ má vlastní číslo 0 $\Leftrightarrow$ $\Ker f ≠ \{¦o\}$.

        Navíc množina vlastních vektorů příslušných k 0 je přesně $\Ker A$ ($\Ker f$).
    \end{pozorovani}

    \begin{pozorovani}
        $A$ má vlastní číslo $\lambda$ $\Leftrightarrow$ $\Ker (A - \lambda I_n) ≠ \{¦o\}$ $\Leftrightarrow$ $A - \lambda I_n$ singulární $\Leftrightarrow$ $\det(A - \lambda I_n) = 0$.

        $f$ má vlastní číslo $\lambda$ $\Leftrightarrow$ $\Ker (f - \lambda·\id_{¦V}) ≠ \{¦o\}$.

        Navíc množina $M_{\lambda}$ vlastních vektorů $A$ (resp. $f$) příslušných k $\lambda$ je v tom případě rovna $\Ker(A - \lambda I_n)$ (resp. $\Ker (f - \lambda·\id_{¦V})$). Speciálně $M_\lambda ≤ ®T^n$ (resp. $M_\lambda ≤ ¦V$).
    \end{pozorovani}

    \begin{definice}[Charakteristický polynom]
        Buď $A$ čtvercová matice nad ®T. Potom charakteristickým polynomem $A$ rozumíme polynom v $\lambda$:
        $$ p_A(\lambda) = \det(A - \lambda I_n). $$ 
    \end{definice}

    \begin{tvrzeni}
        Buď $A = (a_{ij})$ matice řadu $n$ nad ®T. A $p_A(\lambda)$ charakteristický polynom. Pak

        \begin{enumerate}
            \item $p_A(\lambda)$ je polynom stupně $n$.
            \item Koeficient u $\lambda^n$ je roven $(-1)^n$.
            \item Koeficient u $\lambda^{n-1}$ je roven $(-1)^{n-1}·(a_{11} + … + a_{nn})$ (tzv. stopa matice $·(-1)^{n-1}$).
            \item Absolutní člen je roven $\det A$.
        \end{enumerate}
    \end{tvrzeni}

    \begin{definice}[Podobné matice]
        Čtvercové matice $X$ a $Y$ jsou podobné, pokud $Y = RXR^{-1}$ pro $R$ regulární.
    \end{definice}

    \begin{tvrzeni}
        $X, Y$ podobné $\implies$ $p_X(\lambda) = p_Y(\lambda)$.
    \end{tvrzeni}

% 31. 3. 2021

    \begin{definice}[Diagonalizovatelný operátor]
        ®T těleso, ¦V konečně generovaný vektorový prostor. $f: ¦V \rightarrow ¦V$ lineární operátor. Pak $f$ je diagonalizovatelný, pokud $\exists$ báze $B$ prostoru ¦V taková, že $[f]_B^B$ je diagonální.
    \end{definice}

    \begin{poznamka}[Značení]
        $$ \diag(\lambda_1, \lambda_2, …, \lambda_n) := \begin{pmatrix} \lambda_1 & & … & \\ & \lambda_2 & … & \\ \vdots & \vdots & \ddots & \\ & & & \lambda_n \end{pmatrix}. $$
    \end{poznamka}

    \begin{tvrzeni}
        Buď $f: ¦V \rightarrow ¦V$ lineární operátor na konečně generovaném VP ¦V nad ®T a buď $B = (¦v_1, …, ¦v_n)$ nějaká báze. Pak:
        $$ [f]_B^B = \diag(\lambda_1, …, \lambda_n) \Leftrightarrow \forall i: ¦v_i \text{ je vlastní vektor příslušný k } \lambda_i. $$ 
    \end{tvrzeni}

    \begin{dusledek}
        Za stejných předpokladů: $f$ diagonalizovatelný $\Leftrightarrow$ ¦V má bázi z vlastních vektorů $f$.
    \end{dusledek}

    \begin{definice}[Diagonalizovatelnost pro matice]
        Buď ®T těleso a $A$ čtvercová matice řádu $n$ nad ®T. Pak $A$ je diagonalizovatelná, pokud je $f_A: ®T^n \rightarrow T^n$ diagonalizovatelný lineární operátor.
    \end{definice}

    \begin{dusledek}
        $A$ je diagonalizovatelná $\Leftrightarrow$ $®T^n$ má bázi z vlastních vektorů $A$.
    \end{dusledek}

    \begin{tvrzeni}
        Buď $A \in ®T^{n \times n}$. Pak následující tvrzení jsou ekvivalentní:
        
        \begin{enumerate}
            \item $A$ je diagonalizovatelná.
            \item $®T^n$ má bázi z vlastních vektorů matice $A$.
            \item $A$ je podobná diagonální matici.
        \end{enumerate}
    \end{tvrzeni}

    \subsection{Lineární nezávislost vlastních vektorů}
        \begin{veta}
            Buď ®T těleso a $f: ¦V \rightarrow ¦V$ lineární operátor na VP ¦V nad ®T. Je-li $(¦v_1, …, ¦v_n)$ posloupnost vlastních vektorů $f$ popořadě příslušných k vlastním číslům $(\lambda_1, …, \lambda_n)$ a $\lambda_1, …, \lambda_n$ jsou po dvou různá, pak je $(¦v_1, …, ¦v_n)$ LN.

            \begin{dukazin}
                Indukcí podle $n$. $n=1$ je triviální ($¦v_1 ≠ 0$). $n>1$: Uvažujme $a_1, …, a_{n-1} \in ®T$ taková, že $a_1¦v_1 + … + a_n¦v_n = ¦o$. $a_1f(¦v_1) + … + a_nf(¦v_n) = 0$ $\Leftrightarrow$ $\lambda_1a_1¦v_1 + … + \lambda_na_n¦v_n = 0$. Také můžeme původní rovnici vynásobit $\lambda_n$: $\lambda_na_1¦v_1 + … + \lambda_na_n¦v_n = 0$. Následně můžeme odečíst tyto rovnice od sebe a dostaneme $(\lambda_1 - \lambda_n)a_1¦v_1 + … + (\lambda_{n-1} - \lambda_n)a_{n-1}¦v_{n-1} = 0$. Ale z IP $(\lambda_1 - \lambda_n)a_1 = … = (\lambda_{n-1} - \lambda_n)a_{n-1} = 0$. Ale vlastní čísla jsou po dvou různá, tedy $a_1 = … = a_{n-1} = 0$.
            \end{dukazin}
        \end{veta}

        \begin{dusledek}
            Má-li lineární operátor $f: ¦V \rightarrow ¦V$ na prostoru dimenze $n$ $n$ po 2 různých vlastních čísel, pak je $f$ diagonalizovatelný.

            Má-li čtvercová matice $A$ řádu $n$ $n$ po 2 různých vlastních čísel, pak je $A$ diagonalizovatelná.
        \end{dusledek}

% 7. 4. 2021

    \subsection{Geometrická násobnost vlastních čísel}
        \begin{definice}[Geometrická násobnost]
            Buď $A$ čtvercová matice řádu $n$ nad ®T, $\lambda \in ®T$ vlastní číslo. Uvažujme podprostor
            $$ M_\lambda := \{¦v \in ®T^n | A¦v = \lambda ¦v\} ≤ ®T^n. $$
            Pak geometrickou násobností $\lambda$ rozumíme $\dim M_\lambda$.
        \end{definice}

        \begin{tvrzeni}
            Máme-li lineární operátor $f: ¦V \rightarrow ¦V$ na konečně dimenzionálním vektorovém prostoru ¦V nad ®T a máme-li vlastní číslo $\lambda$ operátoru $f$, pak geometrická násobnost $\lambda$ $≤$ algebraická násobnost $\lambda$.

            \begin{tvrzeniin}[Pomocné tvrzení o determinantech]
                Buď ®T těleso, $1 ≤ k < n$ a buď $A$ matice blokově trojúhelníkovéůo tvaru, tj. $a_{ij} = 0$ pro $i > k ≥ j$. Pak $\det A = (\det B)·(\det D)$, kde $B$ je prvních $k$ sloupců a řádků, $D$ je posledních $n-k$ sloupců a řádků.

                \begin{dukazin}
                    Indukcí podle $k$: $k = 1$: Rozvoj $\det A$ podle 1. sloupce nám dá chtěnou rovnost. $k > 1$: taktéž vezmeme rozvoj podle prvního sloupce: $\det A = a_{11}\det M_{11} - a_{21} \det M_{21} + …$. $M_{ij}$ jsou ale takové matice pro $k \leftarrow k-1$. Tedy je spočítáme: $\det A = a_{11}(\det B_{11})(\det D) - a_{21}(\det B_{21})(\det D) + … = (\det B)·(\det D)$.
                \end{dukazin}
            \end{tvrzeniin}

            \begin{dukazin}
                Buď $k$ geometrická násobnost vlastního čísla $\lambda$. Tj. $\dim M_\lambda = k$, kde $M_\lambda = \{¦v \in ¦V | f(v) = \lambda ¦v\} ≤ ¦V$. Vezmeme si bázi $(¦v_1, …, ¦v_k)$ prostoru $M_\lambda$ a doplníme na bázi $B = (¦v_1, …, ¦v_n)$ prostoru ¦V. $A = [f]_B^B$ splňuje předpoklady předchozího tvrzení ($B = \diag(\lambda, …, \lambda)$). Tedy $\det A - \mu·I_n = \(\lambda - \mu\)^k·\det D$. Tedy $\lambda$ je minimálně $k$-násobným kořenem $\det A - \mu·I_n$, tedy má algebraickou násobnost $≥$ geometrické. 
            \end{dukazin}
        \end{tvrzeni}

        \begin{veta}
            Buď $f: ¦V \rightarrow ¦V$ lineární operátor na VP dimenze $n$ nad ®T. Pak následující tvrzení jsou ekvivalentní:

            \begin{enumerate}
                \item $f$ je diagonalizovatelný.
                \item $f$ má $n$ vlastních čísel včetně algebraických násobností a současně geometrická násobnost každého vlastního čísla je rovna jeho algebraické násobnosti.
            \end{enumerate}
            
            \begin{dukazin}
                $1) \implies 2)$: $f$ je diagonalizovatelný, tedy máme bázi $B = (¦v_1, …, ¦v_n)$ z vlastních vektorů. Řekněme, že $¦V_1, …, ¦v_{m_1}$ jsou vlastní vektory příslušné vlastnímu číslu $\lambda_1$, …, $¦v_{m_1 + m_2 + … + m_{k-1} + 1}, …, ¦v_{v_n}$ jsou vlastní vektory příslušné vlastnímu číslu $\lambda_k$. $\forall i \in [k]:$ Geometrická násobnost i algebraická násobnost $\lambda_i$ je $≥ m_i$. Na druhou stranu součet algebraických násobností je nejvýše $n$, tedy násobnosti $\lambda_i$ jsou $m_i$.

                $2) \implies 1)$: Buďte $\lambda_1, …, \lambda_k$ po dvou různá vlastní čísla, $m_i$ algebraická (tj. i geometrická) násobnost $\lambda_i$ = $\dim M_{\lambda_i}$ a $n = m_1 + … + m_k$. Zvolme si bázi $M_{\lambda_i}$ $B_i = (¦v_1^i, …, ¦v_{m_i}^i)$. Ukáže se, že $B = (¦v_1^1, …, v_{m_k}^k)$ je báze ¦V. $B$ má $n = \dim ¦V$ prvků, tedy stačí dokázat, že $B$ je LN. Uvažujme $0 = a_1^1¦v_1^1 + … + a_{m_k}^k¦v_{m_k}^k$. Součty násobků prvků jednotlivých prostorů $M_{\lambda_i}$ jsou zase v $M_{\lambda_i}$, takže jsou (až na právě ty nulové) LN, protože jsou to vlastní vektory příslušné po dvou různým vlastním číslům. Tedy musí být všechny nulové, ale v jednotlivých $M_{\lambda_i}$ byla $(¦v_1^i, …, ¦v_{m_i}^i)$ báze, tedy všechny koeficienty musí být nulové. 
            \end{dukazin}
        \end{veta}

% 9. 4. 2021

        \begin{poznamka}
            Dále se probírali lineární systémy v $®R^2$. Viz přednáška.
        \end{poznamka}

% 14. 4. 2021

    \subsection{Jordanův kanonický tvar}
        \begin{definice}[Jordanova buňka]
            Buď ®T těleso, $k ≥ 1$ a $\lambda \in ®T$. Pak Jordanovou buňkou řádu $k$ příslušnou k prvku $\lambda$ rozumíme matici:
            $$ J_{\lambda, k} = \begin{pmatrix} \lambda & 1 & & … & 0 \\ & \lambda & 1 & … & \\ & & \ddots & ddots & & \\ & & & & & 1 \\ 0 & & & & & \lambda \end{pmatrix}. $$ 
        \end{definice}

        \begin{definice}
            Řekneme, že čtvercová matice $J$ nad tělesem ®T je v Jordanově kanonickém tvaru, pokud je blokově diagonální a čtvercové bloky na diagonále jsou Jordanovy buňky. Tj.
            $$ J = \diag(J_{\lambda_1, k_1}, …, J_{\lambda_s, k_s}). $$ 
        \end{definice}

        \begin{definice}[Zobecnění diagonalizovatelnosti]
            Buď $f: ¦V \rightarrow ¦V$ lineární operátor na konečně generovaném VP nad ®T. Řekneme, že existuje Jordanův kanonický tvar (operátoru $f$), pokud existuje báze $B$ prostoru ¦V taková, že $[f]_B^B$ je v Jordanově tvaru.

            Podobně je-li $A$ čtvercová matice nad ®T, pak $A$ má Jordanův kanonický tvar, pokud pro $f_A$ existuje Jordanův kanonický tvar.
        \end{definice}

        \begin{pozorovani}
            $A$ má jordanův kanonický tvar $\Leftrightarrow$ $\exists$ matice v Jordanově tvaru, která je podobná matici $A$.
        \end{pozorovani}

        \begin{tvrzeni}[Mocnění matice v J. tvaru]
            Buď ®T těleso a $J = \diag(J_{\lambda_1, k_1}, …, J_{\lambda_s, k_s})$ matice v Jordanově tvaru.

            \begin{pozorovaniin}
                $J^n = \diag(J_{\lambda_1, k_1}^m, …, J_{\lambda_s, k_s}^m)$.
            \end{pozorovaniin}

            \begin{tvrzeniin}[Mocnění Jordanovy buňky příslušné k 0]
                Pro $m < k$: $J_{0, k}^m = \(0 \mid| \begin{array}{ccc} 1 & … & 0 \\ \vdots & \diag & \vdots \\ 0 & … & 1 \\ 0 & 0 & 0 \\ \vdots & \vdots & \vdots \end{array}\)$.

                Pro $m ≥ k$: $J_{0, k}^m = 0$.
            \end{tvrzeniin}

            \begin{tvrzeniin}[Mocniny obecné Jordanovy buňky]
                $$ J_{\lambda, k}^m = (\lambda·I_k + J_{0, k})^m \overset{\text{Binomická věta, komutativita $I_k$}}{=} \sum_{i=0}^m \binom{m}{i} \lambda^i · J_{0, k}^{m-i}. $$
            \end{tvrzeniin}
        \end{tvrzeni}

        \begin{definice}[Jordanův řetízek, zobecněné vlastní vektory]
            Buď $f: ¦V \rightarrow ¦V$ lineární operátor na VP ¦V, buď $\lambda$ vlastní číslo $f$ a buď $(¦V_1, …, ¦v_k)$ posloupnost vektorů. Řekneme, že $(¦v_1, …, ¦v_k)$ je Jordanův řetízek délky $k$ příslušný vlastnímu číslu $\lambda$ s počátkem $¦v_1$, pokud
            $$ ¦v_k \overset{f - \lambda\id}{\longrightarrow} ¦v_{k-1} \overset{f - \lambda\id}{\longrightarrow} … \overset{f - \lambda\id}{\longrightarrow} ¦v_1 \overset{f - \lambda\id}{\longrightarrow} ¦o. $$
            Vektory $¦v_1, …, ¦v_k$ se pak nazývají zobecněné vlastní vektory příslušné $\lambda$.
        \end{definice}

        \begin{tvrzeni}
            Buď $f: ¦V \rightarrow ¦V$ lineární operátor na konečně generovaném VP ¦V s bází $B = (¦v_1, …, ¦v_k)$ pak $[f]_B^B = J_{\lambda, k} \Leftrightarrow B$ je Jordanův řetízek délky $k$ příslušný k $\lambda$ s počátkem ve $¦v_1$.
        \end{tvrzeni}

        \begin{tvrzeni}
            Buď $f: ¦V \rightarrow ¦V$ lineární operátor na konečně generovaném VP ¦V a $B$ báze ¦V, pak $[f]_B^B = \diag(J_{\lambda_1, k_1}, …, J_{\lambda_s, k_s})$, právě když $B = B_1…B_s$ je spojením Jordanových řetízků $B_1, …, B_s$, kde $B_i$ je délky $k_i$ a příslušný $\lambda_i$.
        \end{tvrzeni}

        \begin{dusledek}
            $f: ¦V \rightarrow ¦V$ má Jordanův kanonický tvar právě tehdy, když existuje báze $B$ prostoru ¦V, která je spojením Jordanových řetízků.
        \end{dusledek}

        \begin{veta}
            Buď ®T těleso, $f: ¦V \rightarrow ¦V$ lineární operátor na VP ¦V nad ®T. Buďte $B_1, …, B_s$ jordanovy řetízky popořadě délek $k_1, …, k_s$ příslušné vlastním číslům $\lambda_1, …, \lambda_s$. Předpokládejme, že $\forall \lambda \in \{\lambda_1, …, \lambda_s\}$ je posloupnost počátků těch $B_i$, které jsou příslušné $\lambda$, LN. Pak už je spojení $B_1…B_s$ LN.

% 16. 4. 2021

            \begin{dukazin}
                Označme $B_i = (¦v_1^i, …, ¦v_{k_i}^i)$, $(f - \lambda_i \id)(¦v_j^i) = ¦v_{j-1}^i$ ($¦v_0^i := ¦o$ kvůli značení). $B:=B_1…B_s$. Indukce podle délky $B$, tj. podle $k = k_1 + k_2 + … + k_s$. Případ $k=1$: Máme nutně $s=1$, $B=B_1=(¦v_1^1)$, z předpokladu $¦v_1^1 ≠ ¦o$, tedy $B$ je LN.

                Pro $k > 1$: BÚNO $\lambda_1 = \lambda_2 = … = \lambda_r$, $\forall i > r: \lambda_i ≠ \lambda_1$. Uvažujme lineární kombinaci $a_1^1¦v_1^1 + … + a_n^{a^s_{k_s}¦v^s_{k_s}} = ¦o$. Na ní aplikujeme $f - \lambda_1 \id_{¦V}: ¦V \rightarrow ¦V$. ¦o zobrazí na ¦o, prvních $¦v^i_1$ pro $i ≤ r$ zobrazí také na ¦o a $¦v^i_j$, kde $i ≤ r$ a $1<j$, zobrazí na $¦v_{j-1}^i$. Ostatní $¦v_i^j$ zobrazí na $\lambda_i¦v_j^i + ¦v_{j-1}^i - \lambda_1¦v_j^i$. Tedy dostaneme lineární kombinaci vektorů z řetízků, ale ubrali jsme vektor řetízků příslušících $\lambda_1$, tedy můžeme použít IP.

                Takto se můžeme zbavit všech členů, kromě: $a_1^1¦v_1^1 + a_1^2¦v_1^2 + … + a_1^r¦v_1^r + … + a_1^s¦v_1^s$. BÚNO shlukneme jednotlivá vlastní čísla. Stejně tak shlukneme vlastní vektory příslušící jednomu vlastnímu číslu, tedy $¦w_k = a_1^i¦v_1^i + …$ jsou vlastní vektory příslušné po dvou různým vlastním číslům (nebo ¦o). Podle věty výše jsou takové vlastní vektory nezávislé, tedy $¦o$. Tudíž $a_i^j = 0$.
            \end{dukazin}
        \end{veta}

        \begin{veta}[Kritérium existence jordanova kanonického tvaru]
            Buď $f: ¦V \rightarrow ¦V$ lineární operátor na prostoru dimenze $n$ nad ®T. Pak následující je ekvivalentní: 1. pro $f$ $\exists$ Jordanův kanonický tvar a 2. $f$ má $n = \dim ¦V$ vlastních čísel včetně algebraické násobnosti.
        \end{veta}

        \begin{dusledek}
            $®T = ®C$: Jordanův kanonický tvar vždy existuje (ze základní věty algebry).
        \end{dusledek}

% 21. 4. 2021

\section{Invariantní podprostory lineárního operátoru}
    \begin{definice}[Invariantní podprostor]
        Buď $f: ¦V \rightarrow ¦V$ lineární operátor na vektorovém prostoru ¦V nad ®T. Pak řekneme, že $¦W ≤ ¦V$ je invariantním podprostorem operátoru $f$, pokud $\forall ¦v \in ¦W : f(¦v) \in ©W$ (jinými slovy $f(¦W) \subseteq ¦W$).
    \end{definice}

    \begin{priklady}[Vždy invariantní podprostory]
        $$ \{¦o\}, ¦V, \Ker f, \Im f, u \text{vlastní vektor} \implies \LO\{¦u\}, (¦u_1, …, ¦u_k) \text{Jordanův řetízek} \implies \LO\{¦u_1, …, ¦u_k\}. $$

        \begin{dukazin}
                $\implies$ viz přednáška. (Nebude u zkoušky?)
        \end{dukazin}
    \end{priklady}

    \begin{pozorovani}
        $\id_{¦V}$, a dokonce $\id_{¦V}·\lambda$ mají (¦V má při těchto zobrazení) všechny podprostory invariantní.
    \end{pozorovani}

    \begin{tvrzeni}
        Buď $f: ¦V \rightarrow ¦V$ lineární operátor a buďte $¦U, ¦W ≤ ¦V$ invariantní podprostory $f$. Pak i $¦U + ¦V$ a $¦U \cap ¦W$ jsou invariantní podprostory $f$.

        \begin{dukazin}
            Pro $¦U + ¦W$: každý vektoru z $¦U + ¦W$ je tvaru $¦u + ¦v$. Potom $f(¦u + ¦v) = f(¦u) + f(¦v) \in ¦U + ¦W$.

            Pro $¦U \cap ¦W$: Buď $¦v \in ¦U \cap ¦W$, tj. $¦v \in ¦U \land ¦v \in ¦W$ $\implies$ $f(¦V) \in ¦U, ¦W$, tj. $f(¦v) \in ¦U \cap ¦W$.
        \end{dukazin}
    \end{tvrzeni}

    \begin{dusledek}
        $\LO$ spojení (konečně mnoha) Jordanových řetízků $f$ je vždy invariantní podprostor $f$.
    \end{dusledek}

    \begin{tvrzeni}
        Buď $f: ¦V \rightarrow ¦V$ lineární operátor na konečně dimenzionálním VP ¦V nad ®T, buď $¦W≤¦V$ invariantní podprostor a $g = f|_{¦W}: ¦W \rightarrow ¦W$. Pak $p_g(\lambda)$ dělí $p_f(\lambda)$.

        \begin{dukazin}
                Zvolíme bázi $C = (¦v_1, …, ¦v_k)$ prostoru ¦W. $C$ rozšíříme na bázi $B = (¦v_1, …, ¦v_k, ¦v_{k+1}, …, ¦v_n)$ prostoru ¦V. Uvažujme $G := [f]_B^B = \([f(¦v_1)]_B | … | [f(¦v_k)]_B | [f(¦v_{k+1})]_B | … | [f(¦v_n)]_B\)$. Prvních $k$ vektorů je z $[¦W]_B$, tedy $G$ je blokově horní trojúhelníková, přičemž levý horní blok je $A = [g]_C^C$. Tedy $p_f(\lambda) = \det(G - \lambda I_n) = \det(A = \lambda I_k)·\det(F - \lambda I_{n-k}) = p_g(\lambda)·…$. (Podle lemma o blokově trojúhelníkové matici. $F$ je pravá dolní matice…)
        \end{dukazin}
    \end{tvrzeni}

    \begin{dusledek}
        Buď $f: ¦V \rightarrow ¦V$, ¦V konečně generovaný, $n = \dim ¦V$. $¦W ≤ ¦V$ invariantní podprostor, $g = f_{¦W}: ¦W \rightarrow ¦W$, $m = \dim ¦W (≤ n)$. Má-li $f$ $n$ vlastních čísel včetně algebraické násobnosti, pak $g$ má $m$ vlastních čísel včetně algebraické násobnosti.

        \begin{dukazin}[Stručně]
            Ekvivalentně dokazujeme: $p_f(\lambda)$ je součinem lineárních polynomů, tedy $p_g(\lambda)$ musí být také součinem lineárních polynomů (viz Algebra), jelikož dělí $p_f(\lambda)$.
        \end{dukazin}
    \end{dusledek}

    \begin{tvrzeni}
        Buď ¦V VP a $¦W ≤ ¦V$ podprostor. Pak jsou-li $f, g: ¦V \rightarrow ¦V$ takové, že ¦W je invariantním podprostorem $f$ i $g$, pak ¦W je invariantním podprostorem $f + g$. Je-li ¦W invariantním podprostorem $f$ a $t \in ®T$, pak ¦W je invariantním podprostorem $t·f$.

        \begin{dukazin}
            $(f + g)(¦w) = f(¦w) + g(¦w) \in ¦W$, tj. ¦W je invariantním podprostorem $f+ g$. Druhá část je analogicky.
        \end{dukazin}
    \end{tvrzeni}

    \begin{dusledek}
        Buď $f: ¦V \rightarrow ¦V$ lineární operátor, $\lambda \in ®T$ a $g = f - \lambda·\id_{¦V}$. Pak $f$ a $g$ mají stejnou množinu invariantních podprostorů.
    \end{dusledek}

    \begin{dukaz}[Kritérium existence jordanova kanonického tvaru]
        $\Leftarrow$: (Nebude u zkoušky?) Položme $n = \dim ¦V$. Následuje indukce podle $n$: $n = 1$ jasné, $n > 1$: Vezměme $\lambda \in ®T$ vlastní číslo $f$ (z předpokladů druhé části věty existuje) a položme $g = f - \lambda\id_{¦V}$. pak máme podprostory $g$ (tedy i $f$): $\Ker g, \dim \Ker g = k > 0$, $\im g$, $\dim \im g = n-k =: m < n$. $h:= f|_{¦W}$ splňuje předpoklad, tedy podle IP má bázi vzniklou spojením Jordanových řetízků.

        Máme tedy řetízky $h$ příslušné nějakému vlastnímu číslu $\lambda$, řekněme, že jich je $r$. Potřebujeme doplnit vektory do $k$. Doplníme je tedy dalšími vlastními vektory. Pak vše jen dáme dohromady a ověříme, že je to báze.
    \end{dukaz}

% 23. 4. 2021

\section{Cayleyho-Hamiltonova věta}
    \begin{poznamka}
        Buď ¦V VP nad ®T a $f: ¦V \rightarrow ¦V$ lineární operátor. Máme VP $Hom(¦V, ¦V)$ nad ®T, jehož prvky jsou lineární operátory: $f, g \in Hom(¦V, ¦V) \implies f+g: ¦V \rightarrow ¦V, ¦v \mapsto f(¦v) + g(¦v)$ a $f \in Hom(¦V, ¦V), t \in ®T \implies tf: ¦V \rightarrow ¦V, ¦v \mapsto t·f(¦v)$, $¦o: ¦V \rightarrow ¦V, ¦v \mapsto ¦o$.

        Je-li $g(x) = \sum c_ix^i$ polynom, definujeme operátor $g(f) = \sum c_i·f^i : ¦V \rightarrow ¦V, ¦v \mapsto \sum c_i f^i(¦v)$.

        Je-li $\dim ¦V = n$, pak $\dim Hom(¦V, ¦V) = n^2$. (Zvolíme-li $B(¦u_1, …, ¦u_n)$ bázi ¦V, dostaneme izomofrismus: $Hom(¦V, ¦V) \rightarrow ®T^{n \times n}, f \mapsto [f]_B^B$).
    \end{poznamka}

    \begin{pozorovani}
        Jsou-li $g(x), h(x)$ polynomy nad ®T, $f: ¦V \rightarrow ¦V$ operátor a je-li $h(x) = g(x)·r(x)$, pak $h(f) = g(f)\circ r(f)$ a $h(f), g(f), r(f): ¦V \rightarrow ¦V$.

        \begin{dukazin}
            $g(x) = \sum c_ix^i$, $r(x) = d_jx^j$, $c_i, d_j \in ®T$, potom $h(x) = g(x)·r(x) = \sum_{k=0}^{d+l}(\sum_{i+j = k} c_id_j)·x^{k}$. Dosadí se $h(f)$ a $g(f)·r(f)$ a vyjdou stejné operátory.
        \end{dukazin}
    \end{pozorovani}

    \begin{veta}[Cayleyho-Hamiltonova]
        ®T je těleso, ¦V konečně generovaný VP nad ®T a $f:¦V \rightarrow ¦V$ lineární operátor. Pak
        $$ p_f(f) = 0. $$

        \begin{dukazin}
            Budeme požadovat, aby $p_f$ byl součinem lineárních polynomů, tj. $p_f(\lambda) = (\lambda_1 - \lambda)^{k_1}·…·(\lambda_r - \lambda)^{k_r}$, $k_i \in ®N$, $\lambda_1, …, \lambda_r \in ®T$ (tj. aby měl $f$ Jordanův kanonický tvar). V případě potřeby budeme pracovat nad rozkladovým tělesem, viz Algebra.

            Buď $B$ báze vzniklá spojením Jordanových řetízků, tj. $[f]_B^B = \diag(J_{\lambda_1, k_1}, …, J_{\lambda_r, k_r})$, $p_f(\lambda) = (\lambda_1 - \lambda)^{k_1}·…·(\lambda_r - \lambda)^{k_r}$. Všimněme si, že z mocnění Jordanových buněk plyne, že $(J_{\lambda_i, k_i} - \lambda_i·I_{k_i})^{k_i} = 0_{k_i \times k_i}$, tedy $p_f(J_{\lambda_i, k_i}) = o_{k_i \times k_i}$. Tedy $[p_f(f)]_B^B = p_f([f_B^B]) = 0_{r\times r}$, jelikož mocnění a násobení blokově diagonální matice odpovídá mocnění a násobení bloků.
        \end{dukazin}
    \end{veta}

    \begin{poznamka}
        Dále se pokračovalo vysvětlováním LA v Googlu ;)
    \end{poznamka}

% 28. 4. 2021

\section{Unitární diagonalizovatelnost}
    \begin{definice}
        Buď $A \in ®C^{n \times n}$ čtvercová komplexní matice řádu $n$. Pak $A$ je unitárně diagonalizovatelná, pokud existuje ortonormální báze $B$ prostoru $®C^n$ taková, že $[f]_B^B = \diag(\lambda_1, …, \lambda_n)$ je diagonální ($\lambda_1, …, \lambda_n \in ®C$ jsou vlastní čísla).
    \end{definice}

    \begin{definice}
        Buď $A \in ®R^{n \times n}$ čtvercová reálná matice řádu $n$. Pak $A$ je ortogonálně diagonalizovatelná, pokud existuje ortonormální báze $B$ prostoru $®R^n$ taková, že $[f]_B^B = \diag(\lambda_1, …, \lambda_n)$ je diagonální ($\lambda_1, …, \lambda_n \in ®R$ jsou vlastní čísla).
    \end{definice}

    \begin{definice}
        Matice $X, Y \in ®C^{n \times n}$ jsou unitárně podobné, pokud $\exists U$ unitární matice taková, že $Y = U^*XU (= U^{-1}XU)$.
    \end{definice}

    \begin{definice}
        Matice $X, Y \in ®R^{n \times n}$ jsou ortogonálně podobné, pokud $\exists U$ ortogonální matice taková, že $Y = U^*XU (= U^{-1}XU)$.
    \end{definice}

    \begin{tvrzeni}
        Buď $A \in ®C^{n \times n}$. Pak následující je ekvivalentní: 1. $A$ je unitárně diagonalizovatelná. 2. $®C^n$ má ortonormální bázi složenou z vlastních vektorů $A$. 3. $A$ je unitárně podobná diagonální matici (v tom případě jsou na diagonále takové diagonální matice vlastní čísla $A$, včetně násobnosti).

        \begin{dukazin}
            Cvičení / opakování.
        \end{dukazin}
    \end{tvrzeni}

    \begin{veta}
        Buď $A \in ®C^{n \times n}$. Pak NTJE: 1. $A$ je unitárně diagonalizovatelná, 2. Platí současně, že $A$ má $n$ vlastních čísel včetně algebraické násobnosti (pro ®C splněno automaticky), geometrická násobnost každého čísla je rovna algebraické a $\forall$ dvojici různých vlastních čísel $\lambda$ a $\mu$, $\lambda ≠ \mu$, platí, že $M_\lambda \perp M_\mu$.

        \begin{dukazin}
            $1 \implies 2$: První dvě vlastnosti plynou z dřívější věty o diagonalizovatelnosti. Navíc z této věty víme, že ortonormální báze $B$ z vlastních vektorů $A$, kterou nám dává předpoklad 1, je tvaru $B = B_1…B_k$, kde $B_i$ je (nutně ortonormální) báze $M_{\lambda_i}$, $\lambda_1, …, \lambda_k$ jsou všechna po 2 různá vlastní čísla $A$. Navíc $B_i \perp B_j \forall i ≠ j$, tj. $M_{\lambda_i} = \LO\{B_i\} \perp \LO\{B_j\} = M_{\lambda_j}$.

            $2 \implies 1$: Ať $\lambda_1, …, \lambda_k$ jsou všechna vlastní čísla $A$, po 2 různá. Zvolíme $\forall i$ ortonormální bázi $B_i$ podprostoru $M_{\lambda_i}$, položíme $B = B_1…B_k$ $\implies$ $B$ je ortonormální báze (je to báze ze zmiňované věty), $[f_A]_B^B$ je diagonální.
        \end{dukazin}
    \end{veta}

    \begin{pozorovani}
        Buď $x \in ®C^m, y \in ®C^n, A \in C^{m \times n}$. Pak platí $x·(Ay) = (A^*x)·y$. ($·$ je skalární součin.)

        \begin{dukazin}
            $x·Ay = x^*(Ay) = (x^*A^{**})y = (A^*x)^*y = A^*x·y$.
        \end{dukazin}
    \end{pozorovani}

% 30. 4. 2021

\section{Spektrální věty}
    \begin{definice}[Normální matice]
        $A \in ®C^{n \times n}$ je normální, pokud $A^*A = AA^*$.
    \end{definice}

    \begin{poznamka}[Vlastnosti]
        $$ A \text{ normální}, t \in ®C \implies t·A \text{ normální}, A^* \text{ normální}. $$ 
        Normální jsou matice: diagonální, Hermitovské, Antihermitovské, Unitární, …
    \end{poznamka}

    \begin{tvrzeni}
        Buď $A \in ®C^{n \times n}$ a $\lambda \in ®C$. Pak $\lambda$ je vlastní číslo $A$ $\Leftrightarrow$ $\overline{\lambda}$ je vlastní číslo $A^*$.

        \begin{dukazin}
            $\lambda$ je vlastní číslo $A$ $\Leftrightarrow$ $(A - \lambda I_n)$ není invertibilní $\Leftrightarrow$ $(A - \lambda I_n)^* = A^* - \overline{\lambda}I_n$ není invertibilní $\Leftrightarrow$ $\overline{\lambda}$ je vlastní číslo $A^*$.
        \end{dukazin}
    \end{tvrzeni}

    \begin{tvrzeni}
        Buď $A$ normální komplexní matice řádu $n$. Pak
        $$ 1) \forall t \in ®C: A - t·I_n \text{ je normální}. $$
        $$ 2) \forall U \text{ unitární}: UAU^* \text{ je normální}. $$

        \begin{dukazin}
            $$ 1) (A - t·I_n)^*(A - t·I_n) = (A^* - \overline{t}·I_n)(A - t·I_n) = A^*A - t·A^* - \overline{t}·A + |t|^2I_n = … = (A - t·I_n)(A - t·I_n)^*. $$
            $$ 2) (UAU^*)^*(UAU^*) = (UA^*U^*)(UAU^*) = UA^*U^*UAU^* = UA^*AAU^* = UAA^*U^* = … = (UAU^*)(UAU^*)^*. $$ 
        \end{dukazin}
    \end{tvrzeni}

    \begin{tvrzeni}
        Je-li $A$ normální řádu $n$, pak $\forall ¦v \in ®C^n: ||A·¦v|| = ||A^*¦v||$.

        \begin{dukazin}
                $$ ||A¦v||^2 = A¦v·A¦v = (A¦v)^*(A¦v) = ¦v^*A^*A¦v = ¦v^*AA^*¦v = (A^*¦v)^*(A^*¦v) = A^*¦v·A^*¦v = ||A^*¦v||^2. $$ 
        \end{dukazin}
    \end{tvrzeni}

    \begin{poznamka}
        $\forall ¦v \in ®C^n: ||A·¦v|| = ||A^*¦v||$ je ekvivalentní normalitě.
    \end{poznamka}

    \begin{tvrzeni}
        Buď $A \in ®C^{n \times n}$ normální, buď $\lambda \in ®C$ a $¦v \in ®C^n$. Pak
        $$ A¦v = \lambda¦v \Leftrightarrow A^*¦v = \overline{\lambda}¦v $$ 
        čili množiny vlastních vektorů $A$ a $A^*$ jsou stejné.

        \begin{dukazin}
            $$ A¦v = \lambda ¦v \Leftrightarrow (A - \lambda I_n)·¦v = ¦o \Leftrightarrow ||(A - \lambda I_n)·¦v|| = 0 \Leftrightarrow ||(A - \lambda I_n)^*¦v|| = 0 \Leftrightarrow $$
            $$ \Leftrightarrow (A - \lambda I_n)^*¦v = ¦o \Leftrightarrow (A^* - \overline{\lambda}I_n)¦v = ¦o \Leftrightarrow A^*¦v = \overline{\lambda}¦v. $$
        \end{dukazin}
    \end{tvrzeni}

    \begin{veta}[Spektrální věta pro normální matice]
        Buď $A \in ®C^{n \times n}$. Pak $A$ je normální $\Leftrightarrow$ $A$ je unitárně diagonalizovatelná.

        \begin{dukazin}
            $\Leftarrow$: Buď tedy $A = UDU^*$, $D$ diagonální. Pak $D$ je normální a $A$ je normální podle tvrzení výše bod 2.

            $\implies$: Důkaz indukcí podle $n$. $n = 1$: $A$ je diagonální a není co dokazovat. $n > 1:$ Buď $A \in ®C^{n \times n}$ normální a $\lambda$ vlastní číslo $A$ a $¦o ≠ ¦v_1 \in ®C^n$ příslušný vlastní vektor. BÚNO $||¦v_1|| = 1$, můžeme doplnit na ortonormální bázi $B = (¦v_1, ¦v_2, …, ¦v_n)$ prostoru $®C^n$. Označme $X = [f_A]_B^B$, tj. $X$ je unitárně podobná $A$, speciálně normální. $X$ má v prvním sloupci i řádku první prvek $\lambda$, jinak 0 (jelikož $¦v_1·A¦v_i = A^*¦v_1·¦v_i = \lambda(¦v_1·¦v_i) = \lambda·0 = 0$). Tedy na její minor použijeme IP, tj. $A$ je unitárně diagonalizovatelná.
        \end{dukazin}
    \end{veta}

    \subsection{Přehled spektrálních vět}
        \begin{poznamka}
            V10.13: $A$ normální $\Leftrightarrow$ $A$ unitárně diagonalizovatelná.

            V10.15: $A$ Hermitovská $\Leftrightarrow$ $A$ unitárně diagonalizovatelná a vlastní čísla jsou reálná.

            Důsl10.16: $A$ symetrická $\Leftrightarrow$ $A$ ortogonálně diagonalizovatelná.

            V10.20: $A$ pozitivně definitní $\Leftrightarrow$ $A$ unitárně diagonalizovatelná a vlastní čísla jsou kladná reálná.

            V10.20: $A$ pozitivně semidefinitní $\Leftrightarrow$ $A$ unitárně diagonalizovatelná a vlastní čísla jsou nezáporná reálná.

            V10.23: $A$ unitární $\Leftrightarrow$ $A$ unitárně diagonalizovatelná a $\forall \lambda$ vlastní číslo platí $|\lambda|=1$.
        \end{poznamka}

% 5. 5. 2021
        
        \begin{pozorovani}
            Buď $A \in ®C^{m \times n}$ libovolná matice, pak $B = A^*A$ je vždy pozitivně semidefinitní.

            \begin{dukazin}
                $$ \forall ¦v \in ®C^n: ¦v^*B¦v = ||A¦v||^2 ≥ 0. $$ 
            \end{dukazin}
        \end{pozorovani}

        \begin{tvrzeni}
            Tvrdí to samé, co předchozí pozorování plus: Každá pozitivně semidefinitní matice $B \in ®C^{n \times n}$ je tvaru $B=A^*A$ pro nějakou matici $A = ®C^{n \times n}$. Je-li $B$ pozitivně definitní, je nutně $A$ regulární.

            \begin{dukazin}
                Víme ze spektrálních vět, že $B = UOU^*$, kde $U$ je unitární, $D = \diag(\lambda_1, …, \lambda_n)$. Označíme $\sqrt{D} := \diag(\sqrt{\lambda_1}, …, \sqrt{\lambda_n})$. Pak $B = (U\sqrt{D}^*)(\sqrt{D}U^*) = A^*A$.

                Je-li $B$ pozitivně definitní, pak $\lambda_i > 0$. Tj. $D$, $\sqrt{D}$ jsou regulární, tím pádem je $A = \sqrt{D}U^*$ regulární.
            \end{dukazin}
        \end{tvrzeni}

        \begin{tvrzeni}
            Ortogonální operátor $f: ®R \rightarrow ®R$ je vždy buď reflexe ($\det[f]_B^B = -1$ pro libovolnou bázi) nebo otočení ($\det[f]_B^B = 1$ pro libovolnou bázi).

            \begin{dukazin}
                Je-li $f_A$ ortogonální, je $A$ ortogonální matice (speciálně je $A$ unitární nad ®C), tedy $A = U·\diag(\lambda_1, \lambda_2)·U^*$, $U$ unitární, $|\lambda_1| = |\lambda_2| = 1$. Tedy buď $\lambda_1, \lambda_2 \in ®R$, potom $\lambda_1, \lambda_2$ mají různá znaménka, pak je to reflexe, když $\lambda_1 = \lambda_2$, pak je to otočení o $0$ nebo $\pi$.

                Nebo $\lambda_1 = \cos \phi + i\sin \phi$ a $\lambda_2 = \cos\phi - i\sin\phi$ a vlastní vektory $¦v$ a $¦v_2 = \overline{¦v}$. Vezmeme bázi $B = (\frac{1}{\sqrt{2}}(¦v + \overline{¦v}), \frac{i}{\sqrt{2}} (¦v - \overline{¦v}))$. Tato báze je ortonormální a $[f_A]_B^B$ je matice otočení.
            \end{dukazin}
        \end{tvrzeni}

        \begin{poznamka}
            V $®R^3$ máme minimálně 1 reálný kořen, zbytek je jako v předchozím, tedy v $®R^3$ jsou to zase rotace a reflexe, tentokrát však i rotace s reflexí.
        \end{poznamka}

    \subsection{Singulární rozklad matice nad ®R nebo ®C (SVD)}
        \begin{veta}
            Komplexní verze: Buď $A$ komplexní matice typu $m \times n$ a hodnosti $r$. Pak existují ortonormální báze $B = (¦v_1, …, ¦v_n)$ prostoru $®C^n$ a $C = (¦u_1, …, ¦u_m)$ prostoru $®C^m$ a $\sigma_1, ., \sigma_r$ kladná reálná čísla taková, že
            $$ [f_A]_C^B = \(\diag(\sigma_1, …, \sigma_r, 0, …, 0)|0\). $$

            Reálná verze: Buď $A$ reálná matice typu $m \times n$ a hodnosti $r$. Pak existují ortonormální báze $B = (¦v_1, …, ¦v_n)$ prostoru $®R^n$ a $C = (¦u_1, …, ¦u_m)$ prostoru $®R^m$ a $\sigma_1, ., \sigma_r$ kladná reálná čísla taková, že
            $$ [f_A]_C^B = \(\diag(\sigma_1, …, \sigma_r, 0, …, 0)|0\). $$

            Maticové verze: Viz skripta.
        \end{veta}

% 7. 5. 2021

        \begin{pozorovani}
            Buď $f: ®C^n \rightarrow ®C^n$ lineární operátor a $B$ ortonormální báze $®C^n$ a $C$ ortonormální báze $®C^m$. Uvažujme $f_{A^*}: ®C^m \rightarrow ®C^n$. Pak $[f_{A^*}]_B^C = \([f_A]_C^B\)^*$.

            \begin{dukazin}
                $$ A = [f_A]_K^K = [\id]_K^C [f_A]_C^B [\id]_B^K, $$ 
                $$ A^* = [f_{A^*}]_K^K = [\id]_K^B [f_{A^*}]_B^C [\id]_C^K. $$
                Jelikož jsou báze ortogonální, tak matice přechodu jsou k sobě hermitovsky sdružené: $[f_A]_C^B = U^*AV$, $[f_{A^*}]=V^*A^*U = (U^*AV)^* = [f_A]_C^B$.
            \end{dukazin}
        \end{pozorovani}

        \begin{definice}
            Buď $A \in 2C^{m \times n}$. Pak singulárními hodnotami matice $A$ rozumíme druhé odmocniny vlastních čísel matice $A^*A$.
        \end{definice}
            
        \begin{dukaz}[Věty výše]
            Označme $\diag_{m \times n}(\sigma_1, …, \sigma_r) = \(\diag(\sigma_1, …, \sigma_r, 0, …, 0)|0\)$.Uvažujme vlastní čísla $\lambda_1 ≥ \lambda_2 ≥ … ≥ \lambda_r > 0 = \lambda_{r+1} = \lambda_n$ matice $A^*A$ řádu $n$. Za bázi $B = (¦v_1, …, ¦v_n)$ zvolíme ortonormální bázi vlastních vektorů $A^*A$, kde $¦v_i$ je příslušný $\lambda_i$. Tedy $[f_{A^*A}]_B^B = \diag(\lambda_1, …, \lambda_r, 0, …, 0)$.

            Pro $i \in [r]$ položíme $\sigma_i = \sqrt{\lambda_i}$. Aby mohl platit závěr věty, musí být $\forall i \in [r]: A¦v_i = \sigma_i¦u_i$. Čili pro $i \in [r]$ položíme $¦u_i = \frac{1}{\sigma_i}A¦v_i$. Ověříme, že $(¦u_1, …, ¦u_r)$ je ortonormální:
            $$ ¦u_i·¦u_j = (\frac{1}{\sigma_i}A¦v_i)·(\frac{1}{\sigma_j}A¦v_j) = \frac{1}{\sigma_i\sigma_j}(A¦v_i·A¦v_j) = \frac{1}{\sigma_i\sigma_j}((A^*A)¦v_i·¦v_j) = \frac{1}{\sigma_i\sigma_j} (\lambda_i¦v_i·¦v_j). $$
            Jelikož $B$ je ortogonální, tak výsledkem tohoto bude nula, pokud $i≠j$, a $\frac{\lambda_i}{\sigma_i^2} = 1$, pokud $i = j$. Následně doplníme $(¦u_1, …, ¦u_r)$ na ortonormální bázi $C = (¦u_1, …, ¦u_m)$ prostoru $®C^m$. Tedy pro $i ≤ r: f_A(¦v_i) = \sigma_i¦u_i$. Pro $i > r: [f_A(¦v_i)]_C = 0$. Tedy $[f_A]_C^B = \diag_{m \times n}(\sigma_1, …, \sigma_r)$.
        \end{dukaz}

        \begin{pozorovani}
            Je-li $A$ normální, pak singulární hodnoty jsou $\sigma_i = |\mu_i|$, kde $\mu_1, …, \mu_r$ jsou nenulová vlastní čísla $A$. Je-li navíc $A$ pozitivně definitní, pak jsou toto všechna vlastní čísla.
        \end{pozorovani}

    \subsection{Aplikace SVD}
        \begin{definice}[Spektrální norma matice]
            Mějme $A \in ®C^{m \times n}$, $f_A: ®C^n \rightarrow ®C^m$. Potom
            $$ ||A|| := \max\{\frac{||A¦x||}{||¦x||} | ¦o ≠ ¦x \in ®C^n \setminus \{¦o\}\} = \max\{||A¦y|| : ¦y \in ®C^n, ||¦y|| = 1\} $$
            se nazývá spektrální norma matice.
        \end{definice}

        \begin{tvrzeni}
            Buď $A \in ®C^{m \times n}$ (nebo $®R^{m \times n}$). Pak $\forall ¦o ≠ ¦x \in ®C^n: ||A¦x|| ≤ \sigma_1||¦x||$, kde $\sigma_1$ je největší singulární hodnota $A$, rovnost nastane přesně pro vlastní vektory ¦x matice $A^*A$ příslušné $\sigma_1^2$. Speciálně $||A|| = \sigma_1$.

            \begin{dukazin}
                Buď $¦x ≠ ¦o$, položme $¦y = \frac{¦x}{||¦x||}$, $[f_A]_C^B = \diag(\sigma_1, …, \sigma_r)$, $\sigma_1 ≥ \sigma_2 ≥ … ≥ \sigma_r > 0$, $B, C$ ortonormální. Pak
                $$ ||A¦y|| = ||[A]C^B·[y]_B|| = ||\sigma_1y_1 + … + \sigma_ry_r|| ≤ ||\sigma_1y_1 + … + \sigma_1y_r|| = \sigma_1||¦y||. $$ 
            \end{dukazin}
        \end{tvrzeni}

% 19. 5. 2021

\section{Bilineární a kvadratické formy}
%    \begin{definice}[Forma]
%        Zobrazení $®T^n \rightarrow ®T$ dané polynomem, jehož všechny členy mají stejný stupeň, se nazývá forma.
%    \end{definice}

    \begin{definice}[Bilineární forma]
        Buď ¦V vektorový prostor nad tělesem ®T. Bilineární formou na ¦V rozumíme zobrazení $f: ¦V \times ¦V \rightarrow ®T$, které splňuje:
        $$ (BL1) \forall ¦v, ¦w \in ¦V\ \forall t \in ®T: f(t¦v, ¦w) = t·f(¦v, ¦w) = f(¦v, t¦w), $$
        $$ (BL2) \forall ¦u, ¦v, ¦w \in ¦W: f(¦u + ¦v, ¦w) = f(¦u, ¦w) + f(¦v, ¦w) \land f(¦w, ¦u + ¦v) = f(¦w, ¦u) + f(¦w, ¦v). $$ 
    \end{definice}

    \begin{definice}[Kvadratická forma]
        Buď ¦V vektorový prostor nad tělesem ®T a $f: ¦V \times ¦V \rightarrow ®T$ bilineární forma. Pak kvadratickou formou na ¦V vytvořenou bilineární formou $f$ (též příslušnou formě $f$) rozumíme zobrazení $f_2: ¦V \rightarrow ®T$, $¦v \mapsto f(¦v, ¦v)$
    \end{definice}

    \subsection{Matice bilineární formy}
        \begin{definice}
            Buď ¦V konečně generovaný vektorový prostor nad ®T s bází $B = (¦v_1, …, ¦v_n)$. Buď $f: ¦V \times ¦V \rightarrow ®T$ bilineární forma. Buď $¦u = x_1¦v_1 + … + x_n¦v_n$, $¦w = y_1¦v_1 + … +y_n¦v_n$. Potom $f(¦u, ¦w)$ lze vyjádřit maticí ($\sum_{i=1}^na_{ij}x_iy_j$). Této matici říkáme matice bilineární formy $f$ vzhledem k bázi $B$. ($[f]_B = (f(¦v_i, ¦v_j))_{i, j}$).
        \end{definice}
\end{document}
