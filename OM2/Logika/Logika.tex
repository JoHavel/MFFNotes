\documentclass[12pt]{article}                   % Začátek dokumentu
\usepackage{../../MFFStyle}                     % Import stylu

\begin{document}

% 4. 3. 2021
\section{Úvod}
    \begin{poznamka}[Domluva]
        ®N jsou přirozená čísla s 0. $n$ značí přirozené číslo.
    \end{poznamka}

    Dále se probírali základy značení a teorie množin.

% 11. 2. 2021

    \begin{definice}[Základy]
            Základem výrokové logiky je 5 symbolů (2 hodnoty + 3 logické spojky): $\top \bot \neg \land \lor$ = pravda, lež, negace, a, nebo.

        Dále jsou to výrokové atomy z nějaké abecedy. Libovolný výrok je pak konečným aplikováním logických spojek.
    \end{definice}

    \begin{definice}[Pravdivostní ohodnocení]
        Pravděpodobnostní ohodnocení je zobrazení $t$ z prvovýroků do $\{0, 1\}$. Toto zobrazení lze jednoznačně rozšířit na $t'$ na všechny výroky:
        $$ t'(\top) = 1, t'(\bot) = 0, t'(\neg a) = 1 - t'(a), t'(a \lor b) = \max\{t'(a), t'(b)\}, t'(a \land b) = \min\{t'(a), t'(b)\} $$ 
    \end{definice}

    \begin{definice}
        Pomocí pravdivostního ohodnocení můžeme zavést implikaci (spojka mezi premisou (antecedent) a závěrem (konsekvent)).
    \end{definice}

    \begin{definice}[Tautologie]
        $p$ je tautologie (notace $\models p$) $≡ t(p) = 1$ pro všechna $t: A \rightarrow \{0, 1\}$. $p$ je splnitelné $≡$ existuje $t: A \rightarrow \{0, 1\}$ takové, že $t(p) = 1$.
    \end{definice}

    \begin{lemma}[Zákony inempotence, komutativity, asociativity, distributivity, absorbce, DeMorganovy]
        Viz skripta.
    \end{lemma}

    \begin{definice}[Model]
        Model (koho, čeho) $\Sigma$ (výrokové teorie) je každé pravděpodobnostní ohodnocení $t$, které přiřazuje 1 všem výrokům ze $\Sigma$. Říkáme, že $p$ je tautologický důsledek $\Sigma$ (píšeme $\Sigma \models p$, říkáme $p$ vyplývá ze $\Sigma$) $≡ t(p) = 1$ pro všechny modely $t$ (koho čeho) $\Sigma$.
    \end{definice}

    \begin{poznamka}
        $\models p$ je totéž, co $\O \models p$.
    \end{poznamka}

    \begin{lemma}
        Vlastnosti $\models$. Viz skripta.
    \end{lemma}

    \begin{definice}[Arita]
        Mějme množinu symbolů $F$ a zobrazení $a: F \rightarrow ®N$. Říkáme, že symbol $f \in F$ má aritu $n$ $≡ a(f) = n$.

        Řekněme, že slovo je přijatelné $≡$ TODO.
    \end{definice}

    \begin{definice}[Arita logických symbolů]
        Aritu symbolů $ar$ definujeme pro $F = A \cup \{\top, \bot, \neq, \lor, \land\}$ jako $ar(x) = 0, x\in A \cup \{\top, \bot\}$, $ar(\neq) = 1$, $ar(\lor, \land) = 2$.
    \end{definice}

    \begin{lemma}
        Buďte $t_1, …, t_m$ a $u_1, …, u_n$ jsou přijatelná slova a $w$ libovolné slovo tak, že $t_1…t_mw=u_1–u_n$. Potom $m ≤ n$, $t_i = u_i$ pro $i \in [m]$ a $w = u_{m+1}…u_n$.

        \begin{dukazin}
            Indukcí podle velikosti $u_1…u_n$.
        \end{dukazin}
    \end{lemma}

% 18. 3. 2021

    \begin{definice}[Modus Ponens (= MP = odvozovací pravidla)]
        Z $p$ a $p \implies q$, odvodíme $q$.
    \end{definice}

    \begin{definice}[Důkaz]
        Formální důkaz (či důkaz) $p$ z $\Sigma$ je sekvence $p_1, …, p_n$, kde $n≥1$ a $p_n = p$ tak, že $\forall k \in [n]$: buď $p_k \in Sigma$, nebo $p_k$ je výrokový axiom (viz skripta), nebo $\exists i, j \in [k-1]$ tak, že $p_k$ lze odvodit pravidlem MP z $p_i$ a $p_j$.

        Říkáme, že $p$ je dokazatelné ze $\Sigma$, a značíme $\Sigma \vdash p$
    \end{definice}

    \begin{tvrzeni}
        Pokud $\Sigma \vdash p$, pak $\Sigma \models p$.

        \begin{dukazin}
            Jednoduchý.
        \end{dukazin}
    \end{tvrzeni}

    \begin{veta}[O úplnosti (1. znění)]
        $$ \Sigma \vdash p \Leftrightarrow \Sigma \models p. $$ 
    \end{veta}

    \begin{veta}[Kompaktnost logiky]
        Pokud $\Sigma \models p$, pak existuje konečná podmnožina $\Sigma_0 \subseteq \Sigma$ tak, že $\Sigma_0 \models p$.

        \begin{dukazin}
            Vyplývá z předchozí věty
        \end{dukazin}
    \end{veta}

    \begin{definice}[Konzistentnost]
        Říkáme, že $\Sigma$ je nekonzistentní, pokud $\Sigma \vdash \bot$, jinak (pokud $\Sigma \not\vdash \bot$) je konzistentní.
    \end{definice}

    \begin{veta}[O úplnosti (2. znění)]
        $\Sigma$ je konzistentní právě tehdy, když má model.
    \end{veta}

    \begin{dusledek}
        $\Sigma$ má model $\Leftrightarrow$ každá konečná podmnožina $\Sigma$ má model.
    \end{dusledek}

    \begin{lemma}[Dedukce]
        Předpokládejme $\Sigma \cup \{p\} \vdash q$. Potom $\Sigma \vdash p \implies q$.

        \begin{dukazin}[Indukcí]
            Pokud je $q$ výrokový axiom, pak $\Sigma \vdash q$ a jelikož $q \implies (p \implies q)$ je výrokový axiom, MP říká $\Sigma \vdash p \implies q$. Pokud $q \in \Sigma \cup \{p\}$, pak buď TODO
        \end{dukazin}
    \end{lemma}

    \begin{dusledek}
        $\Sigma \vdash p$ tehdy a pouze tehdy, když $\Sigma \cup \{\neg\}$ je nekonzistentní.

        \begin{dukazin}
            $\implies$: Předpokládejme, že $\Sigma \vdash p$. Jelikož $p \implies (\neg p \implies \bot)$ je výrokový axiom, můžeme 2krát použít MP a získat $\Sigma \cup \{p\}$ TODO
        \end{dukazin}
    \end{dusledek}

    \begin{dusledek}
        Z druhého znění věty o úplnosti vyplývá první znění.
    \end{dusledek}

    \begin{definice}
        Říkáme, že $\Sigma$ je kompletní (úplná, ale s větou o úplnosti nemá nic společného), pokud $\Sigma$ je konzistentní a pro všechna $p$ je buď $\Sigma \vdash p$ nebo $\Sigma \vdash \neg p$.
    \end{definice}

    \begin{lemma}[Lindenbaum]
        Nechť $\Sigma$ je konzistentní. Pak existuje kompletní $\Sigma'$ tak, že $\Sigma \subseteq \Sigma'$.

        \begin{dukazin}
            Zornovo lemma. TODO. Pokud je axiomů konečně, tak můžeme udělat důkaz bez Zornova lemmatu.
        \end{dukazin}
    \end{lemma}

    \begin{definice}[Pravdivostní ohodnocení v závislosti na $\Sigma$]
        $t_\Sigma: A \rightarrow \{0, 1\}$, $t_\Sigma(a) = 1$, pokud $\Sigma \vdash a$, jinak $t_\Sigma(a)= 0$.
    \end{definice}

    \begin{lemma}
        Předpokládejme, že $\Sigma$ je kompletní, potom pro každé $p$ máme
        $$ \Sigma \vdash p \Leftrightarrow t_\Sigma(p) = 1. $$
        Nevoli $t_\Sigma$ je model $\Sigma$.

        \begin{dukazin}
            Indukcí podle počtu spojek. TODO.
        \end{dukazin}
    \end{lemma}

% 25. 3. 2021

\section{Predikátorová logika}
    \begin{definice}[Jazyk]
        Jazyk ($L$) je disjoin sjednocení množiny relací ($L^r$) (každé relaci $R \in L^r$ přiřadíme aritu $a(R) \in ®N$) a množiny funkčních symbolů ($L^f$) ($F \in L^f$ má aritu $a(F) \in ®N$).
    \end{definice}

    \begin{definice}[Struktura]
        Struktura ©A pro $L$ je trojice $(A, (R^{©A})_{R \in L^r}, (F^{©A}_{F \in L^f}))$ sestávající z množiny $A$ (tzv. nosič), pro každou $m$-ární relaci $R \in L^r$ máme její vyjádření $R_A \in A^m$.
    \end{definice}

    \begin{definice}[Podstruktura, zúžení]
        $©X$ je podstruktura struktury $©Y$, značíme $©X \subseteq ©Y$, pokud $X \subseteq Y$ a všechny operace jsou uzavřené na relace i funkce. Taktéž říkáme, že $©Y$ je rozšíření $©A$.

        Zúžení funkce $F$ na podstrukturu $©X$, značené $F|_{©X}$ je, jak bychom čekali.
    \end{definice}

    \begin{definice}[Homomorfismus]
        Ať ©A a a ©B jsou struktury (pro tentýž jazyk). Homomorfismus $h: ©A \rightarrow ©B$ je zobrazení $h: A \rightarrow B$ tak, že $\forall m$-nární $R \in L^r$ a každé $(a_1, …, a_m)\in A^m$ máme $(a_1, …, a_m)\in R^{©A} \implies (ha_1, …, ha_m) \in R^{©B}$. $\forall n$-nární $F \in L^f$ a každé $(a_1, …, a_n) \in A^n$ je $h(F^{©A}(a_1, …, a_n)) = F^{©B}(ha_1, …, ha_n)$.
    \end{definice}

    \begin{definice}[Silný homomorfismus]
        Pokud nahradíme implikaci v předchozí definici ekvivalencí, dostaneme tzv. silný homomorfismus. Speciálním případem je tzv. vnoření, TODO.
    \end{definice}

    \begin{definice}[Kongruence]
        Kongruence je ekvivalence taková, že pokud jsou v relaci nějaké prvky, tak jsou v relaci i kongruentní prvky. Stejně tak obraz kongruentních prvků je kongruentní prvek k obrazu původních.
    \end{definice}

    \begin{definice}[Kvocient / faktorstruktura]
        Nechť ©A je struktura a $\sim$ kongruence. Potom $©A/\sim$, tzv. faktostruktura, je struktura, kde nosná množina je $A/\sim$ a relace a funkce jsou přepsané tak, aby nové prvky byly v relaci právě tehdy, pokud byly jim odpovídající původní prvky.
    \end{definice}

    \subsection{Proměnné a formule}
        \begin{definice}[Proměnné]
            Proměnné: $Var = \{v_0, v_1, v_2, …\}$ je spočetná (nekonečná) množina.
        \end{definice}

        \begin{poznamka}
            Většinou by nevadila ani nespočetná. Naopak se spočetná by nám rozbíjela skládání výroků.
        \end{poznamka}

        \begin{definice}[Termy]
            $L$-term je slovo na abecedě $L^f \cup Var$ získané jako: každá proměnná je $L$-term a kdykoliv je $F \in L^f$ $n$-nární relace a $t_1, …, t_n$ $L$-termy, pak je $Ft_1…t_n$ $L$-term.
        \end{definice}

        \begin{definice}[Uzavřený term]
            Uzavřený term se nazývá ten term, který neobsahuje proměnné.
        \end{definice}

        \begin{definice}[Generátory]
            Mějme strukturu a množinu (oindexovanou) prvků z ní. Pokud tuto množinu uzavřeme na relace a funkce, pak dostaneme podstrukturu, která se nazývá generovaná danou množinou prvků (a ty se nazývají generátory).
        \end{definice}

% 1. 4. 2021

        \begin{definice}[Symboly]
            V predikátorové logice máme: $\top, \bot, \neg, \land, \lor, =, \forall, \exists$.
        \end{definice}

        \begin{definice}[Atomická formule]
            Atomická $L$-formule je slovo z abecedy $L \cup Var \cup \{\top, \bot, =\}$, které je tvaru buď $\top, \bot$, nebo termy jsou v relaci ($Rt_1…t_m$, kde $R \in L^r$ je $m$-nární relace a $t_1, …, t_m$ jsou $L$-termy), nebo $=t_1t_2$ (kde $t_1$ a $t_2$ jsou $L$-termy).
        \end{definice}

        \begin{definice}[Formule]
            $L$-formule je slovo na abecedě $L \cup Var \cup \{\top, \bot, \neg, \lor, \land, =, \exists, \forall\}$, které je buď atomická formule, nebo $\neg\phi, \lor\phi\psi, \land\phi\psi$, kde $\phi$ a $\psi$ jsou $L$-formule, nebo $\exists x\phi, \forall x\phi$, kde $\phi$ je formule a $x$ je proměnná.
        \end{definice}

        \begin{definice}[Podformule]
            Podformule je podslovo formule, které je také formule.
        \end{definice}

        \begin{definice}[Vázaný a volný výskyt]
            Pokud se proměnná vyskytuje v podformuli tvaru $\exists x \phi$ nebo $\forall x \phi$, pak se nazývá vázaná (má na tomto místě vázaný výskyt), pokud se vyskytuje jinde, pak je volná (volný výskyt).
        \end{definice}

        \begin{definice}[Sentence (= uzavřená formule)]
            Sentence je formule, kde všechny výskyty proměnné jsou vázané.
        \end{definice}

        \begin{poznamka}
            Píšeme $\phi(x_1, …, x_n)$, abychom zvýraznili, že právě proměnné $x_1, …, x_n$ jsou volné v $\phi$.

            Do formule dosazujeme $(\phi(t_1/x_1, …, t_n/x_n))$ naráz a nahrazujeme všechny volné výskyty dané proměnné.

            Místo $\phi(t_1/x_1, …, t_n/x_n)$ budeme psát $\phi(t_1, …, t_n)$.
        \end{poznamka}

        \begin{lemma}
            Nechť $\phi$ je $L$-formule, $x_1, …, x_n$ různé proměnné a $t_1, …, t_n$ jsou $L$-termy. Potom $\phi(t_1/x_1, …, t_n/x_n)$ je $L$-formule. Pokud $t_1, …, t_n$ nemají volné proměnné a $\phi = \phi(x_1, …, x_n)$, potom $\phi(t_1, …, t_n)$ je $L$-sentence.
        \end{lemma}

        \begin{definice}
            Jazyk rozšiřujeme o tzv. jména, tj. konstantní symboly reprezentující prvky, o kterých se chceme bavit. Tzv. expanze struktury.
        \end{definice}

        \begin{definice}[Pravdivost (Tarského definice splňování)]
            $L_A$-sentence $\sigma$ je pravdivá v $L$-struktuře $A$ (píšeme $A \models \sigma$ a čteme $\sigma$ je pravdivá / splněna v $A$) takto:
            
            \begin{itemize}
                \item $A \models \top$ a $A \not\models \bot$,
                \item $A \models Rt_1…t_m$ právě tehdy, pokud $(t^A_1, …, t^A_m) \in R^A$ pro $m$-nární relaci $R \in L^r$ a $L_A$ termy bez volných proměnných $t_1, …, t_m$,
                \item $A \models t_1 = t_2$ právě tehdy, když $t_1^A = t_2^A$ pro $L_A$-termy bez volných proměnných $t_1, t_2$,
                \item $\sigma = \neg \sigma_1$, potom $A \models \sigma$ právě tehdy, pokud $A \not\models \sigma_1$,
                \item $\sigma = \sigma_1 \lor \sigma_2$, potom $A \models \sigma$ právě tehdy, pokud $A \models \sigma_1$ nebo $A \models \sigma_2$,
                \item $\sigma = \sigma_1 \land \sigma_2$, potom $A \models \sigma$ právě tehdy, pokud $A \models \sigma_1$ a $A \models \sigma_2$,
                \item $\sigma = \exists x \phi(x)$, potom $A \models \sigma$ tehdy a jen tehdy, když $A \models \phi(\underline{a})$ pro nějaké $a \in A$,
                \item $\sigma = \forall x \phi(x)$, potom $A \models \sigma$ tehdy a jen tehdy, když $A \models \phi(\underline{a})$ pro všechna $a \in A$,
            \end{itemize}
        \end{definice}

        \begin{definice}
            $\phi(x_1, …, x_n)$ definuje množinu $\phi^A = \{(a_1, …, a_n) : A \models \phi(a_1, …, a_n)\}$.

            Pokud existuje $L$-formule definující $S \subseteq A^n$, potom říkáme, že formule je 0-definovatelná v $A$.
        \end{definice}

        \begin{definice}
            Formule se nazývá pozitivní, pokud neobsahuje negaci ($\neg$).
        \end{definice}

% 8. 4. 2021

        \begin{definice}
            Buďte $©A, ©B$ dvě $L$-struktury, $C \subseteq A$ a $h: C \rightarrow B$ zobrazení. Řekneme, že $h$ je $(©A, ©B)$-podobnost, pokud pro každou $L_C$-sentenci $\sigma$ platí $A \models \sigma \Leftrightarrow B \models \sigma_h$. Existuje-li nějaká $(©A, ©B)$-podobnost (kde $C ≠ \O$), říkáme, že ©A je elementárně ekvivalentní s ©B, píšeme $©A ≡ ©B$. Je-li dokonce $C = A$, říkáme, že $h$ je elementární vnoření ©A do ©B.
        \end{definice}

        \begin{tvrzeni}
            Buďte ©A, ©B dvě $L$-struktury a $h: A \rightarrow B$ izomorfismus. Pak $h$ je elementární vnoření.

            \begin{dukazin}
                Již víme, že $h(t^{©A}) = t_h^{©B}$ pro každý uzavřený $L_A$-term $t$. Tvrzení dokážeme indukcí…
            \end{dukazin}
        \end{tvrzeni}

        \begin{definice}
            Říkáme, že ©A je model $\Sigma$, když $©A \mapsto \sigma$ pro všechny $\sigma \in \Sigma$.
        \end{definice}

        \begin{definice}
            Říkáme, že $\sigma$ vyplývá z $\Sigma$ (píšeme $\Sigma \models \sigma$), pokud $\sigma$ je pravdivý v každém modelu (koho, čeho) $\Sigma$.
        \end{definice}

        \begin{definice}
            Formule $\sigma$ je validní v ©A, $©A \models \phi$, jestliže všechny ©A-instance jsou pravdivé v ©S.
        \end{definice}

        
        \begin{definice}[Axiomy predikátorové logiky]
            TODO
        \end{definice}

        \begin{definice}[Axiomy rovnítka]
            TODO
        \end{definice}

        \begin{definice}
            $t$ je substituovatelný za $y$ v $\phi$, jestliže žádná proměnná (žádný její výskyt) v $t$ se nestane vázanou.
        \end{definice}

% 15. 4. 2021

        \begin{definice}
            Qvantifikátorové axiomy v $L$ jsou formule $\phi(t/y) \implies \exists y\phi$ a $\forall y \phi \implies \phi(t/y)$.
        \end{definice}

        \begin{veta}[O korektnosti predikátorového počtu]
            Každý logický axiom v $L$ je validní v každé $L$-struktuře.

            \begin{lemmain}
                Mějme nějaké atomy $\alpha_1, …, \alpha_n$, které nejsou v $L$, a buďte $\phi_i = \phi(x_1, …, x_m)$, $\forall 1 ≤ i ≤ n$. Definujeme pravdivostní ohodnocení $t:\{\alpha_1, …, \alpha\} \rightarrow \{0, 1\}$ tak, že $t(\alpha_i) = 1$ pokud $©A \models_i(a_1, …, a_m)$. Potom $p(\phi_1, …, \phi_n)$ je $L$-formule a TODO.
            \end{lemmain}

            \begin{definicein}[$L$-tautologie]
                $L$-tautologie je formule tvaru $p(\phi_1, …, \phi_n)$ pro nějakou tautologii $p(\alpha_1, …, \alpha_n) \in Prop\{\alpha_1, …, \alpha_n\}$ a formule $\phi_1, …, \phi_n$.
            \end{definicein}
        \end{veta}

        \begin{definice}[Logická pravidla]
            Modus Ponens (MP): z $\phi$ a $\phi \implies \psi$ odvodíme $\psi$.

            Generalizační pravidla (G): pokud se proměnná $x$ nevyskytuje volně v $\phi$, potom z $\phi \implies \psi$ odvodíme $\phi \implies x\psi$ a z $\psi \implies \phi$ odvodíme $\exists x \psi \implies \phi$.
        \end{definice}

        \begin{definice}[Důkaz]
            Formální důkaz, nebo prostě důkaz $\phi$ z $\Sigma$ je posloupnost $\phi_1, …, \phi_n$ formulí, kde $n ≥ 1$ a $\phi_n = \phi$, takových, že $\forall k \in [n]$: je buď $\phi_k \in \Sigma$ nebo $\phi_k$ je logický axiom, nebo $\phi_k$ může být odvozen z $\phi_i$ a $\phi_j$ ($\phi_j$) pomocí MP (G), pro nějaké $i, j$. Značíme $\Sigma \vdash \phi$.
        \end{definice}

        \begin{veta}[Kompletnost predikátorové logiky]
            $$ \Sigma \vdash \phi \Leftrightarrow \Sigma \models \phi. $$ 
        \end{veta}

        \begin{veta}[Kompaktnost]
            Pokud $\Sigma \models \sigma$, pak existuje konečná podmnožina $\Sigma_0$ (koho, čeho) $\Sigma$, že $\Sigma_0 \models \sigma$.
        \end{veta}

        \begin{definice}
            $\Sigma$ je konzistentní, pokud $\Sigma \nvdash \bot$, jinak (pokud $\Sigma \vdash \bot$) ji nazýváme nekonzistentní.
        \end{definice}

        \begin{lemma}
            Ať $\Sigma \vdash \phi$. Potom $\Sigma \vdash \forall x \phi$..

            \begin{dukazin}[Náznak]
                Použijeme MP a G na konkrétní $L$-tautologie.
            \end{dukazin}
        \end{lemma}

        \begin{lemma}[Dedukce]
            Nechť $\Sigma \cup \{\sigma\} \vdash \phi$, potom $\Sigma \vdash \sigma \implies \phi$.

            \begin{dukazin}
                Indukcí.
            \end{dukazin}
        \end{lemma}

        \begin{veta}
            Pokud každá konečná podmnožina (koho, čeho) $\Sigma$ má model, pak i $\Sigma$ má model.
        \end{veta}

% 22. 4. 2021

        \begin{lemma}
            Předpokládejme $\Sigma \vdash \phi$ a $t$ je substituovatelné za $x$ v $\phi$. Potom $\Sigma \vdash \phi(t/x)$.

            \begin{dukazin}
                MP $\Sigma \vdash \forall x \phi$ s $\forall x \phi \implies \phi(t/x)$.
            \end{dukazin}
        \end{lemma}

        \begin{lemma}[Důsledky axiomů rovnosti]
            TODO.
        \end{lemma}

        \begin{definice}
            Ať $L_T$ je množina $L$-termů bez proměnných. Definujeme binární relaci $\sim_\Sigma$ na $T_L$:
            $$ t_1 \sim_\Sigma t_2 \Leftrightarrow \Sigma \vdash t_1 = t_2. $$ 
        \end{definice}

        \begin{lemma}
            $\sim_\Sigma$ je relace ekvivalence.
        \end{lemma}

        \begin{definice}[Kanonická struktura pro $\Sigma$]
            $A_\Sigma := T_L/\sim_\Sigma$. $R^{A_\Sigma}$ nebo $F^{A_\Sigma}$ je potom relace nebo funkce, přijímající bloky ekvivalence.
        \end{definice}

        \begin{definice}
            Kompletní teorie je konsistentní a pro každou $\sigma$ lze v této teorii dokázat $\sigma$ nebo $\neg\sigma$.
        \end{definice}

        \begin{lemma}[Lindenbaum]
            Ať $\Sigma$ je konzistentní. Potom $\Sigma \subseteq \Sigma'$ pro nějaké úplné $\Sigma'$.
        \end{lemma}

% 29. 4. 2021

        \begin{definice}
            $\Sigma$ (henkinovský) svědek sentence $\exists x \phi(x)$ je konstantní term $t \in T_L$ tak, že $\Sigma \vdash \phi(t)$. Říkáme, že $\Sigma$ je henkinovská teorie (má svědky), jestliže existuje svědek pro každou sentenci $\exists x \phi(x)$.
        \end{definice}

        \begin{veta}
            Nechť $L$ má konstantní symbol a předpokládejme $\Sigma$ je konzistentní. Potom následující podmínky jsou ekvivalentní: 1) $\forall \sigma$ máme $\Sigma \vdash \sigma \Leftrightarrow ©A_\Sigma \models \sigma$ a 2) $\Sigma$ je kompletní a má svědky.
        \end{veta}

        \begin{lemma}
            Ať $\Sigma$ je množina $L$-sentencí a $c$ je konstantní symbol, který není v jazyku $L$. Definujme $L_c = L \cup \{c\}$. Potom když $\phi(y)$ je $L$-formule a $\Sigma \vdash_{L_c} \phi(c)$, tak $\Sigma \vdash_L \phi(y)$.
        \end{lemma}

        \begin{lemma}
            Nechť $\Sigma$ je konzistentní a $\Sigma \vdash \exists y\phi(y)$. Ať $c$ je konstanta, která není v $L$. Položme $L_c := L \cup \{c\}$. Potom $\Sigma \cup \{\phi(c)\}$ je konzistentní množina $L_c$-sentencí. Obdobně pro více $c_i$.
        \end{lemma}

        \begin{lemma}[Rozšíření teorie o svědky]
            TODO!
        \end{lemma}

        \begin{lemma}
            Pokud máme jazyky $L_0 \subseteq L_1 \subseteq …$ a konzistentní teorie $\Sigma_0 \subseteq \Sigma_1 \subseteq …$. Potom sjednocení $\Sigma_{∞}$ je konzistentní množina $L_{∞}$-sentencí.
        \end{lemma}

% 6. 5. 2021
    
    \subsection{Prenexní tvar}

        TODO?

        \begin{definice}[Prenexní tvar]
                Formule je v prenexním tvaru, pokud je tvaru $Q_1x_1…Q_nx_n\phi$, kde $x_1, …, x_n$ jsou různé proměnné, $Q_i \in \{\exists, \forall\}$ a $\phi$ je bez kvantifikátorů.
        \end{definice}

% 13. 5. 2021

\section{Trochu teorie modelů}
    \begin{veta}[Löwenheim-Skolem (spočetná verze)]
        Nechť $L$ je spočetný jazyk a $\Sigma$ má model. Potom $\Sigma$ má spočetný model.

        \begin{dukazin}
            Jelikož množina proměnných je spočetná a $L$ je spočetný, tak i množina $L$-sentencí je spočetná. Tedy i 
            $$ L \cup \{c_\sigma | \Sigma \vdash \sigma \land \sigma = \exists x \phi(x) \} $$ 
            je spočetný, tedy přidání svědků nezvětší $L$ nad spočetnost. To znamená, že množina $L_{∞}$-termů je spočetná, tj. $©A_{\Sigma_{∞}}$ je spočetná a $©A_{\Sigma_{∞}}|_L$ je spočetný model $\Sigma$.
        \end{dukazin}
    \end{veta}

    \begin{tvrzeni}[Vaughtův test]
        Nechť $L$ je spočetné a $\Sigma$ má model a všechny spočetné modely $\Sigma$ jsou isomorfní. Pak $\Sigma$ je kompletní.

        \begin{dukazin}
            Kdyby nebyla kompletní, pak existuje $\sigma$ tak, že $\Sigma \nvdash \sigma$ a $\Sigma \nvdash \neg\sigma$. Potom z Löwenheim-Skolemovy věty existují 2 spočetné modely, ve kterých je $\sigma$ (v prvním) a $\neg\sigma$ (ve druhém), které z $\Sigma$ dostaneme tak, že přihodíme $\neg \sigma$ a $\sigma$. Tedy máme 2 izomorfní modely, v nichž v jednom je $\neg\sigma$ a v druhém $\sigma$. Spor.
        \end{dukazin}
    \end{tvrzeni}

    \begin{veta}[Löwenheim-Skolem (obecná verze)]
        Nechť $L$ je jazyk mohutnosti $\kappa$ a $\Sigma$ má nekonečný model. Potom $\Sigma$ má model mohutnosti $\kappa$.

        \begin{dukazin}
            Podobně jako předchozí, jen přidáme $c_\lambda ≠ c_\mu$, abychom měli právě mohutnost $\kappa$.
        \end{dukazin}
    \end{veta}


    \begin{tvrzeni}[Vaughtův test]
        Nechť $L$ je mohutnosti nejvýše $\kappa$, $\Sigma$ má model, všechny modely jsou nekonečné, a všechny modely $\Sigma$ mohutnosti $\kappa$ jsou isomorfní. Pak $\Sigma$ je kompletní.
    \end{tvrzeni}

% 20. 5. 2021

% 27. 5. 2021

    \begin{definice}
        $\Sigma'$ se nazývá konzervativní nad $\Sigma$, pokud pro každou $L$-sentenci $\sigma$ je
        $$ \Sigma' \vdash_{L'} \sigma \Leftrightarrow \Sigma \vdash_L \sigma. $$ 
    \end{definice}

    \begin{tvrzeni}
        Pokud $\phi(x_1, …, x_n, y)$ je $L$-formule. $f_{\phi}$ je $n$-ární funkční symbol, který není v $L$ a pro který položíme $L' := L\cup \{f_{\phi}\}$ a 
        $$ \Sigma' := \Sigma \cup \{\forall x_1…\forall x_n (\exists y \phi(x_1, …, x_n, y) \implies \phi(x_1, …, x_n, f_{\phi(x_1, …, x_n)}))\} $$ 
    \end{tvrzeni}

    TODO?

% 3. 6. 2021

    \begin{definice}[Presburgerova aritmetika]
        Uvažujme jazyk $K = \{0, S, +\}$, kde $0$ je konstantní symbol, $S$ je unární funkční a $+$ binární funkční symbol. Presburgerova aritmetika je $K$-teorie obsahující právě následující axiomy:
        
        \begin{enumerate}
            \item $Sx ≠ 0$;
            \item $Sx = Sy \implies x = y$;
            \item $x ≠ 0 \implies \exists y: x = Sy$;
            \item $x + 0 = x$;
            \item $x + Sy = S(x+y)$;
        \end{enumerate}
        
        a navíc schéma axiomů indukce (pro každou $K$-formuli $\phi$):
        $$ (\phi(0/x) \land \forall x (\phi(x) \implies \phi(Sx/x))) \implies \forall x \phi. $$ 
    \end{definice}

    \begin{veta}
        Presburgerova aritmetika je kompletní.
    \end{veta}

    \begin{definice}[Robinsonova a Peanova aritmetika (tj. včetně $·$)]
        Rozšíříme $K$ na $L = K \cup \{·\}$, kde $·$ je binární funkční symbol, a přidejme k předchozím axiomům navíc $x·0 = 0$ a $x·Sy = x·y + x$. Navíc schéma indukce nyní uvažujme pro všechny $L$-formule. Výsledné $L$-teorii se říká Peanova aritmetika (P nebo PA). Její (konečnou) podteorii, která vznikne vypuštěním všech axiomů indukce, nazýváme Robinsonova aritmetika (Q či RA).
    \end{definice}

    \begin{poznamka}
        V RA nelze dokázat ani asociativitu, ani komutativitu $+$ a $·$, ani vztah $\forall x: x ≠ Sx$.
    \end{poznamka}

    \begin{definice}
        Buďte $x, y$ dvě různé proměnné a $\phi$ nějaká $L$-formule. Formuli $\forall x(x≤y \implies \phi)$ zkráceně zapisujeme jako $\forall x ≤ y \phi$. Formuli $\exists x(x≤y \implies \phi)$ zkráceně zapisujeme jako $\exists x ≤ y \phi$. Formuli nazveme omezenou, pokud se v rámci její rekurzivní definice v kvantivikačním kroku používá místo $\forall v \phi$, resp. $\exists v \phi$, kde $v$ je proměnná, pouze omezená kvantifikace $\forall v ≤ z\phi$ resp. $\exists v ≤ z\phi$, kde $z$ je nějaká proměnná různá od $v$.

        Formuli nazveme $\Sigma_1$-formulí, je-li tvaru $\exists x \psi$, kde $\psi$ je omezená.
    \end{definice}

    \begin{veta}[$\Sigma_1$-úplnost RA]
            je-li $\phi$ uzavřená $\Sigma_1$-formule taková, že $®N \models \phi$, pak $RA \vdash \phi$.
    \end{veta}

    \subsection{Gödelovské kódování}
        Kóduje veškeré formule do přirozených čísel. (Např. v PA.)

        \begin{definice}
            Teorii $T$ nazýváme $\Sigma_1$-teorií, pokud existuje $\Sigma_1$-formule $\tau(x)$ taková, že $\sigma \in T$ právě tehdy, když $N \models \tau(\sigma/x)$.
        \end{definice}

        \begin{veta}[Autoreferenční lemma]
            Nechť $\phi(y)$ je $L$-formule. Pak existuje $L$-sentence $psi$ taková, že $RA \vdash \psi \Leftrightarrow \phi(\psi/y)$.
        \end{veta}

        TODO Věty o neúplnosti.

\end{document}
