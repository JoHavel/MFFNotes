\documentclass[12pt]{article}                   % Začátek dokumentu
\usepackage{../../MFFStyle}                     % Import stylu

\begin{document}

% 3. 3. 2021
\section{Úvod}
    \begin{poznamka}[Informační zdroje]
        Stránky, diskuze na google docs, Moodle.
    \end{poznamka}

    \begin{poznamka}[Proč algebra]
        Diofanctické rovnice (Fermatovy věty, Gaussova celá čísla), kořeny polynomů (Grupy polynomů), geometrie (nekonstruovatelnost), studium abstraktních struktur běžných objektů.
    \end{poznamka}

\section{Obory}
    \begin{definice}[Okruh]
            Okruh $R$ je pětice $(R, +, ·, -, 0)$, kde $+, ·: R \times R \rightarrow R$, $-: R \rightarrow R$, $0 \in R$ tak, že $\(\forall a, b, c \in R\)$:
        $$ a + (b + c) = (a + b) + c, $$ 
        $$ a + b = b + a, a + 0 = a, a + (-a) = 0, $$
        $$ a·(b·c) = (a·b)·c, a·(b+c) = a·b + a·c, (b+c)·a = b·a + c·b. $$ 
    \end{definice}

    \begin{definice}[Komutativní okruh]
        Komutativní okruh je okruh, pro který platí $a · b = b · a$.
    \end{definice}

    \begin{definice}[Okruh s jednotkou]
        Okruh s jednotkou je okruh, který má prvek $1 \in R: a·1 = a$.
    \end{definice}

    \begin{definice}[Obor (integrity)]
        Obor (integrity) je komutativní okruh s jednotkou tak, že $0 ≠ 1 \land (a≠0 \land b≠0 \implies a·b ≠ 0)$.
    \end{definice}

    \begin{definice}[Těleso]
        Těleso je komutativní okruh s 1, že $0 ≠ 1$ a $\forall 0≠a \in R\ \exists b \in R: a · b = 1$.
    \end{definice}

    \begin{definice}[Podokruh]
        Podokruh $S$ okruhu $R$ je $(S, +|_S, ·|_S, -|_S, 0)$, kde $0 \in S$ a $\forall a, b \in S: a+b \in S \land a·b \in S \land -a \in S$. Značíme $R ≤ S$.
    \end{definice}

% 5. 3. 2021

    \begin{definice}[Podobor]
        $S$ je podobor oboru $R$ tehdy, když $S ≤ R$ a $S$ je obor.
    \end{definice}

    \begin{definice}[Podtěleso]
        $S$ je podtěleso tělesa $R$ tehdy, když $S ≤ R$ a $S$ je těleso.
    \end{definice}

    \begin{definice}[Gaussova čísla]
        $®Z[i] = \{a+b_i | a, b \in ®Z\}$ jsou tzv. Gaussova celá čísla.
    
        $®Q[i] = \{a+bi | a, b \in ®Q\}$ jsou tzv. Gaussova racionální čísla..
    \end{definice}

    \subsection{Základní vlastnosti}
        \begin{tvrzeni}
            Mějme množinu $X$ s asociativní (tj. $(a*b)*c = a*(b*c)$) operací $*:X \times X \rightarrow X$. Pak hodnota výrazu $a_1*a_2*a_3*…*a_n$ nezávisí na uzávorkování.

            \begin{dukazin}
                Indukcí.
            \end{dukazin}
        \end{tvrzeni}

        \begin{tvrzeni}[Základní vlastnosti oborů]
            Buď $R$ okruh a $a, b, c \in R$.
            $$ 1) a + c = b + c \implies a = b, $$
            $$ 2) a·0 = 0, $$
            $$ 3) -(-a) = a,\ -(a + b) = -a + (-b), $$
            $$ 4) -(a·b) = (-a)·b = a·(-b),\  (-a)·(-b) = a·b, $$ 
            $$ 5) \text{Je-li $R$ obor, pak } a·c = b·c \land c≠0 \implies a=b. $$ 

            \begin{dukazin}
                $$ 1) (a+c) + (-c) = (b + c) + (-c) \implies a+0 = b+0 \implies a = b, $$
                $$ 2) 0 + a·0 = a·0 = a·(0 + 0) = a·0 + a·0 \implies 0 = a·0. $$
            \end{dukazin}
        \end{tvrzeni}

        \begin{tvrzeni}[Každé těleso je obor]
            Z existence $a^{-a}$ vyplývá $a≠0, b≠0 \implies ab ≠ 0$.

            \begin{dukazin}[Sporem]
                $a ≠ 0, b ≠ 0, ab=0 \implies b = (a^{-1}·a)·b = a^{-1}·(ab) = a^{-1}·0$ a podle předchozího tvrzení (část 2) $b = 0$ \lightning.
            \end{dukazin}
        \end{tvrzeni}

        \begin{tvrzeni}
            Každý konečný obor je těleso.
            
            \begin{dukazin}
                Viz skripta.
            \end{dukazin}
        \end{tvrzeni}

        \begin{definice}
            Nechť $R$ je okruh s jednotkou 1. Charakteristika $R$ je nejmenší přirozené číslo $n$ tak, že $ \underbrace{1+1+…+1}_{n\text{-krát}}$, pokud takové $n$ neexistuje, říkáme, že charakteritika je 0 (případně $∞$).

            Prvek $\underbrace{1+1+…+1}_{n\text{-krát}}$ značíme $n$, obdobně $\underbrace{-1-1-…-1}_{n\text{-krát}}$ značíme $-n$.
        \end{definice}

        \begin{tvrzeni}
            Každý obor má charakteristiku 0 nebo $p$.

            \begin{dukazin}
                Pro 1 je to cvičení. V případě, že charakteristika je $n = k·l$, $k, l≠1$, pak $0 = k·l$. Jsme v oboru, tedy $k = 0$ nebo $l=0$. Spor s minimalitou $n$.
            \end{dukazin}
        \end{tvrzeni}

    \subsection{Izomorfismus}
        \begin{definice}[Homomorfismus]
            Nechť $R, S$ jsou okruhy. Zobrazení $\phi: R \rightarrow S$ je homomorfismus okruhů, pokud $\forall a, b \in R:$
            $$ \phi(a + b) = \phi(a) + \phi(b) \land \phi(a·b) = \phi(a) · \phi(b). $$

            Je-li homeomorfismus $\phi$ bijekce, nazývá se izomorfismus.
        \end{definice}

        \begin{poznamka}
            Inverzní zobrazení k izomorfismu je izomorfismus.
        \end{poznamka}

        \begin{definice}
            Okruhy $R, S$ jsou izomorfní, pokud existuje izomorfismus $\phi: R \rightarrow S$. Značíme $R \simeq S$.
        \end{definice}

        \begin{priklady}
            Tzv. prvookruh (tj. všechny prvky tvaru $1+1+…+1$ nějakého okruhu s jedničkou) je izomorfní $®Z_n$ resp. (v tomto případě musíme zahrnout i $-1-1-…-1$) $®Z$.
        \end{priklady}

    \subsection{Podílové těleso}
        \begin{definice}[Multiplikativní množina]
            Nechť $R$ je obor. Pak $M \subseteq R$ je multiplikativní množina, pokud $0 \notin M, 1 \in M$ a $a, b \in M \implies a·b \in M$.

            \begin{prikladyin}
                Nejdůležitější MM je $M = R \setminus \{0\}$.
            \end{prikladyin}
        \end{definice}

        \begin{definice}[Podílové těleso]
            Nechť $R$ je obor a $M$ multiplikativní množina. Definujeme relaci $\sim$ na $R \times M$:
            $$ (a, b) \sim (c, d) ≡ ad = bc. $$

            Blok $[(a, b)]_{\sim}$ nazýváme zlomek a značíme $\frac{a}{b}$.

            Na $Q = \{\frac{a}{b} | a \in R, b \in M\}$ definujeme operace
            $$ \frac{a}{b} + \frac{c}{d} = \frac{ad + bc}{bd},\ \frac{a}{b}\frac{c}{d} = \frac{ac}{bd},\ -\frac{a}{b} = \frac{-a}{b},\ 0 = \frac{0}{1}, 1 = \frac{1}{1}. $$ 
            Tedy $Q$ je okruh s jednotkou. $(Q, +, -, ·, 0, 1)$ se nazývá lokalizace oboru $R$ v MM $M$. Pokud $M = R \setminus \{0\}$, pak se nazývá podílové těleso.
        \end{definice}

% 10. 3. 2021
        
        \begin{tvrzeni}
            Máme $R$, $N$, $Q$ z předchozí definice. 1) $Q$ je obor. 2) $\{\frac{a}{1} | a \in R\}$ je podobor $Q$, který je izomorfní s $R$. 3) Je-li $M = R \setminus \{0\}$, pak $Q$ je těleso.

            \begin{dukazin}
                1) Ověříme axiomy. Triviální. Důležitý je hlavně součin ne0 prvků.

                2) Ověříme uzavřenost a obsah jedničky. Ověříme, že zjevné zobrazení je izomorfizmus.

                3) Ověříme axiomy. Na tři řádky.
            \end{dukazin}
        \end{tvrzeni}

\section{Polynomy}
    \subsection{Obory polynomů}
        \begin{poznamka}[Značení]
            V celé sekci Polynomů je $R$ komutativní okruh s jednotkou.
        \end{poznamka}

        \begin{definice}[Polynom]
            Polynom v proměnné $x$ nad okruhem $R$ je výraz tvaru
            $$ a_0 + a_1·x + a_2·x^2 + a_n·x^n, $$ 
            kde $n ≥ 0$, $a_1, …, a_n \in R$ a $a_n ≠ 0$ vyjma $n = 0$. $a_1, …, a_n$ jsou koeficienty, $x$ proměnná. Navíc se dodefinovává $a_m = 0 \forall m > n$.

            Číslo $n = \deg f$ je stupeň polynomu $f$. $\deg 0 = -1$. $a_n$ se nazývá vedoucí koeficient a $a_0$ absolutní člen.

            $f$ je monický, pokud $a_n = 1$. Množinu všech polynomů značíme $R[x]$.
        \end{definice}

        \begin{definice}[Operace na $R\[x\]$]
                $$ \sum_{i=0}^m a_ix^i + \sum_{i=0}^n b_ix^i = \sum_{i=0}^{max(m, n)} (a_i + b_i)x^i;\ -\sum_{i=0}^m a_ix^i = \sum_{i=0}^m -a_ix^i; $$
                $$ \(\sum_{i=0}^m a_ix^i\)\(\sum_{i=0}^n b_ix^i\) = \sum_{i=0}^{m+n} \sum_{j+k = i, j, j ≥ 0} (a_j·b_k)x^i $$ 
        \end{definice}

        \begin{tvrzeni}
            $R[x]$ je komutativní okruh s jednotkou. Navíc je-li $R$ obor, pak i $R[x]$ je obor $\land \deg(fg) = \deg f + \deg g$ $\forall f, g \in R[x], f≠0≠g$.

            \begin{dukazin}
                Jednoduché, ve skriptech. Druhá část přes vedoucí koeficienty (jsou nenulové).
            \end{dukazin}
        \end{tvrzeni}

        \begin{definice}[Polynom více proměnných]
            Induktivní definicí: Polynom v proměnných $x_1, x_2, …, x_m$ nad okruhem $R$ je polynom v proměnné $x_m$ nad okruhem $R[x_1, …, x_{m-1}]$.

            Značíme $R[x_1, …, x_m] = (R[x_1, …, x_{m-1}])[x_m]$.

            Každý $f \in R[x_1, …, x_m]$ jde jednoznačně napsat v distribuovaném tvaru (je potřeba dokázat, ale tím pádem nezáleží na pořadí proměnných):
            $$ \sum_{k_1, …, k_m}^n a_{k_1, …, k_m}x_1^{k_1}·…·x_m^{k_m}. $$ 
        \end{definice}

    \subsection{Hodnota polynomu}
        \begin{definice}
            $R ≤ S$ obory. $f = a_0 + a_1·x + … + a_n·x^n \in R[x], u \in S$. Hodnota polynomu $f$ po dosazení $u$ je definována:
            $$ f(u):= a_0 + a_1·u + … + a_n·u^n \in S. $$
            (Operace jsou v oboru $S$.)

            Zobrazení $S \rightarrow S$, $u \mapsto f(u)$ nazýváme polynomiální zobrazení dané polynomem $f$.
        \end{definice}

    \subsection{Dělení polynomu se zbytkem}
        \begin{definice}
            $f, g \in R[x]$. $g$ dělí $f$, zapisujeme $g | f$, $≡$ $\exists h \in R[x]$ tak, že $f = gh$.

            Je-li $R$ obor a $g | f ≠ 0$ $\implies \deg g ≤ \deg f$ z tvrzení výše.
        \end{definice}

        \begin{tvrzeni}[Dělení polynomů se zbytkem]
            Nechť $R$ je obor, $Q$ podílové těleso. $f, g \in R[x], g≠0$. Pak existuje právě jedna dvojice $q, r \in Q[x]$:
            $$ f = gq + r \land \deg r < \deg g. $$

            Je-li navíc $g$ monický, pak $q, r \in R$.

            $f div g := q$ a $f mod g := r$.

            \begin{dukazin}
                    $q_0 = 0, r_0 = f$. Induktivně ($l(f) := $ vedoucí koeficient polynomu $f$):
                $$ q_{i+1} = q_i + \frac{l(r_i)}{l(g)} x^{\deg r_i - \deg g},\ r_{i+1} = r_i - \frac{l(r_i)}{l(g)}x^{\deg r_i - \deg g}·g. $$
                Vidíme, že stupeň $r_i$ se snižuje, a když $\deg r_i < \deg g$, tak skončíme a $r = r_i, q = q_i$.

                Jednoznačnost:
                $$ f = gq + r = g\tilde{q} + \tilde{r} \implies g(q - \tilde{q}) = \tilde{r} - r \implies g | \tilde{r} - r \implies \tilde{r} - r = 0. $$ 
            \end{dukazin}
        \end{tvrzeni}

    \subsection{Kořeny a dělitelnost}
        \begin{definice}
                Ať $R ≤ S$ jsou obory, $f \in R[x]$, $a \in S$. Pak $a$ je kořen $f$ $≡$ $f(a) = 0$.
        \end{definice}

        \begin{tvrzeni}
            Buď $R$ obor, $f \in R[x]$, $a \in R$. $a$ je kořen $f$ $\Leftrightarrow x-a | f$.

            \begin{dukazin}
                $\implies:$ $f = (x-a)·g$ pro nějaké $g \in R[x]$ $\implies f(a) = (a - a)·g(a) = 0$.
            
                Buď $q, r \in R[x]$ podíl a zbytek při dělení $f$ monickým polynomem $x - a$. $f = (x - a)·q + r$, $\deg r < \deg (x - a) = 1$ $\implies r$ je konstantní polynom. Dosadíme $a$:
                $$ 0 = f(a) = (a-a)q(a) + r(a) = r(a). $$ 
                $r$ je konstantní $\implies r = 0$. $f = (x - a)·q + 0 \implies x - a | f$.
            \end{dukazin}
        \end{tvrzeni}

        \begin{pozorovani}
            $$ f mod x-a = f(a) $$ 
        \end{pozorovani}

        \begin{veta}[Počet kořenů]
            $R$ obor, $0 ≠ f \in R[x]$.  Pak $f$ má nejvýše $\deg f$ kořenů v $R$.

            \begin{dukazin}
                Indukcí dělením $x - \text{kořen}$.
            \end{dukazin}
        \end{veta}

        \begin{definice}[Vícenásobný kořen]
            Ať $f \in R[x], a \in R$. Pak $a$ je $n$-násobný kořen $f$ $≡$ $(x - a)^n | f$ a $(x - a)^{n-1} \not| f$.
        \end{definice}

\section{Číselné obory}
    \subsection{Okruhová a tělesová rozšíření}
        \begin{definice}
            Nechť $R ≤ S$ jsou komutativní okruhy, $a_1, …, a_n \in S$. Definujeme $R[a_1, …, a_n]$ jako nejmenší podokruh okruhu $S$, který obsahuje $R$ a $a_1, …, a_n$. Ten nazveme okruhové rozšíření $R$ o prvky $a_1, …, a_n$.

            Nechť $R ≤ S$ jsou tělesa, $a_1, …, a_n \in S$. Definujeme $R(a_1, …, a_n)$ jako nejmenší podtěleso tělesa $S$, které obsahuje $R$ a $a_1, …, a_n$. To nazveme tělesové rozšíření $R$ o prvky $a_1, …, a_n$.
        \end{definice}

        \begin{tvrzeni}
            Mějme $R ≤ S$ komutativní okruhy s 1, $a \in R$. Pak $R[a] = \{f(a) | f \in R[x]\}$. Jsou-li $R, S$ navíc tělesa, pak $R(a) = \{\frac{f(a)}{g(a)} | f, g \in q[R], g(a) ≠ 0\}$.

            \begin{dukazin}
                Dokážeme, že je to podokruh, že obsahuje $R$ i $a$ a že je nejmenší takový.
            \end{dukazin}
        \end{tvrzeni}

% 12. 3. 2021

        \begin{pozorovani}
            Ať $T ≤ S$ jsou tělesa, potom $T[a] \subseteq T(a)$.

            Ale např. $®Q[i] = ®Q(i)$.
        \end{pozorovani}

        \begin{tvrzeni}
            Nechť $T ≤ S$ jsou tělesa, $a$ není kořenem žádného nenulového polynomu z $T[x]$. Pak $T[a] ≠ T(a)$.

            \begin{dukazin}
                Podle předchozího tvrzení $T[a] = \{f(a) | f \in T[x]\}$. Kdyby $T[a] = T(a)$, pak $T[a]$ je těleso, tedy $a^{-1} \in T[a] \implies a^{-1} = f(a)$ pro nějaký $f \in T[x]$, tedy $a·f(a) - 1 = 0$. Tedy $a$ je kořenem $x·f - 1$. \lightning.
            \end{dukazin}
        \end{tvrzeni}

    \subsection{Algebraická a transcendentní čísla}
        \begin{definice}
            $a \in ®C$ je algebraické, pokud je kořenem nějakého nenulového polynomu $f \in ®Z[x]$.

            Jinak $a$ je transcendentní.
        \end{definice}

        \begin{poznamka}[První důkaz transcendentního čísla]
            Luvil? $\sum_{i=1}^∞ 10^{-i!}$.

            Další čísla (19. stol): $\pi, e$.

            Cantor: náhodné reálné číslo je transcendentní (tj. algebraická čísla jsou spočetná / mají míru 0).
        \end{poznamka}

        \begin{tvrzeni}
            Množina algebraických čísel je spočetná.

            \begin{dukazin}
                Indexem polynomu $f = a_0 + a_1x + … + a_nx^n \in ®Z[x], f ≠ 0$ nazvěme číslo $|a_0| + |a_1| + … + |a_n| + n \in N$. Indexů existuje jen konečně mnoho daného indexu (díky započítání stupně do indexu). Všechny polynomy seřadím podle rostoucího indexu. Nyní už je zřejmě $®Z[x]$ spočetná. Navíc každý polynom má konečně kořenů, tedy, tedy i kořenů je spočetně mnoho.
            \end{dukazin}
        \end{tvrzeni}

        \begin{tvrzeni}
            Množina reálných čísel je nespočetná.
        \end{tvrzeni}

\section{Elementární teorie čísel}
    \subsection{Dělitelnost a základní věta aritmetiky}
        \begin{definice}[Dělitelnost v celých číslech]
            Ať $a, b \in ®Z$, $b$ dělí $a$, značíme $b|a$, pokud $\exists c \in ®Z: a = bc$.

            $±1$ a $±a$ se nazývají nevlastní dělitelé, ostatní jsou vlastní.
        \end{definice}

        \begin{tvrzeni}
            Mějme $a, b \in ®Z$, $b≠0$. Pak $\exists! q, r \in ®Z: a = qb+r, 0≤r<|b|$. Značíme $a \div b = q$ a $a \mod b = r$. Navíc $b|a \Leftrightarrow a \mod b = 0$
        \end{tvrzeni}

        \begin{definice}[Prvočíslo a složené číslo]
            Prvočíslo je $p \in ®Z, p > 1$, které má pouze nevlastní dělitele. Ostatní přirozená čísla $>1$ jsou složená.
        \end{definice}

        \begin{veta}[Základní věta aritmetiky]
            $\forall a \in ®Z, a > 1$ existují po dvou různá prvočísla $p_1, …, p_n$ a $k_1, …, k_n \in ®N$ tak, že $a = p_1^{k_1}·…·p_n^{k_n}$. Tento rozklad je až na pořadí jednoznačný.

            \begin{dukazin}
                Později.
            \end{dukazin}
        \end{veta}

    \subsection{NSD}
        \begin{definice}[NSD, NSN]
            Největší společný dělitel $a, b \in ®Z$ je největší $c \in ®N$ takové, že $c | a, c|b$. Značíme ho $\NSD(a, b)$ (neexistuje pro $a=b=0$).

            Nejmenší společný násobek $a,b \in ®Z\setminus \{0\}$ je nejmenší $c \in ®N$ tak, že $a | c$ a $b | c$. Značíme ho $\NSN(a, b)$.
        \end{definice}

        \begin{poznamka}
            Základní věta aritmetiky $\implies$ $a·b = \NSD(a, b)·\NSN(a, b)$.

            Rychlý algoritmus na hledání NSN je Euklidův algoritmus.
        \end{poznamka}

        \begin{tvrzeni}[Bézoutova rovnost]
            $\forall a, b \in ®Z$, $a ≠ 0$ nebo $b ≠ 0$, $\exists u, v \in ®Z$ (Bézoutovy koeficienty) tak, že $a·u + b·v = \NSD(a, b)$.

            \begin{dukazin}
                Rozšířený Euklidův algoritmus.
            \end{dukazin}
        \end{tvrzeni}

        \begin{lemma}
            Ať $p$ je prvočíslo, $a, b \in ®Z$. Pak $p|a·b \implies p|a \lor p|b$.

            \begin{poznamkain}
                V obecném oboru neplatí. Např. v $®Z[\sqrt{5}]$ $2|(\sqrt{5} + 1)(\sqrt{5} - 1) = 4$, ale $2\not|\sqrt{5} ± 1$
            \end{poznamkain}

            \begin{dukazin}
                BÚNO $p \not| a$, tedy chceme, aby $p|b$. $p$ je prvočíslo, tudíž nemá vlastní dělitele $\implies$ $\NSD(p, a) = $ buď $p$ (to by ale $p|a$), nebo $1$. Dle tvrzení o Bézoutově rovnosti $\exists u, v \in ®Z: pu+av = 1$. Vynásobíme $b$: $pbu + abv = b$. Ale $p | ab$, takže $p | pbu + abv = b$.
            \end{dukazin}
        \end{lemma}

        \begin{lemma}
            $p$ prvočíslo, $a_1, …, a_n \in ®Z$. $p|a_1 · … · a_n \implies \exists i: p | a_i$.

            \begin{dukazin}
                Indukcí z předchozího tvrzení.
            \end{dukazin}
        \end{lemma}

        \begin{dukaz}[Základní věta aritmetiky]
            Existence: pro spor ať $a$ je nejmenší přirozené číslo, které nemá rozklad na součin. Buď je $a$ prvočíslo, ale pak má rozklad $a = a^1$. Nebo je $a$ složené, tedy $a=b·c, 1 < b, c, < a$, ale $a$ bylo nejmenší číslo, které nemá rozklad, tedy $b$ i $c$ mají rozklad. Ale pak součin těchto rozkladů je $a$.

            Jednoznačnost: $a$ nejmenší přirozené číslo, které má 2 rozklady: $a = p_1^{k_1} · … · p_m^{k_m} = q_1^{l_1}·…·q_n^{l_n}$. Pak $p_1 | q_1^{l_1}·…·q_n^{l_n}$. Podle předchozího lemmatu $\exists i: p_1 | q_i$. Jsou to prvočísla, tedy $p_1 = q_i$. Potom $p_1^{k_1-1} · … · p_m^{k_m} = q_1^{l_1}·…·q_i^{k_i - 1}…·q_n^{l_n}$ jsou dva rozklady čísla $< a$. \lightning.
        \end{dukaz}

    \subsection{Kongruence}
        \begin{poznamka}[Historie]
            Symbol $≡$ zavedl v roce 1801 Gauss.
        \end{poznamka}

        \begin{definice}
            $a, b, m \in ®Z, m≠0$. $a$ je kongruentní s $b$ modulo $m$ ($a ≡ b (\mod m)$), pokud $m|a-b$. (Ekvivalentně $a, b$ dávají stejný zbytek po dělení $m$.)
        \end{definice}

        \begin{pozorovani}
            Být kongruentní $\mod m$ je ekvivalence.
        \end{pozorovani}

        \begin{tvrzeni}[Vlastnosti kongruence]
            $a, b, c, d, m \in ®Z, m≠0$. $a ≡ b \mod m, c ≡ d \mod m$.
            $$ a+c = b+d \mod m, \qquad a-c ≡ b-d \mod m, \qquad a·c ≡ b·d \mod m, \qquad a^k≡b^k \mod m, k \in ®N. $$
            $$ c≠0 \implies a≡b \mod m \Leftrightarrow ac≡bc \mod mc, \qquad \NSD(c, m) = 1 \implies a≡b \mod m \Leftrightarrow ac≡bc \mod m. $$
            \begin{dukazin}
                Z definice rozepsáním.
                $$ a≡b \mod m \Leftrightarrow \exists q: a-b = mq \Leftrightarrow ac - bc = mcq \Leftrightarrow ac ≡ bc \mod mc. $$
                $$ cu + mv = 1, cu = 1 - mv \implies (ac ≡ bc \mod m \Leftrightarrow a≡a(1-mv)≡auc≡buc≡b(1-mv)≡b \mod m). $$ 
            \end{dukazin}
        \end{tvrzeni}

    \subsection{Eulerova věta a RSA}
        \begin{definice}[Eulerova funkce]
                Eulerova funkce $\phi(n)$ značí (pro $n \in ®N$) počet $k \in \{1, 2, …, n\}$ nesoudělných s $n$, čili $\NSD(k, n)=1$.
        \end{definice}

        \begin{tvrzeni}
            $n = p_1^{k_1}·…·p_m^{k_m}$ prvočíselný rozklad, $n > 1$. Pak $\phi(n) = p_1^{k_1-1}(p_1-1)·…·p_m^{k_m - 1}(p_m - 1)$.
            
            \begin{dukazin}
                Příště.
            \end{dukazin}
        \end{tvrzeni}

        \begin{veta}[Eulerova]
            Pokud $a, m$ jsou nesoudělná přirozená čísla, pak $a^{\phi(m)} ≡ 1 \mod m$.

            Speciálním případem je Malá Fermatova věta: $p$ prvočíslo, $p\not|a \implies a^{p-1}≡1(\mod p)$.

            \begin{dukazin}
                $\Phi_m$ nechť značí množinu $\{k \in [m]|\NSD(k, m) = 1\}$. $\phi(m) = |\Phi_m|$.

                Lemma: $a, m$ nesoudělná přirozená čísla, $m≠1$. Definujeme zobrazení $f_a: \Phi_m \rightarrow \Phi_m$, $k \mapsto ka \mod m$. Pak $f_a$ je dobře definované $\land$ je to bijekce.

                Důkaz $k, a$ nesoudělná s $m \implies k·a$ nesoudělné s $m \implies k·a \mod m$ nesoudělné s $m \implies k·a \mod m \in \Phi_m$. $f_a(k) = f_a(l) \implies k·a ≡ l·a \mod m \implies k≡l \mod m$ ($a$ je nesoudělné s $m$, tedy můžeme použít tvrzeni výše) $\implies k=l$. $f_a$ je prosté a na konečné množině, tedy je bijekce.

                $$ \prod_{b \in \Phi_m}b = \prod_{b \in \Phi_m} f_a(b) = \prod_{b \in \Phi_m} (ab \mod m) ≡ a^{\phi(m)} \prod_{b \in \Phi_m} b $$
                $c = \prod_{b \in \Phi_m} b$, $c ≡ a^{\phi(m)}c \mod m$ a $c$ je nesoudělné s $m$, tedy dle tvrzení výše je $1 ≡ a^{\phi(n)} \mod m$.
            \end{dukazin}
        \end{veta}

        \begin{poznamka}
            Lemma z posledního důkazu nám říká, že každý prvek z $\Phi_m$ má inverzi v okruhu $®Z_m$.

            Ten můžeme najít buď přes Eulerovu větu, nebo přes Bézoutovu větu. (Druhý způsob je zpravidla rychlejší.)
        \end{poznamka}

% 17. 3. 2021

        \begin{poznamka}[RSA (Rivest Shamir Adleman)]
            Šifrovací algoritmus založený na Eulerově větě.
        \end{poznamka}

    \subsection{Čínská zbytková věta}
        \begin{poznamka}
            Špatně: Uvedená v knize umění války (počítání vojáků).

            Správně: vymyslel ji čínský matematik, který se jmenoval stejně jako legendární generál, autor knihy výše.
        \end{poznamka}

        \begin{veta}[Čínská zbytková]
            Nechť $m_1, …, m_n \in ®N$ po dvou nesoudělná čísla. Označíme $M = m_1·…·m_N$. Ať $u_1, …, u_n \in ®Z$. Pak\footnote{$$ [M-1]_0 = \{0, 1, …, M-1\} $$ } $\exists! x \in [M-1]_0$ tak, že $x ≡ u_1 \mod m_1, …, x ≡ u_n \mod m_n$.

            \begin{dukazin}
                Jednoznačnost: Ať $x, y \in [M-1]_0$, pro které platí všechny kongruence. Potom $\forall i: m_i | x - y$, tedy $M | x - y$. Ale $|x - y| < M$, tudíž $x - y = 0$.

                Existence: $f: [M-1]0 \rightarrow [m_1-1]_0 \times … \times [m_n - 1]_0$, $x \mapsto (x \mod m_1, …, x \mod m_n)$. Korektní definice zobrazení (mimochodem je to dokonce isomorfismus okruhů). $f$ je prosté (díky jednoznačnosti). Množiny jsou stejně velké, tedy je to dokonce bijekce, a proto existuje inverze, tudíž prvek $(u_1, …, u_n)$ musí mít obraz při zobrazení $f^{-1}$, který z definice splňuje vlastnosti hledaného prvku..
            \end{dukazin}
        \end{veta}

        \begin{dukaz}[Vzorec pro eulerovu formuli]
            1) $\phi(p^k) = p^{k-1}(p-1)$. 2) $a, b$ nesoudělná $\implies \phi(ab) = \phi(a)·\phi(b)$. Následně se vzorec dokáže aplikováním hodněkrát 2 na rozklad a jedničky nakonec.

            1) Počet čísel soudělných s $p^k$ z množiny $[p^k]$ je $p^{k-1}$, tedy počet nesoudělných je $p^k - p^{k-1}$.

            2) Funkce z důkazu čínské zbytkové věty je bijekce. Uvažujme zúžení $f$ na $\Phi_{a·b}$. Chceme: obraz zúžení je $\Phi_a \times \Phi_b$, tedy $\phi(ab) = |\Phi_{ab}| = |\Phi_a \times \Phi_b| = \phi(a)·\phi(b)$. Důkaz:

            a) $f$ zobrazí $\Phi$ do $\Phi_a \times \Phi_b$, čili, že $\NSD(x, a·b) = 1$ implikuje $\NSD(x \mod a, a) = 1, \NSD(x \mod b, b) = 1$. b) $f$ zobrazí $\Phi_{a, b}$ na $\Phi_a \times \Phi_b$, čili pokud $\NSD(u, a) = 1$, $\NSD(v, b) = 1$, pak to jediné $x$, které se zobrazí na $(u, v)$, leží v $\Phi_{a, b}$.

            $$ \NSD(x, ab) == 1 \Leftrightarrow \NSD(x, a) = 1 \land \NSD(x, b) = 1 \Leftrightarrow \NSD(x \mod a, a) = 1 \land \NSD(x \mod b, b) = 1. $$ 
            a) je zleva doprava a b) je zprava doleva.
        \end{dukaz}

\section{Abstraktní dělitelnost}
    \subsection{Dělitelnost a asociovanost}
        \begin{definice}[Dělitelnost, asociovanost, inverz]
            $R$ obor, $a, b \in R$. $b$ dělí $a$ v $R$, značíme $b|a$, pokud existuje $c \in R$ tak, že $a = b·c$.

            $a, b$ jsou asociované v $R$, pokud $a|b$, $b|a$. Značíme $a||b$.

            $a \in R$ je invertibilní, pokud existuje $b \in R$ tak, že $a·b = 1$ (značíme $b = a^{-1}$).
        \end{definice}

        \begin{pozorovani}
            $a$ je invertibilní $\Leftrightarrow$ $a || 1$.

            Relace $|$ je reflexivní $\land$ tranzitivní.
        \end{pozorovani}

        \begin{tvrzeni}
            $R$ obor, $a, b \in R$. Pak $a||b$ $\Leftrightarrow$ $\exists$~invertibilní prvek $q \in R$ tak, že $a = bq$.

            \begin{dukazin}
                $\Leftarrow$: ($a = bq \implies b | a) \land (b = aq^{-1} \implies a | b)$.

                $\implies$: $a = 0 \implies b = 0$. Ať $a ≠ 0$, $(b|a \implies a = bu) \land (a|b \implies b=av) \implies a = bu = auv$. Můžeme vykrátit $a ≠ 0$, tj. $1 = uv$, a $u, v$ jsou tedy invertibilní.
            \end{dukazin}
        \end{tvrzeni}

        \begin{definice}[Kongruence]
            $a, b, m \in R: a≡b \mod m$, pokud $m | a - b$.
        \end{definice}

        \begin{pozorovani}
            Je to ekvivalence, zachovává se přičtením a odečtením, ale nemusí platit krácení.
        \end{pozorovani}

    \subsection{Kvadratická rozšíření ®Z}
        \begin{definice}[Kvadratické rozšíření ®Z]
            Kvadratické rozšíření ®Z je $®Z[\sqrt{s}] = \{a + b\sqrt{s}|a, b \in ®Z\}$, kde $s \in ®Z$, $s$ není druhá mocnina celého čísla.
            \begin{dukazin}[Tvar $®Z\[sqrt{s}\]$]
                Dokáže se uzavřenost.
            \end{dukazin}
        \end{definice}

        \begin{definice}
            Norma na oboru $®Z[\sqrt{s}]$ je zobrazení $\ni : ®Z[\sqrt{s}] \rightarrow ®N \cup \{0\}, a+b\sqrt{s} \mapsto |a^1 - b^2s|$.
        \end{definice}

        \begin{tvrzeni}
            $\forall u, v \in ®Z[\sqrt{s}]$ platí:
            
            \begin{enumerate}
                \item $\ni(u·v) = \ni(u)·\ni(v)$,
                \item $\ni(u) = 1 \Leftrightarrow u$ je invertovatelné.
                \item Pokud $u|v$ a $v|u$, pak $\ni(u)|\ni(v)$ (víme z 1)) a $\ni(u)≠\ni(v)$.
            \end{enumerate}

            \begin{dukazin}
                1) vezmu a ověřím. Nebo využiji, že $\ni(u) = |u·u'|$, kde $u' = a - b\sqrt{s}$, $u = a + b\sqrt{s}$. Zjistíme, že $(u·v)' = u'·v'$. Potom $|u·v·(u·v)'| = |u·u'|·|v·v'|$.

                2) $\Leftarrow$: $u·u^{-1} = 1 \implies \ni(u·u^{-1}) = \ni(1) = 1$. A z 1) už plyne $\ni(u) = 1$. $\implies$: $\ni(u) = 1 \implies u·u' = 1 \implies u'$ je hledaná inverze.

                3) $u = 0 \implies v = 0 \implies v|u$. Ať tedy $v=uc$ pro $c \in ®Z[\sqrt{s}]$. Ať $\ni(u) = \ni(v) = \ni(u·c) = \ni(u)·\ni(c) \implies \ni(c) = 1 \implies c$ je invert $\implies v||u$, čili $v|u$ spor.
            \end{dukazin}
        \end{tvrzeni}

% 19. 3. 2021

        \begin{upozorneni}
            Norma nesplňuje trojúhelníkovou nerovnost!
        \end{upozorneni}

        \begin{tvrzeni}[Dělení Gaussových čísel se zbytkem]
            $$ \forall \alpha, \beta \in ®Z[i], \beta ≠ 0\ \exists \gamma, \delta \in ®Z[i]: \alpha = \beta·\gamma + \delta \land \ni(\delta)<\ni(\beta). $$

            \begin{dukazin}
                $®Z[i] \subseteq ®C$, tudíž berme $\frac{\alpha}{\beta} \in ®C$. Zvolme $\gamma \in ®Z[i]$ jako nejbližší hodnotu k $\frac{\alpha}{\beta}$. Položme $\delta = \alpha - \beta·\gamma$. $\frac{\delta}{\beta} = \frac{\alpha}{\beta} - \gamma$, tj. $|\frac{\delta}{\beta}| ≤ \frac{\sqrt{2}}{2}$, tj. $\ni{\delta} ≤ \(\frac{\sqrt{2}}{2}\)^2|\beta|^2 < 1\ni(\beta)$.
            \end{dukazin}
        \end{tvrzeni}

        \begin{poznamka}
            Takováto definice dělení se zbytkem funguje ještě pro $®Z[\sqrt{-2}]$ a $®Z[\sqrt{2}]$, ale pro ostatní $®Z[\sqrt{s}]$ už nefunguje.
        \end{poznamka}

    \subsection{Největší společný dělitel}
        \begin{definice}[Největší společný dělitel, nesoudělnost a největší společný násobek]
            Pro $a, b \in R$, $R$ obor řekneme, že $c \in R$ je největší společný dělitel $a, b$, značíme $c = \NSD(a, b)$, pokud 1) $c|a \land c|b$ a 2) $\forall d|a, d|b: d|c$.

            $a, b$ jsou nesoudělné, pokud $\NSD(a, b) = 1$.

            Obdobně definujeme $\NSN(a, b) = c ≡ a|c \land b|c \land \forall d, a|d, b|d: c|d$.
        \end{definice}

        \begin{poznamka}
            $\NSD$ nemusí existovat. Zároveň není jednoznačně určený. Ale je jednoznačně určený až na asociovanost.
        \end{poznamka}

    \subsection{Ireducibilní prvky a rozklady}
        \begin{definice}[Vlastní dělitel a ireducibilní prvek]
            $R$ obor. $a \in R \setminus \{0\}$. $b \in R$ je vlastní dělitel $a$, pokud $b|a$ a $b \not||1$ a $b\not||a$.

            $a≠0$ je ireducibilní, pokud $a \not||1$ a nemá žádné vlastní dělitele.
        \end{definice}

        \begin{definice}[Ireducibilní rozklad]
            Ireducibilní rozklad prvku $a$ je zápis $a || p_1^{k_1}p_2^{k_2}…p_n^{k_n}$, kde $p_1, …, p_n$ jsou ireducibilní prvky a $p_i \not|| p_j$, pro $i≠j$, a kde $k_1, …, k_n \in ®N$.

            Řekneme, že $a$ má jednoznačný ireducibilní rozklad, pokud má právě 1 rozklad až na pořadí a asociovanost.
        \end{definice}

    \subsection{Prvočinitelé}
        \begin{definice}[Prvočinitel]
            $R$ obor, pak $p \in R, p\not||1$ je prvočinitel, pokud $\forall a, b \in R: p|a·b \implies p|a \lor p|b$.
        \end{definice}

        \begin{pozorovani}
            $p$ je prvočinitel $\implies$ $p$ je ireducibilní.

            \begin{dukazin}
                Ať $p = ab$. Pak $p|a·b$ $\overset{\text{prvočinitel}}{\implies} p|a \lor p|b$. Zároveň zřejmě $a|p$ a $b|p$, tedy $p||a \implies b||1$ nebo $p||b \implies a||1$. Tedy $a, b$ jsou nevlastní dělitelé.
            \end{dukazin}
        \end{pozorovani}

\section{Existence a jednoznačnost ireducibilního rozkladu}
    \subsection{Gaussovské obory}
        \begin{definice}[Gaussovský obor]
            Obor $R$ je gaussovský, pokud $\forall a \in R, a≠0, a\not||1$, má jednoznačný ireducibilní rozklad.
        \end{definice}

        \begin{priklad}[Otevřený problém]
            $®Z[\sqrt{s}]$ je gaussovský pro $∞$ mnoho $s$. (Čeká se, že ano.)
        \end{priklad}

        \begin{poznamka}[Rozšíření definice ireducibilního rozkladu]
            $a||1$, pak řekneme, že ireducibilní rozklad $a$ je $a||1 = …^0$.
        \end{poznamka}

        \begin{tvrzeni}[Vlastnosti gaussovských oborů]
            $R$ je gaussovský obor a $a, b \in R$, $a, b ≠ 0$. Ať navíc je $a || p_1^{k_1}·…·p_n^{k_n}$ je ireducibilní rozklad. Pak $b|a \Leftrightarrow b || p_1^{l_1}·…·p_n^{l_n}$ (nemusí být rozklad, protože $l_i$ smí být 0), kde $\forall i: 0≤l_i≤k_i$.

            \begin{dukazin}
                $\Rightarrow$: Ať $b = rp_1^{l_1}·…·p_n^{l_n}$ a $a = q·p_1^{k_1}·…·p_n^{k_n}$, kde $r || 1 || q$. Chci: $b|a$, čili $\exists c: a=b·c$. $c = q·r^{-1}·p_1^{k_1-l_1}·…·p_n^{k_n - l_n}$.

                $\implies$: $b|a \implies \exists c: a = b·c$. Ať $b||q_1^{s_1}·…·q_u^{s_u}$, $c || r_1^{t_1}·…·r_v^{t_v}$ jsou ireducibilní rozklady. Zkombinujeme na rozklad $b·c: B·C||q_1^{s'_1}·…·q_u^{s'_u}·r_{i_1}^{t_{i_1}}·…·r_{i_w}^{t_{i_w}}$ (vyfiltrujeme z rozkladu $c$ ty $r_i$, který jsou asociovány s nějakým $q_j$). Máme 2 rozklady $b·c = a$. Z jednoznačnosti rozkladů $q_i = p_{\pi(i)} \land s'_i = k_{\pi(i)} ≥ s_i$. Tudíž $b || p_{\pi(1)}^{s_1}·…·p_{\pi(n)}^{s_n}$, kde $s_i ≤ k_{\pi(i)}$ (a doplníme chybějící $p_j^0$).
            \end{dukazin}
        \end{tvrzeni}

% 24. 3. 2021

        \begin{dusledek}[Dělitelnost v gaussovských oborech]
            $R$ gaussovský obor. Pak $\forall a, b \in R$, $a≠0 \lor 0≠b$ $\implies$ existuje $\NSD(a, b)$. Každý ireducibilní prvek je prvočinitel. Neexistuje posloupnost $a_1, a_2, a_3, … \in R: a_{i+1}|a_i \land a_i \not|| a_{i+1}$.

            \begin{dukazin}
                Mějme rozklady $a || p_1^{k_1}·…·p_n^{k_n}$ a $b || p_1^{l_1}·…·p_n^{l_n}$ (doplněné tak, aby měli shodná prvočísla, ale $k_i ≠ 0 \lor l_i ≠ 0$).

                Ať $a, b ≠ 0$, potom existuje jednoznačný rozklad na prvočinitele. Potom každé (a jenom ty) $c || p_1^{m_1}·…·p_n^{m_n}$, kde $0 ≤ m_i ≤ \min(k_i, l_i)$ dělí $a$ i $b$, tedy $c$ s největšími $m_i$ a to už je zřejmě $\NSD(a, b)$.

                Nechť $p | a·b$ a zároveň je ireducibilní, tj. $p = p_i$ pro nějaké $i$. Toto $p_i$ musí být v nenulové mocnině v $a$ nebo v $b$, tedy $p$ dělí jedno z nich.

                Definujeme normu $\ni(a) = k_1 + … + k_n$. Jelikož máme jednoznačný ireducibilní rozklad, tak $\ni$ je dobře definovaná. Pokud $b|a$, pak $\ni(b) ≤ \ni(a)$, pokud navíc $b \not || a$, pak $\ni(b) < \ni(a)$. Posloupnost $\ni(a_i)$ je pak nekonečná klesající posloupnost v ®N. \lightning. 
            \end{dukazin}
        \end{dusledek}

    \subsection{Zobecněná základní věta aritmetiky}
        \begin{veta}[Zobecněná základní věta aritmetiky]
            $R$ je gaussovský $\Leftrightarrow$ existuje $\NSD$ všech dvojic prvků (krom 0, 0) $\land$ neexistuje nekonečná posloupnost vlastních dělitelů $a_1, a_2, a_3, … \in R: a_{i+1}|a_i \land a_i \not||a_{i+1}$.

            \begin{dukazin}[$\implies$]
                Je dokázáno.
            \end{dukazin}

            \begin{dukazin}[Existence rozkladů]
                Sporem s druhou částí: Ať $a_1 = a$, $a_1 \not||1$ a $a$ nemá ireducibilní rozklad. Mějme $a_i \not|| 1$ a nemá ireducibilní rozklad. Tedy není ireducibilní (jinak by bylo samo sobě rozkladem) $\implies a_i = b·c$ pro nějaké $b, c \not||1$. Kdyby $b, c$ měly ireducibilní rozklad, pak by i. rozklad mělo i $a_i$. Takže aspoň jeden z nich nemá IR. Označíme ho $a_{i+1}$. Tudíž $a_{i+1}|a_i \land a_{i+1}\not||1 \land a_{i+1}$ nemá IR. Indukcí tedy vyrobíme nekonečnou posloupnost, kterou mi podmínky zakazují. \lightning.
            \end{dukazin}

            \begin{lemmain}
                $R$ obor, $a, b \in R$, $c \in R$, $c ≠ 0$. Předpokládejme, že existuje $\NSD(a, b)$, $\NSD(ca, cb)$. Pak $\NSD(ca, cb) = c·\NSD(a, b)$.

                \begin{dukazin}
                    Ve skriptech. Triviální.
                \end{dukazin}
            \end{lemmain}

            \begin{lemmain}
                Buď $R$ obor, ve kterém existuje $\NSD$ všech dvojic prvků. Pak je každý ireducibilní prvek prvočinitel.

                \begin{dukazin}
                    Buď $p$ ireducibilní a ať $p|a·b$. Ať $p \not| a$. $\NSD(p, a)$ existuje, tedy $\NSD(p, a) = 1$, neboť $p$ je ireducibilní. Podle předchozího lemmatu $\NSD(pb, ab) = b·\NSD(p, a) = b$. Zároveň $p | pb$ a $p|ab$, $b$ je $\NSD \implies p|b$.
                \end{dukazin}
            \end{lemmain}

            \begin{dukazin}[Jednoznačnost rozkladu]
                Sporem: Mezi všemi prvky s nejednoznačnými rozklady vyberme ten, který má nejkratší rozklad, čili má minimální $k_1+…+k_n$. Nechť tedy $a || p_1^{k_1}·…·p_n^{k_n} || q_1^{l_1}·…·q_m^{l_m}$. $p_1$ je ireducibilní a dělí $a$, tedy (podle předchozího lemmatu) dělí $q_i$ pro nějaké $i$. To ale znamená, že $p_1^{k_1 - 1}·…·p_n^{k_n} || …$. To jsou ale zase dva různé ireducibilní rozklady, ale to je spor s minimalitou.
            \end{dukazin}
        \end{veta}

\section{Eukleidův algoritmus a Bézoutova rovnost}
    \subsection{Eukleidovské obory}
        \begin{definice}[Eukleidovský obor]
            $R$ je obor. $R$ je eukleidovský, pokud na něm existuje tzv. eukleidovská norma, čili zobrazení $\ni: R \rightarrow ®N_0$ tak, že $\ni(0) = 0$, $a|b \land b ≠ 0 \implies \ni(a) ≤ \ni(b)$, $\forall a, b \in R, b ≠ 0\ \exists q, r \in R: a = bq + r \land \ni(r) < \ni(b)$.
        \end{definice}
        
        \begin{pozorovani}
            $a = 0 \Leftrightarrow \ni(a) = 0$. (Z ostré nerovnosti v třetí podmínce.)
        \end{pozorovani}

        \begin{pozorovani}
            Tělesa jsou eukleidovská ($\ni(0) = 0$, $\ni(a≠0) = 1$). ®Z je eukleidovské $\ni(a) = |a|$. $®Z[i]$ je eukleidovské. ®T těleso, $R = T[x]$ je eukleidovský obor ($\ni(f) = 1 + \deg f$).

            $®Z[x]$ není eukleidovské (ale je gaussovské). ($\NSD(x + 1, x - 1) ≠ f(x)·(x + 1) + g(x)·(x-1)$. Tj. neplatí Bézoutova rovnost.)
        \end{pozorovani}

        \begin{poznamka}
            Eukleidův algoritmus funguje normálně, jen dělení se zbytekm je určené podle definice Eukleidovských oborů.
        \end{poznamka}

        \begin{veta}[Správnost eukleidova algoritmu]
            V eukleidovském oboru $R$ najde rozšířený Eukleidův algoritmus pro jakýkoliv vstup $a, b \in R$ hodnotu $\NSD(a, b)$ a Bézoutovy koeficienty $u, v$ splňující $\NSD(a, b) = u·a + v·b$.

            \begin{dukazin}
                EA skončí, neboť norma se zmenšuje a je nezáporná. Stačí ukázat, že $\NSD(a_{i-1}, a_i) = \NSD(a_{i+1}, a_i)$ a $a_i = u_i·a + v_i·b$. Obojí plyne z $a_{i-1} = a_{i}q+a_{i+1}$
            \end{dukazin}
        \end{veta}

% 26. 3. 2021

        \begin{poznamka}[Oprava]
            $\NSD(0, 0) = 0$, tento případ tedy nemusel být v tvrzení výše vynecháván…<F2> 
        \end{poznamka}

        \begin{lemma}
            $R$ eukleidovský obor, $a, b \in R \setminus \{0\}$. Pokud $a|b$ a $a \not|| b$, pak $\ni(a) < \ni(b)$.

            \begin{dukazin}
                Ať $b = a·u$ pro nějaké $u \in R$. Víme, že $\exists q, r \in R$, $a = bq + r$, $\ni(r) < \ni(b)$. $a\not||b$ $\implies$ $b \not|a \implies r≠0$. $r = a-bq = a(1-uq) \implies a|r$. Z definice dělení se zbytkem je $\ni(a) ≤ \ni(r) < \ni(b)$.
            \end{dukazin}
        \end{lemma}

        \begin{veta}
            Eukleidovské obory jsou gaussovské.

            \begin{dukazin}
                $R$ eukleidovský. Podle jedné z předchozích vět: gaussovský $\Leftrightarrow$ $\exists \NSD$ a $\nexists$ řetězec vlastních dělitelů. $\NSD$ v eukleidovském existuje. Podle lemmatu výše se norma vlastních dělitelů zmenšuje, tedy opravdu takový řetězec neexistuje.
            \end{dukazin}
        \end{veta}

        \begin{dusledek}
            $®Z[i]$ je gaussovský. $®T[x]$ je gaussovský.
        \end{dusledek}

    \subsection{Diofantické rovnice, rozklad v $®Z\[i\]$}
       Viz přednáška, nebude u zkoušky.

    \subsection{Obory hlavních ideálů}
        \begin{definice}
            $R$ je komutativní okruh. Ideál v $R$ je neprázdná podmnožina $I \subseteq R$ tak, že $a, b \in I \implies a+b \in I, -a \in I$, $a \in I, r \in R \implies r·a \in I$.
        \end{definice}

        \begin{priklady}
            $R = ®Z$, $I = n®Z$ pro libovolné $n \in ®Z$. (Dále dokážeme, že jiný v ®Z neexistuje.)
        \end{priklady}

        \begin{tvrzeni}[Definice hlavních ideálů]
            $R$ komutativní okruh, $a \in R$. Pak $a·R = \{a·r|r \in R\} = \{u \in R\,|\,a|u\}$ je ideál v $R$. Navíc je to nejmenší (vůči inkluzi) ideál v $R$, který obsahuje $a$. Takovému ideálu se říká hlavní.

            \begin{dukazin}
                $ar, as \in aR \implies ar + as = a(r + s) \in aR, -ar = a·(-r) \in aR$, $ar \in aR, t \in R \implies art \in aR$. Tedy $R$ je ideál.

                Buď $I$ ideál v $R$, $a \in I$. Z uzavřenosti plyne, že $ar \in I \forall r \in R \implies aR \subseteq I$. Tedy $aR$ je nejmenší.
            \end{dukazin}
        \end{tvrzeni}

        \begin{poznamka}
            Hlavní, protože je tam ten hlavní prvek $a$, který ho vytváří.
        \end{poznamka}

        \begin{definice}
            Hlavním ideálům $0R = \{0\}$ a $1R = R$ se říká nevlastní, ostatním se říká vlastní.
        \end{definice}

        \begin{pozorovani}
            $$ a|b \Leftrightarrow aR \supseteq bR. $$

            \begin{dukazin}
                Triviální, viz přednáška.
            \end{dukazin}

            \begin{dusledekin}
                $$ a||b \Leftrightarrow aR = bR. $$
            \end{dusledekin}
        \end{pozorovani}

        \begin{veta}
            V eukleidovském oboru je každý ideál hlavní.

            \begin{dukazin}
                $R$ eukleidovský obor, $I$ ideál. Pokud $I = \{0\} \implies I = 0R$. Ať $I \supset \{0\}$. Buď $0 ≠ a \in I$ (libovolný) prvek s nejmenší možnou normou $\ni(a)$. Dokážeme, že $I = aR$. Zřejmě $aR \subseteq I$, protože $a \in I$. Pro spor ať existuje $b \in I \setminus aR$. Vydělíme se zbytkem: $b = aq + r, \ni(r)<\ni(a)$. Ale máme $r = b-aq$, přičemž $b, a, aq \in I$, tudíž $r = b-aq \in I$, ale z minimality normy $a$ je $r = 0$, tudíž $a | b$. \lightning.
            \end{dukazin}
        \end{veta}

        \begin{definice}[Obor hlavních ideálů (OHI)]
            Pokud $R$ je obor tak, že každý ideál je hlavní, pak se $R$ nazývá obor hlavních ideálů (OHI).
        \end{definice}

        \begin{priklady}
            $®Z[x]$ není OHI. $®Z[\frac{1 + \sqrt{-19}}{2}$ je OHI, ale není euklidovský (těžké dokázat).
        \end{priklady}

        \begin{tvrzeni}
            $R$ komutativní okruh s 1. $R$ je těleso $\Leftrightarrow$ $R$ má pouze nevlastní ideály.

            \begin{dukazin}
                Ať $I ≠ \{0\}$. Buď $0 ≠ a \in I$. $R$ těleso $\implies a^{-1} \in R$. Z uzavřenosti na násobení $1 = a·a^{-1} \in I$, tudíž $R = 1·R \in I$, tj. $I = R = 1R$.
            \end{dukazin}
        \end{tvrzeni}

        \begin{tvrzeni}
            $R$ komutativní okruh.

            1) $I, J$ ideály v $R$ $\implies$ $I \cap J$ je ideál v $R$.

            2) $I, J$ ideály v $R$. Pak $I + J = \{a + b | a \in I, b \in J\}$ je ideál. Navíc je to nejmenší ideál, který obsahuje $I, J$.

            3) Mějme ideály $I_j$ v $R$ pro $j \in ®N$ tak, že $I_1 \subseteq I_2 \subseteq I_3 \subseteq …$. Pak $\bigcup_{j=®N}I_j$ je ideál v R.

            \begin{dukazin}
                1) $a, b \in I \cap J, r \in R \implies a, b \in I, a,v \in J$. $I$ ideál $\implies a + b, -a, ra \in I$. $J$ ideál $\implies a + b, -a, ra \in J$. Tedy $a + b, -a, ra \in I \cap J$.

                2) Ať $a + b \in I + J, c + d \in I + J, r \in R$, kde $a, c \in I$, $b, d \in J$. Pak $(a+b) + (c+d) = (a + c) + (b + d) \in I + J$. $· \land -$ obdobně. $I + J$ ideál.

                Zřejmě $I \subseteq I + J$, neboť je $a + 0 \in I + J$. Stejně tak pro $J$, tj. $I \cup J \subseteq I + J$. Druhý 'směr' plyne z uzavřenosti na součet.

                3) Uzavřenost na $+$: Ať $a, b \in \bigcup I_j$. Tudíž $a \in I_j, b \in I_k$ pro nějaká $j, k$, BÚNO $j ≤ k$. Máme $I_j \subseteq I_k$, tedy $a \in I_k$. $I_k$ je ideál, tedy je uzavřený na součet. Uzavřenost na $· \land -$ snadná (stačí vzít 1 ideál).
            \end{dukazin}
        \end{tvrzeni}

% 31. 3. 2021

        \begin{veta}
            Buď $R$ OHI. Pak $R$ je gaussovský a platí v něm Bézoutova rovnost.

            \begin{dukazin}
                $R$ OHI. Chceme 1) existuje NSD 2) neexistují řetězce vlastních dělitelů (zobecněná věta algebry):

                1) $a, b \in R$. Buď $I = aR + bR$, (protože OHI) existuje $c \in R, cR = I$. $aR, bR \subseteq cR \implies c|a, b$. Buď $d |a, b$ $\implies aR, bR \subseteq dR \implies aR+bR = cR \subseteq dR \implies d|c$. Tedy $c = \NSD(a, b)$. Navíc $c \in aR + bR = \{ar+bs\}$, tj. $c=ar+bs$ pro nějaké $r, s \in R$.

                2) Pro spor uvažujme takovou posloupnost dělitelů $…|a_2|a_1$, tj. $a_1R \subset a_2R \subset …$. $I = \bigcup_{i=1}^∞ a_iR$ je ideál, tj. (protože OHI) $I = bR$, pro nějaké $b \in I$. Ale tím pádem $\exists i: b \in a_iR$. Pak $bR \subseteq a_iR \subset a_{i+1}R \subset … \subseteq I = bR$. \lightning.
            \end{dukazin}
        \end{veta}

\section{Polynomy nad gaussovskými obory (bez důkazů)}
    \begin{definice}[Primitivní polynom]
        $R$ obor, $f \in R[x]$ je primitivní, pokud jsou jeho koeficienty nesoudělné (čili $\forall c \in R:$ pokud $c$ dělí všechny koeficienty, pak $c||1$).
    \end{definice}

    \begin{veta}[Gaussovo lemma]
        $R$ gaussovský obor, $f, g$ primitivní polynomy v $R[x]$ $\implies f·g$ primitivní v $R[x]$.
    \end{veta}

    \begin{tvrzeni}
        $R$ je gaussovský, $Q$ podílové těleso $R$. $f, g$ primitivní polynomy v $R[x]$. Pak $f|g$ v $R[x]$ $\Leftrightarrow$ $f|g$ v $Q[x]$.
    \end{tvrzeni}

    \begin{definice}[Značení]
        $f = \sum_{i=0}^n a_ix^i \in R[x], a_n ≠ 0$ ($R$ gaussovský). $c(f) = \NSD(a_0, a_1, …, a_n)$ je obsah (content) polynomu.

        $PP(f) = \frac{1}{c(f)}·f$ je primitivní část (primitive part) $f$.
    \end{definice}

    \begin{veta}
        $R$ gaussovský, $Q$ podílové těleso, $f, g \in R[x]$. Pak:
        $$ \exists \NSD_{R[x]} (f, g) = c·h, c = \NSD_R(c(f), c(g)), h \in R[x]\text{ je primitivní tak, že } h = \NSD_{Q[x]}(f, g). $$
        $f$ je ireducibilní v $R[x]$ $\Leftrightarrow$ $\deg f = 0$ a $f$ je ireducibilní v $R$, nebo $\deg f > 0$, $f$ je primitivní a $f$ je ireducibilní v $Q[x]$.
    \end{veta}

    \begin{veta}[Gaussova]
        $R$ gaussovský obor $\implies$ $R[x]$ gaussovský obor.
    \end{veta}

    \begin{dusledek}
        $R$ gaussovský $\implies$ $R[x_1, …, x_n]$ gaussovský $\implies$ $R[x_1, x_2, x_3, …]$ gaussovský.
    \end{dusledek}

    \subsection{Ireducibilita polynomů (i s důkazy)}
        \begin{tvrzeni}[Existence racionálního kořene]
            Nechť $R$ je gaussovský, $Q$ je podílové těleso. Má-li $f = \sum_{i=1}^n a_ix^i \in R[x]$, $a_n ≠ 0$ kořen $\frac{r}{s} \in Q$ (pro $\NSD(r, s) = 1$), pak $r | a_0, s | a_n$.

            \begin{dukazin}
                    $0 = f\(\frac{r}{s}\) = \sum a_i \(\frac{r}{s}\)^i$ přenásobíme $s^n$: $0 = a_0s^n + a_1rs^{n-1} + … + a_nr^n \implies r|a_0s^n$. Ale $\NSD(r, s) = 1$, tedy z gaussovskosti $r|a_0$. Stejně tak $s|a_nr^n \implies s|a_n$.
            \end{dukazin}    
        \end{tvrzeni}

        \begin{tvrzeni}[Einsteinovo kritérium]
            $R$ obor, $f = \sum_{i=0}^n a_ix^i \in R[x]$ primitivní, $a_n ≠ 0$. Pokud existuje prvočinitel $p \in R$ tak, že $p|a_0, a_1, …, a_{n-1}, p^2\not|a_0$, pak $f$ je ireducibilní.

            \begin{dukazin}
                Pro spor $f = g · h$, $g = \sum_{i=0}^k b_ix^i$, $h = \sum_{i=0}^l c_ix^i \in R[x]$, l, k > 0.
                $$ a_0 + a_1x + a_2x^2 + … = (b_0 + b_1x + …)(c_0 + c_1x + …) = b_0c_0 + (b_0v_1 + b_1c_0)x + … \implies a_0 = b_0c_0. $$
                Tudíž $p|a_0 = b_0c_0 \implies $ BÚNO $p|b_0$, pak $p\not|c_0$, neboť $p^2\not|a_0$. $p|a_1 = b_0c_1 + b_1c_0 \implies p|b_1$, …, $p|b_i$ $\forall i ≤ n-1$. $p$ dělí všechny koeficienty $b_i$ pro $i ≤ k ≤ n-1$, ale jelikož $h$ má stupeň alespoň 1, tak $p$ dělí všechny koeficienty $b_i$, tj. $p | g | f$. \lightning.
            \end{dukazin}
        \end{tvrzeni}

\section{Čínská zbytková věta a interpolace}
    \begin{veta}[ČZV pro polynomy]
        ®T těleso. Ať $m_1, m_2, …, m_n \in ®T[x]$ jsou po 2 nesoudělné polynomy, $d = \sum \deg m_i$. Ať $u_1, …, u_n \in ®T[x]$. Pak $\exists! f \in ®T[x]$ stupně $<d$ tak, že $f ≡ u_1 \mod m_1, …, f ≡ u_n \mod m_n$.

        \begin{dukazin}
            Jednoznačnost: Ať $f, g$ jsou řešení, $\deg f, \deg g < d$, čili $f≡g≡u_i \mod m_i \forall i$. Tedy $m_i | f-g \forall i$. $m_i$ jsou po dvou nesoudělné a $®T[x]$ je gaussovské, tj. $m_1·…·m_n |f - g$, tj. $\deg(f-g) > d$ (\lightning) nebo $f-g = 0$.

            Existence: $P_k = \{f \in T[x]|\deg f < k\}$ je vektorový prostor nad ®T dimenze $k$ ($x^i$ je báze). $d_i = \deg m_i$. $\phi: P_d \rightarrow P_{d_1}\times … \times P_{d_n}$, $f \mapsto (f \mod m_1, …, f \mod m_n)$. Zřejmě $P_{d_i}$ má dimenzi $d_i$ a $\phi$ je dobře definované a navíc homomorfismus vektorových zobrazení. Navíc z jednoznačnosti (1. bodu důkazu) je prosté, tj. z porovnání dimenzí je $\phi$ bijekce. Tedy hledaný polynom je $\phi^{-1}(u_1 \mod m_1, …, u_n \mod m_n)$.
        \end{dukazin}
    \end{veta}

% 7. 4. 2021

    \begin{dusledek}[Věta o interpolaci]
        ®T těleso. Mějme po 2 různé body $a_1, …, a_n \in ®T$ a libovolné hodnoty $u_1, …, u_n \in ®T$. $\exists! f \in ®T[x], \deg f < n$ tak, že $\forall i: f(a_i) = u_i$.

        \begin{dukazin}
            $f ≡ f(a) (\mod x - a)$ (už jsme ukázali), tedy $f ≡ u_i (\mod x - a_i)$ a použijeme čínskou zbytkovou větu.
        \end{dukazin}
    \end{dusledek}

    \begin{dusledek}[Zobrazení na konečných tělesech jsou polynomiální]
        ®T je konečné těleso. Pro $\forall \phi: ®T \rightarrow ®T$ zobrazení $\exists! f \in ®T[x]$, $\deg f < |®T|$ tak, že $\phi(a) = f(a)$.
    \end{dusledek}

\section{Faktorokruh modulo polynom}
    \begin{definice}[Faktorokruh]
        ®T těleso. Buď $m \in ®T[\alpha]$ polynom stupně $n ≥ 1$. Faktorokruh $®T[\alpha] / (m)$ je množina všech polynomů z $T[\alpha]$ stupně $< n$ se standardním $+$ a $-$ a s operací násobení modulo $m$, čili $f \odot g = f·g \mod m$.

        Čili $®T[\alpha]/(m) = \(\{f \in ®T[\alpha] | \deg f < n\}, +, -, \odot, 0, 1\)$.
    \end{definice}

    \begin{pozorovani}
        Jde o komutativní okruh s 1. (Ověříme axiomy.)
    \end{pozorovani}

    \begin{tvrzeni}[Faktor podle ireducibilního polynomu]
        ®T těleso, $m \in ®T[\alpha], \deg m ≥ 1$. Pak následující je ekvivalentní: 1) $T[\alpha]/(m)$ je těleso, 2) $T[\alpha]/(m)$ je obor, 3) $m$ je ireducibilní prvek v $®T[\alpha]$.

        \begin{dukazin}
            $1 \implies 2$ zřejmé (jedno z prvních tvrzení), $2 \implies 3$: Ať $m = fg$ pro $f, g \in ®T[\alpha]$, $\deg f, \deg g ≥ 1$. Pak v $®T[\alpha] / (m)$ platí $f\odot g = fg \mod m = m \mod m = 0$, čili $®T[\alpha] / (m)$ není obor.

            $3 \implies 1$: Buď $f ≠ 0$ polynom, $\deg f < \deg m$. $m$ ireducibilní, $f$ má menší stupeň než $m \implies m, f$ jsou nesoudělné. Bézout: $1 = \NSD(f, m) = uf + vm$ pro nějaké $u, v \in T[\alpha]$. Buď $\tilde{u} = u \mod m$. Pak v $®T[\alpha] / (m)$ platí: $\tilde{u} \odot f = \tilde{u}f \mod m ≡ uf ≡ 1 (\mod m)$. Tedy $\tilde(u)\odot f = 1$ v $®T[\alpha] / (m)$. Tedy $\tilde{u}$ je inverz.
        \end{dukazin}
    \end{tvrzeni}

    \begin{poznamka}
        Dál budeme $\odot$ značit jako $·$.
    \end{poznamka}

    \subsection{Kořenová, rozkladová nadtělesa}
        \begin{tvrzeni}
            ®T těleso, $f \in ®T[x]$, $\deg f ≥ 1$. Pak existuje $S ≥ T$, ve kterém ma $f$ kořen.

            \begin{dukazin}
                Buď $m = \sum_{i=0}^n a_ix^i \in ®T[x]$ nějaký ireducibilní dělitel $f$. $S = T[\alpha] / (m(\alpha))$. Z předchozího tvrzení je $S$ tělesso a $S ≥ T$ (neboť ®T jsou tam konstantní polynomy). Chceme $m(x)$ má v $S$ kořen (pak má triviálně i $f$ kořen v $S$).

                $$ m(\alpha) = \sum a_i \odot (\alpha \odot … \odot \alpha) = \sum(a_i \alpha^i \mod m) = $$
                $$ = a_0 \mod m + a_1\alpha \mod m + … + a_n \alpha^n \mod m = a_0 + a_1\alpha  + … + a_{n-1} \alpha^{n-1} + (- a_0 - a_1 \alpha - … - a_{n-1}\alpha^{n-1}) = 0. $$ 
            \end{dukazin}
        \end{tvrzeni}

        \begin{veta}
            ®T těleso, $f \in ®T[x]$, $\deg f ≥ 1$. Pak existuje těleso $S ≥ T$, kde se $f$ rozkládá na součin polynomů stupně 1.

            \begin{dukazin}
                Indukcí podle $f$. $\deg f = 1 \implies f = ax + b$ a má kořen $-a^{-1}b \in ®T$.

                $\deg f > 1$. Podle předchozího tvrzení buď $U ≥ T$ tak, že $f(u) = 0$ pro nějaké $u \in U$. Pak $f = (x - u)·g$ pro nějaké $g \in U[x]$, $\deg g = \deg f - 1$. Následně použijeme indukční předpoklad pro $g$.
            \end{dukazin}
        \end{veta}

        \begin{definice}
            ®T těleso, $f \in ®T[x]$, $\deg f ≥ 1$. Kořenové nadtěleso je (libovolné) těleso $®S ≥ ®T$, ve kterém existuje $a \in ®S$ tak, že $®S = ®T(a)$ a $f(a) = 0$.

            Rozkladové nadtěleso $f$ je (libovolné) těleso $®S ≥ ®T$, že existují $a_1, …, a_n \in ®S: ®S = ®T(a_1, …, a_n)$ a $f||(x - a_1)·…·(x - a_n)$.
        \end{definice}

        \begin{dusledek}[Existence kořenového a rozkladového nadtělesa]
            ®T těleso, $f \in ®Ť[x]$, $\deg ≥ 1$. Pak existuje kořenové i rozkladové nadtěleso $f$ nad ®T.

            \begin{dukazin}
                $\exists ®S ≥ ®T$ tak, že $f(a) = 0$ pro $a \in ®S_0$. Kořenové nadtěleso pak je $®S = ®T(a) ≤ ®S_0$. Obdobně rozkladové.
            \end{dukazin}
        \end{dusledek}

% 9. 4. 2021

\section{Konečná tělesa}
    \begin{pozorovani}[Konečná tělesa]
        Nechť $®T = ®Z_p[\alpha]/(m)$, kde $p$ je prvočíslo, $m$ ireducibilní polynom v $®Z_p[\alpha]$, $\deg m = k$. Potom ®T je těleso s $p^k$ prvky. Značíme ho $®F_{p^k}$ (podle dalšího pozorování je jediné této mohutnosti).
    \end{pozorovani}

    \begin{pozorovani}[Vlastnosti konečných těles]
        \ 
        \begin{itemize}
            \item $\forall k\ \forall p$ prvočíslo $\exists$ ireducibilní polynom stupně $k$ v $®Z_p[\alpha]$ $\implies \exists$ konečné těleso velikosti $p^n$.
            \item Každé konečné těleso lze takto zkonstruovat.
            \item Na volbě $m$ (daného stupně) nezáleží.
        \end{itemize}

        \begin{dukazin}
            Bez důkazu.
        \end{dukazin}
    \end{pozorovani}

    \begin{poznamka}
        Díky pozorování, že nad konečným tělesem je každá funkce polynomiální a že posloupnost jedniček je vlastně $®F_{2^k}$, stačí v kryptografii zkoumat jen polynomy.

        Navíc násobení na tomto tělese používá symetrická šifra AES (advanced encryption standard), která počítá s maticemi $4 \times 4$ nad $®F_{256}$.
    \end{poznamka}

    \begin{poznamka}
        Další využití je v konečné geometrii, např. eliptické křivky jsou Diofantické rovnice tvaru $y^2 = x^3 + ax + b$ nad $®F_{p^k}$ (řešení tvoří grupu a dělá se s tím něco jako v RSA).
    \end{poznamka}

    \subsection{Sdílení tajemství}
        \begin{definice}
            $(k, n)$-schéma sdílení tajemství je situace, kdy se $n$ lidí dělí o tajemství a k odhalení je potřeba alespoň (libovolných) $k$ z nich.
        \end{definice}

        \begin{definice}[Tajemství]
            Za tajemství budeme uvažovat posloupnost $0$ a $1$, na kterou se budeme dívat v $®Z_2^m$ nebo $®F_{2^m}$.
        \end{definice}

        \begin{poznamka}
            Pro $k = n$ se $(k, n)$-schéma nazývá maskování hodnot: Pro každého člověka vyberu hodnotu $a_i \in T$ a zveřejním hodnotu $c = t + \sum_{i=1}^n a_i$. ($t$ je tajemství.)
        \end{poznamka}

        \begin{definice}[Shamirův protokol]
                Vlastník zvolí polynom $f \in ®T[x]$, $\deg f < k$ tak, že $f(0) = t$. Vyberu $n$ po dvou různých prvků $0 ≠ a_1, …, a_n \in ®T$, které se zveřejní, a jednotlivým účastníkům se dá $f(a_1), …, f(a_n)$.

            Když se potká $k$ lidí, tak mají $k$ hodnot polynomu, tedy mohou polynom interpolovat a zjistit konstantní člen, tj. $f(0) = t$.
        \end{definice}

\section{Symetrické polynomy}
    \begin{definice}
            $R$ komutativní okruh. Polynom $f \in R[x_1, …, x_n]$ je symetrický, pokud po libovolném permutování proměnných se $f$ nezmění. (formálně: $f(x_1, …, x_n) = f(x_{\pi(1)}, …, x_{\pi(n)})$ pro každou permutaci $\pi \in S_n$.)
    \end{definice}

    \begin{tvrzeni}[Viétovy vztahy]
        $®T$ těleso, $f = \sum a_i x^i \in ®T[x]$, $\deg f = n ≥ 1$. Ať $f = a_n(x - u_1)·…·(x - u_n)$ v nějakém nadtělese $®S ≥ ®T$. Pak
        $$ \frac{a_{n-i}}{a_n} = (-1)^i s_i(u_1, …, u_n) = \sum_{j_1 < j_2 < … < j_i}x_{j_1}·…·x_{j_i}. $$

        \begin{dukazin}
            Berme $g = a^{-1}_n f$. Z rovnosti
            $$ (y - x_1)·…·(y - x_n) = y^n - s_1y^{n-1} + … + (-1)^ns_n $$
            dostaneme
            $$ g = \sum \frac{a_i}{a_n}x^i = (x - u_1)·…·(x - u_n) = x^n + \sum_{i=1}^n (-1)^i s_i(u_1, …, u_n)x^{n-i}. $$
            Porovnáním koeficientů dostaneme chtěnou rovnost.
        \end{dukazin}
    \end{tvrzeni}

    \begin{veta}[Základní věta o symetrických polynomech]
        Buď $R$ obor, $f \in R[x_1, …, x_n]$ symetrický polynom. Pak $\exists! g \in R[z_1, …, z_n]$ tak, že $f = g(s_1, …, s_n)$.

        \begin{dukazin}
            Později.
        \end{dukazin}
    \end{veta}

    \begin{definice}[Term]
        Term v proměnných $x_1, …, x_n$ je výraz $x_1^{k_1}x_2^{k_2}…x_n^{k_n}, k_i \in ®N_0$.
    \end{definice}

    \begin{definice}[Uspořádání termů]
        Relaci $<$ na termech definujeme jako $x_1^{k_1}…x_n^{k_n} < x_1^{l_1}…x_n^{l_n}$, pokud $\exists i ≥ 0$ tak, že $k_1 = l_1, k_2=l_2, …, k_i = l_i, k_{i + 1} < l_{i+1}$.

        Definujeme $t ≤ s$, pokud $t = s \lor t < s$.
    \end{definice}

    \begin{lemma}
        Relace $≤$ má vlastnosti: 1) Je to lineární uspořádání. 2) Pro libovolné termy $t_1 > t_2$, $s_1 > s_2$ platí $t_1s_1 > t_2s_2$. 3) Neexistuje $∞$ klesající řetězec termů $t_1 > t_2 > …$

        \begin{dukazin}
            Domácí cvičení.
        \end{dukazin}
    \end{lemma}

    \begin{definice}[Vedoucí člen polynomu]
        $R$ obor, $f \in R[x_1, …, x_n]$. Vedoucí člen $f$ je ten člen, který má největší term. Značí se $l(f)$.
    \end{definice}

    \begin{lemma}
        $R$ obor, $f, g \in R[x_1, …, x_n]$. Pak 1) $l(fg) = l(f)·l(g)$. 2) Je-li $f$ symetrický a $l(f) = a·x_1^{k_1}…x_n^{k_n}$, potom $k_1 ≥ k_2 ≥ … ≥ k_n$.
        \begin{dukazin}
            1) $l(f)$, $l(g)$ jsou největší členy v $f$, $g$. Podle předchozího lemmatu víme, že $>$ se zachovává násobením $\implies l(f)·l(g)$ je největší ze všech členů v $fg$. Navíc $R$ je obor $\implies$ koeficient v $l(f)·l(g)$ není nulový.

            2) Kdyby $k_i < k_j$ pro $i < j$, mohli bychom prohodit proměnné $x_i, x_j$. Ze symetrie $f$ je $a·x_1^{k_1}…x_i^{k_j}…x_j^{k_i}…x_n^{k_n}$ je taktéž v $f$, ale je větší než $l(f)$, což je spor.
        \end{dukazin}
    \end{lemma}

% 14. 4. 2021    

    \begin{lemma}
        $k_1≥k_2≥…≥k_n$ nezáporná celá. Pak $\exists!$ ($l_1, …, l_n$) nezáporné celé tak, že $l(s_1^{l_1}…s_n^{l_n}) = x_1^{k_1}·…·x_n^{k_n}$.

        \begin{dukazin}
            $$ l(s_1^{l_1}·…·s_n^{l_n}) = l(s_1)^{l_1}·…·l(s_n)^{l_n} = x_1^{l_1}·(x_1x_2)^{l_2}·…·(x_1x_2…x_n)^{l_n} = x_1^{l_1 + l_2 + … + l_n}·…·x_n^{l_n}. $$ 
            Tedy řeším systém $l_1 + … l_n = k_1$, $l_2+…+l_n = k_2$, …, $l_n=k_n$, tj. $l+n = k_n ≥ 0$, $l_i = k_i - k_{i+1} ≥ 0$.
        \end{dukazin}
    \end{lemma}

    \begin{definice}[Gaussův algoritmus]
        $R$ obor, vstup $f \in R[x_1, …, x_n]$ symetrický, výstup $g \in R[z_1, …, z_n]$ tak, že $g(s_1, …, s_n) = f$.
        
        $f_1 = f$, $g_1 = 0$.

        $i=1, 2, 3, …$: dělej: Najdi $l_1, …, l_n$ tak, že $l(f_i) = c·l(s_1^{l_1}·…·s_n^{l_n})$ pro nějaké $c \in R$ podle předchozího lemmatu. $f_{i+1} = f_i - c·s_1^{l_1}·…·s_n^{l_n}, g_{i+1} = g_i + c·z_1^{l_1}·…·z_n^{l_n}$. Pokud je $f_{i+1}$ konstantní, zastavím se a vrátím $g_{i+1} + f_{i+1}$.

        \begin{dukazin}
            Ověříme, že $f_i$ je symetrický polynom -- zřejmé z definice $f_i$. $g_i \in R[z_1, …, z_n]$ -- jasné z definice $g_i$. $f_i + g_i(s_1, …, s_n) = f$ -- vidíme, nebo ověříme indukcí. A skončí, jelikož zmenšujeme vedoucí člen a neexistuje nekonečná klesající posloupnost.
        \end{dukazin}
    \end{definice}

    \begin{dukazin}[Základní věta o symetrických polynomech]
        Existenci dokazuje Gaussův algoritmus. Jednoznačnost: Ať $f = g_1 (s_1, …, s_n) = g_2(s_1, …, s_n)$, $g_1 ≠ g_2$. $g = g_1 - g_2 = \sum a_it_i$, kde $t_i$ jsou po dvou různé jednotlivé termy (v proměnných $z_i$), $a_i≠0$. $t_i(s_1, …, s_n)$ mají různé vedoucí členy podle lemmatu výše. Vezměme lexikograficky největší z vedoucích členů $t_i(s_1, …, s_n)$. Ten je tedy striktně větší než ostatní, tedy $\sum a_it_i(s_1, …, s_n) ≠ 0$, tudíž $0 = g(s_1, …, s_n) - g_2(s_1, …, s_n)$.
    \end{dukazin}

    \begin{dusledek}[Hodnota symetrického polynomu na kořenech]
        ®T těleso, $f \in T[x]$, $\deg f ≥ 1$. Buď $®U ≥ ®T$ nadtěleso, kde $f || (x - u_1)·…·(x-u_n)$. $\forall$ symetrický polynom $s \in T[x_1, …, x_n$ platí: $s(u_1, …, u_n) \in ®T$.

        \begin{dukazin}
                $f = \sum a_ix^i$. Viétovy vztahy $s_i(u_1, …, u_n) = (-1)^i\frac{a_{n-i}}{a_n} \in ®T$. Z předchozí věty $\exists g \in ®T[z_1, …, z_n$ tak, že $f = g(s_1, …, s_n)$ $\implies$ $f(u_1, …, u_n) = g(s_1(u_1, …, u_n), …, s_n(u_1, …, u_n)) \in ®T$.
        \end{dukazin}
    \end{dusledek}

\section{Základní věta algebry}
    \begin{veta}[Základní věta algebry]
        Každý komplexní polynom stupně $≥1$ má kořen.

        \begin{dusledekin}
            $$ \forall f \in ®C[x], \deg f ≥ 1: f||(x-u_1)·…·(x-u_n). $$
        \end{dusledekin}

        \begin{dusledekin}
            Každý polynomiální zobrazení $®C \rightarrow ®C$ je na.
        \end{dusledekin}

        \begin{dukazin}[Jeden z mnoha, nejvíce algebraický]
            Lemma: předpokládejme, že každý reálný polynom stupně $≥1$ má (komplexní) kořen. Pak má každý komplexní polynom stupně $≥1$ nějaký kořen.

            Důkaz: $f \in ®C[x], \deg f ≥ 1, f = \sum a_ix^i.$ $\overline{f} = \sum\overline{a_i}x^i$. Uvažujme $g = f·\overline{f} = \sum_k\(\sum_{i+j = k} a_i\overline{a_j}\)x^k$. Ten má pro $i=j$ reálný koeficient a pro $i≠j$ má koeficienty $a_i\overline{a_j} + \overline{a_i}a_j \in ®R$. Buď $z \in ®C$ kořen $g$. Potom $f(z) = 0$ (OK) nebo $\overline{f}(z) = 0$ (tj. $f(\overline{z})=0$, OK).

            Lemma: Komplexní polynom stupně 2 má komplexní kořen. Důkaz $\frac{-b ± \sqrt{b^2 - 4a·c}}{2} \in ®C$. (Jediný zádrhel je odmocnina, ale existenci odmocniny z komplexního čísla ukážeme přes exponenciální tvar.)

            Lemma: Reálný polynom lichého stupně má kořen. Důkaz: Vynechán (věta o střední hodnotě a spojitost polynomů).

            Díky 1. lemmatu stačí, že $\forall f \in ®R[x], \deg f ≥ 1$, má kořen v ®C. $\deg f = n = 2^{k}m, m$ liché. Indukcí podle $k$: $k=0 \implies f$ má lichý stupeň, tedy tvrzení je splněno díky přdchozímu lemmatu.

            Ať $k ≥ 1$. Ať $S ≥ ®C$ je nadtěleso, ve kterém $f||(x-u_1)·…·(x-u_n)$ (díky větě z dřívějška). Chceme $\exists i: u_i \in ®C$. Trik. Vezmeme $a \in ®Z$ a definujeme $h_a = \prod_{i < j}(x - (u_i + v_j + a·u_i·u_j)) \in S[x]$. Chceme $h_a \in ®R[x]$. $\tilde h_a = \prod_{i<j}(x - (y_i + y_j + a·y_i·y_j)) \in (®Z[x])[y_1, …, y_n]$ je symetrický polynom v proměnných $y_1, …, y_n$ (s koeficienty ze $®Z[x]$). 

            Z věty výše $\exists g_a \in (®Z[x])[z_1, …, z_n]$ tak, že $\tilde h_a = g_a(s_1, …, s_n)$. Dosadíme $y_i = u_i$: $h_a = \tilde h_a(u_1, …, u_n) = g_a(s_1(u_1, …, u_n), …)$. Z viétových vztahů $s_1(u_1, …, u_n), …, s_n(u_1, …, u_n) \in ®R$. Tedy $h_a$ je polynom v $®R[x]$. $\deg h_a = \binom{n}{2} = 2^{k-1}·(m·(2^k·m - 1))$, takže má menší mocninu dvojky ve stupni, tedy aplikujeme IP. Proto má $h_a$ kořen v ®C, tudíž $\forall a \in ®Z\ \exists i < j: u_i + u_j + au_iu_j$, tedy nějaká dvojice $i, j$ se vyskytne nekonečněkrát ($a$ je nekonečně, dvojic je konečně). Stačí, že $\exists a≠b: u_i + u_j + au_iu_j \in ®C$ a $u_i + u_j + bu_iu_j \in ®C$, tudíž $(a-b)v_i·u_j \in ®C$ a $c=u_i + u_j \in ®C$. Tedy $u_i, u_j$ jsou kořeny $x^2 - cx + (u_iu_j)\in ®C[x]$, tedy podle 3. lemmatu existuje kořen $x \in ®C$, tj. $u_i \in ®C$ nebo $u_j \in ®C$.
        \end{dukazin}
    \end{veta}

\section{Grupy}
    \begin{definice}[Grupa, abelovská grupa]
        Grupa je čtveřice $(G, *, ', e)$, kde $G$ je množina (tzv. nosná), $*$ je binární operace na $G$, $'$ je unární operace ($a'$ je tzv. inverzní prvek k $a$) a $e \in G$ (tzv. jednotka) tak, že $\forall a, b, c \in G$:
        $$ a*(b*c) = (a*b)*c, \qquad a*e = e*a = a, \qquad a*a' = a'*a = e. $$

        Jestliže $\forall a, b \in G: a*b = b*a$, pak je grupa abelovská (čili komutativní).
    \end{definice}

    \begin{poznamka}
        Existují 2 zápisy: aditivní $(G, +, -, 0)$ (typicky abelovská) a multiplikativní $(G, ·, ^{-1}, 1)$.
    \end{poznamka}

    \begin{definice}[Podgrupa]
        Ať $(G,  *, ', e)$ je grupa, $H \subseteq G$ podmnožina. Pokud je $H$ uzavřené na operace, čili $e \in H, \forall a, b \in H: a*b \in H, a' \in H$, pak $H$ je podgrupa $G$. Značíme $H ≤ G$.

        $G, \{e\}$ jsou nevlastní podgrupy, ostatní jsou vlastní.
    \end{definice}

    \begin{priklady}[Symetrická grupa]
        $X$ neprázdná množina, ($S_X := \{\text{permutace na } X\}$, operace $\circ$ skládání, $^{-1}$ inverzní, $\id_X$) je symetrická grupa. Pokud je $X = \{1, 2, …, n\}$, pak značíme $S_n := S_X$.
    \end{priklady}

    \subsection{Vlastnosti permutací}
        \begin{definice}[Cyklus]
            Cyklus je posloupnost $a_1, …, a_k \in X$ navzájem různých prvků přičemž $\pi(a_1) = a_2, \pi(a_2) = a_3, …, \pi(a_k) = a_1$. Cyklus značíme $(a_1 a_2 … a_k)$.
        \end{definice}

        \begin{definice}[Rozklad na cykly]
            Rozklad na cykly je zápis $(a_{11} a_{12} … a_{1k_1})(a_{21} … a_{2k_2})…(a_{m1} … a_{mk_m})$, kde $a_{ij}$ jsou po dvou různé prvky. Cykly délky 1 typicky nepíšeme.

            Každá permutace na konečné množině jde (jednoznačně) rozložit na cykly.
        \end{definice}

        \begin{definice}[Transpozice]
            Transpozice je cyklus délky 2.

            Každá permutace ($X$ konečná, stejně jako kdekoliv dále) jde napsat jako složení transpozic.
        \end{definice}

        \begin{definice}[Sudá a lichá permutace, znaménko]
            Sudá permutace je ta permutace, kterou lze rozložit na sudý počet transpozic. Jinak je permutace lichá.

            Znaménko permutace $\sgn \pi = 1$ pokud je daná permutace sudá, jinak $\sgn \pi = -1$. $\sgn(\pi^{-1}) = \sgn \pi$. $\sgn \pi = (-1)^{n-m} = (-1)^{m_0}$, kde $m$ je počet cyklů a $m_0$ je počet sudých cyklů ($n$ počet prvků v množině).
        \end{definice}

        \begin{definice}[Konjugované]
            $\pi, \sigma \in S_n$ jsou konjugované, pokud $\exists \rho \in S_n: \sigma = \rho\circ\pi\circ\rho^{-1}$.
        \end{definice}

        \begin{tvrzeni}
            $\pi, \sigma$ jsou konjugované, právě když mají stejný počet cyklů každé délky.
            
            \begin{dukazin}
                Viz skripta.
            \end{dukazin}
        \end{tvrzeni}
        
        \begin{priklady}[Permutační grupy]
            Permutační grupy = podgrupy $S_n$: Alternující grupa $A_n ≤ S_n$ jsou všechny sudé permutace $n≥2$. Digedrální grupa $D_{2n} ≤ S_n$ jsou všechny symetrie pravidelného $n$-úhelníku.

            $|S_n| = n!$, $|A_n| = \frac{n!}{2}$, $|D_{2n}| = 2n$.
        \end{priklady}

        \begin{priklady}[Geometrické grupy]
            $D_{2n}$, $E_n$ (euklidovská grupa -- symetrie $®R^n$), symetrie projektivního prostoru.
        \end{priklady}

        \begin{priklady}[Maticové grupy]
            $GL_n(®T)$ je grupa regulárních matic $n \times n$ nad ®T, $SL_n(®T)$ je grupa podgrupa regulárních matic s $\det = 1$, $O_n(®T)$ je grupa ortogonálních matic, čili $A·A^T = I_n$.
        \end{priklady}

        \begin{priklady}[Okruhové grupy]
            $R$ okruh. $(R, +, -, 0)$ je aditivní grupa okruhu $R$ (je ablovská), pokud je navíc $R$ (komutativní) okruh s 1 a $R^*$ množina všech invertibilních prvků, pak $(R^*, ·, ^{-1}, 1)$ je multiplikativní grupa okruhu $R$.
        \end{priklady}

        \begin{priklady}[Komplexní jednotky]
            $\(\{z \in ®C\ |\ |z| = 1\}, ·, ^{-1}, 1\)$ a její podgrupy tzv. cyklotomické grupy $®C_n = \{\text{kořeny } x^n - 1\} = \{\zeta_n^j | j \in [n]\}$, kde $\zeta_n = e^{2\pi i/n}$.

            Priferova $p$-grupa $®C_{p^∞} = \bigcup_{k=1}^∞ ®C_{p^k}$.
        \end{priklady}

        \begin{definice}[Direktní součin grup]
            Direktní součin grup $(G_i, *_i, '_i, e_i)$, $i \in [n]$, je grupa $\prod G_i = G_1\times … \times G_n = \{(a_1, …, a_n) | a_i \in G_i\}$, kde operace $*, ', e$ jsou definovány „po složkách“:
            $$ (a_1, …, a_n)*(b_1, …, b_n) = (a_1*_1b_1, …, a_n*_nb_n), \qquad (a_1, …, a_n)' = (a_1^{'_1}, …), \qquad e = (e_1, …, e_n). $$ 

            Pro $G_1 = … = G_n = G$ jde o direktní mocniny $G^n$.
        \end{definice}

        \begin{tvrzeni}[Základní vlastnosti grup]
            $(G, *, ', e)$, $a, b, c \in G$:
            $$ a*c = b*c \implies a = b, $$
            $$ a*c = a \implies c = e, $$ 
            $$ (a')' = a, \qquad (a*b)' = b'*a'. $$
        \end{tvrzeni}

    \subsection{Mocniny a řád prvku}
        \begin{definice}[Mocnina prvku]
            $G$ grupa, $a \in G$, $n \in ®Z$.
            $$ a^n = \begin{cases} 1 & n = 0 \\ \underbrace{a·a·…·a}_{n\times} & n > 0 \\ \underbrace{a^{-1}·a^{-1}·…·a^{-1}}_{n\times} & n < 0 \end{cases}. $$ 
        \end{definice}

        \begin{tvrzeni}
            $G$ grupa, $a, b \in G$, $k, l \in ®Z$. Pak $a^{k+l} = a^k·a^l$, $a^{k·l} = (a^k)^l = (a^l)^k$. Pokud je navíc $G$ abelovská, potom $(ab)^k = a^kb^k$.

            \begin{dukazin}
                Pro $k, l > 0$ je to jasné: $a^{k+l} = \underbrace{a·a·…·a}_{k+l} = \underbrace{a·a·…·a}_{k}·\underbrace{a·a·…·a}_{l} = a^k·a^l$. Když $k = 0$ nebo $l = 0$, pak je to ještě jasnější. $k > 0, l < 0, k+l>0$ a podobně rozebereme každé zvlášť.

                Zbytek analogicky.
            \end{dukazin}
        \end{tvrzeni}

        \begin{definice}[Řád grupy]
            Řád grupy $G$ je počet prvků nosné množiny $G$ (tj. $|G|$), resp. $∞$.

            Řád prvku $a \in G$ je nejmenší $n \in ®N$ tak, že $a^n = 1$ (pokud neexistuje, pak $∞$). Značíme $\ord(a)$.
        \end{definice}

        \begin{tvrzeni}[Řád permutace]
            Řád permutace $\pi \in S_n$ je nejmenší společný násobek délek cyklů $\pi$.

            \begin{dukazin}
                Cyklus délky $k$ má zřejmě řád $k$. Pro disjunktní cykly $C_1, …, C_m$ máme $\pi = (C_1 \circ … \circ C_m)^k$. Protože jsou disjunktní, tak je to to samé jako $C_1^k \circ … \circ C_m^k$. Tedy $\pi^k = \id \Leftrightarrow C_1^k = \id, …, C_m^k = \id$ $\Leftrightarrow k$ je násobek délek všech cyklů. Tedy $\ord \pi = \min k = \NSN(…)$.
            \end{dukazin}
        \end{tvrzeni}

\section{Podgrupy}
    \begin{lemma}
        Průnik podgrup je podgrupa.

        \begin{dukazin}
            $G$ grupa, $H_i ≤ G$ pro $i \in I$. $H = \bigcup_{i \in I} H_i \subseteq G$. $H$ je uzavřené na operace: jednoduché ověřit.
        \end{dukazin}
    \end{lemma}

    \begin{definice}
        Buď $X \subseteq G$ podmnožina $G$. Podgrupa generovaná množinou $X$ je nejmenší (vzhledem k inkluzi) podgrupa $G$, která obsahuje $X$. Značíme $\<X\>_G$.

        \begin{dukazin}
            $\<X\>_G = \bigcap \{H ≤ G | X \subseteq H\}$.
        \end{dukazin}
    \end{definice}

% 21. 4. 2021 Písemka

% 23. 4. 2021

    \begin{tvrzeni}
        $G$ grupa $\O ≠ X \subseteq G$. Pak
        $$ \<X\>_G = [a_1^{k_1}·…·a_n^{k_n} | n \in ®N_0, a_1, …, a_n \in X, k_1, …, k_n \in ®Z]. $$ 
    
        \begin{upozorneni}
            $a_i$ nemusí být různé.

            Vyjádření typicky není jednoznačné.
        \end{upozorneni}

        \begin{dukazin}
            $M = $ množina napravo. Chceme $M$ je podgrupa, $M \supseteq X$, $M$ je nejmenší podgrupa.

            1. $(a_1^{k_1}·…·a_n^{k_n})·(b_1^{l_1}·…·b_m^{k_m}) \in M$. $(a_1^{k_1}·…·a_n^{k_n})^{-1} = a_n^{-k_n}·…·a_1^{-k_1} \in M$. $1 \in M$ (pro $n = 1$ nebo $k_i = 0$).

            2. $a \in X \implies a^1 \in M$. 3. Buď $H ≤ G$, $H \supseteq X$. $\forall a \in X: a \in H \implies \forall k \in ®Z: a^k \in H$. $a_1, …, a_n \in X \implies a_1^{k_1}, … \in H \implies a_1^{k_1}·… \in H \implies M \subseteq H$.
        \end{dukazin}
    \end{tvrzeni}

    \begin{poznamka}[Značení]
        $$ \<\{b_1, …, b_m\}\>_G ≡ \<b_1, …, b_m\>_G. $$
    \end{poznamka}

    \begin{dusledek}
        $G$ grupa $a \in G$. $\<a\>_G = \{a^k | k \in ®Z\}$.
    \end{dusledek}

    \begin{dusledek}
        $G$ abelovská grupa, $a_1, …, u_n \in G$. $\<u_1, …, u_n\>_G = \{u_1^{k_1}·…·u_n^{k_n} | k_i \in ®Z\}$.
    \end{dusledek}

    \begin{tvrzeni}[Generátory permutačních grup]
        1. Grupa $S_n$ je generovaná množinou všech transpozic (viz Lingebra).

        2. Grupa $A_n$ je generována množinou všech trojcyklů. 
    \end{tvrzeni}

    \begin{tvrzeni}[Řád prvku a řád podgrupy]
        $G$ je grupa, $a \in G$. Pak $\ord a = |\<a\>_G|$.

        \begin{dukazin}
            Z minulého tvrzení $\<a\>_G = \{a^k | k \in ®Z\}$. $a^i = a^j \Leftrightarrow a^{i-j} = 1 \Leftrightarrow \ord a | i - j$ nebo $i-j = 0$ pro $\ord a = ∞$. Pro $\ord a = ∞$ je tvrzení jasné, pro $\ord a = n < ∞$ víme, že $a^i = a^j \Leftrightarrow n|i-j \Leftrightarrow i ≡ j \mod n$. Pak $\<a\>_G = \{a^0, a^1, …, a^{n-1}\}$, tj. $|\<a\>_G| = n = \ord a$.
        \end{dukazin}
    \end{tvrzeni}

    \subsection{Lagrangeova věta}
        \begin{veta}
            $G$ grupa, $H ≤ G$. Pak $|H|$ dělí $|G|$.
        \end{veta}

        \begin{definice}[Rozkladové třídy, transverzála, index podgrupy]
            $H ≤ G$. Množiny $aH = \{a·b | b \in H\}$ pro $a \in G$ nazýváme rozkladové třídy podgrupy $H$.

            Podmnožina $T \subseteq G$ s vlastností $|T \cap aH| = 1$ pro $\forall a \in G$ se nazývá transverzála rozkladu $G$ podle $H$.

            Počet různých rozkladových tříd se nazývá index podgrupy $H$ v grupě $G$ a značí se $[G:H] = |\{aH | a \in G\}|$.
        \end{definice}

        \begin{poznamka}
            Někdy se těmto definicím říká levá transverzála a levá rozkladová třída. Pravé by se definovaly symetricky. Index je shodný (viz dále).
        \end{poznamka}

        \begin{lemma}[Disjunktnost rozkladových tříd]
            $H ≤ G$. $\forall a, b \in G: aH = bH$ nebo $aH \cap bH = \O$.

            \begin{dukazin}
                Ať $aH \cap bH ≠ \O$. Buď $c \in aH \cap bH$. Tedy $c = ah_1 = bh_2$ pro nějaké $h_1, h_2 \in H$. Vezmeme $ah \in aH$ a máme $ah = ch_1^{-1}h = bh_2h_1^{-1}h \in bH$. Symetricky $bh \in aH$, tedy $aH = bH$.
            \end{dukazin}
        \end{lemma}

        \begin{dusledek}
            Mohutnost transverzály se rovná mohutnosti množiny rozkladových tříd.
        \end{dusledek}

        \begin{lemma}[Velikost rozkladových tříd]
            $H ≤ G$. Pro $\forall a \in G: |aH| = H$.

            \begin{dukazin}
                Zobrazení $f: G \rightarrow G$, $x \mapsto ax$ je prosté: $ax = ay \implies x = y$. $f(H) = aH$, tedy $f|_H$ je bijekce mezi $H$ a $aH$. Tedy $|H| = |aH|$.
            \end{dukazin}
        \end{lemma}

        \begin{veta}[Lagrangeova věta podruhé]
            $H ≤ G$. Pak $|G| = |H|·[G:H]$.

            \begin{dukazin}
                Buď $T$ nějaká transverzála. Pak $G = \bigcup_{a \in T} aH$ je disjunktní sjednocení podle lemmatu výše. Tedy
                $$ |G| = \sum_{a \in T} |aH| = \sum_{a \in T} |H| = |H|·|T| = |H|·[G:H]. $$ 
            \end{dukazin}
        \end{veta}

        \begin{dusledek}
            $G$ grupa, $a \in G$. Pak $\ord a | |G|$.
        \end{dusledek}
        
        \begin{poznamka}
            Z Lagrangeovy věty plyne Eulerova věta.
        \end{poznamka}

        \begin{tvrzeni}[Rovnost rozkladových tříd]
            $H ≤ G$. $\forall a, b \in G$ platí $aH = bH \Leftrightarrow a^{-1}b \in H$. $Ha = Hb \Leftrightarrow ab^{-1} \in H$.

            \begin{dukazin}
                $\implies$: $aH = bH \ni b = b·1$, čili $b \in aH$, tedy $b = ah$ pro nějaké $h \in H$. Pak $a^{-1}b = h \in H$.

                $\Leftarrow$: $a^{-1}b = h$ pro nějaké $h \in H$ $\implies$ $b·1 = a·h \in aH \cap bH \implies aH = bH$.

                Analogicky pro pravé.
            \end{dukazin}
        \end{tvrzeni}

        \begin{dusledek}
            Levých a pravých rozkladových tříd je stejně, neboť zobrazení $aH \rightarrow Ha^{-1}$ je bijekce.
        \end{dusledek}

    \subsection{Loydova patnácka (nebude se zkoušet)}
        Místo prázdného políčka uvažujme 16. Každý stav hry lze popsat permutací $\pi \in S_{16}$. Tah je přechod z $\pi$ do $\pi\circ(ij)$, kde $\pi(i) = 16$.

        \begin{pozorovani}
            Každý tah změní znaménko permutace.
        \end{pozorovani}

        \begin{definice}[Invariant pro L. 15]
            $I(\pi) = \sgn(\pi)·(-1)^{d(\pi)}$, kde $d(pi)$ je Newyorská vzdálenost prázdného pole od pravého dolního rohu.

            \begin{dukazin}[Invariant]
                Tah změní $\sgn \pi$ i $d(\pi)$.
            \end{dukazin}
        \end{definice}

        \begin{veta}
            Loydova 15 je řešitelná $\Leftrightarrow$ $I(\pi) = 1$.

            \begin{dukazin}
                $\implies$: Zřejmé (z toho, že je $I(\pi)$ invariant a na konci má být $I = 1$). $\implies$: BÚNO $\pi(16) = 16$. Dívejme se tedy na $\pi \in A_{15}$. Potom pomocí $\pi \circ (5, 6, 7, 8, 11, 10, 9)$ (dolní obdélníček $2 \times 4$), $\pi \circ (1, 2, 3, 4, 7, 6, 5)$ (prostřední obdélníček), $\pi \circ (9, 10, 11, 12, 15, 14, 13)$ (dolní obdélníček). Toto už generuje $A_{15}$ (cvičení).
            \end{dukazin}
        \end{veta}

% 28. 4. 2021

\section{Grupové homomorfismy}         
    \subsection{Základní vlastnosti}
        \begin{definice}
            $G = (G, ·, ^{-1}, 1), H = (H, *, ', e)$  jsou grupy. Zobrazení $\phi: G \rightarrow H$ je homomorfismus grup, pokud $\forall a, b \in G: \phi(a·b) = \phi(a)*\phi(b)$, $\phi(a^{-1}) = \phi(a)'$, $\phi(1) = e$.
        \end{definice}
        
        \begin{lemma}
            $G, H$ jsou grupy jako výše, $\phi: G \rightarrow H$ zobrazení. Pak $\phi$ je homomorfismus $\Leftrightarrow$ $\phi(a·b) = \phi(a)*\phi(b)$.
            
            \begin{dukazin}
                $\Leftarrow$: $\phi(1) \stackrel{?}{=} e$: $e*\phi(1) = \phi(1) = \phi(1·1) = \phi(1)*\phi(1)$. Zkrátíme $\phi(1)$ na obou stranách $\implies$ $e = \phi(1)$.
                
                $\phi(a)*\phi(a)' = e = \phi(1)' = \phi(a·a^{-1}) = \phi(a)*\phi(a^{-1})$. Zkrátíme $\phi(a)$ a dostáváme $\phi(a^{-1}) = \phi(a)'$.
                
                $\implies$: triviální
            \end{dukazin}
        \end{lemma}
        
        \begin{definice}[Obraz a jádro homomorfismu]
            $\phi: G \rightarrow H$ homomorfismus. Obraz $\phi$ je jeho obor hodnot, čili množina:
            $$ \im(\phi) = \{\phi(a) | a \in G\} \subseteq H. $$
            
            Jádro $\phi$ je množina
            $$ \Ker(\phi) = \{a \in G | \phi(a) = e\}\subseteq G. $$
        \end{definice}
        
        \begin{tvrzeni}[Jádro a obraz jsou podgrupy]
            $\phi: G \rightarrow H$ homomorfismus grup. Pak $\im(\phi) ≤ H$ a $\Ker(\phi) ≤ G$.
            
            \begin{dukazin}
                $e = \phi(1) \implies e \in \im(\phi)$. Pokud $\phi(a), \phi(b) \in \im(\phi)$, pak $\phi(a)'=\phi(a^{-1})' \in im(\phi)$ a $\phi(a)*\phi(b) = \phi(a·b) \in \im(\phi)$.
                
                $\phi(1) = e \implies 1 \in \Ker(\phi)$. Pokud $a, b \in \Ker \phi$, pak $\phi(a^{-1}) = \phi(a)' = e' = e \implies a^{-1} \in \Ker(\phi)$. $a·b$ podobně.
            \end{dukazin}
        \end{tvrzeni}

        \begin{tvrzeni}
            $\phi: G \rightarrow H$ homomorfismus grup. Pak $\phi$ je prosté $\Leftrightarrow$ $\Ker(\phi) = \{1\}$.
            
            \begin{dukazin}
                $\implies$: Pro $a≠0$, prostota $\implies$ $\phi(a) ≠ \phi(1) = e$ $\implies$ $a \notin \Ker(\phi)$.
                
                $\Leftarrow$: $\phi(a)=\phi(b) \Leftrightarrow \phi(a·b^{-1}) = e \Leftrightarrow a·b^{-1} \in \Ker(\phi)$. Ale $\Ker(\phi) = \{1\}$, tedy $a·b^{-1} = 1 \implies a=b$. Tedy $\phi$ je prosté.
            \end{dukazin}
        \end{tvrzeni}
        
        \begin{pozorovani}[Bez důkazu]
            Homomorfismus je určený svými hodnotami na generátorech. Na rozdíl od LA však nelze volit libovolně:
        \end{pozorovani}
        
        \begin{tvrzeni}[Řád prvku a jeho obrazu]
            $\phi: G \rightarrow H$ homomorfismus. Pro $a \in G$ platí: $\ord_H(\phi(a)) | \ord_G(a)$. ($x|∞$). Je-li $\phi$ prosté, pak $\ord(\phi(a)) = \ord(a)$.
            
            \begin{dukazin}
                $\ord(a) = ∞$ zřejmé. Ať $\ord(a) = n \in ®N$. Pak $\phi(a)^n = \phi(a^n) = \phi(1) = e$. Tedy $\ord(\phi(a)) | n$ (neboť vydělím $n$ číslem $\ord(\phi(a))$ se zbytkem a dostanu, že zbytek $= 0$).
                
                Prostota: civčení.            
            \end{dukazin}
        \end{tvrzeni}
        
        \begin{tvrzeni}
            Mějme grupy $G, H, K$ a homomorfismy $\phi: G \rightarrow H, \psi: H \rightarrow K$. Pak $\psi \circ \phi$ je homomorfismus $G \rightarrow K$. Je-li $\phi$ bijekce, pak $\phi^{-1}: H \rightarrow G$ je homomorfismus.
            
            \begin{dukazin}
                Viz skripta.
            \end{dukazin} 
        \end{tvrzeni}
        
    \subsection{Izomorfismus}
        \begin{definice}[Izomorfismus]
            Bijektivní homomorfismus $\phi: G \rightarrow H$ je izomorfismus.
        \end{definice}
        
        \begin{dusledek}
            Inverze k izomorfismu je izomorfismus.
        \end{dusledek}
        
        \begin{dukazin}
            Grupy $G, H$ jsou izomorfní, pokud existuje izomorfismus $\phi: G \rightarrow H$. Značíme $G \simeq H$.
        \end{dukazin}
        
        \begin{pozorovani}
            Relace „být izomorfní“ je ekvivalence na třídě všech grup.
            
            \begin{dukazin}
                Reflexivní (identita je izomorfismus), tranzitivní (tvrzení výše o homomorfismu a skládání bijekcí), reflexivní (dusledek výše).
            \end{dukazin}
        \end{pozorovani}
        
        \begin{tvrzeni}[Algebraická verze ČZV]
            Mějme $m_1, …, m_n$ po dvou nesoudělná přirozená čísla, $M = m_1·…·m_n$. Zobrazení $\phi: ®Z_M \rightarrow Z_{m_1} \times … \times ®Z_{m_n}$, $a \mapsto (a \mod m_1, …, a \mod m_n)$ je izomorfismus okruhů. Restrikce $\phi|®Z_M^*$ je izomorfismus multiplikativních grup $®Z_M^* \rightarrow ®Z_{m_1}^* \times … \times ®Z_{m_n}^*$.
            
            \begin{dukazin}
                $\phi: ®Z_M \rightarrow ®Z_{m_1} \times … \times ®Z_{m_n}$ je bijekce, viz ČZV. Snadno se ověří, že je to homomorfismus pro operace $+$ i $·$.
            \end{dukazin}
        \end{tvrzeni}

    \subsection{Neizomorfismus}
        \begin{tvrzeni}
            Buď $\phi: G \rightarrow H$ surjektivní homomorfismus. Pokud $G = \<X\>$, pak $H = \<\phi(X)\>$, kde $\phi(X) = \{\phi(a) | a \in X\}$.

            \begin{dukazin}
                Cvičení / skripta.
            \end{dukazin}
        \end{tvrzeni}

        \begin{upozorneni}
            Na rozdíl od VP generující množiny mohou být různě velké, např. $®Z = \<1\> = \<2\>$.
        \end{upozorneni}

% 30. 4. 2021

\section{Cyklické grupy}
    \subsection{Základy}
        \begin{definice}
            Grupa $G$ je cyklická, pokud je generovaná 1 prvkem, čili $G = \<a\>_G$ pro nějaké $a \in G$.
        \end{definice}

        \begin{dusledek}[Různých předchozích tvrzení]
            $\<a\> = \{a^k | k \in ®Z\} \implies \<a\>$ je abelovská (násobení je akorát sčítání mocnin).
            
            $\ord a = |\<a\>|$.
        \end{dusledek}

        \begin{veta}[Klasifikace cyklických grup]
            $G$ cyklická grupa. 1) Je-li $G$ nekonečná, pak je izomorfní $(®Z, +, -, 0)$. 2) Je-li $G$ konečná řádu $n$, pak je izomorfní $(®Z_n, +, -, 0)$.

            \begin{dukazin}
                1) $|G| = ∞$. Pak $G = \{…, a^{-1}, a^0, a^1, a^2, …\}$ (po dvou různé mocniny). Definujeme izomorfismus $\phi: G \rightarrow ®Z$, $a^k \mapsto k$. Už víme, že je bijekce. Homomorfismus je to, protože $k + l = \phi(a^k) + \phi(a^l) = \phi(a^k·a^l) = \phi(a^{k+l}) = k+l$.

                2) $|G| = n$. $\phi:G = \{1, a, …, a^{n-1}\} \rightarrow ®Z_n$, $a^k \mapsto k$. To že je $\phi$ bijekce už víme, homomorfismus: $k + l \mod n = \phi(a^k) + \phi(a^l) \mod n = \phi(a^k·a^l) = \phi(a^{k+l \mod n}) = k+l \mod n$.
            \end{dukazin}
        \end{veta}

        \begin{tvrzeni}
            Každá podgrupa cyklické grupy je cyklická.

            \begin{dukazin}
                $H ≤ G = \<a\>$. 1) $H = \{1\}$, pak je generována 1. 2) $H$ obsahuje prvek $a^l$ pro $l≠0$. Tedy $a^{l'} \in H$ pro $l' > 0$. Buď $k > 0$ nejmenší tak, že $a^k \in H$. Pak $H = \<a^k\>$. Buď $a^n \in H$, vydělíme se zbytkem: $n = ki + j$, $0 ≤ j < k$. Pak ale $a^k > a^j = a^{ki + j}·a^{k(-i)} \in H$. Tudíž $a^k | a^n$ (když $a^j = 1$) nebo \lightning. Tedy $H \subseteq \<a^k\>$. A triviálně $\<a^k\> \subseteq H$.
            \end{dukazin}
        \end{tvrzeni}

        \begin{tvrzeni}[Generátory podgrup]
            $G = \<a\>$. 1) $\<a^k, a^l\> = \<a^{\NSD(k, l)}\>$. 2) Je-li $|G| = n \in ®N$, pak $\<a^k\> = \<a^{\NSD(k, n)}\>$.

            \begin{dukazin}
                1) $\subseteq$: jasný, neboť $\NSD(k, l)|k,l$. $\supseteq$: použijeme Bézoutovu rovnost, $\NSD(k, l) = kx + ly$. Tedy $a^{\NSD(k, l)} = \(a^k\)^x·\(a^l\)^y \in \<a^k, a^l\>$.

                2) Volíme v 1) $l = n$. Víme, že $a^n = 1$, tedy
                $$ \<a^{\NSD(k, n)}\> = \<a^k, a^n\> = \<a^k, 1\> = \<a^k\>. $$ 
            \end{dukazin}
        \end{tvrzeni}

        \begin{tvrzeni}[Generátory cyklických grup]
            $G = \<a\>$. 1) $|G| = ∞$, pak generátory jsou právě $a^1, a^{-1}$. 2) $|G| = n$, pak generátory jsou právě $a^k$, kde $k \in \{1, …, n\}$ a $\NSD(k, n) = 1$.

            \begin{dukazin}
                1) jasné (viz skripta). 2) Z tvrzení výše $\<a^k\> = \<a^{\NSD(k, n)}\> = H$, $H = G \Leftrightarrow \NSD(k, n) = 1$. $\implies$: Kdyby $d = \NSD(k, n) > 1$, pak $H = \<c^d\> = \{1, a^d, …, a^{d(\frac{n}{d} - 1)}\} ≠ G$. $\Leftarrow$: triviální.
            \end{dukazin}
        \end{tvrzeni}

        \begin{tvrzeni}[Řády prvků]
            Cyklická grupa konečného řádu $n$ obsahuje právě $\phi(d)$ prvků řádu $d|n$. (A 0 řádu $d\nmid n$.)

            \begin{dukazin}
                    $G$ cyklická, $|G| = n$. Každý prvek řádu $d|n$ je generátor cyklické podgrupy řádu $d$. Víme, že taková podgrupa existuje v $G$ právě 1, neboť podle tvrzení výše jsou všechny podgrupy tvaru $\<a^k\>$ pro $k|n$ a taková podgrupa má řád $n/k$. Tedy $b$ je generátor $\<a^{n/d}\>$. Ta má podle předchozího tvrzení právě $\phi(d)$ generátorů.
            \end{dukazin}
        \end{tvrzeni}

        \begin{tvrzeni}
            Pro $n \in ®N: \sum_{d|n} \phi(d) = n$.

            \begin{dukazin}
                Spočteme $|®Z_N|$ dvěma způsoby: 1) $|®Z_n| = n$. 2) $®Z_n = \bigcup_{d|n}\{b \in ®Z_n | \ord b = d\}$. Tj. $|®Z_n| = \sum_{d|n}\phi(d)$.
            \end{dukazin}
        \end{tvrzeni}

    \subsection{Multiplikativní grupy konečných těles}
        \begin{lemma}
            $G$ konečná grupa taková, že $\forall k$ grupa $G$ obsahuje nejvýše $k$ prvků $a$: $a^k = 1$. Pak $G$ je cyklická.

            \begin{dukazin}
                $n = |G|$. $U_k =$ počet prvků řádu $k$ v $G$. Lagrange $\implies$ $U_k = 0$ pro $k \nmid n$. Spočtu prvky $G$ podle řádů (jako v předchozím tvrzení): $n = \sum_{k|n} U_k$. Buď $a$ prvek řádu $k$. $\<a\>$ je cyklická řádu $k$ a všechny prvky $b \in \<a\>$ splňují $b^k = 1$. $\<a\>$ obsahuje $k$ takových prvků.

                Předpoklad: $G$ obsahuje nejvýše $k$ prvků $c$ tak, že $c^k = 1$. Tedy $c \in \<a\>$. Speciálně každý prvek řádu $k$ leží v $\<a\>$ a je generátor $\<a\>$. Z tvrzení výše $\<a\>$ má $\phi(k)$ generátorů $\implies U_k = \phi(k)$. Tedy pro $k|n$ máme $U_k = 0$ nebo $\phi(k)$.

                $\sum_{k|n} \phi(k) = n ≤ \sum_{k|n} U_k ≤ \sum_{k|n} \phi(k)$. Tedy platí rovnost, tedy $U_k = \phi(k) \forall k | n$. Speciálně $U_N = \phi(n) > 1$ a existuje prvek řádu $n$ a ten generuje $G$.
            \end{dukazin}
        \end{lemma}

        \begin{veta}
            Buď ®T těleso a $G$ konečná podgrupa multiplikativní grupy $®T^*$. Pak $G$ je cyklická.

            \begin{dukazin}
                Pro předchozí lemma chci $G$ obsahuje nejvýše $k$ prvků tak, že $a^k = 1$. Ale každé takové $a$ je kořen polynomu $x^k - 1$. Nad tělesem ®T má $x^k - 1$ nejvýše $\deg(x^k - 1) = k$ kořenů.
            \end{dukazin}
        \end{veta}

        \begin{dusledek}
            Speciálně $|®T| < ∞$, pak $®T^*$ je cyklické. Její generátory se nazývají primitivní prvky.

            $®T = ®Z_p \implies ®Z_p^* = \{1, 2, …, p-1\}$ je cyklická, čili existuje $g$ tak, že $®Z_p^* = \{g^0, g^1, …, g^{p-2}\}$.
        \end{dusledek}

    \subsection{Diskrétní logaritmus a kryptografie}
        \begin{definice}[Diskrétní logaritmus a exponenciála]
            $|G| = n$, zobrazení $G = \<a\> = \{a_0, a_1, …, a^{n-1}\} \overset{\simeq}{\rightarrow} ®Z_n, a^k \mapsto k$ je tzv. diskrétní logaritmus. Diskrétní exponenciála je pak $®Z_n \overset{\simeq}{\rightarrow} G, k \mapsto a^k$.
        \end{definice}

        \begin{poznamka}
            Počítání diskrétní exponenciály je rychlé (rozdělení na mocniny 2), ale diskrétního logaritmu pomalé.
        \end{poznamka}

% 5. 5. 2021

\section{Působení grupy na množině}
    \subsection{Abstraktní grupa jako grupa permutací}
        \begin{definice}[Působení grupy na množině]
            Působení grupy $G$ na množině $X$ je libovolný homomorfismus $\pi: G \rightarrow S_X$. Hodnotu permutace $\pi(g)$ na prvku $x \in X$ často značíme $g(x)$.

            \begin{dusledekin}
                $\pi(1) = \id$, $\pi(g^{-1}) = \pi(g)^{-1}$, $(g·h)(x) = g(h(x))$.
            \end{dusledekin}
        \end{definice}

        \begin{veta}[Cayleyova representace]
            Každou grupu jde vnořit do nějaké symetrické grupy. (Čili existuje homomorfismus $\phi: G \rightarrow S_x$).

            \begin{dukazin}
                Dokonce vezmeme $X = G$. (Tedy pokud je $G$ konečná, pak vnořujeme do $S_{|G|}$). Pro $a \in G$ uvažujeme levou translaci $L_a: G \rightarrow G, x \mapsto a·x$. $L_a$ je zřejmě permutace na $G$, neboť můžeme zinvertovat $a$. Zobrazení $G \rightarrow S_G$, $a \mapsto L_a$ je homomorfismus (čili také $G$ působí na $X = G$), což snadno ověříme.
            \end{dukazin}
        \end{veta}

        \begin{definice}
            Relace tranzitivity $\sim$ na $X$: $x \sim y$ pokud $\exists g \in G: g(x) = y$.
        \end{definice}

        \begin{lemma}
            $\sim$ je ekvivalence na $X$.

            \begin{dukazin}
                Cvičení / skripta.
            \end{dukazin}
        \end{lemma}

        \begin{definice}[Orbita]
            Třídy ekvivalence $\sim$ se nazývají orbity.

            Orbitu obsahující $x \in X$ značíme $[x] = \{y \in X | y \sim x\} = \{g(c) | g \in G\}$.
        \end{definice}

        \begin{definice}[Pevný bod, stabilizátor]
            Bod $x \in X$ je pevný bod prvku $g \in G$, pokud $g(x) = x$.

            Množinu všech pevných bodů $g \in G$ značíme $X_g = \{x \in X | g(x) = x\}$.

            Stabilizátor prvku $x \in X$ je množina $G_x = \{g \in G | g(x) = x\}$.
        \end{definice}

        \begin{lemma}
            Stabilizátor $G_x$ je podgrupa $G$.

            \begin{dukazin}
                $1 \in G_x$, neboť $1(x) = x$. $g, h \in G_x$, čili $g(x) = x, h(x) = x$, pak $(g·h)x = g(h(x)) = g(x) = x \implies g·h \in G_x$, $g^{-1} = x \implies g^{-1} \in G_x$.
            \end{dukazin}
        \end{lemma}

        \begin{tvrzeni}[Velikost orbity vs. index stabilizátoru]
            $$ \forall x \in X: |[x]| = [G : G_x]. $$

            \begin{dukazin}
                Najdeme bijekci mezi $[x]$ a množinou $\{gG_x | g \in G\}$. Uvažujme $\phi: \{gG_x | g \in G\} \rightarrow [x]$, $gG_x \mapsto g(x)$. $g(x) \in [x]$. Je $\phi$ vůbec dobře definovaná? $gG_x = hG_c \implies g(x) = h(x)$, neboť podle dřívějšího tvrzení $gG_x = hG_x \Leftrightarrow h^{-1}g \in G_x \Leftrightarrow h^{-1}g(x) = x \Leftrightarrow g(x) = h(x)$. Zároveň je $\phi$ prosté (díky zpětným implikacím v předchozím) a je na, neboť pro $g(x) \in [x]$ mám $g(x) = \phi(gG_x)$.
            \end{dukazin}
        \end{tvrzeni}

        \begin{dusledek}[Spolu s lagrangeovou větou]
            $$ |G| = |G_x|·[G : G_x] = |G_x|·|[x]|. $$
        \end{dusledek}


    \subsection{Burnside}
        \begin{veta}[Burnsideova]
            Ať konečná grupa $G$ působí na konečné množině $X$. Pak:
            $$ |X/\sim| = \frac{1}{|G|}·\sum_{g \in G} |X_g|. $$

            \begin{dukazin}
                Nechť $M = \{(g, x) \in G \times X | g(x) = x\}$. Spočtu velikost $M$ dvěma způsoby:
                $$ |M| = \sum_{g \in G}|X_g| = \sum_{x \in X}|G_x|. $$
                $$ \frac{1}{|G|}·\sum_g |X_g| = \frac{1}{|G|}·\sum_x |G_x| = \sum_{x \in X}\frac{1}{|[x]|} = \sum_{o \in X/\sim} \sum_{x \in o} \frac{1}{|o|} = \sum_{o \in X/\sim} 1 = |X/\sim|. $$ 
            \end{dukazin}
        \end{veta}

% 7. 5. 2021

        \begin{definice}[Tranzitivní]
            Buď $G$ permutační grupa, čili $G ≤ S_X$. $G$ je tranzitivní, pokud má jenom 1 orbitu ve svém působení na $X$.
        \end{definice}

        \begin{veta}[Jordanova]
            Každé konečná tranzitivní grupa $G$, $|G|≥2$, obsahuje aspoň 1 permutaci bez pevného bodu.

            \begin{dukazin}
                Burnside: počet orbit (= 1 z tranzitivity) = průměrný počet pevných bodů. $\id \in G$ má $n ≥ 2$ pevných bodů, tedy nadprůměrný počet. To znamená, že $\exists g \in G$, které má podprůměrný počet pevných bodů, tedy 0.
            \end{dukazin}
        \end{veta}

        \begin{veta}[Cauchyova]
            Buď $G$ konečná grupa a $p$ prvočíslo tak, že $p | |G|$. Pak v $G$ existuje prvek řádu $p$.

            \begin{dukazin}
                $X = \{(a_1, a_2, …, a_p) \in G^{p} | a_1a_2…a_p = 1\}$. Mohutnost $X$ spočítáme tak, že víme, že $a_1, …, a_{p-1}$ můžeme zvolit libovolně a následně dopočítáme $a_p = a_{p-1}^{-1}…a_1^{-1}$, tedy $|X| = |G|^{p-1}$. $Z_p$ působí na $X$ rotací složek (0 je identita, 1 rotace o 1, …). $|[x]| | |Z_p| = p$, tudíž každá orbita má velikost $1$ nebo $p$. Existuje orbita velikosti $1$ a sice $(1, 1, …, 1)$. Zároveň $X$ je disjunktní sjednocení orbit, $p$ dělí $|X| = |G|^p$. Tedy počet $1$-prvkových orbit je $kp$ pro nějaké $k ≥ 1$, tudíž existuje alespoň $p-1$ $1$-prvkových orbit různých od $1, 1, …, 1$.

                Buď $(a_1, …, a_p)$ tato jiná orbita. Pak $a_1 = a_2$, $a_2 = a_3$, …, tedy je tvaru $(a, a, …, a) \in X$. Ale z definice $X$ je $a^p = 1$, zároveň není $a = 1$, tedy $\ord a = p$.
            \end{dukazin}
        \end{veta}

\section{Faktorgrupy}
    \subsection{Normální podgrupy}
        \begin{tvrzeni}
            $G$ je grupa, $H ≤ G$. NTJE:
            1) $aH = Ha$ $\forall a \in G$,
            2) $aha^{-1} \in H$ pro každé $h \in H$, $a \in G$.
        \end{tvrzeni}

        \begin{definicein}[Normální podgrupa]
            Podgrupa $H$ je normální v grupě $G$, pokud splňuje tyto podmínky. Značíme $H \trianglelefteq G$.
        \end{definicein}

        \begin{dukazin}
            $\implies$: Buď $h \in H$, $a \in G$. Víme: $ah \in aH = Ha \implies$ existuje $h' \in H$ tak, že $ah = h'a \implies aha^{-1} = h' \in H$.

            $\Leftarrow$: Dokážeme $aH \subseteq Ha$ (a analogicky $Ha \subseteq aH$). Buď $ah \in aH$. Pak $h' = aha^{-1} \in H$, a tedy $ah = h'a \in Ha$.
        \end{dukazin}

        \begin{priklady}
            $\{1\} \trianglelefteq G$ a $G \trianglelefteq G$. V abelovských grupách jsou všechny podgrupy normální. $SL_n(T) \trianglelefteq GL_n(T)$. Naopak $\{\id, (1\ 2)\} \ntrianglelefteq S_3$. Ale $A_n \trianglelefteq S_n$.
        \end{priklady}

        \begin{tvrzeni}
            Jádro homomorfismu je normální podgrupa.

            \begin{dukazin}
                Hom. $\phi: G \rightarrow H$. $\Ker \phi = \{g \in G | \phi(g) = 1_H\} ≤ G$. Buď $a \in G$, $k \in \Ker \phi$.. $aka^{-1} \in \Ker \phi$, protože
                $$ \phi(aka^{-1}) = \phi(a)\phi(k)\phi(a)^{-1} = \phi(a)\phi(a)^{-1} = 1. $$
            \end{dukazin}
        \end{tvrzeni}

    \subsection{Konstrukce faktorgrupy}
        \begin{definice}
            $G$ grupa, $N \trianglelefteq G$. Definujeme relaci na $G$: $a \sim b \Leftrightarrow ab^{-1} \in N$. $ab^{-1} \in N \Leftrightarrow Na = Nb \Leftrightarrow aN = bN$ (z normality). Třídy ekvivalence $[a] = aN = Na$.

            Faktorgrupa $G$ podle podgrupy $N$ = množina těchto bloků = $\{aN | a \in G\}$ (neboli $G/\sim = \{[a] | a \in G\}$), kde operace jsou $[a]·[b] = [ab]$, $[a]^{-1} = [a^{-1}]$ a neutrální prvek je $[1]$.

            \begin{dukazin}[Operace jsou dobře definované a $G/\sim$ je fakt grupa]
                Ať $[a] = [c]$, $[b] = [d]$. Potom $[ab] = [cd]$, protože $a \sim c$ a $b \sim d$, tedy $ac^{-1} \in N$ a $bd^{-1} \in N$, tudíž $ab(cd)^{-1} = abd^{-1}c^{-1} = (ac^{-1})·(c(bd^{-1})c^{-1}) \in N$. Obdobně pro $^{-1}$.

                Ověříme axiomy grupy…
            \end{dukazin}
        \end{definice}

        \begin{veta}
            $\phi: G \rightarrow H$ homomorfismus grup.

            1) (věta o homomorfismu) Je-li $N \subseteq \Ker \phi$ a $N \trianglelefteq G$, pak zobrazení $\psi: G/N \rightarrow H$, $[a] \mapsto \phi(a)$ je dobře definované a je to grupový homomorfismus.

            \begin{dukazin}
                Dobře definované: $[a] = [b] \implies ab^{-1} \in N \implies ab^{-1} \in \Ker \phi \implies \phi(ab^{-1}) = 1$. Tedy $\phi(a) = \phi(b)$.

                Homomorfismus: $\psi([a]) · \psi([b]) = \phi(a)·\phi(b) = \phi(ab) = \psi([ab]) = \psi([a]·[b])$.
            \end{dukazin}

% 14. 5. 2021

            2) (1. věta o isomorfismu) $G/\Ker \phi \simeq \Im \phi$.

            \begin{dukazin}
                Použiji 1) pro $N = \Ker \phi$. $\psi: G/\Ker \rightarrow H$. $\psi$ je prosté: $[a] = [b] \Leftrightarrow ab^{-1} \in \Ker \phi \Leftrightarrow \phi(ab^{-1}) = 1 \Leftrightarrow \phi(a) = \phi(b) \Leftrightarrow \psi([a])=\psi([b])$. Když se na $\psi$ dívám jenom jako na zobrazení $G/\Ker \phi \rightarrow \Im \phi ≤ H$, tak je na.
            \end{dukazin}
        \end{veta}

        \begin{tvrzeni}[2. věta o isomorfismu]
            $N \trianglelefteq G$. 1) $N \trianglelefteq H \trianglelefteq G$, pak $H/N$ je normální podgrupa $G/N$.

            2) Je-li $K \trianglelefteq G/N$, pak existuje $H \trianglelefteq G$ tak, že $K = H/N$.

            3) $N \trianglelefteq H \trianglelefteq G$, pak $(G/N)/(H/N) \simeq G/H$.

            (1) i 3) platí i pro podgrupy, které nejsou normální.)

            \begin{dukazin}
                1) se ověří.

                2) $K \trianglelefteq G/N \implies K = H/N$ pro nějaké $H$, a sice $H = \{a \in G | [a] = aN \in K\}$. Ověří se, že $K = H/N$.

                3) 1. věta o isomorfismu: Uvažujme homomorfismus $\phi: G/N \rightarrow G/H$, $aN \mapsto aH$. Ověříme, že je dobře definován: $aN = bN \Leftrightarrow ab^{-1} \in N \implies ab^{-1} \in H \Leftrightarrow aH = bH$ a že je to homomorfismus: $\phi(aN·bN) = \phi(abN) = (ab)H \overset{?}{=} \phi(aN)·\phi(bN) = aH·bH = (ab)H$. $\Im \phi = G/H$ zřejmě, $\Ker \phi = \{aN | \phi(aN) = aH = 1·H\}$, ale $aH = 1·H \Leftrightarrow a·1^{-1} = a \in H$. Tedy $\Ker \phi = H/N$.
            \end{dukazin}
        \end{tvrzeni}

        \begin{tvrzeni}[3. věta o isomorfismu]
            $N \trianglelefteq G$, $H ≤ G$. Pak $HN$ je podgrupa $G$. $H \cap N \trianglelefteq H \land HN/N \simeq H/(H \cap N)$.

            \begin{dukazin}
                Bez důkazu (jednoduchý).
            \end{dukazin}
        \end{tvrzeni}

    \subsection{Řešitelné grupy}
        \begin{definice}[Řešitelná grupa, stupeň řešitelnosti]
            Grupa $G$ je řešitelná, pokud $\exists k \in ®N$ a normální podgrupy $N_0, N_1, …, N_k \trianglelefteq G$ tak, že $\{1\} = N_0 ≤ N_1 ≤ … ≤ N_k = G$ a každá faktorgrupa $N_i / N_{i-1}$, pro $i \in [k]$, je abelovská.

            Nejmenší $k$, pro které tento řetězec v $G$ existuje, se nazývá stupeň řešitelnosti $G$.
        \end{definice}

        \begin{definice}[Metaabelovská grupa]
            Grupa stupně řešitelnosti 2 se nazývá metaabelovská.
        \end{definice}

        \begin{veta}[Feit-Thompson]
            Každá grupa lichého řádu je řešitelná.

            Jednodušší varianta: Každá grupa řádu $p^k$ ($p$ prvočíslo) je řešitelná.

            \begin{dukazin}
                Extrémně těžký.
            \end{dukazin}
        \end{veta}

        \begin{definice}[Derivovaná podgrupa]
            $G$ grupa. Její derivovaná podgrupa je $G' = \<aba^{-1}b^{-1} | a, b \in G\>$.
        \end{definice}

        \begin{lemma}
            $N \trianglelefteq G$. 1) $N' \trianglelefteq G$. 2) $G/N$ je abelovská $\Leftrightarrow$ $G' ≤ N$.

            \begin{dukazin}
                1) ověří se (viz skripta). 2) $G/N$ abelovská $\Leftrightarrow$ $[a][b] = [b][a]$ $\Leftrightarrow$ $[aba^{-1}b^{-1}] = [1]$ $\Leftrightarrow$ $aba^{-1}b^{-1} \in N$ $\Leftrightarrow$ $G' \subseteq N$.
            \end{dukazin}
        \end{lemma}

        \begin{tvrzeni}[Řešitelnost podgrup a faktogrup]
            $G$ grupa. 1) $G$ řešitelná a $H ≤ G \implies H$ řešitelná.

            2) $G$ řešitelná, $K \trianglelefteq G$ $\implies$ $G/K$ řešitelná.

            3) Pokud $\exists N \trianglelefteq G$ tak, že $N$ je řešitelná a $G / N$ je řešitelná, pak je $G$ řešitelná.

            \begin{dukazin}
                1) $N_i \trianglelefteq G$, $\{1\} = N_0 ≤ N_1 ≤ … ≤ N_k = G$, $N_i / N_{i-1}$ abelovská. Uvažujme $\{1\} = N_0 \cap H ≤ N_1 \cap H ≤ … ≤ N_k \cap H = H$, dokážeme, že tato posloupnost prokazuje řešitelnost $H$. Triviálně $N_i \cap H \trianglelefteq H$. $(N_i\cap H)/(N_{i-1} \cap H) = (N_i \cap H)/((N_i \cap H) \cap N_{i-1})$. Podle 3. věty o isomorfismu je to isomorfní s $N_{i-1}(N_i \cap H) / N_{i-1} ≤ N_i / N_{i-1}$, která je abelovská, tedy i její podgrupa je abelovská, tedy $(N_i \cap H)/(N_{i-1} \cap H)$ je abelovská.

                2) analogicky (2. a 3. věta o isomorfismu). 3. podobně (nezkouší se, viz skripta).
            \end{dukazin}
        \end{tvrzeni}

        \begin{dusledek}
            $G$ grupa s normálními podgrupami $N_0, …, N_k$ tak, že $\{1\} = N_0 ≤ N_1 ≤ … ≤ N_k = G$ a $N_i/N_{i-1}$ jsou řešitelné. Pak $G$ je řešitelná.

            \begin{dukazin}
                Indukcí podle $k$ a použitím třetího bodu předchozího tvrzení.
            \end{dukazin}
        \end{dusledek}

\section{Číselná tělesa a kořeny polynomů}
    \subsection{Okruhové homeomorfismy a faktorokruhy}
        \begin{tvrzeni}[Obraz a jádro]
            $R, S$ okruhy, $\phi: R \rightarrow S$ homomorfismus okruhů. 1) $\Im \phi$ je podokruh $S$. 2) $\Ker \phi$ je ideál v $R$.

            \begin{dukazin}
                1) $\phi(a) + \phi(b) = \phi(a+b) \in \Im \phi$. $\phi(a)·\phi(b) = \phi(a·b) \in \Im \phi$.

                2) $\Ker \phi$ je uzavřené na $+$ (ověří se). Také je uzavřené na násobení $R$: $a \in \Ker \phi, r \in R \implies \phi(ra) = \phi(r)·\phi(a) = \phi(r) = 0$.
            \end{dukazin}
        \end{tvrzeni}

        \begin{tvrzeni}
            $\phi: R \rightarrow S$ je prostý homomorfismus okruhů $\Leftrightarrow$ $\Ker \phi = \{0\}$.
        \end{tvrzeni}

        \begin{tvrzeni}
            $\phi: R \rightarrow S, \psi: S \rightarrow T$ homomorfismy okruhů. Pak 1) $\psi \circ \phi: R \rightarrow T$ je homomorfismus okruhů.

            2) Je-li $\phi$ bijekce (čili $\phi$ je isomorfismus), pak $\phi^{-1}: S \rightarrow R$ je homomorfismus okruhů (a tedy isomorfismus).
        \end{tvrzeni}

% 19. 5. 2021

    \subsection{Faktorokruh podle ideálu}
        \begin{definice}
            $R$ okruh, $I$ ideál v $R$. Definujeme ekvivalenci na $R$ $a \sim b \Leftrightarrow a-b \in I$.

            Uvažujme jenom grupu $(R, +, -, 0)$. Potom $\sim$ je ta stejná ekvivalence podle podgrupy $I$.

            $a \sim b \Leftrightarrow a + I = b + I$ … třídy ekvivalence jsou rozkladové třídy podle podgrupy. $[a] = a + I$.

            Na blocích definujeme $[a] + [b] = [a+b]$, $-[a] = [-a]$, $[0]$ je neutrální prvek, $[a]·[b] = [a·b]$.

            Množina bloků $[a]$ s těmito operacemi je faktorokruh podle ideálu $I$: $R/I = \(\{[a] | a \in R\}, +, -, [0], ·\)$.

            \begin{dukazin}[Dobrá definovanost]
                1) Operace na blocích jsou dobře definované: $+, -$ víme z grup. Buď nyní $[a] = [c]$, $[b] = [d]$, chceme $[ab] = [cd]$. Z definice bloků víme, že $a - c \in I$ a $b - d \in I$, tudíž $ab - cd = a(b - d) + (a - c)d \in I$.

                2) $R/I$ je okruh: Ověří se axiomy okruhu pro $R/I$ z axiomů pro $R$. (Pozor, axiomy oboru se nezachovávají.)
            \end{dukazin}
        \end{definice}

        \begin{veta}
            $\phi: R \rightarrow S$ homomorfismus okruhů.

            1) (věta o homomorfismu) $I \subseteq \Ker \phi$ ideál v $R$. Pak $\phi: R/I \rightarrow S$, $[a] \mapsto \phi(a)$ je dobře definovaný okruhový homomorfismus.

            2) (1. věta o isomorfismu) $R / \Ker \phi \simeq \Im \phi$, $[a] \mapsto \phi(a)$.

            \begin{dukazin}
                Použije se věta pro grupy a člověk si rozmyslí, že tato zobrazení jsou homomorfismy i vůči násobení.
            \end{dukazin}
        \end{veta}

        \begin{poznamka}
            Platí i 2. a 3. věta o isomorfismu. (Viz skripta.)
        \end{poznamka}

    \subsection{Kdy je faktorokruh obor / těleso?}
        \begin{definice}[prvoideál, maximální ideál]
            Ideál $I$ v okruhu $R$ je prvoideál, pokud $\forall a, b \in R: ab \in I \implies a \in I$ nebo $b \in I$. A je maximální, pokud je $I$ maximální vlastní ideál v $R$, čili neexistuje ideál $j$ tak, že $I \subset J \subset R$.
        \end{definice}

        \begin{veta}
            $R$ komutativní okruh s $1$, $I$ ideál. 1) Pak $R/I$ je obor $\Leftrightarrow$ $I$ prvoideál. 2) Potom $R/I$ je těleso $\Leftrightarrow$ $I$ je maximální.

            \begin{dukazin}
                1) $R/I$ obor $\Leftrightarrow$ $([a]·[b] = 0 \implies [a] = 0 \lor [b] = 0)$ $\Leftrightarrow$ $(a·b \in I \implies a \in I \lor b \in I)$ $\Leftrightarrow$ $I$ je prvoideál.

                2) Z tvrzení 7.6: $R/I$ těleso $\Leftrightarrow$ nemá žádné vlastní ideály. Z druhé věty o isomorfismu: Ideály v $R/I$ jsou právě $J/I$ pro $J \supseteq I$. Tedy $R/I$ je těles $\Leftrightarrow$ $I$ je maximální ideál.
            \end{dukazin}
        \end{veta}

\section{Tělesové rozšíření jako vektorový prostor}
    \begin{definice}[Rozšíření těles]
        Rozšíření těles jsou tělesa $®T$, $®S$ tak, že $®T ≤ ®S$. (Čili $®T$ je podtěleso $®S$ a $®S$ je rozšíření $®T$).
    \end{definice}

    \begin{definice}[Stupeň rozšíření]
        Dimenze vektorového prostoru $®S_{®T}$ (vektorový prostor nad ®T odpovídající ®S) je stupeň rozšíření $®S≥®T$. Neboli $[®S:®T] = \dim ®S_{®T}$.
    \end{definice}
\end{document}
