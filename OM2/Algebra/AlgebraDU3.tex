\documentclass[12pt]{article}                   % Začátek dokumentu
\usepackage{../../MFFStyle}                     % Import stylu

\let\oldmod=\mod
\def\mod{\!\!\!\!\oldmod}

\begin{document}
    \begin{priklad}[3.1]
        Nalezněte ireducibilní rozklad polynomu $x^4 + 1$ nad tělesy ®C, ®R a $®Z_5$.

        \begin{reseni}
            Ze střední školy víme, že rovnice $x^4 + 1 = 0$ má 4 řešení tvaru $e^{k\frac{2\pi}{4} + \frac{\pi}{4}}$ (jelikož $-1 = e^\pi$), tedy rozklad v ®C bude 
            $$ x^4 + 1 = \(x - e^{\pi/4}\)·\(x - e^{3\pi/4}\)·\(x - e^{5\pi/4}\)·\(x - e^{7\pi/4}\) = $$
            $$ \(x - \frac{\sqrt{2}}{2} - i\frac{\sqrt{2}}{2}\)·\(x + \frac{\sqrt{2}}{2} - i\frac{\sqrt{2}}{2}\)·\(x + \frac{\sqrt{2}}{2} + i\frac{\sqrt{2}}{2}\)·\(x - \frac{\sqrt{2}}{2} + i\frac{\sqrt{2}}{2}\), $$
            jelikož polynomy stupně 1 jsou v tělese ireducibilní.

            Pro rozklad v reálných číslech si můžeme všimnout, že prostřední dva činitelé a krajní dva jsou rozklady rozdílu čtverců:
            $$ x^4 + 1 = \(x + \frac{\sqrt{2}}{2} - i\frac{\sqrt{2}}{2}\)·\(x + \frac{\sqrt{2}}{2} + i\frac{\sqrt{2}}{2}\)·\(x - \frac{\sqrt{2}}{2} + i\frac{\sqrt{2}}{2}\)·\(x - \frac{\sqrt{2}}{2} - i\frac{\sqrt{2}}{2}\) = $$
            $$ = \(x^2 + x\sqrt{2} + 1\)·\(x^2 - x\sqrt{2} + 1\). $$
            To už je ireducibilní rozklad, protože dělitelé polynomu mají menší stupeň. A polynom stupně 0 je invertibilní, a pokud by existoval polynom stupně jedna dělící $x^4 + 1$, pak by existoval i kořen v reálných číslech, ale my víme, že všechny 4 kořeny jsou komplexní ($®C \setminus ®R$).
            
            V $®Z_5$ polynom $x^4 + 1$ nemá kořeny, jelikož pro $x ≡ 0$ je to 1 a pro $x \not≡ 0$ je $x^4 ≡ 1$ (z Eulerovy věty), tedy $x^4 + 1 ≡ 2$. Tudíž hledáme rozklad na polynomy stupně dva. Když jsme nepochodili s kořeny $x^4 + 1$, můžeme zkusit najít kořeny $y^2 + 1$. Jednoduchým otestováním se dostaneme ke kořenům 2 a 3. Tedy:
            $$ x^4 + 1 ≡ \(x^2 - 3\)·\(x^2 - 2\) ≡ \(x^2 + 2\)·\(x^2 + 3\). $$
        \end{reseni}
    \end{priklad}

    \begin{priklad}[3.2]
        Nalezněte (nějaký) ireducibilní rozklad prvku $16 + i\sqrt{5}$ v oboru $®Z[i\sqrt{5}]$.

        \begin{reseni}
            Víme, že norma na $®Z[i\sqrt{5}]$ je dána $\nu(a + b·i\sqrt{5}) = |a^2 + 5b^2|$, tedy $\nu(16 + i\sqrt{5}) = |256 + 5| = 261 = 3^2·29$. Víme, že pokud $a | b$, pak $\nu(a) | \nu(b)$, tedy zkusíme rozklad na prvky norem $9$ a $29$, jelikož $3$ je moc malá, protože pro $|b| > 0$ je $\ni(a+bi\sqrt{5}) ≥ 5$, tedy jediným prvkem $®Z[i\sqrt{5}]$ s normou 3 je 3 a ta už na první pohled $16 + i\sqrt{5}$ nedělí.

            Prvky s normou $9 = 2^2 + 5·1^2$ a $29 = 3^3 + 5·2^2$ jsou například $2 + i\sqrt{5}$ a $3 + 2i\sqrt{5}$, ale ty vynásobené mezi sebou nedají $16 + i\sqrt{5}$. Tak zkusíme změnit znaménko a:
            $$ 16 + i\sqrt{5} = (2 - i\sqrt{5})(3 + 2i\sqrt{5}), $$
            kde norma prvního je $9$, tedy všichni nevlastní dělitelé musí mít normu 3 a to už víme, že nejde, a norma druhého je $29$, což je prvočíslo, tedy jsme opravdu našli ireducibilní rozklad.
        \end{reseni}
    \end{priklad}

\pagebreak

    \begin{priklad}[3.3]
        Nalezněte největšího společného dělitele čísel $4 + 6i$ a $3 − 7i$ v oboru $®Z[i]$.

        \begin{reseni}
            Víme, že $®Z[i]$ je eulerovský, tedy použijeme eukleidův algoritmus:
            $$ 3 - 7i, 4 + 6i: \frac{3 - 7i}{4 + 6i} = \frac{(3-7i)(4-6i}{52} = \frac{-30 - 46i}{52} = \frac{-52 - 52i}{52} + \frac{22 + 6i}{52} ≐ -1 - i. $$
            Zaokrouhlujeme tak, aby norma zbytku byla nejmenší. Zbytek můžeme dopočítat z předchozího výpočtu, nebo: $3 - 7i - (4 + 6i)·(-1-i) = 3 - 7i - 2 + 10i = 1 + 3i$.

            $$ 4 + 6i, 1 + 3i: \frac{4 + 6i}{1 + 3i} = \frac{(4 + 6i)(1-3i)}{10} = \frac{22 - 6i}{10} = \frac{20 - 10i}{10} + \frac{2 + 4}{10} ≐ 2 - i. $$
            $$ 4 + 6i - (2 - i)·(1 + 3i) = 4 + 6i - 5 - 5i = -1 + i. $$

            $$ 1 + 3i, -1 + i: \frac{1 + 3i}{-1 + i} = \frac{(1 + 3i)(-1 - i)}{2} = 1 - 2i. $$ 
            Tudíž zbytek nula a $\NSD(4+6i, 3-7i) = -1 + i$ (a všechna sdružená čísla).
        \end{reseni}
    \end{priklad}

    \begin{priklad}[3.4]
        Zvolme pevné $z \in ®C$. Ukažte, že množina $\{f \in ®Q[x] | f(z) = 0\}$ tvoří ideál okruhu $®Q[x]$.

        \begin{dukazin}
            Stačí ukázat uzavřenost na sčítání, opačný prvek a násobení libovolným prvkem $®Q[x]$:
            $$ f, g \in ®Q[x] \land f(z) = g(z) = 0 \implies (f+g)(z) = 0 + 0 = 0, $$
            $$ f \in ®Q[x] \land f(z) = 0 \implies (-f)(z) = -0 = 0, $$ 
            $$ f, g \in ®Q[x] \land f(z) = 0 \implies (f·g)(z) = f(z)·g(z) = 0·g(z) = 0. $$ 
        \end{dukazin}
    \end{priklad}

\end{document}
