\documentclass[12pt]{article}                   % Začátek dokumentu
\usepackage{../../MFFStyle}                     % Import stylu

\let\oldmod=\mod
\def\mod{\!\!\!\!\!\!\oldmod}

\begin{document}
    \begin{priklad}[6.1]
        Dětská stavebnice obsahuje $8$ červených a $8$ modrých destiček ve tvaru rovnostranného trojúhelníka. Kolika způsoby z nich lze sestavit velký trojúhelník o čtyřnásobné hraně

        \begin{itemize}
            \item[(a)] až na otočení,
            \item[(b)] až na otočení a převrácení?
        \end{itemize}

        \begin{reseni}[(a)]
                Grupa symetrií otočení trojúhelníku je 3 prvková (vše je plně určeno, kam se zobrazí daný vrchol trojúhelníku). $\id$ klasicky zachovává všechny trojúhelníky, tedy $|G_{\id}| = \binom{16}{8}$ (použijeme všechny trojúhelníky a jen volíme, kterých 8 bude červených, zbytek budou modré).

            Naopak stabilizátor obou netriviálních rotací je prázdný, protože obě rotace jsou 5 trojcyklů a 1 „jednocyklus“, tudíž počet shodných trojúhelníčků musí dávat zbytek 0 nebo 1 po dělení 3. Ale $8$ děleno $3$ dává zbytek 2. Podle Burnsideovy věty je počet způsobů
            $$ \frac{1}{3}\(\binom{8}{16} + 2·0\) = 4290. $$ 
        \end{reseni}

        \begin{reseni}[(b)]
            Tentokrát máme grupu $D_{2·3}$, která má $6$ prvků. 3 prvky se shodují, tedy není třeba je řešit. Další 3 jsou osové symetrie. Ty zachovávají ty trojúhelníky, které mají „levých“ 6 trojúhelníčků shodných s „pravými“ 6.

            To znamená, že buď 4 z nich jsou červené a „prostřední 4“ modré. Nebo jsou 3 červené a 3 modré, tedy potom ještě vybíráme 2 ze 4 „prostředních“ a nebo jsou 4 z nich modré. Potom Burnsideova věta dává
            $$ \frac{1}{6}\(\binom{16}{8} + 2·0 + 3·\(\binom{6}{4} + \binom{6}{3}\binom{4}{2} + \binom{6}{2}\)\) = 2222. $$ 
        \end{reseni}
    \end{priklad}

\pagebreak

    \begin{priklad}[6.2]
        Popište pomocí první věty o izomorfismu, jak vypadá faktogrupa $®R^*/\{-1, 1\}$.

        \begin{reseni}
            Absolutní hodnota je zřejmě homeomorfismus $®R^* \rightarrow ®R^+$ (při násobení v absolutní hodnotě můžeme všechny znaménka vytknout a „smazat“, tedy vše probíhá jako v $®R^+$), kde $®R^+$ jsou kladná reálná čísla s násobením. Navíc jádro je $\{-1, 1\}$. Tedy podle 1. věty o izomorfismu je $®R/\{-1, 1\} \simeq ®R^+$.
        \end{reseni}
    \end{priklad}

    \begin{priklad}[6.3]
        Nalezněte minimální polynom $m_{a, ®Q}$, kde $a = 1 + \sqrt{5}$.

        \begin{reseni}
            Jelikož $1 + \sqrt{5}$ zřejmě není prvek ®Q, tak $m_{a, ®Q}$ musí být stupně nejméně 2. Všimneme si, že $a^2 = 1 + 2\sqrt{5} + 5$, kde se můžeme zbavit $\sqrt{5}$ odečtením $2a$. Potom už zbývá jen $4$, ale to už je prvek ®Q. Tedy $a^2 - 2a - 4 = 0$. Neboli $m_{a, ®Q} = x^2 - 2x - 4$.
        \end{reseni}
    \end{priklad}

    \begin{priklad}[6.4]
        Určete stupeň rozšíření $[®Q(\sqrt[4]{5}):®Q]$ a nalezněte nějakou bázi $®Q(\sqrt[4]{5})$ nad ®Q.

        \begin{reseni}
            Z tvrzení o stupni jednoduchého rozšíření je $[®Q(\sqrt[4]{5}):®Q] = \deg m_{\sqrt[4]{5}, ®Q}$. Tedy stačí najít minimální polynom. Polynom $x^4 - 5 = 0$ je nerozložitelný, jelikož je primitivní nad ®Z, z Einsteinova kritéria je tedy ireducibilní nad ®Z ($5|5$, $5|0$, $5|0$, $5|0$, $5\nmid 1$, $5^2\nmid 5$), a z toho už víme, že je ireducibilní i v podílovém tělese -- v ®Q. Tedy $m_{\sqrt[4]{5}, ®Q} = x^4 - 5$. Tedy $[®Q(\sqrt[4]{5}):®Q]=4$.

            Jako báze se nabízí $(1, \sqrt[4]{5}, \sqrt[4]{5}^2, \sqrt[4]{5}^3)$. To že tato posloupnost generuje celé $®Q(\sqrt[4]{5})$ nad ®Q se ověří snadno (obsahuje $\sqrt[4]{5}$ a jakýkoliv násobek lineární kombinace této posloupnosti lze vyjádřit zase jako lineární kombinaci, jelikož násobením $\sqrt[4]{5}^i$ spolu vznikají zase jen $\sqrt[4]{5}^j$, ze kterých můžu vytknout celé číslo a zbytek bude $\sqrt[4]{5}^k$, pro $k < 4$). A je lineárně nezávislá, jelikož generuje celé $®Q(\sqrt[4]{5})$ nad ®Q a takový generátor musí mít dimenzi alespoň 4. (Pokud by byla lineárně závislá, tak umíme vytvořit posloupnost délky 3, která generuje tento VP.)
        \end{reseni}
    \end{priklad}

\end{document}
