\documentclass[12pt]{article}                   % Začátek dokumentu
\usepackage{../../MFFStyle}                     % Import stylu

\let\oldmod=\mod
\def\mod{\!\!\!\!\!\!\oldmod}

\begin{document}
    \begin{priklad}[5.1]
        Dokažte, že množina všech nekonstantních lineárních funkcí $®R \rightarrow ®R$ spolu s operací skládání funkcí a vhodně zvolenou jednotkou a operací inverze tvoří grupu.

        \begin{dukazin}
            Nechť $Y$ je množina nekonstantních lineárních funkcí $®R^2 \rightarrow ®R^2$,
            $$ X = \{\begin{pmatrix} a & b \\ 0 & 1 \end{pmatrix} \middle| a, b \in ®R \land a ≠ 0 \} $$
            a $h: X \rightarrow Y$, $\begin{pmatrix} a & b \\ 0 & 1 \end{pmatrix} \mapsto f(x)=ax + b$. Zřejmě se jedná o bijekci, jelikož každá nekonstantní lineární funkce se dá vyjádřit v tomto tvaru ($a, b \in ®R$, $a≠0$) a toto vyjádření je jednoznačné (a opačně každý tento tvar vyjadřuje nějakou nekonstantní lineární funkci).

            Dále si můžeme všimnout, že $X ≤ GL_2(®R)$, jelikož jsou tyto matice horní trojúhelníkové s nenulovými prvky na diagonále, tedy regulární, tudíž stačí ověřit jen uzavřenost na násobení:
            $$ \begin{pmatrix} a & b \\ 0 & 1 \end{pmatrix}\begin{pmatrix} c & d \\ 0 & 1 \end{pmatrix} = \begin{pmatrix} ac & ad+b \\ 0 & 1 \end{pmatrix} \in X. \qquad(a≠0 \land c≠0 \implies ac≠0) $$

            Následně si představme, že $Y$ je opravdu grupa (tj. že existuje inverze, jednotka a $Y$ je uzavřená na $\circ$). Potom můžeme ukázat, že $h$ je homomorfismus (tj. izomorfismus). Stačí ověřit, že obraz součinu je součin obrazů, tj.
            $$ „f(x) = ac·x + ad+b“ = h \begin{pmatrix} ac & ad+b \\ 0 & 1 \end{pmatrix} \overset{?}{=} $$
            $$ h \begin{pmatrix} a & b \\ 0 & 1 \end{pmatrix}·h\begin{pmatrix} c & d \\ 0 & 1 \end{pmatrix} = „f(x) = a·x + b“ \circ „f(x) = c·x + d“ = „f(x) = a·(c·x + d) + b“ $$ 

            Tím jsme dokázali, že $Y$ je grupa, neboť z $h$ umíme odvodit inverzi a jednotku. Jednotka je obraz jednotkové matice (jednotky v $GL_2$), tedy $f(x) = 1·x$. Inverze k $f(x) = ax + b$ je $f(x)\frac{1}{a}x - \frac{b}{a}$ (tj. inverzní funkce), což nám vyjde i z $h\(\(h^{-1}(„f(x) = ax + b“)\)^{-1}\)$.
        \end{dukazin}
    \end{priklad}

\pagebreak

    \begin{priklad}[5.2]
        Vypište všechny prvky grupy $GL_2(®Z_2)$ a uveďte jejich řády.

        \begin{reseni}
            Matice
            $$ \begin{pmatrix} a & b \\ c & d \end{pmatrix} $$
            má determinant $a·d - b·c$, což v $®Z_2$ znamená, že právě jedna diagonála bude obsahovat 2 jedničky. To nám dává
            $$ GL_2(®Z_2) = \{\begin{pmatrix} 1 & 0 \\ 0 & 1 \end{pmatrix}, \begin{pmatrix} 0 & 1 \\ 1 & 0 \end{pmatrix}, \begin{pmatrix} 0 & 1 \\ 1 & 1 \end{pmatrix}, \begin{pmatrix} 1 & 0 \\ 1 & 1 \end{pmatrix}, \begin{pmatrix} 1 & 1 \\ 1 & 0 \end{pmatrix}, \begin{pmatrix} 1 & 1 \\ 0 & 1 \end{pmatrix}\}. $$

            $GL_2(®Z_2)$ má 6 prvků, tedy podle Lagrangeovy věty má prvek řád buď $1$ (pouze jednotková matice), $2$ (ty matice, které v druhé mocnině dají jednotkovou), $3$ (všechny ostatní, vzhledem k tomu, že:), $6$ (žádné, jelikož pak by byla grupa cyklická, ale (jak uvidíme) $GL_2(®Z_2)$ má $3≠\phi(2)$ prvky řádu 2). Podíváme se tedy na druhé mocniny:
            $$  \begin{pmatrix} 0 & 1 \\ 1 & 0 \end{pmatrix}^2 = \begin{pmatrix} 1 & 0 \\ 1 & 1 \end{pmatrix}^2 = \begin{pmatrix} 1 & 1 \\ 0 & 1 \end{pmatrix}^2 = \begin{pmatrix} 1 & 0 \\ 0 & 1 \end{pmatrix}, $$
            $$ \begin{pmatrix} 1 & 1 \\ 1 & 0 \end{pmatrix}^2 = \begin{pmatrix} 0 & 1 \\ 1 & 1 \end{pmatrix}, \qquad \begin{pmatrix} 0 & 1 \\ 1 & 1 \end{pmatrix}^2 = \begin{pmatrix} 1 & 1 \\ 1 & 0 \end{pmatrix}. $$

            Tudíž matice mají řád
            $$  \ord \begin{pmatrix} 0 & 1 \\ 1 & 0 \end{pmatrix} = \ord \begin{pmatrix} 1 & 0 \\ 1 & 1 \end{pmatrix} = \ord \begin{pmatrix} 1 & 1 \\ 0 & 1 \end{pmatrix} = 2, $$
            $$ \ord \begin{pmatrix} 1 & 0 \\ 0 & 1 \end{pmatrix} = 1, \qquad \ord \begin{pmatrix} 1 & 1 \\ 1 & 0 \end{pmatrix} = \ord \begin{pmatrix} 0 & 1 \\ 1 & 1 \end{pmatrix} = 3. $$
        \end{reseni}
    \end{priklad}

\pagebreak

    \begin{priklad}[5.3]
        Nalezněte všechny grupové homomorfismy ze $(S_3, \circ, \ ^{-1}, \id)$ do $(®Z_6, +, -, 0)$ a popište jejich obrazy a jádra.

        \begin{reseni}
            Řád obrazu prvku musí dělit řád tohoto prvku, tedy transpozice (prvky grupy $S_3$ řádu 2) se musí zobrazovat na $3$ nebo $0$ a trojcykly (prvky řádu 3) na $3$ nebo $0$.

            $(2\ 3) \circ (1\ 3) = (1\ 2\ 3)$ a $(2\ 3) \circ (1\ 2) = (3\ 2\ 1)$, takže trojcykly se musí zobrazovat na nulu, protože jejich obraz musí být součet 0 a 0, 3 a 0 nebo 3 a 3 (z vlastností homomorfismu), ale to 2 a 4 nejsou. Díky tomu musí být obraz všech transpozic shodný, neboť složení dvojice transpozic dá trojcyklus, tedy součet jejich obrazů musí být nula, a to $3 + 0$ není.

            Tedy máme 2 homomorfismy: triviální (všechno zobrazí na $0$), kde obrazem je tedy $\{0\}$ a jádrem je $S_3$, a zobrazení $h$ takové, které zobrazuje trojcykly a identitu na $0$ a transpozice na 3, tudíž obraz je $\{0, 3\}$ a jádro je $\{\id, (1\ 2\ 3), (3\ 2\ 1)\}$.

            Zbývá dokázat, že $h$ je opravdu homomorfismus. Pro to stačí dokázat $h(x\circ y) = h(x) + h(y)$, $\forall x, y \in S_3$. To je pro $x = \id$ nebo $y=\id$ jednoduché. Složení trojcyklů v $S_3$ je identita a $0 + 0 = 0$, tedy pro ně to platí také. Složení trojcyklu a transpozice má liché znamínko, tedy to musí být transpozice a $0 + 3 = 3 + 0 = 3$. Nakonec složení dvou různých transpozic má sudé znamínko a není to identita, tedy v $S_3$ je to trojcyklus a $3 + 3 = 0$.
        \end{reseni}
    \end{priklad}

\end{document}
