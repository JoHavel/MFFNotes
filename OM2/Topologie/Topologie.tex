\documentclass[12pt]{article}                   % Začátek dokumentu
\usepackage{../../MFFStyle}                     % Import stylu

\begin{document}

% 5. 3. 2021
\begin{poznamka}[Úmluva]
    Všechny topologické prostory v tomto semestru budou Hausdorffovy ($T_2$). Tedy regulární jsou automaticky $T_3$, úplně regulární jsou automaticky $T_\pi$ a normální jsou $T_4$.

    Speciálně např. kompaktní prostory jsou $T_4$.
\end{poznamka}

\section{Parakompaktní prostory}
    \begin{poznamka}[Připomenutí]
        Pokrytí, otevřené pokrytí, podpokrytí.
    \end{poznamka}

    \begin{definice}[Zjemnění]
        Ať $X$ je množina a ©S je pokrytí $X$. Řekneme, že systém $©T \in ©P(X)$ je zjemnění ©S, pokud $©T$ je pokrytí a $\forall T \in ©T\ \exists S \in ©S: T \subseteq S$.
    \end{definice}

    \begin{definice}[Lokálně konečný systém]
        Ať ®X je TP, $©S \subseteq ©P(®X)$. ©S se nazývá lokálně konečný, pokud
        $$ \forall x \in ®X\ \exists U \in ©U(x): \{S \in ©S| S \cap U ≠ \O\} \text{ je konečná.} $$

        Systém ©S se nazve diskrétní, pokud
        $$ \forall x \in ®X\ \exists U \in ©U(x): \left|\{S \in ©S| S \cap U ≠ \O\}\right| ≤ 1. $$

        Systém ©S se nazve $\sigma$-lokálně konečný (resp. $\sigma$-diskrétní), pokud $\exists S_n$, že $Š = \bigcup_{n=1}^∞$, že $©S_n$ jsou lokálně konečné (resp. diskrétní), $n \in ®N$.
    \end{definice}

    \begin{poznamka}
        Diskrétní systém je lokálně konečný. $\sigma$-diskrétní systém je $\sigma$-lokálně konečný.
    \end{poznamka}

    \begin{lemma}[Uzávěr lokálně konečného prostoru]
        Ať ®X je TP, $©A \subseteq ©P(X)$ lokálně konečný systém. Pak $\{\overline{A}| A \in ©A\}$ je opět lokálně konečný a platí $\overline{\bigcup ©A} = \bigcup \{\overline{A}| A \in ©A\}$.

        \begin{dukazin}
            Ať $x \in ®X$ je libovolné. Existuje $U \in ©U(x): \{A \in ©A: A \cap U ≠ \O\}$ je konečná. Ať $V = \int U, V \in ©U(x)$. $\{A \in ©A: A \cap V ≠ \O\}$ je zřejmě konečná. $V \cap A ≠ \O \Leftrightarrow V \cap \overline{A} ≠ \O$. Tedy $\{A \in ©A: \overline{A} \cap V ≠ \O\} = \{A \in ©A: A \cap V ≠ \O\}$. Tedy $\{\overline{A}| A \in ©A\}$ je konečná.

            $\supseteq:$ $\bigcup ©A \supseteq A, A \in ©A$, tedy $\overline{\bigcup ©A} \supseteq \overline{A} \implies \overline{\bigcup ©A} \supseteq \bigcup \{\overline{A}| A \in ©A\}$.

            $\subseteq:$ Ať $x \in \overline{\bigcup ©A}$. $\exists U \in ©U(x)$ otevřená, že $\{A \in ©A: A \cap U ≠ 0\} = \{A_1, …, A_n\}$. $x \in \overline{A_1 \cup … \cup A_n} \stackrel{\text{konečné}}{=} \overline{A_1} \cup … \cup \overline{A_n}$. $\exists i ≤ n: x \in \overline{A_i}$.
        \end{dukazin}
    \end{lemma}

    \begin{definice}[Parakompaktní]
        TP ®X se nazývá parakompaktní, pokud každé jeho otevřené pokrytí má lokálně konečné otevřené zjemnění.
    \end{definice}

    \begin{poznamka}
        Kompaktní $\implies$ parakompaktní (protože podpokrytí je zjemnění a konečné je lokálně konečné).

        Diskrétní TP $\implies$ parakompaktní.
    \end{poznamka}

    \begin{tvrzeni}
        Uzavřený podprostor parakompaktního TP je parakompaktní.

        \begin{dukazin}
            ®X je parakompaktní TP a $F \subseteq ®X$ uzavřená. Ať ©U je otevřené pokrytí $F$ (otevřenými množinami v $F$). Z definice podprostoru $\forall U \in ©U\ \exists V_U$ otevřená v $®X: U = F \cap V_U$. Uvažujme $©V = \{V_U|U \in ©U\} \cup \{F \setminus  F\}$. ©V je otevřené pokrytí ®X. Existuje otevřené lokálně konečné zjemnění ©W tohoto ©V. $\{F \cap W | W \in ©W\}$ je otevřené pokrytí $F$ a zároveň lokálně konečné. Navíc je to i zjemnění ©U.
        \end{dukazin}
    \end{tvrzeni}

    \begin{veta}[Charakterizace parakompaktnosti]
        Pro regulární TP ®X jsou následující podmínky ekvivalentní:
        
        \begin{itemize}
            \item[a)] ®X je parakompaktní.
            \item[b)] Každé otevřené pokrytí ®X má otevřené $\sigma$-lokálně konečné zjemnění.
            \item[c)] Každé otevřené pokrytí ®X má lokálně konečné zjemnění (libovolnými množinami).
            \item[d)] Každé otevřené pokrytí ®X má uzavřené lokálně konečné zjemnění. 
        \end{itemize}

        \begin{dukazin}
            $a) \implies b):$ každé lokálně konečné zjemnění je $\sigma$-lokálně konečné.

            $b) \implies c):$ Ať ®U je otevřené pokrytí ®X. Podle b) existuje otevřené zjemnění $©V = \bigcup_{n=1}^∞ ©V_n$, $©V_n$ lokálně konečný systém. $W_n:=\bigcup ©V_n$ je otevřené $\{W_n | n \in ®N\}$ je otevřené pokrytí ®X. Ať $A_n := W_n \setminus \bigcup_{i < n} W_i$. $\{A_n | n \in ®N\}$ je lokálně konečné pokrytí ®X (každé $x \in ®X$ je v nějakém $W_n$, takže už není ve větších $A_n$). $\{A_n \cap V | n \in ®N, V \in ©V_n\}$ je lokálně konečné zjemnění ©U.

% 5. 3. 2021

            $c) \implies d):$ Ať ©U je otevřené pokrytí ®X. Pro každé $x \in ®X$ existuje $U_x \in U: x \in U_x$. Nyní máme bod v otevřené množině, tedy z regularity existují otevřené množiny $V_x \subseteq ®X: x \in V_x \subseteq \overline{V_x} \subseteq U_x$. $©V:=\{V_x | x \in ®X\}$ je otevřené pokrytí ®X. ©V má lokálně konečné zjemnění $©W$ podle $c)$. $\{\overline{W} | W \in ©W\}$ je lokálně konečný systém podle lemmatu „Uzávěr lokálně konečného systému“. Navíc je i pokrytí a zjemňuje ©U.

            $d) \implies a)$ Ať ©U je otevřené pokrytí ®X. Z d) existuje lokálně konečné uzavřené zjemnění ©V. Pro $x \in ®X$ existuje $W_x$ otevřené okolí $x$ protínající jen konečně mnoho prvků z ©V. $©W:=\{W_x | x \in ®X\}$ je otevřené pokrytí ®X. Z d) existuje lokálně konečné uzavřené zjemnění ©A toho ©W. Pro $V \in ©V$ označíme $V^*:=®X \setminus \bigcup\{A \in ©A | A \cap V = \O\}$. Zřejmě $V^* \supseteq V$. Tedy $\{V^* | V \in ©V\}$ je otevřené (odčítáme uzavřenou množinu, neboť $A$ jsou uzavřené a množina je lokálně konečná, tedy podle lemmatu … je uzavřené i sjednocení) pokrytí.

                    Ať $x \in ®X$. $\exists U$ okolí $x$, které protíná jen konečně prvků $A_1, …, A_n \in ©A$. Zřejmě $U \subseteq A_1 \cup … \cup A_n$. Každé $A_i$ je podmnožinou nějakého $W_y$, tj. (podle volby $W_y$) $A_i$ protíná jen konečně mnoho prvků z ©V. Navíc je-li $V \in ©V$ a $A \in ©A$, že $A \cap V = \O$, pak $A \cap V^* = \O$. Tedy každé $A_i$ protíná pouze konečně mnoho prvků $V^*$, $V \in @V$. Pro každé $V \in @V$ fixujeme $U_v \in ©U: V \subseteq U_V$. Zřejmě $V \subseteq U_V \cap V^*$. Pak $\{U_V \cap V^* | V \in ©V\}$ je otevřené pokrytí ®X, které je lokálně konečné a které je zjemnění $©U$.
        \end{dukazin}
    \end{veta}

    \begin{dusledek}
        Každý Lindelöfův regulární prostor je parakompaktní.

        \begin{dukazin}
            Ať ©U je otevřené pokrytí ®X. Z lindolöfovosti existuje spočetné pokrytí $©V \subseteq ©U$. ©V je $\sigma$-lokálně konečné otevřené zjemnění ©U. Tedy platí b) z minulé věty.
        \end{dukazin}
    \end{dusledek}

    \begin{definice}[Skrčení]
        Ať $X$ je množina a $©S \subseteq ©P(X)$ (pokrytí $X$). Indexovaný systém $\{T_S: S \in ©S\} \subseteq ©P(X)$ se nazývá skrčení systému ©S, pokud (je to pokrytí) a $T_S \subseteq S, S \in ©S$.
    \end{definice}

    \begin{poznamka}[Nadmutí]
        Skrčení je speciální případ zjemnění.

        Podobně jako skrčení lze definovat pojem nadmutí.
    \end{poznamka}

    \begin{lemma}[O skrčení]
        Ať ®X je normální TP. Pak každé lokálně konečné (stačí bodově konečné) otevřené pokrytí ®X má uzavřené skrčení, jehož vnitřky tvoří pokrytí.

        \begin{dukazin}
            Ať $©U = \{U_\alpha: \alpha < \kappa\}$, $\kappa$ kardinál, ©U je lokálně kompaktní, otevřené pokrytí ®X. Nyní $F_0:=®X \setminus \bigcup \{U_\alpha: 0 < \alpha < \kappa\}$ uzavřená, $F_0 \subseteq U_0$ (z toho, že ©U je pokrytí). Z normality existuje otevřená $V_0 \subseteq ®X: F_0 \subseteq V_0\subseteq \overline{V_0} \subseteq U_0$.

            Nyní indukcí: Nechť máme zkonstruované $V_\beta: \forall \beta < \alpha < \kappa$. Označíme $F_\alpha := ®X \setminus \(\bigcup \{V_\beta: \beta < \alpha\}\cup \bigcup \{U_\gamma: \alpha < \gamma < \kappa\}\)$. Z normality zas $V_\alpha \subseteq ®X: F_\alpha \subseteq V_\alpha\subseteq \overline{V_\alpha} \subseteq U_\alpha$.

            $©V = \{\overline{V_\alpha}: \alpha < \kappa\}$ je skrčení ©U, $\Int\overline{V_\alpha} \supseteq V_\alpha$ a $\bigcup_{\alpha < \kappa} V_\alpha = ®X$, tedy $\bigcup_{\alpha < \kappa} \Int\overline{V_\alpha} = ®X$.
        \end{dukazin}
    \end{lemma}

    \begin{definice}[Kolektivně normální]
        TP ®X se nazývá kolektivně normální, pokud pro každý diskrétní systém ©F z uzavřených množin existuje disjunktní systém otevřených množin $\{U(F): F \in ©F\}$, že $F \subseteq U(F), F \in ©F$ (tj. otevřené nadmutí).
    \end{definice}

    \begin{poznamka}
        Každý kolektivně normální prostor je normální.
    \end{poznamka}

    \begin{tvrzeni}
        Každý parakompaktní prostor už je kolektivně normální, tedy i normální.
        
        \begin{dukazin}
            Ukážeme nejprve, že ®X je regulární. Ať $F \subseteq ®X$ uzavřená, $x \in ®X \setminus F$. Pro $y \in F$ existuje otevřené okolí $U_y$ bodu $y$, že $x \notin \overline{U_y}$. $©U := \{U_y: y \in F\}\cup \{®X \setminus F\}$ otevřené pokrytí ®X. Ať ©V je lokálně konečné otevřené zjemnění ©U. $G := \bigcup \{V \in ©V: V \cap F ≠ \O\}$. Z lemmatu $\overline{G} = \bigcup\{\overline{V}: V \in ©V, V \cup F ≠ \O\} \not\ni x$. $G \supset F$, $G$ otevřená. Tedy ®X je regulární.

            Ať ©F je diskrétní soubor z uzavřených množin. Pro $F \in ©F$ uvážíme $\bigcup\{H \in ©F: H≠F\}$ … uzavřená z lemmatu o uzávěru sjednocení lokálně kompaktního systému. Pro $x \in F$ existuje (z první části důkazu) $U_x$ otevřená, že $x \in U_x$, $\overline{U_x} \cap H = \O$ pro $H ≠ F, H \in ©F$. $\{U_x : x \in F \in ©F\} \cup \{®X \setminus \bigcup F\}$ je otevřené pokrytí ®X. Ať ©V je otevřené lokálně konečné zjemnění. Pro $F \in ©F: V(F):= \{V \in ©V: V \cup F ≠ \O\} \setminus \bigcup \{\overline{V}: V \in ©V, V \cap H ≠ \O \text{ pro nějaké } H \in ©F, H ≠ F\}$. Platí $F \subseteq V(F)$. Pro $F, F' \in ©F, F ≠ F' \implies V(F) \cap V(F') = \O$. $\{V(F): F \in ©F\}$ je disjunktní otevřené nadmutí ©F.
        \end{dukazin}
    \end{tvrzeni}

    \begin{definice}[Hvězda]
        Ať ®X je množina a $©S \subseteq ©P(®X)$, $x \in ®X$, $A \subseteq ®X$.

        Hvězda bodu $x$ vzhledem k ©S je $\st(x, ©S) = \bigcup \{S \in ©S: x \in S\}$.

        Hvězda množiny $A$ vzhledem k @S je $\st(A, ©S) = \bigcup_{x \in A}\st(x, ©S)$.
    \end{definice}

    \begin{definice}[Barycentrické a hvězdovité zjemnění]
        Ať ©U, ©V jsou pokrytí ®X. Řekneme, že ©U barycentricky zjemňuje ©V, pokud $\{\st(x, ©U): x \in ®X\}$ zjemňuje ©V.

        Řekneme, že ©U hvězdovitě zjemňuje ©V, pokud $\{\st(U, ©U): U \in ©U\}$ zjemňuje ©V.
    \end{definice}

    \begin{priklady}
            Ať $(®X, \rho)$ je MP. Ať ©U, ©V, ©W jsou pokrytí ®X tvořená po řadě všemi $\epsilon, 2\epsilon, 3\epsilon$ koulemi ($\epsilon > 0$ pevné). Pak ©U zjemňuje barycentricky ©V a hvězdovitě ©W.
    \end{priklady}

    \begin{lemma}[Dvojité barycentrické zjemnění je hvězdovité]
        Ať $X$ je množina, ©U pokrytí ©X, ©V barycentrické zjemnění ©U a ©W barycentrické zjemnění ©V. Potom ©W je hvězdovité zjemnění ©U.

% 12. 3. 2021

        \begin{dukazin}
            Mějme $W \in ©W$ libovolně. Chceme najít $U \in ©U: \st(W, ©W) \subseteq U$. $W = \O$ triviální. $W ≠ \O:$ Fixujeme $x_0 \in W$. Pro každé $x \in ®X$ existuje $V_x \in ©V$: $\st(x, ©W) \subseteq V_x$. Nyní 
            $$ \st(W, ©W) = \bigcup\{T \in ©W: T \cap W ≠ \O\} = \bigcup\{\{T \in ©W| x \in T\} | x \in W\} = \bigcup\{\st(x, ©W) | x \in W\} \subseteq \bigcup\{V_x| x \in W\} \subseteq \st(x_0, ©V), $$
            protože $W \subseteq V_x$ pro každé $x \in W$. ©V barycentricky zjemňuje ©U, tedy existuje $u \in ©U: \st(x_0, ©V) \subseteq U$.
        \end{dukazin}
    \end{lemma}

    \begin{veta}[Charakterizace parakompaktnosti pomocí hvězdovitých zjemnění]
        Pro TP ®X je ekvivalentní:

        \begin{itemize}
            \item[a)] ®X je parakompaktní.
            \item[b)] Každé otevřené pokrytí ®X má barycentrické zjemnění.
            \item[c)] Každé otevřené pokrytí ®X má hvězdovité zjemnění.
            \item[d)] Každé otevřené pokrytí ®X má otevřené $\sigma$-diskrétní zjemnění a ®X je regulární.
        \end{itemize}

        \begin{dukazin}
            $a) \implies b)$ Ať ©U je otevřené pokrytí ®X. Z a) vyplývá, že existuje jeho lokálně konečné otevřené zjemnění ©V. Víme, že ®X je parakompaktní, tedy normální. Z lemmatu o skrčení existuje uzavřené pokrytí $©W = \{W_V | V \in ©V\}$, $W_V \subseteq V$. ©V je lokálně konečné, tedy i ©W je lokálně konečné. Pro $x \in ®X$ definujeme $A_x = \bigcap\{V | x \in W_V\}$. Jde o konečný průnik (vzhledem k lokální kompaktnosti), tedy $A_x$ je otevřená. Položme $B_x = \bigcup\{W \in ©W | x \notin W\}$. Podle lemmatu o sjednocení lokálně konečného systému je $B_x$ uzavřená. Zřejmě $x \in A_x \setminus B_x =: C_x$ je otevřená. Tedy $©C = \{C_x|x \in ®X\}$ je otevřené pokrytí ®X.

                Ukážeme, že ©C barycentricky zjemňuje ©U: Ať $y \in ®X$. Chceme najít $V \in ©U$: $\st(y, ©C) \subseteq V$. Víme, že existuje $V \in ©V: y \in W_V$. Ať $x \in \st(y, ©C)$. Pak $y \in C_x = A_x \setminus B_x$, tedy $y \notin B_x$, tudíž $x \in W_V \subseteq V$ (kdyby ne, pak $W_V \subseteq B_x$, tedy $y \notin C_x$).

            $b) \implies c)$ k otevřenému pokrytí můžeme najít barycentrické zjemnění, ke kterému můžeme najít barycentrické zjemnění. Pak c) vyplývá z předchozího lemmatu.

            $c) \implies d)$ ®X je regulární: Ať $F \subseteq ®X$ uzavřená, $x \in ®X \setminus F$. Uvažujme otevřené pokrytí $\{®X \setminus F, ®X \setminus {x}\}$. Podle c) existuje otevřené hvězdovité zjemnění ©U. $\exists U \in ©U: x \in U$. Nutně $U \cap F = \O$. Pak $\overline{U} \subseteq \st(U, ©U) \subseteq ®X \subseteq ®X \setminus F$. Tedy ®X je regulární.

                Ať $©U_0$ je otevřené pokrytí ®X. Chceme najít $\sigma$-diskrétní zjemnění toho $©U_0$. Použijeme podmínku c) spočetně nekonečněkrát, abychom induktivně našli otevřená pokrytí $©U_1, ©U_2, …$, že $©U_{n+1}$ hvězdovitě zjemňuje $©U_n, n≥0$. Oindexujme prvky $©U_0: ©U_0 = \{U_i|i \in I\}$. Pro $i \in I$ a pro $n \in ®N$ uvažujme $U_{i, n} := \{x \in ®X|x \text{ má okolí } V: \st(V, ©U_n) \subseteq U_i\}$. Pro každé $n \in ®N: \{U_{i, n}| i \in I\}$ je otevřené zjemnění ©U, ale ne nutně pokrytí.

                Pomocné tvrzení: Pokud $x \in U_{i, n}, u \notin U_{i, n+1}$, pak neexistuje $U \in ©U_{n+1}$, že $x, y \in U$. Důkaz: Pro $U \in ©U_{n+1}$ existuje $W \in ©U_n: \st(U, ©U_{n+1})\subseteq W$. Tedy pokud $x \in U\cap U_{i, n}$, pak $W \subseteq \st(x, ©U_n) \subseteq U_i$. Pak $\st(U, ©U_{n+1})\subseteq U_i$ a $u \subseteq U_{i, n+1}$. Tedy $y \notin U$, protože $y \notin U_{i, n+1}$.

                Uvažme dobré uspořádání $<$ na $I$. Ať $V_{i_0, n} = U_{i_0, n} \setminus \bigcup \{U_{i, n+1} | i < i_0\}$, $i_0 \in I, n \in ®N$. Ukážeme, že $** = \{V_{i_0, n} | i_0 \in I, n \in ®N\}$ je hledané $\sigma$-diskrétní zjemnění $©U_0$. Pro $i_1 ≠ i_2, i_1, i_2 \in I$, pak $i_1 < i_2$ nebo naopak. Podle toho buď $V_{i_2, n} \subseteq ®X \setminus U_{i_1, n+1}$ nebo $V_{i_1, n} \subseteq ®X \setminus U_{i_2, n+1}$. Podle pomocného tvrzení platí, že pokud $x \in V_{i_1, n}$ a $y \in V_{i_2, n}$, pak neexistuje $U \in ©U_{n+1}$, že $x, y \in U$. To nám říká, že $\forall n \in ®N: \{V_{i, n} | i \in I\}$ je diskrétní. Zbývá už jen ukázat, že $**$ je pokrytí: Ať $y \in ®X$. Existuje $<$-nejmenší $i(y) \in I: y \in U_{i(y), n}$ pro nějaké $n \in ®N$. Nyní $y \notin U_{i, n+2}$ pro $i <i(y)$. Podle pomocného tvrzení použitého na $n+1$ platí $\st(y, ©U_{n+2}) \cap \bigcup \{U_{i, n+1}|i < i(y)\} = \O$. Tedy $y \in V_{i(y)}, n$.

                $d) \implies a)$ Víme, že ®X je regulární, tedy můžeme aplikovat charakterizaci parakompaktnosti z minulého týdne, jelikož $\sigma$-diskrétní $\implies \sigma$-lokálně konečný.
        \end{dukazin}
    \end{veta}

    \begin{veta}[Stone]
        Každý metrizovatelný prostor je parakompaktní.

        \begin{dukazin}
            Ukážeme, že každé otevřené pokrytí ©U má barycentrické zjemnění. Fixujeme na nějakém tom prostoru ®X kompatibilní metriku $\rho ≤ 1$. Navíc búno $®X \notin ©U$. Pro každé $x \in ®X$ a $U \in ©U$, že $x \in U$, existuje největší možné $\epsilon_{x, U}>0$, že $B(x, 5\epsilon_{x, U})$. Položíme $©V = \{B(x, \epsilon_{x, U})|x \in U \in ©U\}$. Ověříme, že $©V$ barycentricky zjemňuje $©U$: Ať $x \in ®X$. Chceme najít $U \in ©U: \st(x, ©V) \subseteq U$. Ať $\epsilon_x = \sup\{\epsilon_{x, U} | x \in U \in ©U\}$. $0 < \epsilon_x ≤ 1$. Existuje $U \in ©U: \epsilon_{x, U} ≥ \frac{\epsilon_x}{2}$.

            Ukážeme, že $\st(x, ©V) \subseteq U$. Ať tedy $x \in B(y, \epsilon_{y, v})$ pro nějaké $y \in V \in ©U$. Chceme $B(y, \epsilon_{y, v}) \subseteq U$. Máme $B(y, 5\epsilon_{y, v}) \subseteq V$ a zároveň $\rho(x, y) < \epsilon_{U, V}$. Z $\triangle$-nerovnosti: $B(x, 4\epsilon_{y, V}) \subseteq V$. Z maximality $\epsilon_{x, V} ≥ \frac{1}{5}4\epsilon_{y, V}$. Také $2\epsilon_{x, U} > \epsilon_x ≥ \epsilon_{x, V}$. Dohromady $2\epsilon_{x, U} > \frac{4}{5} \epsilon_{y, V}$, tj. $5\epsilon_{x, U} > 2\epsilon_{y, V}$. Pro $z \in B(y, \epsilon_{y, V}): \rho(x, z) < 2\epsilon_{y, V}$, a tedy $\rho(x, z) < 5\epsilon_{x, U}$. Proto $z \in U$. Tudíž $B(y, \epsilon_{y, v}) \subseteq U$.
        \end{dukazin}
    \end{veta}

% 19. 3. 2021

    \begin{definice}
        Pro funkci $f: X \rightarrow ®R$ značíme $\supp f = \overline{\{x \in X: f(x) ≠ 0\}}$.
    \end{definice}

    \begin{veta}[Rozklad jednotky]
        Ať ®X je parakompaktní prostor, ©U otevřené pokrytí ®X. Pak existuje rozklad jednotky podřízený tomuto pokrytí, tj. systém spojitých funkcí $f_i: X \rightarrow [0, 1]$, $i \in I$, že $\{\supp f_i: i \in I\}$ je lokálně konečné zjemnění ©U a $\sum_{i \in I} f_i(x) = 1, \forall x \in ®X$.

        \begin{dukazin}
            ®X parakompaktní, tedy normální. Tedy existuje otevřené pokrytí ©W takové, že $\{\overline{W}: W \in ©W\}$ zjemňuje ©U. Ať ©V je lokálně konečné otevřené zjemnění ©W. Víme, že existuje uzavřené skrčení $\{F_V: V \in ©V\}$, $F_V \subseteq V$. Z normality existují spojité funkce $g_V: ®X \rightarrow [0, 1]$, $g_V|_{F_V} = 1$, $g_V|_{®X \setminus V} = 0$. Položme $g(x):=\sum_{V \in ©V} g_V(x)$. Funkce $g$ je spojitá, protože spojitost je lokální pojem a $g$ je lokálně součet konečně mnoha nenulových spojitých funkcí. Navíc zřejmě $g ≥ 1$, protože $\{F_V: V\in©V\}$ je pokrytí ®X. Tedy položme $f_V:= \frac{g_V}{g}$.
        \end{dukazin}
    \end{veta}

    \begin{veta}[Michaelova selekční]
        Zdola polospojitá (vícehodnotová) funkce z parakompaktního prostoru do neprázdných uzavřených konvexních podmnožin Banachova prostoru má spojitou selekci.
    \end{veta}

    \begin{veta}[Dugunjiho]
        Ať ®X je metrizovatelný a $A \subseteq ®X$ uzavřená. Pak existuje lineární zobrazení $L: C(A, ®R) \rightarrow C(®X, ®R)$, že $L(f)$ rozšiřuje $f$ pro $f \in C(A, ®R)$.
    \end{veta}

\section{Metrizační věty}
    \begin{poznamka}[Opakování]
        Uryshonova metrizační věta: Regulární prostor se spočetnou bází je metrizovatelný.
    \end{poznamka}

    \begin{veta}[Bing, Nagata, Smirnov]
        Pro regulární prostor ®X jsou následující podmínky ekvivalentní:
        
        \begin{itemize}
            \item[a)] ®X je metrizovatelný.
            \item[b)] ®X má $\sigma$-diskrétní bázi.
            \item[c)] ®X má $\sigma$-lokálně konečnou bázi.
        \end{itemize}

        \begin{dukazin}
            $a) \implies b)$: Ať $©B_n$ je otevřené pokrytí ®X koulemi o poloměru $\frac{1}{n}$. ®X je parakompaktní podle Stoneovy věty. Z charakterizace parakompaktnosti máme, že $©B_n$ má $\sigma$-diskrétní otevřené zjemnění $©V_n$. $\bigcup_{n \in ®N} ©V_n$ je opět $\sigma$-diskrétní, navíc je to báze.

            $b) \implies c)$: triviální.

            $c) \implies a)$ Ať $B = \bigcup_{n=1}^∞$ je báze $®X$, $©B_n$ lokálně konečný soubor. Uvědomíme si, že ®X je parakompaktní: Je-li totiž $©U$ otevřené pokrytí ®X, pak $\{B \in ©B: \exists U \in ©U: B \subseteq U\}$ je zjemnění $U$ a vzhledem k tomu, že $B$ je báze, tak je to i pokrytí. Navíc je $\sigma$-lokálně konečné. Tedy z charakterizace parakompaktnosti to máme.

            Z parakompaktnosti dostáváme normalitu ®X. Pro $n, k \in ®N$ a $B \in ©B_n$ položme $V_{k, n, B}:= \bigcup\{C \in ©B_k: \overline{C} \subseteq B\}$. $©B_k$ je lokálně konečný, tedy (z lemmatu o uzávěru lokálně konečného systému) $\overline{V_{k, n , B}} \subseteq B$. Tedy existují (z normality) spojité funkce $f_{k, n, B}: ®X \rightarrow [0, 1]$, $f_{k, n, B}(x) = 0$ pro $x \in ®X \setminus B$ a $1$ pro $x \in \overline{V_{k, n, B}}$.

            Definujeme $M_{k, n} \subseteq [0, 1]^{©B_n}$ následovně $M_{k, n} = \{\phi: ©B_n \rightarrow [0, 1]: \{B \in ©B_n: \phi(B) ≠ 0\} \text{ je konečná}\}$. Na $M_{k, n}$ uvažme metriku $\rho_{k, n}{\phi, \psi} := \sum_{B \in ©B_n} |\phi(B) - \psi(B)|$. Ať $g_{k, n}: ®X \rightarrow M_{k, n}$, $g_{k, n} = \triangle_{B \in ©B_n} f_{k, n, B}$, $g_{k, n}(x) = (f_{k, n, B}(x))_{B \in ©B_n}$.

            Ověříme, že $g_{k, n}: ®X \rightarrow (M_{k, n}, \rho_{k, n})$ je spojité: Ať $x \in ®X$, $\epsilon > 0$, existuje $U$ okolí $x$ protínající jen konečně prvků $B_1, …, B_m \in ©B_n$. $f_{k, n, B_1}, …, f_{k, n, B_m}$ jsou spojitá, tedy existuje $V \subseteq U$ okolí $x$, že $|f_{k, n, B}(x) - f_{k, n, B_i}(y)| < \frac{\epsilon}{m}$ pro $i ≤ m, y \in V$. Nyní
            $$ \rho_{k, n}(g_{k, n}(x), g_{k, n}(y)) = \sum_{i=1}^m |g_{k, n}(x)(B_i) - g_{k, n}(y)(V_i)| = \sum |f_{k, n, B}(x) - f_{k, n, B_i}(y)| < m·\frac{\epsilon}{m} = \epsilon. $$

            Pokud systém $\{g_{k, n}: k, n \in ®N\}$ odděluje body a uzavřené množiny, pak $\delta:= \triangle_{k, n \in ®N}g_{k, n}:®X \rightarrow \prod_{k, n \in ®N} M_{k, n}$ je vnoření (podle lemmatu o Tichonovově vnoření). Tím jsme vnořili ®X do spočetného součinu metrizovatelných prostorů, tedy do metrizovatelného prostoru, tedy ®X je metrizovatelné.

            $\{g_{k, n}: k, n \in ®N\}$ odděluje body a uzavřené množiny: Ať $F \subseteq ®X$ je uzavřená, $x \in ®X\setminus F$. Existuje $n \in ®N$ a $B \in ©B_n: x \in B \subseteq X \setminus F$. Z regularity existuje $C \in ©B_k, k \in ®N$. $g_{k, n}(x)(B) = f_{k, n, B}(x) = 1$ a $g_{k, n}(y)(B) = f_{k, n, B}(y) = 0$ pro $y \in ®X \setminus B \supseteq F$.
        \end{dukazin}
    \end{veta}

    \begin{definice}
        Ať ®X je TP. Posloupnost otevřených pokrytí $©V_n$ prostoru ®X se nazývá development, pokud pro každé $x \in ®X: \{\st(x, ©V_n)| n \in ®N\}$ je báze okolí v bodě $x$.
    \end{definice}

    \begin{poznamka}
        Je-li $(X, \rho)$ MP, pak $©V_n:= \{B(x, \frac{1}{n}): x \in ®X\}, n \in ®N$ je development ®X.
    \end{poznamka}

    \begin{veta}[Bing]
        TP ®X je metrizovatelný $\Leftrightarrow$ je kolektivně normální a má development.

        \begin{dukazin}
            $\implies$: metrizovatelný $\implies$ má development (podle předchozí poznámky) a metrizovatelný $\implies$ parakompaktní $\implies$ kolektivně normální.

% 26. 3. 2021

            $\Leftarrow$: Dokážeme ve 4 částech:

            1. Pro diskrétní soubor $©F = \{F_\alpha\}_{\alpha \in A}$ uzavřených množin v ®X existuje diskrétní soubor otevřených množin $©W = \{W_\alpha\}_{\alpha \in A}$, že $F_\alpha \subseteq W_\alpha$: Dle kolektivní normality existují otevřené disjunktní $U_\alpha: \alpha \in A, F_\alpha \subseteq U_\alpha$. Položme $F = \bigcup ©F$, $Z = ®X \setminus \bigcup_{\alpha \in A} U_\alpha$. $F$ uzavřená (sjednocení lokálního systému uzavřených množin), $Z$ uzavřená. ®X je kolektivně normální, tedy speciálně normální, tedy existují otevřené disjunktní $V, W$, že $Z \subseteq V$ a $F \subseteq W$. Položme $W_\alpha := U_\alpha \cap W$. Systém $\{W_\alpha\}$ už je diskrétní (je-li $x \in Z$, pak $x \in V$ a $V \cap W_\alpha = \O$, je-li naopak $x$ v $U_\alpha$, pak $U_\alpha \cap W_\beta = \O$ pro $\beta ≠ \alpha$) a $F_\alpha \subseteq W_\alpha$.

            2. Ať $©V_n$ je development prostoru ®X. Buď $\kappa ≥ \omega$ a očíslujme $©V_n = \{V_{\alpha, n} | \alpha < \kappa\}$ (s případným opakováním prvků). Položme $D_{\alpha, n, k} = \{x \in V_{\alpha, n} | \st(x, V_k) \subseteq V_{\alpha, n}\}$ a $C_{\alpha, n, k} = D_{\alpha, n, k} \setminus \bigcup_{\beta < \alpha} V_{\beta, n}$. $D_{\alpha, n, k}$ (a tudíž i $C_{\alpha, n, k}$) je uzavřená:

                Volme $x \in \overline{D_{\alpha, n, k}}$. Pak pro libovolné $V \in ©V_k$, že $x \in V$ platí, že existuje $y \in V \cap D_{\alpha, n, k}$. Pak $V \subseteq \st(y, ©V_k) \subseteq V_{\alpha, n}$. Tedy $\st(x, ©V_k) = \bigcup\{V \in ©V_k | x \in V\} \subseteq V_{\alpha, n}$. Tedy $x \in D_{\alpha, n, k}$, tudíž $D_{\alpha, n, k}$ je uzavřená.

            3. Pro pevná $n, k \in ®N$ je $\{C_{\alpha, n, k} | \alpha < \kappa\}$ diskrétní: Buď $y \in ®X$ libovolné. Pak existuje nejmenší $\beta < \kappa: y \in V_{\beta, n}$. Najděme $V \in ©V_k: y \in V$. Pro $\alpha > \beta: V_{\beta, n}$ je disjunktní s $C_{\alpha, n, k}$ a pro $\alpha < \beta: V$ je disjunktní s $C_{\alpha, n, k}$ (kdyby existovalo $z \in V \cap C_{\alpha, n, k}$ pak $\st(z, ©V_k) \subseteq V_{\alpha, n}$, speciálně $y \in V_{\alpha, n}$, což je spor s minimalitou $\beta$). Tedy $V \cap V_{\beta, n}$ je okolí bodu $y$, které protíná nejvýše jeden prvek systému $\{c_{\alpha, n, k} | \alpha \in A\}$ (a sice prvek $C_{\beta, n, k}$).

            4. $\{C_{\alpha, n, k}\}$ je diskrétní soubor uzavřených množin (podle 2, 3). Podle 1 existuje diskrétní soubor otevřených nadmnožin $\{V_{\alpha, n, k} | \alpha < \kappa\}$. Tedy $©V_{n, k} := \{V_{\alpha, n, k} \cap V_{\alpha, n} | \alpha < \kappa\}$ je diskrétní (zmenšili jsme jeho množiny). Ukážeme, že $©V := \bigcup_{n, k \in ®N} ©V_{n, k}$ je báze ®X:

            Ať $U \subseteq ®X$ je otevřená, $x \in U$. $\exists n \in ®N: \st(x, ©V_n) \subseteq U$. Najdeme $\alpha$ nejmenší možné, že $x \in V_{\alpha, n}$. Zřejmě $V_{\alpha, n} \subseteq U$. Opět z vlastností developmentu existuje $k \in ®N: \st(x, ©V_k) \subseteq V_{\alpha, n}$. Nyní $x \in C_{\alpha, n, k}$, tedy $x \in V_{\alpha, n, k} \cap V_{\alpha, n} \subseteq U$. Tudíž ©V je báze ®X.

            ©V je $\sigma$-diskrétní báze ®X, tedy podle metrizační věty Bing-Nagata-Smirnov je ®X metrizovatelný.
        \end{dukazin}
    \end{veta}

\section{Uniformní prostory}
    \begin{poznamka}
        Zavedeno např. díky tomu, že stejnoměrnou spojitost nelze charakterizovat pomocí topologie.

        Matematici Weil(1936), Tukey(1940) … prvotní zkoumání UP.
    \end{poznamka}

    \begin{definice}[Značení]
        Pro množinu ®X značíme $\triangle(X) = \{(x, x) | x \in X\}$.

        Pro $E \subseteq X \times X$ značíme $E^{-1} = \{(y, x) | (x, y) \in E\}$.

        Pro $C, D \in X \times X$ značíme $C \circ D = \{(x, z) \in X \times X | \exists y \in X: (x, y) \in C \land (y, z) \in D\}$.

        $E[x] = \{y \in X | (x, y) \in E\}$.
    \end{definice}

    \begin{definice}[Uniformní prostor (UP)]
        Dvojice $(®X, ©D)$ se nazývá uniformní prostor (UP), pokud $®X$ je množina a $©D \subseteq ©P(X \times X), ©D ≠ 0$ splňující

        \begin{enumerate}
            \item $\forall D \in ©D: \triangle(®X) \subseteq D$,
            \item $\forall C, D \in ©D: C \cap D \in ©D$,
            \item $\forall D \in ©D\ \exists C \in ©D: C \circ C \subseteq D$,
            \item $\forall D \in ©D: D^{-1} \in ©D$,
            \item $\forall D \in ©D\ \forall E \subseteq X \times X: D \subseteq E \implies E \in ©D$,
            \item $\forall x, y \in X: x≠y \implies \exists D \in ©D: (x, y) \notin D$. $(\Leftrightarrow \bigcap ©D = \triangle(®X).)$
        \end{enumerate}

        Prvky systému ©D nazýváme okolí diagonály.
    \end{definice}

    \begin{definice}[Báze uniformity]
        Systém $©B \subseteq ©P(®X^2)$ se nazývá báze uniformity (resp. báze uniformity ©D), pokud uzavřením ©B na nadmnožiny dostaneme ©D.
    \end{definice}

    \begin{definice}[Subbáze uniformity]
        Systém $©S \subseteq ©P(®X^2)$ tvoří subbázi uniformity (resp. uniformity ©D), pokud uzavřením na konečné průniky dostaneme bázi uniformity (resp. bázi uniformity ©D).
    \end{definice}

    \begin{definice}[Uniformní zobrazení]
        Jsou-li $(®X, ©D)$ a $(®Y, ©E)$ UP, $f: ®X \rightarrow ®Y$ se nazývá uniformní (stejnoměrně spojité), pokud $\forall E \in ©E: (f \times f)^{-1}(E) \in ©D$. ($\Leftrightarrow \forall E \in ©E\ \exists D \in ©D: (f \times f)(D) \subseteq E$.) ($\Leftrightarrow \forall E \in ©E\ \exists D \in ©D\ \forall x, y \in ®X: (x, y) \in D \implies (f(x), f(y)) \in E$.)
    \end{definice}

    \begin{definice}[Uniformní izomorfismus]
        Zobrazení $f$ se nazývá uniformní izomorfismus, pokud $f$ je bijekce a $f$ i $f^{-1}$ jsou uniformní.
    \end{definice}

    \begin{lemma}
        Systém $©B \subseteq ©P(®X^2)$ tvoří bázi nějaké uniformity na ®X, pokud
        $$ a) \bigcap ©B = \triangle(®X), $$ 
        $$ b) \forall C, D \in ©B\ \exists E \in ©B: E \subseteq C \cap D, $$
        $$ c) \forall D \in ©B\ \exists C \in ©B: C \circ C \subseteq D, $$ 
        $$ d) \forall D \in ©B\ \exists E \in ©B: E \subseteq D^{-1}. $$

        \begin{dukazin}
            $©D := \{C \subseteq ®X \times ®X | \exists B \in ©B: B \subseteq C\}$. Následně ověříme podmínky.
        \end{dukazin}
    \end{lemma}

% 9. 4. 2021

    \begin{tvrzeni}[Vytvoření UP z MP a TP z UP]
        Je-li $(®X, \rho)$ metrický prostor a $D_\epsilon = \{(x, y)| \rho(x, y) < \epsilon\}$, potom $\{D_\epsilon | \epsilon > 0\}$ je báze nějaké uniformity na ®X -- značíme ji $©D_\rho$. Tato uniformita se nazývá generovaná metrikou $\rho$.

        Je-li $(®X, ©D)$ UP, pak systém $\tau_{©D} = \{A \subseteq X | \forall x \in A\ \exists D \in ©D: D[x] \subseteq A\}$ je topologie na ®X a pro každé $x \in ®X$ tvoří systém $©B(x) := \{D[x] | D \in ©D\}$ bázi okolí v bodě $x$. Topologie $\tau_{©D}$ se nazývá generovaná uniformitou ©D.

        Pokud místo systému ©D použijeme v definici topologie $\tau_{©D}$ nějakou bázi ©D, pak dostaneme stejnou topologii. Zároveň také $\tau_{©D_\rho}$ je systém všech otevřených množin v $(®X, \rho)$.

        \begin{dukazin}
            Ověříme definice.
        \end{dukazin}
    \end{tvrzeni}

    \begin{definice}
        UP $(®X, ©D)$ se nazývá metrizovatelný, pokud existuje metrika $\rho$ na ®X, že $©D = ©D_\rho$. TP $(®X, \tau)$ se nazývá metrizovatelný, pokud existuje uniformita ©D na ®X, že $\tau = \tau_{©D}$
    \end{definice}

    \begin{definice}
        Ať $©U \subseteq ©P(®X)$, $A \subseteq ®X$, pak značíme $\st(A, ©U) = \bigcup\{U \in ©U | U \cap A ≠ \O\}$. $©U^* = \{\st(U, ©U) | U \in ©U\}$.

        Pro ©U, ©V pokrytí ®X, definujeme jejich společné zjemnění $©U \wedge ©V = \{U \cap V | U \in ©U, V \in ©V\}$.
    \end{definice}

    \begin{poznamka}
        Pro ©U, ©V pokrytí množiny ®X: ©U hvězdovitě zjemňuje ©V, pokud $©U^*$ zjemňuje ©V.
    \end{poznamka}

    \begin{definice}
        Ať ®X je množina, $¦U \subseteq ©P(©P(®X))$ se nazývá pokrývací uniformita na ®X, pokud:

        \begin{itemize}
            \item $\forall ©U \in ¦U: ©U$ je pokrytí ®X,
            \item Je-li $©U \in ¦U, ©V$ pokrytí ®X a ©U zjemňuje ©V, pak $©V \in ¦U$,
            \item $\forall ©U \in ¦U\ \exists ©V \in ¦U: ©V$ hvězdovitě zjemňuje ©U,
            \item $\forall ©U, ©V \in ¦U: ©U \wedge ©V \in ¦U$,
            \item $\forall x, y \in ®X, x ≠ y\ \exists ©U \in ¦U\ \forall U \in ©U: |\{x, y\} \cap U| ≤ 1$.
        \end{itemize}

        Prvky systému ¦U se nazývají uniformní pokrytí.
        
        Je-li ¦U pokrývací uniformita na ®X, pak položme
        $$ ©D_{¦U} := \{D \subseteq ®X \times ®X | \exists ©U \in ¦U\ \forall u \in ©U: U \times U \subseteq D\}. $$

        Je-li ©D (diagonální) uniformita na ®X, pak položme
        $$ ©U_{©D} := \{©U \in ©P(©P(®X)) | \exists D \in ©D: \{D[x] | x \in ®X\} \text{ zjemňuje} ©U\}. $$

        Přiřazení $¦U \mapsto ©D_{¦U}$ a $©D \mapsto ¦U_{©D}$ jsou navzájem inverzní bijekce systému všech pokrývacích uniformit na ®X a systém všech uniformit.
    \end{definice}

    \begin{lemma}[O pseudometrice]
        Ať $(®X, ©D)$ je UP a $D_i \in ©D$, $D_i = D_i^{-1}$, $i \in ®N_0$, $D_0 = ®X \times X$, $D_{i+1} \circ D_{i+1} \circ D_{i+1} \subseteq D_i$. Pak existuje pseudometrika $d$ na ®X, že pro každé $i ≥ 1$: $\{(x, y) : d(x, y) < \frac{1}{2^i}\} \subseteq D_i \subseteq \{(x, y) | d(x, y) ≤ \frac{1}{2^i}\}$.

% 16. 4. 2021

        \begin{dukazin}
            Položme $d(x, y) := \inf\{\frac{1}{2^{i_1}} + \frac{1}{2^{i_2}} + … + \frac{1}{2^{i_k}} | x_0, …, x_k \in ®X \land (x_{j-1}, x_j) \in D_{i_j} \land x = x_0, y=x_k\}$. $d(x, y)$ je pseudometrika na ®X. $D_i \subseteq \{(x, y) | d(x, y) ≤ \frac{1}{2^i}\}$ vidíme z toho, že pro $(x, y) \in D_i$ zvolíme $k=1$. Zbývá dokázat $\{(x, y) | d(x, y) < \frac{1}{2^i}\} \subseteq D_i$. Tedy chceme, že $d(x, y) < \frac{1}{2^i}$, pak $(x, y) \in D_i$, tj. že pro každou posloupnost $x_0, …, x_k$, kde $(x_{j_-1}, x_j) \in D_{i_j}$: pokud $\frac{1}{2^{i_1}} + … + \frac{1}{2^{i_k}} < \frac{1}{2^i}$, pak $(x_0, x_k) \in D_i$. To dokážeme indukcí podle $k$:

            Pro $k=1: \frac{1}{2^{i_1}} < \frac{1}{2^i}$, tj. $i < i_1$, tedy $(x_0, x_k) \in D_{i_1} \subseteq D_i$. Nyní předpokládejme, že $m > 1$ a pro všechna $k < m$ uvedené tvrzení platí. Uvažme posloupnost $x_0, …, x_m$, že $(x_{j-1}, x_j) \in D_{i_j}, j = 1, …, m$, a $\frac{1}{2^{i_1}} + … + \frac{1}{2^{i_m}} < \frac{1}{2^i}$. Zřejmě buď $\frac{1}{2^{i_1}} < \frac{1}{2^{i+1}}$, nebo $\frac{1}{2^{i_m}} < \frac{1}{2^{i+1}}$. Ze symetrie obou případů můžeme BÚNO předpokládat platnost první nerovnosti.

            Ať $n ≤ m-1$ je největší takové, že $\frac{1}{2^{i_1}} + … + \frac{1}{2^{i_n}} < \frac{1}{2^{i+1}}$. Pokud $n < m-1$, pak $\frac{1}{2^{i+1}} + … + \frac{1}{2^{i_{n+1}}} ≥ \frac{1}{2^{i+1}}$, tedy $\frac{1}{2^{i_{n+1}}} + … + \frac{1}{2^{i_m}} < \frac{1}{2^{i+1}}$. Podle indukčního předpokladu $x_0, n \in D_{i+1}$, $(x_{n+1}, x_m) \in D_{i+1}$. Navíc $\frac{1}{2^{i_{n+1}}} < \frac{1}{2^i}$, tedy $i < i_{n+1}$, $i + 1 ≤ i_{n+1}$, $D_{i_{n+1}} \subseteq D_{i+1}$. $(x_0, x_m) = (x_0, x_n)\circ(x_n, x_{n+1})\circ(x_{n+1}, x_m) \in D_{i+1}\circ D_{i+1}\circ D_{i+1} \subseteq D_i$.

            Pokud $n = m-1$, pak podle IP $x_0, x_{m-1} \in D_{i+1}$, a jelikož $\frac{1}{2^{i_m}} < \frac{1}{2^i}$, tak $(x_{m-1}, x_m)\in D_{i_m} \subseteq D_{i+1}$. Tedy $(x_0, x_m) \in D_{i+1}\circ D_{i+1} \subseteq D_i$.
        \end{dukazin}
    \end{lemma}

    \begin{veta}[Metrizovatelnost UP]
        UP $(®X, ©D)$ je metrizovatelný, právě když má spočetnou bázi uniformity.

        \begin{dukazin}
            $(\implies)$: Ať $d$ je metrika na ®X generující ©D. Pak $\{\{(x, y): d(x, y) < \frac{1}{n}\} | n \in ®N\}$ je báze ©D.

            $(\Leftarrow)$: Ať $\{C_n | n \in ®N\}$ je báze ©D. Indukcí najdeme posloupnost $D_n \in ©D$, že jsou splněny předpoklady předchozího lemmatu. A že $D_i \subseteq C_i$: Předpokládejme, že $D_0, …, D_n$ máme ($D_0 = ®X \times ®X$), pak víme, že $\exists E: E \circ E \subseteq D_n$, $\exists F: F \circ F \subseteq E$. Tedy $F \circ F \circ F \subseteq D_n$. Ať $D_{n+1} := (F \cap C_{i+1}) \cap (F \cap C_{i+1})^{-1}$. Tedy $D_{n+1}\circ D_{n+1} \circ D_{n+1} \subseteq D_n$, $D_{n+1} \subseteq C_{i+1}$. Tedy podle lemmatu o pseudometrice existuje pseudometrika $d$ na ®X, že
            $$ \{(x, y) | d(x, y) < \frac{1}{2^i}\} \subseteq D_i \subseteq \{(x, y) | d(x, y) ≤ \frac{1}{2^i}\}. $$
            Pro $x, y \in ®X, x≠y$ $\exists C \in ©D: (x, y) \notin C$. $\exists i \in ®N: C_i \subseteq C$. $D_i \subseteq C_i$. $(x, y)\notin D_i$. Tedy $d(x, y) ≥ \frac{1}{2^i} > 0$. Tedy $d$ je metrika. $d$ generuje uniformitu ©D: Tj. pro $\epsilon > 0$ $\{(x, y | d(x, y) < \epsilon)\} \in ©D$. To platí díky vlastnosti z předchozího lemmatu. A pro $D \in ©D\ \exists \epsilon > 0: \{(x, y) | d(x, y) < \epsilon\} \subseteq D$. $D \in ©D$ dané $\exists i \in ®N: C_i \subseteq D$, $D_i \subseteq C_i \subseteq D$. $\epsilon := \frac{1}{2^i}$. To máme také díky vlastnosti z předchozího lemmatu.
        \end{dukazin}
    \end{veta}

    \begin{veta}[Jemná uniformita]
        Ať $(®X, \tau)$ je TP. Všechny otevřené podmnožiny $®X \times ®X$ obsahující $\delta(®X)$ tvoří bázi nějaké uniformity na ®X právě tehdy, když ®X je parakompaktní.

        \begin{poznamkain}[Reformulace]
            Ať $(®X, \tau)$ je TP. Všechna otevřená pokrytí ®X tvoří bázi nějaké pokrývací uniformity na ®X, právě když ®X je parakompaktní.
        \end{poznamkain}

        \begin{dukazin}
            Pokud všechna otevřená pokrytí ®X tvoří bázi pokrývací uniformity, pak speciálně každé otevřené pokrytí má hvězdovité otevřené zjemnění. Tedy podle charakterizační věty je ®X parakompaktní.

            Je-li ®X parakompaktní, tak podle charakterizační věty má každé otevřené pokrytí ®X otevřené hvězdovité zjemnění a snadno se ověří, že $\{©U | ©U \text{ je pokrytí } ®X \text{, které je zjemňované nějakým otevřeným pokrytím } ®X\}$ je pokrývací uniformita na ®X.
        \end{dukazin}
    \end{veta}

    \begin{veta}[Uniformizovatelnost TP]
        TP je uniformizovatelný (tj. generován nějakou uniformitou), právě když je Tichonovův.

        \begin{dukazin}
            Ať $(®X, \tau)$ je generovaný uniformitou ©D. Buď $F \subseteq ®X$ uzavřená, $x \in ®X \setminus F$. Pak existuje $D \in ©D: D[x] \subseteq ®X \setminus F$. Ať $d$ je pseudometrika z předchozího lemmatu, kde volíme $D_1 = D$. Pak $\B_d(x, \frac{1}{2}) \subseteq D[x]$. Definujeme $f: ®X \rightarrow ®R$, $f(y) = d(x, y)$, $0 ≤ f ≤ 1$, $f(x) = 0$ a $f$ je spojitá. Pro $y \in F: f(y) ≥ \frac{1}{2}$. Tedy $®X$ je $T_\pi$.

            Ať $(®X, \tau)$ je Tichonovův. Pak uvažme $\beta®X$. $\beta®X$ je (para)kompaktní. Tedy na $\beta®X$ máme jemnou uniformitu, která generuje topologii na $\beta®X$. Tuto jemnou uniformitu na $\beta®X$ můžeme zúžit na ®X a ta již generuje topologii $\tau$.
        \end{dukazin}
    \end{veta}
\end{document}
