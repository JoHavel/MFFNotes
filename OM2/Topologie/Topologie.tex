\documentclass[12pt]{article}                   % Začátek dokumentu
\usepackage{../../MFFStyle}                     % Import stylu

\begin{document}

% 5. 3. 2021
\begin{poznamka}[Úmluva]
    Všechny topologické prostory v tomto semestru budou Hausdorffovy ($T_2$). Tedy regulární jsou automaticky $T_3$, úplně regulární jsou automaticky $T_\pi$ a normální jsou $T_4$.

    Speciálně např. kompaktní prostory jsou $T_4$.
\end{poznamka}

\section{Parakompaktní prostory}
    \begin{poznamka}[Připomenutí]
        Pokrytí, otevřené pokrytí, podpokrytí.
    \end{poznamka}

    \begin{definice}[Zjemnění]
        Ať $X$ je množina a ©S je pokrytí $X$. Řekneme, že systém $©T \in ©P(X)$ je zjemnění ©S, pokud $©T$ je pokrytí a $\forall T \in ©T\ \exists S \in ©S: T \subseteq S$.
    \end{definice}

    \begin{definice}[Lokálně konečný systém]
        Ať ®X je TP, $©S \subseteq ©P(®X)$. ©S se nazývá lokálně konečný, pokud
        $$ \forall x \in ®X\ \exists U \in ©U(x): \{S \in ©S| S \cap U ≠ \O\} \text{ je konečná.} $$

        Systém ©S se nazve diskrétní, pokud
        $$ \forall x \in ®X\ \exists U \in ©U(x): \left|\{S \in ©S| S \cap U ≠ \O\}\right| ≤ 1. $$

        Systém ©S se nazve $\sigma$-lokálně konečný (resp. $\sigma$-diskrétní), pokud $\exists S_n$, že $Š = \bigcup_{n=1}^∞$, že $©S_n$ jsou lokálně konečné (resp. diskrétní), $n \in ®N$.
    \end{definice}

    \begin{poznamka}
        Diskrétní systém je lokálně konečný. $\sigma$-diskrétní systém je $\sigma$-lokálně konečný.
    \end{poznamka}

    \begin{lemma}[Uzávěr lokálně konečného prostoru]
        Ať ®X je TP, $©A \subseteq ©P(X)$ lokálně konečný systém. Pak $\{\overline{A}| A \in ©A\}$ je opět lokálně konečný a platí $\overline{\bigcup ©A} = \bigcup \{\overline{A}| A \in ©A\}$.

        \begin{dukazin}
            Ať $x \in ®X$ je libovolné. Existuje $U \in ©U(x): \{A \in ©A: A \cap U ≠ \O\}$ je konečná. Ať $V = \int U, V \in ©U(x)$. $\{A \in ©A: A \cap V ≠ \O\}$ je zřejmě konečná. $V \cap A ≠ \O \Leftrightarrow V \cap \overline{A} ≠ \O$. Tedy $\{A \in ©A: \overline{A} \cap V ≠ \O\} = \{A \in ©A: A \cap V ≠ \O\}$. Tedy $\{\overline{A}| A \in ©A\}$ je konečná.

            $\supseteq:$ $\bigcup ©A \supseteq A, A \in ©A$, tedy $\overline{\bigcup ©A} \supseteq \overline{A} \implies \overline{\bigcup ©A} \supseteq \bigcup \{\overline{A}| A \in ©A\}$.

            $\subseteq:$ Ať $x \in \overline{\bigcup ©A}$. $\exists U \in ©U(x)$ otevřená, že $\{A \in ©A: A \cap U ≠ 0\} = \{A_1, …, A_n\}$. $x \in \overline{A_1 \cup … \cup A_n} \stackrel{\text{konečné}}{=} \overline{A_1} \cup … \cup \overline{A_n}$. $\exists i ≤ n: x \in \overline{A_i}$.
        \end{dukazin}
    \end{lemma}

    \begin{definice}[Parakompaktní]
        TP ®X se nazývá parakompaktní, pokud každé jeho otevřené pokrytí má lokálně konečné otevřené zjemnění.
    \end{definice}

    \begin{poznamka}
        Kompaktní $\implies$ parakompaktní (protože podpokrytí je zjemnění a konečné je lokálně konečné).

        Diskrétní TP $\implies$ parakompaktní.
    \end{poznamka}

    \begin{tvrzeni}
        Uzavřený podprostor parakompaktního TP je parakompaktní.

        \begin{dukazin}
            ®X je parakompaktní TP a $F \subseteq ®X$ uzavřená. Ať ©U je otevřené pokrytí $F$ (otevřenými množinami v $F$). Z definice podprostoru $\forall U \in ©U\ \exists V_U$ otevřená v $®X: U = F \cap V_U$. Uvažujme $©V = \{V_U|U \in ©U\} \cup \{F \setminus  F\}$. ©V je otevřené pokrytí ®X. Existuje otevřené lokálně konečné zjemnění ©W tohoto ©V. $\{F \cap W | W \in ©W\}$ je otevřené pokrytí $F$ a zároveň lokálně konečné. Navíc je to i zjemnění ©U.
        \end{dukazin}
    \end{tvrzeni}

    \begin{veta}[Charakterizace parakompaktnosti]
        Pro regulární TP ®X jsou následující podmínky ekvivalentní:
        
        \begin{itemize}
            \item[a)] ®X je parakompaktní.
            \item[b)] Každé otevřené pokrytí ®X má otevřené $\sigma$-lokálně konečné zjemnění.
            \item[c)] Každé otevřené pokrytí ®X má lokálně konečné zjemnění (libovolnými množinami).
            \item[d)] Každé otevřené pokrytí ®X má uzavřené lokálně konečné zjemnění. 
        \end{itemize}

        \begin{dukazin}
            $a) \implies b):$ každé lokálně konečné zjemnění je $\sigma$-lokálně konečné.

            $b) \implies c):$ Ať ®U je otevřené pokrytí ®X. Podle b) existuje otevřené zjemnění $©V = \bigcup_{n=1}^∞ ©V_n$, $©V_n$ lokálně konečný systém. $W_n:=\bigcup ©V_n$ je otevřené $\{W_n | n \in ®N\}$ je otevřené pokrytí ®X. Ať $A_n := W_n \setminus \bigcup_{i < n} W_i$. $\{A_n | n \in ®N\}$ je lokálně konečné pokrytí ®X (každé $x \in ®X$ je v nějakém $W_n$, takže už není ve větších $A_n$). $\{A_n \cap V | n \in ®N, V \in ©V_n\}$ je lokálně konečné zjemnění ©U.

% 5. 3. 2021

            $c) \implies d):$ Ať ©U je otevřené pokrytí ®X. Pro každé $x \in ®X$ existuje $U_x \in U: x \in U_x$. Nyní máme bod v otevřené množině, tedy z regularity existují otevřené množiny $V_x \subseteq ®X: x \in V_x \subseteq \overline{V_x} \subseteq U_x$. $©V:=\{V_x | x \in ®X\}$ je otevřené pokrytí ®X. ©V má lokálně konečné zjemnění $©W$ podle $c)$. $\{\overline{W} | W \in ©W\}$ je lokálně konečný systém podle lemmatu „Uzávěr lokálně konečného systému“. Navíc je i pokrytí a zjemňuje ©U.

            $d) \implies a)$ Ať ©U je otevřené pokrytí ®X. Z d) existuje lokálně konečné uzavřené zjemnění ©V. Pro $x \in ®X$ existuje $W_x$ otevřené okolí $x$ protínající jen konečně mnoho prvků z ©V. $©W:=\{W_x | x \in ®X\}$ je otevřené pokrytí ®X. Z d) existuje lokálně konečné uzavřené zjemnění ©A toho ©W. Pro $V \in ©V$ označíme $V^*:=®X \setminus \bigcup\{A \in ©A | A \cap V = \O\}$. Zřejmě $V^* \supseteq V$. Tedy $\{V^* | V \in ©V\}$ je otevřené (odčítáme uzavřenou množinu, neboť $A$ jsou uzavřené a množina je lokálně konečná, tedy podle lemmatu … je uzavřené i sjednocení) pokrytí.

                    Ať $x \in ®X$. $\exists U$ okolí $x$, které protíná jen konečně prvků $A_1, …, A_n \in ©A$. Zřejmě $U \subseteq A_1 \cup … \cup A_n$. Každé $A_i$ je podmnožinou nějakého $W_y$, tj. (podle volby $W_y$) $A_i$ protíná jen konečně mnoho prvků z ©V. Navíc je-li $V \in ©V$ a $A \in ©A$, že $A \cap V = \O$, pak $A \cap V^* = \O$. Tedy každé $A_i$ protíná pouze konečně mnoho prvků $V^*$, $V \in @V$. Pro každé $V \in @V$ fixujeme $U_v \in ©U: V \subseteq U_V$. Zřejmě $V \subseteq U_V \cap V^*$. Pak $\{U_V \cap V^* | V \in ©V\}$ je otevřené pokrytí ®X, které je lokálně konečné a které je zjemnění $©U$.
        \end{dukazin}
    \end{veta}

    \begin{dusledek}
        Každý Lindelöfův regulární prostor je parakompaktní.

        \begin{dukazin}
            Ať ©U je otevřené pokrytí ®X. Z lindolöfovosti existuje spočetné pokrytí $©V \subseteq ©U$. ©V je $\sigma$-lokálně konečné otevřené zjemnění ©U. Tedy platí b) z minulé věty.
        \end{dukazin}
    \end{dusledek}

    \begin{definice}[Skrčení]
        Ať $X$ je množina a $©S \subseteq ©P(X)$ (pokrytí $X$). Indexovaný systém $\{T_S: S \in ©S\} \subseteq ©P(X)$ se nazývá skrčení systému ©S, pokud (je to pokrytí) a $T_S \subseteq S, S \in ©S$.
    \end{definice}

    \begin{poznamka}[Nadmutí]
        Skrčení je speciální případ zjemnění.

        Podobně jako skrčení lze definovat pojem nadmutí.
    \end{poznamka}

    \begin{lemma}[O skrčení]
        Ať ®X je normální TP. Pak každé lokálně konečné (stačí bodově konečné) otevřené pokrytí ®X má uzavřené skrčení, jehož vnitřky tvoří pokrytí.

        \begin{dukazin}
            Ať $©U = \{U_\alpha: \alpha < \kappa\}$, $\kappa$ kardinál, ©U je lokálně kompaktní, otevřené pokrytí ®X. Nyní $F_0:=®X \setminus \bigcup \{U_\alpha: 0 < \alpha < \kappa\}$ uzavřená, $F_0 \subseteq U_0$ (z toho, že ©U je pokrytí). Z normality existuje otevřená $V_0 \subseteq ®X: F_0 \subseteq V_0\subseteq \overline{V_0} \subseteq U_0$.

            Nyní indukcí: Nechť máme zkonstruované $V_\beta: \forall \beta < \alpha < \kappa$. Označíme $F_\alpha := ®X \setminus \(\bigcup \{V_\beta: \beta < \alpha\}\cup \bigcup \{U_\gamma: \alpha < \gamma < \kappa\}\)$. Z normality zas $V_\alpha \subseteq ®X: F_\alpha \subseteq V_\alpha\subseteq \overline{V_\alpha} \subseteq U_\alpha$.

            $©V = \{\overline{V_\alpha}: \alpha < \kappa\}$ je skrčení ©U, $\Int\overline{V_\alpha} \supseteq V_\alpha$ a $\bigcup_{\alpha < \kappa} V_\alpha = ®X$, tedy $\bigcup_{\alpha < \kappa} \Int\overline{V_\alpha} = ®X$.
        \end{dukazin}
    \end{lemma}

    \begin{definice}[Kolektivně normální]
        TP ®X se nazývá kolektivně normální, pokud pro každý diskrétní systém ©F z uzavřených množin existuje disjunktní systém otevřených množin $\{U(F): F \in ©F\}$, že $F \subseteq U(F), F \in ©F$ (tj. otevřené nadmutí).
    \end{definice}

    \begin{poznamka}
        Každý kolektivně normální prostor je normální.
    \end{poznamka}

    \begin{tvrzeni}
        Každý parakompaktní prostor už je kolektivně normální, tedy i normální.
        
        \begin{dukazin}
            Ukážeme nejprve, že ®X je regulární. Ať $F \subseteq ®X$ uzavřená, $x \in ®X \setminus F$. Pro $y \in F$ existuje otevřené okolí $U_y$ bodu $y$, že $x \notin \overline{U_y}$. $©U := \{U_y: y \in F\}\cup \{®X \setminus F\}$ otevřené pokrytí ®X. Ať ©V je lokálně konečné otevřené zjemnění ©U. $G := \bigcup \{V \in ©V: V \cap F ≠ \O\}$. Z lemmatu $\overline{G} = \bigcup\{\overline{V}: V \in ©V, V \cup F ≠ \O\} \not\ni x$. $G \supset F$, $G$ otevřená. Tedy ®X je regulární.

            Ať ©F je diskrétní soubor z uzavřených množin. Pro $F \in ©F$ uvážíme $\bigcup\{H \in ©F: H≠F\}$ … uzavřená z lemmatu o uzávěru sjednocení lokálně kompaktního systému. Pro $x \in F$ existuje (z první části důkazu) $U_x$ otevřená, že $x \in U_x$, $\overline{U_x} \cap H = \O$ pro $H ≠ F, H \in ©F$. $\{U_x : x \in F \in ©F\} \cup \{®X \setminus \bigcup F\}$ je otevřené pokrytí ®X. Ať ©V je otevřené lokálně konečné zjemnění. Pro $F \in ©F: V(F):= \{V \in ©V: V \cup F ≠ \O\} \setminus \bigcup \{\overline{V}: V \in ©V, V \cap H ≠ \O \text{ pro nějaké } H \in ©F, H ≠ F\}$. Platí $F \subseteq V(F)$. Pro $F, F' \in ©F, F ≠ F' \implies V(F) \cap V(F') = \O$. $\{V(F): F \in ©F\}$ je disjunktní otevřené nadmutí ©F.
        \end{dukazin}
    \end{tvrzeni}

    \begin{definice}[Hvězda]
        Ať ®X je množina a $©S \subseteq ©P(®X)$, $x \in ®X$, $A \subseteq ®X$.

        Hvězda bodu $x$ vzhledem k ©S je $\st(x, ©S) = \bigcup \{S \in ©S: x \in S\}$.

        Hvězda množiny $A$ vzhledem k @S je $\st(A, ©S) = \bigcup_{x \in A}\st(x, ©S)$.
    \end{definice}

    \begin{definice}[Barycentrické a hvězdovité zjemnění]
        Ať ©U, ©V jsou pokrytí ®X. Řekneme, že ©U barycentricky zjemňuje ©V, pokud $\{\st(x, ©U): x \in ®X\}$ zjemňuje ©V.

        Řekneme, že ©U hvězdovitě zjemňuje ©V, pokud $\{\st(U, ©U): U \in ©U\}$ zjemňuje ©V.
    \end{definice}

    \begin{priklady}
            Ať $(®X, \rho)$ je MP. Ať ©U, ©V, ©W jsou pokrytí ®X tvořená po řadě všemi $\epsilon, 2\epsilon, 3\epsilon$ koulemi ($\epsilon > 0$ pevné). Pak ©U zjemňuje barycentricky ©V a hvězdovitě ©W.
    \end{priklady}

    \begin{lemma}[Dvojité barycentrické zjemnění je hvězdovité]
        Ať $X$ je množina, ©U pokrytí ©X, ©V barycentrické zjemnění ©U a ©W barycentrické zjemnění ©V. Potom ©W je hvězdovité zjemnění ©U.

% 12. 3. 2021

        \begin{dukazin}
            Mějme $W \in ©W$ libovolně. Chceme najít $U \in ©U: \st(W, ©W) \subseteq U$. $W = \O$ triviální. $W ≠ \O:$ Fixujeme $x_0 \in W$. Pro každé $x \in ®X$ existuje $V_x \in ©V$: $\st(x, ©W) \subseteq V_x$. Nyní 
            $$ \st(W, ©W) = \bigcup\{T \in ©W: T \cap W ≠ \O\} = \bigcup\{\{T \in ©W| x \in T\} | x \in W\} = \bigcup\{\st(x, ©W) | x \in W\} \subseteq \bigcup\{V_x| x \in W\} \subseteq \st(x_0, ©V), $$
            protože $W \subseteq V_x$ pro každé $x \in W$. ©V barycentricky zjemňuje ©U, tedy existuje $u \in ©U: \st(x_0, ©V) \subseteq U$.
        \end{dukazin}
    \end{lemma}

    \begin{veta}[Charakterizace parakompaktnosti pomocí hvězdovitých zjemnění]
        Pro TP ®X je ekvivalentní:

        \begin{itemize}
            \item[a)] ®X je parakompaktní.
            \item[b)] Každé otevřené pokrytí ®X má barycentrické zjemnění.
            \item[c)] Každé otevřené pokrytí ®X má hvězdovité zjemnění.
            \item[d)] Každé otevřené pokrytí ®X má otevřené $\sigma$-diskrétní zjemnění a ®X je regulární.
        \end{itemize}

        \begin{dukazin}
            $a) \implies b)$ Ať ©U je otevřené pokrytí ®X. Z a) vyplývá, že existuje jeho lokálně konečné otevřené zjemnění ©V. Víme, že ®X je parakompaktní, tedy normální. Z lemmatu o skrčení existuje uzavřené pokrytí $©W = \{W_V | V \in ©V\}$, $W_V \subseteq V$. ©V je lokálně konečné, tedy i ©W je lokálně konečné. Pro $x \in ®X$ definujeme $A_x = \bigcap\{V | x \in W_V\}$. Jde o konečný průnik (vzhledem k lokální kompaktnosti), tedy $A_x$ je otevřená. Položme $B_x = \bigcup\{W \in ©W | x \notin W\}$. Podle lemmatu o sjednocení lokálně konečného systému je $B_x$ uzavřená. Zřejmě $x \in A_x \setminus B_x =: C_x$ je otevřená. Tedy $©C = \{C_x|x \in ®X\}$ je otevřené pokrytí ®X.

                Ukážeme, že ©C barycentricky zjemňuje ©U: Ať $y \in ®X$. Chceme najít $V \in ©U$: $\st(y, ©C) \subseteq V$. Víme, že existuje $V \in ©V: y \in W_V$. Ať $x \in \st(y, ©C)$. Pak $y \in C_x = A_x \setminus B_x$, tedy $y \notin B_x$, tudíž $x \in W_V \subseteq V$ (kdyby ne, pak $W_V \subseteq B_x$, tedy $y \notin C_x$).

            $b) \implies c)$ k otevřenému pokrytí můžeme najít barycentrické zjemnění, ke kterému můžeme najít barycentrické zjemnění. Pak c) vyplývá z předchozího lemmatu.

            $c) \implies d)$ ®X je regulární: Ať $F \subseteq ®X$ uzavřená, $x \in ®X \setminus F$. Uvažujme otevřené pokrytí $\{®X \setminus F, ®X \setminus {x}\}$. Podle c) existuje otevřené hvězdovité zjemnění ©U. $\exists U \in ©U: x \in U$. Nutně $U \cap F = \O$. Pak $\overline{U} \subseteq \st(U, ©U) \subseteq ®X \subseteq ®X \setminus F$. Tedy ®X je regulární.

                Ať $©U_0$ je otevřené pokrytí ®X. Chceme najít $\sigma$-diskrétní zjemnění toho $©U_0$. Použijeme podmínku c) spočetně nekonečněkrát, abychom induktivně našli otevřená pokrytí $©U_1, ©U_2, …$, že $©U_{n+1}$ hvězdovitě zjemňuje $©U_n, n≥0$. Oindexujme prvky $©U_0: ©U_0 = \{U_i|i \in I\}$. Pro $i \in I$ a pro $n \in ®N$ uvažujme $U_{i, n} := \{x \in ®X|x \text{ má okolí } V: \st(V, ©U_n) \subseteq U_i\}$. Pro každé $n \in ®N: \{U_{i, n}| i \in I\}$ je otevřené zjemnění ©U, ale ne nutně pokrytí.

                Pomocné tvrzení: Pokud $x \in U_{i, n}, u \notin U_{i, n+1}$, pak neexistuje $U \in ©U_{n+1}$, že $x, y \in U$. Důkaz: Pro $U \in ©U_{n+1}$ existuje $W \in ©U_n: \st(U, ©U_{n+1})\subseteq W$. Tedy pokud $x \in U\cap U_{i, n}$, pak $W \subseteq \st(x, ©U_n) \subseteq U_i$. Pak $\st(U, ©U_{n+1})\subseteq U_i$ a $u \subseteq U_{i, n+1}$. Tedy $y \notin U$, protože $y \notin U_{i, n+1}$.

                Uvažme dobré uspořádání $<$ na $I$. Ať $V_{i_0, n} = U_{i_0, n} \setminus \bigcup \{U_{i, n+1} | i < i_0\}$, $i_0 \in I, n \in ®N$. Ukážeme, že $** = \{V_{i_0, n} | i_0 \in I, n \in ®N\}$ je hledané $\sigma$-diskrétní zjemnění $©U_0$. Pro $i_1 ≠ i_2, i_1, i_2 \in I$, pak $i_1 < i_2$ nebo naopak. Podle toho buď $V_{i_2, n} \subseteq ®X \setminus U_{i_1, n+1}$ nebo $V_{i_1, n} \subseteq ®X \setminus U_{i_2, n+1}$. Podle pomocného tvrzení platí, že pokud $x \in V_{i_1, n}$ a $y \in V_{i_2, n}$, pak neexistuje $U \in ©U_{n+1}$, že $x, y \in U$. To nám říká, že $\forall n \in ®N: \{V_{i, n} | i \in I\}$ je diskrétní. Zbývá už jen ukázat, že $**$ je pokrytí: Ať $y \in ®X$. Existuje $<$-nejmenší $i(y) \in I: y \in U_{i(y), n}$ pro nějaké $n \in ®N$. Nyní $y \notin U_{i, n+2}$ pro $i <i(y)$. Podle pomocného tvrzení použitého na $n+1$ platí $\st(y, ©U_{n+2}) \cap \bigcup \{U_{i, n+1}|i < i(y)\} = \O$. Tedy $y \in V_{i(y)}, n$.

                $d) \implies a)$ Víme, že ®X je regulární, tedy můžeme aplikovat charakterizaci parakompaktnosti z minulého týdne, jelikož $\sigma$-diskrétní $\implies \sigma$-lokálně konečný.
        \end{dukazin}
    \end{veta}

    \begin{veta}[Stone]
        Každý metrizovatelný prostor je parakompaktní.

        \begin{dukazin}
            Ukážeme, že každé otevřené pokrytí ©U má barycentrické zjemnění. Fixujeme na nějakém tom prostoru ®X kompatibilní metriku $\rho ≤ 1$. Navíc búno $®X \notin ©U$. Pro každé $x \in ®X$ a $U \in ©U$, že $x \in U$, existuje největší možné $\epsilon_{x, U}>0$, že $B(x, 5\epsilon_{x, U})$. Položíme $©V = \{B(x, \epsilon_{x, U})|x \in U \in ©U\}$. Ověříme, že $©V$ barycentricky zjemňuje $©U$: Ať $x \in ®X$. Chceme najít $U \in ©U: \st(x, ©V) \subseteq U$. Ať $\epsilon_x = \sup\{\epsilon_{x, U} | x \in U \in ©U\}$. $0 < \epsilon_x ≤ 1$. Existuje $U \in ©U: \epsilon_{x, U} ≥ \frac{\epsilon_x}{2}$.

            Ukážeme, že $\st(x, ©V) \subseteq U$. Ať tedy $x \in B(y, \epsilon_{y, v})$ pro nějaké $y \in V \in ©U$. Chceme $B(y, \epsilon_{y, v}) \subseteq U$. Máme $B(y, 5\epsilon_{y, v}) \subseteq V$ a zároveň $\rho(x, y) < \epsilon_{U, V}$. Z $\triangle$-nerovnosti: $B(x, 4\epsilon_{y, V}) \subseteq V$. Z maximality $\epsilon_{x, V} ≥ \frac{1}{5}4\epsilon_{y, V}$. Také $2\epsilon_{x, U} > \epsilon_x ≥ \epsilon_{x, V}$. Dohromady $2\epsilon_{x, U} > \frac{4}{5} \epsilon_{y, V}$, tj. $5\epsilon_{x, U} > 2\epsilon_{y, V}$. Pro $z \in B(y, \epsilon_{y, V}): \rho(x, z) < 2\epsilon_{y, V}$, a tedy $\rho(x, z) < 5\epsilon_{x, U}$. Proto $z \in U$. Tudíž $B(y, \epsilon_{y, v}) \subseteq U$.
        \end{dukazin}
    \end{veta}

% 19. 3. 2021

    \begin{definice}
        Pro funkci $f: X \rightarrow ®R$ značíme $\supp f = \overline{\{x \in X: f(x) ≠ 0\}}$.
    \end{definice}

    \begin{veta}[Rozklad jednotky]
        Ať ®X je parakompaktní prostor, ©U otevřené pokrytí ®X. Pak existuje rozklad jednotky podřízený tomuto pokrytí, tj. systém spojitých funkcí $f_i: X \rightarrow [0, 1]$, $i \in I$, že $\{\supp f_i: i \in I\}$ je lokálně konečné zjemnění ©U a $\sum_{i \in I} f_i(x) = 1, \forall x \in ®X$.

        \begin{dukazin}
            ®X parakompaktní, tedy normální. Tedy existuje otevřené pokrytí ©W takové, že $\{\overline{W}: W \in ©W\}$ zjemňuje ©U. Ať ©V je lokálně konečné otevřené zjemnění ©W. Víme, že existuje uzavřené skrčení $\{F_V: V \in ©V\}$, $F_V \subseteq V$. Z normality existují spojité funkce $g_V: ®X \rightarrow [0, 1]$, $g_V|_{F_V} = 1$, $g_V|_{®X \setminus V} = 0$. Položme $g(x):=\sum_{V \in ©V} g_V(x)$. Funkce $g$ je spojitá, protože spojitost je lokální pojem a $g$ je lokálně součet konečně mnoha nenulových spojitých funkcí. Navíc zřejmě $g ≥ 1$, protože $\{F_V: V\in©V\}$ je pokrytí ®X. Tedy položme $f_V:= \frac{g_V}{g}$.
        \end{dukazin}
    \end{veta}

    \begin{veta}[Michaelova selekční]
        Zdola polospojitá (vícehodnotová) funkce z parakompaktního prostoru do neprázdných uzavřených konvexních podmnožin Banachova prostoru má spojitou selekci.
    \end{veta}

    \begin{veta}[Dugunjiho]
        Ať ®X je metrizovatelný a $A \subseteq ®X$ uzavřená. Pak existuje lineární zobrazení $L: C(A, ®R) \rightarrow C(®X, ®R)$, že $L(f)$ rozšiřuje $f$ pro $f \in C(A, ®R)$.
    \end{veta}

\section{Metrizační věty}
    \begin{poznamka}[Opakování]
        Uryshonova metrizační věta: Regulární prostor se spočetnou bází je metrizovatelný.
    \end{poznamka}

    \begin{veta}[Bing, Nagata, Smirnov]
        Pro regulární prostor ®X jsou následující podmínky ekvivalentní:
        
        \begin{itemize}
            \item[a)] ®X je metrizovatelný.
            \item[b)] ®X má $\sigma$-diskrétní bázi.
            \item[c)] ®X má $\sigma$-lokálně konečnou bázi.
        \end{itemize}

        \begin{dukazin}
            $a) \implies b)$: Ať $©B_n$ je otevřené pokrytí ®X koulemi o poloměru $\frac{1}{n}$. ®X je parakompaktní podle Stoneovy věty. Z charakterizace parakompaktnosti máme, že $©B_n$ má $\sigma$-diskrétní otevřené zjemnění $©V_n$. $\bigcup_{n \in ®N} ©V_n$ je opět $\sigma$-diskrétní, navíc je to báze.

            $b) \implies c)$: triviální.

            $c) \implies a)$ Ať $B = \bigcup_{n=1}^∞$ je báze $®X$, $©B_n$ lokálně konečný soubor. Uvědomíme si, že ®X je parakompaktní: Je-li totiž $©U$ otevřené pokrytí ®X, pak $\{B \in ©B: \exists U \in ©U: B \subseteq U\}$ je zjemnění $U$ a vzhledem k tomu, že $B$ je báze, tak je to i pokrytí. Navíc je $\sigma$-lokálně konečné. Tedy z charakterizace parakompaktnosti to máme.

            Z parakompaktnosti dostáváme normalitu ®X. Pro $n, k \in ®N$ a $B \in ©B_n$ položme $V_{k, n, B}:= \bigcup\{C \in ©B_k: \overline{C} \subseteq B\}$. $©B_k$ je lokálně konečný, tedy (z lemmatu o uzávěru lokálně konečného systému) $\overline{V_{k, n , B}} \subseteq B$. Tedy existují (z normality) spojité funkce $f_{k, n, B}: ®X \rightarrow [0, 1]$, $f_{k, n, B}(x) = 0$ pro $x \in ®X \setminus B$ a $1$ pro $x \in \overline{V_{k, n, B}}$.

            Definujeme $M_{k, n} \subseteq [0, 1]^{©B_n}$ následovně $M_{k, n} = \{\phi: ©B_n \rightarrow [0, 1]: \{B \in ©B_n: \phi(B) ≠ 0\} \text{ je konečná}\}$. Na $M_{k, n}$ uvažme metriku $\rho_{k, n}{\phi, \psi} := \sum_{B \in ©B_n} |\phi(B) - \psi(B)|$. Ať $g_{k, n}: ®X \rightarrow M_{k, n}$, $g_{k, n} = \triangle_{B \in ©B_n} f_{k, n, B}$, $g_{k, n}(x) = (f_{k, n, B}(x))_{B \in ©B_n}$.

            Ověříme, že $g_{k, n}: ®X \rightarrow (M_{k, n}, \rho_{k, n})$ je spojité: Ať $x \in ®X$, $\epsilon > 0$, existuje $U$ okolí $x$ protínající jen konečně prvků $B_1, …, B_m \in ©B_n$. $f_{k, n, B_1}, …, f_{k, n, B_m}$ jsou spojitá, tedy existuje $V \subseteq U$ okolí $x$, že $|f_{k, n, B}(x) - f_{k, n, B_i}(y)| < \frac{\epsilon}{m}$ pro $i ≤ m, y \in V$. Nyní
            $$ \rho_{k, n}(g_{k, n}(x), g_{k, n}(y)) = \sum_{i=1}^m |g_{k, n}(x)(B_i) - g_{k, n}(y)(V_i)| = \sum |f_{k, n, B}(x) - f_{k, n, B_i}(y)| < m·\frac{\epsilon}{m} = \epsilon. $$

            Pokud systém $\{g_{k, n}: k, n \in ®N\}$ odděluje body a uzavřené množiny, pak $\delta:= \triangle_{k, n \in ®N}g_{k, n}:®X \rightarrow \prod_{k, n \in ®N} M_{k, n}$ je vnoření (podle lemmatu o Tichonovově vnoření). Tím jsme vnořili ®X do spočetného součinu metrizovatelných prostorů, tedy do metrizovatelného prostoru, tedy ®X je metrizovatelné.

            $\{g_{k, n}: k, n \in ®N\}$ odděluje body a uzavřené množiny: Ať $F \subseteq ®X$ je uzavřená, $x \in ®X\setminus F$. Existuje $n \in ®N$ a $B \in ©B_n: x \in B \subseteq X \setminus F$. Z regularity existuje $C \in ©B_k, k \in ®N$. $g_{k, n}(x)(B) = f_{k, n, B}(x) = 1$ a $g_{k, n}(y)(B) = f_{k, n, B}(y) = 0$ pro $y \in ®X \setminus B \supseteq F$.
        \end{dukazin}
    \end{veta}

    \begin{definice}
        Ať ®X je TP. Posloupnost otevřených pokrytí $©V_n$ prostoru ®X se nazývá development, pokud pro každé $x \in ®X: \{\st(x, ©V_n)| n \in ®N\}$ je báze okolí v bodě $x$.
    \end{definice}

    \begin{poznamka}
        Je-li $(X, \rho)$ MP, pak $©V_n:= \{B(x, \frac{1}{n}): x \in ®X\}, n \in ®N$ je development ®X.
    \end{poznamka}

    \begin{veta}[Bing]
        TP ®X je metrizovatelný $\Leftrightarrow$ je kolektivně normální a má development.

        \begin{dukazin}
            $\implies$: metrizovatelný $\implies$ má development (podle předchozí poznámky) a metrizovatelný $\implies$ parakompaktní $\implies$ kolektivně normální.

% 26. 3. 2021

            $\Leftarrow$: Dokážeme ve 4 částech:

            1. Pro diskrétní soubor $©F = \{F_\alpha\}_{\alpha \in A}$ uzavřených množin v ®X existuje diskrétní soubor otevřených množin $©W = \{W_\alpha\}_{\alpha \in A}$, že $F_\alpha \subseteq W_\alpha$: Dle kolektivní normality existují otevřené disjunktní $U_\alpha: \alpha \in A, F_\alpha \subseteq U_\alpha$. Položme $F = \bigcup ©F$, $Z = ®X \setminus \bigcup_{\alpha \in A} U_\alpha$. $F$ uzavřená (sjednocení lokálního systému uzavřených množin), $Z$ uzavřená. ®X je kolektivně normální, tedy speciálně normální, tedy existují otevřené disjunktní $V, W$, že $Z \subseteq V$ a $F \subseteq W$. Položme $W_\alpha := U_\alpha \cap W$. Systém $\{W_\alpha\}$ už je diskrétní (je-li $x \in Z$, pak $x \in V$ a $V \cap W_\alpha = \O$, je-li naopak $x$ v $U_\alpha$, pak $U_\alpha \cap W_\beta = \O$ pro $\beta ≠ \alpha$) a $F_\alpha \subseteq W_\alpha$.

            2. Ať $©V_n$ je development prostoru ®X. Buď $\kappa ≥ \omega$ a očíslujme $©V_n = \{V_{\alpha, n} | \alpha < \kappa\}$ (s případným opakováním prvků). Položme $D_{\alpha, n, k} = \{x \in V_{\alpha, n} | \st(x, V_k) \subseteq V_{\alpha, n}\}$ a $C_{\alpha, n, k} = D_{\alpha, n, k} \setminus \bigcup_{\beta < \alpha} V_{\beta, n}$. $D_{\alpha, n, k}$ (a tudíž i $C_{\alpha, n, k}$) je uzavřená:

                Volme $x \in \overline{D_{\alpha, n, k}}$. Pak pro libovolné $V \in ©V_k$, že $x \in V$ platí, že existuje $y \in V \cap D_{\alpha, n, k}$. Pak $V \subseteq \st(y, ©V_k) \subseteq V_{\alpha, n}$. Tedy $\st(x, ©V_k) = \bigcup\{V \in ©V_k | x \in V\} \subseteq V_{\alpha, n}$. Tedy $x \in D_{\alpha, n, k}$, tudíž $D_{\alpha, n, k}$ je uzavřená.

            3. Pro pevná $n, k \in ®N$ je $\{C_{\alpha, n, k} | \alpha < \kappa\}$ diskrétní: Buď $y \in ®X$ libovolné. Pak existuje nejmenší $\beta < \kappa: y \in V_{\beta, n}$. Najděme $V \in ©V_k: y \in V$. Pro $\alpha > \beta: V_{\beta, n}$ je disjunktní s $C_{\alpha, n, k}$ a pro $\alpha < \beta: V$ je disjunktní s $C_{\alpha, n, k}$ (kdyby existovalo $z \in V \cap C_{\alpha, n, k}$ pak $\st(z, ©V_k) \subseteq V_{\alpha, n}$, speciálně $y \in V_{\alpha, n}$, což je spor s minimalitou $\beta$). Tedy $V \cap V_{\beta, n}$ je okolí bodu $y$, které protíná nejvýše jeden prvek systému $\{c_{\alpha, n, k} | \alpha \in A\}$ (a sice prvek $C_{\beta, n, k}$).

            4. $\{C_{\alpha, n, k}\}$ je diskrétní soubor uzavřených množin (podle 2, 3). Podle 1 existuje diskrétní soubor otevřených nadmnožin $\{V_{\alpha, n, k} | \alpha < \kappa\}$. Tedy $©V_{n, k} := \{V_{\alpha, n, k} \cap V_{\alpha, n} | \alpha < \kappa\}$ je diskrétní (zmenšili jsme jeho množiny). Ukážeme, že $©V := \bigcup_{n, k \in ®N} ©V_{n, k}$ je báze ®X:

            Ať $U \subseteq ®X$ je otevřená, $x \in U$. $\exists n \in ®N: \st(x, ©V_n) \subseteq U$. Najdeme $\alpha$ nejmenší možné, že $x \in V_{\alpha, n}$. Zřejmě $V_{\alpha, n} \subseteq U$. Opět z vlastností developmentu existuje $k \in ®N: \st(x, ©V_k) \subseteq V_{\alpha, n}$. Nyní $x \in C_{\alpha, n, k}$, tedy $x \in V_{\alpha, n, k} \cap V_{\alpha, n} \subseteq U$. Tudíž ©V je báze ®X.

            ©V je $\sigma$-diskrétní báze ®X, tedy podle metrizační věty Bing-Nagata-Smirnov je ®X metrizovatelný.
        \end{dukazin}
    \end{veta}

\section{Uniformní prostory}
    \begin{poznamka}
        Zavedeno např. díky tomu, že stejnoměrnou spojitost nelze charakterizovat pomocí topologie.

        Matematici Weil(1936), Tukey(1940) … prvotní zkoumání UP.
    \end{poznamka}

    \begin{definice}[Značení]
        Pro množinu ®X značíme $\triangle(X) = \{(x, x) | x \in X\}$.

        Pro $E \subseteq X \times X$ značíme $E^{-1} = \{(y, x) | (x, y) \in E\}$.

        Pro $C, D \in X \times X$ značíme $C \circ D = \{(x, z) \in X \times X | \exists y \in X: (x, y) \in C \land (y, z) \in D\}$.

        $E[x] = \{y \in X | (x, y) \in E\}$.
    \end{definice}

    \begin{definice}[Uniformní prostor (UP)]
        Dvojice $(®X, ©D)$ se nazývá uniformní prostor (UP), pokud $®X$ je množina a $©D \subseteq ©P(X \times X), ©D ≠ 0$ splňující

        \begin{enumerate}
            \item $\forall D \in ©D: \triangle(®X) \subseteq D$,
            \item $\forall C, D \in ©D: C \cap D \in ©D$,
            \item $\forall D \in ©D\ \exists C \in ©D: C \circ C \subseteq D$,
            \item $\forall D \in ©D: D^{-1} \in ©D$,
            \item $\forall D \in ©D\ \forall E \subseteq X \times X: D \subseteq E \implies E \in ©D$,
            \item $\forall x, y \in X: x≠y \implies \exists D \in ©D: (x, y) \notin D$. $(\Leftrightarrow \bigcap ©D = \triangle(®X).)$
        \end{enumerate}

        Prvky systému ©D nazýváme okolí diagonály.
    \end{definice}

    \begin{definice}[Báze uniformity]
        Systém $©B \subseteq ©P(®X^2)$ se nazývá báze uniformity (resp. báze uniformity ©D), pokud uzavřením ©B na nadmnožiny dostaneme ©D.
    \end{definice}

    \begin{definice}[Subbáze uniformity]
        Systém $©S \subseteq ©P(®X^2)$ tvoří subbázi uniformity (resp. uniformity ©D), pokud uzavřením na konečné průniky dostaneme bázi uniformity (resp. bázi uniformity ©D).
    \end{definice}

    \begin{definice}[Uniformní zobrazení]
        Jsou-li $(®X, ©D)$ a $(®Y, ©E)$ UP, $f: ®X \rightarrow ®Y$ se nazývá uniformní (stejnoměrně spojité), pokud $\forall E \in ©E: (f \times f)^{-1}(E) \in ©D$. ($\Leftrightarrow \forall E \in ©E\ \exists D \in ©D: (f \times f)(D) \subseteq E$.) ($\Leftrightarrow \forall E \in ©E\ \exists D \in ©D\ \forall x, y \in ®X: (x, y) \in D \implies (f(x), f(y)) \in E$.)
    \end{definice}

    \begin{definice}[Uniformní izomorfismus]
        Zobrazení $f$ se nazývá uniformní izomorfismus, pokud $f$ je bijekce a $f$ i $f^{-1}$ jsou uniformní.
    \end{definice}

    \begin{lemma}
        Systém $©B \subseteq ©P(®X^2)$ tvoří bázi nějaké uniformity na ®X, pokud
        $$ a) \bigcap ©B = \triangle(®X), $$ 
        $$ b) \forall C, D \in ©B\ \exists E \in ©B: E \subseteq C \cap D, $$
        $$ c) \forall D \in ©B\ \exists C \in ©B: C \circ C \subseteq D, $$ 
        $$ d) \forall D \in ©B\ \exists E \in ©B: E \subseteq D^{-1}. $$

        \begin{dukazin}
            $©D := \{C \subseteq ®X \times ®X | \exists B \in ©B: B \subseteq C\}$. Následně ověříme podmínky.
        \end{dukazin}
    \end{lemma}

% 9. 4. 2021

    \begin{tvrzeni}[Vytvoření UP z MP a TP z UP]
        Je-li $(®X, \rho)$ metrický prostor a $D_\epsilon = \{(x, y)| \rho(x, y) < \epsilon\}$, potom $\{D_\epsilon | \epsilon > 0\}$ je báze nějaké uniformity na ®X -- značíme ji $©D_\rho$. Tato uniformita se nazývá generovaná metrikou $\rho$.

        Je-li $(®X, ©D)$ UP, pak systém $\tau_{©D} = \{A \subseteq X | \forall x \in A\ \exists D \in ©D: D[x] \subseteq A\}$ je topologie na ®X a pro každé $x \in ®X$ tvoří systém $©B(x) := \{D[x] | D \in ©D\}$ bázi okolí v bodě $x$. Topologie $\tau_{©D}$ se nazývá generovaná uniformitou ©D.

        Pokud místo systému ©D použijeme v definici topologie $\tau_{©D}$ nějakou bázi ©D, pak dostaneme stejnou topologii. Zároveň také $\tau_{©D_\rho}$ je systém všech otevřených množin v $(®X, \rho)$.

        \begin{dukazin}
            Ověříme definice.
        \end{dukazin}
    \end{tvrzeni}

    \begin{definice}
        UP $(®X, ©D)$ se nazývá metrizovatelný, pokud existuje metrika $\rho$ na ®X, že $©D = ©D_\rho$. TP $(®X, \tau)$ se nazývá metrizovatelný, pokud existuje uniformita ©D na ®X, že $\tau = \tau_{©D}$
    \end{definice}

    \begin{definice}
        Ať $©U \subseteq ©P(®X)$, $A \subseteq ®X$, pak značíme $\st(A, ©U) = \bigcup\{U \in ©U | U \cap A ≠ \O\}$. $©U^* = \{\st(U, ©U) | U \in ©U\}$.

        Pro ©U, ©V pokrytí ®X, definujeme jejich společné zjemnění $©U \wedge ©V = \{U \cap V | U \in ©U, V \in ©V\}$.
    \end{definice}

    \begin{poznamka}
        Pro ©U, ©V pokrytí množiny ®X: ©U hvězdovitě zjemňuje ©V, pokud $©U^*$ zjemňuje ©V.
    \end{poznamka}

    \begin{definice}
        Ať ®X je množina, $¦U \subseteq ©P(©P(®X))$ se nazývá pokrývací uniformita na ®X, pokud:

        \begin{itemize}
            \item $\forall ©U \in ¦U: ©U$ je pokrytí ®X,
            \item Je-li $©U \in ¦U, ©V$ pokrytí ®X a ©U zjemňuje ©V, pak $©V \in ¦U$,
            \item $\forall ©U \in ¦U\ \exists ©V \in ¦U: ©V$ hvězdovitě zjemňuje ©U,
            \item $\forall ©U, ©V \in ¦U: ©U \wedge ©V \in ¦U$,
            \item $\forall x, y \in ®X, x ≠ y\ \exists ©U \in ¦U\ \forall U \in ©U: |\{x, y\} \cap U| ≤ 1$.
        \end{itemize}

        Prvky systému ¦U se nazývají uniformní pokrytí.
        
        Je-li ¦U pokrývací uniformita na ®X, pak položme
        $$ ©D_{¦U} := \{D \subseteq ®X \times ®X | \exists ©U \in ¦U\ \forall u \in ©U: U \times U \subseteq D\}. $$

        Je-li ©D (diagonální) uniformita na ®X, pak položme
        $$ ©U_{©D} := \{©U \in ©P(©P(®X)) | \exists D \in ©D: \{D[x] | x \in ®X\} \text{ zjemňuje} ©U\}. $$

        Přiřazení $¦U \mapsto ©D_{¦U}$ a $©D \mapsto ¦U_{©D}$ jsou navzájem inverzní bijekce systému všech pokrývacích uniformit na ®X a systém všech uniformit.
    \end{definice}

    \begin{lemma}[O pseudometrice]
        Ať $(®X, ©D)$ je UP a $D_i \in ©D$, $D_i = D_i^{-1}$, $i \in ®N_0$, $D_0 = ®X \times X$, $D_{i+1} \circ D_{i+1} \circ D_{i+1} \subseteq D_i$. Pak existuje pseudometrika $d$ na ®X, že pro každé $i ≥ 1$: $\{(x, y) : d(x, y) < \frac{1}{2^i}\} \subseteq D_i \subseteq \{(x, y) | d(x, y) ≤ \frac{1}{2^i}\}$.

% 16. 4. 2021

        \begin{dukazin}
            Položme $d(x, y) := \inf\{\frac{1}{2^{i_1}} + \frac{1}{2^{i_2}} + … + \frac{1}{2^{i_k}} | x_0, …, x_k \in ®X \land (x_{j-1}, x_j) \in D_{i_j} \land x = x_0, y=x_k\}$. $d(x, y)$ je pseudometrika na ®X. $D_i \subseteq \{(x, y) | d(x, y) ≤ \frac{1}{2^i}\}$ vidíme z toho, že pro $(x, y) \in D_i$ zvolíme $k=1$. Zbývá dokázat $\{(x, y) | d(x, y) < \frac{1}{2^i}\} \subseteq D_i$. Tedy chceme, že $d(x, y) < \frac{1}{2^i}$, pak $(x, y) \in D_i$, tj. že pro každou posloupnost $x_0, …, x_k$, kde $(x_{j_-1}, x_j) \in D_{i_j}$: pokud $\frac{1}{2^{i_1}} + … + \frac{1}{2^{i_k}} < \frac{1}{2^i}$, pak $(x_0, x_k) \in D_i$. To dokážeme indukcí podle $k$:

            Pro $k=1: \frac{1}{2^{i_1}} < \frac{1}{2^i}$, tj. $i < i_1$, tedy $(x_0, x_k) \in D_{i_1} \subseteq D_i$. Nyní předpokládejme, že $m > 1$ a pro všechna $k < m$ uvedené tvrzení platí. Uvažme posloupnost $x_0, …, x_m$, že $(x_{j-1}, x_j) \in D_{i_j}, j = 1, …, m$, a $\frac{1}{2^{i_1}} + … + \frac{1}{2^{i_m}} < \frac{1}{2^i}$. Zřejmě buď $\frac{1}{2^{i_1}} < \frac{1}{2^{i+1}}$, nebo $\frac{1}{2^{i_m}} < \frac{1}{2^{i+1}}$. Ze symetrie obou případů můžeme BÚNO předpokládat platnost první nerovnosti.

            Ať $n ≤ m-1$ je největší takové, že $\frac{1}{2^{i_1}} + … + \frac{1}{2^{i_n}} < \frac{1}{2^{i+1}}$. Pokud $n < m-1$, pak $\frac{1}{2^{i+1}} + … + \frac{1}{2^{i_{n+1}}} ≥ \frac{1}{2^{i+1}}$, tedy $\frac{1}{2^{i_{n+1}}} + … + \frac{1}{2^{i_m}} < \frac{1}{2^{i+1}}$. Podle indukčního předpokladu $x_0, n \in D_{i+1}$, $(x_{n+1}, x_m) \in D_{i+1}$. Navíc $\frac{1}{2^{i_{n+1}}} < \frac{1}{2^i}$, tedy $i < i_{n+1}$, $i + 1 ≤ i_{n+1}$, $D_{i_{n+1}} \subseteq D_{i+1}$. $(x_0, x_m) = (x_0, x_n)\circ(x_n, x_{n+1})\circ(x_{n+1}, x_m) \in D_{i+1}\circ D_{i+1}\circ D_{i+1} \subseteq D_i$.

            Pokud $n = m-1$, pak podle IP $x_0, x_{m-1} \in D_{i+1}$, a jelikož $\frac{1}{2^{i_m}} < \frac{1}{2^i}$, tak $(x_{m-1}, x_m)\in D_{i_m} \subseteq D_{i+1}$. Tedy $(x_0, x_m) \in D_{i+1}\circ D_{i+1} \subseteq D_i$.
        \end{dukazin}
    \end{lemma}

    \begin{veta}[Metrizovatelnost UP]
        UP $(®X, ©D)$ je metrizovatelný, právě když má spočetnou bázi uniformity.

        \begin{dukazin}
            $(\implies)$: Ať $d$ je metrika na ®X generující ©D. Pak $\{\{(x, y): d(x, y) < \frac{1}{n}\} | n \in ®N\}$ je báze ©D.

            $(\Leftarrow)$: Ať $\{C_n | n \in ®N\}$ je báze ©D. Indukcí najdeme posloupnost $D_n \in ©D$, že jsou splněny předpoklady předchozího lemmatu. A že $D_i \subseteq C_i$: Předpokládejme, že $D_0, …, D_n$ máme ($D_0 = ®X \times ®X$), pak víme, že $\exists E: E \circ E \subseteq D_n$, $\exists F: F \circ F \subseteq E$. Tedy $F \circ F \circ F \subseteq D_n$. Ať $D_{n+1} := (F \cap C_{i+1}) \cap (F \cap C_{i+1})^{-1}$. Tedy $D_{n+1}\circ D_{n+1} \circ D_{n+1} \subseteq D_n$, $D_{n+1} \subseteq C_{i+1}$. Tedy podle lemmatu o pseudometrice existuje pseudometrika $d$ na ®X, že
            $$ \{(x, y) | d(x, y) < \frac{1}{2^i}\} \subseteq D_i \subseteq \{(x, y) | d(x, y) ≤ \frac{1}{2^i}\}. $$
            Pro $x, y \in ®X, x≠y$ $\exists C \in ©D: (x, y) \notin C$. $\exists i \in ®N: C_i \subseteq C$. $D_i \subseteq C_i$. $(x, y)\notin D_i$. Tedy $d(x, y) ≥ \frac{1}{2^i} > 0$. Tedy $d$ je metrika. $d$ generuje uniformitu ©D: Tj. pro $\epsilon > 0$ $\{(x, y | d(x, y) < \epsilon)\} \in ©D$. To platí díky vlastnosti z předchozího lemmatu. A pro $D \in ©D\ \exists \epsilon > 0: \{(x, y) | d(x, y) < \epsilon\} \subseteq D$. $D \in ©D$ dané $\exists i \in ®N: C_i \subseteq D$, $D_i \subseteq C_i \subseteq D$. $\epsilon := \frac{1}{2^i}$. To máme také díky vlastnosti z předchozího lemmatu.
        \end{dukazin}
    \end{veta}

    \begin{veta}[Jemná uniformita]
        Ať $(®X, \tau)$ je TP. Všechny otevřené podmnožiny $®X \times ®X$ obsahující $\delta(®X)$ tvoří bázi nějaké uniformity na ®X právě tehdy, když ®X je parakompaktní.

        \begin{poznamkain}[Reformulace]
            Ať $(®X, \tau)$ je TP. Všechna otevřená pokrytí ®X tvoří bázi nějaké pokrývací uniformity na ®X, právě když ®X je parakompaktní.
        \end{poznamkain}

        \begin{dukazin}
            Pokud všechna otevřená pokrytí ®X tvoří bázi pokrývací uniformity, pak speciálně každé otevřené pokrytí má hvězdovité otevřené zjemnění. Tedy podle charakterizační věty je ®X parakompaktní.

            Je-li ®X parakompaktní, tak podle charakterizační věty má každé otevřené pokrytí ®X otevřené hvězdovité zjemnění a snadno se ověří, že $\{©U | ©U \text{ je pokrytí } ®X \text{, které je zjemňované nějakým otevřeným pokrytím } ®X\}$ je pokrývací uniformita na ®X.
        \end{dukazin}
    \end{veta}

    \begin{veta}[Uniformizovatelnost TP]
        TP je uniformizovatelný (tj. generován nějakou uniformitou), právě když je Tichonovův.

        \begin{dukazin}
            Ať $(®X, \tau)$ je generovaný uniformitou ©D. Buď $F \subseteq ®X$ uzavřená, $x \in ®X \setminus F$. Pak existuje $D \in ©D: D[x] \subseteq ®X \setminus F$. Ať $d$ je pseudometrika z předchozího lemmatu, kde volíme $D_1 = D$. Pak $\B_d(x, \frac{1}{2}) \subseteq D[x]$. Definujeme $f: ®X \rightarrow ®R$, $f(y) = d(x, y)$, $0 ≤ f ≤ 1$, $f(x) = 0$ a $f$ je spojitá. Pro $y \in F: f(y) ≥ \frac{1}{2}$. Tedy $®X$ je $T_\pi$.

            Ať $(®X, \tau)$ je Tichonovův. Pak uvažme $\beta®X$. $\beta®X$ je (para)kompaktní. Tedy na $\beta®X$ máme jemnou uniformitu, která generuje topologii na $\beta®X$. Tuto jemnou uniformitu na $\beta®X$ můžeme zúžit na ®X a ta již generuje topologii $\tau$.
        \end{dukazin}
    \end{veta}

% 23. 4. 2021
    
    \subsection{Operace s uniformními prostory}
        \begin{definice}
            Ať $(®X, ©D)$ je UP, $®Y \subseteq ®X$. Pak uniformní podprostor $(®Y, ©D_{®Y})$ je definován následovně $©D_{®Y} := \{D \cap (®Y \times ®Y) | D \in ©D\}$.

            Jsou-li $(®X_i, ©D_i)$ UP, pak suma těchto UP je definována jako $\(\bigcup ®X_i, \{\bigcup D_i | D_i \in ©D_i\}\)$. Součin pak jako $\(\prod ®X_i | \{\prod D_i | D_i \in ©D_i, \Fin(D_i ≠ ®X_i \times ®X_i)\}\)$. (Tedy jsou různé od identity jen v konečně mnoha případech.)
        \end{definice}

    \subsection{Úplnost a totální omezenost}
        \begin{definice}[Net]
            Net $(x_i)_{i \in I}$ UP $(®X, ©D)$ se nazývá cauchyovský, pokud $\forall D \in ©D\ \exists i_0 \in I\ \forall i, j ≥ i_0: (x_i, x_j \in D)$.
        \end{definice}

        \begin{definice}[Úplný a totálněomezený prostor]
            UP $(®X, ©D)$ se nazývá úplný, pokud každý cauchyovský net v $(®X, ©D)$ je konvergentní v $(®X, \tau_{©D})$.

            UP $(®X, ©D)$ se nazývá totálně omezený, pokud $\forall E \in ©D\ \exists K \subseteq ®X$ konečná: $E[K] = ®X$.
        \end{definice}

        \begin{poznamka}
            MP je totálně omezený (úplný) $\Leftrightarrow$ UP jím generovaný je totálně omezený (úplný).
        \end{poznamka}

        \begin{poznamka}
            UP $(®X, ©D)$ je totálně omezený $\Leftrightarrow$ $\forall D \in ©D\ \exists$ konečné pokrytí $U_1, …, U_n$ prostoru ®X, že $(U_1 \times U_1) \cup … \cup (U_n \times U_n) \subseteq D$.
        \end{poznamka}

        \begin{veta}
            Buď $(®X, ©D)$ UP. Pak $(®X, \tau_{©D})$ je kompaktní $\Leftrightarrow$ $(®X, ©D)$ je úplný a totálně omezený.

            \begin{dukazin}
                $\implies$ Je-li $D \in ©D$, pak $\{\Int D[x] | x \in ®X\}$ je otevřené pokrytí ®X. Tedy z kompaktnosti existuje konečné podpokrytí $\{\Int D[x_1], …, \Int D[x_n]\}$. Tedy pro $K := \{x_1, …, x_n\}$ je $D[K] = \bigcup D[x_i] \supseteq \bigcup \int D[x_i] = ®X$.

                Ať $(x_i)_{i \in I}$ je cauchyovský net v $(®X, ©D)$. Z charakterizace kompaktnosti víme, že $(x_i)_{i \in I}$ má hromadný bod -- řekněme $x$. Ukážeme, že $x$ je limitou netu $x_i$ (tj. $x_i$ konverguje k $x$). Ať $U$ je okolí $x$. Pak existuje symetrické $D \in ©D: (D \circ D)[x] \subseteq U$. Z cauchyovskosti existuje $i_0$, že pro $i, j ≥ i_0: (x_i, x_j) \in D$

            \end{dukazin}
        \end{veta}

        TODO

% 16. 4. 2021 Druhá část

\section{Topologické grupy}
    \begin{definice}[Topologická grupa]
        $(®G, ·, \tau)$ se nazývá topologická grupa (TG), pokud $(®G, ·)$ je grupa, $(®G, \tau)$ je TP a $·: ®G \times ®G \rightarrow ®G$ je spojité, $^{-1}: ®G \rightarrow ®G$ je spojité.
    \end{definice}

    \begin{pozorovani}
        Ať ®G je TG. Pak:

        \begin{itemize}
            \item $^{-1}$ je homeomorfismus.
            \item $\forall g \in ®G: L_g: ®G \rightarrow ®G$, $L_g(h) := g·h$ (tzv. levá translace) je homeomorfismus. ($L^{-1}_g = L_{g^{-1}}$.)
            \item $\forall x, y \in ®G\ \exists h: ®G \rightarrow ®G$ homeomorfismus: $h(x) = y$. ($L_{yx^{-1}}(x) = y$.)
            \item $\forall U$ okolí $e$ $\exists V$ okolí $e$: $V·V^{-1} := \{u·v^{-1} | u \in V, v \in V\} \subseteq U$.
            \item Pro $U$ otevřenou v ®G a $M \subseteq ®G$ libovolnou $M·U$ je otevřené.
            \item Uzávěr podgrupy $H ≤ ®G$ je opět grupa.
            \item Uzávěr normální (algebraicky) podgrupy je normální (algebraicky) podgrupa.
            \item Je-li $H$ podgrupa ®G s neprázdným vnitřkem, pak je obojetná.
            \item Součin TG se součinovou topologií a operací po složkách je TG.
        \end{itemize}
    \end{pozorovani}

    \begin{tvrzeni}[Uniformita na TG]
        Ať ®G je TG. Pak systém $\{D_U | U \text{ je okolí } e\}$, kde $D_U = \{(x, y) | x·y^{-1} \in U\}$, je bází nějaké uniformity na ®G. (Tzv. pravá uniformita na ®G). Tato uniformita je kompatibilní s topologií ®G.

        \begin{dukazin}
            $\delta(G) \subseteq D_u$. Ať $U$ je okolí $e$. Chceme najít $¦V$ okolí $e: D_{V} \circ D_{V} \subseteq D_{U}$. Uvažme spojité zobrazení $(x, y) \mapsto x·y$, $x, y \in ®G$. $(e, e) \rightarrow e$.

            Tedy existuje $V$ okolí $e$: $¦V·¦V \subseteq U$. Nyní $D_V \circ D_V \subseteq D_u$: Ať $(a, b) \in D_V \circ D_V$. Pak existuje $c \in G: (a, c) \in D_V$, $(c, b \in D_V)$. $a·c^{-1} \in V$ a $cb^{-1} \in V$, tedy $a·c^{-1}·c·b^{-1} \in V·V \subseteq U$. $a·b^{-1} \in U$. $(a, b) \in D_U$. Pro $V, W$ okolí $e$: $D_V \cap D_W \supseteq D_{V \cap W}$. Tedy $\{D_U | U \text{ okolí } e\}$ tvoří bázi uniformity.

            Tato uniformita je kompatibilní s původní topologií na ®G: Ať $V$ okolí $e$, $x \in ®G$. $D_V[x] = \{y \in ®G | xy^{-1} \in V\} = \{y \in ®G | y \in V^{-1}·x\}i = V^{-1}·x$ je okolí $x$.
        \end{dukazin}
    \end{tvrzeni}

    \begin{veta}
        Každá TG je Tichonovova.

        \begin{dukazin}
            Ať $(®G, ·, \tau)$ je TG. Ať ©D je pravá uniformita na $(®G, ·, \tau)$. Víme, že $(®G, \tau_{©D})$ je Tichonovův. Předchozí tvrzení dává, že $\tau_{©D} = \tau$.
        \end{dukazin}
    \end{veta}

    \begin{veta}[Metrizovatelnost TG]
        TG je metrizovatelná, právě když má spočetný charakter.
        
        \begin{dukazin}
                $\implies$: Triviální (každý met. prostor má spočetný charakter). $\Leftarrow$: Ať $(®G, ·, \tau)$ má spočetný charakter. Ať $\{U_n : n \in ®N\}$ je báze okolí v $e$. $D_n := \{(x, y) \in ®G \times ®G | xy^{-1} \in U_n\}$, $\{D_n | n \in ®N\}$ je báze pravé uniformity ©D. Tedy ©D má spočetnou bázi. Tedy $(®G, ©D)$ je metrizovatelný, tedy $(©G, \tau)$ je metrizovatelný.
        \end{dukazin}
    \end{veta}

    \begin{poznamka}[Informativně]
        Věta (Birkhoff Kahutani): Každá metrizovatelná grupa má zleva (zprava) invariantní metriku, tj. metrika $\rho$, že $\rho(x, y) = \rho(c·x, c·y)$, $\forall c, x, y \in ®G$.
    \end{poznamka}

    \begin{tvrzeni}[Spojitost homomorfismu]
        Ať $®G$, $®H$ jsou TG. $f: ®G \rightarrow ®H$ je homomorfismus grup. $f$ je spojité $\Leftrightarrow$ $f$ je spojité v $e \in ®G$.

        \begin{dukazin}
            $\implies$: zřejmě. $\Leftarrow$: Ať $x \in ®G$, $(x_i)_{i \in I}$ je net v ®G, který konverguje k $x$. Chceme, že $(f(x_i))_{i \in I}$ konverguje k $f(x)$ v ®H. $(x_i · x^{-1})_{i \in I}$ je net v ®G, konverguje k $x·x^{-1} = e \in ®G$. $f$ spojité v $e$, tedy $f(x_i·x^{-1})$ konverguje k $f(e) = e \in ®H$. Tudíž $f(x_i)·\(f(x)\)^{-1} \rightarrow e \in ®H$. Tudíž (po vynásobení $f(x)$) $f(x_i) \rightarrow f(x)$. Tedy $f$ spojité v $x$. $x \in ®G$ libovolné. Tedy $f$ je spojité.
        \end{dukazin}
    \end{tvrzeni}

    \begin{veta}[Faktor TG]
        Buď $N$ uzavřená normální podgrupa TG ®G. Pak faktogrupa $®G / N$ s kvocientovou topologií je TG a přirozená projekce (= kvocientové zobrazení) $\pi: ®G \rightarrow ®G / N$, $\pi(g) := g·N$, je spojitý a otevřený homeomorfismus.

        \begin{dukazin}
            Víme: $\pi$ je spojitý homeomorfismus. $\pi$ je otevřené: je-li $U \subseteq ®G$ otevřené, pak $\pi(U) = \{xN: x \in U\} \subseteq G/N$, $\pi^{-1}(\pi(U)) = U·N$ je otevřením. Tedy z definice kvocientové topologie je $\pi(U)$ otevřená množina.

            $®G/N$ je TG: Násobení je spojité: $f: ®G \times G \rightarrow ®G / N$, $f(g, h) := g·h·N$ je spojité, jelikož je to složení součinu a kvocientového zobrazení. Poznamenejme, že můžeme přirozeně identifikovat $®G \times ®G / N \times N$ a $(®G/N) \times (®G/N)$. Operace součinu na $G/N$ je kvocientem spojitého zobrazení $f$. Tedy je také spojité podle charakterizace spojitosti a projektivně vytvořeného prostoru.

            Spojitost $^{-1}$ se ukáže obdobně.

            Zbývá ověřit, že $®G / N$ je Hausdorffův. Stačí ověřit, že je $T_1$. $N$ je uzavřená, tedy $G\setminus N$ je otevřená. $\pi^{-1}(G / N \setminus \{N\}) = ®G \setminus N$ je otevřená. Tedy z definice kvocientové topologie $G/N \setminus \{N\}$ je otevřená v $G/N$, …
        \end{dukazin}
    \end{veta}

% 30. 4. 2021

    \begin{tvrzeni}[O homomorfismu]
        Ať $®G, ®H$ jsou TG a $f: G \rightarrow H$ spojitý homomorfismus. Pak $N := f^{-1}(e)$ je normální (uzavřená) podgrupa $G$ a existuje spojitý homomorfismus $\overline{f}: G / N \rightarrow H$, že $\overline{f}\pi = f$ (kde $\pi: G \rightarrow G/N$ je přirozená projekce).

        \begin{poznamkain}
            $\overline{f}$ nemusí být vnoření topologických prostorů.
        \end{poznamkain}

        \begin{dukazin}
            Bez důkazu.
        \end{dukazin}
    \end{tvrzeni}

    \begin{tvrzeni}[O izomorfismu]
        Ať ®G je TG, $K \subseteq H$ její uzavřené normální podgrupy. Pak $H/K$ je uzavřenou normální podgrupou $G/K$ a $(G/K)/(H/K)$ je izomorfně homeomorfní s $G/H$.

        \begin{dukazin}
            Bez důkazu.
        \end{dukazin}
    \end{tvrzeni}

\section{Souvislé prostory}
    \begin{definice}
        TP se nazývá souvislý, pokud je neprázdný a nelze ho vyjádřit jako sjednocení dvou disjunktních otevřených neprázdných množin. (Tj. obsahuje právě dvě obojetné množiny, sám sebe a prázdný prostor.)
    \end{definice}

    \begin{tvrzeni}
        Pro neprázdný TP ®X je ekvivalentní: a) ®X je souvislý, b) Je-li $®X = A \cup B$ a $\overline{A}\cap B = \O = \overline{B} \cap A$, pak $A = \O$ nebo $B = \O$. c) ®X neobsahuje vlastní obojetnou podmnožinu. d) Každé spojité zobrazení $f: ®X \rightarrow \{0, 1\}$ je konstantní.

        \begin{dukazin}
            Přímočaré.
        \end{dukazin}
    \end{tvrzeni}

    \begin{tvrzeni}
        Spojitý obraz souvislého prostoru je souvislý.

        \begin{dukazin}
            Ať $f: ®X \rightarrow ®Y$ je spojité zobrazení, ®X souvislé, $f$ na. Sporem. ®Y není souvislý. Potom z minulého tvrzení $\exists g: ®Y \rightarrow \{0, 1\}$ spojitý nekonstantní. Potom ale $g\circ f: ®X \rightarrow \{0, 1\}$ je spojité a nekonstantní $\implies$ ®X není souvislý. \lightning.
        \end{dukazin}
    \end{tvrzeni}

    \begin{tvrzeni}[Sjednocení souvislých množin]
        Ať $C_i \subseteq ®X, C_i$ souvislé, $i \in I, 0 \in I, C_i \cap C_0 ≠ \O$ pro $i \in I$. Pak $\bigcup C_i$ je souvislé.

        \begin{dukazin}
            Ať $O$ je neprázdná obojetná množina v $\bigcup C_i$. Existuje $j \in I$, $C_j \cap O ≠ \O$. $C_j \cap O$ je obojetná v $C_j$, $C_j$ je souvislá, tedy $C_j \subseteq O$. Tedy $C_0 \cap O ≠ \O$, tj. $C_0 \subseteq O$. Je-li $i \in I$ libovolná, pak $C_i \cap O ≠ \O$ a opět $C_i \subseteq O$.Tudíž $O = \bigcup C_i$, tj. $\bigcup C_i$ je souvislá.
        \end{dukazin}
    \end{tvrzeni}

    \begin{dusledek}
        Jsou-li $C_i$ souvislé v TP ®X, $i \in I$ a $\bigcap C_i ≠ \O$, pak $\bigcup_{i \in I}C_I$ je souvislá.

        \begin{dukazin}
            Předchozí s $C_0 := \{x_0\} \subseteq \bigcap C_i$.
        \end{dukazin}
    \end{dusledek}

    \begin{dusledek}
        Je-li $A \subseteq ®X$ souvislá a $A \subseteq M \subseteq \overline{A}$, pak $M$ je souvislá.

        \begin{dukazin}
            $C_a := A \cup \{a\}$ pro $a \in M \setminus A$. Vzhledem k předchozímu stačí ověřit, že $C_a$ je souvislá $(M = \bigcup_{a \in M \setminus A} C_a)$.
        \end{dukazin}
    \end{dusledek}

    \begin{veta}
        Buď ®X Tichonovův prostor. Pak ®X je souvislý $\Leftrightarrow$ $\beta ®X$ je souvislý.

        \begin{dukazin}
            $\implies$: $\overline{®X} = \beta ®X$, ®X je souvislá $\implies$ $\overline{®X}$ je souvislá $\implies$ $\beta®X$ je souvislá.

            $\Leftrightarrow$: Ať $\beta®X$ je souvislý. Ať $f: ®X \rightarrow \{0, 1\}$ je spojité zobrazení. Z vlastností $\beta ®X$ existuje spojité rozšíření $\overline{f}: \beta®X \rightarrow \{0, 1\}$. $\beta ®X$ je souvislý $\implies$ $\overline{f}$ je konstantní $\implies$ $f$ je konstantní $\implies$ ®X je souvislý.
        \end{dukazin}
    \end{veta}

    \begin{veta}[Součin souvislých prostorů]
        Ať $®X_i: i \in I$ jsou TP. Pak $\prod_{i \in I}®X_i$ je souvislý $\Leftrightarrow$ $\forall i \in I: ®X_i$ je souvislý.

        \begin{dukazin}
            Pokud některý $®X_i = \O$, tvrzení platí. Dále ať $®X_i ≠ \O$, $i \in I$. $\implies$: Je-li $\prod ®X_i$ souvislý, $\pi_j: \prod ®X_i \rightarrow ®X_j$ je spojité a na, tedy i $®X_j$ je souvislý.

            $\Leftarrow$: Nejprve pro dva prostory ®X, ®Y: $®X \times ®Y = (\{x_0\} \times ®Y) \cup \bigcup_{y \in ®Y} ®X \times \{y\}$. Tedy podle tvrzení výše je $®X \times ®Y$ souvislý. Indukcí dokážeme pro konečně mnoho. Obecně: $\forall i \in I$ fixujeme bod $x_i \in ®X_i$.
            $$ M:= \{(y_i)_{i \in I} \in \prod ®X_i: y_i = x_i \text{ pro všechna $i \in I$ až na konečně mnoho výjimek}\}. $$
            $$ M = \bigcup_{K \subseteq I \text{ konečná}} \( \prod_{i \in K}®X_i \times \prod_{i \in I \setminus K} \{x_i\}\). $$
            To znamená, že $M$ je souvislé, protože je sjednocením souvislých množin s průnikem obsahujícím $(x_i)_{i \in I}$. Ale $M$ je hustá v $\prod ®X_i$. $\overline{M}$ je souvislá, tedy $\prod ®X_i$ je souvislá.
        \end{dukazin}
    \end{veta}

    \begin{definice}[Komponenta souvislosti]
        Ať ®X je TP, $x \in ®X$, pak komponenta souvislosti bodu $x$ je největší souvislá množina, která $x$ obsahuje. Značíme ji $C_x$.

        \begin{dukazin}[Existence]
            Plyne z jednoho z důsledků: $\bigcup \{C | x \in C \land C \text{ je souvislá}\}$ je souvislá a maximální. Navíc je to vždy uzavřená množina.
        \end{dukazin}

        \begin{poznamkain}
            Komponenty souvislosti tvoří rozklad, tj. $C_x = C_y$ nebo $C_x \cap C_y = \O$.
        \end{poznamkain}
    \end{definice}

    \begin{tvrzeni}
        Jsou-li $®X_i$ TP a $x_i \in ®X_i$, $C_i$ komponenta bodu $x_i$ v $®X_i$. Pak $\prod C_i$ je komponenta $(x_i)_{i \in I}$ v $\prod ®X_i$.
        
        \begin{dukazin}
            Cvičení.
        \end{dukazin}
    \end{tvrzeni}

    \begin{definice}[Kvazikomponenty]
        Ať ®X je TP. Množina $Q \subseteq ®X$ se nazývá kvazikomponenta bodu $x \in ®X$ v prostoru ®X, pokud $Q = \bigcap \{Z | x \in Z \land Z \text{ obojetná}\}$. Značíme ji $Q_x$.

        \begin{poznamkain}
            $\forall x \in ®X: C_x \subseteq Q_x$. Navíc $Q_x$ je uzavřená. A opět tvoří rozklad prostoru ®X.
        \end{poznamkain}
    \end{definice}

    \begin{priklady}[TP ®X, že $C_x ≠ Q_x$]
            $®X \subseteq ®R^2$ skládající se ze 2 bodů $x, y$ a úseček k ním konvergujícím (\verb!:|| |  |   |!). $C_x = \{x\}$, $Q_x = \{x, y\}$.
    \end{priklady}

    \begin{lemma}[O průniku v kompaktu]
        Buď ®X kompaktní TP, ©A soubor uzavřených množin v ®X. $U \subseteq ®X$ otevřená a $\bigcap ©A \subseteq U$. Pak existuje konečný systém $©F \subseteq ©A: \bigcap ©F \subseteq U$.

        \begin{dukazin}
            Kdyby ne, pak $\forall ©F \subseteq ©A$ konečné: $\bigcap ©F \setminus U ≠ \O$. Tedy $©A \cup \{®X \setminus U\}$ má konečnou průnikovou vlastnost. ®X kompaktní: $\bigcap ©A \cap (®X \setminus U) ≠ \O$. Tedy $\bigcap ©A \not \subseteq U$. \lightning.
        \end{dukazin}
    \end{lemma}

    \begin{veta}
        V kompaktním TP komponenty a kvazikomponenty splývají.

        \begin{dukazin}
            Ať $x \in ®X$. $C_x \subseteq Q_x$. Pro $Q_x \subseteq C_x$ stačí dokázat, že $Q_x$ je souvislá. Předpokládejme $E \cup F$, $E, F$ uzavřené (v $Q_x$, a tedy i v ®X) disjunktní množiny. BÚNO $x \in E$. ®X je normální, tedy existují otevřené disjunktní množiny $U, V$: $E \subseteq U$, $F \subseteq V$. $Q_x = E \cup F \subseteq U \cup V$. Podle předchozího lemmatu existují obojetné množiny $Q_1, …, Q_n: Q_1 \cap … \cap O_n \subseteq U \cup V$. Otevřené $O \cap U = O \setminus V$ uzavřené, tedy $x \in O \cap U$, tedy $Q_x \subseteq O \cap U$. Tedy $F = \O$. Proto $Q_x$ je souvislá.
        \end{dukazin}
    \end{veta}

% 7. 5. 2021

    \begin{definice}[Křivková a oblouková souvislost]
        TP ®X se nazývá obloukově (resp. křivkově) souvislý, pokud $\forall x, y \in ®X, x≠y \exists f: [0, 1] \rightarrow f([0, 1])$ homeomorfní (resp. $f$ spojité), že $f(0) = x$, $f(1) = y$.

        \begin{poznamkain}
            Obecně je mezi nimi rozdíl, v Hausdorffových prostorech je to totéž.
        \end{poznamkain}
    \end{definice}

\section{Kontinua}
    \begin{definice}[Kontinuum]
        Kontinuum je kompaktní souvislý prostor.

        Jednoprvkové kontinuum se nazývá degenerované, ostatní nedegenerovaná.
    \end{definice}

    \begin{tvrzeni}
        Ať $K_n$ je klesající (vzhledem k inkluzi) posloupnost kontinuí, pak $\bigcap_{n = 1}^∞ K_n$ je opět kontinuum.

        \begin{dukazin}
            $K := \bigcap_{n = 1}^∞ K_n$ je zřejmě kompaktní. Předpokládejme, že $K = E \cup F$, kde $E, F$ jsou uzavřené disjunktní. Chceme $E = \O$ nebo $F = \O$. $K_1$ je normální, tedy existují otevřené disjunktní $U, V$, že $E \subseteq U$ a $F \subseteq V$. Podle lemmatu o průniku v kompaktu existuje $n \in ®N$, že $K_n \subseteq U \cup V$, tedy $K_1 = (K_n \cap U) \cup (K_n \cap V)$. $K_n$ je sovislá, tedy $K_n \cap U = \O$ nebo $K_n \cap V = \O$. Tedy $E = \O$ nebo $F = \O$.
        \end{dukazin}
    \end{tvrzeni}

    \begin{tvrzeni}[Bum do hranice (Boundary bumping theorem)]
        Ať ®X je kontinuum, $A$ vlastní uzavřená podmnožina ®X. Pak každá komponenta množiny $A$ protíná hranici $A$.

        \begin{dukazin}
            $A = \O$ nemá žádnou komponentu. Tedy $A ≠ \O, x \in A$, $C_x$ … komponenta bodu $x$ v $A$. Pro spor předpokládejme, že $C_x \cap \partial A = \O$. $A \subset ®X$, ®X je kontinuum, tedy $\partial A ≠ \O$. Víme, že $C_x$ je kvazikomponenta ®X, $C_x \subseteq A \setminus \partial A$. Tedy podle lemmatu o průniku v kompaktu existuje obojetná množina $Z$ v $A$, že $C_x \subseteq Z \subseteq A \setminus \partial A$. $Z$ je uzavřená v ®X, neprázdná, vlastní. Navíc $Z$ je otevřená v $A$ a neprotíná hranici, tedy $Z$ je otevřená v ®X ($Z$ je otevřená v otevřené $A \setminus \partial A$). $Z$ je tedy obojetná vlastní podmnožina v ®X, \lightning.
        \end{dukazin}
    \end{tvrzeni}

    \begin{veta}[Sierpinski]
        Ať ®X je kontinuum, $X_n, n \in ®N$ po dvou disjunktní uzavřené množiny v ®X, že $X = \bigcup X_n$. Pak všechna $X_n = \O$ až na jednu výjimku.

        \begin{dukazin}
            Je-li kontinuum $®D$ spočetným sjednocením uzavřených neprázdných disjunktních množin $Y_i, i \in ®N$, pak pro každé $i \in ®N$ existuje kontinuum $®C \subseteq ®D$, které je disjunktní s $Y_i$, ale není obsaženo v žádném $Y_j, j \in ®N$ (protíná alespoň 2). Důkaz:

            Fixujeme $j ≠ i$: existují disjunktní otevřené $U, V$: $Y_j \subseteq U$, $Y_i \subseteq V$. Buď $C$ komponenta libovolného bodu z $Y_j$ v množině $\overline{U}$. Podle bum do hranice víme, že $C \cap \partial \overline{U} ≠ \O$. Tj. $\partial \overline{U}\cap Y_j = \O$, tedy $C \nsubseteq Y_j$. $C \subseteq \overline{U} \subseteq ®D \setminus V$, tedy $C \cap Y_i = \O$.

            Důkaz věty: Sporem: Existují alespoň dva indexy $n ≠ m: X_n ≠ \O ≠ X_m$. Kdyby $\{k \in ®N: X_n ≠ \O\}$ byla konečná, pak ®X je sjednocením konečně mnoha disjunktních uzavřených neprázdných množin, tyto množiny by již byly obojetné, spor se souvislostí ®X.

            BÚNO (vyházíme prázdné a přeindexujeme) $X_n ≠ \O, \forall n \in ®N$. Indukcí najdeme posloupnost kontinuí $C_n$, že $C_{n+1} \subseteq C_n \land C_n \cap X_n = \O \land C_n$ není obsaženo v žádném $X_i$, $i \in ®N$. Podle prvního odstavce $®X = ®D$, $X_j = Y_j$, $i = 1$ existuje $C = C_j$. Dále uvažme $C_1$. To protíná nekonečně mnoho z množin $X_i$. V druhém kroku indukce použijeme první odstavec na $®D = C_1$, $\{Y_i | i \in ®N\} = \{C_1 \cap X_i | C_1 \cap X_i ≠ \O\}$, …

            $\bigcap C_n ≠ \O$, ať tedy $c \in \bigcap C_n$. $c \notin \bigcup X_n = ®X$. \lightning.
        \end{dukazin}
    \end{veta}

    \begin{definice}[Rozložitelné a nerozložitelné kontinuum]
        Kontinuum ®X se nazývá rozložitelné, pokud existují dvě vlastní podkontinua $®A, ®B$ (ne nutně disjunktní), že $®X = ®A \cup ®B$. Jinak je nerozložitelné.
    \end{definice}

    \begin{veta}[Charakterizace nerozložitelnosti]
        Kontinuum ®X je nerozložitelné, právě když každé jeho vlastní podkontinuum je v něĺ řídké (tj. uzávěr má prázdný vnitřek).

        \begin{dukazin}
            $\Leftarrow$ Kdyby $X = ®A \cup ®B$, ®A, ®B vlastní podkontinua, pak ®A, ®B řídké. $®A \cup ®B$ je řídká v ®X, \lightning.

            $\implies$: Ať $®Y \subseteq ®X$ je vlastní podkontinuum ®X, $\Int ®Y ≠ \O$. Ať $M = \overline{®X \setminus ®Y}$. Pak $M$ je souvislá: $®X = M \cup ®Y$ a $M$ je kontinuum, tedy ®X je rozložitelné. Nebo $M$ je nesouvislá: $M = E \cup F$, kde $E, F$ jsou uzavřené disjunktní neprázdné podmnožiny v $M$. Potom $(®Y \cup E) \cup (®Y \cup F)$. Ale tyto dvě množiny jsou vlastní uzavřené podmnožiny ®X, které jsou souvislé (neboť každá komponenta $E$ nebo $F$ protíná jejich hranici a ta je podmnožinou ®Y a sjednocení souvislých protínajících se ve stejném bodě je souvislé).
        \end{dukazin}
    \end{veta}

% 14. 5. 2021

\section{Nesouvislost (absence souvislosti)}
    \begin{definice}
        TP ®X se nazývá
        
        \begin{itemize}
            \item dědičně nesouvislý, pokud jeho jediné souvislé podprostory jsou jednobodové (tj. komponenty bodů jsou jednoprvkové).
            \item totálně nesouvislý, jestliže $\forall x, y \in ®X, x ≠ y, \exists$ obojetná $Z \subseteq ®Z: x \in Z \land y \notin Z$.
            \item nuldimenzionální, pokud ®X má bázi tvořenou obojetnými množinami (tj. pro $x \in ®X$, $U$ okolí $x$ existuje obojetná $Z \subseteq ®X: x \in Z \subseteq U$).
            \item silně nuldimenzionální, pokud ®X je normální a pro uzavřené disjunktní $E, F \subseteq ®X$ existuje obojetná $Z \subseteq ®X: E \subseteq Z \subseteq ®X \setminus F$.
        \end{itemize}
    \end{definice}

    \begin{poznamka}
        Definici silné nuldimenzionality lze zobecnit na Tichonovovy prostory: Tichonovův prostor ®X se nazve silně nuldimenzionální, pokud $\forall E, F \subseteq$, $E$, $F$ funkcionálně oddělené množiny (tj. $\exists f: ®X \rightarrow ®R$ spojitá, že $E \subseteq f^{-1}(0)$, $F \subseteq f^{-1}(1)$) existuje obojetná $Z \subseteq ®X: E \subseteq Z \subseteq ®X \setminus F$.
    \end{poznamka}

    \begin{tvrzeni}
        Silně nuldimenzionální $\implies$ nuldimenzionální $\implies$ totálně nesouvislý $\implies$ dědičně nesouvislý.

        \begin{dukazin}
            Snadné cvičení.
        \end{dukazin}
    \end{tvrzeni}

    \begin{poznamka}
        Obecně nelze žádnou z výše uvedených implikací obrátit.
    \end{poznamka}

    \begin{veta}
        Ať ®X je kompaktní TP. Pak je ekvivalentní:

        \begin{enumerate}
            \item ®X je dědičně nesouvislý.
            \item ®X je totálně nesouvislý.
            \item ®X je nuldimenzionální.
            \item ®X je silně nuldimenzionální.
        \end{enumerate}

        \begin{dukazin}
            Implikace směrem nahoru už máme, tedy stačí $1) \implies 4)$. Ale ukážeme nejprve $1) \implies 3)$ a pak $3) \implies 4)$. Ať $x \in ®X$, $U$ okolí $x$. Víme, že $C_x = \{x\}$. V kompaktním prostoru splývají komponenty a kvazikomponenty: $C_x = Q_x = \bigcap \{Z: x \in Z, Z \text{ je obojetná v } ®X\}$. Podle lemmatu o průniku v kompaktu existují obojetné $Z_1, …, Z_n$, že $x \in \bigcap Z_i$, $x \in \bigcap Z_i \subseteq U$. Tedy $1) \implies 3)$.

            Zbývá $3) \implies 4)$: Kompaktní prostor je normální. Ať $E, F$ jsou uzavřené disjunktní. Pro každé $x \in E$ existuje obojetná $Z_x \subseteq ®X$, že $x \in Z_x \subseteq X \setminus F$, díky nuldimenzionalitě. $\{Z_x | x \in E\}$ je otevřené pokrytí $E$. $E$ uzavřená v ®X, tedy $E$ kompaktní. $\exists x_1, …, x_n \in E: E \subseteq Z_{x_1} \cup … \cup Z_{x_n}$, což je obojetná $\subseteq ®X \setminus F$/ Tedy ®X je silně nuldimenzionální.
        \end{dukazin}
    \end{veta}

    \begin{veta}[Nuldimenzionalita v $\beta®X$]
        Buď ®X je Tichonovův prostor. Pak $\beta®X$ je nuldimenzionální, právě když ®X je silně nuldimenzionální.

        \begin{dukazin}
            $\implies$: $\beta®X$ je kompaktní, tedy $\beta®X$ je silně nuldimenzionální. Ať $f: ®X \rightarrow ®R$ je spojitá. Chceme, že $f^{-1}$ a $f^{-1}(1)$ oddělíme obojetnou: Existuje spojité rozšíření $\overline{f}: \beta®X \rightarrow [0, 1]$ funkce $f$. $\overline{f}^{-1}(0)$, $\overline{f}^{-1}(1)$ funkcionálně oddělené v $\beta ®X$. Tedy existuje obojetná $Z \subseteq \beta®X$, že $\overline{f}^{-1}(0)\subseteq Z \subseteq \beta®X \setminus \overline{f}^{-1}(1)$. Nyní $Z \cap ®X$ obojetná v ®X. $f^{-1}(0) = \overline{f}^{-1}(0)\cap®X \subseteq Z \cap ®X \subseteq ®X \setminus f^{-1}(1)$. $Z \cap ®X$ je obojetná a odděluje $f^{-1}(0)$ a $f^{-1}(1)$.

            $\Leftarrow$: $\beta®X$ je kompaktní, tedy stačí ukázat (díky předchozí větě), že ®X je totálně nesouvislý. Ať $a, b \in ®X$, $a ≠ b$. Existuje spojitá funkce $f: \beta®X \rightarrow [0, 1]$, $f(a) = 0$, $f(b) = 1$. $A := f^{-1}\(\[0, \frac{1}{3}\)\)$, $B := f^{-1}\(\(\frac{2}{3}, 1\]\)$. Jistě $a \in A$ a $b \in B$. $A, B$ jsou funkcionálně oddělené v $\beta®X$, otevřené. Tj. $A\cap®X$, resp. $B \cap ®X$, je hustá v $A$, resp. $B$. Navíc jsou oddělené v ®X. Ze silné nuldimenzionality ®X existuje obojetná $Z \subseteq ®X: A \cap ®X \subseteq Z \subseteq ®X \setminus B$. Uvažujme spojitou funkce $g: ®X \rightarrow \{0, 1\}$, $g|_Z = 0$, $g|_{X \setminus Z} = 1$. $g$ lze spojitě rozšířit na $\overline{g}: \beta®X \rightarrow \{0, 1\}$. Ať $W := \overline{g}^{-1}(0)$. $W \supseteq Z$, $W$ obojetná, $W \cap B = \O$. $W \supseteq A$. $a \in W$, $b \notin W$.
        \end{dukazin}
    \end{veta}

    \begin{poznamka}
        Podprostor nuldimenzionálního prostoru je nuldimenzionální. Součin nuldimenzionálních je nuldimensionální. Tedy $\{0, 1\}^\kappa$ je vždy nuldimenzionální.
    \end{poznamka}

    \begin{tvrzeni}
        Každý nuldimenzionální prostor ®X lze vnořit do $\{0, 1\}^A$, kde za $A$ lze volit váhu (nebo bázi) ®X.

        \begin{dukazin}
            Ať ©B je báze $X$, která je tvořena obojetnými množinami, pro kterou je $|©B| = w(®X)$. Pro $B \in ©B$ uvažme $f_B(x) = 0$, $x \in B$ a $1$ jinak. $\{f_B|B \in ©B\}$ odděluje body a uzavřené množiny, tedy podle lemmatu o Tichonovově vnoření je $\triangle_{B \in ©B} : x \in ®X \mapsto (f_B(x))_{B \in ©B}$ je vnoření $®X \rightarrow \{0, 1\}^{©B}$.
        \end{dukazin}
    \end{tvrzeni}

    \begin{dusledek}
            Každý kompaktní metrický (má spočetnou váhu) nuldimenzionální prostor lze vnořit do Cantorova diskontinua.
    \end{dusledek}
    
    \begin{veta}[Topologická charakterizace Cantorova diskontinua]
        Každý neprázdný nuldimenzionální kompaktní metrizovatelný prostor je homeomorfní Cantorovu diskontinuu.

        \begin{dukazin}
            Bez důkazu.
        \end{dukazin}
    \end{veta}

    \begin{poznamka}[Stonova dualita]
        Kompaktní nuldimenzionální TP se spojitými zobrazeními se dají přiřadit Booleovým algebrám s homomorfismy
    \end{poznamka}

% 21. 5. 2021

\section{Topologická dimenze}
    \begin{definice}[Malá induktivní dimenze, Menger-Urysohn]
        Pro ®X regulární definujeme

        \begin{itemize}
            \item $\ind \O = -1$.
            \item $\ind ®X ≤ n ≥ 0$, pokud $\forall x \in ®X\ \forall U$ okolí $x$ $\exists$ otevřená $V: x \in V \subseteq U: \ind(\partial V) ≤ n-1$.
            \item $\ind ®X = n$, pokud $\ind ®X ≤ n$ a neplatí $\ind ®X ≤ n-1$.
            \item $\ind ®X = ∞$, pokud $\forall n \in ®N: \ind ®X \not≤ n$.
        \end{itemize}
    \end{definice}

    \begin{poznamka}
        $\ind ®X = -1 \Leftrightarrow X = \O$. $\ind ®X ≤ 0 \Leftrightarrow ®X$ je nuldimenzionální ($\partial V = \O \Leftrightarrow V$ je obojetná). $\ind([0, 1]) = 1$ (není nuldimenzionální). $\ind[0, 1]^n ≤ n$ indukcí. (Naopak už pro $n = 2$ se těžko dokazuje, že platí rovnost. Viz dále.)
    \end{poznamka}

    \begin{definice}[Velká induktivní dimenze, Brouwer-čech]
        Pro ®X normální definujeme
        
        \begin{itemize}
            \item $\Ind \O = -1$.
            \item $\Ind ®X ≤ n ≥ 0$, pokud $\forall E \subseteq ®X$ uzavřenou a $\forall U$ otevřenou: $E \subseteq U\ \exists V$ otevřená: $E \subseteq V \subseteq U$ a $\Ind(\partial V) ≤ n-1$.
            \item $\Ind ®X = n$, pokud $\Ind ®X ≤ n$ a neplatí $\Ind ®X ≤ n-1$.
            \item $\Ind ®X = ∞$, pokud $\forall n \in ®N: \ind ®X \not≤ n$.
        \end{itemize}
    \end{definice}

    \begin{poznamka}
        $\Ind ®X ≤ 0 \Leftrightarrow ®X$ je silně nuldimenzionální.
    \end{poznamka}

    \begin{definice}[Pokrývací dimenze, Čech-Lebesgue]
        Říkáme, že systém ©A podmnožin množiny $X$ je řádu $n$, pokud $n$ je největší přirozené číslo pro které existují po dvou různé $A_1, …, A_{n+1} \in ©A: A_1 \cap A_2 \cap … \cap A_{n+1} ≠ \O$.

        Pro normální prostor ®X definujeme
        
        \begin{itemize}
            \item $\dim \O = -1$.
            \item $\dim ®X ≤ n ≥ 0$, pokud každé konečné otevřené pokrytí ®X má konečné otevřené zjemnění řádu nejvýše $n$.
            \item $\dim ®X = n$, pokud $\dim ®X ≤ n$ a neplatí $\dim ®X ≤ n-1$.
            \item $\dim ®X = ∞$, pokud $\forall n \in ®N: \dim ®X \not≤ n$.
        \end{itemize}
    \end{definice}

    \begin{veta}
        Ať ®X je normální TP. Pak $\dim ®X ≤ 0$ $\Leftrightarrow$ ®X je silně nuldimenzionální.

        \begin{dukazin}
            $\implies$: Ať $E, F$ jsou uzavřené disjunktní. Ať $©U = \{®X \setminus E, ®X \setminus F\}$ otevřené pokrytí ®X. $\exists$ konečné otevřené zjemnění ©V toho ©U, které má řád $≤ 0$, tedy prvky ©V jsou navzájem disjunktní. Tedy každé $V \in ©V$ je obojetná, $\bigcup \{V \in ©V|V \subseteq ®X \setminus F\}$ obojetná, obsahující $E$ (protože pokud $V \cap E ≠ \O$ pro $V \in ©V$, pak $V \subseteq ®X \setminus F$) a disjunktní s $F$.

            $\Leftarrow$: Ať ©U je otevřené konečné pokrytí ®X. Podle lemmatu o skrčení existuje uzavřené $F_U: U \in ©U, F_U \subseteq U$, $\bigcup \{F_U | U \in ©U\}$. ®X silně nuldimenzionální, tedy $\exists$ obojetné $C_U: F_U \subseteq C_U \subseteq U$, $U \in ©U$. $\{C_U | U \in ®U\}$ je konečné obojetné pokrytí ®X zjemňující ©U. Toto pokrytí můžeme „rozdisjunktnit“ a dostaneme disjunktní otevřené (obojetné) konečné pokrytí řádu 0 (zjemnění ©U).
        \end{dukazin}
    \end{veta}

    \begin{poznamka}
        $\dim [0, 1] = 1$. $\dim [0, 1]^n = n$ (zase dosti těžké dokázat).
    \end{poznamka}

    \begin{tvrzeni}[Dimenze (uzavřeného) podprostoru]
        Je-li ®X regulární a $A \subseteq ®X$, pak $\ind A ≤ \ind ®X$.

        Je-li ®X normální a $A \subseteq ®X$ uzavřená, pak $\Ind A ≤ \Ind ®X$ a $\dim A ≤ \dim ®X$.

        \begin{dukazin}
            Indukcí podle $\ind ®X$ ($\Ind ®X, \dim ®X$): Pokud $\ind ®X = -1$, pak $®X = \O = A$, $\ind A = -1$. Předpokládejme, že pro všechny regulární prostory ®Y, kde $\ind ®Y ≤ n-1$, a $B \subseteq ®Y$ je $\ind(B) ≤ \ind(®Y)$. $\ind ®X = n$, $A \subseteq ®X$: chceme $\ind A ≤ n$. Ať $x \in A$, $U$ okolí $x$ v $A$. Existuje $W$ okolí $x$ v ®X takové, že $W \cap A = U$. $\exists V$ otevřené okolí $x$ v ®X: $\ind(\partial V) ≤ n-1$, $x \in V \subseteq W$. $x \in V \cap A \subseteq U$, $V \cap A$ otevřená v $A$ a $\partial_A(V \cap A) \subseteq \partial_{®X}V$, tedy podle indukčního předpokladu $\ind \partial_A(V \cap A) ≤ \ind \partial_{®X}V ≤ n-1$, tj. $\ind A ≤ n$. Ostatní 2 dimenze analogicky.
        \end{dukazin}
    \end{tvrzeni}

    \begin{veta}[Charakterizace pokrývací dimenze]
        Pro normální prostor ®X je ekvivalentní:
        
        \begin{enumerate}
            \item $\dim ®X ≤ n$.
            \item Každé konečné otevřené pokrytí ®X má otevřené skrčení řádu $≤ n$.
            \item Každé konečné otevřené pokrytí ®X má uzavřené skrčení řádu $≤ n$.
            \item Každé konečné otevřené pokrytí ®X má konečné uzavřené zjemnění řádu $≤ n$.
        \end{enumerate}

        \begin{dukazin}
            $1) \implies 2)$: Ať $©U = \{U_1, …, U_n\}$ otevřené pokrytí ®X. Dle 1) existuje ©V konečné otevřené zjemnění toho ©U řádu nejvýše $n$. $V_i' := \bigcup \{V \in ©V|V \subseteq U_i \land V \nsubseteq \bigcup V_j': j < i\}$. $©V' = \{V'_i | i \in [n]\}$ je skrčení ©U řádu $≤ n$.

            $2) \implies 3)$ je z lemmatu o skrčení. $3) \implies 4)$ je triviální.

            $4) \implies 1)$: Ať ©U je konečné otevřené pokrytí ®X a ©F je jeho uzavřené konečné zjemnění řádu $≤ n$. $©F = \{F_1, …, F_m\}$. Pro $i ≤ m$ fixujeme $U_i \in ©U: F_i \subseteq U_i$. Indukcí najdeme $©V = \{V_1, …, V_m\}$, že $F_i \subseteq V_i \subseteq \overline{V_i} \subseteq U_i$, že řád ©V $≤ n$. Ať $A_1 = \bigcup \{H_1 \cap … \cap  H_{n+1} | H_i \in ©F \setminus \{F_1\}, H_i \text{ jsou po dvou různé}\}$. $A_1$ uzavřená. Navíc $A_1 \cap F_1 = \O$, protože řád ©F $≤ n$. Ať $V_1$ je otevřená, že $F_1 \subseteq V_1$ a $\overline{V_1} \cap A_1 = \O$. Nyní $\{\overline{V_1}, F_2, F_3, …, F_m\}$ má řád nejvýše $n$.

            Předpokládejme, že máme otevřené $V_1, …, V_{j-1}$, že $F_i \subseteq V_i \subseteq \overline{V_i} \subseteq U_i$, $i ≤ j-1$ a systém $\{\overline{V_1}, \overline{V_2}, …, \overline{V_{j-1}}, F_j, F_{j+1}, …, F_m\}$ má řád $≤ n$. Hledáme $V_j$. Ať $A_j$ je sjednocením průniků $n+1$ různých množin vybraných ze systému $\{\overline{V_1}, \overline{V_2}, …, \overline{V_{j-1}}, F_{j+1}, F_{j+2}, …, F_m\}$. $A_j$ je uzavřená a disjunktní s $F_j$. Opět z normality existuje otevřená $V_j: F_j \subseteq V_j \subseteq \overline{V_j} \subseteq U_j \setminus A_j$. Řád $\{\overline{V_1}, …, \overline{V_j}, F_{j+1}, …, F_n\} ≤ n$.

            $©V:= \{V_1, …, V_m\}$ je otevřené zjemnění ©U řádu $≤ n$.
        \end{dukazin}
    \end{veta}

    \begin{veta}[Součtová pro $\dim$]
        Je-li ®X normální TP a $F_i \subseteq ®X$ uzavřené, $i \in ®N$, $\dim F_i ≤ n$ a $X = \bigcup_{i \in ®N}F_i$, pak $\dim ®X ≤ n$.
    \end{veta}

    \begin{tvrzeni}
        Pro normální TP ®X: $\ind ®X ≤ \Ind ®X$.

        \begin{dukazin}
            Přímočarý, indukcí podle $\Ind ®X$.
        \end{dukazin}
    \end{tvrzeni}

% 28. 5. 2021

    \begin{veta}[Součinová věta pro $\dim$]
        Je-li ®X normální TP a $F_i \subseteq ®X$ uzavřené, $i \in ®N$, $\dim F_i ≤ n$ a $®X = \bigcup_{i \in ®N}F_i$, pak $\dim ®X ≤ n$.

        \begin{dukazin}
            Ať $©U = \{U_i | i \in [k]\}$ je otevřené pokrytí ®X. $F_0 := \O$. Induktivně definujeme otevřené pokrytí $©U_0, ©U_1, …$, že $©U_j = \{U_{j, i} | i \in [k]\}$, $U_{j, i} \subseteq U_{j-1, i}$ pro $j ≥ 1$ a $U_{0, i} \subseteq U_i$, $\ord\{\overline{U_{j, i}} \cap F_j | i \in [k]\} ≤ n$.

            $j = 0: U_{j, 0} := U_j$. Předpokládejme, že máme $©U_j$, $j < m ≥ 1$, splňující podmínky výše. $\{F_m \cap U_{m-1, i}|i \in [k]\}$ je otevřené pokrytí $F_m$. Podle charakterizace dimenze $\dim$ má otevřené skrčení $\{V_1, …, V_k\}$ řádu nejvýše $n$. Ať $W_i := (U_{m-1, i}, F_m) \cup V_i \subseteq U_{m-1, i}$. $\{W_i | i \in [k]\}$ je otevřené pokrytí ®X. Řád $\{F_m \cap W_i | i \in [k]\} ≤ n$. Podle lemmatu o skrčení existuje skrčení tohoto $\{W_i | i \in [k]\}$ na $\{U_{m, i} | i \in [k]\}$, že $\overline{U_{m, i}} \subseteq W_i$. $©U_m := \{U_{m, i} | i \in [k]\}$ splňuje požadované podmínky. Nyní $\forall x \in ®X\ \forall j \in ®N: \exists i(x) \in [k]: x \in U_{j, i(x)}$. $\exists i(x) \in [k] : \{j | i(x) = i(j, x)\}$ je nekonečné. $x \in \bigcap_{j=1}^∞$. Tedy $\{\bigcap_{j=1}^∞ \overline{U_{j, i}} | i \in [k]\}$ je uzavřené pokrytí ®X a navíc skrčení ©U. K tomu je řádu $≤n$. Tedy podle charakterizace $\dim$ je $≤ n$.
        \end{dukazin}
    \end{veta}

    \begin{veta}
        Ať ®X je normální, pak $\Ind \beta ®X = \Ind ®X$, $\dim \beta ®X = \dim ®X$.

        \begin{dukazin}
            Bez důkazu. Pro důkaz viz Engelking: Theory of dimensions Finite and Infinite.
        \end{dukazin}
    \end{veta}

    \subsection{Dimenze separabilních metrizovatelných prostorů}
        \begin{veta}
            Je-li ®X separabilní metrizovatelný, pak $\dim ®X = \ind ®X = \Ind ®X$.

            Jsou-li ®X, ®Y neprázdné separabilní metrizovatelné, pak $\dim ®X \times ®Y ≤ \dim ®X + \dim ®Y$.
        \end{veta}

        \begin{veta}
            Je-li ®X separabilní metrizovatelný prostor, pak existuje metrizovatelná kompaktifikace $\tilde{®X}$, že $\dim ®X = \dim \tilde{®X}$.
        \end{veta}

        \begin{lemma}[O oddělování]
            Pokud pro separabilní metrizovatelný prostor ®X je $\dim ®X ≤ n$, pak pro každou posloupnost disjunktních dvojic $(A_1, B_1), …, (A_{n+1}, B_{n+1})$ uzavřených množin existují otevřené množiny $U_1, …, U_{n+1}: A_i \subseteq U_i \subseteq \overline{U_i} \subseteq ®X \setminus B_i$, že $\bigcap_{i = 1}^{n+1} \partial U_i = \O$.
        \end{lemma}

    \subsection{Dimenze eukleidovských prostorů}
        \begin{veta}[Bronworova o pevném bodě]
            Každé spojité zobrazení $f: [0, 1]^n \rightarrow [0, 1]^n$ má pevný bod, tj. $x \in [0, 1]^n: f(x) = x$.
        \end{veta}

        \begin{veta}
            $$ \dim [0, 1]^n = \dim ®R^n = n. $$

            \begin{dukazin}[Myšlenka]
                $\dim [0, 1]^n ≤ n$: Indukcí podle $n$: $n = 1$ je cvičení. Předpokládejme, že $\dim [0, 1]^n ≤ n$, chceme $\dim [0, 1]^{n+1} ≤ n+1$: Otevřené koule (při maximové metrice) v $[0, 1]^{n+1}$ jsou „krychličky“, jejichž hranice jsou sjednocením konečně mnoha množin homeomorfních s $[0, 1]^n$. Ze součtové věty je tedy dimenze hranice každé krychle v $[0, 1]^{n+1}$ nejvýše $n$-dimenzionální. $®R^n$ je spočetným sjednocením prostorů homeomorfních $[0, 1]^n$ tedy ze součtové věty $\dim ®R^n ≤ n$.

                $\dim [0, 1]^n ≥ n$: Ať pro spor $\dim [0, 1]^n ≤ n-1$. $A_i := \{x \in [0, 1]^n | x_i = 0\}$ a $B_i := \{x \in [0, 1]^n | x_i = 1\}$. Zřejmě $\bigcup A_i \cup \bigcup B_i$ tvoří hranici $[0, 1]^n$ v $®R^n$. Podle lemmatu o oddělování existují otevřené množiny $U_1, …, U_n: A_i \subseteq U_i \subseteq \overline{U_i} \subseteq [0, 1]^n \setminus B_i$, $i \in [n]$ a $\partial U_1 \cap … \cap \partial U_n = \O$. Definujme $f_i : [0, 1]^n \rightarrow [0, 1]: f_i(x) = \frac{1}{2}\frac{\rho(x, \partial U_i)}{\rho(x, \partial U_i) + \rho(x, A_i)} + \frac{1}{2}$, pro $x \in U_i$, a $-\frac{1}{2}\frac{\rho(x, \partial U_i)}{\rho(x, \partial U_i) + \rho(x, B_i)} + \frac{1}{2}$ jinak. $f_i$ je spojité, $f(x) = 1$ pro $x \in A_i$, $f(x) = 0$ pro $x \in B_i$. $f_i^{-1}(1/2) = \partial U_i$. $f := f_1 \triangle … \triangle f_n$, $f: [0, 1]^n \rightarrow [0, 1]^n$ spojité. Bod $a = (1/2, …, 1/2)$ neleží v obrazu $f$, protože $\bigcap_{i=1}^n \partial U_i = \O$.

                Ať $g: [0, 1]^n \setminus \{a\} \rightarrow [0, 1]^n$ je projekce z bodu $a$ na $[0, 1]^n$. $g \circ f: [0, 1]^n \rightarrow [0, 1]^n$ spojité, ale nemá pevný bod.
            \end{dukazin}
        \end{veta}

        \begin{veta}[Vnoření do euklidovských prostorů]
            Separabilní metrizovatelný TPP dimenze $≤ n$ lze vnořit do $®R^{2n+1}$. (Menší dimenze ®R by už nefungovala.)

            Dokonce lze takový prostor vnořit do $\{x \in ®R^{2n+1} | x \text{ má nejvýše $n$ racionálních souřadnic}\}$ (který má dimenzi $n$, takže se mu říká univerzální ?).
        \end{veta}

% 4. 6. 2021

\section{Hyperprostory}

        \begin{definice}[Hyperprostory]
            ®X TP. $©K(®X) = \{K \subseteq ®X | K \text{ kompaktní, } K ≠ \O\}$. Na $©K(®X)$ lze zavést přirozeným způsobem topologii. Subbáze této topologie je tvořena množinami
            $$ \Gamma(U) = \{K \in ©K(®X): K \cap U ≠ \O\}, $$
            $$ \Lambda(V) = \{K \in ©K(®X) | K \subseteq V\}, $$ 
            pro $U, V$ otevřené v ®X.
        \end{definice}

        \begin{tvrzeni}
            Je-li ®X kompaktní, pak $©K(®X)$ je kompaktní.

            \begin{poznamkain}
                Opačná implikace je triviální, protože ®X je uzavřený v $©K(®X)$.
            \end{poznamkain}

            \begin{dukazin}
                Aplikace Alexandrovova lemmatu: Ať $\{\Gamma(U_i), \Lambda(V_j) | i \in I, j \in J\}$ je pokrytí $©K(®X)$. Chceme vybrat konečné podpokrytí. $U:= \bigcup_{i \in I}U_i$ je otevřené. $F := ®X \setminus U$ uzavřená. Tedy $F$ je kompaktní. Buď $F = \O$, tak $\{U_i | i \in I\}$ je pokrytí ®X, lze z něj tedy vybrat konečné podpokrytí $U_1, …, U_n$ a $\{\Gamma(U_i): i \in [n]\}$ je pokrytí $©K(®X)$. Nebo $F ≠ \O$, tedy $F \in ©K(®X)$ a $F \notin \Gamma(U_i)$ pro žádné $i \in I$. $\exists j \in J: F \in \Gamma(V_j)$. Tedy $F \subseteq V_j$.

                $M:= ®X \setminus V_j \subseteq U$ je uzavřená, tedy kompaktní, $M \subseteq \bigcup_{i \in I} U_i$, $\exists i_1, …, i_n$, že $M \subseteq U_{i_1} \cup … \cup U_{i_n}$. Tvrdíme, že $\Gamma(U_{i_1}) \cup … \cup \Gamma(U_{i_n}) \cup \Lambda(V_j) = ©K(®X)$: Ať $K \in ©K(®X)$, pak $K \subseteq V_j \implies K \in \Lambda(V_j)$, $K \notin V_j \implies K \cup M ≠ \O. \exists i_r: K \cap U_{i_r} ≠ \O$, Tedy $©K(®X)$ je kompaktní podle A. lemmatu.
            \end{dukazin}
        \end{tvrzeni}

        \begin{tvrzeni}
            Je-li ®X kontinuum, pak $©K(®X)$ je kontinuum.

            Je-li ®X úplně metrizovatelný, pak $©K(®X)$ je úplně metrizovatelný.
        \end{tvrzeni}

        \begin{poznamka}
            $®Q = [0, 1]^{®N}$. Každý kompaktní metrický prostor lze vnořit do ®Q. Tedy $©K(®Q)$ obsahuje (jako body) topologické kopie všech kompaktních metrických prostorů.

            Nejčastější prostor je Cantorovo diskontinuum.
        \end{poznamka}

\end{document}
