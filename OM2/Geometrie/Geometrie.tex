\documentclass[12pt]{article}                   % Začátek dokumentu
\usepackage{../../MFFStyle}                     % Import stylu

\begin{document}

% 1. 3. 2021
\section{plochy}
    \begin{definice}[Regulární plocha]
        Nechť $k < n$ jsou přirozená čísla. Nechť je $\phi$ spojitě diferencovatelné zobrazení otevřené podmnožiny $©O \subseteq ®R^k$ do $®R^n$. Řekněme, že $\phi$ je regulární, pokud je to homeomorfismus $©O$ na $M = \phi(©O)$ a pokud má Jacobiho matice $J\phi$ hodnost rovnou $k$ ve všech bodech $©O$. Množinu $\phi(©O)$ pak nazveme lokální $k$-plochou.

        Řekněme, že množina $M \subseteq ®R^n$ je $k$-plocha pokud pro každý bod $x \in M$ existuje okolí $U_x$ v $®R^n$ takové, že $M \cap U$ je lokální $k$-plocha.
    \end{definice}

% 2. 3. 2021
    Podobný začátek jako Analýza na varietách (AnVar).

\end{document}
