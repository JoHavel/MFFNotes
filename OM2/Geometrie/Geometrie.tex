\documentclass[12pt]{article}                   % Začátek dokumentu
\usepackage{../../MFFStyle}                     % Import stylu

\begin{document}

% 1. 3. 2021
\section{plochy}
    \begin{definice}[Regulární plocha]
        Nechť $k < n$ jsou přirozená čísla. Nechť je $\phi$ spojitě diferencovatelné zobrazení otevřené podmnožiny $©O \subseteq ®R^k$ do $®R^n$. Řekněme, že $\phi$ je regulární, pokud je to homeomorfismus $©O$ na $M = \phi(©O)$ a pokud má Jacobiho matice $J\phi$ hodnost rovnou $k$ ve všech bodech $©O$. Množinu $\phi(©O)$ pak nazveme lokální $k$-plochou.

        Řekněme, že množina $M \subseteq ®R^n$ je $k$-plocha pokud pro každý bod $x \in M$ existuje okolí $U_x$ v $®R^n$ takové, že $M \cap U$ je lokální $k$-plocha.
    \end{definice}

% 2. 3. 2021
    Podobný začátek jako Analýza na varietách (AnVar).

% 5. 3. 2021
    \begin{definice}[Difeomorfismus]
        Standardně.
    \end{definice}

    \begin{veta}[Věta o lokálním difeomorfismu]
        Pokud je Jakobián nenulový, pak existuje difeomorfní okolí.
    \end{veta}

    \begin{definice}[Hladký bod hranice]
        Nechť $\Omega \subseteq ®R^n$ je otevřená podmnožina. Označme symbolem $®H^n$ otevřený podprostor. Řekněme, že bod $a \in H(\Omega) = \overline{\Omega} \setminus \Omega$ je hladký bod hranice, pokud existuje okolí $U$ bodu $a$ a difeomorfismus $\Phi$ na? $U$ takový, že
        $$ \Phi(\Omega \cap U) = \Phi(U)\cap ®H^n. $$ 
        (Narovnání hranice pomocí difeomorfismu.)

        Množinu všech hladkých bodů hranice značíme $H^*(\Omega)$.
    \end{definice}

    \begin{definice}[Vnější algebra vektorového prostoru]
        Nechť ¦V je vektorový prostor nad reálnými čísly a $\{e_1, …, e_n\}$ je jeho pevně zvolená báze. Vnější algebra $\Lambda^*(¦V)$ vektorového prostoru ¦V je definována jako algebra nad tělesem reálných čísel, jejíž báze je množina
        $$ \{e_I | I \subseteq \{1, … n\}\}. $$
        A prvky báze splňují
        $$ e_I \wedge e_J := \begin{cases} 0 & I \cap J ≠ \O \\ \sgn\binom{I, J}{I \cup J} e_{I \cup J} \end{cases}. $$

        Vzhledem k bilinearitě násobení v algebře je tímto výrazem násobení vektorů již plně definováno.
    \end{definice}

    \begin{poznamka}
        $e_\O$ je podle definice jednotka.

        $\Lambda^k(¦V)$, což je lineární obal bází $\Lambda^*(¦V)$ velikosti $k$, se nazývá $k$-tá vnější algebra a její prvky jsou $k$-vektory.
    \end{poznamka}

    \begin{veta}
        Pro vektorový prostor ¦V s bází $e_1, …, e_n$ a pro libovolná $k, l \in \{1, …, n\}$.
        \begin{enumerate}
            \item $\dim \Lambda^k(¦V) = \binom{n}{k}, \dim \Lambda^*(¦V) = 2^n.$
            \item $\wedge$ je asociativní.
            \item $e_I = e_{i_1} \wedge … \wedge e_{i_k}, |I|=k$.
            \item Je-li $\omega \in \Lambda^k(¦V), \tau \in \Lambda^l(¦V)$, pak $\omega \wedge \tau = (-1)^{kl}\tau \wedge \omega$.
            \item Nechť $¦v_1, …, ¦v_k \in ¦V$ jsou vektory. Potom (matice $V_I$ má za sloupce vektory $¦v_i$ a řádky jsou vybrány pouze ty s indexem $I$)
                $$¦v_1 \wedge … \wedge ¦v_k = \sum_{I \subseteq [n], |I|=k} \det ¦V_I·e_I.$$
        \end{enumerate}

        \begin{dukazin}
            Jednoduchý. (Ve skriptech anvar…).
        \end{dukazin}
    \end{veta}

\end{document}
