\documentclass[12pt]{article}					% Začátek dokumentu
\usepackage{../../MFFStyle}					    % Import stylu



\begin{document}

% 14. 02. 2022

\section*{Organizační úvod}
\begin{poznamka}[Organizační úvod]
	Na zkoušku není nutné mít zápočet. Přednáška má stránku (se slidy, kvízem). Záznamy z loňského roku budou.
\end{poznamka}

\section{Úvod}

% 21. 02. 2022

\begin{definice}[Prostředí]
	Prostředí může být s plnou informací nebo s částečnou informací (podle toho, zda agent dostává svými senzory vše, nebo ne), dále může být buď deterministické nebo stochastické (podle toho, zda je plně určené svým stavem a akcí), dále je buď epizodní nebo sekvenční (podle toho, zda se pořád opakuje to samé, např. návštěva u lékaře, nebo zda se neopakuje), dále statické nebo dynamické (podle toho, zda ho ovlivňuji jen já, nebo i něco jiného, semidynamické je, když přemýšlení ovlivňuje můj výkon, např. hry s hodinami), dále diskrétní nebo spojité, dále jedno-agentová nebo více-agentové (kompetitivní/kooperativení).
\end{definice}

\begin{definice}[Reflex agent]
	Simple RA: Na základě pozorování světa vrátí akci. (V podstatě neprocedurální funkce beroucí svět a vracející akci.)

	Model-based RA: Kromě vracení akce i mění svůj stav (pomocí stavového modelu).
\end{definice}

\begin{definice}[Goal-based agent]
	Funguje podobně jako Reflexní agent, ale má ještě navíc nějaký cíl (který lze měnit např. příkazem), který ovlivňuje akci.
\end{definice}

\begin{definice}[Stav]
	Stav může být reprezentován buď atomicky (nemá žádnou strukturu) nebo Factored? (stav je vektor hodnot) nebo strukturovaně (stav je množina objektů spojených různými relacemi).
\end{definice}

\begin{definice}[Problem solving agent]
	PSA je typ goal-based agenta, který používá atomickou reprezentaci stavů, cíl je jedním ze stavů a akce jsou popisy, jak se stavy mění.

	Úkolem je najít sekvenci akcí, která dosáhne cílového stavu. Hledá se pomocí nějakého search algoritmu.
\end{definice}

\begin{definice}[Dobře definovaný problém]
	Dobře definovaný problém má počáteční stav, přechodový model (který má rozumnou míru abstrakce, např. neovládá každý sval zvlášť), jím implikované stavy a test určující cílové stavy.

	Tím je implicitně definovaný search tree. (Na něm je algoritmus tree search, který strom prochází tak, že do „množiny“ postupně přidává syny prvků, které v ní už jsou. Často je však problém s opakováním stavů.)
\end{definice}

\begin{definice}[Graph search]
	Graph search je skoro totéž jako tree search, jen si u každého stavu pamatuje, zda již byl navštíven, nebo ne.

	Search tree tohoto algoritmu má každý stav nanejvýš jednou.
\end{definice}

\begin{definice}[Kompletní algoritmus]
	Algoritmus je kompletní, když správně najde řešení respektive dokáže, že neexistuje, pro všechny vstupy.
\end{definice}

\begin{poznamka}[Neinformované prohledávání (prohledávání obecného stavového prostoru)]
	Následně se probíral breadth-first search, depth-first search a backtracking (na rozdíl od DFS nenačte hned všechny následníky vrcholu, ale jde jeden po druhém, což nemusí být vždy možné).
\end{poznamka}

\begin{definice}[Informované (heuristické) algoritmy, best-first search, A$^*$]
	Algoritmy, které využívají pro rozhodování používají navíc tzv. heuristiku.

	Patří mezi ně např. best-first search, který prohledává stav, kde je nejmenší evaluační funkce $f(n)$, která kromě vzdálenosti od počátku ($g(n)$) bere v potaz i heuristiku $h(n)$. Ten podle volby $f(n)$ může být např. greedy best-first search: $f(n) = h(n)$, nebo A$^*$: $f(n) = g(n) + h(n)$.
\end{definice}

\begin{definice}[Přípustná a monotónní heuristika]
	Přípustná heuristika je taková, která vrací hodnotu mezi nulou a nejlepší cestou.

	Monotónní (nebo také konzistentní) heuristika je taková, která splňuje „trojúhelníkovou nerovnost“ (tedy rozdíl heuristik nemůže být větší než cesta mezi nimi).
\end{definice}

\begin{tvrzeni}
	Je-li heuristika monotónní (a nezáporná), pak už je přípustná.

	\begin{dukazin}
		$$ h(\text{start}) - h(\text{cíl}) ≤ |\text{nejkratší cesta}|. $$
	\end{dukazin}
\end{tvrzeni}

\begin{tvrzeni}
	A$^*$ v tree-search je optimální (první nalezená cesta je nejkratší).

	\begin{dukazin}
		V otevřených vrcholech je vždy vrchol nejkratší cesty (jelikož počáteční stav je a vždy když vrchol uzavíráme, tak přidáme všechny sousedy).

		Cíl musíme najít po nejkratší cestě, protože v cíli je $f(n) = g(n)$ a my jsme ho poprvé potkali při nejmenším $f(n)$.
	\end{dukazin}
\end{tvrzeni}

\begin{tvrzeni}
	Je-li použitá heuristika monotónní, pak A$^*$ v graph-search je optimální.

	\begin{dukazin}
		Jednoduchý, podobně jako u tree-search, navíc se dokazuje jen, že do každého vrcholu přijde po nejkratší cestě.
	\end{dukazin}
\end{tvrzeni}

\begin{definice}[Dominance]
	Heuristika $h_1$ dominuje heuristice $h_2$, když $\forall n: h_1(n) ≥ h_2(n)$.
\end{definice}

% 28. 02. 2022

\begin{definice}[Forward checking]
	Testuje budoucí pozice, jestli tam vůbec může být cíl.
\end{definice}

\begin{definice}[Problém splňování]
	Problém splňování? sestává z konečného počtu proměnných (popisujících svět), oborů hodnot pro každou proměnnou a konečné množiny podmínek.

	Řeší se Backtrackingem.
\end{definice}

\begin{definice}[Přípustné řešení]
	Přípustné řešení? je takové řešení, které je úplné (každá proměnná má přiřazenou hodnotu) a konzistentní (každá podmínka je splněna).
\end{definice}

\begin{definice}[Hranová konzistence]
	Hrana (dvojice vrcholů) je konzistentní, pokud existují hodnoty z oborů hodnot proměnných tak, že když je dosadíme, tak jsou splněny všechny podmínky mezi těmito proměnnými.

	Problém splňování je hranově konzistentní, pokud každá hrana je konzistentní.
\end{definice}

\begin{definice}[k-konzistence]
	Podobně jako hranová konzistence, jen pro více vrcholů zároveň (a splnění všech, klidně binárních, podmínek mezi nimi).
\end{definice}

\begin{definice}[Fail-first princip (pořadí proměnných)]
	První zkoušíme dosadit do té proměnné, která nejpravděpodobněji selže.

	Heuristiky na selhání jsou: dom heuristika = proměnná s nejmenším množstvím hodnot, deg heuristika = vybírá proměnnou, která je v nejvíce podmínkách.
\end{definice}

\begin{definice}[Succeed-first principe (pořadí hodnot)]
	Vybírá hodnotu, která má největší pravděpodobnost na úspěch.
\end{definice}

\begin{definice}[Programování s podmínkami]
	Programování s podmínkami je deklarativní přístup řešení (kombinatorických) problémů.
\end{definice}

\begin{veta}
	Pokud je problém $\forall i \in [n]$ $i$-konzistentní, pak ho lze vyřešit bez backtracku.
\end{veta}

% 07. 03. 2022

\begin{definice}[CNF (Conjunctive normal form)]
	Literál je proměnná nebo její negace, klauzule je disjunkce literálů a formule v CNF je konjunkce klauzulí.
\end{definice}

\begin{tvrzeni}[Z logiky]
	Každá sentence ve výrokové logice je logicky ekvivalentní formuli v CNF.
\end{tvrzeni}

\begin{definice}[DPLL]
	Jmenuje se podle autorů. Řeší SAT podobně jako jsme řešili CSP.

	Nejdřív se ohodnocují jen ty proměnné, které se vyskytují pouze jednou (ty ohodnotíme BÚNO). Pak se vyřeší klauzule o jedné proměnné (ty jsou nutně určené). Nakonec se vyzkouší něco doplnit s backtrackingem.

	V modernějších řešeních se ještě hledá, zda se SAT nedá rozdělit na disjunktní (co do proměnných) části. Pak se využívají heuristiky jako v CSP a dá se (stejně jako CSP) používat náhodný restart.
\end{definice}

\begin{definice}[Znalostní agent]
	Znalostní agent funguje tak, že odvozuje neznámé části světa podle známých.
\end{definice}

% 14. 03. 2022

\section{Dynamický SAT}
\begin{definice}[Fluent]
	V podstatě dvojice \verb|proměnná| – \verb|čas|. (Ohodnocení true znamená, že v \verb|čase| platí \verb|proměnná|.)
\end{definice}

\begin{definice}[Pozorovací model]
	Vezme pozorované prostředí a převede je do modelu (např. přidá souřadnice, nebo čas).
\end{definice}

\begin{definice}[Přechodový model]
	Popisuje, jak evolvuje svět.
\end{definice}

\begin{definice}[Frame problem]
	V přechodovém modelu se musí logickými formulemi zadat i proměnné, které se nemění. Pro jednoduchost se používá tzv. axiom následujícího stavu (pokud akce nemění proměnnou, tak se nemění zapíšeme jako proměnná je ekvivalentní sobě konjunkce akce měnící proměnnou).
\end{definice}

\begin{definice}[Precondition axioms]
	Podmínky pro provedení akce se dají zapsat jako implikace.
\end{definice}

\begin{definice}[Situační kalkulus]
	Výroková logika má nevýhodu, že nemá kvantifikátory, tedy musíme popis všeho většinou nagenerovat (například jedno pole je soused vedlejšího). S tím nám pomůže situační kalkulus, kde se používá logika 1. řádu.
\end{definice}

\begin{definice}[Klasické plánování]
	Využívá popis situačního kalkulu, ale řeší problém pomocí např. A$^*$.
\end{definice}

\begin{poznamka}
	Negace se lze zbavit tak, že se předpoklad duplikuje, a duplikát představuje negaci.
\end{poznamka}

\end{document}
