\documentclass[12pt]{article}					% Začátek dokumentu
\usepackage{../../MFFStyle}					    % Import stylu



\begin{document}

% 14. 02. 2022

\section*{Organizační úvod}
\begin{poznamka}[Organizační úvod]
	Bude Moodle. DL1. Kódem je kód cvičení.

	Prerekvizity nejsou formální, ale je vhodné mít za sebou ADS1 + logiku.

	Zkouška má povinnou písemnou část s materiály z přednášky a překladači Prologu a Haskellu. Zadání i odevzdání prostřednictvím Moodle UK. Zápočet ke zkoušce je doporučený.

	Zápočet: Zápočtový program v Prologu / Haskellu + cvičící si mohou volit další podmínky.
\end{poznamka}

\section{Úvod}
\begin{definice}[Neprocedurální programování]
	Programování bez přiřazovacího příkazu. Lze tak programovat dvěma způsoby: Logické programování = popisujeme problém, který chceme řešit prostředky matematické logiky (Prolog). Funkcionální programování = program je definice funkcí, výpočet je pak aplikace funkce na argumenty (Haskell, Lisp).
\end{definice}

\begin{definice}[Prolog]
	U zrodu stáli Robert Kowalski (Edinburgh) a Alain Colmerauer (Marseille).

	Aplikace: výuka a výzkum, zpracování přirozeného jazyka, AI, automatické dokazování vět, expertní systémy, dotazovací systémy, systémy řízení, …

	Existuje ohromné množství implementací. My budeme používat SWI Prolog.
\end{definice}

\begin{definice}[Syntax Prologu]
	Před tečkou je term = funktor použitý na konstantu (např. zena(eva). nebo rodic(eva, kain).). Tím (několika řádky začínající vždy stejným funktorem) jsme vytvořili proceduru, která definuje predikát (funktor/počet parametrů) tak, že dává true právě tehdy, když je použitý na tuto konstantu.

	Funktor použitý na proměnnou (standardně s velkým prvním písmenem) je definice proměnné.

	Čárka je konjunkce. Disjunkce se píše jako středník (využívá se také, když chceme vrátit další možné splnění formule). Konjunkce má vyšší prioritu.

	Definice pomocí predikátoru se dělá pomocí \verb|:-|.

	Nerovná se značí \verb|\==|.
\end{definice}

\end{document}
