\documentclass[12pt]{article}					% Začátek dokumentu
\usepackage{../../MFFStyle}					    % Import stylu



\begin{document}

% 14. 02. 2022

\section*{Organizační úvod}
\begin{poznamka}[Organizační úvod]
	Nahrávky budou. (Z minulého roku anglicky, z letoška česky.)
\end{poznamka}

\section{Úvod}
\begin{definice}[Strojově čitelný soubor]
	Strojově čitelný soubor je vlastnost konkrétního souboru, ne formátu (jelikož do formátu můžu nacpat data v jiném formátu).

	Strojová čitelnost se špatně definuje.
\end{definice}

\begin{definice}[Binární soubor]
	Binární soubor je takový, kde je struktura popsána na úrovni bitů (bit po bitu). Není čitelný textovými editory.
\end{definice}

TODO!!!

% 21. 02. 2022

\section{RDF}
\begin{definice}[RDF -- resource description framework]
	RDF je formát popisu grafu, kde se každé tvrzení (tedy trojice) má tvar „subjekt predikát objekt“, tj. „kdo co s-čím“. Vše se identifikuje pomocí IRI odkazující na definici (nebo v případě některých objektů (často Stringů/čísel/datumů) -- literálem).
\end{definice}

\begin{poznamka}
	Uri budeme často zkracovat (takové zkrácení se zapisuje jako např. \verb|@prefix dcterms: https://…|). Obecné zkratky lze najít na \url{prefix.cc}.
\end{poznamka}

\begin{definice}[Literál]
	Literál má dvě části -- text odpovídající formátu a uri na ?XML schéma toho typu. Nebo je tvaru \verb|"text"@jazyk|.
\end{definice}

\begin{priklady}
	Nejčastější predikát je \verb|rdf:type| -- „je typu“.
\end{priklady}

\begin{definice}[Blank node]
	Existují i nepojmenované uzly.
\end{definice}

\begin{definice}[RDF serializace]
	(Jak zapsat RDF do textu.)

	\begin{itemize}
		\item RDF 1.1 N-Triples = každá trojice se zapíše jako \verb|<uri> <uri> <uri> .   # comment|.
		\item RDF 1.1 Turtle = použijí se prefixy, středníky na shodný subjekt a čárku na shodný subjekt i predikát + se používají relativní IRI (base se definuje pomocí \verb|@base IRI|, implicitní je URL dokumentu) + multiline stringy a odescapované znaky + \verb|rdf: type| má zkratku \verb|a| + blank nody se píší pomocí hranatých závorek + běžné literály nemusí mít typ.
		\item RDF 1.1 N-Quads = místo trojice se kóduje i pojmenování grafu.
		\item RDF Trig = Turtle + pojmenované grafy (jsou reprezentovány jako bloky).
	\end{itemize}
\end{definice}

\begin{definice}[Reifikace]
	Pokud chci něco říct o naší trojici, můžu to udělat tak, že si definuji (zase pomocí trojic) objekt, který jako subjekt bude mít subjekt, atd. a navíc bude mít doplňující informace. Tato metoda se nazývá reifikace.
\end{definice}

\begin{definice}[Pojmenovaný graf, dataset]
	Vztahy lze seskupit do tzv. pojmenovaného grafu.

	Pojmenované grafy + defaultní graf se nazývá dataset.
\end{definice}

\begin{definice}[RDFS]
	Nadstavba RDF, které umožňuje definovat třídy a dědičnost. \verb|rdfs:Class|, \verb|rdfs:subClassOf|, \verb|rdf:Property|, \verb|rdfs:range|, \verb|rdfs:domain|, \verb|rdfs:subPropertyOf|.

	Oproti OOP není třeba definovat třídy, lze definovat property jako takové.

	Také umožňuje label, comment, seeAlso: \verb|rdfs:label|, \verb|rdfs:comment|, \verb|rdfs:seeAlso|, \verb|rdfs:isDefinedBy?|.
\end{definice}

\begin{definice}[rdf:List a jiné kolekce]
	Ve specifikaci RDF je přímo definován spojový seznam (\verb|rdf:List| + anonymní prvky + \verb|rdf:nil|).

	\verb|rdf:_i|, kde $i$ je libovolné číslo jsou predikáty náležení do kolekce (\verb|rdf:TODO|).
\end{definice}

\begin{definice}[Open World Assumption (OWA)]
	Tvrzení může být pravdivé, i když to nevíme. (Tj. máme i odpověď nevím.)
\end{definice}

TODO!!!

\end{document}
