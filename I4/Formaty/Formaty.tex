\documentclass[12pt]{article}					% Začátek dokumentu
\usepackage{../../MFFStyle}					    % Import stylu



\begin{document}

% 14. 02. 2022

\section*{Organizační úvod}
\begin{poznamka}[Organizační úvod]
	Nahrávky budou. (Z minulého roku anglicky, z letoška česky.)
\end{poznamka}

\section{Úvod}
\begin{definice}[Strojově čitelný soubor]
	Strojově čitelný soubor je vlastnost konkrétního souboru, ne formátu (jelikož do formátu můžu nacpat data v jiném formátu).

	Strojová čitelnost se špatně definuje.
\end{definice}

\begin{definice}[Binární soubor]
	Binární soubor je takový, kde je struktura popsána na úrovni bitů (bit po bitu). Není čitelný textovými editory.
\end{definice}

TODO!!!

% 21. 02. 2022

\section{RDF}
\begin{definice}[RDF -- resource description framework]
	RDF je formát popisu grafu, kde se každé tvrzení (tedy trojice) má tvar „subjekt predikát objekt“, tj. „kdo co s-čím“. Vše se identifikuje pomocí IRI odkazující na definici (nebo v případě některých objektů (často Stringů/čísel/datumů) -- literálem).
\end{definice}

\begin{poznamka}
	Uri budeme často zkracovat (takové zkrácení se zapisuje jako např. \verb|@prefix dcterms: https://…|). Obecné zkratky lze najít na \url{prefix.cc}.
\end{poznamka}

\begin{definice}[Literál]
	Literál má dvě části -- text odpovídající formátu a uri na ?XML schéma toho typu. Nebo je tvaru \verb|"text"@jazyk|.
\end{definice}

\begin{priklady}
	Nejčastější predikát je \verb|rdf:type| -- „je typu“.
\end{priklady}

\begin{definice}[Blank node]
	Existují i nepojmenované uzly.
\end{definice}

\begin{definice}[RDF serializace]
	(Jak zapsat RDF do textu.)

	\begin{itemize}
		\item RDF 1.1 N-Triples = každá trojice se zapíše jako \verb|<uri> <uri> <uri> .   # comment|.
		\item RDF 1.1 Turtle = použijí se prefixy, středníky na shodný subjekt a čárku na shodný subjekt i predikát + se používají relativní IRI (base se definuje pomocí \verb|@base IRI|, implicitní je URL dokumentu) + multiline stringy a odescapované znaky + \verb|rdf: type| má zkratku \verb|a| + blank nody se píší pomocí hranatých závorek + běžné literály nemusí mít typ.
		\item RDF 1.1 N-Quads = místo trojice se kóduje i pojmenování grafu.
		\item RDF Trig = Turtle + pojmenované grafy (jsou reprezentovány jako bloky).
	\end{itemize}
\end{definice}

\begin{definice}[Reifikace]
	Pokud chci něco říct o naší trojici, můžu to udělat tak, že si definuji (zase pomocí trojic) objekt, který jako subjekt bude mít subjekt, atd. a navíc bude mít doplňující informace. Tato metoda se nazývá reifikace.
\end{definice}

\begin{definice}[Pojmenovaný graf, dataset]
	Vztahy lze seskupit do tzv. pojmenovaného grafu.

	Pojmenované grafy + defaultní graf se nazývá dataset.
\end{definice}

\begin{definice}[RDFS]
	Nadstavba RDF, které umožňuje definovat třídy a dědičnost. \verb|rdfs:Class|, \verb|rdfs:subClassOf|, \verb|rdf:Property|, \verb|rdfs:range|, \verb|rdfs:domain|, \verb|rdfs:subPropertyOf|.

	Oproti OOP není třeba definovat třídy, lze definovat property jako takové.

	Také umožňuje label, comment, seeAlso: \verb|rdfs:label|, \verb|rdfs:comment|, \verb|rdfs:seeAlso|, \verb|rdfs:isDefinedBy?|.
\end{definice}

\begin{definice}[rdf:List a jiné kolekce]
	Ve specifikaci RDF je přímo definován spojový seznam (\verb|rdf:List| + anonymní prvky + \verb|rdf:nil|).

	\verb|rdf:_i|, kde $i$ je libovolné číslo jsou predikáty náležení do kolekce (\verb|rdf:TODO|).
\end{definice}

\begin{definice}[Open World Assumption (OWA)]
	Tvrzení může být pravdivé, i když to nevíme. (Tj. máme i odpověď nevím.)
\end{definice}

TODO!!!

% 28. 02. 2022

\begin{definice}[Otevřená data 5 hvězdičkova klasifikace dat]
	První hvězdička je za uvedenou licenci, druhá je za strojovou čitelnost, třetí je za otevřený formát, čtvrtá za URI odkazy, pátá za připojení do systému LOD.
\end{definice}

\section{SPARQL}
\begin{definice}[SPARQL]
	SPARQL je dotazovací jazyk nad daty v RDF. SPARQL endpoint je HTTP služba pro dotazování v SPARQL na daných open datech.

	\begin{poznamkain}
		Doporučovaný user formulář je yasgui.
	\end{poznamkain}

	Funguje tak, že se píší RDF trojice s \verb|?nazevproměnné| v místě, kde chceme něco doplnit (a zjistit, co to je). To jsou tzv. datové vzory.

	Výsledkem je pak tabulka řešení, kde je v každém řádku jeden match a v každém sloupci jedna proměnná, v políčkách je tam pak doplněno.

	Do dotazu lze připsat \verb|OPTIONAL| a výsledek pak bude matchovat, i když tato část bude chybět a v tabulce pak bude \verb|NOT BOUND|. Také lze přidat \verb|FILTER| pro podmínky s proměnnými.

	Oproti SQL máme ještě RDF operátory: \verb|bound, isIri, isBlank, isLiteral| a přístup k literálu: \verb|str, language, typeOf|?

	Taktéž fungují \verb|/| jako v cestě k souboru, která se navíc zadává Regexem.

	Jedním dotazem se můžeme ptát na více SPARQL endpointů, což uděláme pomocí příkazu \verb|SERVICE|.
\end{definice}

TODO!!!

% 07. 03. 2022

\section{Nejčastější slovníky RDF}
\begin{definice}[Dublin Core metadata]
	Jeden z prvních slovníků, vznikl na popis knih (a dalších děl). Jsou to pojmy se zkratkou dcterms (Dublin Core Metadata Initiative).
\end{definice}

\begin{definice}[skos]
	Konceptuální slovník. Důležité jsou např. \verb|skos:prefLabel, skos:altLabel, skos:hiddenLabel|. Dále třeba \verb|notation| a různé typy \verb|skos:semanticRelation|.
\end{definice}

\begin{definice}[GoodRelations]
	Slovník pro e-komerci.
\end{definice}

\begin{definice}[Schema.org]
	Založen firmami Google, Microsofte, Yahoo a Yandex. Integruje existující slovníky. Určeno pro jednoduchou anotaci webových stránek, ne k dobré strukturalizaci.
\end{definice}

\begin{definice}[Wikidata]
	Komunitní RDF data. Má k sobě také slovník. Běží na softwaru Wikibase.
\end{definice}

TODO!!!

% 14. 03. 2022 Přednáška nebyla, je záznam z minulého roku.

% 21. 03. 2022

\section{Hierarchické datové formáty}
\begin{definice}[Dokumentově orientované XML]
	Dokument, do kterého se vloží značky (tj. bez značek je stále čitelný).
\end{definice}

\begin{definice}[Datově orientované XML]
	To jsou pouze data se značkami (tj. bez značek je „nečitelný“).
\end{definice}

\begin{definice}[XML 1.0 a XML 1.1]
	1.0 má list povolených znaků, 1.1 zakázaných. Aplikace ale zamrzly u 1.0.
\end{definice}

TODO syntaxe XML

% 28. 03. 2022

TODO!!!

% 04. 04. 2022

TODO (Nebyl jsem)

% 11. 04. 2022

\section{Relační datový model}
\begin{definice}[Relační datový model]
	V relačním modelu máme tabulku, která má řádky (záznamy) a sloupce (klíče). Pak máme primární klíč, jehož hodnoty určují jednoznačně každý záznam. A foreign klíč, tj. že se odkazujeme na cizí tabulku.
\end{definice}

\begin{definice}[SQL dump]
	Vytvoří SQL příkaz, který vytvoří přesnou kopii dat.
\end{definice}

\begin{poznamka}[Before CSV]
	Delimiter-Separated Values (DSV): (delmiter = označerní kusu dat vs. separator = odděluje kusy dat (neoznačuje začátek + konec)), skoro jako CSV, jen jiné separátory a jiné kódování.

	Tab-Separated Values (TSV): už má specifikaci, odděluje tabulátory.
\end{poznamka}

\begin{definice}[Comma-Separated Values (CSV)]
	Kódování defaultně UTF-8 (od 2014, předtím US-ASCII, jsou možná i jiná), oddělovačem je čárka, escape znakem je uvozovka (\verb|""| je odescapovaná uvozovka, escapuje tak, že se věc uzavře do uvozovek), 
\end{definice}

\begin{poznamka}
	Dále jsme si povídali o správných a špatných příkladech.
\end{poznamka}

\begin{definice}[URI Fragment Identifiers pro csv]
	Když máme uri csv, můžeme na konec přidat \verb|#col=rozsah|, \verb|#row=rozsah| nebo \verb|#cell=rozsah| rozsah může mít \verb|-| a \verb|*|.
\end{definice}

\begin{poznamka}
	Dále jsme se bavili o tom, jak popisovat schéma csv.
\end{poznamka}

TODO!!!

% 25. 04. 2022 (Host)

\section{Prostorové informace}
\begin{poznamka}[Otázky, pro které potřebujeme prostorové informace]
	Jak je to daleko? Kterým směrem to je? Co je nejbližší? Co největší, nejvyšší, atd.
\end{poznamka}

\begin{definice}[Prostorové informace]
	Zabývá se jimi hlavně norma ISO/TC 211. Obor se nazývá Geoinformatika, Geomatika apod.

	Geografická data = Geoprostorová data = Prostorová data -- data a informace, která mají implicitní/explicitní lokaci oproti zemi.

	K souřadnicím se většinou používá tzv. referenční elipsoid, který máme „umístění“ kolem země a projektujeme na něj. Má chybu cca půl metru? Ale existuje i mnoho dalších možností.

	Implicitní = souřadnice, vzdálenosti směry; Explicitní = pojmenování, adresy, geografická jména.
\end{definice}

\begin{definice}[Typy]
	Points, Multipoints, Lines (Linestrings), Multilines, Polygons, Multipolygons, Surface.

	Dále lze používat i křivky jako kružnice, ale spíš se to nedělá (kružnice se rozseká a udělá se z ní Polygon).
\end{definice}

\begin{definice}[Points, Multipoints]
	Points -- určují „bodový“ objekt.
	Mutipoints jsou například zastávky -- jednomu objektu odpovídá více „diskrétních“ bodů.
\end{definice}

\begin{definice}[Lines, Multilines, Polygons, Multipolygons]
	Jako Points, jen s liniovými/mnohoúhelníkovými objekty.
\end{definice}

\begin{definice}[Well-Known Text (WKT)]
	OGC standart (nebo v placeném ISU 19125?), téměř všechny knihovny předpokládají WGS-84 (referenční elipsoid, souřadnice ve stupních). Je tvaru \verb|TYP(souradnice1x souradnice1y souradnice2x souradnice2y …)|
\end{definice}

\begin{definice}[Geometry Markup Language (GLM)]
	XML, které kromě typů obsahuje i odkaz na souřadnicový systém, dimenzi, atd.
\end{definice}

\begin{definice}[Další formáty]
	GeoJSON, Shapefile, GeoPackage, CSV, GeoSPARQL, …
\end{definice}

\begin{definice}[Feature]
	(Časky geografický popis objektu?, nepoužívá se.)

	Objekt, který může mít nějaký geografický popis (např. dům). Featura má atributy (které mluví o negeografických vlastnostech) a geometrii (geografická data).
\end{definice}

\begin{poznamka}
	Geografické objekty lze kreslit třeba v geojsonu… (Lze tam i převádět z geoJSONu.)
\end{poznamka}

\begin{definice}[Prostorové relace]
	Topologické: Within, Touches, Crosses, Overlaps.

	Směrové: Left, Right.

	Vzdálenostní: Closer Further.

	A ještě všechno může být určené v čase.
\end{definice}

\begin{definice}[Prostorové operace]
	Buffer (nafoukne objekt na polygon všech bodů, které jsou blíže než nějaká určená vzdálenost od nějakého bodu objektu). Union, Difference, Intersection, Clip (odříznutí), Distance, Convex Hull.
\end{definice}

\begin{definice}[Geografické informační softwary a knihovny pro zpracování prostorových informací]
	QGIS (zdarma), PostGIS (rozšíření pro PostgreSQL), ESRI ArcGIS (velký komerční projekt).
\end{definice}

TODO?

\end{document}
