\documentclass[12pt]{article}					% Začátek dokumentu
\usepackage{../../MFFStyle}					    % Import stylu



\begin{document}

% 5. 3. 2021
\section{Úvod}
    \begin{definice}[Metrika, metrický prostor]
        $M$ množina, $d: M \times M \rightarrow [0, ∞)$ je metrika, pokud $\forall x, y, z \in M$ platí:
        $$ d(x, y) = 0 \Leftrightarrow x = y, $$ 
        $$ d(y, x) = d(x, y), $$
        $$ d(x, y) ≤ d(x, z) + d(z, y). $$ 

        Dvojice $(M, d)$ se pak nazývá metrický prostor.
    \end{definice}

    \begin{definice}[Norma a normovaný lineární prostor (NLP)]
        Ať ¦V je vektorový prostor nad $®F \in \{®R, ®C\}$, pak $||·||: ¦V \rightarrow [0, ∞)$ je norma, pokud $\forall ¦x, ¦y \in ¦V$
        $$ ||¦x|| = 0 \Leftrightarrow ¦x = ¦o, $$
        $$ \forall \lambda \in ®F: ||\lambda ¦x|| = |\lambda|·||¦x||, $$
        $$ ||¦x + ¦y|| ≤ ||¦x|| + ||¦y||. $$ 


        Dvojice $(¦V, ||·||)$ se pak nazývá normovaný lineární prostor.
    \end{definice}

    \begin{definice}[Otevřená a uzavřená koule]
        Ať $(®M, d)$ je MP, $x \in ®M$, $r > 0$. Pak otevřená koule o středu $x$ a poloměru $r$ je množina $\B(x, r):=\{y \in ®M | d(x, y) < r\}$. Uzavřená koule o středu $x$ a poloměru $r$ je množina $\overline{\B}(x, r):= \{y \in ®M | d(x, y) ≤ r\}$.
    \end{definice}

    \begin{veta}
        $(®R^d, ||·||_p)$ je NLP pro $d \in ®N, p \in [1, ∞]$.

        \begin{dukazin}
                1. krok: $B = \{ x \in ®R^d; ||x||_p ≤ 1\}$ je konvexní množina (tj. $\forall \lambda \in (0, 1)\ \forall x, y \in B: \lambda x + (1 - \lambda)y \in B$). Pro $p=∞$: 
                $$ \forall i \in [d]: |\lambda x_i + (1-\lambda) y_i| ≤ \lambda|x_i| + (1 - \lambda)|y_i| ≤ \lambda · 1 + (1-\lambda)·1 = 1 $$
                Pro $p < ∞$:
                $$ \forall i \in [d]: |\lambda x_i + (1-\lambda) y_i|^p ≤ \lambda|x_i|^p + (1 - \lambda)|y_i|^p, $$
                protože $t \mapsto t^p$ je konvexní funkce. Dopočítáním obou nerovností získáme, že je to opravdu konvexní množina.

                2. krok: Pokud $||·||$ splňuje $(i) + (ii)$ a $B$ je konvexní, pak $||·||$ je norma. Zvolme $¦x, ¦y \in ¦V$, BÚNO $¦x, ¦y ≠ ¦o$, položme $\tilde{¦x} := \frac{¦x}{||¦x||}$, $\tilde{¦y} := \frac{¦y}{||¦y||}$, tedy:
                $$ \frac{¦x + ¦y}{||¦x|| + ||¦y||} = \frac{||¦x||}{||¦x|| + ||¦y||} \tilde{¦x} + \frac{||¦y||}{||¦x|| + ||¦y||}\tilde{¦y} \in B \text{ (zlomky jsou } \lambda, 1-\lambda). $$ 
                $$ ||\frac{¦x + ¦y}{||¦x|| + ||¦y||}|| ≤ 1 \implies \frac{||¦x + ¦y||}{||¦x|| + ||¦y||} ≤ 1. $$ 

                3. $||·||_p$ zřejmě splní $(i) + (ii)$ a $B$ je konvexní podle 1. kroku. Tedy $||·||_p$ je norma.
        \end{dukazin}
    \end{veta}

    \begin{poznamka}[Značení]
        $$ l_p^d := \(®R^d, ||·||_p\). $$ 
    \end{poznamka}

    \begin{definice}[Konvergence]
        Ať $(®M, d)$ je MP, $\{x_n\}_{n = 1}^∞$ posloupnost v ®M, $x \in ®M$. Pak $(x_n)$ konverguje k $x$ pokud $d(x_m, x)$ konverguje k 0. Píšeme $x_n \rightarrow x$ nebo také $\lim_{n \rightarrow ∞} x_n = x$.
    \end{definice}

% 14. 3. 2021

\section{Otevřené a uzavřené množiny}
    \begin{definice}[Vnitřek, vnějšek, hranice]
        Ať $(®M, d)$ je MP. $A \subseteq ®M$. Pak $x_0 \in ®M$ je vnitřní bod $A$ $≡$ $\exists r > 0: \B(x_0, r) \subseteq A$. Dále vnitřek (interior) množiny $A$ je množina
        $$ \Int(A) = \{x_0 \in ®M | x_0 \text{ je vnitřní bod } A\}. $$ 

        Dále $x_0 \in ®M$ je vnější bod $A$ $≡$ $\exists r>0: \B(x_0, r) \subseteq ®M \setminus A$. Vnějšek (exterior) množiny $A$ je množina
        $$ \Ext(A) = \{x_0 \in ®M | x_0 \text{ je vnější bod } A\}. $$

        Nakonec $x_0 \in ®M$ je hraniční bod $A$ $≡$ $x \in ®M \setminus (\Int(A) \cup \Ext(A))$. Hranice množiny $A$ je množina
        $$ \partial A = \{x_0 \in ®M | x_0 \text{ je hraniční bod } A\}. $$ 
    \end{definice}

    \begin{pozorovani}
        Zřejmě $\Int(A) \subseteq A$.

        Zřejmě $\Ext(A) = \Int(®M \setminus A) \subseteq ®M \setminus A$.
    \end{pozorovani}

    \begin{definice}[Otevřená a uzavřená množina]
        Buď $(®M, d)$ MP a $A \subseteq ®M$. Pak $A$ je otevřená $≡$ $A \cap \partial A = \O$.

        Dále uzávěr množiny $A$ je množina $\overline{A} = A \cup \partial A$. Množina $A$ je poté uzavřená $≡$ $\partial A \subseteq A$.
    \end{definice}

    \begin{pozorovani}
        Zřejmě $A$ je otevřená $\Leftrightarrow$ $A = \Int(A)$.

        Otevřená koule je otevřená množina.
    \end{pozorovani}

    \begin{lemma}
        Ať $(®M, d)$ je MP, $A \subseteq ®M$. Pak $x \in \overline{A} \Leftrightarrow \exists (x_n) \subseteq ®N\times A: x_n \rightarrow x$. Zároveň následující podmínky jsou ekvivalentní:
        $$ a)\ A \text{ je uzavřená,} \qquad b)\ A = \overline{A}, \qquad \forall(x_n \in A): x_n \rightarrow x \in ®M \implies x \in A. $$ 

        \begin{dukazin}
            $\implies$: Ať $x \in \overline{A}$. Pokud $x \in A$, polož $x_n = x$. Pokud $x \notin A$, pak $x \in \partial A$, tedy $\forall n\ \exists x_n \in \B(x, \frac{1}{n}) \cap A$. Pak $x_n \rightarrow x$ ($0≤d(x_n, x) < \frac{1}{n} \rightarrow 0$).

            $\Leftarrow$ Ať $(x_n)$ je posloupnost v $A$, $x_n \rightarrow x$. Pokud $x \in A$, jsme hotovi. Pokud $x \notin A$, pak $\forall \epsilon > 0\ \exists r_0 \forall n ≥ n_0 : x_n \in \B(x, \epsilon) \cap A$. Tedy $x \in \overline{A}$.

            $a) \Leftrightarrow b)$ $A$ je uzavřená $\overset{\text{def}}{\Leftrightarrow} \partial A \subseteq A$ $\Leftrightarrow A = A \cup \partial A = \overline{A}$.

            $b) \implies c) \implies a)$ $A = \overline{A} \implies \forall(x_n): x_n \rightarrow x \implies x \in A$ $\overline{\text{První část}}{\implies} \partial A \subseteq A$.
        \end{dukazin}
    \end{lemma}

    \begin{veta}[Základní vlastnosti otevřených množin]
        Ať $(®M, d)$ je MP. Pak

        \begin{itemize}
            \item[(i)] ®M a $\O$ jsou otevřené.
            \item[(ii)] Sjednocení libovolně mnoha otevřených je otevřené.
            \item[(iii)] Průnik konečně mnoha otevřených je otevřený.
        \end{itemize}

        \begin{dukazin}
            (i) Triviální. (ii) $x \in \bigcup_i M_i$, pak $\exists j: x \in M_j$. Potom $M_j$ je otevřená, tedy existuje $r > 0: \B(x, r) \subseteq M_j \subseteq \bigcup_i M_i$. Tedy $\bigcup_i M_i$ je otevřená. (iii) $x \in \bigcap_i M_i$, pak $\forall i\ \exists r_i: \B(x, r_i) \subseteq M_i$. Polož $r = \min_i r_i > 0$ (protože $i$ je z konečné množiny, tedy existuje minimum a to je jistě jeden z těch poloměrů, tedy $> 0$), pak $\B(x, r) \subseteq \bigcap_i M_i$. Tedy $\bigcap_i M_i$ je otevřená.
        \end{dukazin}
    \end{veta}

    \begin{veta}[Vztah otevřená a uzavřené množiny]
        Ať $(®M, d)$ je MP, $A \subseteq M$. Pak $A$ je otevřená $\Leftrightarrow$ $®M \setminus A$ je uzavřená.

        \begin{dukazin}
            $\implies$: Zvol $(x_n)$ posloupnost v $®M \setminus A$, $x_n \rightarrow x$. Sporem. Nechť $x \in A$. Potom $\exists \epsilon > 0: \B(x, \epsilon) \subseteq A$, ale pak $\exists n: x_n \in A$. \lightning.

            $\Leftarrow$: Zvol $x \in A$. Protože $®M \setminus A$ je uzavřená, tedy $\partial(®M \setminus A) \subseteq ®M \setminus A)$, $x \notin \partial(®M \setminus A)$, tedy $\exists \epsilon > 0: \B(x, \epsilon) \cap A = \O$ (to nelze) nebo $\B(x, \epsilon)\cap(®M\setminus A) = \O$. Tedy $\exists \epsilon > 0: \B(x, \epsilon) \cap (®M \setminus A) = \O$, tj. $\B(x, \epsilon) \subseteq A$, tedy $A$ je otevřená.
        \end{dukazin}
    \end{veta}

    \begin{veta}[Základní vlastnosti uzavřených množin]
        Ať $(®M, d)$ je MP, $A \subseteq ®M$. Pak
        
        \begin{itemize}
            \item[(i)] ®M a $\O$ jsou uzavřené.
            \item[(ii)] Průnik libovolně mnoha uzavřených množin je uzavřený.
            \item[(iii)] Sjednocení konečně mnoha uzavřených množin je uzavřené.
        \end{itemize}
        
        \begin{dukazin}
            Plyne z věty výše a de-Morganových pravidel.
        \end{dukazin}
    \end{veta}

    \begin{veta}
        Ať $(®M, d)$ je MP, $A \subseteq ®M$. Pak 
        $$ \Int(A) = \bigcup\{G \subseteq A | G \text{ otevřené}\}, $$
        $$ \overline{A} = \bigcap\{F \supseteq A | F \text{ uzavřené}\}. $$

        \begin{dukazin}
            $\subseteq$: $x \in \Int(A) \implies \exists \epsilon > 0: \B(x, \epsilon) \subseteq A$, stačí položit $G = \B(x, \epsilon)$.

            $\supseteq$: Ať $G \subseteq A$ otevřená, pak $G = \Int(G) \subseteq \Int(A)$.

            $\subseteq$: $x \in \overline{A}$, pak $\exists (x_n)$ v $A$: $x_n \rightarrow x$. Zvol $F \supseteq A$ uzavřená, pak $x_n \rightarrow x \in F$ (z uzavřené se nedá vykonvergovat).

            $\supseteq$: Položme $F = \overline{A} \supseteq A$.
        \end{dukazin}
    \end{veta}

% 19. 3. 2021

\section{Spojitost v metrických prostorech}
    \begin{definice}[Spojitost v bodě, spojitost, ($k$-)Lipschitzovskost]
        Ať $(®M, d), (®N, e)$ jsou MP, $f: ®M \rightarrow ®N$, $a \in ®M$. Potom $f$ je spojitá v $a$ $≡ \forall \epsilon > 0\ \exists \delta > 0\ \forall x \in ®M:$ $d(x, a) < \delta \implies e(f(x), f(a)) < \epsilon$.

        $f$ je spojitá na ®M $≡ \forall a \in ®M:$ $f$ je spojitá v $a$.

        $f$ je $k$-Lipschitzovská ($k > 0$) $≡ \forall x, y \in ®M: e(f(x), f(y)) ≤ k·d(x, y)$.

        $f$ je Lipschitzovská $≡$ $\exists k>0:$ $f$ je $k$-Lipschitzovská.
    \end{definice}

    \begin{pozorovani}
        $f$ je $k$-Lipschitzovská $\implies$ $f$ je spojitá.
    \end{pozorovani}

    \begin{definice}[Značení]
        Ať $(®M, d)$ je MP, $A \subseteq ®M$, $x \in ®M$. Pak $\dist(x, A):=\inf_{a \in A} d(x, a)$.
    \end{definice}

    \begin{lemma}
        Ať $(®M, d)$ je MP, $A \subseteq ®M$. Pak
        $$ (i) \forall x \in ®M: d(x, A) = d(x, \overline{A}), $$ 
        $$ (ii) \forall x \in ®M: d(x, A) = 0 \Leftrightarrow x \in \overline{A}, $$ 
        $$ (iii) \dist(·, A): ®M \rightarrow ®R \text{ je 1-Lipschitzovská}. $$ 

        \begin{dukazin}
            $(i)$ $≥$: Jasné (infimum přes menší množinu). $≤$: Pro $n \in ®N$ zvolme $y_n \in \overline{A}$: $d(x, y_n) < \dist(x, \overline{A}) + \frac{1}{n}$. Zvolme dále $x_n \in \B\(y_n, \frac{1}{n}\) \cap A$, pak $\dist(x, A) ≤ d(x, x_n) ≤ d(x, y_n) + d(y_n, x_n) < \dist(x, \overline{A}) + \frac{1}{n}$, celkem $\forall n \in ®N: \dist(x, A) < \dist(x, \overline{A}) + \frac{2}{n}$ $\implies \dist(x, A) ≤ \dist(x, \overline{A})$.

            $(ii)$: BÚNO $A$ je uzavřená (jinak podle $(i)$). $\Rightarrow$: Jasné (do $\inf$ dosadíme $x$). $\Rightarrow$ $\forall n \ \exists x_n \in \B(x, \frac{1}{n}) \cap A$ protože $d(x, A) = 0$. Pak ale $x_n \rightarrow x$, tedy $x \in A$ z uzavřenosti.

            $(iii)$: Zvolme $x, y \in ®M$. BÚNO $d(x, A) ≥ d(y, A)$. Fixujeme $n \in ®N$. Zvolme $y_n \in A: d(y, y_n) < \dist(y, A) + \frac{1}{n}$. Pak
            $$ |d(x, A) - d(y, A)| = d(x, A) - d(y, A) < d(x, y_n) - \(d(y, y_n) - \frac{1}{n}\) \overset{\triangle}{≤} \frac{1}{n} + d(x, y). $$

            $\implies$ ($n$ bylo libovolné, přejdeme k limitě) $|d(x, A) - d(y, A)| ≤ 1·d(x, y)$.
        \end{dukazin}
    \end{lemma}

    \begin{lemma}
        Ať $(®M, d)$ je MP. Pak
        $$ (i) \forall x ≠ y \in ®M\ \exists f: ®M \rightarrow ®R \text{ 1-Lipschitzovská, že } f(x) ≠ f(y), $$
        $$ (ii) \text{ Projekce } \pi_i: (®R^d, ||·||_p) \rightarrow ®R, (x_1, …, x_d) \mapsto x_i \text{ jsou Lipschitzovské}, d \in ®N, p \in [1, ∞]. $$

        \begin{dukazin}
            $(i)$ Zvol $f:=d(·, \{x\})$.

            $(ii)$ $\forall \vec{x}, \vec{y} \in ®R^d: |\pi_i(x_1, …, x_d) - \pi_i(y_1, …, y_d)| = |x_i - y_i|$
            $$ ≤ \begin{cases} p = ∞: & ||\vec{x} - \vec{y}||_∞ \\ p≠∞: & \sqrt[p]{\sum_{j=1}^d |x_j - y_j|^p} \end{cases}. $$ 
        \end{dukazin}
    \end{lemma}

    \begin{tvrzeni}
        Ať $(®M, d)$, $(®N, e)$ jsou MP, $f: ®M \rightarrow ®N$. Pak následující tvrzení jsou ekvivalentní:
        
        \begin{itemize}
            \item[(i)] $f$ je spojitá,
            \item[(ii)] $f^{-1}(U)$ je otevřená, kdykoliv $U \subseteq ®N$ je otevřená,
            \item[(iii)] $f^{-1}(F)$ je uzavřená, kdykoliv $F \subseteq ®N$ je uzavřená.
        \end{itemize}

        \begin{dukazin}
            (ii) $\Leftrightarrow$ (iii): Z věty o doplňcích a toho, že $f^{-1}(®N \setminus U) = ®M \setminus f^{-1}(U)$.

            (i) $\implies$ (ii): Nechť $U \subseteq ®N$ otevřená, $x \in f^{-1}(U)$. Pak $f(x) \in U \implies \exists \epsilon > 0: \B(f(x), \epsilon) \subseteq U$. $\implies$ ($f$ spojitá) $\exists \delta > 0: y \in \B(x, \delta) \implies f(y) \in \B(f(x), \epsilon) \subseteq U$, pak $\B(x, \delta) \subseteq f^{-1}(U)$.

            (ii) $\implies$ (i): Nechť $x \in ®M, \epsilon > 0$. Pak $f^{-1}(\B(f(x), \epsilon))$ je otevřená dle (ii). $\implies \exists \delta > 0: \B(x, \delta) \subseteq f^{-1}(\B(f(x), \epsilon))$. Tedy $d(x, y) < \delta \implies f(y) \in \B(f(x), \epsilon)$.
        \end{dukazin}
    \end{tvrzeni}

    \begin{definice}[Stejnoměrná spojitost]
        Ať $(®M, d)$ a $(®N, e)$ jsou MP, $f: ®M \rightarrow ®N$. Pak $f$ je stejnoměrně spojitá, pokud
        $$ \forall \epsilon > 0\ \exists \delta > 0\ \forall x, y \in ®M: d(x, y) < \delta \implies e(f(x), f(y)) < \epsilon. $$
    \end{definice}

    \begin{dusledek}
        $f$ je stejnoměrně spojitá $\implies$ $f$ je spojitá. (Ale naopak to neplatí.)

        $f$ je Lipschitzovská $\implies$ $f$ je stejnoměrně spojitá. (Stejně tak tohle naopak neplatí.)
    \end{dusledek}

    \begin{definice}[Izometrie]
        Ať $(®M, d)$ a $(®N, e)$ jsou MP, $f: ®M \rightarrow ®N$. Pak $f$ je izometrie, pokud $\forall x, y \in ®M: d(x, y) = e(f(x), f(y))$.
    \end{definice}

    \begin{dusledek}
        Izometrie je 1-Lipschitzovská. (Ale ne naopak.)
    \end{dusledek}

    \begin{definice}[Homeomorfismus]
        Ať $(®M, d)$ a $(®N, e)$ jsou MP, $f: ®M \rightarrow ®N$. Pak $f$ je homeomorfismus, pokud $f$ je spojitá bijekce a $f^{-1}$ je spojitá.
    \end{definice}

    \begin{dusledek}
        Izometrie na je homeomorfismus. (Ale opačně to neplatí.)
    \end{dusledek}

    \begin{lemma}
        $I$ interval, $f: I \rightarrow ®R$, že $|f'(x)| ≤ C, \forall x \in \Int(I)$ $\implies f$ je C-Lipschitzovská.

        \begin{dukazin}
            Ať $a < b \in I$ $\implies$ (Lagrange) $\exists \zeta \in (a, b): |\frac{f(b) - f(a)}{b-a}| = |f'(\zeta)| ≤ C$, tj. $|f(b) - f(a)| ≤ C|b - a|$.
        \end{dukazin}
    \end{lemma}

% 26. 3. 2021

    \begin{definice}[Topologicky ekvivalentní]
        Řekneme, že $\sigma$ a $\sigma_1$ jsou topologicky ekvivalentní, pokud
        $$ \{A \subseteq ®Y | A \text{ je otevřená v } (®Y, \sigma) \} = \{A \subseteq ®Y | A \text{ je otevřená v } (®Y, \sigma_1)\}. $$
    \end{definice}

    \begin{tvrzeni}
        Buďte $(®X, \rho)$, $(®Y, \sigma)$ MP, $f: (®X, \rho) \rightarrow (®Y, \sigma)$ homeomorfismus. Definujeme pro všechna $y, y' \in ®Y$ zobrazení $\sigma_1: ®Y \times ®Y \rightarrow [0, ∞)$ předpisem
        $$ \sigma_1(y, y') = \rho(f^{-1}(y), f^{-1}(y')). $$ 
        Pak $\sigma_1$ je metrika na ®Y, $f:(®X, \rho) \rightarrow (®Y, \sigma_1)$ je izometrie a metriky $\sigma$ a $\sigma_1$ jsou topologicky ekvivalentní.

        \begin{dukazin}
            Metrika: Banální, cvičení pro nás. Izometrie: Nechť $x, x' \in ®X$ jsou libovolné body. 
            $$ \sigma_1(f(x), f(x')) = \rho\(f^{-1}(f(x)), f^{-1}(f(x'))\) = \rho(x, x'), $$
            a tedy $f$ je izometrie.

            Topologická ekvivalence: Nechť $U \subseteq ®Y$ je otevřená vzhledem k $\sigma$. Pak $f^{-1}(U)$ je otevřená ($f$ je homeomorfismus), ale $f$ je izometrie, tedy $f^{-1}$ je izometrie, tudíž $f^{-1}$ je spojitá. Tj.
            $$ U = f\(f^{-1}(U)\) = \(f^{-1}\)^{-1}\(f^{-1}(U)\) \text{ je otevřená.} $$
            Podobně pokud $U$ je $\sigma_1$-otevřená, je $\sigma$-otevřená.
        \end{dukazin}
    \end{tvrzeni}

    \begin{veta}
        Buďte $\rho_1$, $\rho_2$ metriky na ®X. Pak $\rho_1$ a $\rho_2$ jsou topologicky ekvivalentní $\Leftrightarrow$
        $$ (\forall x \in ®X\ \forall \epsilon > 0\ \exists \delta > 0\ \forall y \in ®X: \rho_1(x, y) < \delta \implies \rho_2(x, y) < \epsilon) \land $$
        $$ \land (\forall x \in ®X\ \forall \epsilon > 0\ \exists \delta > 0\ \forall y \in ®X: \rho_2(x, y) < \delta \implies \rho_1(x, y) < \epsilon). $$

        \begin{dukazin}
            Snadné cvičení.
        \end{dukazin}
    \end{veta}

    \begin{definice}[Diametr, omezená množina]
        Buď $(®X, \rho)$ MP, $A \subseteq ®X$. Definujeme $\diam_\rho(A) = \sup\{\rho(x, y): x, y \in A\}$.

        Řekneme, že $A$ je omezená, pokud $\diam_\rho(A) < ∞$.
    \end{definice}

    \begin{definice}[Omezená metrika]
        $\rho$ je na ®X omezená, pokud je množina $®X$ omezená.
    \end{definice}

\section{Operace s metrickými prostory}
    \begin{definice}[Operace]
        Je-li $(®X, \rho)$ MP, $®Y \subseteq ®X$, pak metrický prostor $(®Y, \rho|_{®Y \times ®Y})$ nazýváme podprostorem prostoru $(®X, \rho)$, značíme $(®Y, \rho)$.

        \begin{dukazin}
            Že $(®Y, \rho)$ je MP je zřejmé.
        \end{dukazin}
    \end{definice}

    \begin{tvrzeni}
        Buď $(®X, \rho)$ MP, $®Y \subseteq ®X$. Pak:

        1) Pokud $G \subseteq ®X$ je otevřená v $(®X, \rho)$, pak $G'=G \cap ®Y$ je otevřená v $(®Y, \rho)$.
        
        2) Pokud $G' \subseteq ®Y$ je otevřená v $(®Y, \rho)$, pak $\exists G \subseteq ®X$ otevřená v $(®X, \rho): G' = G \cap ®Y$.

        \begin{dukazin}
            1) Nechť $y \in G'$. Protože $G$ je otevřená v ®X, tak $\exists r > 0: \B_{®X, \rho}(y, r) \subseteq G$. Tedy $\B_{®Y, \rho}(y, r) = \B_{®X, \rho}(y, r) \cap ®Y \subseteq G \cap ®Y = G'$.

            2) Nechť je dána $G'$ otevřená v $(®Y, \rho)$. Pak $\forall x \in G'\ \exists \epsilon(x) > 0: \B_{®Y, \rho}(x, \epsilon(x)) \subseteq G'$. Zřejmě $G' = \bigcup_{x \in G'}\B_{®Y, \rho}(x, \epsilon(x))$.

            Položme $G = \bigcup_{x \in G'}\B_{®X, \rho}(x, \epsilon(x))$. Potom je $G \cap ®Y = G'$. $G$ je otevřená, jelikož je sjednocením otevřených množin.
        \end{dukazin}
    \end{tvrzeni}

    \begin{definice}[Součet MP]
        Mějme MP $\{®X_\alpha, \rho_\alpha\}_{\alpha \in I}$, které splňují $\forall \alpha \in I\ \forall x, y \in ®X_\alpha: \rho_\alpha(x, y) ≤ 1$. Sumou prostorů $\(®X_\alpha, \rho_\alpha\)$ nazýváme prostor
        $$ \sum_{\alpha \in I} (®X_\alpha, \rho) = (®X, \rho), $$
        kde
        $$ ®X = \coprod_{\alpha \in I}®X_\alpha = \{(x, \alpha) | x \in ®X_\alpha, \alpha \in I\}, $$
        $$ \rho((x, \alpha), (y, \beta)) = 1, \text{ pokud } \alpha ≠ \beta, \rho_{\alpha}(x, y), \text{ pokud } \alpha = \beta. $$
    \end{definice}

% 9. 4. 2021

    \begin{definice}[Součin (spočetně mnoha) MP]
        Buďte $\{®X_i, \rho_i\}_{i \in ®N}$ MP, že $\forall i \in ®N: \diam(®X_i) ≤ 1$. Součinem prostorů $\(®X_i, \rho_i\)$ nazýváme metrický prostor (je nutno, ale jednoduché dokázat)
        $$ \prod_{i \in ®N} := (®X, \rho), \qquad ®X = \prod_{i \in ®N}®X_i \land \forall f, g \in ®X: \rho(f, g) = \sum_{i \in ®N} \frac{\rho_i(f(i), g(i))}{2^i}. $$
    \end{definice}

    \begin{tvrzeni}
        Buďte $(®X_i, \rho_i)$, $i \in ®N$, MP, kde $\diam ®X_i ≤ 1$. Nechť $(®X, \rho) = \prod_{i \in ®N}(®X_i, \rho_i)$ a budiž $\{f_n\}_{n=1}^∞ \subseteq ®X$ posloupnost bodů v ®X, $f \in ®X$. Pak $\lim_{n \rightarrow ∞} f_n = f$ v $(®X, \rho)$ $\Leftrightarrow$ $\forall i \in ®N: \lim_{n \rightarrow ∞} f_n(i) = f(i)$ v $(®X_i, \rho_i)$.

        \begin{dukazin}
            $\implies$: Nechť $f_n \rightarrow f$. Budiž dáno libovolné $i_0 \in ®N$. Nechť $\epsilon > 0$. Najdeme $n_0 \in ®N$, že $\forall n ≥ n_0: \rho(f_n, f) < \epsilon·2^{-i_0}$. Tedy pro
            $$ n ≥ n_0: \epsilon · 2^{-i_0} > \rho(f_n, f) = \sum_{i=1}^∞ \frac{\rho_i(f_n(i), f(i))}{2^i} ≥ \frac{\rho_{i_0}(f_n(i_0), f(i_0))}{2^{i_0}}, $$
            tj. $\rho_{i_0}(f_n(i_0), f(i_0)) < \epsilon$. tedy $\lim_{n \rightarrow ∞} f_n(i_0) = f(i_0)$.

            $\Leftarrow$: Nechť $\forall i \in ®N: f_n(i) \rightarrow f(i)$. Nechť $\epsilon > 0$. Najděme $i_0 \in ®N$, že $\sum_{i = i_0 + 1}^∞ \frac{1}{2^i} < \frac{\epsilon}{2}$. Pro všechna $i \in \{1, 2, …, i_0\}$ najdeme $n_i$, že $\forall n ≥ n_i: \rho_i(f_n(i), f(i)) < \frac{\epsilon}{i_0}$. Položme $\tilde{n} := \max \{n_1, n_2, …, n_{i_0}\}$. Pak $n ≥ \tilde n$: jest
            $$ \rho(f_n, f) = \sum_{i=1}^∞\frac{\rho_i(f_n(i), f(i))}{2^i} = \sum_{i=1}^{i_0} \frac{\rho_i(f_n(i), f(i))}{2^i} + \sum_{i=i_0+1}^∞ \frac{\rho_i(f_n(i), f(i))}{2^i} ≤ $$
            $$ ≤ \sum_{i=1}^{i_0} \frac{\epsilon/i_0}{2} + \sum_{i = i_0+1}^∞ \frac{1}{2^i} < \frac{\epsilon}{2} + \frac{\epsilon}{2} = \epsilon. $$ 
        \end{dukazin}
    \end{tvrzeni}

\section{Totálně omezené a separabilní MP}
    \begin{definice}[$\epsilon$-síť, $\epsilon$-separovanost]
        Buď $(®M, d)$ MP, $A \subseteq ®M$, $\epsilon > 0$.

        Řekneme, že $A$ je $\epsilon$-síť pro ®M, jestliže $\forall x \in ®M\ \exists a \in A: d(x, a) < \epsilon$.

        $A$ je $\epsilon$-separovaná, pokud $\forall x, y \in A: d(x, y) ≥ \epsilon$.

    \end{definice}

    \begin{definice}[Totálně omezený prostor, separabilní prostor]
        ®M je totálně omezený, jestliže $\forall \epsilon > 0\ \exists A \subseteq ®M: A$ je konečná $\epsilon$-síť pro ®M.

        ®M je separabilní, pokud $\exists A \subseteq ®M$ spočetná: $\overline{A} = ®M$. (Tj. $A$ je hustá v ®M.)
    \end{definice}

    \begin{veta}
        MP $(®M, d)$ je totálně omezený, právě když $\forall \epsilon > 0$ je každá $\epsilon$-separovaná množina konečná.

        \begin{dukazin}
            $\implies$: Nechť $\epsilon > 0$, $B \subseteq ®M$ je $\epsilon$-separovaná. Chceme $B$ je konečná. Protože ®M je TO, existuje konečná $\frac{\epsilon}{4}$-síť $A \subseteq ®M$. Pro každé $x \in B$ zvolíme nějaký bod $a_x \in A: d(x, a_x) < \frac{\epsilon}{4}$. Pak pro $x≠y$, $x, y \in B$ platí $a_x ≠ a_y$: $d(a_x, a_y) ≥ d(x, y) - d(x, a_x) - d(y, a_y) ≥ \epsilon - \frac{\epsilon}{4} - \frac{\epsilon}{4} = \frac{\epsilon}{2} > 0$. Tj. zobrazení $B \rightarrow A: x \mapsto a_x$ je prosté. Ale $A$ je konečná tedy $B$ je konečná.

            $\Leftarrow$ Nechť $\epsilon > 0$; chceme najít konečnou $\epsilon$-síť. Vezmeme si $B \subseteq M$ $\epsilon$-separovaná, která je maximální (co do inkluze). Tvrdíme, že $B$ je automaticky $\epsilon$ síť: Zvolme $x \in ®M$. Pak existuje $b \in B: d(x, b) < \epsilon$. Kdyby ne: $\forall b \in B: d(x, b) ≥ \epsilon$. To by znamenalo, že $B \cup \{x\} \supset B$ je $\epsilon$-separovaná, což je spor s maximalitoui B. Tj. $B$ je opravdu $\epsilon$-síť pro ®M.

            Druhá část důkazu však potřebuje tzv. Zornovo lemma, abychom mohli brát $B$ maximální.
        \end{dukazin}
    \end{veta}

% 16. 4. 2021

    \begin{veta}
        Buď $(®M, d)$ MP, $®N \subseteq ®M$. Pokud $(®M, d)$ je TO, pak i $(®N, d)$ je TO. (Tedy TO se zachovává na podprostory.)

        \begin{dukazin}
            Nechť $A \subseteq ®N$ je $\epsilon$-separovaná. Chceme: $A$ je konečná. (Pak $(®N, d)$ je TO podle V18). Ale $A \subseteq ®N \subseteq ®M$, tedy $A \subseteq ®M$ je $\epsilon$-separovaná v ®M. Ale ®M je TO, takže $A$ musí být konečná.
        \end{dukazin}
    \end{veta}
    
    \begin{veta}
        $(®M, d)$ je MP, $®N \subseteq ®M$. Je-li $(®N, d)$ TO, pak $(\overline{®N}, d)$ je TO. (TO se zachovává na uzávěr.)

        \begin{dukazin}
            $\epsilon > 0$ dáno, chceme konečnou $\epsilon$-síť v prostoru $(\overline{®N}, d)$. Nechť $A \subseteq ®N$ je konečná $\epsilon/2$-síť pro ®N (ta existuje, neboť $(®N, d)$ je TO). Chceme $Á$ je $\epsilon$-sít pro $(\overline{®N}, d)$. Zvolme libovolný bod $x \in \overline{®N}$. Chceme $\exists y \in A: d(x, y) < \epsilon$.

            Protože $x \in \overline{®N}$, existuje $z \in ®N: d(x, z) < \frac{\epsilon}{2}$. Protože $A$ je $\epsilon/2$-síť pro ®N, existuje $y \in A: d(y, z) < \frac{\epsilon}{2}$. Tedy $d(x, y) ≤ d(x, z) + d(y, z) < \epsilon/2 + \epsilon/2 = \epsilon$. Tudíž $A$ je $\epsilon$-síť pro $\overline{®N}$.
        \end{dukazin}
    \end{veta}

    \begin{veta}
        Nechť $(®M_\alpha, d_\alpha)$, $\alpha \in I$ jsou MP, $\diam M_\alpha ≤ 1$, $\forall \alpha \in I$. Pak $\sum_{\alpha \in I}(®M_\alpha, d_\alpha)$ je TO $\Leftrightarrow$ $I$ je konečná a $\forall \alpha \in I:$ $(®M_\alpha, d_\alpha)$ je TO.

        \begin{dukazin}
            $\implies$: Nechť $\sum$ je TO. Nechť $\epsilon > 0$. Pokud $\epsilon ≥ 1$, pak libovolná jednobodová množina je $\epsilon$-síť. $\epsilon < 1$. Nechť $A \subseteq \sum(®M_\alpha, d_\alpha)$ je konečná $\epsilon$-síť. Položme $A_\alpha := \{x \in ®M_\alpha | (x, \alpha) \in A\}$. Potom $A_\alpha$ je zřejmě $\epsilon$-síť $(®M_\alpha, d_\alpha)$.

            $\Leftarrow$: Podle předpokladů, pro dané $\epsilon > 0$, $\forall \alpha \in I\ \exists A_\alpha \subseteq ®M_\alpha: A_\alpha$ je konečná $\epsilon$-síť (protože $(®M_\alpha, d_\alpha)$ jsou TO). $A \bigcup_{\alpha \in I} A_\alpha\times\{\alpha\}$ je $\epsilon$-síť pro $\sum_{\alpha \in I}(®M_\alpha, d_\alpha)$.
        \end{dukazin}
    \end{veta}

    \begin{veta}
        Nechť $(®M_i, d_i)$ jsou MP, $i \in ®N$, a nechť $\forall i: \diam ®M_i ≤ 1$. Pak $(®M, d) = \prod_{i \in ®N}(®M_i, d_i)$ je TO $\Leftrightarrow$ $\forall i \in ®N: (®M_i, d_i)$ je TO.

        \begin{dukazin}
            $\implies$: (Lze provést i důkaz přímo z definic) Zvolme $a = (a_i)_{i=1}^∞ \in ®M$. Definujeme zobrazení $\phi: (®M_i, d_i) \rightarrow (®M, d)$ tak, že $\phi(x) = (a_1, a_2, …, a_{i-1}, x, a_{i+1}, a_{i+2}, …) \in ®M$. Pak $\phi$ je izometrie. Tedy $(®M_i, \frac{d_i}{2^i})$ lze chápat jako podprostor (neboli izometrickou kopii podprostoru) $(®M, d)$. Ale ®M je TO, tedy i jeho podprostor je TO.

        $\Leftarrow$: Nechť $\epsilon > 0$ je dáno. Zvolíme $i_0 \in ®N: \sum_{i=i_0}^∞ \frac{1}{2^i} < \frac{\epsilon}{2}$. Pro $i \in \{1, …, i_0 - 1\}$ najdeme konečné $\epsilon/2$-síťě $A_i$ pro $®M_i$. $S = \{(x_i) \in ®M | x_i \in A_i, i \in \{1, …, i_0 - 1\} \lor x_i = a_i, i ≥ i_0\}$. $S$ je konečná, neboť $|S| \prod_{i=1}^{i_0 - 1}|A_i|$. $S$ je $\epsilon$-síť pro ®M.
        \end{dukazin}
    \end{veta}

% 23. 4. 2021

    \begin{veta}
        Buď $(®M, d)$ MP. Pak následující je ekvivalentní: 1) ®M je separabilní, 2) existuje spočetná množina otevřených podmnožin $B_n \subseteq ®M$ tak, že $\forall G \subseteq ®M$ otevřená: $\exists I \subseteq ®N: G = \bigcup_{n\in I}B_n$, 3) každý podprostor ®M je separabilní, 4) každá $\epsilon$-separovaná množina je spočetná.

        \begin{dukazin}
            $1) \implies 2)$: Uvažujme nějakou spočetnou hustou podmnožinu $D \subseteq ®M$. $D = \{x_i: i \in ®N\}$. $B_{(i, j)} := \B(x_i, 1/j)$. $®N^2$ je spočetná, tedy těchto koulí je spočetně mnoho. Nechť je dána libovolná otevřená $G \subseteq ®M$. Chceme $G = \bigcup_{(i, j) \in I}B_{(i, j)}$, kde $I := \{(i, j) \in ®N^2 | B_{i, j} \subseteq G\}$. Zvolme $x \in G$. Chceme $\exists(i, j) \in I: x \in B_{(i, j)}$. Protože $G$ je otevřená, $\exists \epsilon > 0: \B(x, \epsilon) \subseteq G$. Podle definice hustoty najdeme $\frac{1}{j} < \frac{\epsilon}{2}$. $\exists i \in ®N: x_i \in (x, \frac{1}{j})$. Pak $B_{(i, j)} \subseteq \B(x, \epsilon) \subseteq G$. Tedy $x \in \bigcup_{(i, j) \in I}B_{(i, j)}$.

            $2) \implies 1)$: Nechť $B_n$ mají vlastnost z $2)$. Chceme ukázat, že $M$ je separabilní. Vybereme $x_n \in B_n$, $n \in ®N$. Pak $D := \{x_n : n \in ®N\}$ je hustá. Skutečně, budiž $\O ≠ G \subseteq ®M$ otevřená, pak $G = \bigcup_{n \in I}B_n$ pro nějaké $I \subseteq ®M$. Tím pádem $\forall n \in I: x_n \in B_n \subseteq G$, tj. $(I≠\O)$ některé prvky $D$ jsou v $G$.

            $2) \implies 3)$: Buď $®O \subseteq ®M$ libovolný podprostor. Chceme: ®O je separabilní. Stačí, že ®O splňuje $2)$, tj. má spočetnou bázi otevřených množin. ®M má spočetnou bázi otevřených množin $\{B_n | n \in ®N\}$. Tvrdíme, že $\{B_n \cap ®O | n \in ®N\}$ má vlastnost z 2) (tj. je to báze) pro ®O. Tedy všechny podprostory ®M jsou separabilní.

            $3) \implies 4)$: Nechť ®M splňuje $3)$. Budiž dáno $\epsilon > 0$ a libovolná $\epsilon$-separovaná podmnožina $A \subseteq ®M$. Chceme $|A| ≤ |®N|$. $A$, jakožto podprostor ®M, je separabilní. Pro libovolné $x \in A$ jest $\B(x, \epsilon)\cap A = \{x\}$. Je-li nyní $D$ spočetná hustá v $(A, d)$, pak $\forall x: A \cap \B(x, \epsilon) ≠ \O$, tj. $\forall x \in A: x \in D$, tedy $D = A$ a $A$ je spočetná.

            $4) \implies 1)$: Zornovo lemma: Pro $\epsilon = \frac{1}{n}$ uvažujme maximální $\frac{1}{n}$-separovanou podmnožinu $S_n \subseteq ®M$. Dokážeme, že $S := \bigcup_{n \in ®N}S_n$ je hustá v ®M: Nechť $x \in ®M$, $\epsilon > 0$ jsou dány. Najděme $n \in ®N : \frac{1}{n} < \epsilon$. Tvrdíme, že existuje $y \in S_n: d(x, y) < \frac{1}{n}$. Kdyby ne, pak $S_n \cup \{x\}$ by byla $\frac{1}{n}$-separovaná, což by byl spor s maximalitou $S_n$. Tedy $y \in S$ a $d(x, y) < \frac{1}{n} < \epsilon$.
        \end{dukazin}
    \end{veta}

    \begin{tvrzeni}
        Pro prostory $(®M_n, d_n)$, $n \in ®N$ splňující $\diam ®M_n ≤ 1$, $n \in ®N$, platí $\prod_{i=1}^∞ (®M_i, d_i)$ je separovaný $\Leftrightarrow$ $\forall i: (®M_i, d_i)$ je separabilní.
        
        \begin{dukazin}
            Cvičení.
        \end{dukazin}
    \end{tvrzeni}

    \begin{dusledek}
        $®M$ je separovaný $\implies$ $®M$ je TO.
    \end{dusledek}

% 30. 4. 2021

\section{Úplné prostory}
    \begin{definice}[Cauchyovská posloupnost, úplný prostor]
        Buď $(®M, \rho)$ MP, $\{x_n\}_{n=1}^∞$ posloupnost prvků ®M. Řekneme, že $\{x_n\}$ je cauchyovská, jestliže platí
        $$ \forall \epsilon > 0\ \exists n_0 \in ®N\ \forall n, m ≥ n_0: \rho(x_n, x_m) < \epsilon. $$

        Řekneme, že prostor $(®M, \rho)$ je úplný, jestliže v něm každá cauchyovská posloupnost konverguje (tj. $\exists x \in ®M: \lim_{n \rightarrow ∞} x_n = x$).
    \end{definice}

% 7. 5. 2021

    \begin{veta}[Cantorův princip]
        Buď $(®M, \rho)$ MP. $®M$ je úplný $\Leftrightarrow$ pro každou posloupnost uzavřených množin $\{F_n\}_{n=1}^∞$, $F_n \subseteq ®M$, kde $F_1 \supseteq F_2 \supseteq …$ a $\diam F_n \rightarrow 0$, platí $\bigcap F_n ≠ \O$.

        \begin{dukazin}
            $\implies$: Vybereme prvky $x_n \in F_n, n \in ®N$ libovolně. Máme tedy $\{x_n\}_{n=1}^∞$. Tato posloupnost je Cauchyovská, jelikož pro $\epsilon > 0$ najdeme $n_0 \in ®N$, že $\forall n ≥ n_0: \diam F_n < \epsilon$. Pak $\forall n, m ≥ n_0: x_n \in F_n, x_m \in F_m \subseteq F_{n_0}$, tj. $\rho(x_n, x_m) < \epsilon$.

            Tedy existuje $x \in ®M: \lim_{n \rightarrow ∞} x_n = x$. Tvrdíme, že $x \in \bigcap F_n$. Nechť je dána $m \in ®N$. Pak $\forall n ≥ m: x_n \in F_m$. Ale $F_m$ je uzavřená, a tedy $\lim_{n \rightarrow ∞} x_n \in F_m$.

            $\Leftarrow$: Nechť platí podmínka věty. Buď $\{x_n\}_{n=1}^∞ \subseteq ®M$ cauchyovská. Chceme dokázat, že je konvergentní. Položme $F_n = \overline{\{x_i | i ≥ n\}}$. Zřejmě $F_1 \supseteq F_2 \supseteq …$. $\diam F_n \rightarrow 0$, protože $\{x_n\}$ je cauchyovská. Tedy podle předpokladů existuje $x \in \bigcap F_n$.

            Ukážeme, že $x = \lim x_n$: Nechť je dáno $\epsilon > 0$. Zvolme $n_0 \in ®N: \diam F_{n_0} < \epsilon$. Pro libovolné $n ≥ n_0$ platí $\rho(x_n ,x) ≤ \diam F_n ≤ \diam F_{n_0} < \epsilon$. Tedy $x_n \rightarrow x$.
        \end{dukazin}
    \end{veta}

    \begin{veta}[Úplnost a podprostory]
        Buď $(®M, \rho)$ MP, $®N \subseteq ®M$ jeho podprostor. Potom $(®N, \rho)$ je úplný $\Leftrightarrow$ ®N je uzavřený v ®M.

        \begin{dukazin}
            $\implies$: Kdyby ®N nebyla uzavřená množina, existovala by posloupnost $\{x_n\}_{n=1}^∞ \subseteq ®N$, že $\lim_{n \rightarrow ∞} x_n \in ®M \setminus ®N$. Ale protože $\{x_n\}$ je konvergentní v ®M, je cauchyovská v ®M, a tedy i v ®N. Nemá ale v ®N limitu a tak ®N není úplný.

            $\Leftrightarrow$: Nechť $®N \subseteq ®M$ je uzavřená v ®M. Nechť $\{x_n\}$ je cauchyovská v ®N, pak je cauchyovská i v ®M, a protože ®M je úplný, tak $\exists x \in ®M: x = \lim_{n \rightarrow ∞} x_n$. Ale ®N je uzavřený, takže $x \in ®N$.
        \end{dukazin}
    \end{veta}

    \begin{veta}[Úplnost a součin]
        Nechť $(®M_i, \rho_i)$, $i \in ®N$ jsou MP, že $\forall i \in ®N: \diam ®M_i ≤ 1$. Pak $\prod_{i = 1}^∞ (®M_i, \rho_i)$ je úplný $\Leftrightarrow$ $\forall i: (®M_i, \rho_i)$ je úplný.

        \begin{dukazin}
            $\implies$: $(®M_i, \rho_i)$ lze chápat jako podprostor součinu $\prod (®M_i, \rho_i)$. Lze si snadno rozmyslet, že je uzavřený (cvičení). Tedy podle předchozí věty je úplný.

            $\Leftarrow$: Nechť $\{x_n\} \subseteq \prod (®M_i, \rho_i) =: ®M$ je cauchyovská posloupnost v ®M, $x_n(i) \in ®M_i$. Tvrdíme, že $\forall i \in ®N: \{x_n(i)\}_{n=1}^∞$ je cauchyovská v $®M_i$. Buď $\epsilon > 0$. Najdeme $n_0 \in ®N$, že $\forall m, n ≥ n_0: \rho(x_n, x_m) < \frac{\epsilon}{2^i}$.
            $$ \rho(x_n, x_m) = \sum_{j=1}^∞ \frac{\rho_j(x_n(j), x_m(j))}{2^j} < \frac{\epsilon}{2^i}. $$
            Tedy $\rho_i(x_n(i), x_m(i)) < \epsilon$. Tudíž existuje limita $x(i) := \lim_{n \rightarrow ∞} x_n(i)$, tj. $x \in \prod_{j=1}^∞ (®M_j, \rho_j)$. Jednoduše (standardní zanedbatelnost „vocasu řady“) dokážeme, že $\rho(x_n, x) \rightarrow 0$.
        \end{dukazin}
    \end{veta}

    \begin{veta}[Bairova]
        Nechť $(®M, \rho)$ je úplný MP. Nechť $U_n$ jsou otevřené, husté podmnožiny ®M, $n \in ®N$. Pak $\bigcap_{n=1}^∞ U_n$ je hustá.

        \begin{dukazin}
            Stačí, že $\bigcap_{n=1}^∞ U_n$ protne libovolnou kouli. Mějme tedy dáno $x, \epsilon$. Buď $B := \B(x, \epsilon)$. Protože $U_1$ je hustá, existuje $x_1 \in U_1 \cap B$. Ale $U_1$ i $B$ jsou otevřené, takže i $U_1 \cap B$ je otevřená, takže existuje $\epsilon_1 > 0$, že $\B(x_1, 2\epsilon_1) \subseteq U_1 \cap B$. Tedy $\overline{\B(x_1, \epsilon_1)} \subseteq U_1 \cap B$. Analogicky pokračujeme (navíc chceme, aby $\epsilon_i > \epsilon_{i+1}$).

            Tím jsme dostali klesající (co do inkluze) posloupnost uzavřených množin, jejichž diametr jde k 0. Tedy průnik těchto množin je neprázdný podle Cantorova principu.
        \end{dukazin}
    \end{veta}

    \begin{dusledek}
        Existuje spojitá funkce, která nemá v žádném bodě derivaci.
    \end{dusledek}

    \begin{definice}[Zajímavost]
        $I$ množina, $l_∞ := \{f: I \rightarrow ®R \text{ je omezená}\}$. $d_∞(f, g) = \sup_{x \in I}|f(x) - g(x)|$.
    \end{definice}

    \begin{tvrzeni}[Zajímavost]
        $(l_∞, d_∞)$ je úplný.
    \end{tvrzeni}

    \begin{veta}[Zajímavost]
        Nechť $(®M, \rho)$ je MP, $\overline{D} = ®M$. Pak existuje izometrie $\phi: (®M, \rho) \rightarrow l_∞(D)$.
    \end{veta}

% 14. 5. 2021

\section{Kompaktnost}

    \begin{definice}[Pokrytí, otevřené pokrytí, konečná průniková vlastnost]
        Ať ©S je systém podmnožin množiny $X$. Řekneme, že ©S je pokrytí $X$, pokud $\bigcup ©S = X$.

        Je-li $(®X, \rho)$ MP, $©S \subseteq ©P(®X)$, říkáme, že ©S je otevřené pokrytí ®X, pokud ©S je pokrytí ®X a každá $S \in ©S$ je otevřená množina v $(®X, \rho)$.

        $©S \subseteq ©P(X)$ má konečnou průnikovou vlastnost $≡$ $\forall n \in ®N\ \forall S_1, …, S_n \in ©S: \bigcap S_i ≠ \O$.
    \end{definice}

    \begin{definice}[Kompaktní prostor]
        MP $(®X, \rho)$ se nazývá kompaktní, pokud z každého otevřeného pokrytí ©S lze vybrat konečnou $©S' \subseteq ©S: \bigcup ©S' = ®X$.
    \end{definice}

    \begin{definice}
        Ať $(®M, \rho)$ je MP, $A \subseteq ®M$. Řekneme, že $x \in ®M$ je hromadným bodem množiny $A$, pokud $\forall \epsilon > 0: (\B(x, \epsilon) \setminus \{x\}) \cap A ≠ \O$.
    \end{definice}

    \begin{veta}[Charakterizace kompaktnosti]
        Pro $(®M, \rho)$ je ekvivalentní:

        \begin{enumerate}
            \item ®M je kompaktní.
            \item Je-li $©F \subseteq ©P(®M)$ soubor uzavřených množin s konečnou průnikovou vlastností, pak $\bigcap ©F ≠ \O$.
            \item Je-li $A \subseteq ®M$ nekonečná, pak $A$ má hromadný bod v ®M.
            \item Z každé posloupnosti v ®M lze vybrat konvergentní podposloupnost.
            \item Každá spojitá funkce $f: ®M \rightarrow ®R$ je omezená.
            \item Každá spojitá funkce $f: ®M \rightarrow ®R$ nabývá svého minima a maxima.
            \item Z každého spočetného otevřeného pokrytí lze vybrat konečné pokrytí.
        \end{enumerate}
    
        \begin{dukazin}
            $1 \implies 2$: Ať ©F je soubor uzavřených množin v $(®M, \rho)$ s konečnou průnikovou vlastností. Pro spor ať $\bigcap ©F = \O$. Ať $©S := \{®M \setminus F | F \in ©F\}$. $®M \setminus F$ je otevřená pro $F \in ©F$. $\bigcup ©S = ®M \setminus \bigcap ©F = ®M \setminus \O = ®M$. Tedy ©S je otevřené pokrytí ®M. ®M je kompaktní, tedy $S_1, …, S_n \in ©S: S_1 \cup S_2 \cup … \cup S_n = ®M$. $\forall i ≤ n\ \exists F_i \in ©F: S_i = ®M \setminus F_i$. $F_1 \cap … \cap F_n = ®M \setminus (S_1 \cup … \cup S_n) = ®M \setminus ®M = \O$. Tedy ©F nemá konečnou průnikovou vlastnost. \lightning.
        \end{dukazin}

        \begin{dukazin}
            $2 \implies 3$: Ať $A \subseteq ®M$ je nekonečná. Ať pro spor $A$ nemá hromadný bod. Uvažujme $a \in A$, $A \setminus \{a\}$ je uzavřená množina (protože jinak by bod z $\overline{A} \setminus A$ byl hromadným bodem). $©F:= \{A \setminus \{a\} | a \in A\}$ je soubor uzavřených množina a navíc má konečnou průnikovou vlastnost (můžeme odečíst pouze konečně mnoho $a_i \in A$, ale těch je nekonečně). Dle 2) je $\bigcap ©F ≠ \O$. Ale $\bigcap \{A \setminus \{a\} | a \in A\} = A \setminus A = \O$. \lightning.
        \end{dukazin}

        \begin{dukazin}
            $3 \implies 4$: Ať $x_n \in ®M$. Pokud $A := \{x_n | n \in ®N\}$ je konečná, najdeme konstantní (tedy konvergentní) podposloupnost. Jinak $A$ má hromadný bod $x \in ®M$. Najdeme podposloupnost konvergující k $x$, indukcí $n_1 := 1$, $n_i < n_{i+1}$: $\rho(x_{n_k}) < \frac{1}{k}$, $k > 1$:

            $\B(x_{n_k}, \frac{1}{k+1}) \cap A$ je nekonečná, tedy existuje $n_{k+1} > n_k$, že je splněna chtěná podmínka. Jednoduše následně ukážeme i to, že tato posloupnost konverguje k $x$.
        \end{dukazin}

        \begin{dukazin}
            $4 \implies 5$: Zase sporem: Ať $f: ®M \rightarrow ®R$ je spojitá neomezená funkce. Tj. $\forall n \in ®N\ \exists x_n \in ®M: |f(x_n)| ≥ n$. Dle 4) ale víme, že posloupnost $x_n$ má konvergentní podposloupnost $x_{n_k}$. $x_{n_k} \rightarrow x \in ®M$. Ale $f$ je spojitá, tedy $f(x_{n_k}) \rightarrow f(x)$. \lightning.

            $5 \implies 6$: $f: ®M \rightarrow ®R$ spojitá, $f$ nenabývá maxima (búno). Dle 5 je $f$ omezená, tedy $\sup_{x \in ®M} f(x) = \alpha < ∞$. $g := \frac{1}{\alpha - f}$, $\forall x \in ®M: f(x) ≠ \alpha$, tedy $g$ je spojitá (a dobře definovaná). $\exists x_n \in ®M: \lim_{n \rightarrow ∞} f(x_n) = \alpha$. $\lim_{n \rightarrow ∞} g(x_n) = ∞$. Tedy $g$ není omezená, spor s $5$.
        \end{dukazin}

        \begin{dukazin}
            $6 \implies 7$: Ať $©S = \{S_1, S_2, …\}$ je spočetné otevřené pokrytí ®M. Položme $V_n := S_1 \cup … \cup S_n$. Ať pro spor ©S nemá konečné podpokrytí. Tedy $V_n \subset ®M$, $V_1 \subseteq V_2 \subseteq …$, $\bigcup V_n = ®M$. Můžeme předpokládat, že $V_i \subset V_{i+1}$. Vybereme $x_i \in V_{i+1} \setminus V_i$ libovolně. Pro $i \in ®N$ ať $\epsilon_i = \min_{j ≤ i} \rho(x_i, x_j)$ a splňuje $\B(x_i, \epsilon_i) \subseteq V_{i + 1}$.
            $$ f(x) = \frac{4k}{\epsilon_k}\(\frac{\epsilon_k}{4} - \rho(x, x_n)\), \text{ pro } x \in \B(x_{n_k}, \frac{\epsilon_k}{4}), \text{ a } f(x) = 0 \text{ jinak}. $$
            $f(x_k) = k$, $k \in ®N$, tedy $f$ není omezená. Ale je spojitá. Spor s $6$.
        \end{dukazin}

        \begin{dukazin}
            $7 \implies 1$: Nejprve ukážeme, že ®M je separabilní, sporem: ®M není separabilní, tedy podle charakterizace separability existuje $\epsilon > 0$ a existuje $A \subseteq ®M$, $A$ nespočetná, $A$ je $\epsilon$-separovaná. Ať $A' \subseteq A$ je spočetná, nekonečná. Zřejmě $A'$ je $\epsilon$-separovaná. $A'$ je uzavřená v ®M. $\{M \setminus A'\} \cup \{\B(a, \frac{\epsilon}{2}) | a \in A'\}$ spočetné otevřené pokrytí ®M. Tento systém ale nemá konečné podpokrytí, což je spor s $7$, tedy ®M je separabilní.

            Ať ©S je otevřené pokrytí ®M. ®M je separabilní, tedy existuje systém $\{B_n | n \in ®N\}$ otevřených množin, že $\forall G \subseteq ®M$ otevřenou $\exists J \subseteq ®N$:
            $$ G = \bigcup \{B_n | n \in J\}. $$
            Tedy $\forall S \in ©S\ \exists J_S \subseteq ®N: S = \bigcup \{B_n | n \in J_S\}$. $J := \bigcup_{S \in ©S} J_S \subseteq ®N$. $\forall n \in J\ \exists S_n \in ©S: B_n \subseteq S_n$. Tj. $\bigcup B_n = \bigcup ©S = ®M$. Tedy $\bigcup_{n=1}^∞ S_n = M$. Tedy $\{S_1, S_2, …\}$ je spočetné podpokrytí ®M. Dle 7 existuje $k \in ®N: S_1 \cup … \cup S_k = ®M$.
        \end{dukazin}
    \end{veta}

    \begin{veta}
        Ať $(®M, \rho)$ je MP. Pak ®M je kompaktní, právě když ®M je úplný a totálně omezený.

        \begin{dukazin}
            $\implies$: Úplnost: Ať $F_1 \supseteq F_2 \supseteq …$ jsou uzavřené neprázdné a $\lim_{n \rightarrow ∞} \diam F_1 = 0$. Chceme, že $\bigcap F_n ≠ \O$. $©F := \{F_1, F_2, …\}$ je systém uzavřených množin s konečnou průnikovou vlastností. Podle předchozí věty je $\bigcap ©F ≠ \O$. Tj. $\bigcap_{n=1}^∞ F_n ≠ \O$.

            Totální omezenost: Ať $\epsilon > 0$. $\{\B(x, \epsilon) | x \in ®M\}$ je otevřené pokrytí ®M. Z kompaktnosti existuje konečné podpokrytí, tj. $\exists n \in ®N\ \exists x_1, …, x_n \in ®M: \B(x_1, \epsilon)\cup … \cup \B(x_n, \epsilon) = ®M$. Tedy $\{x_1, …, x_n\}$ je $\epsilon$-síť pro ®M. Tedy $(®M, \rho)$ je totálně omezený.

            $\Leftarrow$: Ukážeme, že každá nekonečná $A \subseteq ®M$ má hromadný bod. Indukcí najdeme klesající posloupnost $B_n, n \in ®N$, uzavřených množin, že $\diam B_n ≤ \frac{1}{n}$, $B_n \cap A$ je nekonečná:

            Z totální omezenosti ®M existuje konečné pokrytí ®M uzavřenými koulemi $U_1, …, U_n$ s diametrem $≤ 1$. $\exists i ≤ n: U_i \cap A$ je nekonečná. Položme $B_1 := U_i$.

            Máme-li $B_1, …, B_n$, víme, že $B_n$ je totálně omezená, $B_n \cap A$ je nekonečná. Z totální omezenosti $B_n$ existují uzavřené $V_1, …, v_l$, že $B_n = V_1 \cup … \cup V_l$, že $B_n = V_1 \cup … \cup V_l$, $\diam V_l ≤ \frac{1}{n+1}$. Existuje $j ≤ l: V_j \cap A$ je nekonečná. $B_{n+1} := V_j$. $B_{n+1} \subseteq B_n$.

            Nyní z úplnosti ®M je $\bigcap B_n ≠ \O$. Ať $a \in \bigcap B_1$. $a$ je hromadným bodem $A$: Ať $\epsilon > 0$, pak $\exists n \in ®N: \frac{1}{n} < \frac{\epsilon}{2}$. $B_n \subseteq \B(a, \epsilon)$, $\B(a, \epsilon) \cap A$ je nekonečná, tedy $a$ je hromadný bod $A$.
        \end{dukazin}
    \end{veta}

% 21. 5. 2021

    \begin{tvrzeni}[Zachovávání kompaktnosti operacemi]
        1. Je-li $(®X, \rho)$ kompaktní MP a $Y \subseteq ®X$ uzavřená, pak $(®Y, \rho|_{®Y})$ je kompaktní.

        2. Je-li $(®X, \rho)$ kompaktní MP a $(®Y, \rho)$ MP, $f: (®X, \rho) \rightarrow (®Y, \sigma)$ spojité, pak $(f(®X), \sigma_{f(®X)})$ je kompaktní.

        3. Je-li $(®X, \rho)$ MP a $®Y \subseteq ®X$, že $(®Y, \rho|_{®Y})$ je kompaktní, pak ®Y je uzavřená v $(®X, \rho)$.

        4. Jsou-li $(®X_i, \rho_i)$ kompaktní MP, $i \in ®N$, $\rho_i ≤ 1$, pak $\prod_{i \in ®N}(®X_i, \rho_i)$ je kompaktní.

        5. Jsou-li $(®X_i, \rho_i)$ MP a $(®X, \rho)$ je suma prostorů $(®X_i, \rho_i)$, $i \in I$. Pak $(®X, \rho)$ je kompaktní $\Leftrightarrow$ $\forall i \in I$: $(®X_i, \rho_i)$ je kompaktní a $\{i \in I | ®X_i ≠ \O\}$ je konečná.

        \begin{dukazin}
            1. Kompaktnost $\Leftrightarrow$ úplnost + totální omezenost. Uzavřený podprostor úplného je úplný, tedy ®Y je úplný (jelikož ®X je úplný). Podprostor totálně omezeného je totálně omezený, tedy ®Y je totálně omezený. Tedy ®Y je kompaktní.

            2. Ať ©U je otevřené pokrytí $f(®X)$. $f$ je spojité, tedy $f^{-1}(U)$ jsou otevřené pro $U \in ©U$. Navíc $\{f^{-1}(U) | U \in ©U\}$ je pokrytí ®X. ®X je kompaktní, tedy existuje konečné podpokrytí $\{f^{-1}(U_1), …, f^{-1}(U_n)\}$ pro $n \in ®N$ a $U_1, …, U_n \in ©U$. $®X = f^{-1}(U_1) \cup … \cup f^{-1}(U_n) = f^{-1}(U_1 \cup … \cup U_n)$, tedy $f(®X) \subseteq U_1 \cup … \cup U_n$, tedy $\{U_1, …, U_n\}\subseteq ©U$ je konečné podpokrytí $f(®X)$. Tedy $f(®X)$ je kompaktní.

            3. ®Y je kompaktní, tedy úplný. Ale tím pádem je uzavřený (úplný podprostor je nutně uzavřený).

            4. Převedením na úplnost a totální omezenost (součin spočetně mnoha úplných MP je úplný, stejně tak součin spočetně mnoha totálně omezených MP je totálně omezený).

            5. $\implies$ $J := \{i \in I | ®X_i ≠ \O\}$ je konečná sporem: Ať $J$ je nekonečná. Pro $j \in J$ vybereme $x_j \in ®X_j$. $A := \{x_k | j \in J\}$. $A$ je nekonečná, ®X kompaktní, tedy podle charakterizace kompaktnosti má $A$ hromadný bod $a \in \sum (®X_i, \rho)$. $\B(a, \frac{1}{2}) \cap A$ nekonečná, $\exists i ≠ j, i, j \in J: x_i, x_j \in \B(a, \frac{1}{2})$. $\rho(x_i, x_j) < 1$, spor, neboť $\rho(x_i, x_j) = 1$.

            $\forall i \in I: ®X_i$ je kompaktní: $®X_i$ je uzavřená podmnožina $\sum(…®X_i, \rho_i)$. Podle 1. je $®X_i$ kompaktní.

            $\Leftarrow$: $\{i \in I | ®X_i ≠ \O\}$ je konečná, $\forall i \in I: ®X_i$ kompaktní. Chceme, že $\sum(®X_i, \rho_i)$ je kompaktní. Ať $A \subseteq \sum(®X_i, \rho_i)$ je nekonečná. Nutně existuje $i \in I: ®X_i \cap A$ má hromadný bod v $®X_i$. Ten je zřejmě hromadným bodem celé množiny $A$. Tedy podle charakterizace kompaktnosti pomocí hromadných bodů nekonečných množin je $\sum (®X_i, \rho_i)$.
        \end{dukazin}
    \end{tvrzeni}

    \begin{lemma}[Lebesgueovo číslo]
        Ať $(®X, \rho)$ je kompaktní MP a ©U otevřené pokrytí ®X. Pak existuje $\delta > 0$, ze pro každé $x \in ®X$ existuje $U \in ©U: \B(x, \delta) \subseteq U$.

        \begin{dukazin}
            ®X kompaktní, tedy existuje konečné podpokrytí ©U: $\{U_1, …, U_n\} \subseteq ©U$. Pokud existuje $i \in \{1, …, n\}$ tak, že $U_i = ®X$, jsme hotovi, $\delta = 1$. Předpokládejme tedy, že $\forall i ≤ n: U_i \subsetneq ®X$. Ať $C_i := ®X \setminus U_i$, $C_i$ uzavřené, $C_i ≠ \O$. Ať $f(x) = \frac{1}{n} \sum_{i=1}^n \dist_\rho(x, C_i)$. $f$ je spojité. $f: ®X \rightarrow [0, ∞)$, $\forall x \in ®X: \exists i \in \{1, …, n\} : x \in U_i$, $x \notin C_i, \dist_\rho(x, C_i) > 0$. tedy $f: ®X \rightarrow (0, ∞)$. ®X kompakt, tedy $f$ nabývá svého minima $\delta := \min f(®X) > 0$.

            Nyní pro $x \in ®X: f(x) ≥ \delta$. $f(x) =$ aritmetický průměr $n$ čísel, tedy $\exists i \in \{1, …, n\}: \dist_\rho(x, C_i) ≥ \delta$. Tj. $\B_\rho(x, \delta) \cap C_i = O$, tj. $\B_\rho(x, \delta) \subseteq U_i$.
        \end{dukazin}
    \end{lemma}

    \begin{dusledek}
        Ať $(®X, \rho)$, $(®Y, \sigma)$ jsou MP, $(®X, \rho)$ kompaktní, $f: (®X, \rho) \rightarrow (®Y, \sigma)$ spojitá. Pak $f$ je stejnoměrně spojitá.

        \begin{dukazin}
                Ať $\epsilon > 0$ dáno. $\{f^{-1}\(\B_\sigma\(y, \frac{\epsilon}{2}\)\) | y \in ®Y\}$ je otevřené pokrytí ®X ($f$ spojitá, tedy vzor otevřené je otevřená). ®X kompaktní. Tedy ať $\delta$ je Lebesgueovo číslo z předchozího lemmatu příslušné uvedenému otevřenému pokrytí, tj. $\forall x \in ®X\ \exists y \in ®Y: \B_\rho(x, \delta) \subseteq f^{-1}\(\B_\sigma\(y, \frac{\epsilon}{2}\)\)$. $f(\B_\rho(x, \delta)) \subseteq \B_\sigma\(y, \frac{\epsilon}{2}\)$. Tedy pokud $x, x' \in ®X, \rho(x, x') < \delta$, pak $x' \in \B_\rho(x, \delta)$ a pak $f(x), f(x') \in \B_\sigma\(y, \frac{\epsilon}{2}\)$ pro nějaké $y \in ®Y$. Tedy $\sigma(f(x), f(x')) < \epsilon$.
        \end{dukazin}
    \end{dusledek}

    \begin{lemma}
        Ať $(®X, \rho)$ a $(®Y, \sigma)$ jsou MP, $(®X, \rho)$ kompaktní, $f: (®X, \rho) \rightarrow (®Y, \sigma)$ spojitá bijekce. Pak $f$ je homeomorfismus.

        \begin{dukazin}
            Potřebujeme pouze ověřit, že $f^{-1}(®Y, \sigma) \rightarrow (®X, \rho)$ je spojité. Ať $G \subseteq ®X$ je otevřená. $(f^{-1})^{-1}(G)$ je otevřená v $(®Y, \sigma)$, jelikož $f(G) = ®Y \setminus f(®X \setminus ®G)$ je otevřený (vzor je uzavřená, tedy kompakt, tedy obraz je kompakt, tedy uzavřený, tedy doplněk je otevřený). Tedy $f^{-1}$ je spojitá.
        \end{dukazin}
    \end{lemma}

    \begin{veta}[Vlastnosti Cantorova diskontinua a Hilbertovy kostky]
        Pro každý neprázdný kompaktní MP $(®X, \rho)$ existuje spojitá surjekce $f: C \rightarrow ®X$. Pro každý separabilní MP $(®Y, \sigma)$ existuje spojitá $g: ®Y \rightarrow Q$, že $g: ®Y \rightarrow g(®Y)$ je homeomorfismus. 
    \end{veta}

% 28. 5. 2021

\section{Souvislost}
    \begin{definice}[Souvislý prostor]
        MP $(®X, \rho)$ se nazývá souvislý, pokud $®X ≠ \O$ a pokud $®X = U \cup V$ pro $U, V$ otevřené, neprázdné, pak $U \cap V ≠ \O$.
    \end{definice}

    \begin{poznamka}
        $(®X, \rho)$ není souvislý, právě když $®X = \O$ nebo $®X = U \cup V$, že $U \cap V = \O$, $U, V$ otevřené neprázdné. (Tj. existuje obojetná množina $ ≠ \O, ®X$.)

        Souvislost je topologický pojem.
    \end{poznamka}

    \begin{veta}[Charakterizace souvislosti]
        Pro neprázdný MP $(®X, \rho)$ je ekvivalentní: 1) ®X je souvislý, 2) ®X má právě 2 obojetné podmnožiny, 3) je-li $f: ®X \rightarrow [0, 1]$ spojitá a $f(®X) \supseteq \{0, 1\}$, pak je $f$ na.

        \begin{dukazin}
            $1) \implies 2):$ $\O, ®X$ jsou vždy obojetné, $®X ≠ \O$. Kdyby existovala obojetná $A \subsetneq ®X$, pak $A ≠ \O, ®X$, $®X \setminus A ≠ \O, ®X$, pak $®X = A \cup (®X \setminus A)$ je sjednocení dvou otevřených neprázdných disjunktních množin, tedy spor se souvislostí.

            $2) \implies 3):$ Ať $f: ®X \rightarrow [0, 1]$ je spojitá, $f(®X) \supseteq \{0, 1\}$. Ať $\exists r \in (0, 1): f^{-1}(r) = \O$. Intervaly $[0, r)$ a $(r, 1]$ jsou otevřené množiny v $[0, 1]$, tedy i jejich vzory (neprázdné disjunktní množiny) jsou otevřené. Tj. $®X = f^{-1}([0, 1]) = f^{-1}([0, r)) \cup f^{-1}(r) \cup f^{-1}((r, 1]) = f^{-1}([0, r)) \cup f^{-1}((r, 1])$. A tedy $\O, ®X, f^{-1}((r, 1]), f^{-1}([0, r))$ jsou obojetné množiny. \lightning.

            $3) \implies 1):$ Ať $®X = U \cup V$, kde $U, V$ jsou otevřené neprázdné. Chceme $U \cap V ≠ \O$. Sporem: $U \cap V = \O$. Definujeme $f: ®X \rightarrow [0, 1]$, $f(x) = 0$, $x \in U$ a $f(x) = 1$, $x \in V$. $f$ je spojitá, jelikož $f^{-1}(G) \in \{\O, ®X, U, V\}$, pro $G \subseteq [0, 1]$, což jsou otevřené množiny. $f$ tedy nenabývá např. hodnoty $1/2$. \lightning.
        \end{dukazin}
    \end{veta}

    \begin{dusledek}
        $[0, 1]$ je souvislá.

        \begin{dukazin}
            Každá spojitá funkce $f: [0, 1] \rightarrow [0, 1]$ má Darbouxovu vlastnost, tedy z předchozí charakterizace je $[0, 1]$ souvislá.
        \end{dukazin}
    \end{dusledek}

    \begin{veta}
        Ať $(®X, \rho)$ je MP a ať pro každou dvojici $a, b \in ®X$ existuje souvislá množina $S(a, b)$, že $S(a, b) \supseteq \{a, b\}$. Pak ®X je souvislý.

        \begin{dukazin}
            Sporem: $®X = U \cup V$, $U ≠ \O ≠ V$, $U, V$ otevřené, $U \cap V = \O$. Ať $a \in U, b \in V$ libovolné. $S(a, b)$ je souvislá, ale $S(a, b) = (S(a, b) \cap U) \cup (S(a, b) \cap V)$, což jsou otevřené neprázdné (obsahují $a$ a $b$), tedy máme spor se souvislostí $S(a, b)$.
        \end{dukazin}
    \end{veta}

    \begin{veta}
        Ať $(®X, \rho)$ je MP a ©S je soubor nějakých souvislých podmnožin ®X, pro který je $\bigcap ©S ≠ \O$, pak $\bigcup ©S$ je souvislý podprostor ®X.

        \begin{dukazin}
                Ať $U, V$ jsou otevřené v $\bigcup ©S$ a $U \cap V = \O$. Chceme dokázat $U = \O$ nebo $V = \O$. Fixujeme $x_0 \in \bigcap ©S$. BÚNO $x_0 \in U$. Pro libovolné $y \in \bigcup ©S$ existuje $S \in ©S: y \in S$. $S = (S \cap U) \cup (S \cap V)$, což jsou otevřené množiny a $S \cap U$ neprázdná, tedy $S \cap V = \O$, jelikož $S$ je souvislá. Tudíž $S \subseteq U$. Celkově $\bigcup ©S \subseteq U$. Tedy $V = \O$.
        \end{dukazin}
    \end{veta}

    \begin{dusledek}
        Je-li $A$ souvislá podmnožina MP $(®X, \rho)$, pak každá množina $M$, splňující $A \subseteq M \subseteq \overline{A}$, je souvislá.

        \begin{dukazin}
            Ať $x \in \overline{A} \setminus A$. Ukážeme, že $A \cup \{x\}$ je souvislá: Ať $A \cup \{x\} = U \cup V$, kde $U, V$ jsou disjunktní otevřené v $A \cup \{x\}$. BÚNO $x \in U$. $U$ otevřená, $x \in \overline{A}$, tedy $A \cap U ≠ \O$.

            $A = (A \cap V) \cup (A \cap U)$, kde $A \cap U ≠ \O$ a $A \cap V = V$ jsou otevřené množiny, tedy $V = \O$. Tedy $A \cup \{x\}$ je souvislá.

            Ať $M$ splňuje předpoklady, $©S := \{A \cup \{x\} | x \in M\}$. $\bigcup ©S = M$, $\bigcap ©S = A ≠ \O$, tedy $M$ je souvislá.
        \end{dukazin}
    \end{dusledek}

    \begin{veta}
        Obraz souvislého MP při spojitém zobrazení je souvislý prostor.

        \begin{dukazin}
            Ať $f: (®X, \rho) \rightarrow (®Y, \sigma)$ na a $(®X, \rho)$ souvislý. Chceme, že ®Y je souvislý. Ať $g: ®Y \rightarrow [0, 1]$ je spojitá a $g(®Y) \supseteq \{0, 1\}$. Chceme $g$ je na. Ale $h := g \circ f$ je spojité a zjevně na $[0, 1]$ podle charakterizace souvislosti, tedy podle char. souv. $g$ je na.
        \end{dukazin}
    \end{veta}

    \begin{definice}
        Ať $(®X, \rho)$ je MP. Množina $M \subseteq ®X$ se nazývá oblouk, pokud $M$ je homeomorfní s $[0, 1]$.

        Body $h(0), h(1)$ se nazývají krajní body oblouku $M$. 

        Prostor $(®X, \rho)$ se nazývá obloukově souvislý, pokud $\forall x, y \in ®X, x ≠ y$ existuje $M \subseteq ®X$ oblouk s krajními body $x$ a $y$.
    \end{definice}

    \begin{dusledek}
        Každý obloukově souvislý prostor ®X je souvislý.

        \begin{dukazin}
            $\forall a, b \in X$, $S(a, b) = \{a\}$, pokud $a = b$, nebo $S(a, b) = $ oblouk s krajními body $a, b$. Z předchozích vět je tak ®X souvislý.
        \end{dukazin}
    \end{dusledek}

    \begin{definice}
        Ať $(®X, \rho)$ je MP. Množina $C_x$ se nazývá komponenta (souvislosti) bodu $x$ v prostoru ®X, pokud $C_x$ je největší souvislá podmnožina ®X obsahující $x$.

        \begin{dukazin}[Korektnost „největší“]
            Z předchozích vět víme, že pokud vezmeme souvislé množiny obsahující bod $x$, tak jejich sjednocení je souvislé.
        \end{dukazin}
    \end{definice}

    \begin{poznamka}
        $\{C_x | x \in ®X\}$ tvoří rozklad MP $(®X, \rho)$.
    \end{poznamka}

    \begin{definice}
        Souvislý kompaktní MP se nazývá kontinuum.
    \end{definice}

    \begin{poznamka}
        Kontinuum je topologický pojem.

        Spojitý obraz kontinua je kontinuum.
    \end{poznamka}

    \begin{veta}
        Ať $K_n$ je kontinuum pro každé $n \in ®N$.
        $$ K_1 \supseteq K_2 \supseteq …  $$
        Pak $\bigcap_{n=1}^∞ K_n$ je kontinuum.

        \begin{dukazin}
            Každé $K_n$ je kompaktní, tedy uzavřené v $K_1$. $\bigcap_{i=1}^∞ K_i$ je uzavřená v $K_1$, $K_1$ kompaktní, tedy $\bigcap_{i=1}^∞ K_i$ je kompaktní.

            Souvislost $K:= \bigcup_{i = 1}^∞ K_i$: Ať pro spor $K$ není souvislé: $K = A \cup B$, $A, B$ jsou otevřené v $K$, $A ≠ \O ≠ B$, $A \cap B = \O$. $\exists U, v$ otevřené disjunktní v $K_1$, že $U \cap K = A$, $V \cap K = B$. $K \subseteq U \cup V$ otevřené v $K_1$. Tvrdíme, že existuje $n \in ®N: K_n \subseteq U \cup V$. To dokážeme sporem: kdyby ne, $K_n \setminus (U \cup V) ≠ \O, n \in ®N$. $F_n := K_n \setminus (U \cup V)$ kompaktní neprázdný, $F_{n+1} \subseteq F_1$. Pak $\bigcap_{n=1}^∞ F_n ≠ \O$, protože $\{F_n | n \in ®N\}$ má konečnou průnikovou vlastnost. $x \in \bigcap F_n \subseteq \bigcap K_n$, $x \in K_n \setminus (U \cup V)$, Spor s $K \subseteq U \cup V$.

            Tedy $K_n = (K_n \cap U) \cup (K_n \cap V)$, což jsou neprázdné otevřené (v $K_n$) disjunktní množiny. \lightning.
        \end{dukazin}
    \end{veta}

% 4. 6. 2021

\section{Hausdorffova metrika}
    \begin{poznamka}[Značení]
        Ať ®X je MP. Označíme
        $$ \kappa(X) = \{K \subseteq ®X | K ≠ \O, K \text{ kompaktní}\}. $$
    \end{poznamka}

    \begin{definice}
        Pro MP $(®X, \rho)$ definujeme $\rho_H: \kappa(®X)\times \kappa(®X) \rightarrow ®R$,
        $$ \rho_H(A, B) = \max\{\sup_{x \in A}\dist_\rho(x, B), \sup_{x \in B}\dist_\rho(x, A)\}. $$
        $\rho_H$ se nazývá Hausdorffova metrika, $(\kappa(®X), \rho_H)$ se nazývá hyperprostor.
    \end{definice}

    \begin{veta}
        $\rho_H$ je metrika na $\kappa(®X)$.

        \begin{dukazin}
            Ať $A \in \kappa(®X): \rho_H(A, A) = 0$.

            Ať $A, B \in \kappa(®X)$, $A ≠ B$, $\rho_H(A, B) \overset{?}{>} 0$. $(A \setminus B) \cup (B \setminus A) ≠ \O$. Ať $x \in (A \setminus B) \cup (B \setminus A)$. $\rho(x, A) > 0$ nebo $\rho(x, B) > 0$. Tedy $\rho_H(A, B) > 0$.

            Symetrie je triviální.

            Pomocné tvrzení: Ať $C \in \kappa(®X)$, $x, y \in ®X$. Pak $\rho(x, C) ≤ \rho(x, y) + \rho(y, C)$. Důkaz: Je-li $c \in C: \rho(x, c) ≤ \rho(x, y) + \rho(y, c)$. Přejdeme k infimu: $\rho(x, C) ≤ \rho(x, y) + \rho(y, C)$.

            Ať $A, B, C \in \kappa(®X)$. Pro $a \in A$, $b \in B$: $\rho(a, C) ≤ \rho(a, b) + \rho(b, C) ≤ \rho(a, b) + \rho_H(B, C)$. Přejdeme k infimu:
            $$ \rho(a, C) ≤ \rho(a, B) + \rho_H(B, C) ≤ \rho_H(A, B) + \rho_H(B, C). $$
            To platí pro všechna $a \in A$, tedy přejdeme k supremu:
            $$ \sup_{a \in A} \rho(a, C) ≤ \rho_H(A, B) + \rho_H(B, C). $$
            Analogicky $\sup_{c \in C} \rho(c, A) ≤ …$. Tedy $\rho_H(A, C) ≤ \rho_H(A, B) + \rho_H(B, C)$.
        \end{dukazin}
    \end{veta}

    \begin{tvrzeni}
        Ať $(®X, \rho)$ je MP, pak zobrazení $f: (®X, \rho) \rightarrow (\kappa(®X), \rho_H)$, $f(x) = \{x\}$ je izometrické vnoření na uzavřenou podmnožinu $\kappa(®X)$.

        \begin{dukazin}
            $\rho_H(f(x), f(y)) = \rho(x, y)$ triviálně.

            $M := \{\{x\} | x \in ®X\}$ je uzavřená v $\kappa(®X)$: Doplněk je otevřená: Ať $A \in \kappa(®X)$, $|A| ≥ 2$ (tj. $A$ je v doplňku $M$). Fixujeme $x, y \in A, x ≠ y: \rho(x, y) =: \epsilon > 0$. $\B_{\rho_H}(A, \frac{\epsilon}{3}) \cap M = \O$ sporem: Kdyby existovalo $z \in ®X: \{z\} \in B_{\rho_H}(A, \frac{\epsilon}{3})$, pak $\rho_H(\{z\}, A) < \frac{\epsilon}{3}$, $\rho(x, z) < \frac{\epsilon}{3}$, $\rho(y, z) < \frac{\epsilon}{3}$.
            $$ \epsilon = \rho(x, y) ≤ \rho(x, z) + \rho(y, z) < \frac{2}{3} \epsilon. $$
            \lightning.
        \end{dukazin}
    \end{tvrzeni}

    \begin{veta}
        Pro MP $(®X, \rho)$ platí:

        \begin{enumerate}
            \item $(®X, \rho)$ je úplný $\Leftrightarrow$ $(\kappa(®X), \rho_h)$ je úplný.
            \item $(®X, \rho)$ je totálně omezený $\Leftrightarrow$ $(\kappa(®X), \rho_H)$ je totálně omezený.
            \item $(®X, \rho)$ je kompaktní $\Leftrightarrow$ $(\kappa(®X), \rho_H)$ je kompaktní.
            \item $(®X, \rho)$ je kontinuum $\Leftrightarrow$ $(\kappa(®X), \rho_H)$ je kontinuum.
        \end{enumerate}

        \begin{dukazin}[Pouze náznak]
                1), 2) $\leftarrow$: Z předchozího je ®X uzavřený podprostor úplného (tot. omezeného) prostoru, tedy úplný (tot. omezený) prostor.

                1) $\implies$: Předpokládejme, že $(A_n)$ je cauchyovská v $(\kappa(®X), \rho_H)$. $B_n := \overline{\bigcup_{k ≥ n} A_k}$. $B := \bigcap B_n$. Lze ukázat, že $B \in \kappa(®X)$, $A_n \rightarrow B$.

                2) $\implies$: Ať $\epsilon > 0$. Ať $A_0$ je konečná $\epsilon$-síť v $(®X, \rho)$. $©P(A_0)\setminus \{\O\}$ (je konečná a) je $\epsilon$-síť v $(\kappa(®X), \rho_H)$.

                3) plyne z 1) a 2). 4) bez náznaku.
        \end{dukazin}
    \end{veta}

    \subsection{Iterované funkční systémy}
        \begin{definice}
            Ať $(®X, \rho)$, $(®Y, \sigma)$ jsou MP, $f: ®X \rightarrow ®Y$ se nazývá kontrakce, pokud existuje $K < 1$, že $\forall x, y \in ®X: \sigma(f(x), f(y)) ≤ K·\rho(x, y)$.
        \end{definice}

        \begin{veta}[Banachova věta]
            Ať $(®X, \rho)$ je neprázdný úplný MP, $f: ®X \rightarrow ®X$ kontrakce. Pak $f$ má právě jeden pevný bod, tj. bod $x \in ®X: f(x) = x$.

            \begin{dukazin}
                Ať $x_0 \in ®X$ libovolně. $x_1 = f(x_0), x_2 = f(x_1), …$. Posloupnost $(x_n)_{n=1}^∞$ je cauchyovská:
                $$ \rho(x_{n+1}, x_n) ≤ K·\rho(x_n, x_{n-1}) ≤ … ≤ K^n·\rho(x_1, x_0). $$
                $\sum_{n=1}^∞ K^n·\rho(x_1, x_0)$ konverguje. Tedy pro $\epsilon > 0\ \exists n_0 \in ®N: \sum_{n=n_0}^∞ K^n·\rho(x_1, x_0) < \epsilon$. Tedy $m ≥ n > n_0$:
                $$ \rho(x_m, x_n) ≤ \rho(x_m, x_{m-1}) + \rho(x_{m-1}, x_{m-2}) + … + \rho(x_{n+1}, x_n) ≤ \sum_{i=n}^m K^i \rho(x_0, x_1) ≤ \epsilon. $$ 
                $(®X, \rho)$ je úplný, $(x_n)$ cauchyovská, tedy $\lim_{n \rightarrow ∞} x_n = x \in ®X$. Tvrdíme, že $f(x) = x$:
                $$ f(x) = f(*\lim_{n \rightarrow ∞} x_n) \overset{\text{$f$ spojitá}}{=} \lim_{n \rightarrow ∞} f(x_n) = \lim_{n \rightarrow ∞} x_{n+1} = x. $$

                Jednoznačnost: Kdyby $f(x) = x$, $f(y) = y$, pak
                $$ \rho(x, y) ≤ K·\rho(f(x), f(y)) = K\rho(x, y) \implies \rho(x, y) = 0 \implies x = y. $$ 
            \end{dukazin}
        \end{veta}

        \begin{definice}
            Ať $(®X, \rho)$ je MP, $f_1, …, f_n: ®X \rightarrow ®X$ kontrakce, $n \in ®N$. Pak $\{f_1, …, f_n\}$ se nazývá iterovaný funkční systém. Množina $K \subseteq ®X$ se nazývá invariantní (vůči tomuto IFS), pokud $K = f_1(K) \cup … \cup f_n(K)$.
        \end{definice}

        \begin{veta}
            Ať $(®X, \rho)$ je úplný neprázdný MP a $\{f_1, …, f_n\}$ IFS na ®X. Pak existuje jediná kompaktní neprázdná invariantní množina pro tento IFS.

            \begin{dukazin}
                Víme $(\kappa(®X), \rho_H)$ je úplný. Definujeme $F: \kappa(®X) \rightarrow \kappa(®X)$, $F(L):=f_1(L) \cup … \cup f_n(L) \in \kappa(®X)$. $F$ je kontrakce. Podle Banachovy věty existuje právě jedno $K \in \kappa(®X): F(K) = K$. Tím je důkaz hotov.
            \end{dukazin}
        \end{veta}
\end{document}
