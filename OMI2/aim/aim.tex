\documentclass[12pt]{article}                   % Začátek dokumentu
\usepackage{../../MFFStyle}                     % Import stylu

\begin{document}

% 9. 3. 2021
\section{Úvod}
    \begin{poznamka}
        Probírali jsme C (typy, literály, operátory, konverze, )
    \end{poznamka}

    \begin{upozorneni}
        Proměnné mohou být uložené s paddingem (když vyžadují zarovnání (\verb|_Alignof(typ)|), např. ve struktuře s char a int se mezi char a int vloží 3 byty). Také nejsou všechny kombinace funkční (např. u float), tedy není dobré jen tak přistupovat na náhodná místa.

        Padding není v poli! A na začátku (a pouze tam) struktur.

        Navíc ve struktuře je zaručeno pořadí, takže char int char zabere 12 bytů (na konec se přidává padding tak, aby zarovnání fungovalo i v poli takových struktur).
    \end{upozorneni}

    \begin{upozorneni}
        „0číslo“ je číslo v oktanové (osmičkové) soustavě.

        V 16 soustavě se používá pro exponent znak \verb|p|, ne \verb|e|.

        Dva řetězce s mezerou uprostřed se spojí.
    \end{upozorneni}

    \begin{upozorneni}
        Na floatech rovná se moc nefunguje. Např. NAN (not a number) není roven ničemu, ani sám sobě.
    \end{upozorneni}

    \begin{upozorneni}
        Menší typy než int se při výpočtu a předávání do funkcí bez definovaných parametrů převádějí na int.
    \end{upozorneni}

\end{document}
