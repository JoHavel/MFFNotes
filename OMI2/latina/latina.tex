\documentclass[12pt]{article}					% Začátek dokumentu
\usepackage{../../MFFStyle}					    % Import stylu



\begin{document}

% 16. 2. 2021
ablativ zřetelný = se zřetelem na (chudí \emph{duchem})

Substantiva 5. deklinace mají v nominativu vždy -és (ale existují i slova jiné deklinace s -és v nominativu) a nejsou to žádná neutra a pouze 1,5 masculina: merídiés (poledne, vždy v masculinu) a diés (den, v masculinu je, když je to abstraktní pojem, ve feminimu je to konkrétní den).

% 23. 2. 2021
superlativ je od slovesa superferró? = vynáším. Komparativ od adjektiv vzniká přidáním -ior (resp. -ius pro neutra) a pak se skloňuje podle 3. deklinace (-ióris). Kromě srovnávání (kromě než se ale často používá i ablativ srovnávací) se komparativ v latině používá k vyjádření větší míry než je běžná (překládá se pak jako poněkud / příliš / trochu / … + pozitiv).

Na rozdíl od komparativu se superlativ (skloňující se podle servus, femina a verbum) tvoří různě u různých adjektiv: u 6 adjektiv -lis (facilis, difficilis, similis, dissimilis, humilis, gracilis) se tvoří příponami -limus, -lima, -limum, u jiných -lis se tvoří -issimus / a / um stejně jako ostatní, krom končících v masculinu na -er, kde se tvoří z genitivu pomocí -(er)rimus / a / um.

Superlativ se používá s genitivem celkovým nebo s dé / é / éx + ablativ celku nebo s inter + akuzativ. Samozřejmě se může použít samostatně (pak se nazývá elativ) a překládáme ho pozitivem s velmi / velice / nesmírně nebo pozitivem s předponou pře-.

% 2. 3. 2021
Římané na rozdíl od Řeků neměli rádi více samohlásek za sebou, tedy pokud adjektiva mají před -us, -a, -um samohlásku, tak se stupňuje opisně pomocí magis a maximé, které předcházejí pozitivu.

Dále existuje 5 častých nepravidelně stupňovaných adjektiv: (bonus, melior, optimus), (malus, peior, pessimus), (magnus, maior, maximus), (parvus, minor, minimum), (multí, plúrés, plúrimí) a navazující (multum, plús, plúrimum) a (?, mínús, ?).

Následuje lekce 8, kde si můžeme vytvořit tahák (ručně psaný) na zápočtový test.

% 9. 3. 2021
Latina nemá osobní zájmeno k 3. osobě, používá ukazovací. Quis a quid (kdo a co) jsou zájmena tzv. substantivně platná, tj. odpovědí je většinou substantivum. Quí, quae a quod (jaký, jaká a jaké) jsou adjektivně platná, tj. odpovědí je zpravidla adjektivum.

S číslovek základních se skloňují první tři, stovky a tisíc. Řadové se skloňují všechny, ale podle prvních 2 deklinací (jsou trojvýchodná). U dvojky (duo, duae, duo) lze v akuzativu potkat duál (stejně jako plurál / singulár byl kdysi součástí jazyků). Stovky se skloňují podle první a druhé deklinace. Tisíce se skloňují pouze v množném čísle a skloňují se jako neutra podle 3. deklinace.

% 16. 3. 2021

% 23. 3. 2021

\section{Konjunktiv}
Od slova spojovat, vznikl jako tvar na spojování vedlejších vět a stále se tak hojně (povinně ve 2 typech vedlejších vět) používá (tzv. vztažné použití konjunktivu). Máme 4 (prézenta, imperfekta, perfekta, plusquamperfekta). Zároveň se často vyskytuje tam, kde není povinně, ale vyjadřuje něco speciálního (samostatné užití konjunktivu) (co nevyjádří indikativ, indikativem se určují děje, které se reálně odehrávají, avšak potřebujeme i další děje (pojďte sem, pracujme, nedělejme to, byli bychom tam šli, máme tam chodit, kéž tam jde, …)).

Konjunktiv prézentu se tvoří z indikativu prézenta (1. a 3. (legó) bez kmenových samohlásek, zbytek s nimi) přidáním 'kmenové' samohlásky é (v 1. konjugaci) a á (v ostatních). Následují běžné osobní koncovky. Sloveso esse je tvaru sim, sís, …

Konjunktiv prézentu vyjadřuje například optimistické přání splnitelné aktuálně nebo v blízké době (překládá se s částicí ať). Zesiluje částicí utinam (kéž) (ale na rozdíl od ostatních ji nepotřebuje). Také se jím dá rozkazovat nebo pobízet (něco mezi rozkazem a přáním). Ještě umí vyjádření možnosti v přítomnosti (dal bych si kafe, zapsal bych si to, …). Nakonec umí vyjádřit rozvažování (vždy v otázce, v první osobě).

Konjunktiv imperfekta naopak s částicí utinam vyjadřuje pesimistické přání v přítomnosti (vnímáme, že nebude splněno, např. kéž bych byl jinde, kéž bychom měli více času). Také vyjadřuje děj, co nenastává (spal bych (ale nespím), šel bych tam (ale nejdu)). Rozvažování (v otázce, v první osobě) nad minulostí (co jsem měl dělat).

% 30. 3. 2021
Konjunktiv je povinný ve vedlejší větě účelové (ut = aby, né = aby ne, né quis = aby nikdo, …), do češtiny se překládá podmiňovacím způsobem. Pro to, aby jsme věděli, kde je konjunktiv přítomný a kde minulý, zavádíme vedlejší (popisuje děj související s minulostí) a hlavní čas (slovesný tvar, který nepopisuje děj související s minulostí). Pokud je ve větě hlavní hlavní čas, pak je vedlejší věta účelová v konjunktivu prézenta, po vedlejším čase konjunktiv imperfekta.

Další typ vedlejších vět (také uvozené ut / né, také se klade imperfektum a prézentum podle stejného pravidla), obsahové věty žádací (resp. první skupina -- věty snahové, v dalším semestru budeme brát ještě další 2 skupiny). U nás jsou pod předmětnými (snažím se o něco, prosím o něco, přeji o něco, připouštím něco, …)

% 13. 4. 2021
\section{Perfektní kmen pravidelných sloves 1.-4. konjugace a slovesa esse}
První dva členy slovesné stupnice jsou prézentní kmen (1. je bez koncovky, 2. je s koncovkou). Třetí a čtvrtý člen nemusí být (podle toho, co je schopno sloveso tvořit), třetí končí vždy na -í (1. os. sg. ind. perfekta akt.), po jeho odtrhnutí dostaneme perfektní kmen, který určuje tvary perfekta, plusquamperfekta a futura II. Čtvrtý člen je supinum a končí na -um, určuje supinový člen (odtrhne se -um a většinou je tak zakončen t, méně často s a poměrně výjimečně x, jinak ne). Od supina se odvozují pasivní tvary perfekta, plusquamperfekta a futura II.

Indikativ perfekta se překládá v minulém čase a dokonavě (je-li to možné).

\newpage
\section{Slovníček}
    lacus, ús, m. = jezero\\
    manus, ús, f. = ruka\\
    exercitus, ús, m. = vojsko\\
    cornú, ús, n. = roh, křídlo (vojska), paroh\\
    genu, ús, n. = koleno, pokolení\\
    gelu, ús, n. = led\\
    veru, ús, n. = rožeň\\
    rés, reí, f. = věc\\
    diés, diéí, f. m. = den\\
    faciés, faciéí, f. = tvář\\
    spés, eí, f. = naděje\\
    gracias agó = děkuji\\

    

    


\end{document}
