\documentclass[12pt]{article}					% Začátek dokumentu
\usepackage{../../MFFStyle}					    % Import stylu



\begin{document}

% 16. 2. 2021
ablativ zřetelný = se zřetelem na (chudí \emph{duchem})

Substantiva 5. deklinace mají v nominativu vždy -és (ale existují i slova jiné deklinace s -és v nominativu) a nejsou to žádná neutra a pouze 1,5 masculina: merídiés (poledne, vždy v masculinu) a diés (den, v masculinu je, když je to abstraktní pojem, ve feminimu je to konkrétní den).

% 23. 2. 2021
superlativ je od slovesa superferró? = vynáším. Komparativ od adjektiv vzniká přidáním -ior (resp. -ius pro neutra) a pak se skloňuje podle 3. deklinace (-ióris). Kromě srovnávání (kromě než se ale často používá i ablativ srovnávací) se komparativ v latině používá k vyjádření větší míry než je běžná (překládá se pak jako poněkud / příliš / trochu / … + pozitiv).

Na rozdíl od komparativu se superlativ (skloňující se podle servus, femina a verbum) tvoří různě u různých adjektiv: u 6 adjektiv -lis (facilis, difficilis, similis, dissimilis, humilis, gracilis) se tvoří příponami -limus, -lima, -limum, u jiných -lis se tvoří -issimus / a / um stejně jako ostatní, krom končících v masculinu na -er, kde se tvoří z genitivu pomocí -(er)rimus / a / um.

Superlativ se používá s genitivem celkovým nebo s dé / é / éx + ablativ celku nebo s inter + akuzativ. Samozřejmě se může použít samostatně (pak se nazývá elativ) a překládáme ho pozitivem s velmi / velice / nesmírně nebo pozitivem s předponou pře-.

% 2. 3. 2021
Římané na rozdíl od Řeků neměli rádi více samohlásek za sebou, tedy pokud adjektiva mají před -us, -a, -um samohlásku, tak se stupňuje opisně pomocí magis a maximé, které předcházejí pozitivu.

Dále existuje 5 častých nepravidelně stupňovaných adjektiv: (bonus, melior, optimus), (malus, peior, pessimus), (magnus, maior, maximus), (parvus, minor, minimum), (multí, plúrés, plúrimí) a navazující (multum, plús, plúrimum) a (?, mínús, ?).

Následuje lekce 8, kde si můžeme vytvořit tahák (ručně psaný) na zápočtový test.

% 9. 3. 2021
Latina nemá osobní zájmeno k 3. osobě, používá ukazovací. Quis a quid (kdo a co) jsou zájmena tzv. substantivně platná, tj. odpovědí je většinou substantivum. Quí, quae a quod (jaký, jaká a jaké) jsou adjektivně platná, tj. odpovědí je zpravidla adjektivum.

S číslovek základních se skloňují první tři, stovky a tisíc. Řadové se skloňují všechny, ale podle prvních 2 deklinací (jsou trojvýchodná). U dvojky (duo, duae, duo) lze v akuzativu potkat duál (stejně jako plurál / singulár byl kdysi součástí jazyků). Stovky se skloňují podle první a druhé deklinace. Tisíce se skloňují pouze v množném čísle a skloňují se jako neutra podle 3. deklinace.

% 16. 3. 2021

\newpage
\section{Slovníček}
    lacus, ús, m. = jezero\\
    manus, ús, f. = ruka\\
    exercitus, ús, m. = vojsko\\
    cornú, ús, n. = roh, křídlo (vojska), paroh\\
    genu, ús, n. = koleno, pokolení\\
    gelu, ús, n. = led\\
    veru, ús, n. = rožeň\\
    rés, reí, f. = věc\\
    diés, diéí, f. m. = den\\
    faciés, faciéí, f. = tvář\\
    spés, eí, f. = naděje\\

    

    


\end{document}
