\documentclass[12pt]{article}                   % Začátek dokumentu
\usepackage{../../MFFStyle}                     % Import stylu

\begin{document}
\section{Úvod}
    \begin{poznamka}[Historie]
        \ 
        \begin{itemize}
            \item První formalizace pojmu algoritmus -- Ada, Countess of Lovelace 1852.
            \item Intenzivnější vývoj s rozvojem počítačů ve 2. čtvrtině 20. století.
            \item Co stroje umí a co ne? -- Church, Turing, Kleene (konečné automaty / neuronové sítě), Post, Markov, Chomsky (zásobníkové automaty a formální teorie konečných automatů, zkoumal Angličtinu).
        \end{itemize}
    \end{poznamka}

    \begin{poznamka}[Cíl]
        Osvojit si abstraktní model počítače, vnímat jak drobné změny v definici vedou k velmi rozdílným důsledkům. Zažít skutečnost alg. nerozhodnutelných problémů a připravit se na přednášku o složitosti a NP-úplnosti.
    \end{poznamka}

    \begin{poznamka}[Praktické využití]
        Korektnost algoritmů, zpracování přirozeného jazyka, lexikální a syntaktická analýza v překladačích. Návrh, popis a verifikace hardwaru (automaty, integrované obvody, stroje). Vyhledávání v textu atd.
    \end{poznamka}

\section{}
    \begin{definice}[Deterministický konečný automat (DFA)]
        Deterministický konečný automat $A = (Q, \Sigma, \delta, q_0, F)$ sestává z: konečné množiny stavů ($Q$), konečné neprázdné množiny vstupních symbolů (abecedy, $\Sigma$), přechodové funkce, tj. zobrazení $Q \times \Sigma \rightarrow Q$ (značí se hranami grafu, $\delta$), počátečního stavu (vede do něj šipka 'odnikud', $q_0 \in Q$) a neprázdné množiny (přijímajících) stavů (značí se dvojitým kruhem / šipku 'ven', $F \subseteq Q$).
    
        \begin{umluvain}
            Přidáváme 0-2 stavy: fail (pokud je nějaký přechod nedefinován, vede sem a všechno z fail vede do fail) a final (pokud je $F$ prázdné, všechny šipky z něj vedou zpět do něj).
        \end{umluvain}
    \end{definice}

    \begin{definice}[Slovo, jazyk]
        Mějme neprázdnou množinu symbolů $\Sigma$. Slovo je konečná (i prázdná) posloupnost symbolů $s \in \Sigma$, prázdné slovo se značí $\lambda$ nebo $\epsilon$.

        Množinu všech slov v abecedě $\Sigma$ značíme $\Sigma^*$ a množinu všech neprázdných $\Sigma^+$.

        Jazyk $L \subseteq \Sigma^*$ je množina slov v abecedě $\Sigma$.
    \end{definice}

    \begin{definice}[Operace: zřetězení, mocnina, délka slova]
        Nad slovy $\Sigma^*$ definujeme operace: Zřetězení slov $u.v$ nebo $uv$, mocnina (počet opakování) $u^n$ ($u^0 = \lambda$, $u^1 = u$, $u^{n+1} = u^n.u$), délka slova $|u|$ ($|\lambda| = 0$, $|auto| = 4$), počet výskytů $s \in \Sigma$ ve slově $u$ značíme $|u|_s$ ($|zmrzlina|_z = 2$).
    \end{definice}

    \begin{definice}[Rozšířená přechodová funkce]
        Mějme přechodovou funkci $\delta: Q \times \Sigma \rightarrow Q$. Rozšířenou přechodovou funkci $\delta^*: Q \times \Sigma^* \rightarrow Q$ (tranzitivní uzávěr $\delta$) definujeme induktivně: $\delta^*(q, \lambda) = q$ a $\delta^*(q, wx) = \delta(\delta^*(q, w), x)$ pro $x \in \Sigma$ a $w \in \Sigma^*$.
    \end{definice}

    \begin{definice}[Jazyky rozpoznatelné konečnými automaty, regulární jazyky]
        Jazykem rozpoznávaným (akceptovaným, přijímaným) konečným automatem $A$ nazveme jazyk $L(A) = \{w | w \in \Sigma^* \land \delta^*(q_0, w) \in F\}$.

        Jazyk je rozpoznatelný konečným automatem, jestliže existuje konečný automat $A$ takový, že $L = L(A)$.

        Třídu jazyků rozpoznatelných konečnými automaty označíme $©F$ a nazveme ji regulární jazyky.
    \end{definice}

    \begin{veta}[!Iterační (pumpin) lemma pro regulární jazyky]
        Mějme regulární jazyk $L$. Pak existuje konstanta $n \in ®N$ (závislá na $L$) tak, že každé $w \in L; |w| ≥ n$ můžeme rozdělit na tři části, $w = xyz$, že $y ≠ \lambda \land |xy| ≤ n \land \forall k \in ®N_0$, slovo $xy^kz$ je také v $L$.

        \begin{dukazin}
            Mějme regulární jazyk $L$, pak existuje DFA $A$ s $n$ stavy, že $L = L(A)$. Vezmeme libovolné slovo $a_1a_2…a_n…a_m = w \in L$ délky $m ≥ n$, $a_i \in \Sigma$. Následně definujeme $\forall i: p_i = \delta^*(q_0, a_1a_2…a_i)$. Platí $p_0 = q_0$. Z Dirichletova principu se některý stav opakuje. Vezmeme první takový, tj. $(\exists i, j)(0 ≤ i < j ≤ n \land p_i = p_j)$.

            Definujeme $x = a_1a_2…a_i$, $y = a_{i+1}a_{i+2}…a_j$ a $z = a_{j+1}a_{j+2}…a_m$, tj. $w = xyz$, $y ≠ \lambda$, $|xy| ≤ n$.
        \end{dukazin}
    \end{veta}

% 9. 3. 2021

    \begin{definice}[Kongruence, konečný index]
        Mějme konečnou abecedu $\Sigma$ a relaci ekvivalence $\sim$ na $\Sigma^*$. Potom $\sim$ je pravá kongruence, jestliže $\forall u, v, w \in \Sigma^*: u \sim v \implies uw \sim vw$. $\sim$ je konečného indexu, jestliže rozklad $\Sigma^* / \sim$ má konečný počet tříd.

        Třídu kongruence $\sim$ obsahující slovo $u$ značíme $[u]_\sim$, resp. $[u]$.
    \end{definice}

    \begin{veta}[Myhill-Nerodova věta]
        Nechť $L$ je jazyk nad konečnou abecedou $\Sigma$. Potom $L$ je rozpoznatelný konečným automatem právě tehdy, když existuje pravá kongruence $\sim$ konečného indexu nad $\Sigma^*$ tak, že $L$ je sjednocením jistých tříd rozkladu $\Sigma^* / \sim$.

        \begin{dukazin}
            $\implies$ definujeme $u \sim v ≡ \delta^*(q_0, u) = \delta^*(q_0, v)$. Zřejmě je to ekvivalence. Je to pravá kongruence (z definice $\delta^*$) a má konečný index (jelikož automat má konečně mnoho stavů).
            $$ L = \{w|\delta^*(q_0, w) \in F\} = \bigcup_{q \in F} [w | \delta^*(q_0, w) = q]_\sim. $$ 

            $\Rightarrow$ abeceda automatu bude $\Sigma$. Stavy budou třídy rozkladu $\Sigma^* / \sim$. Počáteční stav je $q_0 = [\lambda]_\sim$. Koncové stavy $F = \{c_1, …, c_n\}$, kde $L = \bigcup_{i \in [n]} c_i$. Přechodová funkce $\delta([u], x) = [ux]$ (korektní z definice pravé kongruence).
        \end{dukazin}
    \end{veta}

    \begin{priklad}
        $L = \{u | u = a^+b^ic^i \land u = b^ic^j \land i, j \in ®N_0\}$ není regulární, ale vždy lze pumpovat první písmeno.

        \begin{dukazin}[Sporem]
            Předpokládejme, že $L$ je regulární. Pak existuje pravá kongruence $\sim$ konečného indexu $m$, $L$ je sjednocení některých tříd $\Sigma^* / \sim$. Vezmeme množinu slov $S = \{ab, abb, abbb, …, ab^{m+1}\}$. Existují dvě slova (Dirichletův princip) $i ≠ j$, která padnou do stejné třídy. $ab^i \sim ab^j \Leftrightarrow ab^ic^i \sim ab^jc^i$, ale $ab^ic^i \in L \land ab^jc^i \notin L$. \lightning.
        \end{dukazin}
    \end{priklad}

    \begin{definice}[Dosažitelné stavy]
        Mějme DFA $A = (Q, \Sigma, \delta, q_0, F)$ a $q \in Q$. Řekneme, že stav $q$ je dosažitelný, jestliže existuje $w \in \Sigma^*$ takové, že $\delta^*(q_0, w) = q$.

        \begin{poznamka}[Hledání dosažitelných stavů]
            'Hloupé' prohledávání do šířky.
        \end{poznamka}
    \end{definice}

    \begin{definice}[Automatový homomorfismus]
        Nechť $A_1, A_2$ jsou DFA se standardním označením a shodnou abecedou. Řekněme, že zobrazení $h: Q_1 \rightarrow Q_2$ je automatovým homomorfismem, jestliže $h({q_1}_0) = {q_2}_0$, $h(\delta_1(q, x)) = \delta_2(h(q), x)$ a $q \in F_1 \Leftrightarrow h(q) \in F_2$.
    \end{definice}

    \begin{definice}[Ekvivalence automatů]
        Dva konečné automaty nad stejnou abecedou jsou ekvivalentní, jestliže rozpoznávají stejný jazyk.
    \end{definice}

    \begin{veta}[O ekvivalenci automatů]
        Existuje-li homomorfismus konečných automatů, pak jsou tyto automaty ekvivalentní.

        \begin{dukazin}
            Triviální.
        \end{dukazin}
    \end{veta}

    \begin{definice}[Ekvivalence stavů]
        Dva stavy jsou ekvivalentní, pokud pro všechna slova dojdeme z obou stavů buď do nepřijímajících, nebo do přijímajících stavů. Pokud dva stavy nejsou ekvivalentní, říkáme, že jsou rozlišitelné.
    \end{definice}

    \begin{poznamka}[Algoritmus pro nalezení eqvivalentních stavů]
        Vytvořím tabulku dvojic stavů a zaškrtám zřejmě rozlišitelné dvojice (přijímající + nepřijímající). Potom pro každou dvojici zkusím všechna písmena a pokud nějaké z nich posune ze stavů do rozlišitelné, pak i tato dvojice je rozlišitelná. Opakuji, dokud se něco mění.
    \end{poznamka}

    \begin{definice}[Redukovaný DFA, redukt]
        DFA je redukovaný, pokud nemá nedosažitelné stavy a žádné dva stavy nejsou ekvivalentní. DFA $B$ je reduktem $A$, jestliže $B$ je redukovaný a $B$ a $A$ jsou ekvivalentní.
    \end{definice}

% 16. 3. 2021

    \begin{poznamka}[Algoritmus na testování ekvivalence reg. jazyků]
        Najdeme jeden a druhý DFA rozpoznávající jeden a druhý jazyk. BÚNO jsou stavy disjunktní. Vytvoříme DFA sjednocením (za počáteční stav vezmeme libovolný z 2 počátečních stavů našich DFA). Potom jsou jazyky ekvivalentní, když jsou ekvivalentní počáteční stavy našich DFA.
    \end{poznamka}
    
\section{NFA}

    \begin{definice}[Nederministický konečný automat (NFA)]
        NFA je DFA, kde přechodová funkce je funkce do potenční množiny stavů. A počáteční stav může být také množina, ale existují obě alternativy.
    \end{definice}

    \begin{definice}[Rozšířená přechodová funkce]
        Pro přechodovou funkci $\delta$ NFA je rozšířená přechodová funkce $\delta^*: Q\times \Sigma^* \rightarrow ©P(Q)$ definovaná indukcí: $\delta^*(q, \lambda) = \{q\}$ a $\delta^*(q, wx) = \bigcup_{p \in \delta^*(q, w)}\delta(p, x)$.
    \end{definice}

    \begin{definice}[Jazyk přijímaný NFA]
        Mějme NFA $A = (Q, \Sigma, \delta, S_0, F)$, pak $L(A) = \{w|\exists q_0 \in S_0:\delta^*(q_0, w)\cap F ≠ \O\}$ je jazyk přijímaný automatem $A$.
    \end{definice}

    \begin{poznamka}[Algoritmus: podmnožinová konstrukce]
        Začínáme s NFA $N = (Q_N, \Sigma, \delta_N, S_0, F_N)$. Cílem je popis deterministického DFA $D = (Q_D, \Sigma, \delta_D, S_0, F_D)$, pro který $L(N) = L(D)$.

        $Q_D$ je množina podmnožin $Q_N$ ($Q_D = ©P(Q_n)$). Počáteční stav DFA označený $S_0$ je prvek $Q_D$. $F_D = \{S|S \in ©P(Q_n) \land S \cap F_N ≠ \O\}$. Přechodová funkce je ($S \in Q_D$, $a \in \Sigma$):
        $$ \delta_D(S, a) = \bigcup_{p \in S} \delta_N(p, a). $$

        \begin{dukazin}
            Triviální, indukcí dokážeme shodné chování $d*$.
        \end{dukazin}
    \end{poznamka}

    \begin{definice}[$\lambda$-NFA]
        $\lambda$-NFA (NFA s $\lambda$ přechody) je NFA, kde $\delta$ je definována pro $Q \times (\Sigma \cup \{\lambda\})$.
    \end{definice}

    \begin{definice}[$\lambda$-uzávěr]
        Pro $q \in Q$ definujeme $\lambda$-uzávěr stavu $q$ (v těchto poznámkách značeno $\overline{q}$) rekurzivně: $q \in \overline{q}$. Je-li $p \in \overline{q}$ a $r \in \delta(p, \lambda)$, pak i $r \in \overline{q}$.

        Pro $S \subseteq Q$ definujeme $\overline{S} = \bigcup_{q \in S} \overline{q}$.
    \end{definice}

    \begin{definice}[Rozšířená přechodová funkce]
        $\delta^*(q, \lambda) = \overline{q}$. $\delta^*(q, wa) = \overline{\bigcup_{\delta^*(q, w)} \delta(p, a)}$.
    \end{definice}

    \begin{veta}
        Jazyk je rozpoznatelný $\lambda$-NFA $\Leftrightarrow$ $L$ regulární.

        \begin{dukazin}
            $\Leftarrow$: triviální. $\implies$: přes podmnožinovou konstrukci.
        \end{dukazin}
    \end{veta}

\section{Množinové operace nad jazyky}
    \begin{definice}[Množinové operace nad jazyky]
        Mějme jazyky $L, M$. Definujeme následující operace:

        \begin{itemize}
            \item binární (konečné) sjednocení $L \cup M = \{w|w \in L \lor w \in M\}$,
            \item průnik $L \cap M = \{w|w \in L \land w \in M\}$,
            \item rozdíl $L - M = \{w | w \in L \land w \notin M\}$,
            \item doplněk (komplement) $\overline{L} = -L = \{w|w\notin L\} = \Sigma^* - L$.
        \end{itemize}
    \end{definice}

    \begin{veta}[Uzavřenost na množinové operace]
        Regulární jazyky jsou uzavřené na 4 operace výše.

        \begin{dukazin}
            Doplněk: doplníme všechny přechody (doplníme FAIL stav). Potom prohodíme přijímající a nepřijímající stavy.

            Průnik sjednoceni a rozdíl přes tzv. součinový automat $(Q_1 \times Q_2, \Sigma, \delta', (q_0, q_1), F)$, kde $\delta'((p_1, p_2), x) = (\delta_1(p_1, x), \delta_2(p_2, x))$ a $F$ je podle toho, zda řešíme průnik, sjednocení nebo rozdíl, $F_1 \times F_2$, $(F_1\times Q_2)\cup(Q_1 \times F_2)$ nebo (po doplnění) $F_1 \times (Q_2 - F_2)$.
        \end{dukazin}
    \end{veta}

    \begin{definice}[Řetězcové operace nad jazyky]
        Mějme jazyky $L, M$. Definujeme následující operace:

        \begin{itemize}
            \item zřetězení $L.M = \{uv|u \in L \land v \in M\}$,
            \item mocninu $L^0 = \{\lambda\}, L^{i+1} = L^i.L$,
            \item pozitivní iteraci $L^+ = \bigcup_{i ≥ 1}L^i$,
            \item obecnou iteraci $L^* = \bigcup_i L^i$,
            \item otočení (zrcadlový obraz, reverze) $L^R = \{u^R|u \in L\}$, $(x_1x_2…x_n)^R = x_n…x_2x_1$,
            \item levý kvocient $M\setminus L = \{v|uv \in L \land u \in M\}$,
            \item levá derivace $\partial_w L = \{w\}\setminus L$,
            \item pravý kvocient $L/M = \{u|uv \in L \land v \in M\}$,
            \item pravá derivace $\partial^R_w L = L/\{w\}$.
        \end{itemize}
    \end{definice}

    \begin{veta}[Uzavřenost regulárních jazyků na řetězcové operace]
        Regulární jazyky jsou uzavřené na 10 operací výše.
    \end{veta}


\end{document}
