\documentclass[12pt]{article}                   % Začátek dokumentu
\usepackage{../../MFFStyle}                     % Import stylu

\begin{document}

    \begin{priklad}[mult]
        Ukažte, dva (velmi podobné) algoritmy, jak je možno násobit $n$ ciferná čísla v čase $O(n^{log_2 3})$.

        \begin{reseni}[První]
            BÚNO je $n$ mocnina dvou (pokud není, tak čísla doplníme na začátku nulami a zvětšíme tak jejich délku maximálně na $2n$). Čísla tedy rozdělíme na poloviny a zapíšeme jako $\overline{AB} = A·2^{n/2} + B$ a $\overline{CD} = C·2^{n/2} + D$. Tedy součin těchto čísel je roven $\overline{AB}·\overline{CD} = A·C·2^n + B·D + (A·D + B·C)·2^{n/2}$.

            Pokud bychom počítali přímo tyto součiny, tak jsme si moc nepomohli, protože pokud budeme počítat $4$krát poloviční vstupy, pak podle 1. příkladu bude složitost $n^{\log_2(4)} = n^2$ (jelikož $4·\(1/2\)^\beta = 1 \implies \beta = \log_{1/2}(1/4) = \log_2(4)$). My si ale můžeme všimnout, že když vezmeme $(B - A)·(c - D) = (A·D + B·C) - A·D - B·C$, tak stačí přičíst jen 2 hledané součiny a máme přesně ten součet třetího a čtvrtého, který chceme. Zároveň rozdíl jistě zachová velikost čísla, tedy i tento 'větší' součin je součin stále polovičních (co do počtu cifer) čísel.

            Tedy náš algoritmus vrátí součin daných čísel pokud je $n = 1$, jinak spustí sám sebe na čísla $A, C$, čísla $B, D$ a čísla $B-A, C-D$. Až dostane výsledky $X = A·C$, $Y = B·D$ a $Z = (B - A)·(c - D)$, tak spočítá $X·2^n + Y + (Z + X + Y)·2^{n/2}$.

            Násobením 2 $n/2$ciferných čísel můžeme dostat až $n$ciferné číslo, tedy $X, Y, Z$ jsou '$n$ciferná'. Navíc násobením $2^n$ se můžeme dostat až na $2n$ciferné číslo. Sčítání a násobení mocninou 2 (tedy bitový posun) zřejmě probíhá v lineárním čase k velikosti vstupu (tj.~v~$O(2n) = O(n)$), jelikož prochází pole cifer jenom jednou. Tedy v jednom kroku děláme $O(n)$ operací a voláme se na 3 poloviční vstupy, tedy rekurence $t(n) ≤ O(n) + 3·t(n/2)$, tj. náš algoritmus běží v čase $O(n^{\log_2(3)})$, jelikož $3·\(1/2\)^\beta = 1 \implies \beta = \log_2(3)$.
        \end{reseni}

        \begin{reseni}[Druhý]
            V podstatě úplně stejný algoritmus funguje pomocí součtů místo rozdílu, jelikož $(A+B)·(C+D) - A·C - B·D = A·D + B·C$. Tento nepatrný rozdíl přináší však jednu nepříjemnost, a to, že součet může mít i o bit více než původní čísla. To však můžeme vyřešit tak, že si zapamatujeme dvě hodnoty $F, G \in \{0, 1\}$ podle toho, zda přeteklo sčítání $(A+B)$ resp. $(C+D)$ (součty však spočítáme jen s $n/2$ bitovou přesností). Výsledek pak vždy spočítáme jako $X·2^n + Y + (Z - X - Y)·2^{n/2} + (C+D)·F·2^n + (A+B)·E·2^n$. Složitost tak zůstane $O(n^{\log_2(3)})$.
        \end{reseni}
    \end{priklad}

\end{document}
