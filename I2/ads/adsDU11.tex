\documentclass[12pt]{article}                   % Začátek dokumentu
\usepackage{../../MFFStyle}                     % Import stylu

\begin{document}

    \begin{priklad}[permBi]
        Algoritmicky popište zobrazení a k němu inverzní zobrazení, mezi čísly od $1$ do $K!$ a permutacemi čísel od $1$ do $K$ (předpokládejte, že máte aritmetiku s dostatečně velkými čísly).

        \begin{reseni}
            Bijekce mezi čísly $1$ až $K!$ a permutacemi je vlastně nějaké uspořádání permutací. Seřazení permutací provedeme tak, že primárně seřadíme podle prvního členu, tento člen následně prohodíme s $K$ a následně seřadíme posloupnosti, které se shodli v prvním členu rekurzivně podle zbytku permutace. Tedy vlastně takové modifikované lexikografické třízení.

            Obraz permutace tedy spočítáme tak, že si na začátku připravíme pole indexů, na kterých se nacházejí daná čísla (které budeme průběžně aktualizovat) (abychom uměli prohazovat prvky), a předpočítáme si $1!, 2!, …, K!$. Obojí umíme v lineárním čase. Také si založíme proměnnou $r$, kde budeme počítat, o kolika permutacích už víme, že jsou menší.

            Následně pro každý (od nejmenšího po největší) index $i \in \{1, …, K\}$ přičtu do $r$: (číslo na aktuálním indexu mínus 1) krát $(K-i)!$, jelikož všechny menší posloupnosti (když pominu předchozí čísla) mají na daném indexu menší číslo ať je zbytek (indexy $i+1, …, K-1$) seřazen jakkoliv. Následně prohodím číslo na aktuálním indexu s číslem $K-i+1$ a pokračuji na další index.

            Obraz čísla $n$ najdeme tím, že si zase připravíme pole s $1!, 2!, …, K!$ a připravíme si pole $1, 2, …, K$. Následně spočítáme $n$ celočíselně vydělený $(K-1)!$, čímž získáme číslo na první pozici, označme ho $m$. Na dalším místě, kde vyjde $m$, bylo (před výpočtem obrazu permutace na tomto indexu) $K$, tedy na pozici číslo $m$ v poli, které jsme si připravily uložíme $K$. Následně „rekurzivně“ generujeme permutaci délky $K-1$ s $r$ rovným $r \% (K-1)!$ (jen nebudeme psát do permutace toto číslo, ale číslo, které je na odpovídajícím indexu v poli).

            Tedy obojí umíme v lineárním čase.
        \end{reseni}
    \end{priklad}

\end{document}
