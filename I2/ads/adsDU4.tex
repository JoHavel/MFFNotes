\documentclass[12pt]{article}                   % Začátek dokumentu
\usepackage{../../MFFStyle}                     % Import stylu

\begin{document}

    \begin{priklad}[rekur]
        Zjednodušte popis funkce $t$ (stačí $\Theta(t)$) dané následující rekurencí: $t(n) = t\(\left\lceil\strut\sqrt{n}\right\rceil\) + c$, pro $n≥3$ a kde $t(n) = n$ pro $1≤n≤2$.

        \begin{reseni}
            Nejprve indukcí dokážeme, že je funkce neklesající. 1. krok: Pro $n = 1, 2$ tvrzení rozhodně platí.

            2. krok: Nechť tvrzení platí pro všechna $n ≤ k$. Chceme dokázat, že $t(k+1) ≥ t(k)$, tedy že
            $$ t\(\left\lceil\strut\sqrt{k+1}\right\rceil\) + c ≥ t\(\left\lceil\strut\sqrt{k}\right\rceil\) + c, $$
            $$ t\(\left\lceil\strut\sqrt{k+1}\right\rceil\) ≥ t\(\left\lceil\strut\sqrt{k}\right\rceil\), $$
            z indukčního předpokladu (a toho, že $\left\lceil\vrule width 0pt height 1em \sqrt{k+1}\right\rceil ≤ k, \forall k > 1$) je to totéž jako:
            $$ \left\lceil\strut\sqrt{k+1}\right\rceil ≥ \left\lceil\strut\sqrt{k}\right\rceil, $$
            horní celá část je také neklesající a odmocnina též tj. toto platí, když $k+1 ≥ k$, což rozhodně je, tudíž $t$ je neklesající.

            Nyní dosadíme $n = (…((2\overbrace{^2)^2)…)^2}^{k\text{-krát}} = 2^{2^k}$. Indukcí dokážeme, že $t\(2^{2^k}\) = 2+k·c$: 1.~krok: $k = 0 \implies t\(2^{2^k}\) = t\(2^1\) = t(2) = 2$.

            2. krok: Ať $t\(2^{2^{k - 1}}\) = 2 + (k-1)·c$. Potom
            $$ t\(2^{2^{k}}\) = t\(\left\lceil\vrule width 0pt height 1.5em \sqrt{2^{2^{k}}}\right\rceil\) + c = t\(\left\lceil 2^{2^{k}/2}\right\rceil\) + c = t\(\left\lceil 2^{2^{k-1}}\right\rceil\) + c = $$
            $$ = t\(2^{2^{k - 1}}\) + c = 2 + (k-1)·c + c = 2 + k·c. $$

            Tedy pro tato konkrétní $n$ je $t(n)$ rovno\footnote{$\log = \log_2$} $2 + \log(\log n)·c$. Ale protože je funkce neklesající, tak pro libovolné $n \in ®N$ je
            $$ 2 + \lfloor\log(\log n)\rfloor·c ≤ t(n) ≤ 2 + \lceil\log(\log n)\rceil·c, $$ 
            $$ 2 + (\log(\log n) - 1)·c ≤ t(n) ≤ 2 + (\log(\log n) + 1)·c. $$
            Z čehož je jasně vidět $t(n) = \Theta(\log(\log n))$.
        \end{reseni}
    \end{priklad}

\end{document}
