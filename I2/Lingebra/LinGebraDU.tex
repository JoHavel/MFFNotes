\documentclass[12pt]{article}					% Začátek dokumentu
\usepackage{../../MFFStyle}					    % Import stylu

\def\degree{^{\circ}}

\begin{document}

\section*{Normy mimo normu}
    \begin{priklad}[8.1]
        Dokažte následující vlastnosti pro libovolnou normu $||·||$:

        \begin{itemize}
            \item[a)] Počátek $¦o\notin S_r$ pro libovolné $r > 0$.
            \item[b)] Množiny $S_r$ a $B_r$ jsou symetrické podle počátku, tedy $−S_r = S_r$ a $−B_r = B_r$.
            \item[c)] Množina $B_r$ je konvexní.
        \end{itemize}

        \begin{dukazin}
            a): Z definice normy je norma lineární, tedy $||¦o|| = ||0·¦x|| = 0·||¦x|| = 0 ≠ r$ pro žádné $r > 0$, tedy $¦o \notin S_r$.

            \noindent b): Jelikož je norma lineární, tak pokud $||s|| = r$ (resp. $||s|| ≤ r$), pak $||-s|| = |-1|·||s|| = ||s|| = r$ (resp. $||-s|| = ||s|| ≤ r$), tedy $s \in S_r \implies -s \in S_r$ (resp. $s \in B_r \implies -s \in B_r$), tedy s každým prvkem je zde obsažen i opačný prvek, tj. $S_r = -S_r$ (resp. $B_r = -B_r$).

            \noindent c): Mějme tedy libovolnou lineární kombinaci $t_1¦x_1 + t_2¦x_2 + … + t_k¦x_k$, kde $t_1 + … + t_k = 1$ a $t_i ≥ 0$, bodů $¦x_1, …, ¦x_k \in B_r$. Z trojúhelníkové nerovnosti a linearity je
            $$ ||t_1¦x_1 + t_2¦x_2 + … + t_k¦x_k|| ≤ ||t_1¦x_1|| + ||t_2¦x_2|| + … + ||t_k¦x_k|| = t_1||¦x_1|| + t_2||¦x_2|| + … + t_k||¦x_k||. $$
            Navíc $t_i||¦x_i|| ≤ t_i·r$, tj. $t_1||¦x_1|| + t_2||¦x_2|| + … + t_k||¦x_k|| ≤ t_1r + t_2r + … + t_kr = r$.
        \end{dukazin}
    \end{priklad}

\pagebreak

    \begin{priklad}[8.2]
        Nechť $||·||$ je libovolné zobrazení $®R^n \rightarrow ®R$, které splňuje (i) a (ii). Dokažte, že $B_r$ je konvexní pro libovolné $r > 0$, právě když $||t¦x + (1−t)¦y|| ≤ t||¦x|| + (1−t) ||¦y||$, kdykoliv $t \in [0,1]$.

        \begin{dukazin}
            $\ \Longleftarrow\ $: Nechť máme $r$ a $¦x, ¦y \in B_r$. Chceme dokázat, že $\{t¦x+(1−t)¦y | t \in [0,1]\} \subseteq B_r$. Mějme tedy nějaké $t \in [0, 1]$:
            $$ ||t¦x + (1−t)¦y|| ≤ t||¦x|| + (1−t) ||¦y|| ≤ t·r + (1-t)·r = r, $$
            neboť $¦x, ¦y \in B_r$, tedy $||¦x||, ||¦y|| ≤ r$. Tudíž i $t¦x + (1−t)¦y \in B_r$.

            \noindent$\implies$: Nejprve pro $¦x = ¦o$ nebo $¦y = ¦o$ nerovnost platí z linearity, jelikož (BÚNO $¦x = ¦o$) je $||t¦x + (1-t)·¦o|| = t||¦x|| = t||¦x|| + (1-t)||0·¦o||$. Nechť tedy $¦x, ¦y ≠ ¦o$, tj. podle nezápornosti $||¦x||, ||¦y|| ≠ 0$. Tudíž můžeme položit $\alpha = ||¦y||/||¦x||$ a $\alpha\tilde{¦y} = ¦y$, takže z linearity $||\tilde{¦y}|| = ||¦y||/\alpha = ||¦x||$. Napíšeme si celou nerovnost znovu:
            $$ ||t¦x + (1−t)\alpha\tilde{¦y}|| \overset{?}{≤} t||¦x|| + (1−t)||\alpha\tilde{¦y}|| = t||¦x|| + (1−t)\alpha||¦x|| = (t + (1-t)\alpha)||¦x||. $$
            Na levé straně už je skoro úsečka mezi vektory na $S_{||¦x||}$. Jen potřebujeme „$t + (1-t)\alpha = 1$“. Není nic jednoduššího než koficienty znormovat (aby dávali součet 1), tj. vydělit nerovnici výrazem $t + (1-t)\alpha$ (je nezáporný, jelikož $0 ≤ t ≤ 1$ a $\alpha > 0$) a použít linearitu:
            $$ \frac{1}{t + (1-t)\alpha}||t¦x + (1−t)\alpha\tilde{¦y}|| = \left|\left|\frac{t}{t + (1-t)\alpha}¦x + \frac{(1-t)\alpha}{t + (1-t)\alpha}\tilde{¦y}\right|\right| \overset{?}{≤} ||¦x||. $$
            Ale $¦x, \tilde{¦y} \in S_{||¦x||} \subseteq B_{||¦x||}$, tedy i úsečka mezi nimi, jejíž bod máme vlevo od menšítka, musí ležet v $B_{||¦x||}$, tedy norma toho bodu je $≤ ||¦x||$, což je přesně, co chceme dokázat.
        \end{dukazin}
    \end{priklad}

\pagebreak

    \begin{priklad}[8.3]
        Nechť $B_1 ≠ \O$ je uzavřená omezená konvexní podmnožina $®R^n$, symetrická podle počátku. Nechť $S_1$ je hranice $B_1$ a nechť $¦o \notin S_1$. Definujme zobrazení $||·||: ®R^n\rightarrow ®R$ následovně: $||¦x|| = |\alpha|$, kde $¦x = \alpha ¦y$ a $¦y \in S_1$. Dokažte, že $||·||$ je norma.

        \begin{dukazin}
            Nejprve dobrá definovanost: Je-li $¦o ≠ ¦x = \alpha_1¦y_1 = \alpha_2¦y_2$, kde $¦y_1, ¦y_2 \in S_1$, potom za prvé je $¦y_1 = ±¦y_2$, protože díky symetrii je $-¦y_1 \in S_1$, tedy pokud by bylo $¦y_2 = \beta¦y_1$ ($\beta$ nemůže být díky $¦o \notin S_1$ nula) BÚNO $0 < \beta < 1$, potom vzhledem k faktu $¦o \notin S_1$ musí $B_1$ obsahovat\footnote{Za předpokladu, že $B_1 ≠ \O$ (pak by bylo zobrazení z prázdné množiny a ne z $®R^n$, tedy by rozbíjelo zadání), $B_1$ obsahuje ¦o, jelikož pro $¦x \in B_1$ ze symetrie i $-¦x \in B_1$ a z konvexnosti i ($t = 1/2$) $$ ¦o = ¦x/2 + (-¦x/2) \in B_1. $$} i nějaké $\epsilon$ okolí ¦o (vzhledem k eukleidovské metrice), tedy obsahuje $(1-\beta)\epsilon$ okolí $¦y_2$ (dokáže se jednoduše přes konvexitu s koeficientem $\beta$), tedy $¦y_2 \notin \partial B_1 = S_1$. Tudíž $\alpha_1 = ± \alpha_2$, a tedy $|\alpha_1| = |\alpha_2|$, tedy $||·||$ je zobrazení (existuje pouze jeden obraz jednoho bodu). Navíc jak jsme již řekli, $B_1$ obsahuje $\epsilon$ okolí ¦o, tedy obsahuje nějaký bod libovolným směrem. A jelikož je omezená a uzavřená, tak každým směrem obsahuje i bod na hranici. Tedy zobrazuje opravdu celé $®R^n$ (a ne jen nějakou podmnožinu).

            Nyní ověříme axiomy normy: Nezápornost -- absolutní hodnota je nezáporná, navíc $\forall ¦x = \alpha¦y ≠ ¦o: \alpha ≠ 0$, tedy $|\alpha| ≠ 0$. Linearita -- jednoduše
            $$ \forall \gamma\ \forall ¦x = \alpha¦y, ¦y \in S_1: ||\gamma¦x|| = ||\gamma·\alpha¦y|| = |\gamma·\alpha| = |\gamma|·||\alpha¦y|| = |\gamma|·||¦x||. $$
            A trojúhelníková nerovnost (už jsme ověřili podmínky předchozí úlohy, tedy ji použijeme)
            $$ \forall ¦x_1, ¦x_2: ||¦x_1 + ¦x_2|| = 2||¦x_1/2 + ¦x_2/2|| ≤ 2||¦x_1/2|| + 2||¦x_2/2|| = ||¦x_1|| + ||¦x_2||. $$
        \end{dukazin}
    \end{priklad}

\section*{Determinant je multilineární alternující forma}
    \begin{priklad}[9.1]
        Pomocí základních vlastností determinantu dokažte následující vlastnosti:

        \begin{itemize}
            \item[1')] První řádek není v ničem speciální, determinant je lineárně závislý na libovolném řádku matice.
            \item[4)] Pokud má determinant nulový řádek, je roven nule.
            \item[5)] Pokud má determinant dva řádky stejné, je roven nule.
            \item[6)] Přičtení libovolného násobku jednoho řádku k jinému nemění determinant.
            \item[7)] Determinant trojúhelníkové matice $T$ (horní či dolní) je roven součinu prvků na hlavní diagonále.
            \item[8)] Determinant je nenulový právě tehdy, když je matice regulární.
            \item[9)] Determinant součinu dvou matic je součin jejich determinantů.
            \item[10)] Determinant matice i její transpozice je stejný, tedy $\det A = \det A^T$.
        \end{itemize}

        \begin{dukazin}
            \begin{itemize}
                \item[1')] Prohodíme daný řádek s prvním, tím před determinanty dostaneme mínus. Následně aplikujeme rovnice z 1) a nakonec prohodíme řádky zpět, čímž dostaneme další mínus před determinant, tedy v celku plus.
                \item[4)] Podle 1') je determinant takové matice roven libovolnému svému násobku (násobíme něčím nulový řádek), tedy je $0$.
                \item[5)] Podle 2) je determinant s těmito řádky prohozenými roven opačnému číslu, je to však ten samý determinant, tedy je nulový.
                \item[6)] Vezmeme matici, která má místo „jiného řádku“ „jeden“. Ta má 2 stejné řádky, tedy její determinant je nula. Aplikujeme na ni 2. rovnost z 1'), tím dostaneme zase nulu, a následně ji pomocí 1. rovnosti přičteme k původní matici, čímž dostaneme, že determinant nové matice je roven determinantu původní plus nula.
                \item[7)] Pokud kdykoliv během výpočtu dostaneme nulový řádek, tak je determinant 0. Postupujeme následovně: Příslušné násobky posledního řádku (resp. prvního řádku u dolní trojúhelníkové) odečteme od všech ostatních, aby jediný nenulový prvek v posledním (resp. prvním) sloupci byl ten na diagonále. Následně totéž provedeme u předposledního (2.), a tak dále. Nakonec dostaneme ekvivalentní (co do hodnoty determinantu) diagonální matici. Pak podle 1') vytkneme všechny prvky na diagonále a dostaneme tak $I$, které má determinant 1 a před tím součin prvků na diagonále.
                \item[8)] Podle 6) a 2. rovnice 1') Gaussova eliminace (konkrétně každý její krok) nemění nenulovost determinantu, tedy pokud GE vyjdou na diagonále nenulová čísla (regulární matice), tak je determinant nenulový, jinak (singulární matice) nulový.
                \item[9)] Asi aktuálně netuším. Zkusím si rozmyslet časem…
                \item[10)] Řešení dalšího příkladu říká, že $P_\pi$ má determinant $\sgn(\pi)$, tedy podle 9)
                    $$ \det A = 1·\det A = (\sgn \pi)^2\det A = \det P_\pi \det A \det P_\pi = \det(P_\pi A P_\pi). $$
                    A můžeme si všimnout, že $A^T$ je $P_\pi A P_\pi$ pro $\pi = \begin{pmatrix} 1 & 2 & … & n-1 & n \\ n & n-1 & … & 2 & 1 \end{pmatrix}$.
            \end{itemize}
        \end{dukazin}
    \end{priklad}

    \begin{priklad}[9.2]
        Jak vypadá determinant permutační matice $P_\pi$? Dokažte pomocí předchozích vlastností i pomocí definice.

        \begin{reseni}
            Z definice determinantu je jasné, že $\det P_\pi = \sgn \pi$, jelikož jediný nenulový člen je ten pro $\pi$ a všechny prvky matice v tomto členu jsou 1.

            K důkazu pomocí vlastností nám stačí 3) a 2), jelikož pomocí 2) proházíme řádky (rozložíme $\pi$ na transpozice a podle těch proházíme řádky), až dostaneme $I$. Protože každé prohození řádku odpovídá složení $\pi$ s 1 transpozicí, tak nám vyjde znaménko $\sgn \pi$. Podle 3) je $\det I = 1$, tedy je $\det P_\pi = \sgn \pi$.
        \end{reseni}
    \end{priklad}

    \begin{priklad}[9.3]
        Dokažte, že determinant definovaný za pomoci základních vlastností je ekvivalentní s klasickou definicí.

        \begin{dukazin}
            Determinant pomocí 1') rozložíme na součet matic majících v každém řádku jen 1 nenulové číslo (nejdříve rozdělíme první řádek, pak pro každou vzniklou matici druhý, …). Pokud má matice nulový sloupec, tak podle 10) a 4) (nebo podle 8)) je její determinant roven nule. Z ostatních matic „vytkneme“ permutační matici, podle 9) rozdělíme determinant na součin a dostaneme diagonální matici (tedy speciální trojúhelníkovou), která podle 7) má determinant roven součinu jejích nenulových prvků. Determinant permutační matice je roven $\sgn \pi$, tedy dostáváme přesně klasickou definici determinantu.
        \end{dukazin}
    \end{priklad}

\pagebreak

\section*{Hadamardovy matice dají nejvíce!}
    \begin{priklad}[10.1]
        Nechť $H$ je Hadamardova matice $n \times n$. Potom $n = 1, 2$ nebo je dělitelné čtyřmi.

        \begin{dukazin}
            Je-li $n > 1$, potom v matici $H·H^T = nI_n$ je $0$. Tato nula musí být výsledkem skalárního součinu 2 vektorů (řádku $H$ a sloupce $H^T$), které jsou tvořeny pouze $1$ a $-1$. To ale znamená, že tato nula je rovna součtu $n$ prvků tvaru $(±1)·(±1)$. Takový součet má ale stejnou paritu jako $n$, neboť se s každým členem parita změní. Ale $0$ je sudá, tedy $n$ je sudé.

            Už u definice je napsáno, že sloupce $H$ jsou ortogonální vektory. Můžeme si všimnout, že i naopak, pokud máme ortogonální bázi vektorů obsahujících pouze $1$ a $-1$, tak už je matice, která má vektory této báze za sloupce, Hadamardova. Tedy můžeme vzít sloupce $H$ a libovolně je propermutovat a vynásobit $±1$, až dostaneme matici, která má v prvním řádku samé jedničky (násobení $±1$) a v druhém řádku nejdříve $1$ a pak $-1$ (permutace sloupců).

            Jelikož (standardní) skalární součin prvního a druhého vektoru musí být $0$ (z $H·H^T = nI_n$), tak druhý řádek musí mít stejně $1$ jako $-1$. Stejně tak i $3.$ řádek pro $n > 2$ musí mít stejně $1$ a $-1$, jelikož taktéž dává ve skalárním součinu s $1.$ řádkem $0$. Zároveň dává $0$ i s $2.$ řádkem, tudíž když označíme $k$ počet $1$ v první polovině 3. řádku a $j$ počet $1$ v druhé, pak $k+j = n/2$ (ze součinu $1.$ a $3.$) a $k - j - n/2 + k + n/2 - j = 2(k-j) = 0$, tj. $k = j$ (ze součinu $2.$ a $3.$). Tedy se shoduje i parita $k$ a $j$, tudíž $k+j$ je sudé a $n = 2(k+j)$ násobek $2·2 = 4$.
        \end{dukazin}
    \end{priklad}

\pagebreak

    \begin{priklad}[10.2]
        Dokažte, že existuje Hadamardova matice $H_n$ velikosti $n\times n$ pro každé $n = 2^k$.

        \begin{dukazin}
            Snadno si ověříme, že $H_2 \otimes H_{2^k} = H_{2·2^{k}}$ a jelikož víme, že $H_2$ existuje, tak existuje i $H_{2^i}$.

            Ještě mě však napadla jiná konstrukce $H_{2^k}$ (vedoucí pravděpodobně na „podobnou“ matici a velmi neefektivní, ale což): Začneme s $\sqrt{n}·I_n$, což je matice, jejíž sloupce tvoří ortogonální bázi se správnými velikostmi vektorů. Tedy pokud najdeme zobrazení zachovávající kolmost a velikost, které nám tuto matici převede na matici $1$ a $-1$, tak jsme vyhráli. $\sqrt{n}·I_n$ vynásobme postupně maticemi ($\frac{1}{\sqrt{2}} = q_1 = \sin(45\degree) = q_2 = \cos(45\degree)$):
            $$ \begin{pmatrix} q_1 & q_2 & 0 & 0 & … \\ -q_2 & q_1 & 0 & 0 & … \\ 0 & 0 & q_1 & q_2 & … \\ 0 & 0 & -q_2 & q_1 & … \\ \vdots & \vdots & \vdots & \vdots & \ddots \end{pmatrix}, \begin{pmatrix} q_1 & 0 & q_2 & 0 & … \\ 0 & q_1 & 0 & q_2 & … \\ -q_2 & 0 & q_1 & 0 & … \\ 0 & -q_2 & 0 & q_1 & … \\ \vdots & \vdots & \vdots & \vdots & \ddots \end{pmatrix}, \begin{pmatrix} q_1 & 0 & 0 & 0 & … \\ 0 & q_1 & 0 & 0 & … \\ 0 & 0 & q_1 & 0 & … \\ 0 & 0 & 0 & q_1 & … \\ \vdots & \vdots & \vdots & \vdots & \ddots \end{pmatrix}… $$
            Po prvním vynásobení budeme mít v matici na diagonále buňky $2\times 2$, po druhém $4 \times 4$, po třetím $8 \times 8$, … Navíc každá matice je soubor rotací o $45$ stupňů (první matice jsou rotace v sousedních souřadnicích, 2. jsou rotace ob jednu souřadnici v budoucích buňkách $4 \times 4$, 3. ob 3 souřadnice, 4. ob 7, …), tedy toto zobrazení (složení všech těchto rotací) zachovává velikost vektorů a kolmost. Navíc si lehko ověříme, že čísla v jednom bloku jsou vždy stejná co do absolutní hodnoty, tedy po $k$ krocích (blok $2^k \times 2^k$) musí být všechna $±1$.
        \end{dukazin}
    \end{priklad}

    \begin{priklad}[10.3]
        Mějme matici $A$ velikosti $n\times n$, že $\left|a_{i,j}\right|≤1$ ve všech pozicích $(i, j)$. Potom platí $|\det(A)| ≤ n^{n/2}$ a rovnosti se nabývá právě tehdy, když $A$ je Hadamardova matice.

        \begin{dukazin}
            Když si determinant představíme jako ($n$-dimenzionální) objem rovnoběžnostěnu daného sloupcovými / řádkovými vektory dané matice, tak je zřejmé, že největší objem má, když jsou vektory na sebe kolmé, a to součin jejich velikostí (jelikož pak je to kvádr, podrobný důkaz je na \url{https://github.com/JoHavel/MFFNotes/blob/master/OM2/LinGebra/LinGebraDU3.pdf}):
            $$ \det(A) ≤ \prod_{i=1}^n \sqrt{\sum_{j=1}^n a_{i, j}^2} ≤ \prod_{i=1}^n \sqrt{n·1} = n^{n/2}. $$

            Jak jsme již řekli v prvním odstavci, maximální je determinant, když jsou vektory ortogonální, a rovnost v nerovnosti výše nastává, právě když všechna $a_{i, j}$ jsou rovna $±1$ (potom je velikost vektoru přesně $\sqrt{n}$), a toto je právě tehdy, když $A$ je Hadamardova.
        \end{dukazin}
    \end{priklad}

\section*{Determinují determinanty perfektní párování?}
    \begin{priklad}[11.1]
        Dokažte, že pokud $\det(I_G) ≠ 0$, graf $G$ má nutně perfektní párování.

        \begin{dukazin}
            Nechť $n$ je řád $I_G = (a_{ij})$. Můžeme si všimnout, že všechna perfektní párování v této situaci lze vzájemně přiřadit permutacím $\pi \in S_n$ takovým, že $\forall i: a_{i, \pi(i)} = 1$, jelikož párování je přesně bijekce (tj. v našem případě permutace) a mezi danými dvojicemi vrcholů musí vést hrana.

            Tedy pokud máme nenulový determinant, tak to znamená, že ve výrazu z definice determinantu ($\det I_G = \sum_{\pi \in S_n}\sgn(\pi)·\prod_{i=1}^n a_{i, \pi(i)}$) musí být nějaký člen součtu nenulový, tj. $\exists \pi_1: \sgn(\pi_1)\prod_{i=1}^n a_{i, \pi_1(i)} ≠ 0$. To ale znamená, že $\forall i: a_{i, \pi_1(i)} ≠ 0$ (protože jsme v tělese). Ale jediná možná nenulová hodnota v naší matici je 1, tj. $\forall i: a_{i, \pi_1(i)} = 1$, a podle prvního odstavce jsme tak našli nějaké perfektní párování.
        \end{dukazin}
    \end{priklad}

    \begin{priklad}[11.2]
        Rozhodněte a zdůvodněte, zda platí i obrácená implikace. Je nějaký vztah mezi hodnotou determinantu a počtem různých perfektních párování?

        \begin{reseni}
            Neplatí, už jen pro matici $I_G = \begin{pmatrix} 1 & 1 \\ 1 & 1 \end{pmatrix}$ je její determinant roven 0, přesto perfektní párování existují.

            V důkazu předchozího příkladu si můžeme všimnout, že každé perfektní párování odpovídá nenulovému členu v determinantu a opačně. Navíc (protože v matici máme jen $1$ nebo $0$) nenulový člen determinantu našich matic může mít hodnotu pouze $1$ nebo $-1$. Tudíž pokud se na to podíváme v $®Z_2$, kde $-1 = 1$, tak determinant je opravdu součet jedniček za každé perfektní párování. Tudíž determinant má stejnou paritu jako počet perfektních párování.

            Nic moc víc nám neříká, jelikož vlastně říká rozdíl počtu perfektních párování odpovídajících sudým a lichým permutacím, ale to vůbec nic neříká o jejich celkovém počtu.
        \end{reseni}
    \end{priklad}

\pagebreak

\section*{Po stopách matic}
    \begin{priklad}[12.1]
        Dokažte pro libovolné matice $A \in ®R^{m \times n}$ a $B \in R^{n \times m}$, že platí
        $$ \tr(AB) = \tr(BA). $$ 

        \begin{dukazin}
            Z definice maticového násobení
            $$ (AB)_{i, k} = \sum_{j=1}^n a_{i, j}b_{j, k} \qquad (BA)_{i, k} = \sum_{j=1}^m b_{i, j}a_{j, k}. $$
            Nás zajímá součet prvků na diagonále:
            $$ \tr(AB) = \sum_{i=1}^m(AB)_{i, i} = \sum_{i=1}^m\sum_{j=1}^n a_{i, j}b_{j, i} \qquad \tr(BA)\sum_{i=1}^n(BA)_{i, i} = \sum_{i=1}^n\sum_{j=1}^m b_{i, j}a_{j, i}. $$
            (Jednoduše ověříme, že meze indexů sedí.) V reálných číslech je násobení komutativní (můžeme prohodit $a$ s $b$) a stejně tak sčítání (můžeme prohodit sumy). Tedy výrazy výše se liší jen pojmenováním indexů a jsou tudíž totožné.
        \end{dukazin}
    \end{priklad}

    \begin{priklad}[12.2]
        Dokažte, že maticová podobnost nemění stopu, tedy $\tr\(SAS^{-1}\) = \tr(A)$.

        \begin{dukazin}
            Podle předchozího tvrzení (a asociativity násobení matic) je:
            $$ \tr\(SAS^{-1}\) = \tr\((SA)S^{-1}\) = \tr\(S^{-1}(SA)\) = \tr\(\(S^{-1}S\)A\) = \tr(A). $$
        \end{dukazin}
    \end{priklad}

    \begin{priklad}[12.3]
        Nechť $\lambda_1, …, \lambda_n$ jsou vlastní čísla matice $A$. Dokažte, že
        $$ \tr(A) = \lambda_1 + … + \lambda_n. $$ 

        \begin{dukazin}
            Předpokládejme, že $n$ je řád $A$ (jinak by tvrzení neplatilo, viz $\begin{pmatrix} 1 & -1 \\ 1 & 0 \end{pmatrix}$). Víme, že každá matice řádu $n$ s $n$ vlastními čísly včetně jejich algebraické násobnosti je podobná matici v Jordanově tvaru. Jordanova matice podobná $A$ (označme ji $J$) má na diagonále $\lambda_1, …, \lambda_n$. Tedy $\tr(A) = \tr\(SJS^{-1}\) = \tr(J) = \lambda_1 + … + \lambda_n$.
        \end{dukazin}
    \end{priklad}

    \begin{priklad}[12.4]
        Nechť $f: ®R^{n \times n} \rightarrow ®R$ je lineární zobrazení, které splňuje $f(AB) = f(BA)$ a pro které $f\(I_n\) = n$. Dokažte, že potom $f = \tr$.

        \begin{dukazin}
            $\(E_{i, j}\)_{i, j \in [n]}$ je báze prostoru $®R^{n \times n}$ a $f$ i $\tr$ jsou lineární zobrazení, tedy jsou jednoznačně určeny obrazy báze (jak víme z prvního semestru), tedy stačí ověřit $\forall i, j \in [n]: f\(E_{i, j}\) = \tr\(E_{i, j}\)$. Můžeme si všimnout, že z podmínky na zobrazení $f$ je
            $$ \forall i, j \in [n]: f(E_{i, i}) = f(E_{i, j}E_{j, i}) = f(E_{j, i}E_{i, j}) = f(E_{j, j}). $$
            Tedy (ještě z linearity) $f(I_n) = \sum_{i=1}^n f(E_{i, i}) = n·E_{k, k}$, tj. $E_{k, k} = 1$ pro libovolné $k$.

            Naopak z linearity je $f(0_{n\times n}) = 0$. Ale („prostřední“ rovnost zase z vlastnosti $f$)
            $$ \forall i, j \in [n], i≠j: 0 = f\(0_{n \times n}\) = f\(E_{i, j}E_{i, i}\) = f\(E_{i, i}E_{i, j}\) = f\(E_{i, j}\). $$
            Tedy opravdu $\forall i, j \in [n]: f\(E_{i, j}\) = \tr\(E_{i, j}\)$.
        \end{dukazin}
    \end{priklad}
\end{document}
