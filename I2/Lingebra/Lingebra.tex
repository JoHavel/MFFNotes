\documentclass[12pt]{article}                   % Začátek dokumentu
\usepackage{../../MFFStyle}                     % Import stylu

\begin{document}
\section{Skalární součin}
    \begin{definice}[Standardní skalární součin vektorů]
        Standardní skalární součin vektorů $¦x, ¦y \in ®R^n$ je definován jako $¦x^T¦y = \sum_{i=1}^n x_iy_i$.
    \end{definice}

    \begin{definice}[Skalární součin nad ®R]
        Buď ¦V vektorový prostor nad ®R. Pak skalární součin je zobrazení $\<·,·\>: ¦V^2 \rightarrow ®R$, splňující pro všechna $¦x, ¦y, ¦z \in ¦V$ a $\alpha \in ®R$:
        $$ \<¦x, ¦x\> ≥ 0 \text {a rovnost nastane jen pro } ¦x=¦o, $$
        $$ \<¦x + ¦y, ¦z\> = \<¦x, ¦z\> + \<¦y, ¦z\>, $$
        $$ \<\alpha ¦x, ¦y\> = \alpha \<¦x, ¦y\>, $$
        $$ \<¦x, ¦y\> = \<¦y, ¦x\>. $$ 
    \end{definice}

    \begin{definice}[Komplexně sdružené číslo]
        Komplexně sdružené číslo k $a + bi \in ®C$ je číslo $\overline{a + bi} = a - bi$.
    \end{definice}
    
    \begin{definice}[Skalární součin nad ®C]
        Buď ¦V vektorový prostor nad ®C. Pak skalární součin je zobrazení $\<·,·\>: ¦V^2 \rightarrow ®C$, splňující pro všechna $¦x, ¦y, ¦z \in ¦V$ a $\alpha \in ®C$:
        $$ \<¦x, ¦x\> ≥ 0 \text {a rovnost nastane jen pro } ¦x=¦o, $$
        $$ \<¦x + ¦y, ¦z\> = \<¦x, ¦z\> + \<¦y, ¦z\>, $$
        $$ \<\alpha ¦x, ¦y\> = \alpha \<¦x, ¦y\>, $$
        $$ \<¦x, ¦y\> = \overline{\<¦y, ¦x\>}. $$ 
    \end{definice}

    \begin{poznamka}
        Skalární součin je bilineární (nad ®C se ale musí komplexně sdružovat druhá složka), tudíž je dán hodnotami pro báze (všechny dvojice bází).
    \end{poznamka}

    \begin{definice}[Norma indukovaná skalárním součinem]
        Norma indukovaná skalárním součinem je definovaná jako
        $$ ||¦x|| := \sqrt{\<¦x, ¦x\>}, \text{ kde } ¦x \in ¦V. $$ 
    \end{definice}

    \begin{definice}[Kolmost]
        Vektory $¦x, ¦y \in ¦V$ jsou kolmé, pokud $\<¦x, ¦y\> = 0$. Značíme: $¦x \perp ¦y$.
    \end{definice}

    \begin{veta}[Pythagorova]
        Pokud $¦x, ¦y \in ¦V$ jsou kolmé, tak
        $$ ||x + y||^2 = ||x||^2 + ||y^2||. $$ 

        \begin{dukazin}
            $$ ||x + y||^2  = \<¦x + ¦y, ¦x + ¦y\> = \<¦x, ¦x\> + \<¦x, ¦y\> + \<¦y, ¦x\> + \<¦y, ¦y\> = ||x||^2 + 0 + 0 + ||y^2|| = ||x||^2 + ||y^2||. $$ 
        \end{dukazin}

        \begin{poznamkain}
            Pro reálná čísla platí i zpětná implikace, pro komplexní obecně ne.
        \end{poznamkain}
    \end{veta}

    \begin{veta}[Cauchyho-Schwartzova nerovnost]
        Pro každé $¦x, ¦y \in ¦V$ platí
        $$ |\<¦x, ¦y\>| ≤ ||¦x||·||¦y||. $$

        \begin{dukazin}
            Pro $¦y = ¦o$ platí, tak předpokládejme $¦y ≠ ¦o$. Uvažujme funkci
            $$ f(t) = \<¦x + t¦y, ¦x + t¦y\> ≥ 0. $$ 
            Pak 
            $$ f(t) = \<¦x, ¦x\> + t\<¦y, ¦x\> + t\<¦x, ¦y\> + t^2\<¦y, ¦y\> = \<¦x, ¦x\> + 2t\<¦y, ¦x\> t^2\<¦y, ¦y\> ≥ 0. $$ 
            Což je kvadratická funkce, která je všude nezáporná, tedy nemůže mít 2 kořeny, tedy má nekladný diskriminant:
            $$ 4\<¦y, ¦x\>^2 - 4\<¦x, ¦x\>\<¦y, ¦y\> ≤ 0. $$
            Odtud $\<¦y, ¦x\>^2 ≤ \<¦x, ¦x\>\<¦y, ¦y\>$ a odmocněním $|\<¦x, ¦y\>| ≤ ||¦x||·||¦y||$.
        \end{dukazin}
    \end{veta}

    \begin{dusledek}[Trojúhelníková nerovnost]
        $$ ||¦x + ¦y|| ≤ ||¦x|| + ||¦y||. $$

        \begin{dukazin}
            Nejprve připomeňme, že pro každé $z = a + bi \in ®C$ platí $z + \overline{z} = 2a = 2\Re(z)$, a dále $a ≤ |z| = \sqrt{a^2 + b^2} = \sqrt{z·\overline{z}}$. Nyní můžeme odvodit (obě strany jsou nezáporné):
            $$ ||¦x + ¦y||^2 = \<¦x + ¦y, ¦x + ¦y\> = \<¦x, ¦x\> + \<¦x, ¦y\> + \<¦y, ¦x\> + \<¦y, ¦y\> = \<¦x + ¦y, ¦x + ¦y\> = \<¦x, ¦x\> + 2\Re(\<¦x, ¦y\>) + \<¦y, ¦y\> ≤ \<¦x , ¦x\> + 2|\<¦x, ¦y\>| + \<¦y, ¦y\> \overset{\text{CS}}{≤} ||¦x|| + 2||¦x||·||¦y|| + ||¦y|| = \(||¦x|| + ||¦y||\)^2. $$
        \end{dukazin}
    \end{dusledek}

    \subsection{Norma obecně}
        \begin{definice}[Norma]
            Buď ¦V vektorový prostor nad ®R nebo ®C. Pak norma zobrazení $||·||: ¦V \rightarrow ®R$, splňující (pro každé $¦x, ¦y \in ¦V$ a $\alpha \in ®R$ resp. ®C):
            $$ ||¦x|| ≥ 0 \text{ a rovnost nastává pouze pro } ¦x = ¦o, $$
            $$ ||\alpha ¦x|| = |\alpha|·||¦x||, $$ 
            $$ ||¦x + ¦y|| ≤ ||¦x|| + ||¦y||. $$ 
        \end{definice}

        \begin{tvrzeni}
            Norma indukovaná skalárním součinem je normou.

            \begin{dukazin}
                Triviální.
            \end{dukazin}
        \end{tvrzeni}

        \begin{priklady}
            Pro $p = 1, 2, …, ∞$ definujeme $p$-normu vektoru $¦x \in ®R^n$ jako
            $$ ||¦x||_p = \(\sum_{i=1}^n |x_i|^p\)^{\frac{1}{p}}. $$ 
        \end{priklady}

        \begin{definice}[Jednotková koule]
            Jednotková koule je množina vektorů, které mají normu nanejvýš 1, a tedy jsou od počátku vzdáleny maximálně 1:
            $$ \{¦x \in ¦V : ||x|| ≤ 1\}. $$

            \begin{poznamkain}
                Jednotková koule je vždy uzavřená, omezená, symetrická dle počátku, konvexní a počátek leží v jejím vnitřku.
            \end{poznamkain}
        \end{definice}

        \begin{definice}[Metrika generovaná normou]
            Každá norma určuje metriku (vzdálenost) předpisem $d(¦x, ¦y) := ||¦x - ¦y||$.
        \end{definice}
\end{document}
