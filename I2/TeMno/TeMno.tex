\documentclass[12pt]{article}                   % Začátek dokumentu
\usepackage{../../MFFStyle}                     % Import stylu

\begin{document}
\section{Úvod}
    \begin{poznamka}[Aplikace]
        Transfinitní indukce, axiom výběru (= princip maximality = Zornovo lemma)
    \end{poznamka}

    \begin{poznamka}[Cíl]
        Vybudování matematiky na pevných základech. Porozumění nekonečen. Důkaz existence nealgebraických (= transcendentních) reálných čísel. Princip kompaktnosti. Banach-Tarského paradox.
    \end{poznamka}

    \begin{poznamka}[Literatura]
        Balcar, Štěpánek -- Teorie množin\\
        Seriál PraSete\\
        Hrbáček, Jech -- Introduction to set theory\\
        Olšák -- Esence teorie množin (videa)
    \end{poznamka}

    \begin{poznamka}[Historie]
        Bernard Bolzano (český matematik, 1781-1848, pojem množina), George Cantor (německý matematik, 1845 - 1918, zavedení aktuálního nekonečna, diagonální metoda, kardinální čísla, uzavřená množina), Bertrand Russell (1902, Russellův paradox = paradox holiče = holí holič holící všechny lidi, kteří se neholí sami, sebe?) + Berriho paradox (nechť m je nejmenší přirozené číslo, které nejde definovat méně než 100 znaky), Zermelo-Fraenkel (zavedli axiomatickou teorii množin).
    \end{poznamka}

    \begin{definice}[Symboly]
        Proměnné pro množiny -- $x, y, z, x_1, x_2, …$.

        Binární predikátorový symbol $=$ a bin. relační symbol $\in$.

        Logické spojky $\neg, \lor, \land, \implies, \Leftrightarrow$.

        Kvantifikátory $\forall, \exists$.

        Závorky $\(\)\{\}\[\]$
    \end{definice}

    \begin{definice}[Formule]
        Atomické ($x = y$, $x \in y$). Jsou-li $\phi$ a $\psi$ formule, pak $\neg \phi$, $\phi \lor \psi$, $\phi \land \psi$, $\phi \implies \psi$, $\phi \Leftrightarrow \psi$ jsou formule. Je-li $\phi$ formule, $x$ proměnná, pak $(\forall x)\phi$, $(\exists x)\phi$ jsou formule. (Vázané vs. volné proměnné -- proměnné formule, které do ní lze dosadit jsou volné, proměnné formule, které do ní nelze dosadit jsou vázané). Každou formuli lze dostat konečnou posloupností aplikací výše zmíněného.
    \end{definice}

    \begin{definice}[Rozšíření jazyka]
        $x ≠ y$ značí $\neg(x=y)$, $x \notin y$ znamená $\neg(x \in y)$, $x \subseteq y$ znamená $(\forall u)(u \in x \implies u \in y)$, $x \subset y$ značí $x \subseteq y \land x ≠ y$. Dále uvidíme $\cup$, $\cap$, $\setminus$, $\{x_1, …, x_n\}$, $\O$, $\{x \in a|\phi(x)\}$.
    \end{definice}

    \begin{definice}[Axiomy logiky]
        Vysvětlují, jak se chovají implikace, kvantifikátory, rovnost, …
    \end{definice}

    \begin{definice}[Axiomy TEMNA]
        Říkají, jak se chová $\in$ a jaké množiny existují. Budeme používat Zermelo-Fraenkelovu teorii (ZF), tedy 9 axiomů (7 + 2 schémata). (Není minimální, tj. lze některé odvodit z jiných) + axiom výběru (AC) s ním se pak ZF značí ZFC.
    \end{definice}

    \begin{definice}[Axiomy ZFC]
        \ 
        \begin{enumerate}
            \item Axiom existence množiny: $(\exists x)(x=x)$.
            \item Axiom extenzionality: $(\forall z)(z \in x \Leftrightarrow z \in y) \implies x=y$.

% 9. 3. 2021

            \item Schéma axiomů vydělení: je-li $\phi(x)$ formule, která neobsahuje volnou proměnnou $z$, potom $(\forall a)(\exists z)(\forall x)(x \in z \Leftrightarrow (x \in a \land \phi(x)))$ je axiom.
            \item Axiom dvojice: $(\forall a)(\forall b)(\exists z)(\forall x)(x \in z \Leftrightarrow (x = a \lor x = b))$.
            \item Axiom sumy: $(\forall a)(\exists z)(\forall x)(x \in z \Leftrightarrow (\exists y)(x \in y \land y \in a))$.
            \item Axiom potence: $(\forall a)(\exists z)(\forall x)(x \in z \Leftrightarrow x \subseteq a)$.
            \item Schéma axiomů nahrazení\footnote{Slogan: Obraz množiny funkcí je množina.} Je-li $\psi(u, v)$ formule s volnými proměnnými $u, v$, jež nemá volné proměnné $w, z$, pak
                    $$ (\forall u)(\forall v)(\forall w)((\psi(u, v) \land \psi(u, w))\implies v=w) \implies (\forall a)(\exists z)(\forall v)(v \in z \Leftrightarrow (\exists u)(u \in a \land \psi(u, v))) $$
                    je axiom.
            \item Axiom fundovanosti (regularity): $(\forall a)(a ≠ \O \implies (\exists x)(x \in a \land x \cap a = \O))$.
        \end{enumerate}

        Později ještě:
        \begin{itemize}
            \item Axiom nekonečna
            \item Axiom výběru
        \end{itemize}
    \end{definice}

    \begin{definice}[Značení]
        $\{x | x \in a \land \phi(x)\}$, zkráceně $\{x \in a | \phi(x)\}$ je množina z axiomů vydělení.
    \end{definice}

    \begin{definice}[Průnik, množinový rozdíl, prázdná množina]
        $a \cap b = \{x | x \in a \land x \in b\}$.

        $a \setminus b = a - b = \{x | x \in a \land x \notin b\}$.

        $\O = \{x | x \in a \land x ≠ x\}$. ($\O$ je díky prvním třem axiomům dobře definována.)
    \end{definice}

    \begin{definice}[Neuspořádaná a uspořádaná dvojice]
        $\{a, b\}$ je neuspořádaná dvojice, $\{a\}$ znamená $\{a, a\}$.

        $(a, b) = <a, b> = \{\{a\}, \{a, b\}\}$ je uspořádaná dvojice.
    \end{definice}

    \begin{lemma}
        $(x, y) = (u, v) \Leftrightarrow (x = u \land y = v)$.

        \begin{dukazin}
            $\Leftarrow$ $x=u$, pak $\{x\} = \{u\}$ z axiomu extenzionality, stejně tak $\{x, y\} = \{u, v\}$, a tedy $\{\{x\}, \{x, y\}\} = \{\{u\}, \{u, v\}\}$.

            $\implies$ $\{\{x\}, \{x, y\}\} = \{\{u\}, \{u, v\}\}$ pak $\{x\} = \{u\}$ nebo $\{x\} = \{u, v\}$, každopádně $x = u$. Nyní $\{u, v\} = \{x\}$ nebo $\{u, v\} = \{x, y\}$ tedy $v = x$ nebo $v = y$. Pokud $v = y$, tak jsme skončili, pokud $v = x$ pak $v = u = x = y$.
        \end{dukazin}
    \end{lemma}

    \begin{definice}[Potenční množina]
        $©P(a) = \{x | x \subseteq a\}$ je potenční množina (potence) $a$ (z axiomu potence).
    \end{definice}

    \begin{definice}[Uspořádaná $n$-tice]
        Jsou-li $a_1, a_2, …, a_n$ množiny, definujeme uspořádanou $n$-tici $(a_1, a_2, …, a_n) = <…>$ následovně: $(a_1) = a_1$, je-li definovaná $(a_1, …, a_k)$, pak $(a_1, …, a_k, a_{k+1}) = ((a_1, a_2, …, a_k), a_{k+1})$.
    \end{definice}

    \begin{lemma}
        $(a_1, a_2, …, a_n) = (b_1, b_2, …, b_n) \Leftrightarrow (a_1 = b_1 \land a_2 = b_2 \land … \land a_n = b_n)$.

        \begin{dukazin}
            Domácí cvičení.
        \end{dukazin}
    \end{lemma}

    \begin{definice}[Značení]
        $$ \bigcup a = \{x | (\exists y)(x \in y \land y \in a\}. $$ 
        
        \begin{definicein}
            Pro $a = \{b, c\}$ definujeme $b \cup c = \bigcup a$.
        \end{definicein}
    \end{definice}

    \begin{definice}[Neuspořádaná $n$-tice]
            Neuspořádanou $n$-tici $\{a_1, …, a_n\}$ ($n$-prvkovou množinu) definujeme rekurzivně: je-li definováno $\{a_1, …, a_k\}$, pak $\{a_1, …, a_k, a_{k+1}\} = \{a_1, …, a_k\} \cup \{a_{k+1}\}$.
    \end{definice}

% 16. 3. 2021

    \begin{poznamka}
        Axiom nahrazení se využívá v: transfinitní rekurzi, definici $\omega + \omega$, větě o typu dobrého uspořádání, Zornově lemmatu (tj. axiom výběru).
    \end{poznamka}

    \begin{priklad}
        Ukažte, že axiom fundovanosti zakazuje existenci konečných cyklů relace $\in$ (tj. takových množin $y$, pro které $y \in . \in … \in . \in y$).

        \begin{dusledekin}
            Všechny množiny lze vygenerovat z $\O$ pomocí operací $\bigcup$ a $©P$ (zhruba).
        \end{dusledekin}
    \end{priklad}

    \subsection{Třídy}
        \begin{definice}
            Nechť $\phi(x)$ je formule, pak $\{x, \phi(x)\}$ (čteme třída všech $x$, pro které platí $\phi(x)$), tzv. třídový term, se nazývá třída (určená formulí $x$).
        \end{definice}

        \begin{dusledek}
            Pokud $\phi(x)$ je tvaru $x \in a \land \phi(x)$, pak je $\{x, \phi(A)\}$ množina z axiomu vydělení. Obdobně pro axiom dvojice, sumy, …
        \end{dusledek}

        \begin{poznamka}
            Je-li $y$ množina, pak $y$ má stejné prvky jako třída ${x, x \in y \land x = x}$.
        \end{poznamka}

        \begin{poznamka}[Vlastní třídy]
            Existují i třídy (tzv. vlastní), které nejsou množiny (např. třída všech množin).
        \end{poznamka}

        \begin{definice}[Rozšíření jazyka]
            Ve formulích na místech volných proměnných připustíme i Třídové termy a proměnné pro třídy (psané velkými písmeny). (Avšak je nebude možné kvantifikovat!)
        \end{definice}

        \begin{definice}[Eliminace (nahrazování) třídových termů]
            $x, y, z, X, Y$ proměnné (3 množinové + 3 třídové), $\phi(x)$, $\psi(y)$ formule základního jazyka, $X$ zastupuje $\{x, \phi(x)\}$ a $Y$ $\{y, \phi(y)\}$.
            $$ z \in X \text{ (schéma formulí pro obecné $X$) zastupuje } z \in \{x, \phi\} \text{ nahradíme } \phi(z). $$
            $$ z = \{X\} \text{ zastupuje } z = \{x, \phi(x)\} \text{ nahradíme } (\forall u)(u \in z \Leftrightarrow \phi(u)). $$
            $$ X \in Y \text{ zastupuje } \{x, \phi(x)\} \in \{y, \phi(y)\} \text{ nahradíme } (\exists u)(u = \{x, \phi(x)\}) \land u \in \{y, \psi(y)\}. $$ 
            $$ X \in y \text{ analogicky předchozí.} $$
            $$ X = Y … … … … (\forall u)(\phi(u) \Leftrightarrow \psi(u)). $$ 
        \end{definice}

        \begin{poznamka}
            Třídy s rozšířenou formulí nepřináší díky eliminaci nic nového.
        \end{poznamka}

        \begin{definice}[Třídové operace]
            $A \cap B$ je $\{x, x \in A \land x \in B\}$, $A \cup B$ je $\{x, x \in A \lor B\}$, $A \setminus B = A - B = \{x, x \in A \land x \notin B\}$.
        \end{definice}

        \begin{definice}[Univerzální třída, doplněk]
            $\{x, x = x\}$ je tzv. univerzální třída a značí se $V$.

            Buď $A$ třída, pak (absolutní) doplněk $A$ je $V - A$, který se značí $-A$.
        \end{definice}

        \begin{definice}[Inkluze]
            $A \subseteq B$ ($A \subset B$) značí „$A$ je (vlastní) částí (= podtřídou) $B$“.
        \end{definice}

        \begin{definice}[Suma a průnik]
            $\bigcup A$ značící sumu třídy $A$ je třída $\{x, (\exists a)(a \in A \land x \in a)\}$. $\bigcap A$ značící průnik třídy $A$ je třída $\{x, (\forall a)(a \in A \rightarrow x \in a)\}$.
        \end{definice}

        \begin{lemma}
            $V$ není množina.

            \begin{dukazin}
                Cvičení.
            \end{dukazin}
        \end{lemma}

        \begin{lemma}
            Je-li $A$ třída, $a$ množina, pak $a \cap A$ je množina.

            \begin{dukazin}
                V podstatě axiom vydělení.
            \end{dukazin}
        \end{lemma}

\end{document}
