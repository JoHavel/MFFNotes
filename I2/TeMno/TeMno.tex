\documentclass[12pt]{article}                   % Začátek dokumentu
\usepackage{../../MFFStyle}                     % Import stylu

\begin{document}
\section{Úvod}
    \begin{poznamka}[Aplikace]
        Transfinitní indukce, axiom výběru (= princip maximality = Zornovo lemma)
    \end{poznamka}

    \begin{poznamka}[Cíl]
        Vybudování matematiky na pevných základech. Porozumění nekonečen. Důkaz existence nealgebraických (= transcendentních) reálných čísel. Princip kompaktnosti. Banach-Tarského paradox.
    \end{poznamka}

    \begin{poznamka}[Literatura]
        Balcar, Štěpánek -- Teorie množin\\
        Seriál PraSete\\
        Hrbáček, Jech -- Introduction to set theory\\
        Olšák -- Esence teorie množin (videa)
    \end{poznamka}

    \begin{poznamka}[Historie]
        Bernard Bolzano (český matematik, 1781-1848, pojem množina), George Cantor (německý matematik, 1845 - 1918, zavedení aktuálního nekonečna, diagonální metoda, kardinální čísla, uzavřená množina), Bertrand Russell (1902, Russellův paradox = paradox holiče = holí holič holící všechny lidi, kteří se neholí sami, sebe?) + Berriho paradox (nechť m je nejmenší přirozené číslo, které nejde definovat méně než 100 znaky), Zermelo-Fraenkel (zavedli axiomatickou teorii množin).
    \end{poznamka}

    \begin{definice}[Symboly]
        Proměnné pro množiny -- $x, y, z, x_1, x_2, …$.

        Binární predikátorový symbol $=$ a bin. relační symbol $\in$.

        Logické spojky $\neg, \lor, \land, \implies, \Leftrightarrow$.

        Kvantifikátory $\forall, \exists$.

        Závorky $\(\)\{\}\[\]$
    \end{definice}

    \begin{definice}[Formule]
        Atomické ($x = y$, $x \in y$). Jsou-li $\phi$ a $\psi$ formule, pak $\neg \phi$, $\phi \lor \psi$, $\phi \land \psi$, $\phi \implies \psi$, $\phi \Leftrightarrow \psi$ jsou formule. Je-li $\phi$ formule, $x$ proměnná, pak $(\forall x)\phi$, $(\exists x)\phi$ jsou formule. (Vázané vs. volné proměnné -- proměnné formule, které do ní lze dosadit jsou volné, proměnné formule, které do ní nelze dosadit jsou vázané). Každou formuli lze dostat konečnou posloupností aplikací výše zmíněného.
    \end{definice}

    \begin{definice}[Rozšíření jazyka]
        $x ≠ y$ značí $\neg(x=y)$, $x \notin y$ znamená $\neg(x \in y)$, $x \subseteq y$ znamená $(\forall u)(u \in x \implies u \in y)$, $x \subset y$ značí $x \subseteq y \land x ≠ y$. Dále uvidíme $\cup$, $\cap$, $\setminus$, $\{x_1, …, x_n\}$, $\O$, $\{x \in a|\phi(x)\}$.
    \end{definice}

    \begin{definice}[Axiomy logiky]
        Vysvětlují, jak se chovají implikace, kvantifikátory, rovnost, …
    \end{definice}

    \begin{definice}[Axiomy TEMNA]
        Říkají, jak se chová $\in$ a jaké množiny existují. Budeme používat Zermelo-Fraenkelovu teorii (ZF), tedy 9 axiomů (7 + 2 schémata). (Není minimální, tj. lze některé odvodit z jiných) + axiom výběru (AC) s ním se pak ZF značí ZFC.
    \end{definice}

    \begin{definice}[Axiomy ZFC]
        \ 
        \begin{enumerate}
            \item Axiom existence množiny: $(\exists x)(x=x)$.
            \item Axiom extenzionality: $(\forall z)(z \in x \Leftrightarrow z \in y) \implies x=y$.

% 9. 3. 2021

            \item Schéma axiomů vydělení: je-li $\phi(x)$ formule, která neobsahuje volnou proměnnou $z$, potom $(\forall a)(\exists z)(\forall x)(x \in z \Leftrightarrow (x \in a \land \phi(x)))$ je axiom.
            \item Axiom dvojice: $(\forall a)(\forall b)(\exists z)(\forall x)(x \in z \Leftrightarrow (x = a \lor x = b))$.
            \item Axiom sumy: $(\forall a)(\exists z)(\forall x)(x \in z \Leftrightarrow (\exists y)(x \in y \land y \in a))$.
            \item Axiom potence: $(\forall a)(\exists z)(\forall x)(x \in z \Leftrightarrow x \subseteq a)$.
            \item Schéma axiomů nahrazení\footnote{Slogan: Obraz množiny funkcí je množina.} Je-li $\psi(u, v)$ formule s volnými proměnnými $u, v$, jež nemá volné proměnné $w, z$, pak
                    $$ (\forall u)(\forall v)(\forall w)((\psi(u, v) \land \psi(u, w))\implies v=w) \implies (\forall a)(\exists z)(\forall v)(v \in z \Leftrightarrow (\exists u)(u \in a \land \psi(u, v))) $$
                    je axiom.
            \item Axiom fundovanosti (regularity): $(\forall a)(a ≠ \O \implies (\exists x)(x \in a \land x \cap a = \O))$.
        \end{enumerate}

        Později ještě:
        \begin{itemize}
            \item Axiom nekonečna
            \item Axiom výběru
        \end{itemize}
    \end{definice}

    \begin{definice}[Značení]
        $\{x | x \in a \land \phi(x)\}$, zkráceně $\{x \in a | \phi(x)\}$ je množina z axiomů vydělení.
    \end{definice}

    \begin{definice}[Průnik, množinový rozdíl, prázdná množina]
        $a \cap b = \{x | x \in a \land x \in b\}$.

        $a \setminus b = a - b = \{x | x \in a \land x \notin b\}$.

        $\O = \{x | x \in a \land x ≠ x\}$. ($\O$ je díky prvním třem axiomům dobře definována.)
    \end{definice}

    \begin{definice}[Neuspořádaná a uspořádaná dvojice]
        $\{a, b\}$ je neuspořádaná dvojice, $\{a\}$ znamená $\{a, a\}$.

        $(a, b) = <a, b> = \{\{a\}, \{a, b\}\}$ je uspořádaná dvojice.
    \end{definice}

    \begin{lemma}
        $(x, y) = (u, v) \Leftrightarrow (x = u \land y = v)$.

        \begin{dukazin}
            $\Leftarrow$ $x=u$, pak $\{x\} = \{u\}$ z axiomu extenzionality, stejně tak $\{x, y\} = \{u, v\}$, a tedy $\{\{x\}, \{x, y\}\} = \{\{u\}, \{u, v\}\}$.

            $\implies$ $\{\{x\}, \{x, y\}\} = \{\{u\}, \{u, v\}\}$ pak $\{x\} = \{u\}$ nebo $\{x\} = \{u, v\}$, každopádně $x = u$. Nyní $\{u, v\} = \{x\}$ nebo $\{u, v\} = \{x, y\}$ tedy $v = x$ nebo $v = y$. Pokud $v = y$, tak jsme skončili, pokud $v = x$ pak $v = u = x = y$.
        \end{dukazin}
    \end{lemma}

    \begin{definice}[Potenční množina]
        $©P(a) = \{x | x \subseteq a\}$ je potenční množina (potence) $a$ (z axiomu potence).
    \end{definice}

    \begin{definice}[Uspořádaná $n$-tice]
        Jsou-li $a_1, a_2, …, a_n$ množiny, definujeme uspořádanou $n$-tici $(a_1, a_2, …, a_n) = <…>$ následovně: $(a_1) = a_1$, je-li definovaná $(a_1, …, a_k)$, pak $(a_1, …, a_k, a_{k+1}) = ((a_1, a_2, …, a_k), a_{k+1})$.
    \end{definice}

    \begin{lemma}
        $(a_1, a_2, …, a_n) = (b_1, b_2, …, b_n) \Leftrightarrow (a_1 = b_1 \land a_2 = b_2 \land … \land a_n = b_n)$.

        \begin{dukazin}
            Domácí cvičení.
        \end{dukazin}
    \end{lemma}

    \begin{definice}[Značení]
        $$ \bigcup a = \{x | (\exists y)(x \in y \land y \in a\}. $$ 
        
        \begin{definicein}
            Pro $a = \{b, c\}$ definujeme $b \cup c = \bigcup a$.
        \end{definicein}
    \end{definice}

    \begin{definice}[Neuspořádaná $n$-tice]
            Neuspořádanou $n$-tici $\{a_1, …, a_n\}$ ($n$-prvkovou množinu) definujeme rekurzivně: je-li definováno $\{a_1, …, a_k\}$, pak $\{a_1, …, a_k, a_{k+1}\} = \{a_1, …, a_k\} \cup \{a_{k+1}\}$.
    \end{definice}

% 16. 3. 2021

    \begin{poznamka}
        Axiom nahrazení se využívá v: transfinitní rekurzi, definici $\omega + \omega$, větě o typu dobrého uspořádání, Zornově lemmatu (tj. axiom výběru).
    \end{poznamka}

    \begin{priklad}
        Ukažte, že axiom fundovanosti zakazuje existenci konečných cyklů relace $\in$ (tj. takových množin $y$, pro které $y \in . \in … \in . \in y$).

        \begin{dusledekin}
            Všechny množiny lze vygenerovat z $\O$ pomocí operací $\bigcup$ a $©P$ (zhruba).
        \end{dusledekin}
    \end{priklad}

    \subsection{Třídy}
        \begin{definice}
            Nechť $\phi(x)$ je formule, pak $\{x, \phi(x)\}$ (čteme třída všech $x$, pro které platí $\phi(x)$), tzv. třídový term, se nazývá třída (určená formulí $x$).
        \end{definice}

        \begin{dusledek}
            Pokud $\phi(x)$ je tvaru $x \in a \land \phi(x)$, pak je $\{x, \phi(A)\}$ množina z axiomu vydělení. Obdobně pro axiom dvojice, sumy, …
        \end{dusledek}

        \begin{poznamka}
            Je-li $y$ množina, pak $y$ má stejné prvky jako třída ${x, x \in y \land x = x}$.
        \end{poznamka}

        \begin{poznamka}[Vlastní třídy]
            Existují i třídy (tzv. vlastní), které nejsou množiny (např. třída všech množin).
        \end{poznamka}

        \begin{definice}[Rozšíření jazyka]
            Ve formulích na místech volných proměnných připustíme i Třídové termy a proměnné pro třídy (psané velkými písmeny). (Avšak je nebude možné kvantifikovat!)
        \end{definice}

        \begin{definice}[Eliminace (nahrazování) třídových termů]
            $x, y, z, X, Y$ proměnné (3 množinové + 3 třídové), $\phi(x)$, $\psi(y)$ formule základního jazyka, $X$ zastupuje $\{x, \phi(x)\}$ a $Y$ $\{y, \phi(y)\}$.
            $$ z \in X \text{ (schéma formulí pro obecné $X$) zastupuje } z \in \{x, \phi\} \text{ nahradíme } \phi(z). $$
            $$ z = \{X\} \text{ zastupuje } z = \{x, \phi(x)\} \text{ nahradíme } (\forall u)(u \in z \Leftrightarrow \phi(u)). $$
            $$ X \in Y \text{ zastupuje } \{x, \phi(x)\} \in \{y, \phi(y)\} \text{ nahradíme } (\exists u)(u = \{x, \phi(x)\}) \land u \in \{y, \psi(y)\}. $$ 
            $$ X \in y \text{ analogicky předchozí.} $$
            $$ X = Y … … … … (\forall u)(\phi(u) \Leftrightarrow \psi(u)). $$ 
        \end{definice}

        \begin{poznamka}
            Třídy s rozšířenou formulí nepřináší díky eliminaci nic nového.
        \end{poznamka}

        \begin{definice}[Třídové operace]
            $A \cap B$ je $\{x, x \in A \land x \in B\}$, $A \cup B$ je $\{x, x \in A \lor B\}$, $A \setminus B = A - B = \{x, x \in A \land x \notin B\}$.
        \end{definice}

        \begin{definice}[Univerzální třída, doplněk]
            $\{x, x = x\}$ je tzv. univerzální třída a značí se $V$.

            Buď $A$ třída, pak (absolutní) doplněk $A$ je $V - A$, který se značí $-A$.
        \end{definice}

        \begin{definice}[Inkluze]
            $A \subseteq B$ ($A \subset B$) značí „$A$ je (vlastní) částí (= podtřídou) $B$“.
        \end{definice}

        \begin{definice}[Suma a průnik]
            $\bigcup A$ značící sumu třídy $A$ je třída $\{x, (\exists a)(a \in A \land x \in a)\}$. $\bigcap A$ značící průnik třídy $A$ je třída $\{x, (\forall a)(a \in A \rightarrow x \in a)\}$.
        \end{definice}

        \begin{lemma}
            $V$ není množina.

            \begin{dukazin}
                Cvičení.
            \end{dukazin}
        \end{lemma}

        \begin{lemma}
            Je-li $A$ třída, $a$ množina, pak $a \cap A$ je množina.

            \begin{dukazin}
                V podstatě axiom vydělení.
            \end{dukazin}
        \end{lemma}

% 23. 3. 2021

        \begin{definice}[Kartézský součin tříd]
            Kartézský součin tříd $A$, $B$, značený $A \times B$, je třída $\{(a, b), a \in A, b \in B\}$ (zkrácený zápis pro $\{x, (\exists a)(\exists b)(x = (a, b) \land a \in A \land b \in B)\}$).
        \end{definice}

        \begin{lemma}
            Jsou-li $x, y$ množiny, pak $x \times y$ je množina.

            \begin{dukazin}[„Vnoření a vydělení“]
                Stačí dokázat, že $x\times y \subseteq ©P(©P(x \cup y))$ (vpravo je množina z axiomu potence, součin pak vydělíme): Pokud $u \in x$ a $v \in y$, pak $\{u\}, \{u, v\} \subseteq x \cup y$, tedy $\{u\}, \{u, v\} \in ©P(x \cup y)$. Tedy $\{\{u\}, \{u, v\}\} \subseteq ©P(x \cup y)$, tj. $\in ©P(©P(x \cup y))$.
            \end{dukazin}
        \end{lemma}

        \begin{definice}[Mocnina]
            $X$ třída, pak $X^1 = X$ a $X^{n+1} = X^n\times X$. (Tj. $X^n$ je třída všech uspořádaných $n$-tic s prvky v $X$.)
        \end{definice}

        \begin{pozorovani}
            $V = V^1 \supset V^2 \supset …$.
        \end{pozorovani}

        \begin{priklad}
            Obecně neplatí $X \times X^2 = X^3$.
        \end{priklad}

    \subsection{relace}
        \begin{definice}[Relace]
            Třída $R$ je (binární) relace pokud, $R \subseteq V \times V$. ($n$-ární relace, pokud $R \subseteq V^n$.)

            $xRy$ je zkratka za $(x, y) \in R$.
        \end{definice}

        \begin{definice}[Definiční obor, obor hodnot, zúžení]
            Je-li $X$ relace (libovolná třída), pak $\Dom(X) := \{u, (\exists v)((u, v) \in X)\}$ je definiční obor třídy $X$.

            Je-li $X$ relace (libovolná třída), pak $\Rang(X) := \{v, (\exists u)((u, v) \in X)\}$ je obor hodnot třídy $X$.

            Je-li navíc $Y$ třída, pak $X''Y$ (nebo také $X[Y]$) $:= \{z, (\exists y)(y \in Y \land (y, z) \in X)\}$ je obraz třídy $Y$ třídou $X$.

            Je-li navíc $Y$ třída, pak $X\upharpoonright Y := \{(y, z), y \in Y \land (y, z) \in X)\}$ je zúžení třídy $X$ na třídu $Y$ (nebo také parcializace).
        \end{definice}
       
        \begin{lemma}
            Je-li $x$ množina, $Y$ třída, pak $\Dom(x)$, $\Rang(x)$, $x \upharpoonright Y$, $x''Y$ jsou množiny.

            \begin{dukazin}[Vnoření a vydělení]
                $\Dom(x) \subseteq \bigcup(\bigcup x))$. Když $u \in \Dom(x)$, $\exists v$, že $(u, v) \in x$, $u \in \{u\} \in (u, v) \in x$, tedy $\{u\} \in \bigcup x$, tj. $u \in \bigcup(\bigcup x))$.

                $\Rang(x) \subseteq \bigcup(\bigcup x))$. (Analogicky.) $x \upharpoonright Y \subseteq x$, $x''Y \subseteq \Rang(x)$.
            \end{dukazin}
        \end{lemma}

        \begin{definice}[Inverzní relace, složení relací]
            $R$, $S$ relace, pak $R^{-1} := \{(u, v), (v, u) \in R\}$ je relace inverzní k $R$. Relace $R \circ S := \{(u, w), (\exists v)(uRv \land vSw)\}$ je složení relací $R$, $S$.
        \end{definice}

        \begin{definice}[Zobrazení, na, do, prosté]
            Relace $F$ je zobrazení (funkce), pokud $(\forall u)(\forall v)(\forall w)(((u, v) \in F \land (u, w) \in F)\implies v = w)$.

            Zkracujeme $(u, v) \in F$ na $F(u) = v$.

            $F$ je zobrazení třídy $X$ do (na) třídy $Y$ $F:X \rightarrow Y$, pokud $\Dom(F) = X$ a $\Rang(F) \subseteq Y$ ($\Rang(F) = Y$).

            Zobrazení $F$ je prosté, pokud inverzní relace $F^{-1}$ je zobrazení. (Zřejmě je pak i $F^{-1}$ prosté).
        \end{definice}

        \begin{definice}[Zkratka]
            $A$ třída, $\phi$ formule, pak $(\exists x \in A)\phi$ je zkratka za $(\exists x)(x \in A \land \phi)$ a $(\forall x \in A)\phi$ je zkratka za $(\forall x)(x \in A \implies \phi)$.

            Obraz (vzor) třídy $X$ zobrazením $F$ je $F[X]$ místo $F''X$ ($F^{-1}[X]$ místo $F^{-1''}X$).
        \end{definice}

        \begin{definice}
            $A$ třída, $a$ množina, pak $^aA := \{f, f: a \rightarrow A\}$ je třída všech zobrazení $a$ do $A$.

            \begin{dukazin}
                Axiom nahrazení říká, že $\Rang f$ je množina, $f \subseteq a \times \Rang(f)$, libovolné $f$ je tedy množina a tato třída je dobře definována.
            \end{dukazin}
        \end{definice}

        \begin{dusledek}
            $^{\O}y = \{\O\}$, $^x\O = \O$ ($x ≠ \O$).
        \end{dusledek}

        \begin{lemma}
            Pro libovolné množiny $x, y$ je $^xy$ množina. Je-li $x ≠ \O$, $Y$ je vlastní třída, pak $^xY$ je vlastní třída.

            \begin{dukazin}
                $f \in ^xy…f: x \rightarrow y…f \subseteq x \times y…f \in ©P(X\times y)$ $\implies ^xy \subseteq ©P(x\times y)$.

                Pro každé $y \in Y$ definujeme konstantní zobrazení $k_y: x \rightarrow Y$ tak, že $(\forall u \in x)(k_y(u) = y)$. Nechť $K = \{k_y, y \in Y\} \subseteq ^xY$. Sporem: pokud $^xY$ je množina, pak $K$ je množina. Použijeme axiom nahrazení $F: K \rightarrow Y$, $F(k_y) = y$, tj. (protože $F$ zobrazuje na $Y$) $Y$ je množina. \lightning.
            \end{dukazin}
        \end{lemma}

\section{Uspořádání}
    \begin{definice}[Reflexivní, antireflexivní, symetrická, slabě antisymetrická, antisymetrická, trichotomická, tranzitivní]
        Relace $R (\subseteq V \times V)$ je na třídě $A$ reflexivní (antireflexivní, symetrická, slabě antisymetrická, antisymetrická, trichotomická, tranzitivní), pokud … (…, …, $(\forall x \in A)(\forall y \in A)((xRy \land yRx) \implies x = y)$, $(\forall x \in A)(\forall y \in A)\neg(xRy \land yRx)$, $(\forall x \in A)(\forall y \in A)(xRy \lor yRx \lor x = y)$, …).
    \end{definice}

    \begin{pozorovani}
        Tyto vlastnosti jsou dědičné (tzn. platí i na každé podtřídě $B \subseteq A$).
    \end{pozorovani}

    \begin{definice}[Uspořádání, porovnatelné]
        Řekneme, že relace $R$ je uspořádání na třídě $A$, je-li $R$ na $A$ reflexivní, slabě symetrická, tranzitivní.

        $x, y \in A$ jsou porovnatelné relací $R$, pokud $xRy \lor yRx$.
    \end{definice}

    \begin{definice}[Značení]
        $x ≤_R y$ znamená $xRy$, $x$ je menší nebo rovno $y$ vzhledem k $R$.
    \end{definice}

    \begin{definice}[Lineární uspořádání]
        Uspořádání $R$ je lineární, je-li $R$ trichotomická relace.
    \end{definice}

    \begin{definice}[Ostré uspořádání]
        Relace $R'$ je ostré uspořádání, je-li $R' = R - \id$ a $R$ je uspořádání.

        Píšeme $x <_R y$, když $xR'y$.
    \end{definice}

    \begin{definice}
        Nechť $R$ je uspořádání na třídě $A$, $X \subseteq A$. Říkáme, že $a \in A$ je (vzhledem k $R$, $A$): majoranta (horní mez, horní závora) třídy $X$, pokud $(\forall x \in X)(x ≤_R a)$, maximální prvek třídy $X$ pokud $a \in X \land (\forall x \in X)(\neg a <_R x)$, největší prvek třídy $X$ pokud $a \in X \land (\forall x \in X)(x ≤_R a)$, supremum třídy $X$, pokud $a$ je nejmenší majoranta.

        Obdobně (dolní mez, dolní závora), minimální, nejmenší, infimum.
    \end{definice}

    \begin{pozorovani}
        Největší $\implies$ maximální. V lineárním uspořádání i naopak.
    \end{pozorovani}

    \begin{definice}[Shora, zdola omezená množina, dolní, horní množina]
        $X$ je shora omezená v $A$, pokud existuje majorita $X$ v $A$. $X$ je dolní množina v $A$, pokud $\forall x \in X)(\forall y \in A)(y ≤_Rx \implies y \in X)$. $x \in A$, pak $(\leftarrow, x]$ je $\{y, y \in A \land y ≤_R x\}$ hlavním ideálem určeným $x$.

        Obdobně zdola uzavřená a horní množina. Lze definovat i pro třídy, ale to se nedělá.
    \end{definice}

    \begin{pozorovani}
        $R$ je uspořádání na $A$, pak pro libovolné $x, y \in A$ platí $x ≤_R y \Leftrightarrow (\leftarrow, x]\subseteq(\leftarrow, y]$.
    \end{pozorovani}

    \begin{definice}[Dobré uspořádání]
        Uspořádání $R$ na třídě $A$ je dobré, pokud každá neprázdná podmnožina $u \subseteq A$ má nejmenší prvek.
    \end{definice}

\section{Srovnávání množin}
    \begin{definice}
        Množiny $x, y$ mají stejnou mohutnost (jsou ekvivalentní), $x \approx y$, pokud existuje prosté zobrazení $x$ na $y$.
    \end{definice}

% 6. 4. 2021

    \begin{definice}
        Množina $x$ má mohutnost menší nebo rovnou mohutnosti $y$, $x \preceq y$, pokud existuje prosté zobrazení $x$ do $y$. (Také říkáme $x$ je subvalentní $y$). $x$ má mohutnost menší než $y$, $x \prec y$, pokud $x \preceq y \land \neg(x \approx y)$.
    \end{definice}

    \begin{pozorovani}
        $x \subseteq y \implies x \preceq y$. $x \subset y \implies x \preceq y$, ale ne nutně $x \prec y$ (viz přirozená čísla).
    \end{pozorovani}

    \begin{lemma}
        Jsou-li $x, y, z$ množiny, pak
        $$ 1) x \approx x. \qquad (\id) $$
        $$ 2) x \approx y \implies y \approx x. \qquad (^-1) $$
        $$ 3) (x \approx y \land y \approx z) \implies x \approx z. \qquad (\circ) $$
        $$ 4) x \preceq x. \qquad (\id) $$
        $$ 5) (x \preceq y \land y \preceq z) \implies x \preceq z. \qquad (\circ) $$
    \end{lemma}

    \begin{pozorovani}
        $(x \approx y) \implies (x \preceq y \land y \preceq x)$.
    \end{pozorovani}

    \begin{definice}
        Zobrazení $H: ©P(x) \rightarrow ©P(x)$ je monotónní vzhledem k inkluzi, pokud pro každé podmnožiny $u, v \subseteq x$ platí $u \subseteq v \implies H(u) \subseteq H(v)$.
    \end{definice}

    \begin{lemma}
        Je-li $H: ©P(x) \rightarrow ©P(x)$ zobrazení monotónní vzhledem k inkluzi, pak existuje podmnožina $c \subseteq c$, že $H(c) = c$.

        \begin{poznamkain}
            Speciální případ Knaster-Tarski, kteří předpokládají jen úplný svaz.
        \end{poznamkain}

        \begin{pozorovaniin}
            $A \subseteq ©P(x) \implies \sup_{\subseteq} A = \bigcup A$.
        \end{pozorovaniin}

        \begin{dukazin}
            Nechť $A = \{u, u \subseteq x \land u \subseteq H(u)\}$. $c = \bigcup A$. $c \subseteq x$ zřejmě, $u \in A \implies u \subseteq c \land u \subseteq H(u) \subseteq H(c)$. Tedy $H(c)$ je majoranta $A$, tedy $c \subseteq H(c)$. Z monotonie $H$ je $H(c) \subseteq H(H(c))$, tedy $H(c) \in A$, tedy $H(c)\subseteq c$.
        \end{dukazin}
    \end{lemma}

    \begin{veta}[Cantor-Bernstein]
        $(x \preceq y \land y \preceq x) \implies x \approx y$.
        
        \begin{dukazin}
            Nechť $f: x \rightarrow y, g: y \rightarrow x$ jsou prostá zobrazení. Uvažujeme 'indukovaná' zobrazení $(\vec{f}) : ©P(x) \rightarrow ©P(y)$, $u \mapsto f[u]$. Definujeme $H: ©P(x) \rightarrow ©P(x)$ tak, že pro $u \subseteq u$ $H(u) = X - g[Y-f[u]$. $H$ je monotónní vzhledem k inkluzi: $u_1 \subseteq u_2 \implies f[u_1] \subseteq f[u_2] \implies y - f[u_1] \supseteq f[u_2] \implies … \implies H(u_1) \subseteq H(u_2)$. Podle lemmatu o pevném bodě existuje $c: H(c) = c$.

            Tedy $c = x - g[y - f(c)]$, tj. $x - c = g[y - f[c]$. Tedy $g^{-1}|_{(x - c)}$ je prosté zobrazení $x - c$ na $y - f[c]$. Tedy definujeme $h: x \rightarrow y$, $h(a) = f(a)$ pro $a \in c$, $H(a) = g^{-1}(a)$ jinak.
        \end{dukazin}
    \end{veta}

    \begin{lemma}
        $x, y, z, x_1, y_1$ množiny, pak $1) x\times y \approx y \times x$, $2) x \times (y \times z) \approx (x \times y) \times z$, $3) (x \approx x_1 \land y \approx y_1) \implies (x\times y) \approx (x_1 \times y_1)$, $4) x \approx y \implies ©P(x) \approx ©P(y)$, $5) ©P(x) \approx ^x2 = ^x\{\O, \{\O\}\}$.

        \begin{dukazin}
            1-4) Triviální. Pro 5) definujeme charakteristickou funkci a je to triviální.
        \end{dukazin}
    \end{lemma}

% 13. 4. 2021

    \begin{definice}[Konečná množina (Tarski)]
        Množina $x$ je konečná (značíme $\Fin(x)$), pokud každá neprázdná podmnožina její potenční podmnožiny má vzhledem k inkluzi maximální prvek.
    \end{definice}

    \begin{pozorovani}
        $Fin(x)$ právě tehdy, když každá neprázdná podmnožina $©P(x)$ má minimální prvek vzhledem k $\subseteq$.

        \begin{dukazin}
            $d: ©P(x) \rightarrow ©P(x), y \mapsto x\setminus y$ vše obrátí.
        \end{dukazin}
    \end{pozorovani}

    \begin{definice}[Dedekindovsky konečná množina]
        Množina $a$ je dedekindovsky konečná, pokud má větší mohutnost než každá její vlastní podmnožina. (Tj. neexistuje prosté zobrazení $a$ do $b$.)
    \end{definice}

    \begin{lemma}
        Je-li $a$ konečná, potom je i dedekindovsky konečná.

        \begin{dukazin}
            Víme $b \subset a \implies b \preceq a$. Chceme tedy $a ≠ b$. Sporem: Předpokládejme, že $b \subset a \land b \approx a$, $Y := \{b, b\subset a \land b \approx a\}$. Víme, že $Y$ je neprázdná, $Y \subseteq ©P(a)$. Nechť $c$ je minimální prvek $Y$ vzhledem k inkluzi. Existuje tedy bijekce $f: a \rightarrow c$. Nechť $d = f[c]$. Zřejmě $f|_c: c \rightarrow d$ je bijekce, tedy $c \approx d$ a z tranzitivity $d \approx a$. Tedy $d \in Y$. Ale $d \subset c$, jelikož $\exists s \in c, f^{-1}(s) \in a \setminus c$, tj. $s \notin d$. To je ale spor s volbou $c$.
        \end{dukazin}
    \end{lemma}

    \begin{poznamka}
        V ZF je opravdu konečnost silnější než dedekindovská konečnost.
    \end{poznamka}

    \begin{poznamka}[Další definice konečnosti]
        Existuje lineární uspořádání, které je dobré a jeho inverze je také dobré uspořádání.

        Existuje lineární uspořádání a každá dvě lineární uspořádání jsou izomorfní.

        Potenční množina od potenční množiny je dedekindovsky konečná.
    \end{poznamka}

    \begin{veta}
        1) Je-li $a$ konečná množina uspořádaná relací (částečným uspořádáním) $≤$, pak každá neprázdná podmnožina $b \subset a$ má maximální i minimální prvek (vzhledem k $≤$).
        2) Každé lineární uspořádání na konečné množině je dobré (každá podmnožina má nejmenší prvek).

        \begin{dukazin}
            1) $\implies$ 2): každá podmnožina má minimální prvek, ale ten je při lineárním uspořádání konečný.

            1): Pro každé $x \in a$ uvažujeme $(\leftarrow, x] := \{y \in a | y ≤ x\}$. To jsou zřejmě podmnožiny $a$. Nechť $u = \{(\leftarrow, x], x \in b\}$. $b \in u \subseteq ©P(a)$, tedy z definice konečnosti má maximální prvek $(\leftarrow, m]$. $m \in b$ je maximální prvek v $b$. Minimum otočením $≤$.
        \end{dukazin}
    \end{veta}

    \begin{definice}[Izomorfismus]
        $F$ je zobrazení $A_1$ do $A_2$, $R_1, R_2$ jsou relace. $F$ je izomorfismus tříd $A_1$ a $A_2$ vzhledem k $R_1$ a $R_2$, pokud $F$ je bijekce (prosté zobrazení na) a $\forall x, y \in A_1: xR_1y \Leftrightarrow F(x)R_2F(y)$.
    \end{definice}

    \begin{definice}[Počátkové vnoření]
        $A$ množina uspořádaná relací $R$, $B$ relací $S$, potom zobrazení $F: A \rightarrow B$ je počátkové vnoření $A$ do $B$, pokud $A_1 = \Dom(F)$ je dolní množina $A$, $B_1 = \Rang(F)$ je dolní podmnožina $B$ a $F$ je izomorfismus $A_1$ a $B_1$ vzhledem k $R, S$.
    \end{definice}

    \begin{lemma}
        Nechť $F, G$ jsou počátková vnoření dvou dobře uspořádaných množin $A$, $B$. Pak $F \subseteq G \lor G \subseteq F$.

        \begin{dukazin}
            Nechť $R$ resp. $S$ je dané uspořádání $A$ resp. $B$. Víme, že $\Dom(F), \Dom(G)$ jsou dolní podmnožiny. $R$ je lineární (jelikož je dobré), tedy $\Dom(F) \subseteq \Dom(G)$ nebo $\Dom(G)\subseteq \Dom(F)$. BÚNO první možnost. Sporem dokážeme, že $(\forall x \in \Dom(F))(F(x) = G(x))$.

            Nechť $x$ je vzhledem k $R$ nejmenší prvek množiny $\{z, z\in A \land F(z)≠G(z)\}$. Tedy pro každé $y <_R x$ je $F(y) = G(y)$. Z linearity $S$ $F(x) <_S G(x)$ nebo $F(x) >_S G(x)$. BÚNO první možnost. Nechť $b = F(x)$. $z \in \Dom(G)$, 1) $z <_R x \implies G(z) = f(z) <_S b$, 2) $z ≥_R x \implies G(z) ≥_s G(x) >_S b$. Tedy $b \notin \Rang G$ a $\Rang G$ není dolní množina. \lightning.
        \end{dukazin}
    \end{lemma}

% 20. 4. 2021

    \begin{veta}[O porovnávání dobrých uspořádání]
        $A$ je množina dobře uspořádaná $R$, $B$ je množina dobře uspořádaná $S$, pak existuje právě jedno zobrazení $F$, které je izomorfismus $A$ a dolní množiny $B$ nebo izomorfismus dolní množiny $A$ a $B$.

        \begin{dukazin}
            $P$ množina všech počátkových vnoření z $A$ do $B$. $F := \bigcup P$ je zobrazení: když $(x, y_1)$ a $(x, y_2)$ existují počátková vnoření $F_1, F_2$ taková, že $(x, y_1) \in F_1$ a $(x, y_2) \in F_2$. Podle lemma: $F_1 \subseteq F_2 \lor F_2 \subseteq F_1$, tedy $y_1 = y_2$.

            $F$ je počátkové vnoření: když $x_1 <_R x_2 \in \Dom(F)$, pak existuje počátkové vnoření $F' \in P$ takové, že $x_2 \in \Dom(F')$, tedy $x_1 \in \Dom(F') \subseteq \Dom F$, tedy $\Dom(F)$ je dolní množina. Symetricky pro $\Rang(F)$.

            TODO.

            $\Dom(F) = A \lor \Rang(F) = B$: Sporem: $A - \Dom(F)$, $B - \Dom(F)$ jsou neprázdné. Mají tedy nejmenší prvky $a, b$. Definujeme $F' := F \cup \{(a, b)\}$. $F'$ je počátkové vnoření, přitom $F' \supseteq F$, \lightning.
        \end{dukazin}
    \end{veta}

    \begin{veta}
        $a$ je konečná množina, pak každé dvě lineární uspořádání na $a$ jsou izomorfní.

        \begin{dukazin}
            $r, s$ dvě lineární uspořádání $a$. Podle věty výše jsou $r, s$ dobrá, tedy $a, r$ je izomorfní dolní množině b v $a, s$ nebo naopak. Tzn. $a \approx b$, tedy z definice konečnosti $b = a$.
        \end{dukazin}
    \end{veta}

    \begin{lemma}[Zachovávání konečnosti]
        $$ 1) (\Fin(x) \land y \subseteq x) \implies \Fin(y). $$
        $$ 2) (\Fin(x) \land y \approx x) \implies \Fin(y). $$
        $$ 3) (\Fin(x) \land y \preceq x) \implies \Fin(y). $$ 


        \begin{dukazin}
            $1)$ z definice. $2)$ $©P(x) \approx ©P(y)$, dokonce jsou izomorfní vzhledem k $\subseteq$. $3)$ z $1)$ a $2)$.
        \end{dukazin}
    \end{lemma}
    
    \begin{lemma}[Sjednocení konečných množin]
        $$ 1) (\Fin(x) \land \Fin(y)) \implies \Fin(x \cup y). $$
        $$ 2) \Fin(x) \implies (\forall y)(\Fin(x \cup \{y\})). $$ 

        \begin{dukazin}
            1) $w \subseteq ©P(x \cup y)$ neprázdná. $©P(x) \supseteq w_1 = \{u, (\exists t \in w)(u = t \cap x)\} ≠ \O$. Tedy $w_1$ má maximální prvek ($v_1$). Nechť $\O ≠ w_2 := \{u, (\exists t \in w)(t \cap x = v_1 \land t \cap y = u)\} \subseteq ©P(y)$. Tedy $w_2$ má maximální prvek $v_2$. $v_1 \cup v_2$ je maximální prvek $w$.

            2): důsledek 1).
        \end{dukazin}
    \end{lemma}

    \begin{definice}[Třída konečných množin]
        Třída všech konečných množin je $\Fin := \{x, \Fin(x)\}$.
    \end{definice}

    \begin{veta}[Princip indukce pro konečné množiny]
        Je-li $X$ třída, pro kterou platí
        $$ 1) \O \in X, \qquad 2) x \in X \implies (\forall y)(x \cup \{y\} \in X), $$
        potom $\Fin \subseteq X$.

        \begin{dukazin}[Sporem]
            Pokud $x \in \Fin \setminus X$, pak označme $w = \{v, v \subseteq x \land v \in X\} = ©P(x) \cap X$. Potom z definice konečnosti má $w$ (která obsahuje minimálně $\O$) maximální prvek $v_1$. $v_0 \subseteq X, v_0 ≠ x$, tedy $\exists y \in x \setminus v_0$ a $v_0 \subseteq v_0 \cup \{y\} \in X$, tím pádem jsme našli větší prvek než $v_0$, který je v $w$, \lightning.
        \end{dukazin}
    \end{veta}

    \begin{lemma}
        $$ \Fin(x) \implies \Fin(©P(x)). $$

        \begin{dukazin}
            Indukcí přes konečné množiny. Nechť $X = \{x, \Fin(©P(x))\}$. 1) $\O \in X$, protože $©P(\O) = \{\O\}$ je konečná. 2) Nechť $x \in X$, $y$ množina. Chceme, aby $x \cup \{y\} \in X$. BÚNO $y \notin x$. Rozdělíme $©P(x \cup \{y\})$ na části $©P(x)$ a $z:=©P(x\cup\{y\}) \setminus ©P(x)$. Platí $©P(x) \approx z$: pro $u \in ©P(x)$ definujeme $f(u) = u \cup \{y\}$. Podle IP $\Fin(©P(x))$, podle lemma $\Fin(z)$ a $\Fin(z \cup ©P(x))$.
        \end{dukazin}
    \end{lemma}

    \begin{dusledek}
        $\Fin(x) \land \Fin(y) \implies \Fin(x \times y)$.
    \end{dusledek}

    \begin{lemma}[Sjednocení konečně mnoha konečných množin]
        $(\Fin(a) \land (\forall b \in a)(\Fin(b))) \implies \Fin(\bigcup a)$.

        \begin{dukazin}[Indukcí]
            Nechť $X = \{x, x \subseteq \Fin \implies \Fin(\bigcup x)\}$. 1) $\O \in X$, protože $\bigcup \O = \O$ je konečná. 2) Nechť $x \in X$, $y$ množina. Nechť $x \cup \{y\} \subseteq \Fin$, speciálně $x \subseteq \Fin$. $\bigcup (x \cup \{y\}) = (\bigcup x)\cup y$, což je podle IP a lemmatu o sjednocení konečné.
        \end{dukazin}
    \end{lemma}

    \begin{dusledek}[Dirichletův princip pro nekonečné množiny]
        Je-li nekonečná množina sjednocením konečně mnoha množin, pak alespoň jedna z nich je nekonečná.
    \end{dusledek}

\end{document}
