\documentclass[12pt]{article}					% Začátek dokumentu
\usepackage{../../MFFStyle}					    % Import stylu



\begin{document}

\section{Latinská abeceda, výslovnost, přízvuk, délka hlásek}
Latina má tu výhodu, že píše latinkou (i když původně pouze velkými písmeny, malými až ve středověku), používala se dlouho na území Evropy a stále se slova z ní vyskytují v jazycích (včetně češtiny).

    Latina je navíc velmi pravidelná a má málo výjimek. 

    V archaické latině se vyskytovala jen jedna hláska (anglické w), později se oddělila na V a U, avšak zápis zůstal V (U v majuskulích (velkých písmenech) tedy „neexistuje“). Stejně jako J (nahrazuje se I).

    K sice existuje, ale zachovalo se jen v několika slovech před „a“ (Kalendae, Karthágo).

    Y a Z jsou pouze ve slovech přejatých z řečtiny.

    Existuje i CH (ve slovníku pod písmenem C, ale vyslovuje se CH).

    Samohlásky existují krátké a dlouhé (dlouhé se značí overline, ale já si je budu značit čárkou, protože overline se píše pomalu).

    Existují i dvojhlásky: ae, oe, ei, ui, au, eu.

    Výslovnost se liší podle národnosti. My budeme používat středoevropskou výslovnost, protože se na základě ní přejímala slova do češtiny (viz iústitia by se četla justitia, ale v SE výslovnosti je to justicia).

    Čtení: ae[e:] a oe[e:], c[k] před a, o, u, souhláskou, nebo na konci slova. Jinak c[c]. qu[kv] (q se nikdy nevyskytuje bez následujícího u). ti[ci] pokud je následována samohláskou (výjimkou jsou: ?), jinak se čte [ti]. s[z] mezi samohlásky, jinak [s].















\newpage
\section{Slovníček}
    iánua = dveře\\
    iústitia = spravedlnost\\



\end{document}
