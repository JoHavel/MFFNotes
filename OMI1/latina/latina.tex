\documentclass[12pt]{article}					% Začátek dokumentu
\usepackage{../../MFFStyle}					    % Import stylu



\begin{document}

\section{Latinská abeceda, výslovnost, přízvuk, délka hlásek}
Latina má tu výhodu, že píše latinkou (i když původně pouze velkými písmeny, malými až ve středověku), používala se dlouho na území Evropy a stále se slova z ní vyskytují v jazycích (včetně češtiny).

    Latina je navíc velmi pravidelná a má málo výjimek. 

    V archaické latině se vyskytovala jen jedna hláska (anglické w), později se oddělila na V a U, avšak zápis zůstal V (U v majuskulích (velkých písmenech) tedy „neexistuje“). Stejně jako J (nahrazuje se I).

    K sice existuje, ale zachovalo se jen v několika slovech před „a“ (Kalendae, Karthágo).

    Y a Z jsou pouze ve slovech přejatých z řečtiny.

    Existuje i CH (ve slovníku pod písmenem C, ale vyslovuje se CH).

    Samohlásky existují krátké a dlouhé (dlouhé se značí overline, ale já si je budu značit čárkou, protože overline se píše pomalu).

    Existují i dvojhlásky: ae, oe, ei, ui, au, eu.

    Výslovnost se liší podle národnosti. My budeme používat středoevropskou výslovnost, protože se na základě ní přejímala slova do češtiny (viz iústitia by se četla justitia, ale v SE výslovnosti je to justicia).

    Čtení: ae[e:] a oe[e:], c[k] před a, o, u, souhláskou, nebo na konci slova. Jinak c[c]. qu[kv] (q se nikdy nevyskytuje bez následujícího u). ti[ci] pokud je následována samohláskou (výjimkou jsou: ?), jinak se čte [ti]. s[z] mezi samohlásky, jinak [s].


\section{Skloňování a časování}
    \subsection{Skloňování obecně}
        1. 5 pádů (nominativ, genitiv, dativ, akuzativ, vokal) je „shodných“ s českými pády. Ablativ (6. pád v latině, 8. v Indoevropské historii) je jiný než český 6 pád. Má původně otázku odkud, ale nabalil na sebe spoustu dalších významů.

        Ve slovníku se vyskytuje: nominativ singuláru, pádová přípona genitivu singuláru (podle toho poznáme, podle čeho se skloňuje), feminimum (f.) / masculinum (m.) / neutrum (n.), překlad.

        V plurálu se vždy shoduje nominativ a vokativ. V singuláru skoro vždy. V plurálu se navíc vždy shoduje dativ a ablativ.

        Přídavná jména (adjektiva) se skloňují podle stejných rodů jako podstatná jména (substantiva).

    \subsection{Substantiva 1. deklinace}
        Genitiv končící na -ae (nominativ končí často na -a). Pouze f. (většinou) / m., žádná n.

        Vzor: fémina, ae, f. - žena
        \begin{itemize}
            \item fémin-a
            \item fémin-ae
            \item fémin-ae
            \item fémin-am
            \item fémin-a!
            \item fémin-á
            \item …
            \item fémin-ae
            \item fémin-árum
            \item fémin-ís
            \item fémin-ás
            \item fémin-ae!
            \item fémin-ís
        \end{itemize}

% 20. 10. 2020

    \subsection{Časování obecně}
        Ve slovníku se udává číslo konjugace, či jiné tvary (tzv. stupnice, maximálně 4 členy (včetně základního) -- présentní kmen, infinitiv présentum aktíva, imperfektní?, supinní?) pro jiné časy.

        Všechna latinská slovesa mají stejné koncovky, ale liší se tím, co předchází.

        Rozkazovací způsoby pasiva zatím můžeme přehlížet.

        \begin{upozorneni}
            Čeština má také infinitiv pasiva.
        \end{upozorneni}

    \subsection{Pravidelná slovesa 1. - 4. konjugace}


% 27. 10. 2020

    \subsection{probrali jsme?}
    Ablativ vyjadřuje odpověď na otázku kým / čím (činitel pasivního děje / nástroj = ablativ instrumentální) a tato odpověď se váže s předložkou á / ab (kým) / neváže s žádnou (čím). Pokud není činitelem osoba, pak se vyjadřuje prostým ablativem (tj. ablativem bez předložky).

    \subsection{3. konjugace}
        Tzv. krátké e kmeny. Ale většina tvarů má i místo e. Navíc existují dva vzory (tzv. smíšená konjugace = přechod mezi 2. a 4.), i někde zůstává a někde mizí.

    \subsection{Substantiva 2. deklinace (o-tvary)}
        Pravděpodobně všechny subs. 2. deklinace měly původně nominativ končící na -os.    

        -us: Většinou maskulina, feminina jsou často jména stromů, zemí, měst a malých ostrovů. Jediný vzor se změnou vokativu oproti nominativu..

        -er: Vždy maskulina. (Substantiva -samohláska-er- si zachovávají e, kdežto -souhláska-er- se mění na -souhláska-r-)

        -um: Pouze neutra. Neutra mají vždy shodný nominativ, akuzativ a vokativ (v jednotném i (zvlášť) v množném)


% 3. 11. 2020

        Předložka in se pojí s ablativem (otázka kde, odpověď na / v) a s akuzativem (otázka kam, odpověď). Ad se naopak pojí pouze s akuzativem.

        Latina často vypouští zájmeno, takže v češtině je správné si do spousty vět doplnit předmět v podobě zájmena.

        Většina obyčejných sloves se váže s bezpředložkovým akuzativem, jiné pády a předložky se dozvíme ze slovníku („zkratky“: alcis = genitiv, alci = dativ, alqm / alqd = akuzativ, alqó / alqá = ablativ).

        Latina nemá přivlastňovací zájmena, nahrazuje je genitivem.

\section{Adjektiva 1. a 2. deklinace, adverbia a zájmena přivlastňovací, osobní a zvratná}
    \subsection{Adjektiva}
        Přídavná jména jsou zakončena (většinou má jedno adjektivum všechny 3 tvary) na -us (-er) v mužském rodě (2. deklinace), -a v ženském (1. deklinace), a -um ve středním (2. deklinace).

% 10. 11. 2020

        Nejenom slovesa, ale i adjektiva se mohou vázat s nějakým (jiným než v češtině) pádem, např. pléna (plný) s 6. pádem, např. glóriá (sláva).

    \subsection{Osobní zájmena}
    Mají vlastní skloňování. Tvary vestrum a nostrum jsou tzv. celkové genitivy (překládají se z vás a znás = celek z kterého se něco vyjímá). Tvary nostrI a vestrí jsou tzv. předmětové genitivy (viz někdy v budoucnu).

    Zvratné zájmeno se používá jen pro 3. osobu (vz. čeština: svůj / můj, se / mně). Zvratné zájmeno nemá nominativ.

    Předložka s (cum) se píše za zájmeno a dohromady s ním (mécum, técum, nóbíscum, vóbíscum, sécum).

    Suum přivlastňuje podmětu, eius přivlastňuje někomu dalšímu.

% 24. 11. 2020 MIDTERM z lingebry (proto tak málo zápisků)

\section{Časy}
    Latina má 6 časů: Présentum (přítomný čas), Imperfektum (minulé trvalé / opakované děje), 

% 1. 12. 2020

    Vazba dativ substantiva / zájmena + sloveso esse = mít (vlastnit).

    Ve slovesu prósum, pródesse (prospívám) je v mnoha tvarech přidané tzv. protetické d, jelikož o a e se těžko vyslovují těsně za sebou.

\section{3. deklinace}
    Všechna ostatní… Poznají se podle genitivu -is. V ablativu mají -e nebo -í, ale neexistuje jednoznačné pravidlo, které by jednoznačně obecně říkalo, co z toho tam má být (některá slova tam mohou mít obojí).

\newpage
\section{Slovníček}
    iánua = dveře\\
    iústitia = spravedlnost\\
    amícus, -?, m. = přítel, kamarád\\
    amica, -ae, f. = kamarádka\\
    aeterna, -ae, f. = věčná\\
    víta, -ae, f. = život\\
    stella, -ae, f. = život\\
    via, -ae, ?. = ?\\
    tenebrae, -árum, f. = tma, temnota (pomnožné)\\
    gloria, -ae, f. = sláva\\
    misericodia, -ae, f. = milosrdenství\\
    properó, 1. = pospíchat\\
    habeó, ére, uí, itum = mít\\
    laudó, laudáre = chválit\\
    naró, 1. = vyprávět\\
    ornó, ornáre, 1. = zdobím\\
    amó, 1. = milovat, mít rád\\
    cónfirmó, 1. = potvrdit\\
    labóró, 1. = pracovat\\
    castígó, 1. = kárat\\
    óró, 1. = modlit se\\
    sileó, -ére = mlčet\\
    inter + akuz. = mezi\\
    doceó, -ére = učit (někoho)\\
    timeó, -ére = bát se\\
    non = ne\\
    exaltó, -áre = oslavovat, vyvyšovat\\
    in aeternum = na věky\\
    maneó, -ére = setrvat, zůstat\\
    líberó, -áre = osvobodit\\
    monstró, -áre = ukázat\\
    ambuló, -áre = chodit, docházet, procházet se\\
    audió, -íre = slyšet, poslouchat\\
    aperió, -íre = otevřít\\
    legó, -ere = číst\\
    capió, -ere = brát, chytat\\
    accipió, -ere = přijímat\\
    magister, magistrí, f. = mistr, učitel\\i
    num = neboť\\
    sed = ale\\
    dé (s ablativem) = z / o\\
    fábula = příběh\\
    narrátó = vyprávět\\
    

    


\end{document}
