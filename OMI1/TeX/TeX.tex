\documentclass[12pt]{article}					% Začátek dokumentu
\usepackage{../../MFFStyle}					    % Import stylu



\begin{document}

\section*{Organizační úvod}
    Bude odevzdávací systém, zatím odevzdávat domácí úkoly. (Musíme splnit něco jako 7 domácích úkolů).

\section{Úvod}
    .tex + .tfm (tex font metric = rozměry písmen) $\leftarrow$ TEX $\leftarrow$ DVI (formát nezávislý na OS) $\leftarrow$ DVI moduly = DVI drivery (dvips, xdvi, pdftex (ten navíc potřebuje .tfm a fonty)).

    \TeX umí primitiva. V tom se ale sázení řídí špatně, tedy existují nadstavby (plain\TeX, nadstavba od autora, nad nim jsou postavené con\TeX t, opmac a \LaTeX) a nadstavby mimo, jako AMS\TeX, Xe\TeX, Bib\TeX.

    tex nebo pdftex spouští plain\TeX nebo s přepínačem -ini ini\TeX, který umí zkompilovat makra do formatu.

    Literatura:
    \begin{itemize}
        \item Knuth: The \TeX book: nejdřív je to takový tutoriál, potom \TeX uvnitř
        \item Olšák: \TeX book naruby: v opačném pořadí, česky
        \item Knuth: \TeX the program (popsaný zdroják \TeX u)
    \end{itemize}

    \subsection{Sazba odstavce}
        Vstup = horizontální seznam $\rightarrow$ (proces předělání se spouští příkazem par) zalámaný odstavec

        Horizontální seznam obsahuje:
        \begin{itemize}
            \item box (h, d, w): písmenko, slitek (ligatura), hbox / vbox
            \item linka (h, d, w): hrule, vrule (liší se tím, kde se mohou vyskytnout, tedy hrule nepatří do horizontálního seznamu)
            \item discretionary break (pevné šířky): nobreak (stav bez rozdělení) pre-break (na konci řádku) post-break (na začátku dalšího), nesmí obsahovat pružné věci

            \item whatsit (?): přepínač jazyka, …
            \item vertikální materiál (nemá vliv na horizontální sazbu, při zlomu vypadne do vertikálního seznamu), objeví se například po řádku s \verb'\vadjust{…}'
            \item lepidlo (= glue) (= pružné výplňky) -- věc s fixní šířkou, roztažitelností a stlačitelností (ty se udávají buď přímo v jednotkách jako px apod, nebo v jednom ze 3 nekonečen (také jednotka) fil, fill, filll).
            \item kern (= pevné výplňky) -- věc pouze s fixní šířkou, automaticky vzniká například pro oddálení kulatých písmen / přiblížení plochých…
            \item penalty (= trest) -- číslo, které říká, jak moc chceme / nechceme zlomit
            \item math on/off -- zapíná a vypíná matematiku

        \end{itemize}
        Prvních 5 je non-discardable (jsou vidět). Ostatní discardable.

        Možná místa zlomu jsou:
        \begin{itemize}
            \item glue (když před sebou má něco non-disc. a není uvnitř math)
            \item před kernem (když za ním je glue a není uvnitř math)
            \item math off (následovaný glue)
            \item v penaltě (ne větší než 10000)
            \item v discretionary breaku (ten se ale přidává až při druhém průchodu = když se nezalomí bez něho)
        \end{itemize}

        V zlomu se z předchozích věcí stane box řádku, u kterého se spočítají rozměry a spočítá se, jak moc je tento zlom špatný (badness), tím se vybere ten nejlepší zlom a vysází se „jeho“ řádek. Následně se zahodí všechny discardable věci a pokračuje se na dalším řádku.

        Badness boxu se počítá (pokud součet nekonečných roztažitelností není nula, pak je badness 0):
        $$ \left\lceil 100 \cdot \(\frac{\text{deformace}}{\text{pružnost(součet roztažitelností / stlačitelností)}}\)^3 \right\rceil $$

        Roztahuje (stlačuje) se v nejvyšším nenulovém nekonečnu a to v poměru hodnot roztažnosti.

        Nestlačuje se o více jak 100\% (dá badness ∞).

        \verb|\hbadness| říká badness, při které už se vypisuje, že se \TeX pokusí (už ho dělá, nevybírá) vysázet box s touto badness.

        \verb|\tolerance| (při prvním průchodu \verb|\pretolerance|, která je zpravidla o dost nižší) říká, že větší badness už se zahazuje. Při třetím průchodu už se používá, že roztažnost zvýšíme o \verb|\emergencystretch|

        Lámání odstavce funguje tak, že se prohledají zlomy (každý bod zlomu se dá považovat za vrchol grafu, hrany jsou ohodnoceny tzv. demerits (počítá se z badness a dalších věcí)).

        Demerits:
        $$ (l + b)^2 \pm p^2 + extras $$ (l = \verb|\linepenalty| -- přidává se vždy (aby se nelámalo všude, kde lze), b = badness, p = penalta na níž se láme resp. \verb|\hyphenpenalty| nebo \verb|\exhyphenpenalty|, extras = \verb|\adjdemerits| (10000, přidává se, když typ řádku není sousední -- aby nebyl velmi roztažen, když předchozí byl stlačen) + \verb|\doublehyphendemerits| (10000, přidává se, když vyjdou 2 neprázdné pre-breaky za sebou) + \verb|\finaldemerits| (5000, neláme se na předposledním řádku odstavce))

        Typ řádku: 0 (roztažení o více jak 100\%), 1 (roztažení o 50-100\%), 2 (změna max. o 50\%), 3 (stlačení o  více jak 50\%)


    \begin{priklad}[Domácí úkol]
        Zkuste vymyslet, co naskládat do horizontálního seznamu, aby se mezera dala zlomit, ale zůstala na dalším řádku.
    \end{priklad}

    Příště tvar odstavce (left / right)skip, jak vzniká hor. seznam, algoritmus na pakování boxů.

% 12. 10. 2020

    \begin{poznamka}[Vzniky horizontálních věcí]
        Boxy (znaky, ligatury, \verb|\hbox to 10cm {…}|, \verb|\vbox spread 5mm{…}| (podle v/h se skládá obsah uvnitř boxu, to nastavuje velikost, spread zvětšuje velikost))

        Linky (\verb|\vrule width 3pt height 10pt depth 1pt|)

        Kerny (\verb|\kern 30pt|)

        Glue (\verb|\hskip 10pt plus 2pt minus 1fil|, \verb|hfil|, \verb|hfill|, \verb|hss|)

        Penalty (\verb|penalty 100|, \verb|\nobreak|, \verb|break|, \verb|\allowbreak|)

        Discr. break (\verb|\discretionary{pre}{post}{no}|)

        Etc.
    \end{poznamka}

    \begin{definice}[Box packing]
        Spočítáme výsledné rozměry a deformace, pokládám věci na baseline, pomocí výšky a hloubky (příkazy \verb|\raise 10pt \hbox{…}| a \verb|\lower| se dají posunovat boxy vůči baseline).

        Deformujeme glue.

        Určí se neurčité rozměry linek.

        Vertikálně se naopak pokládají referenčními body na jednu linku vlevo (zase existuje \verb|\moveleft| a \verb|\moveright|). Výška se pak určuje jako součet výšek a hloubek boxů uvnitř. Hloubka se počítá podle posledního boxu a pak se minimuje na \verb|\boxmaxdepth| u explicitních \verb|\maxdepth| u stránkových zlomů.

        \begin{poznamkain}
            Na určování rozměrů se hodí tzv. podpěry (linky nulové šířky s nenulovou výškou nebo hloubkou).
        \end{poznamkain}

        \begin{poznamkain}
            Lze získat rozměry před i po deformaci.
        \end{poznamkain}
    \end{definice}

    \begin{definice}[Sázení písmenek]
        Font dimen: sklon std.mezera, roztažnost, smrštitelnost, ex, em, extra mezera…

        Spacefaktor je na počátku 1000. Pokud sf znaku $≠ 0$: přenastavýme (leda že by sf znaku > 1000 a my jsme < 1000, pak nastavíme na 1000). Mezera: velikost: std.mezera + extramezera (pokud $sf≥1000$), roztažnost: $fd3 \cdot sf/1000$, smrštitelnost: $fd4 \cdot 1000/sf$.

        \begin{poznamka}[Nastavení pro angličitnu]
                A-Z: 999, a-z: 1000, .!?: 3000, ,: 1250, (): 0
        \end{poznamka}

        Existuje i explicitní \verb|\spacefactor 1234|, \verb|\frenchspacing| (nastavuje češtinu), \verb|\nonfrenchspacing| (nastavuje angličtinu), \verb|\spaceskip=5mm| (spaceskip překřičí font dimen aktuálního fontu) a \verb|\xspace=3mm| (použije se, když je moc velký sf).
    \end{definice}

    \begin{poznamka}[Některé příkazy]
        \verb|\line{…} = \hbox to \hsize{…}|

        \verb|\centerline{…} = \line{\hss…\hss}|

        \verb|\rlap{…} = \hbox to 0pt{…\hss}| (Box nulové šířky s vyčuhujícím materiálem doleva)

        \verb|\llap{…} = \hbox to 0pt{\hss…}| (Box nulové šířky s vyčuhujícím materiálem doprava)
    \end{poznamka}

    \begin{definice}[Dělení slov]
        Pro každý jazyk má \TeX trie, jak slovo dělit.

        Existuje makro \verb|\chyph|, které přepne do češtiny.
    \end{definice}

% 19. 10. 2020

    \begin{definice}[Módy fungování v \TeX u]
        (Přesněji řečeno módy hlavního procesoru)
        \begin{itemize}
            \item Vertikální hlavní (stránkový)
            \item Vertikální vnitřní (\verb|\vbox|)
            \item Horizontální odstavcový (zalamování)
            \item Horizontální vnitřní / restricted (\verb|\vbox|)
            \item Matematický vnitřní (\verb|$|)
            \item Matematický display (\verb|$$|)
        \end{itemize}

        Na začátku je \TeX v hlavním vertikálním módu. Teprve ve chvíli, kdy najde něco, co by mělo být v odstavci (písmenko, \verb|\noindent|, \verb|\indent|, \verb|\leavevmode| (jako písmenko), \verb|\hskip|, \verb|\vrule|), tak se \TeX přesune do odstavcového horizontálního.

        Zpět se přesouvá příkazy (\verb|\par| (tj. i 2 odřádkování), vertikálními povely: \verb|\par|, \verb|\vskip|, …, \verb|}| ukončující \verb|\vbox|), což vyvolá odstavcový zlom a vrácení se do hlavního vertikálního.

        Obdobně ostatní přechody (pozor, lze přecházet i z Vertikálního vnitřního do odstavcového, naopak nelze přecházet z vertikálních do matematických, tam se automaticky přechází přes horizontální).
    \end{definice}
    
    \begin{poznamka}[Co dostane lámací algoritmus]
        Na začátku prázdný \emph{box} šířky \verb|\parindent|. Následuje horizontální materiál odstavce a „ocásek“, ve kterém je \verb|\unskip| (odstranění poslední mezery), \verb|\nobreak|, glue velikosti \verb|\parfillskip = 0 plus 1 fil|, \verb|\break|. 
        
        \begin{poznamkain}[Co lze]
            Přenastavit \verb|\parfillskip = \parindent| (pak bude odstavec, když to vyjde, symetrické).

            Přenastavit \verb|\parfillspi = 1cm plus 1fil| (např. když máme malou mezeru mezi odstavci a chceme uživatele upozornit na konec odstavce, i když vychází do konce řádku).
        \end{poznamkain}
    \end{poznamka}

    \begin{poznamka}[Sestavení řádku]
        Horizontální materiál řádku se obalí \verb|\leftskip = 0pt| zleva a \verb|\rihtskip = 0pt| zprava a zavře se do \verb|\hbox| velikosti \verb|\hsize|.

        Sázení na praporek lze vytvořit tím, že nastavíme \verb|\rightskip = 0pt plus 1 fil|, ale pak se budou řádky snadno lámat (nebudou se rozdělovat slova, budou kratší řádky). Správně na to existuje makro \verb|\raggedright|, které udělá \verb|\rightskip 0pt plus 4em\spaceskip=…\xspaceskip=…| (nastaví mezislovní a písmenné mezery na pevné, aby se neroztahovali podle smršťování a roztahování té mezery na konci).

        Centrování \verb|\leftskip = \rightskip = 0,4 plus 2em\parfillskip = 0pt|.
    \end{poznamka}

    \begin{poznamka}[Tvar odstavce]
        Vykousnutí se nastavuje \verb|\handindent| (rozměr, o kolik se odsadí) a \verb|\hadgafter| (číslo, kolik řádků se odsadí), když se nastaví záporné hodnoty, vykusují se intuitivně ostatní rohy odstavce. Na konci odstavce se nuluje.

        Následuje \verb|\parshape = n p1 w1 … pn wn| (kolik se má odsadit, o kolik které, poslední se opakuje do nekonečna). Také se nuluje.

        Když zrovna nejsme ve vertikálním módu, tak se v \verb|\prevgraf| uchovává počet řádků v předchozím odstavci.

        Existuje makro \verb|\everypar|, které spustí nastavený kód každý odstavec.

        Můžeme si objednat zmenšení / zvětšení počtu řádků \verb|\looseness = n| (- je kratší, pokud nelze vyplnit, bude ignorováno). 
    \end{poznamka}

    \begin{poznamka}[Výsledek lámání odstavce: vertikální materiál]
        \begin{itemize}
            \item $\forall$ řádek jako box + posunutí referenčního bodu (žádné glue).
            \item Dále z boxů vypadají vertikální věci (\verb|vadjust|, \verb|mark|).
            \item Penalty mezi řádky (\verb|\interlinepenalty = 0| + \verb|clubpenalty = 150| (po prvním řádku) + \verb|widowpenalty = 150| (před posledním řádkem) + \verb|\brokenpenalty = 100| (po pre-break) + \verb|\displaywidowpenalty| (aby nebyla osamocená display matematika)).
            \item Ještě se objeví vertikální (zde řádkové) mezery, ale ty probereme zvlášť.
        \end{itemize}
    \end{poznamka}

    \begin{definice}[Řádkování]
        Algoritmus, aby se pokud možno dodrželo řádkování (ale řádky mohou být různě široké). Řídí se 3 parametry: \verb|\baselineskip| (glue), \verb|lineskiplimit = 0pt| (dimen) a \verb|lineskip = 1pt| (glue).

        Vypočítá mezeru jako $skip = bls - d_{horni} - h_{spodni}$. Pokud vyjde $skip < lsl$, nastaví se $skip = ls$. (Při více stránkách není dobré nastavovat pružnost těchto mezer).

        vskip, kern, penalty ignorujeme, hrule algoritmus potlačí.

        \begin{poznamkain}[Jak je to doopravdy]
                V registru \verb|\prevdepth| = hloubka posledního boxu (\verb|-1000pt|: algoritmus potlačen), linka nastaví právě ten dolní limit.

                \verb|\nointerlineskip| je \verb|\prevdepth = -1000pt|. \verb|\offinterlineskip| úplně zastaví tento algoritmus \verb|\baselineskip = -1000pt, \lineskip = 0, \lineskiplimit = \maxdimen|
        \end{poznamkain}
    \end{definice}

    \begin{poznamka}[Usazení 1. řádku na stránce (pouze hlavní vertikální mód)]
        Snažíme se spočítat glue tak, aby výška mezery + výška 1. řádku vyšla \verb|\topskip|, ale není nikdy záporný.

        Rozdíl proti řádkovému: nemáme limit (vždy je 0pt) a uvažujeme linky.
    \end{poznamka}

% 26. 10. 2020

    \begin{poznamka}[Ještě k předchozímu]
        Na začátku odstavce se vloží \verb|\parskip|.
    \end{poznamka}

\section{registry}
    \begin{definice}
        Registry jsou zabudované (konkrétní počet; pojmenované; spousta nastavení, o kterých jsme mluvili) a uživatelské (0…255 každého typu, často (u dalších „TeXů“) i více).

        Typy:
        \begin{itemize}
            \item \verb|\count| -- číslo (31 bitů + znaménko)
            \item \verb|\dimen| -- rozměr (30 bitů + znaménko ve sp $= 2^{-16}$ pt\footnote{pt = 1in/72,27}, tj. 14 celá část, 16 desetinná)
            \item \verb|\skip| -- roztažnost (13+16 bitů)
            \item \verb|\muskip| -- matematický (speciální jednotky)
            \item toks, box, …
        \end{itemize}

        Registry se obnovují po konci grupy na začínající stav.

        Použití: lze do nich dosazovat (\verb|\count74=32|\footnote{Číslo lze napsat číslicemi s desetinnou tečkou, apostrof a osmičková soustava, 2 apostrofy a šestnáctková ve velkých písmenech, obrácený apostrof znak resp lomítko znak, hodnota registru a backslash pojmenovaný znak (pomocí *chardef*ch=kód (*=lomítko) to však \TeX užívá spíše uvnitř).}, \verb|\parskip=10pt| (rovnítka lze vyměnit za mezeru, či vynechat)), lze ho použít jako jednotky, vypsat ho (\verb|\the\count5|) (do pdf), vypsat ho (\verb|\showthe\count5|) (do logu), použít jako pointer (\verb|\count\count5|), automaticky konvertovat dimen $\rightarrow$ skip nebo skip $\rightarrow$ dimen $\rightarrow$ count. 
    \end{definice}

    \begin{definice}[Aritmetika]
        \verb|\advanced registr by hodnota| (by lze vynechat nebo napsat BY)\\
        \verb|\multiply| (pouze celými čísly)\\
        \verb|\divide| (pouze celými čísly), zaokrouhluje se k nule
    \end{definice}

    \begin{definice}[Alokace registrů]
        \ 
        \begin{itemize}
            \item count 0…9 = číslo stránky
            \item box 255 = přenos obsahu do output rutiny
            \item reg. 0…9 = pracovní (krom čísel stránek)
            \item \verb|\countdef\jmeno=cislo| -- nastavuje přezdívku za registr s číslem cislo
            \item \verb|\newcount\pocitadlo| -- (plain) alokuje nějaký registr (interně \verb|\countdef\pocitadlo|)
            \item \verb|\newinsert\…| -- (plain) alokuje vše, co potřebuje na insert, viz dále 
        \end{itemize}
    \end{definice}

    \begin{definice}[Boxový registr]
        Obsahuje nic, hbox nebo vbox.

        Lze ho nastavit (\verb|\setbox0=\hbox{…}|), přemístit na aktuální místo (\verb|\box0|), vložit na aktuální místo (\verb|\copy0|), přemístit / vložit jejich obsah na aktuální místo (\verb|\unhbox0|, \verb|unvbox|, \verb|unhcopy|, \verb|\unvcopy|), měřit / měnit rozměry (\verb|\wd0|, \verb|\ht0|, \verb|\dp0|)\footnote{toho využívají plainová makra *phantom$\{…\}$, *hphantom$\{…\}$, *vphantom$\{…\}$ (*=lomítko), které vytvoří prázdné boxy velikosti jejich obsahů.}, (\verb|\showbox0|), (\verb|\newbox\cs|).
    \end{definice}

\section{Stránkový zlom}

    \begin{definice}[Obsah vertikálního seznamu]
        \ 
        \begin{itemize}
            \item box
            \item linka
            \item odkaz na insert (plovoucí obsah)
            \item mark
            \item whatsit (třeba zápis do souboru \verb|\write|, \verb|\special| viz dále)
            \item glue
            \item kern
            \item penalty
        \end{itemize}

        Prvních 5 je non-discardable.
    \end{definice}

    \begin{definice}[Stránkový zlom]
        Místa zlomu: glue, před nímž je non-disc., kern za glue, penalta $< 10000$.

        Nebyl dostatek pamětí na obtížnější, tedy se postupně přidávají prvky, počítá se cost ta je na začátku 100000, protože by se obsah moc roztáhl, pak jsou rozumné a někdy dojde na nekonečno, kde algoritmus najde zpětně nejlepší zlom (pamatuje si ho, ze stejných vybere ten poslední = nejplnější) a tam zlomí.

        Cost\footnote{c = cost, b = badness, p = penalta, q = dodatečná penalta}: 1) $b<∞, p≤-10000, q<1000: c := p$, 2) $b < 10000, p \in \(-10000,10000\), q < 10000: c:= b+p+q$, 3) $b=10000 (underfull), \text{ostatní konečné jako v 2)}: c:=100000$, 4) jinak: $c:=+∞$.

        Pamatuje si \verb|\pagetotal|, kde si pamatuje, co už má na stránce, \verb|\pagestretch…|, kde si pamatuje počty roztažností, a \verb|\pagegoal|, kde si pamatuje výšku (bez plovoucích tedy \verb|\vsize|).
    
    \end{definice}

    \begin{priklad}
        Plain má makro \verb|\raggedbottom|, který nechá vlát dole (pružný konec stránk), vytvořte ho.
    \end{priklad}

\end{document}
