\documentclass[12pt]{article}					% Začátek dokumentu
\usepackage{../../MFFStyle}					    % Import stylu



\begin{document}

\section*{Organizační úvod}
    Bude odevzdávací systém, zatím odevzdávat domácí úkoly. (Musíme splnit něco jako 7 domácích úkolů).

\section{Úvod}
    .tex + .tfm (tex font metric = rozměry písmen) $\leftarrow$ TEX $\leftarrow$ DVI (formát nezávislý na OS) $\leftarrow$ DVI moduly = DVI drivery (dvips, xdvi, pdftex (ten navíc potřebuje .tfm a fonty)).

    \TeX umí primitiva. V tom se ale sázení řídí špatně, tedy existují nadstavby (plain\TeX, nadstavba od autora, nad nim jsou postavené con\TeX t, opmac a \LaTeX) a nadstavby mimo, jako AMS\TeX, Xe\TeX, Bib\TeX.

    tex nebo pdftex spouští plain\TeX nebo s přepínačem -ini ini\TeX, který umí zkompilovat makra do formatu.

    Literatura:
    \begin{itemize}
        \item Knuth: The \TeX book: nejdřív je to takový tutoriál, potom \TeX uvnitř
        \item Olšák: \TeX book naruby: v opačném pořadí, česky
        \item Knuth: \TeX the program (popsaný zdroják \TeX u)
    \end{itemize}

    \subsection{Sazba odstavce}
        Vstup = horizontální seznam $\rightarrow$ (proces předělání se spouští příkazem par) zalámaný odstavec

        Horizontální seznam obsahuje:
        \begin{itemize}
            \item box (h, d, w): písmenko, slitek (ligatura), hbox / vbox
            \item linka (h, d, w): hrule, vrule (liší se tím, kde se mohou vyskytnout, tedy hrule nepatří do horizontálního seznamu)
            \item discretionary break (pevné šířky): nobreak (stav bez rozdělení) pre-break (na konci řádku) post-break (na začátku dalšího), nesmí obsahovat pružné věci

            \item whatsit (?): přepínač jazyka, …
            \item vertikální materiál (nemá vliv na horizontální sazbu, při zlomu vypadne do vertikálního seznamu), objeví se například po řádku s \verb'\vadjust{…}'
            \item lepidlo (= glue) (= pružné výplňky) -- věc s fixní šířkou, roztažitelností a stlačitelností (ty se udávají buď přímo v jednotkách jako px apod, nebo v jednom ze 3 nekonečen (také jednotka) fil, fill, filll).
            \item kern (= pevné výplňky) -- věc pouze s fixní šířkou, automaticky vzniká například pro oddálení kulatých písmen / přiblížení plochých…
            \item penalty (= trest) -- číslo, které říká, jak moc chceme / nechceme zlomit
            \item math on/off -- zapíná a vypíná matematiku

        \end{itemize}
        Prvních 5 je non-discardable (jsou vidět). Ostatní discardable.

        Možná místa zlomu jsou:
        \begin{itemize}
            \item glue (když před sebou má něco non-disc. a není uvnitř math)
            \item před kernem (když za ním je glue a není uvnitř math)
            \item math off (následovaný glue)
            \item v penaltě (ne větší než 10000)
            \item v discretionary breaku (ten se ale přidává až při druhém průchodu = když se nezalomí bez něho)
        \end{itemize}

        V zlomu se z předchozích věcí stane box řádku, u kterého se spočítají rozměry a spočítá se, jak moc je tento zlom špatný (badness), tím se vybere ten nejlepší zlom a vysází se „jeho“ řádek. Následně se zahodí všechny discardable věci a pokračuje se na dalším řádku.

        Badness boxu se počítá (pokud součet nekonečných roztažitelností není nula, pak je badness 0):
        $$ \left\lceil 100 \cdot \(\frac{\text{deformace}}{\text{pružnost(součet roztažitelností / stlačitelností)}}\)^3 \right\rceil $$

        Roztahuje (stlačuje) se v nejvyšším nenulovém nekonečnu a to v poměru hodnot roztažnosti.

        Nestlačuje se o více jak 100\% (dá badness ∞).

        \verb|\hbadness| říká badness, při které už se vypisuje, že se \TeX pokusí (už ho dělá, nevybírá) vysázet box s touto badness.

        \verb|\tolerance| (při prvním průchodu \verb|\pretolerance|, která je zpravidla o dost nižší) říká, že větší badness už se zahazuje. Při třetím průchodu už se používá, že roztažnost zvýšíme o \verb|\emergencystretch|

        Lámání odstavce funguje tak, že se prohledají zlomy (každý bod zlomu se dá považovat za vrchol grafu, hrany jsou ohodnoceny tzv. demerits (počítá se z badness a dalších věcí)).

        Demerits:
        $$ (l + b)^2 \pm p^2 + extras $$ (l = \verb|\linepenalty| -- přidává se vždy (aby se nelámalo všude, kde lze), b = badness, p = penalta na níž se láme resp. \verb|\hyphenpenalty| nebo \verb|\exhyphenpenalty|, extras = \verb|\adjdemerits| (10000, přidává se, když typ řádku není sousední -- aby nebyl velmi roztažen, když předchozí byl stlačen) + \verb|\doublehyphendemerits| (10000, přidává se, když vyjdou 2 neprázdné pre-breaky za sebou) + \verb|\finaldemerits| (5000, neláme se na předposledním řádku odstavce))

        Typ řádku: 0 (roztažení o více jak 100\%), 1 (roztažení o 50-100\%), 2 (změna max. o 50\%), 3 (stlačení o  více jak 50\%)


    \begin{priklad}[Domácí úkol]
        Zkuste vymyslet, co naskládat do horizontálního seznamu, aby se mezera dala zlomit, ale zůstala na dalším řádku.
    \end{priklad}

    Příště tvar odstavce (left / right)skip, jak vzniká hor. seznam, algoritmus na pakování boxů.

% 12. 10. 2020

    \begin{poznamka}[Vzniky horizontálních věcí]
        Boxy (znaky, ligatury, \verb|\hbox to 10cm {…}|, \verb|\vbox spread 5mm{…}| (podle v/h se skládá obsah uvnitř boxu, to nastavuje velikost, spread zvětšuje velikost))

        Linky (\verb|\vrule width 3pt height 10pt depth 1pt|)

        Kerny (\verb|\kern 30pt|)

        Glue (\verb|\hskip 10pt plus 2pt minus 1fil|, \verb|hfil|, \verb|hfill|, \verb|hss|)

        Penalty (\verb|penalty 100|, \verb|\nobreak|, \verb|break|, \verb|\allowbreak|)

        Discr. break (\verb|\discretionary{pre}{post}{no}|)

        Etc.
    \end{poznamka}

    \begin{definice}[Box packing]
        Spočítáme výsledné rozměry a deformace, pokládám věci na baseline, pomocí výšky a hloubky (příkazy \verb|\raise 10pt \hbox{…}| a \verb|\lower| se dají posunovat boxy vůči baseline).

        Deformujeme glue.

        Určí se neurčité rozměry linek.

        Vertikálně se naopak pokládají referenčními body na jednu linku vlevo (zase existuje \verb|\moveleft| a \verb|\moveright|). Výška se pak určuje jako součet výšek a hloubek boxů uvnitř. Hloubka se počítá podle posledního boxu a pak se minimuje na \verb|\boxmaxdepth| u explicitních \verb|\maxdepth| u stránkových zlomů.

        \begin{poznamkain}
            Na určování rozměrů se hodí tzv. podpěry (linky nulové šířky s nenulovou výškou nebo hloubkou).
        \end{poznamkain}

        \begin{poznamkain}
            Lze získat rozměry před i po deformaci.
        \end{poznamkain}
    \end{definice}

    \begin{definice}[Sázení písmenek]
        Font dimen: sklon std.mezera, roztažnost, smrštitelnost, ex, em, extra mezera…

        Spacefaktor je na počátku 1000. Pokud sf znaku $≠ 0$: přenastavýme (leda že by sf znaku > 1000 a my jsme < 1000, pak nastavíme na 1000). Mezera: velikost: std.mezera + extramezera (pokud $sf≥1000$), roztažnost: $fd3 \cdot sf/1000$, smrštitelnost: $fd4 \cdot 1000/sf$.

        \begin{poznamka}[Nastavení pro angličitnu]
                A-Z: 999, a-z: 1000, .!?: 3000, ,: 1250, (): 0
        \end{poznamka}

        Existuje i explicitní \verb|\spacefactor 1234|, \verb|\frenchspacing| (nastavuje češtinu), \verb|\nonfrenchspacing| (nastavuje angličtinu), \verb|\spaceskip=5mm| (spaceskip překřičí font dimen aktuálního fontu) a \verb|\xspace=3mm| (použije se, když je moc velký sf).
    \end{definice}

    \begin{poznamka}[Některé příkazy]
        \verb|\line{…} = \hbox to \hsize{…}|

        \verb|\centerline{…} = \line{\hss…\hss}|

        \verb|\rlap{…} = \hbox to 0pt{…\hss}| (Box nulové šířky s vyčuhujícím materiálem doleva)

        \verb|\llap{…} = \hbox to 0pt{\hss…}| (Box nulové šířky s vyčuhujícím materiálem doprava)
    \end{poznamka}

    \begin{definice}[Dělení slov]
        Pro každý jazyk má \TeX trie, jak slovo dělit.

        Existuje makro \verb|\chyph|, které přepne do češtiny.
    \end{definice}
    
\end{document}
