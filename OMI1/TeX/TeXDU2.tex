\chyph

%\hsize=10cm
%\advance \hsize by 6pt

\hyphenation{nej-ab-surd-něj-ší}

\dimen10=10cm%
\advance\dimen10 by 6pt%

\hbox to \dimen10 {%
\vbox {%
\hrule height 1pt%
\hbox{%
\vrule width 1pt%
\kern 2pt%
\hsize=10cm%
\vbox{%
\vrule width 0pt height 10.5pt%
             Málo známý zákon, který lidem zakazuje, aby umírali na půdě Parlamentu, vybrali
             televizní diváci stanice UK Gold jako nejabsurdnější britské legislativní
             ustanovení. Jako druhý nejabsurdnější byl vybrán zákon, podle něhož je nalepit
             známku na dopis vzhůru nohama (v~Británii mají poštovní známky na sobě portrét
             královny) velezradou. Na třetím místě se umístil zákon, podle něhož smí být na
             veřejnosti v~Liverpoolu "nahoře bez" pouze prodavačky v~obchodech s~tropickými
             rybami.
\vrule width 0pt depth 5.5pt \par%
}%
\kern 2pt \vrule width 1pt%
}%
\hrule height 1pt%
}%
}%


 Jako nejabsurdnější mezinárodní zákon byl vybrán zákon z~Ohia, kde je
 protizákonné opíjet ryby.

 V~první desítce nejabsurdnějších britských zákonů byl dále zákon zakazující
 pojídání vánočního pečiva (mince pies) dne 25. prosince a~zákon zakazující
 vstup do Dolní sněmovny v~brnění.

 Britská Asociace právníků poukázala na to, že v~Británii zůstává stále
 v~platnosti ještě celá řada absurdních zákonů. Platí například zákaz střílet
 z~děla v~blízkosti obytných domů (zákon londýnské Metropolitní police z~r.
 1839), používat na sněhu či ledu klouzačky (zákon městských policií z~r. 1847)
 a~zákon zakazující hnát stádo dobytka ulicemi Londýna (Zákon o~ulicích hlavního
 města, 1867).

 K~dalším nezvyklým britským zákonům patří: Když někdo ve Skotsku zazvoní
 u~dveří vašeho domu a~chce použít vašeho záchodu, musíte mu to umožnit. Těhotná
 žena se v~Británii smí kdekoliv vymočit, třeba i~do policistovy helmy. Je
 protizákonné nesdělit daňovému úřadu to, co nechcete, aby daňový úřad věděl,
 ale je povoleno daňovému úřadu nesdělit, co vám nevadí, aby věděl. Zákon
 povoluje zavraždit Skota uvnitř středověkých hradeb starobylého anglického
 města York, ale pouze v~tom případě, pokud má u~sebe luk a~šíp.

\end
