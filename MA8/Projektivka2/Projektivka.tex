\documentclass[12pt]{article}					% Začátek dokumentu
\usepackage{../../MFFStyle}					    % Import stylu

\begin{document}

% 21. 02. 2024

\section{Ohniska kuželoseček}
\subsection{Konstrukce s imaginárními elementy}
\begin{poznamka}
	Všimněme si, že projektivita dvou soumístných soustav určuje jednoznačně pár samodružných elementů, ale opačně ne. Pokud však vezmeme involuci, tak ta už má jednoznačnou korespondenci involuce s párem samodružných elementů.
\end{poznamka}

\begin{priklad}[Konstrukce]
	Je-li dána projektivita soumístných bodových soustav na přímce, určete involuci, která má tytéž samodružné body. (Totéž duálně.)

	\begin{reseni}[Duální]
		Zvolíme pomocnou kružnici procházející daným bodem. Převedeme soustavy na bodové soustavy na kružnici. Vezmeme direkční přímku za poláru a najdeme k ní (přes tečny) pól. Nyní uvažujme involuci se středem v tomto bodě. Obraz v hledané involuci najdeme tak, že vzor převedeme na kružnici, zobrazíme v této involuci, a vrátíme zpět.
	\end{reseni}

	\begin{poznamkain}
		Pokud direkční přímka vyjde mimo kružnici, budou samodružné body komplexní a pól najdeme tak, že leží na polárách k bodům (pólům) ležícím na dané poláře.
	\end{poznamkain}
\end{priklad}

\begin{veta}
	Pro eliptickou involuci (bodových soustav na přímce) existují právě dva body v rovině, z nichž se tato involuce promítá absolutní involucí (to znamená involucí kolmic).
	
	\begin{dukazin}
		Pro eliptickou involuci se její páry rozdělují. Tedy nad úsečkami vzor – obraz si uděláme Thaletovy kružnice a hledané body budou jejich průsečíky.
	\end{dukazin}
\end{veta}

\begin{definice}
	Body z předchozí věty se nazývají pomocné body eliptické involuce.
\end{definice}

\begin{poznamka}[Platí]
	Absolutní involuce je eliptická involuce, jejíž samodružné přímky jsou imaginární. Nazývají se izotropické přímky a jejich směry jsou $[0:1:i]$ a $[0:1:-1]$.
\end{poznamka}

\begin{poznamka}
	Izotropické body leží na každé kružnici v rovině. Každé izotropická přímka je kolmá sama na sebe (v reálném skalárním součinu, z definice absolutní involuce)
\end{poznamka}

\subsection{Ohnisko středových kuželoseček}
\begin{dusledek}
	Pokud kuželosečka není kružnice, pak izotropické body na ní neleží, tedy z každého izotropického bodu k takové kuželosečce existují 2 tečny (? 4 imaginární přímky). Lze ukázat, že ze 6 průsečíků těchto 4 přímek jsou vždy dva reálné.
\end{dusledek}

\begin{definice}[Ohnisko]
	Těmto dvěma bodům budeme říkat ohniska dané kuželosečky.
\end{definice}

\begin{veta}
	Bod je ohniskem kuželosečky $\Leftrightarrow$ involuce sdružených polár indukovaná v tomto bodě kuželosečkou je involuce absolutní.

	\begin{dukazin}
		Samodružné přímky involuce sdružených polár jsou právě tečny z tohoto bodu.
	\end{dukazin}
\end{veta}

\begin{veta}
	1. Kuželosečka má 2 ohniska ($E, F$) (pro kružnici splývající), jsou umístěna symetricky podle středu na jedné z os kuželosečky. Ohniska jsou samodružné body involuce bodů na této ose, jejíž páry jsou vyťaty sdruženými kolmými polárami. A tedy i páry tečna+jejich normála (kolmice v bodě dotyku = pól tečny).

	2. Každé z ohnisek je pomocným bodem eliptické involuce, kterou na druhé ose vytínají sdružené kolmé poláry (a tedy i dvojice tečna+normála).

	3. Každá kružnice opsaná trojúhelníku danému druhou osou a sdruženými kolmými polárami protíná původní osu v ohniscích. (Vyplývá z předchozí části.)

	\begin{dukazin}
		Bez důkazu.
	\end{dukazin}
\end{veta}

\begin{definice}[Hlavní osa, vedlejší osa]
	Ose z předchozí věty se říká hlavní osa, druhé pak vedlejší.
\end{definice}

\begin{priklad}[Konstrukce]
	Dány osy elipsy s vrcholy, najděte ohniska.

	\begin{reseni}[Podobné hledání hyperoskulační kružnice.]
		K spojnici hlavního a vedlejšího vrcholu umíme najít pól (průsečík tečen = kolmic na osy). Z tohoto pólu vedeme kolmici, čímž jsme získali dvojici kolmých sdružených polár, tedy použijeme předchozí větu, bod 3.
	\end{reseni}

	Totéž pro hyperbolu: na hlavní ose máme zadané vrcholy, na vedlejší náhradní body.

	\begin{reseni}
		Polára bude tentokrát průsečík „těch druhých dvou kolmic v hlavním a vedlejším vrchole“, neboť pomocné body jsou takové, že přesně tento bod leží na asymptotě (tečně v nevlastním bodě).
	\end{reseni}
\end{priklad}

\subsection{Ohnisko paraboly}
\begin{definice}[Ohnisko]
	(Stejná.) Ohnisko paraboly je reálný průsečík izotropických tečen.

	\begin{poznamkain}
		Tuto definici splňují 2 body: vlastní ohnisko $F$ a nevlastní ohnisko = střed = směr průměrů = směr osy.
	\end{poznamkain}

	\begin{poznamkain}
		Polára vlastního ohniska = řídící přímka.
	\end{poznamkain}
\end{definice}

\begin{veta}
	1. Bod je ohniskem paraboly $\Leftrightarrow$ involuce sdružených polár v tomto bodě je involuce absolutní. (Tj. sdružené poláry v $F$ jsou vzájemně kolmé.)

	2. Spojnice vlastního a nevlastního ohniska = osa paraboly, vlastní ohnisko půlí každou úsečku vyťatou na ose sdruženými kolmými polárami (speciálně tečnou a její normálou).
\end{veta}

\begin{priklad}[Konstrukce]
	Zkonstruujte ohnisko paraboly zadané 4 tečnami.

	\begin{reseni}
		Najdeme osu a bod dotyku na libovolné nevrcholové tečně. Z něj vedeme kolmici a použijeme předchozí větu bod 2.
	\end{reseni}
\end{priklad}

\begin{veta}
	Ohnisko jsou pro kuželosečku 2 podmínky.

	\begin{dukazin}
		Ohnisko zadává 2 izotropické tečny, tedy 2 podmínky.
	\end{dukazin}

	\begin{poznamkain}
		2 ohniska + 1 bod (mimo osu = jejich spojnice) nezadávají jednoznačně kuželosečku, zadávají však jednoznačně elipsu a hyperbolu. A tyto dvě kuželosečky se v daném bodu protínají kolmo (úhel mezi tečnami).
	\end{poznamkain}
\end{veta}


\end{document}
