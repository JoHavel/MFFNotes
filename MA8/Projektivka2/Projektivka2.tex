\documentclass[12pt]{article}					% Začátek dokumentu
\usepackage{../../MFFStyle}					    % Import stylu

\begin{document}

% 21. 02. 2024

\section{Ohniska kuželoseček}
\subsection{Konstrukce s imaginárními elementy}
\begin{poznamka}
	Všimněme si, že projektivita dvou soumístných soustav určuje jednoznačně pár samodružných elementů, ale opačně ne. Pokud však vezmeme involuci, tak ta už má jednoznačnou korespondenci involuce s párem samodružných elementů.
\end{poznamka}

\begin{priklad}[Konstrukce]
	Je-li dána projektivita soumístných bodových soustav na přímce, určete involuci, která má tytéž samodružné body. (Totéž duálně.)

	\begin{reseni}[Duální]
		Zvolíme pomocnou kružnici procházející daným bodem. Převedeme soustavy na bodové soustavy na kružnici. Vezmeme direkční přímku za poláru a najdeme k ní (přes tečny) pól. Nyní uvažujme involuci se středem v tomto bodě. Obraz v hledané involuci najdeme tak, že vzor převedeme na kružnici, zobrazíme v této involuci, a vrátíme zpět.
	\end{reseni}

	\begin{poznamkain}
		Pokud direkční přímka vyjde mimo kružnici, budou samodružné body komplexní a pól najdeme tak, že leží na polárách k bodům (pólům) ležícím na dané poláře.
	\end{poznamkain}
\end{priklad}

\begin{veta}
	Pro eliptickou involuci (bodových soustav na přímce) existují právě dva body v rovině, z nichž se tato involuce promítá absolutní involucí (to znamená involucí kolmic).
	
	\begin{dukazin}
		Pro eliptickou involuci se její páry rozdělují. Tedy nad úsečkami vzor – obraz si uděláme Thaletovy kružnice a hledané body budou jejich průsečíky.
	\end{dukazin}
\end{veta}

\begin{definice}
	Body z předchozí věty se nazývají pomocné body eliptické involuce.
\end{definice}

\begin{poznamka}[Platí]
	Absolutní involuce je eliptická involuce, jejíž samodružné přímky jsou imaginární. Nazývají se izotropické přímky a jejich směry jsou $[0:1:i]$ a $[0:1:-1]$.
\end{poznamka}

\begin{poznamka}
	Izotropické body leží na každé kružnici v rovině. Každé izotropická přímka je kolmá sama na sebe (v reálném skalárním součinu, z definice absolutní involuce)
\end{poznamka}

\subsection{Ohnisko středových kuželoseček}
\begin{dusledek}
	Pokud kuželosečka není kružnice, pak izotropické body na ní neleží, tedy z každého izotropického bodu k takové kuželosečce existují 2 tečny (? 4 imaginární přímky). Lze ukázat, že ze 6 průsečíků těchto 4 přímek jsou vždy dva reálné.
\end{dusledek}

\begin{definice}[Ohnisko]
	Těmto dvěma bodům budeme říkat ohniska dané kuželosečky.
\end{definice}

\begin{veta}
	Bod je ohniskem kuželosečky $\Leftrightarrow$ involuce sdružených polár indukovaná v tomto bodě kuželosečkou je involuce absolutní.

	\begin{dukazin}
		Samodružné přímky involuce sdružených polár jsou právě tečny z tohoto bodu.
	\end{dukazin}
\end{veta}

\begin{veta}
	1. Kuželosečka má 2 ohniska ($E, F$) (pro kružnici splývající), jsou umístěna symetricky podle středu na jedné z os kuželosečky. Ohniska jsou samodružné body involuce bodů na této ose, jejíž páry jsou vyťaty sdruženými kolmými polárami. A tedy i páry tečna+jejich normála (kolmice v bodě dotyku = pól tečny).

	2. Každé z ohnisek je pomocným bodem eliptické involuce, kterou na druhé ose vytínají sdružené kolmé poláry (a tedy i dvojice tečna+normála).

	3. Každá kružnice opsaná trojúhelníku danému druhou osou a sdruženými kolmými polárami protíná původní osu v ohniscích. (Vyplývá z předchozí části.)

	\begin{dukazin}
		Bez důkazu.
	\end{dukazin}
\end{veta}

\begin{definice}[Hlavní osa, vedlejší osa]
	Ose z předchozí věty se říká hlavní osa, druhé pak vedlejší.
\end{definice}

\begin{priklad}[Konstrukce]
	Dány osy elipsy s vrcholy, najděte ohniska.

	\begin{reseni}[Podobné hledání hyperoskulační kružnice.]
		K spojnici hlavního a vedlejšího vrcholu umíme najít pól (průsečík tečen = kolmic na osy). Z tohoto pólu vedeme kolmici, čímž jsme získali dvojici kolmých sdružených polár, tedy použijeme předchozí větu, bod 3.
	\end{reseni}

	Totéž pro hyperbolu: na hlavní ose máme zadané vrcholy, na vedlejší náhradní body.

	\begin{reseni}
		Polára bude tentokrát průsečík „těch druhých dvou kolmic v hlavním a vedlejším vrchole“, neboť pomocné body jsou takové, že přesně tento bod leží na asymptotě (tečně v nevlastním bodě).
	\end{reseni}
\end{priklad}

\subsection{Ohnisko paraboly}
\begin{definice}[Ohnisko]
	(Stejná.) Ohnisko paraboly je reálný průsečík izotropických tečen.

	\begin{poznamkain}
		Tuto definici splňují 2 body: vlastní ohnisko $F$ a nevlastní ohnisko = střed = směr průměrů = směr osy.
	\end{poznamkain}

	\begin{poznamkain}
		Polára vlastního ohniska = řídící přímka.
	\end{poznamkain}
\end{definice}

\begin{veta}
	1. Bod je ohniskem paraboly $\Leftrightarrow$ involuce sdružených polár v tomto bodě je involuce absolutní. (Tj. sdružené poláry v $F$ jsou vzájemně kolmé.)

	2. Spojnice vlastního a nevlastního ohniska = osa paraboly, vlastní ohnisko půlí každou úsečku vyťatou na ose sdruženými kolmými polárami (speciálně tečnou a její normálou).
\end{veta}

\begin{priklad}[Konstrukce]
	Zkonstruujte ohnisko paraboly zadané 4 tečnami.

	\begin{reseni}
		Najdeme osu a bod dotyku na libovolné nevrcholové tečně. Z něj vedeme kolmici a použijeme předchozí větu bod 2.
	\end{reseni}
\end{priklad}

\begin{veta}
	Ohnisko jsou pro kuželosečku 2 podmínky.

	\begin{dukazin}
		Ohnisko zadává 2 izotropické tečny, tedy 2 podmínky.
	\end{dukazin}

	\begin{poznamkain}
		2 ohniska + 1 bod (mimo osu = jejich spojnice) nezadávají jednoznačně kuželosečku, zadávají však jednoznačně elipsu a hyperbolu. A tyto dvě kuželosečky se v daném bodu protínají kolmo (úhel mezi tečnami).
	\end{poznamkain}
\end{veta}

% 06. 03. 2024

\section{Analytická geometrie}
\begin{definice}[Projektivní prostor, geometrický bod, aritmetický zástupce]
	(Reálný) projektivní prostor dimenze $n$ je množina
	$$ ®RP^n = \{\<v\>, v \in ®R^{n+1} \setminus ¦o\} = \text{ množina všech přímek (procházejících počátkem) v } ®R^{n+1}. $$
	Prvek $\<v\> \in ®RP^n$ se nazývá geometrický bod a $v$ jeho aritmetický zástupce

	\begin{poznamkain}[Platí]
		$\<v\> = \<w\>$ (tj. stejné geometrické body) $\Leftrightarrow$ $\exists α \in ®R \setminus \{0\}: w = α·v$ (tj. aritmetičtí zástupci se liší pouze násobkem $≠ 0$).
	\end{poznamkain}
\end{definice}

\begin{definice}[Homogenní souřadnice]
	Je-li $v = (x_0, …, x_n) \in ®R^{n+1} \setminus \{¦o\}$, pak homogenní souřadnice geometrického bodu $\<v\>$ jsou $[x_0:…:x_n]$.

	\begin{poznamkain}
		Jsou určeny až na násobek $≠0$.
	\end{poznamkain}
\end{definice}

\begin{definice}[Projektivní přímka, projektivní rovina, projektivní prostor]
	$®RP^1$ říkáme projektivní přímka.
	$®RP^2$ říkáme projektivní rovina.
	$®RP^3$ říkáme projektivní prostor.
	%$®RP^{n-1}$ říkáme (v $®RP^n$) projektivní nadrovina.
\end{definice}

\begin{poznamka}[Značení]
	Místo $\<a\>$ budeme psát $A$.
\end{poznamka}

\begin{poznamka}
	$®RP^1$: Dva body $A, B$ jsou totožné $\Leftrightarrow$ vektory $a, b$ jsou lineárně závislé.

	$®RP^2$: Tři body $A, B, C$ leží na jedné přímce (po dvou různé) $\Leftrightarrow$ $a, b, c$ jsou lineárně závislé (po dvou lineárně nezávislé), tj. leží v jedné rovině.

	$®RP^3$: Čtyři body $A, B, C, D$ leží v rovině $\Leftrightarrow$ $a, b, c, d$ jsou lineárně závislé, tj. leží v jednom prostoru.

	Obecně $®RP^n$: $n+1$ bodů $A_0, …, A_n$ leží v $n-1$-dimenzionálním projektivním prostoru $\Leftrightarrow$ vektory $a_0, …, a_n$ leží v nadrovině v $®R^{n+1}$ (jsou lineárně závislé).
\end{poznamka}

\begin{poznamka}
	Procesu „zakážu ¦o a ztotožníme násobky“ říkáme projektivizace.
\end{poznamka}

\begin{definice}[Projektivní rozšíření afinního prostoru, vlastní bod, nevlastní bod]
	Projektivní rozšíření afinního prostoru $®R^n$ na projektivní prostor $®RP^n$ (= kanonické vnoření $®R^n$ do $®RP^n$) je zobrazení, které bodu $[x_1, …, x_n]$ přiřadí $[1:x_1:…:x_n]$ a vektoru $(x_1, …, x_n)$ přiřadí $[0:x_1:…:x_n]$.

	Prvním říkáme body vlastní, druhý nevlastní.
\end{definice}

TODO?

\begin{definice}[Homogenní souřadnice přímky]
	V $®RP^2$ zavádíme homogenní souřadnice přímky $a_0 x_0 + a_1 x_1 + a_2 x_2 = 0$ jako homogenní trojici $(a_0:a_1:a_2)$.

	\begin{poznamkain}
		Opět určeny až na násobek $≠ 0$.
	\end{poznamkain}
\end{definice}

\begin{priklady}
	$(0:1:0)$ je osa $y$, $(0:0:1)$ je osa $x$, $(1:0:0)$ je nevlastní přímka.
\end{priklady}

\begin{priklad}[Hledání průsečíku dvou přímek]
	TODO?
\end{priklad}

\begin{priklad}[Incidence bodů]
	$X = [x_0:x_1:x_2]$, $a = (a_0:a_1:a_2)$. $X \in a$ $\Leftrightarrow$ $a_0 x_0 + a_1 x_1 + a_2 x_2 = 0$.

	\begin{dusledekin}
		Lze zaměnit bod za přímku $\implies$ dualita.

		Tj. například spojnice bodů se počítá stejně jako průsečík přímek.
	\end{dusledekin}
\end{priklad}

\begin{poznamka}[Trik na nalezení spojnice (/průsečíku)]
	Dány body $Y = [y_0:y_1:y_2]$, $Z = [z_0:z_1:z_2]$. Chceme rovnici jejich spojnici: $X \in a = YZ \Leftrightarrow a_0x_0 + a_1x_1 + a_2x_2 = 0$ pro hledané souřadnice $(a_0:a_1:a_2)$ $\Leftrightarrow$ vektory $X$, $Y$ a $Z$ jsou závislé $\Leftrightarrow$ $\det\((X|Y|Z)^T\) = 0$ $\Leftrightarrow$
	$$ \Leftrightarrow (y_1·z_2 - y_2·z_1)·x_0 + (-y_0·z_2 + y_2·z_0)·x_1 + (y_0·z_1 - y_1·z_0)·x_2 = 0. $$
\end{poznamka}

\subsection{Dvojpoměr}
\begin{definice}[Dvojpoměr]
	Dvojpoměr 4 vektorů v rovině $a, b, c, d$, po dvou lineárně nezávislých, ale po třech lineárně závislých (tj. BÚNO $c = α_1 a + β_1 b$, $d = α_2·a + β_2 b$) definujeme jako $(abcd) := \frac{α_2·β_1}{α_1·β_2} \in ®R$.

	\begin{poznamkain}
		Zřejmě tato hodnota nezávisí na volbě (nenulového) násobku každého vektoru.
	\end{poznamkain}

	Dvojpoměr 4 bodů $A, B, C, D \in ®RP^n$ ležících na jedné přímce definujeme jako $(ABCD) := (abcd)$.
\end{definice}

\begin{tvrzeni}[Už jsme si dokázali]
	$A, B, C, D$ jsou čtyři různé $\implies$ $(ABCD) ≠ 0, 1, ∞$. (Např. $C = A \lor B = D \Leftrightarrow (A B C D) = 0$.)

	$A, B, C, D$ vlastní $\implies$ $(ABCD) = \frac{(ABC)}{(ABD)}$. $A, B, C$ vlastní, $D$ nevlastní $\implies$ $(ABCD) = (ABC)$.

	Věta o 4 determinantech (pro $A, B, C, D \in ®RP^1$):
	$$ (ABCD) = \frac{(a_0·c_1 - a_1·c_0)·(b_0·d_1 - b_1·d_0)}{(a_0·d_1 - a_1·d_0)·(b_0·c_1 - b_1·c_0)}. $$
\end{tvrzeni}

\begin{definice}[Harmonická čtveřice]
	$$ (ABCD) = -1 $$
\end{definice}

\begin{priklad}[Pak jsme počítali. V jednu chvíli nám vyšlo:]
	Parametrizace přímky procházející $A, B$ je $t_1·A + t_2·B$, kde například $t_1 + t_2 = 1$; lépe $t·A + (1 - t)·B$.
\end{priklad}

\begin{definice}[Projektivní souřadný systém (PSS)]
	Projektivní souřadný systém v $®RP^n$ je $(n+1)$-tice různých bodů $A_0, …, A_n \in ®RP^n$. Pak $\forall X \in ®RP^n$ definujeme souřadnice bodu $X$ vůči PSS $(A_0, …, A_n)$ jako homogenní $(n+1)$-tici $[x_0:…:x_n]$ takovou, že $x = \sum_{i=0}^n x_i·a_i$.
\end{definice}

% 13. 03. 2024

TODO!!! (Projektivita na $®RP^n$: je dána regulární maticí $(n + 1) \times (n + 1)$ určenou až na násobek $≠$ 0 (píšeme $A \sim k·A$, pro $k ≠ 0$).)

TODO!!! (Ukázání si, že taková matice zachovává dvojpoměr.)

TODO!!! (Projektivita je dána svými hodnotami na $n+2$ bodech.)

TODO!!! (A mnoho dalšího.)

% 20. 03. 2024

\subsection{Samodružné body projektivit}
\begin{definice}[Samodružný bod projektivity]
	$\<¦v\> \in ®RP^n$ je samodružný bod projektivity dané maticí $A$ $≡$ $\<A·¦v\> = \<¦v\>$.
	
	\begin{poznamkain}
		$$ \Leftrightarrow \ \exists λ \in ®R (®C) \setminus \{0\}: A·¦v = λ·¦v \qquad \(\Leftrightarrow (A - λE)·¦v = ¦o \Leftrightarrow p_A(λ) = \det(A - λE) = 0\) $$
		$\Leftrightarrow$ $λ ≠ 0$ je vlastním číslem matice $A$ $¦o ≠ ¦v$ je vlastním vektorem matice $A$ příslušným vlastnímu číslu $λ$.
	\end{poznamkain}

	%\begin{poznamkain}
		%?To znamená, že samodružných bodů (včetně násobnosti) je $n$.
	%\end{poznamkain}

	\begin{poznamka}
		Matice projektivity je regulární, tedy nemá vlastní číslo nula.
	\end{poznamka}
\end{definice}

TODO? (Hromada lineární algebry.)

\subsection{Klasifikace projektivit na projektivní přímce}
\begin{poznamka}
	Klasifikace projektivit na projektivní přímce podle možných Jordanových tvarů:
	\begin{itemize}
		\item $J_A = \begin{pmatrix} λ_1 & 0 \\ 0 & λ_2 \end{pmatrix} \sim \begin{pmatrix} 1 & 0 \\ 0 & λ \end{pmatrix} \qquad \forall λ_1, λ_2, λ \in ®R \setminus \{0\}$. Pro $λ = 1$ je to identická projektivita. Pro $λ ≠ 1$ má dva reálné samodružné body.
		\item $J_A = \begin{pmatrix} λ & 1 \\ 0 & λ \end{pmatrix} \qquad \forall λ \in ®R \setminus \{0\}$. Tehdy má jediný samodružný bod (a ten je reálný). Navíc je podobná matici $\begin{pmatrix} λ & λ \\ 0 & λ \end{pmatrix}$, což je násobek $\begin{pmatrix} 1 & 1 \\ 0 & 1 \end{pmatrix}$.
		\item $J_A = \begin{pmatrix} λ & 0 \\ 0 & \overline{λ} \end{pmatrix} \qquad \forall λ \in ®C \setminus ®R$. Tehdy má dva komplexní samodružné body. 
	\end{itemize}
\end{poznamka}

\begin{dusledek}
	To dokazuje větu ze zimního semestru ($\exists$ 2/1/0 samodružné body projektivity soustav).
\end{dusledek}

\subsection{Charakteristika projektivity}
\begin{poznamka}[Opakování zimního semestru]
	Jsou-li $S, T$ samodružné body projektivity na $®RP^1$, pak její charakteristika je číslo $w = (XX'ST)$ pro libovolný pár $X \mapsto X'$.

	Věta: Hodnota $w$ nezávisí na volbě bodu $X$.
\end{poznamka}

\begin{veta}
	Hodnota $w$ nezávisí na volbě bodu $X$, a je-li projektivita dána maticí $A$, platí
	$$ w = \frac{\tr A + \sqrt{D}}{\tr A - \sqrt{D}}, \qquad D := (\tr A)^2 - 4 \det A. $$

	\begin{dukazin}
		Pro dvojpoměr platí (věta o čtyřech determinantech)
		$$ (XX'ST) = \frac{[XS]·[X'T]}{[XT]·[X'S]}, \qquad [AB] = \begin{vmatrix} a_0 & b_0 \\ a_1 & a_2 \end{vmatrix}. $$
		Pro danou $A$ spočítejme její vlastní čísla: $p_A(λ) = λ^2 - (\tr A)·λ + \det A$, tj. $λ_{1, 2} = \frac{\tr A ± \sqrt{D}}{2}$.

		Pak pro samodružné body platí: $S' = S$, $T' = T$, tedy $¦s' = A·¦s = λ_1·¦s$, $¦t' = A·¦t = λ_2·¦t$. Pak
		$$ [¦x'¦s] = [¦x' \frac{1}{λ_1}¦s'] = \frac{1}{λ_1}[¦x'¦s'] = \frac{1}{λ_1}·|A|·[¦x ¦s], \quad [¦x'¦t] = … = \frac{1}{λ_2}·|A|·[¦x¦t]. $$
		Dosadíme: $w = \frac{[¦x¦s]·\frac{1}{λ_2}·|A|·[¦x¦t]}{[¦x¦t]·\frac{1}{λ_1}·|A|·[¦x¦s]} = \frac{λ_1}{λ_2} = \frac{\tr A + \sqrt{D}}{\tr A - \sqrt{D}}$.
	\end{dukazin}
\end{veta}

\subsection{Involuce}
\begin{poznamka}[Opakování zimního semestru]
	Involuce je projektivita soumístných soustav splňující $w = -1$ $\Leftrightarrow$ $\forall X: X'' = X$ $\Leftrightarrow$ $\exists X: X'' = X$.
\end{poznamka}

\begin{definice}[Involuce]
	Involuce je projektivita (na $®RP^n$) daná maticí $A$, která splňuje $A^2 \sim E$.
\end{definice}

\begin{veta}
	Nechť matice $A \in ®R^{2 \times 2}$ zadává neidentickou projektivitu na $®RP^1$ ($A \nsim E$). Pak NáPoJE:
	\begin{enumerate}
		\item $A^2 \sim E$ (je to involuce);
		\item $\tr A = 0$;
		\item $w = -1$.
	\end{enumerate}

	\begin{dukazin}
		Pišme $A = \begin{pmatrix} a & b \\ c & d \end{pmatrix}$. Pak $A^2 = \begin{pmatrix} a^2 + bc & (a + d)·b \\ (a+d)·c & bc + d^2 \end{pmatrix}$.

		„$1. \implies 2.$“: $A^2 \sim E$ máme, chceme $\tr A = a + d = 0$. Předpoklad nám dává $(a + d)·b = 0 = (a + d)·c$. Pro spor $(a + d) ≠ 0$. Pak $b = c = 0$ a $A = \diag(a, d)$, tedy $\diag(a^2, d^2) \sim \diag(1, 1)$, tedy $a^2 = d^2$, tj. $a = ±d$. Takže buď $a + d = 0$ nebo $A \sim E$. \lightning.

		„$2. \implies 1.$“: předpokládáme $a + d = 0$. Pak ale $A^2 = \diag(a^2 + bc, bc + d^2) \overset{a = -d}= \diag{a^2+bc, a^2+bc} \sim E$.

		„$2. \Leftrightarrow 3.$“: $w = \frac{\tr A + \sqrt{D}}{\tr A - \sqrt{D}}$, tedy pro $\tr A = 0$ je $w = -1$, a pokud $w = -1$, pak $\tr A + \sqrt{D} = -\tr A + \sqrt{D}$, tedy $\tr A = -1$. (Přitom $D = (\tr A)^2 - 4\det A = -4\det A ≠ 0$.)
	\end{dukazin}
\end{veta}

\begin{dusledek}
	Stará definice involuce sedí s tou novou.
\end{dusledek}

\subsection{Parabolická involuce}
\begin{poznamka}
	Parabolická involuce odpovídá singulární matici $A$, která splňuje $\tr A = 0$.
	
	\begin{poznamkain}
		$$ A = \begin{pmatrix} a & b \\ c & - a \end{pmatrix} \implies c = \frac{-a^2}{b}. $$
		$$ \tilde A = \begin{pmatrix} a^2 & ba \\ -a^2 & -ab \end{pmatrix} \implies \begin{pmatrix} a^2 & ba \\ -a^2 & -ab \end{pmatrix}·\begin{pmatrix} x_0 \\ x_1 \end{pmatrix} = \begin{pmatrix} b·(ax_0 + bx_1) \\ -a·(ax_0 + bx_1) \end{pmatrix} \sim \begin{pmatrix} b \\ -a \end{pmatrix} $$
	\end{poznamkain}

	\begin{dusledekin}
		Tedy parabolická involuce zobrazuje všechny body do jednoho bodu.
	\end{dusledekin}
\end{poznamka}

\begin{poznamka}
	Klasifikace involucí na $®RP^1$ podle Jordanova tvaru:
	\begin{itemize}
		\item $J_A \sim \diag(1, -1)$ $\implies$ $p_A(λ) = λ^2 + \det A = λ^2 - a^2$, tedy 2 reálné samodružné body, tj. hyperbolická involuce;
		\item $J_A = \diag(λ, \overline{λ})$ $\implies$ $p_A(λ) = λ^2 + \det A = λ^2 + a^2$, tedy 2 imaginární body, tj. eliptická involuce.
		\item $J_A = \begin{pmatrix} 0 & 1 \\ 0 & 0 \end{pmatrix}$, tj. parabolická involuce.
	\end{itemize}
\end{poznamka}

% 27. 03. 2024

\section{Projektivity na $®RP^2$}
\begin{priklady}[Známé projektivity]
	Všechny eukleidovské shodnosti. Všechny afinity (násobení bodu v $®R^2$ regulární maticí). Tj. i stejnolehlosti.
\end{priklady}

\begin{poznamka}[Působení projektivity na přímku]
	Nechť je dána projektivita dána maticí $A$ ($3 \times 3$ regulární). Na bodech $x \mapsto Ax$. Na přímkách tedy $p^T \mapsto p^T A^{-1}$, protože projektivita zachovává incidenci.
\end{poznamka}

\begin{dusledek}
	Projektivita na $®RP^2$ má samodružné body a samodružné přímky. (A jsou to vlastní vektory matice $A^{-T}$.)
\end{dusledek}

\begin{veta}
	Nechť $A$ má $n$ různých vlastních čísel. Označme $v_1, …, v_n$ vlastní vektory $A$ (odpovídající $λ_1, …, λ_n$) a $u_1, …, u_n$ vlastní vektory $A^T$ (odpovídající stejným $λ_1, …, λ_n$). Pak pro $i ≠ j: \<u_j, v_i\> = 0$.

	\begin{dukazin}
		$$ λ_j · \<u_j, v_i\> =  λ_j ¦u_j^T·¦v_i = (λ_j ¦u_j)^T·¦v_i = (A^T·¦u_j)^T·¦v_i = ¦u_j^T·A·¦v_i = ¦u_j^T ·(λ_i ¦v_i) = λ_i ¦u_j^T·¦v_i = λ_i · \<u_j, v_i\> \overset{i ≠ j}\implies \<u_j, v_i\> = 0. $$
	\end{dukazin}
\end{veta}

\begin{dusledek}
	Jsou-li $¦u_1, …, ¦u_n$ samodružné přímky a $¦V_1, …, ¦V_n$ samodružné body projektivity, pak $i ≠ j \implies ¦V_i \in ¦u_j$.
\end{dusledek}

\begin{definice}[Silně a slabě samodružná]
	Samodružná přímka je silně samodružná, pokud se každý její bod zobrazí sám na sebe, a slabě samodružná v opačném případě.

	Samodružný bod je silně samodružný, pokud se každá jím procházející přímka zobrazí sama na sebe, a slabě samodružný v opačném případě.
\end{definice}

\begin{poznamka}
	Přímka/bod je silně samodružná/-ý právě tehdy, pokud příslušný vlastní podprostor má dimenzi $≥$ 2 (tj. existují 2 lineárně nezávislé vlastní vektory).
\end{poznamka}

\subsection{Klasifikace projektivit na $®RP^2$}
\begin{poznamka}[Hrubá klasifikace]
	Jordanovy buňky mohou být buď 3, 2 nebo 1.
\end{poznamka}

\begin{poznamka}[Podpřípady]
	3 buňky:
	\begin{itemize}
		\item $J_{1a} = \diag (λ_1, λ_2, λ_3)$ (3 různá reálná vlastní čísla). $\implies$ 3 slabě samodružné body a 3 slabě samodružné přímky.
		\item $J_{1b} = \diag (λ_1, \overline{λ_1}, λ_3)$, kde $λ_1$ je komplexní číslo, které není reálné, a $λ_2$ je reálné číslo. Je podobná s maticí $\diag (R_φ, λ_2)$, rotace + stejnolehlost (= spirální podobnost). 1 slabě samodružný bod a 1 slabě samodružná přímka.
		\item $J_{1c} = \diag (λ_1, λ_2, λ_2) \sim \diag (λ_1', 1, 1)$. Jeden silně a jeden slabě samodružný bod, jedna silně a jedna slabě samodružná přímka. Toto zobrazení je perspektivní (nebo také středová) kolineace.
		\item $J_{1d} = \diag (λ_1, λ_1, λ_1) \sim E$ je identita.
	\end{itemize}
	2 buňky:
	\begin{itemize}
		\item $J_{2a} = \diag \(\begin{pmatrix} 1 & 1 \\ 0 & 1 \end{pmatrix}, λ\)$, kde $λ ≠ 1$. 2 slabě samodružné přímky a 2 slabě samodružné body. Je to stejnolehlost složená s elací (tj. s osovou afinitou, kde směr je rovnoběžný s osou).
		\item $J_{2b} = \diag \(\begin{pmatrix} 1 & 1 \\ 0 & 1 \end{pmatrix}, 1\)$. 1 silně samodružná přímka a jeden silně samodružný bod.
	\end{itemize}
	1 buňka:
	\begin{itemize}
		\item $J_3 = \begin{pmatrix} 1 & 1 & 0 \\ 0 & 1 & 1 \\ 0 & 0 & 1 \end{pmatrix}$ (to je jediná až na podobnost matice se 3 Jordanovy buňky). Jeden slabě samodružný bod a jedna slabě samodružná přímka.
	\end{itemize}
\end{poznamka}

% 03. 04. 2024

TODO? (definice bilineární formy; antisymetrické a symetrické BL formy; (jednoznačný) rozklad na symetrickou a antisymetrickou část; kvadratická forma ($g_2$) určená BL formou $g$ neboli její symetrickou částí $g_s$; zpětná rekonstrukce $g_s$ z $g_2$; polární báze (to je ta, ve které je matice kvadratické formy diagonální); Sylvestrův zákon setrvačnosti; signatura; Sylvestrovo kritérium; regulární matice má počet nul v signatuře nulový a opačně)

\begin{definice}[Vrchol symetrické bilineární formy]
	Vrchol symetrické bilineární formy $g$ na $®R^n$ je množina $V(g) := \{¦u \in ®R^n | \forall ¦v \in ®R^n : g(¦u, ¦v) = 0\}$.

	\begin{poznamkain}
		Ekvivalentně $g(¦v, ¦u) = 0$. Maticově $\forall ¦v: ¦u^T G ¦v = 0 \Leftrightarrow ¦u^t·G = ¦o$ nebo $G·¦u = ¦o$.

		Čili $V(g) = \Ker G$.
	\end{poznamkain}
\end{definice}

\section{Kvadriky}
\begin{definice}[Kvadrika]
	Kvadrika v $®RP^n$ určená kvadratickou formou $g_2$ (na $®R^{n+1}$) je množina $Q_g := \{\<¦u\> \in ®RP^n | g_2(¦u) = 0\}$.
\end{definice}

\begin{priklady}
	TODO?

	$$ G = \begin{pmatrix} 1 & 0 & 0 \\ 0 & 1 & 0 \\ 0 & 0 & 1 \end{pmatrix} $$
	je formálně reálná kuželosečka (nemá reálné body).

	TODO?
\end{priklady}

\begin{dusledek}
	Kvadrika na $®RP^n$ je dána symetrickou maticí $(n+1) \times (n+1)$ určenou až na násobek $≠0$.
\end{dusledek}

\begin{dusledek}
	Kvadrika v $®RP^n$ je určena symetrickou maticí $3\times 3$, tedy 6 čísly, až na násobek, tedy o jedno číslo méně, tj. 5 čísly.

	\begin{dusledekin}
		$X = \<¦x\> \in Q_g \Leftrightarrow x^T·G·x = 0$, tedy zadání 5 bodů je stejné jako zadání 5 lineárních rovnic (o 6 neznámých).
	\end{dusledekin}
\end{dusledek}

\begin{definice}[Regulární kvadrika, singulární kvadrika]
	Kvadrika je regulární/singulární, pokud její matice vůči libovolné bázi je regulární/singulární.

	\begin{poznamkain}
		To je ekvivalentní s nulou/nenulou v počtu nul v signatuře matice.
	\end{poznamkain}
\end{definice}

\begin{definice}[Vrchol kvadriky]
	Vrchol kvadriky je množina $V(Q) := \{X | ¦x \in V(g) = \Ker G\}$.
\end{definice}

\end{document}
