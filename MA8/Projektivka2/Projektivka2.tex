\documentclass[12pt]{article}					% Začátek dokumentu
\usepackage{../../MFFStyle}					    % Import stylu

\begin{document}

% 21. 02. 2024

\section{Ohniska kuželoseček}
\subsection{Konstrukce s imaginárními elementy}
\begin{poznamka}
	Všimněme si, že projektivita dvou soumístných soustav určuje jednoznačně pár samodružných elementů, ale opačně ne. Pokud však vezmeme involuci, tak ta už má jednoznačnou korespondenci involuce s párem samodružných elementů.
\end{poznamka}

\begin{priklad}[Konstrukce]
	Je-li dána projektivita soumístných bodových soustav na přímce, určete involuci, která má tytéž samodružné body. (Totéž duálně.)

	\begin{reseni}[Duální]
		Zvolíme pomocnou kružnici procházející daným bodem. Převedeme soustavy na bodové soustavy na kružnici. Vezmeme direkční přímku za poláru a najdeme k ní (přes tečny) pól. Nyní uvažujme involuci se středem v tomto bodě. Obraz v hledané involuci najdeme tak, že vzor převedeme na kružnici, zobrazíme v této involuci, a vrátíme zpět.
	\end{reseni}

	\begin{poznamkain}
		Pokud direkční přímka vyjde mimo kružnici, budou samodružné body komplexní a pól najdeme tak, že leží na polárách k bodům (pólům) ležícím na dané poláře.
	\end{poznamkain}
\end{priklad}

\begin{veta}
	Pro eliptickou involuci (bodových soustav na přímce) existují právě dva body v rovině, z nichž se tato involuce promítá absolutní involucí (to znamená involucí kolmic).
	
	\begin{dukazin}
		Pro eliptickou involuci se její páry rozdělují. Tedy nad úsečkami vzor – obraz si uděláme Thaletovy kružnice a hledané body budou jejich průsečíky.
	\end{dukazin}
\end{veta}

\begin{definice}
	Body z předchozí věty se nazývají pomocné body eliptické involuce.
\end{definice}

\begin{poznamka}[Platí]
	Absolutní involuce je eliptická involuce, jejíž samodružné přímky jsou imaginární. Nazývají se izotropické přímky a jejich směry jsou $[0:1:i]$ a $[0:1:-1]$.
\end{poznamka}

\begin{poznamka}
	Izotropické body leží na každé kružnici v rovině. Každé izotropická přímka je kolmá sama na sebe (v reálném skalárním součinu, z definice absolutní involuce)
\end{poznamka}

\subsection{Ohnisko středových kuželoseček}
\begin{dusledek}
	Pokud kuželosečka není kružnice, pak izotropické body na ní neleží, tedy z každého izotropického bodu k takové kuželosečce existují 2 tečny (? 4 imaginární přímky). Lze ukázat, že ze 6 průsečíků těchto 4 přímek jsou vždy dva reálné.
\end{dusledek}

\begin{definice}[Ohnisko]
	Těmto dvěma bodům budeme říkat ohniska dané kuželosečky.
\end{definice}

\begin{veta}
	Bod je ohniskem kuželosečky $\Leftrightarrow$ involuce sdružených polár indukovaná v tomto bodě kuželosečkou je involuce absolutní.

	\begin{dukazin}
		Samodružné přímky involuce sdružených polár jsou právě tečny z tohoto bodu.
	\end{dukazin}
\end{veta}

\begin{veta}
	1. Kuželosečka má 2 ohniska ($E, F$) (pro kružnici splývající), jsou umístěna symetricky podle středu na jedné z os kuželosečky. Ohniska jsou samodružné body involuce bodů na této ose, jejíž páry jsou vyťaty sdruženými kolmými polárami. A tedy i páry tečna+jejich normála (kolmice v bodě dotyku = pól tečny).

	2. Každé z ohnisek je pomocným bodem eliptické involuce, kterou na druhé ose vytínají sdružené kolmé poláry (a tedy i dvojice tečna+normála).

	3. Každá kružnice opsaná trojúhelníku danému druhou osou a sdruženými kolmými polárami protíná původní osu v ohniscích. (Vyplývá z předchozí části.)

	\begin{dukazin}
		Bez důkazu.
	\end{dukazin}
\end{veta}

\begin{definice}[Hlavní osa, vedlejší osa]
	Ose z předchozí věty se říká hlavní osa, druhé pak vedlejší.
\end{definice}

\begin{priklad}[Konstrukce]
	Dány osy elipsy s vrcholy, najděte ohniska.

	\begin{reseni}[Podobné hledání hyperoskulační kružnice.]
		K spojnici hlavního a vedlejšího vrcholu umíme najít pól (průsečík tečen = kolmic na osy). Z tohoto pólu vedeme kolmici, čímž jsme získali dvojici kolmých sdružených polár, tedy použijeme předchozí větu, bod 3.
	\end{reseni}

	Totéž pro hyperbolu: na hlavní ose máme zadané vrcholy, na vedlejší náhradní body.

	\begin{reseni}
		Polára bude tentokrát průsečík „těch druhých dvou kolmic v hlavním a vedlejším vrchole“, neboť pomocné body jsou takové, že přesně tento bod leží na asymptotě (tečně v nevlastním bodě).
	\end{reseni}
\end{priklad}

\subsection{Ohnisko paraboly}
\begin{definice}[Ohnisko]
	(Stejná.) Ohnisko paraboly je reálný průsečík izotropických tečen.

	\begin{poznamkain}
		Tuto definici splňují 2 body: vlastní ohnisko $F$ a nevlastní ohnisko = střed = směr průměrů = směr osy.
	\end{poznamkain}

	\begin{poznamkain}
		Polára vlastního ohniska = řídící přímka.
	\end{poznamkain}
\end{definice}

\begin{veta}
	1. Bod je ohniskem paraboly $\Leftrightarrow$ involuce sdružených polár v tomto bodě je involuce absolutní. (Tj. sdružené poláry v $F$ jsou vzájemně kolmé.)

	2. Spojnice vlastního a nevlastního ohniska = osa paraboly, vlastní ohnisko půlí každou úsečku vyťatou na ose sdruženými kolmými polárami (speciálně tečnou a její normálou).
\end{veta}

\begin{priklad}[Konstrukce]
	Zkonstruujte ohnisko paraboly zadané 4 tečnami.

	\begin{reseni}
		Najdeme osu a bod dotyku na libovolné nevrcholové tečně. Z něj vedeme kolmici a použijeme předchozí větu bod 2.
	\end{reseni}
\end{priklad}

\begin{veta}
	Ohnisko jsou pro kuželosečku 2 podmínky.

	\begin{dukazin}
		Ohnisko zadává 2 izotropické tečny, tedy 2 podmínky.
	\end{dukazin}

	\begin{poznamkain}
		2 ohniska + 1 bod (mimo osu = jejich spojnice) nezadávají jednoznačně kuželosečku, zadávají však jednoznačně elipsu a hyperbolu. A tyto dvě kuželosečky se v daném bodu protínají kolmo (úhel mezi tečnami).
	\end{poznamkain}
\end{veta}

% 06. 03. 2024

\section{Analytická geometrie}
\begin{definice}[Projektivní prostor, geometrický bod, aritmetický zástupce]
	(Reálný) projektivní prostor dimenze $n$ je množina
	$$ ®RP^n = \{\<v\>, v \in ®R^{n+1} \setminus ¦o\} = \text{ množina všech přímek (procházejících počátkem) v } ®R^{n+1}. $$
	Prvek $\<v\> \in ®RP^n$ se nazývá geometrický bod a $v$ jeho aritmetický zástupce

	\begin{poznamkain}[Platí]
		$\<v\> = \<w\>$ (tj. stejné geometrické body) $\Leftrightarrow$ $\exists α \in ®R \setminus \{0\}: w = α·v$ (tj. aritmetičtí zástupci se liší pouze násobkem $≠ 0$).
	\end{poznamkain}
\end{definice}

\begin{definice}[Homogenní souřadnice]
	Je-li $v = (x_0, …, x_n) \in ®R^{n+1} \setminus \{¦o\}$, pak homogenní souřadnice geometrického bodu $\<v\>$ jsou $[x_0:…:x_n]$.

	\begin{poznamkain}
		Jsou určeny až na násobek $≠0$.
	\end{poznamkain}
\end{definice}

\begin{definice}[Projektivní přímka, projektivní rovina, projektivní prostor]
	$®RP^1$ říkáme projektivní přímka.
	$®RP^2$ říkáme projektivní rovina.
	$®RP^3$ říkáme projektivní prostor.
	%$®RP^{n-1}$ říkáme (v $®RP^n$) projektivní nadrovina.
\end{definice}

\begin{poznamka}[Značení]
	Místo $\<a\>$ budeme psát $A$.
\end{poznamka}

\begin{poznamka}
	$®RP^1$: Dva body $A, B$ jsou totožné $\Leftrightarrow$ vektory $a, b$ jsou lineárně závislé.

	$®RP^2$: Tři body $A, B, C$ leží na jedné přímce (po dvou různé) $\Leftrightarrow$ $a, b, c$ jsou lineárně závislé (po dvou lineárně nezávislé), tj. leží v jedné rovině.

	$®RP^3$: Čtyři body $A, B, C, D$ leží v rovině $\Leftrightarrow$ $a, b, c, d$ jsou lineárně závislé, tj. leží v jednom prostoru.

	Obecně $®RP^n$: $n+1$ bodů $A_0, …, A_n$ leží v $n-1$-dimenzionálním projektivním prostoru $\Leftrightarrow$ vektory $a_0, …, a_n$ leží v nadrovině v $®R^{n+1}$ (jsou lineárně závislé).
\end{poznamka}

\begin{poznamka}
	Procesu „zakážu ¦o a ztotožníme násobky“ říkáme projektivizace.
\end{poznamka}

\begin{definice}[Projektivní rozšíření afinního prostoru, vlastní bod, nevlastní bod]
	Projektivní rozšíření afinního prostoru $®R^n$ na projektivní prostor $®RP^n$ (= kanonické vnoření $®R^n$ do $®RP^n$) je zobrazení, které bodu $[x_1, …, x_n]$ přiřadí $[1:x_1:…:x_n]$ a vektoru $(x_1, …, x_n)$ přiřadí $[0:x_1:…:x_n]$.

	Prvním říkáme body vlastní, druhý nevlastní.
\end{definice}

TODO?

\begin{definice}[Homogenní souřadnice přímky]
	V $®RP^2$ zavádíme homogenní souřadnice přímky $a_0 x_0 + a_1 x_1 + a_2 x_2 = 0$ jako homogenní trojici $(a_0:a_1:a_2)$.

	\begin{poznamkain}
		Opět určeny až na násobek $≠ 0$.
	\end{poznamkain}
\end{definice}

\begin{priklady}
	$(0:1:0)$ je osa $y$, $(0:0:1)$ je osa $x$, $(1:0:0)$ je nevlastní přímka.
\end{priklady}

\begin{priklad}[Hledání průsečíku dvou přímek]
	TODO?
\end{priklad}

\begin{priklad}[Incidence bodů]
	$X = [x_0:x_1:x_2]$, $a = (a_0:a_1:a_2)$. $X \in a$ $\Leftrightarrow$ $a_0 x_0 + a_1 x_1 + a_2 x_2 = 0$.

	\begin{dusledekin}
		Lze zaměnit bod za přímku $\implies$ dualita.

		Tj. například spojnice bodů se počítá stejně jako průsečík přímek.
	\end{dusledekin}
\end{priklad}

\begin{poznamka}[Trik na nalezení spojnice (/průsečíku)]
	Dány body $Y = [y_0:y_1:y_2]$, $Z = [z_0:z_1:z_2]$. Chceme rovnici jejich spojnici: $X \in a = YZ \Leftrightarrow a_0x_0 + a_1x_1 + a_2x_2 = 0$ pro hledané souřadnice $(a_0:a_1:a_2)$ $\Leftrightarrow$ vektory $X$, $Y$ a $Z$ jsou závislé $\Leftrightarrow$ $\det\((X|Y|Z)^T\) = 0$ $\Leftrightarrow$
	$$ \Leftrightarrow (y_1·z_2 - y_2·z_1)·x_0 + (-y_0·z_2 + y_2·z_0)·x_1 + (y_0·z_1 - y_1·z_0)·x_2 = 0. $$
\end{poznamka}

\subsection{Dvojpoměr}
\begin{definice}[Dvojpoměr]
	Dvojpoměr 4 vektorů v rovině $a, b, c, d$, po dvou lineárně nezávislých, ale po třech lineárně závislých (tj. BÚNO $c = α_1 a + β_1 b$, $d = α_2·a + β_2 b$) definujeme jako $(abcd) := \frac{α_2·β_1}{α_1·β_2} \in ®R$.

	\begin{poznamkain}
		Zřejmě tato hodnota nezávisí na volbě (nenulového) násobku každého vektoru.
	\end{poznamkain}

	Dvojpoměr 4 bodů $A, B, C, D \in ®RP^n$ ležících na jedné přímce definujeme jako $(ABCD) := (abcd)$.
\end{definice}

\begin{tvrzeni}[Už jsme si dokázali]
	$A, B, C, D$ jsou čtyři různé $\implies$ $(ABCD) ≠ 0, 1, ∞$. (Např. $C = A \lor B = D \Leftrightarrow (A B C D) = 0$.)

	$A, B, C, D$ vlastní $\implies$ $(ABCD) = \frac{(ABC)}{(ABD)}$. $A, B, C$ vlastní, $D$ nevlastní $\implies$ $(ABCD) = (ABC)$.

	Věta o 4 determinantech (pro $A, B, C, D \in ®RP^1$):
	$$ (ABCD) = \frac{(a_0·c_1 - a_1·c_0)·(b_0·d_1 - b_1·d_0)}{(a_0·d_1 - a_1·d_0)·(b_0·c_1 - b_1·c_0)}. $$
\end{tvrzeni}

\begin{definice}[Harmonická čtveřice]
	$$ (ABCD) = -1 $$
\end{definice}

\begin{priklad}[Pak jsme počítali. V jednu chvíli nám vyšlo:]
	Parametrizace přímky procházející $A, B$ je $t_1·A + t_2·B$, kde například $t_1 + t_2 = 1$; lépe $t·A + (1 - t)·B$.
\end{priklad}

\begin{definice}[Projektivní souřadný systém (PSS)]
	Projektivní souřadný systém v $®RP^n$ je $(n+1)$-tice různých bodů $A_0, …, A_n \in ®RP^n$. Pak $\forall X \in ®RP^n$ definujeme souřadnice bodu $X$ vůči PSS $(A_0, …, A_n)$ jako homogenní $(n+1)$-tici $[x_0:…:x_n]$ takovou, že $x = \sum_{i=0}^n x_i·a_i$.
\end{definice}

\end{document}
