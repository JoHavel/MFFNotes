\documentclass[12pt]{article}					% Začátek dokumentu
\usepackage{../../MFFStyle}					    % Import stylu

\newcommand{\str}{^\wedge}

\begin{document}

% 20. 02. 2024

\section{$Σ_1^1$ sets and trees on $ω$}
\begin{poznamka}[Notation]
	\ 

	\begin{itemize}
		\item $®S := ω^{<ω}$;
		\item $ν|_k = (ν(0), …, ν(k-1))$, $ν \in ®S \cup ω^ω$ ($ν|_0 = \O$, empty sequence);
		\item $t \prec s ≡ \exists s' \in ®S \cup ©N: s = t\str s'$ ($t \in ®S, s \in ®S \cup ©N$);
		\item $©N := ω^ω$;
		\item $|s|$ is the length of $s$, $s \in ®S$ ($s = (s(0), …, s(k-1)) \implies |s| = k$);
		\item $s \in ®S$, $ν \in ®S \cup ©N$: $s\str ν = (s(0), …, s(|s| - 1), ν(0), …)$.
	\end{itemize}
\end{poznamka}

\begin{definice}[Souslin set (on TP space)]
	$X$ topological space. We say $S \subset X$ be Souslin $\Leftrightarrow$ $\exists (F_s)_{s \in S}$ Souslin scheme of closed subset of $X$ such that $S = ©A_s(F_s) = \bigcup_{ς \in ©N} \bigcap_{n \in ω} F_{ς|_n}$.
\end{definice}

\begin{poznamka}
	a) $P$ Polish topological space, then $A \in Σ_1^1 \Leftrightarrow A$ Souslin in $P$. (We already know.)

	b) $P$ topological space, then $A \subset P$ Souslin $\Leftrightarrow$ $\exists F \in ∏_1^0(©N \times P): A = ∏_P(F)$. (Difficult.)

	c) We will assume only regular Souslin scheme (RSS): $F_{s\str t} \subset F_s$, $s, t \in ®S$ and $F_\O = P$.
\end{poznamka}

\subsection{Souslin operation and trees}
\begin{definice}[Trees on $ω$, infinite branch, ill-founded trees, well-founded trees]
	We define set of trees ©T by $©T := \{T \in ©P(®S) | \forall  s \in T, t \in T: t \prec s \implies t \in T\}$.

	$T \in ©T$ has infinite branch $≡$ $\exists ς \in ©N \forall n \in ω: ς|_n \in T$ (i.e. $ς \in [T]$) (i.e. $[T] ≠ \O$).

	Trees with infinite branches are called ill-founded (IF). The set of IF trees is denoted by $©T_I$. Trees without infinite branches are called well-founded (WF). The set of WF trees is denoted by $©T_W$.

	$©T_s := \{T \in ©T | s \in T\}$ are all trees containing $s \in ®S$.

	\begin{poznamkain}
		$®T_I = \bigcup_{ς \in ©N} \bigcap_{n \in ω} ©T_{ς|_n}$.
	\end{poznamkain}

	$©T^* := ©T \setminus \{\O\}$, $©T_W^* = ©T_W \setminus \{\O\}$.
\end{definice}

\begin{lemma}
	Let $X$ be a topological space, $(F_s)_{s \in ®S}$ RSS of closed subsets of $X$, $S := ©A_s(F_s)$. Define $f(x): X \rightarrow ©T^*$ by $f(x) := \{s \in ®S | x \in F_s\}$. Then $F_s = f^{-1}(®T_s)$ and $S = f^{-1}(©T_I)$.

	\begin{dukazin}[?]
		a) „$f: X \rightarrow ©T$“: $s \in f(x) \implies x \in F_s \implies F_s \subset F_t \implies x \in F_t \implies t \in f(x)$ ($t \prec s$).

		b) $x \in F_s \Leftrightarrow s \in f(x) \Leftrightarrow f(x) \in ©T_s \Leftrightarrow x \in f^{-1}(®T_s)$

		c) lemma $\impliedby$ b) and the next remark.
	\end{dukazin}
\end{lemma}

\begin{poznamka}
	TODO!!! $©T \rightarrow ©T^*$.

	\begin{dukazin}
		„$\implies$“: lemma?. „$\impliedby$“: $S = f^{-1}(®T_I) = f^{-1}\(\bigcup_{} \bigcap_{n \in ω} ©T_{ς|_n}\) = \bigcup_{ς \in ©N} \bigcap_{n \in ω} f^{-1}(®T_{ς|_n})$, where $f^{-1}(®T_{ς|_n}) \in ∏_1^0(X)$ $\implies$ Souslin.
	\end{dukazin}
\end{poznamka}

\subsection{Trees as PTS (compact)}
\begin{poznamka}[Topology on trees]
	$©P(®S) = \{A \subset ®S\} = \{0, 1\}^{®S}$ (product topology of product of discrete topologies) which is compact and homeomorphic to $2^ω$. We assume on ®T subspace topology.
\end{poznamka}

\begin{tvrzeni}
	$®T, ©T^* \in ∏_0^1(\{0, 1\}^®S)$, $\{®T_s, ®T^* \setminus ®T_s, s \in ®S\}$ form a subbase of topology in ®T.

	\begin{poznamkain}
		©T, $©T^*$ is compact metric space, so PTS.
	\end{poznamkain}

	\begin{dukazin}
		$S \in \{0, 1\} \setminus ®T$ $\Leftrightarrow$ $\exists s, t \in ®S, s \prec t:$ $t \in S \land s \notin s$ $\implies$ $\{0, 1\} \setminus ®T = \bigcup_{t \in ®S} \bigcup_{s \prec t} (\{T, χ_T(t) = 1\} \cap \{T; χ_T(s) = 1\})$.

		$\{T | χ_T(t) = 1\}$, $\{T | χ_T(s) = 0\}$ is subbase of product topology.

		$$ ©T^* = ©T \cap \{A \in \{0, 1\} | χ_A(\O) = 1\} \implies ©T^* \in ∏_1^0(©T) \implies ©T^* \text{ is compact}. $$
	\end{dukazin}
\end{tvrzeni}

\subsection{Properties of $f$ from the lemma}
\begin{definice}
	$T \in ®T$, $ς \in ©N$. $h_ς(T) := \sup\{k \in ω | ς|_k \in T\} \in ω \cup \{∞\}$.
\end{definice}

\begin{poznamka}[Remind Lebesgue–H?–Banach characterization]
	$X, Y$ metric spaces, $Y$ separable, $1 ≤ α < ω_1$, $f: X \rightarrow Y$. Then $f$ is Baire$_α$ $\Leftrightarrow$ $f$ is $Σ_{α + 1}^0(X)$-measurable.
\end{poznamka}

\begin{tvrzeni}
	$X$ metrizable (we need only $Σ_1^0(X) \subset Σ_2^0(X)$), $S \subset X$ Souslin. Then there exists $f: X \rightarrow ®T$ such that:
	\begin{enumerate}
		\item $f^{-1}(®T_I) = S$;
		\item $f^{-1}(®T_s) \in ∏_1^0(X)$, $s \in ®S$;
		\item $h_ς ∘ f$ is upper semi-continuous ($h_ς ∘ f: X \rightarrow ®R^*$), $ς \in ©N$ (i.e $\{x \in X | h_ς(f(x)) < n\}$ is open $\forall ς \in ©N, n \in ®R^*$);
		\item $f$ is Baire$_1$ (i.e. $Σ_2^0$-measurable).
	\end{enumerate}

	\begin{dukazin}
		1. and 2. is from the lemma. „4.“: $®T$ separable metric space. So, it is enough to prove it for subbase. $f^{-1}(®T_s) \in ∏_1^0 \subset Σ_2^0$, $f^{-1}(®T \setminus ®T_s) \in Σ_1^0 \subset Σ_2^0(X)$. „3.“: $\{x \in X | h_ς(f(x)) < n\} = f^{-1}(\{T \in ®T | ς|_n \notin T\}) = f^{-1}(®T \setminus ®T_{ς|_n})$ is open (by the lemma). And $\{x \in X | h_ς(f(x)) < ∞\} = \bigcup_{n \in ω} \{x \in X | h_ς(f(x)) < n\}$.
	\end{dukazin}
\end{tvrzeni}

\subsection{Examples of $Σ_1^1$ non-$Δ_1^1$ sets}
\begin{poznamka}
	$$ Σ_1^1(X) \setminus ∏_1^1(X) = Σ_1^1(X) \setminus Δ_1^1(X) \overset{?} ≠ \O. $$
\end{poznamka}

\begin{lemma}
	$©T_I \in Σ_1^1(©T) \setminus Δ_1^1(©T), ©T_W \in ∏_1^1(©T) \setminus Δ_1^1(©T)$.

	\begin{dukazin}
		1. $©T_I \in Σ_1^1(®T) \impliedby ®T_I = \bigcup \bigcap ©T_{ς|_n}$ souslin in PTS.

		2. „$©T_I \notin Δ_1^1(®T)$“: By continuity $©T_I \in Δ_1^1 \implies ©T_W \in Δ_1^1 \implies ©T_W \in Σ_1^1 \implies ©T_W$ souslin. \lightning.
	\end{dukazin}
\end{lemma}

\begin{poznamka}
	$f_I$, $f_W$ are mappings from the lemma for $S = ©T_I$ and $S = ©F_W$. Clearly $f_I = \id$.
\end{poznamka}

\begin{definice}
	$f: ©T \rightarrow ©T$ by $f(T) := f_I(T) \cap f_W(T) = T \cap f_w(T)$. $f(T) \in ©T \impliedby (A, B \in ©T \implies A \cap B \in ©T)$.
	$$ T \in ©T_W \implies f(T) = T \cap f_W(T) \subset T \implies f(T) \in ©T_W. $$
	$$ T \in ©T_I \implies f(T) \subset f_w(T) \in ©T_W \impliedby (\text{the lemma } \implies f^{-1}(©T_I) = ©T_W \implies f^{-1}(©T_W) = ©T_I) \implies f(T) \in ©T_W. $$
	$\implies f: ©T \rightarrow ©T_W \implies h_ς ∘ f: ©T \rightarrow ω$. From the previous proposition $h_ς ∘ f$ is usc, so $h_ς ∘ f$ is usc real function on compact set. Thus $m(ς) := \max_{T \in ®T} h_ς(f(T)) \in ω$.
\end{definice}

% 27. 02. 2024


\begin{dukaz}[The previous lemma]
	By contradiction $©T_I \in Δ_1^1(©T^*) \implies ©T_W^* \in Σ_1^1(©T^*)$. $f(T) = f_I(T) \cap f_W(T)$, $f: ©T^* \rightarrow ©T^*$, $f: ©T^* \rightarrow ©T_W^*$. $\exists m(ς) := \max_{T \in ©T^*} h_ς(f(T)) \in ω$.

	Define $T_0 \in ©T^*: s \in T_0 \Leftrightarrow ς \in ©N: ς|_{m(ς) + 1} \succ s$. $T_0 \in ©T^*$, $\{\O\} \in T_0$, $T_0 \in ©T$ trivial. $T_0 \in ©T_W^*$. By contradiction $ς \in [T_0] \implies ς|_{m(ς) + 2} \in T_0 \implies \exists ν \in ©N: ς|_{m(ς) + 2} \prec ν|_{m(ν) + 1} \implies ν|_{m(ς) + 1} = ς|_{m(ς) + 1}$. Definition of $m(ν)$ gives $\exists T \in ©T^*: m(ν) = h_ν(f(T)) \implies ν|_{m(ν)} \in f(T) \implies ς|_{m(ς) + 1} \in f(T) \implies h_ς(f(T)) ≥ m(ς) + 1$. \lightning.

	Clearly
	$$ T_0 \supseteq \bigcup_{T \in ©T^*}(T). T_0 \in ©T_W^* \implies f_W(T_0) \in ©T_I \implies \exists ς_0 \in [f_W(T_0)] \implies $$
	$$ \implies h_{ς_0}(f(T_0)) = \min \{k \in ω | ς_0|_k \in T_0 \cap f_W(T_0)\} = \min\{k \in ω | ς_0|_k \in T_0\} \supseteq m(ς_0) + 1. \text{\lightning}. $$
\end{dukaz}

\begin{veta}
	$X$ PTS, $A \in Σ_1^1(X)$, $\card(A) > \card(ω)$. Then there exists $B \subset A$ such that $B \in Σ_1^1(X) \setminus Δ_1^1(X)$.

	\begin{dukazin}
		$\card(A) > ω \implies \exists C \subset A$ homeomorphic copy of $2^ω \sim 2^{®S}$. $2^{®S} \overset{h}\hookrightarrow A$ then  $h(©T_I) \in Σ_1^1(X) \setminus Δ_1^1(X)$. Homeomorphism of $Σ_1^1$, $Δ_1^1$ set is $Σ_1^1$, $Δ_1^1$ set.
	\end{dukazin}
\end{veta}

\begin{poznamka}
	Let $Γ$ be class of subsets of PTS and $X$ be PTS. We say that $A$ is $Γ(X)$-hard $≡$ $\forall B \in Γ(©N)\ \exists f \in Δ_1^1, f: ©N \rightarrow X: f^{-1} = B$. $A$ is $Γ(X)$-complete $\Leftrightarrow A \in Γ$ and $A \in Γ$-hard.

	From the previous theorem $A \in Σ_1^1$-complete $\implies$ $A \in Σ_1^1 \setminus Δ_1^1$ (same for $∏_1^1$). ($A \in Δ_1^1 \implies f^{-1}(A) \in Δ_1^1$, but there are $Σ_1^1 \setminus Δ_1^1$ subsets of ©N).
\end{poznamka}

\begin{poznamka}
	$Σ_1^1$-complete $= Σ_1^1 \setminus Δ_1^1$ $\impliedby$ $Σ_1^1$-determinacy.
\end{poznamka}

\begin{poznamka}
	$©T_I \in Σ_1^1$-complete, $©T_W^* \in ∏_1^1$-complete.
\end{poznamka}

\begin{definice}[Universal set]
	$X$ PTS, $Γ$ class of subsets of PTS. We say that $A$ is $Γ(X)$-universal $≡$ $A \in Γ(X \times ©N) \land Γ(X) = \{A^s | s \in ©N\}$.
\end{definice}

\begin{poznamka}
	$X$ PTS. Then
	\begin{enumerate}
		\item there exists $Σ_1^0(X)$-universal set;
		\item there exists $∏_1^0(X)$-universal set;
		\item there exists $Σ_1^1(X)$-universal set.
	\end{enumerate}

	\begin{dukazin}
		„1.“: $\{B_n\}$ base of $X$. $G := \bigcup_{n \in ω, s \in ω} (B_{s(0)} \cup B_{s(1)} \cup … \cup B_{s(n-1)}) \times B(s)$ ($B(s) = \{ς \in ©N | s \prec ς\}$). $G \in Σ_1^0(X \times ©N)$ trivial. $ς \in ©N \implies G^{ς} \in Σ_1^0(X)$ trivial ($G^ς = \bigcup_{n \in ω} (B_{ς(0)} \cup B_{ς(1)} \cup … \cup B_{ς(n-1)})$ open). $U \in Σ_1^0(X) \implies \exists ς \in ©N: U = \bigcup_{n \in ω} B_{ς(n)} = G^ς$.

		„2.“: G $Σ_1^0(X)$-universal $\implies$ $(X \times ©N) \setminus G$ is $∏_1^0(X)$-universal.

		„3.“: $Y = ©N \times X$. Let $F \in ∏_0^1(Y \times ©N)$ be $∏_1^0(Y)$-universal. $∏: ©N \times X \times ©N \rightarrow X \times ©N$ be projections on 2nd and 3rd coordinate. $A := ∏(F)$. $A$ is $Σ1^1(X)$-universal. Clearly $A \in Σ_1^1(X \times ©N)$, $A^ς \in Σ_1^1(X)$ for $ς \in ©N$ trivial. Let $B \in Σ_1^1(X) \implies \exists C \in ∏_1^0(©N \times X): B = ∏_2(C) \implies \exists ς \in ©N: C = F^ς$.
		$$ A^ς = (∏_{2, 3}(F))^ς = ∏_2(F^ς) = π_2(C) = B. $$
	\end{dukazin}
\end{poznamka}

\begin{poznamka}
	Let $A \in Σ_1^1(©N^2)$ be $Σ_1^1(©N)$ universal. Then
	$$ M := \{x \in ©N | (x, x) \notin A\} \in Σ_1^1(©N) \impliedby (M \in Σ_1^1 \implies \exists ς \in ©N: M = A^ς.) (ς \in M?: ς \in M \implies (ς, ς) \in A \implies ς \notin M; ς \notin M \implies (ς, ς) \notin A \implies ς \in M). $$
	$$ \{x \in ©N | (x, x) \in ©A\} \in Σ_1^1(©N) \impliedby \text{ diagonal is closed } \implies \{x \in ©N | (x, x) \in A\} \in Σ_1^1 \setminus Δ_1^1. $$
\end{poznamka}

\subsection{Derivative of trees}
\begin{definice}[Derivative]
	$T \in ©T$. $T' := \{s \in ®S | \exists n \in ω: s\str n \in T\}$. $T^{(0)} := T$. $ς < ω_1: T^{(α + 1)} = \(T^α\)'$, $λ$-limit ordinal: $T^{(λ)} := \bigcap_{α < λ} T^{(α)}$. $d_α(T) := T^{(α)}$, $α < ω_1$, $d_α: ©T \rightarrow ©T$.
\end{definice}

\begin{veta}
	$\forall α < ω_1: d_α \in Δ_1^1(©T^2)$.

	\begin{dukazin}
		$d_α(T) \in ©T$ ($T \in ©T$) trivial.

		a) $d_1^{-1}(©T_s) = \{T \in ©T| \exists n \in ω: s\str \in T\} = \bigcup_{n \in ω} ©T_{s\str n} \in \sum_1^0(©T)$.
		$$ \implies d_1^{-1}(©T \setminus ©T_s) \in ∏_1^0(©T), \qquad d_1^{-1}(\O) = \{\O, \{\O\}\} \in ∏_1^0(©T) \implies $$
		$$ \implies (G \in Σ_1^0(©T)) \implies d_1^{-1}(G) \in Σ_2^0(©T) \implies $$
		$\implies$ $d_1$ is in the first Borel class.

		b) $d_0$-id $\implies$ continuous.

		Induction: c) $α = β + 1$, $d_β \in Δ_1^1 \implies d_α = d_1 ∘ d_β \in Δ_1^1$.

		d) $λ$ limit ordinal, $λ < ω_1$, $\forall α < λ: d_α \in Δ_1^1$.
		$$ d_λ^{-1}(©T_s) = \{T \in ©T | \bigcap_{α \in λ} d_α(T) \ni s\} = \bigcap_{α < λ} d_α^{-1}(©T_s) \in Δ_1^1 \implies $$
		$$ \implies d_λ^{-1}(©T \setminus ©T_s) \in Δ_1^1, \qquad d_λ^{-1}(\O) = \{T \in ©T | \exists α < λ: d_α(T) = \O\} = \bigcup_{α < λ} d_α^{-1}(\O) \in Δ_1^1. $$
	\end{dukazin}
\end{veta}

% 05. 03. 2024

\subsection{Luzin–Sierpinski index (rank, norm)}
\begin{definice}
	$T \in ©T^*$, $i(T) := \min \{α < ω_1 | T^{(α)} = \{\O\}\}$, if exists, otherwise $ω_1$.
\end{definice}

\begin{poznamka}[Notation]
	$T_s := \{t \in ®S | s\str t \in T\}, T \in ©T^*, s \in T$.
\end{poznamka}

\begin{poznamka}[Other indices]
	$T_s \in ©T^*$, $T \in ©T^*$, $s \in T$ trivial.

	Hausdorff index $:= \min \{α < ω_1 | d^{(α)}(T) = d^{(α + 1)}(T)\}$.

	Derivation of sets: $X$ PTS, $K \in ©K(X)$, $K' := \{x \in K | x \text{ is not isolated point in } K\}$. $K^{(α + 1)} := (K^{(α)})'$, $K^{(0)} := K$, $K^{(λ)} := \bigcap_{α < λ} K^{(α)}$ ($λ$ limit ordinal).
\end{poznamka}

\begin{lemma}
	$T_s \in ©T^*$, $i(T_s) = \sup \{\min \{ω_1, i(T_{s\str n}) 1\} | s\str n \in T\}$ ($\sup \O := 0$).

	\begin{dukazin}
		$s \in T \implies T_s ≠ \O$, $T \in T_s$, $l < t$: $s\str t \in T \implies s \str l < s\str t \implies s\str l \in T \implies l \in T_s$.
		$$ i(T_s) = ω_1 \Leftrightarrow T_s \in ©T_I \Leftrightarrow \exists n \in ω: T_{s \str n} \in ©T_I \Leftrightarrow \exists n \in ω: i(T_{s\str n}) = ω_1. $$
		„$i(T_s) < ω_1 \Leftrightarrow T_s \in ©T_W^*$“: $α := \sup_{n \in ω: s\str n \in T} i(T_{s\str n}) + 1$, clearly $\forall n \in ω: s\str T$, $i(T_{s\str n}) ≤ i(T_s) < ω_1 \implies 0 < α < ω_1$. „$α = i(T_s)$“:
		$$ T_s^{(α)} = \bigcup_{s\str n \in T}(\{\O\} \cup n\str T_{s\str n})^{(α)} \subseteq \bigcup_{s\str n \in T}(\{\O\} \cup n\str T_{s\str n}) = \{\O\} \implies i(T_s) ≤ α. $$
		Assume $β < α \implies \exists s\str n \in T: i(T_{s\str n}) + 1 > β \implies T_s^β \supset (\{\O\} \cup n\str T_{s\str n})^{(β)} \supsetneq \{\O\} \impliedby i(\{O\} \cup n\str T_{s\str n}) = i(T_{s \str n}) + 1$. $\implies β < i(T_s) \implies α ≤ i(T_s)$.
	\end{dukazin}
\end{lemma}

\begin{veta}
	a) $T \in ©T_W^* \Leftrightarrow i(T) < ω_1$. b) $i(©T_W^*) = ω_1$ (i.e. $\{i(T) | T \in ©T_W^*\} = \{α < ω_1\}$).

	\begin{dukazin}
		„a)“: $T \in ©T_W^*$. $T ≠ \{\O\} \implies \exists s \in T: |s| ≥ 1$, $\forall n \in ω: s\str n \notin T \implies s \notin T' \implies T' \subsetneq T$. And $\card(T) < ω_1$ $\implies$ $i(T) < ω_1$. $i(\{\O\}) = 0$. It can't happen:
		$$ T ≠ \O, \quad \{\O\}, \quad T' = \O $$
		$$ T \in ©T_I \implies \exists ς \in [T] \implies ς \in [T'] \implies T' \in ©T_I \implies \forall α < ω_1: ς \in [T^{(α)}] \implies T^{(α)} ≠ \{\O\} \implies i(T) = ω_1. $$

		„b)“: $i(\{\O\}) = 0$. Induction $\forall α < ω_1\ \exists T_α \in ©T_W^*: i(T_α) = α$: First step is done; Second: $T_{α + 1} := 1\str T_α \cup \{\O\} \implies i(T_{α + 1}) = α + 1$; Assume $λ$ is limit ordinal, $α \nearrow λ$. $T_λ:=\{\O\} \cup \{n \str T_{α_n} | n \in ω\}$. ($i(T_λ) = \sup \{i(T_{α_n} + 1)\} = λ$.)
	\end{dukazin}
\end{veta}

\subsection{Decomposition of $©T_W^*$ and cosouslin sets}
\begin{definice}
	$α < ω_1: ©T_W(α) := \{T \in ©T^* | i(T) = α\}$.
\end{definice}

\begin{veta}
	$©T_W(α) \in Δ_1^1(©T)$, $α < ω_1$.

	\begin{dukazin}
		$©T_W(α) = d_α^{-1}(\{\O\})$, $d_α \in Δ_1^1$.
	\end{dukazin}
\end{veta}

\begin{poznamka}
	$C$ cosouslin in $X$ ($X \setminus C = S$, which is souslin). $\exists Δ_1^1 f: X \rightarrow ©T^*$: $f^{-1}(©T_I) = S = f^{-1}(©T_W^*) = C$. Define $C_α = f^{-1}(©T_W(α))$, $α < ω_1$. It is called a decomposition of $C$ on $Δ_1^1$ subsets. If $\{α | C_α ≠ \O\}$ is countable $\implies$ $C \in Δ_1^1$. „Inverse implication“ is going to be in some weeks (Theorem 15).
\end{poznamka}

\begin{poznamka}
	$$ A \in ∏_1^1(X) \setminus ∏_2^0(x) \implies ©K(A) \in ∏_1^1-\text{complete}. $$
	$$ A \in ∏_2^0(X) \Leftrightarrow ©K(A) \in ∏_2^0(©K(X)). $$
\end{poznamka}

\subsection{Luzin–Sierpinski index as partial ordering}
\begin{poznamka}[Goal]
	Study $\{(T_1, T_2) \in (©T_W^*)^2 | i(T_1) ≤ i(T_2)\}$.
\end{poznamka}

\begin{definice}
	$f: ®S \rightarrow ®S$ is strategy $≡$ $\forall s \in ®S: |f(s)| = |s|$ (respect length) and $\forall s, t \in ®S: s < t \implies f(s) < f(t)$ (monotone.)
\end{definice}

\begin{poznamka}
	a) $f$ strategy. We define $\overline{f}: ω^ω \rightarrow ω^ω$ by $f(ς) = ®T \Leftrightarrow \forall n \in ω: T|_n = f(ς|_n)$.

	b) For first $|s|$ steps of player I describes $f$ first $|s|$ steps of player II (strategy for II player).

	c) $T \in ©T^*: f(T), f^{-1}(T) \in ©T^*$.

	d) $α < ω_1: (f^{-1}(T))^{(α)} \subset f^{-1}(T^{(α)})$.

	\begin{dukazin}
		„a)“, „b)“ trivial. „c)“: $s \in f(T), t < s \implies \exists x \in T: f(x) = s \implies |x| = |s| ≥ |t| \implies x|_{|t|} \in T \implies f(x|_{|t|}) \in f(T), f(x|_{|t|}) < f(x) = s, |f(x|_{|t|})| = |t| \implies f(x|_{|t|}) = t \implies f(T) \in ©T^*$. $f^{-1}(T) \in ©T^*$ similar.

		„d)“: By induction: First step ($α = 0$) is trivial. For $α = 1$: $s \in (f^{-1}(T))' \implies \exists n \in ω: s\str n \in f^{-1}(T) \implies f(s\str n) \subset f(s), f(s \str n) \in T \implies f(s) \in T \implies f(s) \in T'$ ($\exists m \in ω: f(s \str m) = f(s)\str m$). For successor ordinal: $(f^{-1})^{(β + 1)} = \((f^{-1}(T))^{(β)}\)' \subset (f^{-1}(T^{(β)})) \subset f^{-1}(T^{(β + 1)})$. For limit ordinal $λ < ω_1$: $(f^{-1}(T))^{(λ)} = \bigcap_{α < λ} (f^{-1}(T))^{(α)} \subseteq \bigcap_{α < λ} f^{-1}(T^{(α)}) = f^{-1}(\bigcap_{α < λ} T^{(α)}) = f^{-1}(T^{(λ)})$.
	\end{dukazin}
\end{poznamka}

\begin{lemma}
	$T_1, T_2 \in ©T_W^*$. $i(T_1) ≤ i(T_2) \Leftrightarrow \exists f: ®S \rightarrow ®S$ strategy such that $T_1 \subset f^{-1}(T_2)$ ($f(T_1) \subset T_2$).

	\begin{dukazin}
		„$\impliedby$“: $T_1 \subset f^{-1}(T_2) \implies i(T_1) ≤ i(f^{-1}(T_2)) ≤ i(T_2)$ (second equation holds, because: $(f^{-1}(T_2))^{(α)} \subset f^{-1}(T_2^{(α)})$, put $α = i(T_2) \implies (f^{-1}(T_2))^{(α)} \subseteq \{\O\} \implies i(f^{-1}(T_2)) ≤ α$).

% 12. 03. 2024

		„$\implies$“: a) $i(T_2) = ω_1 \implies T_2 \in ©T_I \implies ς \in [T_2]$. Define $f(s)$ by $f(s) = ς|_s$. Clearly $f$ is strategy and $f(T_1) \subset \{ς|_k, k \in ω\} \subset T_2$.

		b) $i(T_2) < ω_1 \implies T_2 \in ©T^*_W$. We will construct $f$ by induction on $|s|$, $s \in ®S$, and we also want $(+_n): i_{T_1}(s) ≤ i_{T_2}(f(s))$, $s \in T_1$, $|s| ≤ n$ $\implies$ $f(s) \in T_2$, $s \in T_1$ (where $i_T(s) = i(T_s)$, $T \in ©T^*$, $s \in T$).

		Firstly $f(\{\O\}) = \{\O\}$. $f$ monotone, respect length and $(+_0): i_{T_1}(\{\O\}) = i(T_1) ≤ i(T_2) = i_{T_2}(\{\O\})$. Let $f$ be defined for $s \in ®S$, $|s| ≤ n$, $n \in ω$, $f$ respect length and be monotone and satisfy $(+_n)$. Let $s \in ω^n$. i) $s_0 \notin T_1$ or $i_{T_1}(s_0) = 0$ TODO!!!

		ii) $i_{T_1}(s_0) > 0$ TODO!!!

		TODO!!!
	\end{dukazin}
\end{lemma}

\subsection{Luzin–Sierpinski index as $∏_1^1$ rank}
\begin{veta}
	$$ A := \{(T_1, T_2) \in (©T^*)^2 | i(T_1) ≤ i(T_2)\} \in Σ_1^1((©T^*)^2). $$
	$$ C := \{(T_1, T_2) \in (©T^*)^2 | T_1 \in ©T^*_W, i(T_1) ≤ i(T_2)\} \in ∏_1^1((©T^*)^2). $$
	$$ B := \{(T_1, T_2) \in (©T^*)^2 | i(T_1) < i(T_2)\} \in ∏_1^1((©T^*)^2). $$
	$$ D := \{(T_1, T_2) \in (©T_W^*)^2 | i(T_1) ≤ i(T_2)\} \text{ bisouslin in } (©T_W^*)^2. $$

	\begin{dukazin} % Obrázkem
		„$A \implies C$“: Define $h: (©T^*)^2 \rightarrow (©T^*)^2$ homeomorphism by $h(T_1, T_2) = (T_2, T_1)$. Then $(©T^*)^2 \setminus A = h(B) \implies B \in ∏_1^1((©T^*)^2)$.

		„$C \implies B$“: $E := \{(T, T) \in (©T^*)^2 | T \in ©T_W^*\} \simeq ©T_W^* \implies E \in ∏_1^1$. $C = B \cup E \in ∏_1^1((©T^*)^2)$.

		„$D$“: $A \cap (©T_W^*)^2$ Souslin, $D = C \cap \(\(©T_W^*\)^2\) \in ∏_1^1\(\(©T^*\)^2\)$ cosouslin.

		„$A$“: $i(T_1) ≤ i(T_2) \Leftrightarrow \exists f \text{ strategy}: f^{-1}(T_2) \supset T_1$. So $A = ∏(F)$, $F := \{(T_1, T_2, f) \in (©T^*)^2 \times ®S^{®S} | T_1 \subseteq f^{-1}(T_2), f \in ©S\}$, where ©S is set of strategies. We show $F \in ∏_1^0$. Clearly $®S^{®S}$ is PTS.

		a) „$©S \subset ∏_1^0(®S^{®S})$“: $f_n \in ©S$, $f_n \rightarrow f$, $f \in ©S$? Set $s < t$, $s, t \in ®S$ $\implies$ $\forall n \in ω$: $f_n(s) < f_n(t)$ ($f_n \in ©S$). (Convergence in product space is point-wise) $\implies$ $\exists n_0 \in ω\ \forall n ≥ n_0: f_n(s) = f(s)$, $f_n(t) = f(t) \implies f(s) < f(t)$. Similarly $\exists n_1\ \forall n ≥ n_1: f_n(s) = f(s) \implies |f(s)| = |f_n(s)| = |s| \implies f \in ©S$.

		b) $f^{-1}(T_2) \supset T_1$ is $∏_1^0$ cond? $T_1^n \rightarrow T_1$, $T_2^n \rightarrow T_2$, $f_n \rightarrow f$ such that $f_n^{-1}(T_2^n) \supset T_1^n$. By contradiction: $\exists v \in T_1 \setminus f^{-1}(T_2)$. $\exists n_0\ \forall n ≥ n_0: f_n(v) = f(v)$, $v \in T_1^n$, $f(v) \notin T_2^n$ $\implies$ $v \in T_1^n \setminus f_n^{-1}(T_2^n)$. \lightning.
	\end{dukazin}
\end{veta}

\begin{definice}
	$©S: L \rightarrow ω_1$ is $∏_1^1$-rank $≡$ $L \in ∏_1^1(X)$, $X$ PTS and $\exists C \in ∏_1^1(X^2), A \in Σ_1^1(X^2)$: $\{(x, y) \in L^2 | S(x) \subseteq ©S(y)\} = C \cap (X \times L) = A \cap (X \times L)$.

	\begin{poznamkain}
		TODO!!!
	\end{poznamkain}

	\begin{dusledekin}
		TODO!!!
	\end{dusledekin}
\end{definice}

\subsection{Boundedness of $∏_1^1$-rank}
\begin{lemma}
	$X$ PTS, $L \subset X$. Let $©S: L \rightarrow ω_1$ be $∏_1^1$-rank, $L \notin Σ_1^1(X)$ and $B \subset L$, $B \in Σ_1^1(X)$. Then $\sup \{©S(x), x \in B\} < ω_1$.

	\begin{dukazin}
		Define $©S(x) = ω_1$, $x \in X \setminus L$. $A$ as in definition of $∏_1^1$-rank. By contradiction: $\sup ©S(B) = ω_1$. Then
		$$ L = \{x \in X | \exists y \in B: ©S(x) ≤ ©S(y)\} = \{x \in X| \exists y \in X: (x, y) \in A \cap (X \times B)\} = ∏_1(A \cap (X \times B)) \in Σ_1^1. \text{\lightning}. $$
	\end{dukazin}
\end{lemma}

\begin{veta}
	Let $B \subset ©T_W^*$, $B \in Σ_1^1(©T^*)$. Then $\sup\{i(T) | T \in B\} < ω_1$.

	\begin{dukazin}
		Trivial.
	\end{dukazin}

	\begin{poznamkain}
		$B \subset X$ PTS, $B \in Δ_1^1(X) \implies B \in ∏_1^1 \implies \exists f \in Δ_1^1, f: X \rightarrow ©T^*: f^{-1}(©T_W^*) B, f(B) \subset ©T_W^*, f(B) \in Σ_1^1 \implies \{α | f^{-1}(©T_W^*(α)) ≠ \O\}$ is countable.

		$\implies \exists α < ω_1: B \subset f^{-1}(\bigcup_{β < α} ©T_W^*(β))$, $X \setminus B = f^{-1}(©T_I^*)$.
	\end{poznamkain}
\end{veta}

\subsection{Luzin first separation principle}

\begin{veta}
	Assume $M$ is metric space, $S \subset M$ souslin, $A \in Σ_1^1(M)$, $A \cap S = \O$. Then there exists $B \in Δ_1^1(M)$ such that $A \subset B \subset M \setminus S$.

	\begin{dukazin}
		$S$ Souslin $\implies$ $S = f^{-1}(©T_I)$, $f \in Δ_1^1$, $f: M \rightarrow ©T^*$. Define $©S(x) := i(f(x))$.
		$$ f(A) \in Σ_1^1(©T^*), f(A)  \subset ©T_W^* \impliedby A \cap S = \O \implies \sup ©S(A) = α < ω_1 \implies $$
		$$ A \subset B = f^{-1}(\bigcup_{β ≤ α} ©T_W^*(β)) \in Δ_1^1, B \cap S = \O. $$
	\end{dukazin}
\end{veta}

\begin{priklad}
	$\exists C_1, C_2 \in ∏_1^1(®R)$, $C_1 \cap C_2 = \O$, $C_1$ cannot be $Δ_1^1$-separated from $C_2$. ($C_1$, $C_2$ are bisouslin in $C_1 \cup C_2$ and cannot be separated by $Δ_1^1(C_1 \cup C_2)$ set.)

% 19. 03. 2024

	\begin{dukazin}
		$C_1 = \{(S, T) \in (©T^*)^2 | i(s) < i(T)\} \in ∏_1^1 \impliedby$ the theorem above. $C_2 = \{(S, T) \in (©T^*)^2 | i(T) < i(S)\} \in ∏_1^1$. $C_1 \cap C_2 = \O$. $M := C_1 \cup C_2$ $\implies$ $C_1$ and $C_2$ are bisouslin in $M$.

		For contradiction $\exists H \in Δ_1^1((©T^*)^2)$. $C_1 \subset H \subset (©T^*)^2 \setminus C_2 \implies \exists α < ω_1: H \in Σ_α^0((©T^*)^2)$. Find $B \in Δ_1^1 \setminus Σ_{α + 1}^0((©T^*)^2)$ $\impliedby$ use $Σ_j^0$ universal sets $\impliedby$ Kechris.

		Find $f_{B^C}$ from the lemma, $f_{B^C}: (©T^*)^2 \rightarrow ©T^*$, $f_{B^C}^{-1}(©T_I) = (©T^*)^2 \setminus B$, $B = f_{B^C}^{-1}(©T_W^*)$ $\implies$ $Σ_1^1 \ni f_{B^C}(B) \subset ©T_W^*$, $f_{B^C} \in B_{ς_1}$ ($f_{B^C}^{-1}(Σ_1^0) \subset Σ_2^0$).

		From the theorem above $\sup_{x \in B} i(f(x)) = α_B < ω_1$. From the other theorem $\exists T \in ©T_W^*: i(T) > α_B$. Define $F(x) = (f(x), T) \in (©T^*)^2$, $x \in (©T^*)^2$. $F \in B_{ς_1}$.

		Then $F^{-1}(C_1) = B \impliedby x \in B \implies i(f(x)) ≤ α_B < i(T)$, $x \in B \implies f(x) \in ©T_I \implies (f(x), T) \notin C_1$, $\in C_2$.
		$$ F^{-1}(C_1) = F^{-1}(H) \impliedby x \in (©T^*)^2 \implies F(x) \subset C_1 \cup C_2. H \in Σ_α^0, F \in B_{ς_1} \implies B = F^{-1}(H) \in Σ_{α + 1}^0 ((©F^*)^2). \text{\lightning.} $$
	\end{dukazin}
\end{priklad}

\subsection{Luzin second separation principle and reduction theorem}
\begin{veta}[Reduction theorem]
	$C_1, C_2$ cosouslin in metric space $M$. Then there exists cosouslin $D_1, D_2 \subset M$ such that
	$$ \forall i = 1, 2: \quad D_i \subset C_i, \qquad D_1 \cap D_2 = \O, \qquad D_1 \cup D_2 = C_1 \cup C_2. $$

	\begin{dukazin}
		From the lemma $\exists f_i: M \rightarrow ©T^*$, $f_i \in Δ_1^1$, $f_i^{-1}(©T_W^*) = C_i$.
		$$ D_1 := \{x \in M | i(f_1(x)) < ω_1, i(f_1(x)) ≤ i(f_2(x))\} \implies D_1 \subset C_1 \quad (i(f_1(x)) ≤ ω_1). $$
		$$ D_1 := \{x \in M | i(f_2(x)) < i(f_1(x))\} \implies D_2 \subset C_2 \quad (i(f_2(x)) ≤ ω_1). $$
		$D_1 \cup D_2 = C_1 \cup C_2$ ($x \in C_1 \cup C_2 \implies i(f_1(x)) < ω_1 \lor i(f_2(x)) < ω$, if $i(f_1(x)) ≤ i(f_2(x))$ then $x \in D_1$ otherwise $x \in D_2$).

		„$D_1 \cap D_2 = \O$“: Define $F = (f_1, f_2) \in Δ_1^1$, $F: M \rightarrow ((©T^*)^2)$ $\impliedby$ $F^{-1}(U_1 \times U_2) = f_1^{-1}(U_1) \cap f_2^{-1}(U_2)$. ($(©T^*)^2$ has countable base.)
		$$ C = \{(T_1, T_2) \in (©T^*)^2 | i(T_1) < ω_1, i(T_1) ≤ i(T_2)\} \in ∏_1^1, $$
		$$ B = \{(T_1, T_2) \in (©T^*)^2 | i(T_2) < i(T_1)\} \in ∏_1^1, $$
		$$ F^{-1}(C) = D_1 \land F^{-1}(B) = D_2 \implies D_1, D_2 \in ∏_1^1 \implies \text{cosouslin}. $$

	\end{dukazin}
\end{veta}

\begin{dusledek}[Luzin second separation principle]
	Let $M$ be metric space, $A_1, A_2$ Souslin in $M$. Then there exists cosouslin $B_1, B_2$ such that $A_2 \setminus A_1 \subset B_1$, $A_1 \setminus A_2 \subset B_2$, $B_1 \cap B_2 = \O$. Moreover, it is possible to manage $B_1 \cup B_2 = M \setminus (A_1 \cap A_2)$ $\implies$ if $A_1 \cap A_2 = \O$, then $B_i$ are bisouslin.

	\begin{dukazin}
		$C_i = M \setminus A_i$, $B_i$ reduction of $C_i$. $B_1 \cup B_2 = C_1 \cup C_2 = M \setminus (A_1 \cap A_2)$, $B_1 \cap B_2 = \O$, $B_i \supset C_i \setminus C_j = A_j \setminus A_i$ ($i ≠ j$). $A_1 \cap A_2 = \O \implies B_1 = M \setminus B_2$.
	\end{dukazin}
\end{dusledek}

\section{Kuratowski–Ulam theorem}
\begin{poznamka}[Notation]
	$A \subset X \times Y$, $X, Y$ sets. $A_X := \{y \in Y | [x, y] \in A\}$. $A^y := \{x \in X | [x, y] \in A\}$.

	$X$ topological space, $T(x)$ statement. $\forall^* x: T(x) \Leftrightarrow \{x \in X | T(x)\}$ is co-meager. $\exists^* x: T(x) \Leftrightarrow \{x \in X | T(x)\}$ is non-meager.
\end{poznamka}

\begin{veta}[Kuratowski–Ulam]
	$X, Y$ be topological spaces with countable base, $A \subset X \times Y$ has Baire property in $X \times Y$. Then
	\begin{enumerate}
		\item $\forall^*x: A_x$ has Baire property in $Y$, $\forall^*y : A^y$ has Baire property in $X$;
		\item $A$ is meager $\Leftrightarrow$ $\forall^* x: A_x$ is meager $\Leftrightarrow$ $\forall^* y: A^y$ is meager;
		\item $A$ is co-meager $\Leftrightarrow$ $\forall^* x: A_x$ is co-meager $\Leftrightarrow$ $\forall^* y: A^y$ is co-meager.
	\end{enumerate}
\end{veta}

\begin{lemma}
	$X, Y$ topological spaces, $Y$ has countable base, $F \subset X \times Y$ nowhere dense. Then $\forall^* x: F_x$ is nowhere dense.

	\begin{dukazin}
		WLOG $Y ≠ \O$. $F \in ∏_1^0(X \times Y)$ (otherwise for $\overline{F}$). Let $U := (X \times Y) \setminus F$. It is open and dense. We want $\forall^* x: \overline{U_x} = Y$.

		$\{V_n\}$ base of $Y$, $V_n ≠ \O$. $U_n := ∏_X(U \cap X \times V_n)$ dense open in $X$. (Open trivial. Dense $\impliedby$ $G \in Σ_1^0(X), G ≠ \O \implies (G \times V_n) \cap U ≠ \O \implies [x, y] \in U \cap (X \times V_n)$.)

		$x \in \bigcap U_n \implies x \in U_n \implies U_x \cap V_n ≠ \O \implies U_x$ is dense in $Y$.
	\end{dukazin}
\end{lemma}

\begin{dukazin}[Kuratowski–Ulam]
	$F \subset X \times Y$ meager $\implies$ $F \subset \bigcup F_n$, $F_n \in ∏_1^0$, nowhere dense. By the previous lemma $\exists M_n \subset X$ co-meager: $\forall x \in M_n$: $(F_n)_x$ is nowhere dense. $M := \bigcap M_n$ co-meager $\implies$ $\forall x \in M$ $\forall n \in ω$: $(F_n)_x$ is nowhere dense $\implies$ $F_x \subset \bigcup (F_n)_x$ is meager.

	Let $A \subset X \times Y$ has Baire property $\implies$ $A = U \triangle M$, $U \in Σ_1^0$, $M$ meager. $A_x = U_x \triangle M_x$ (open $\triangle$ meager for co-meager many $x$) $\implies$ $\forall^* x: A_x$ has Baire property. This implies 1.

	Clearly 2. $\Leftrightarrow$ 3. using complements. It remains to show 2. $\impliedby$.
\end{dukazin}

\begin{lemma}
	$X, Y$ topological spaces with countable base, $A \subset X$, $B \subset Y$. Then $A \times B$ is meager $\Leftrightarrow$ $A$ or $B$ is meager.

	\begin{dukazin}
		„$\implies$“: $A \times B$ meager, $A$ non-meager. Then by the previous lemma $\exists x \in A: (A \times B)_x = B$ meager.

		„$\impliedby$“: $A$ is meager, $A \subseteq \bigcup F_n$, $F_n \in ∏_1^0$, nowhere dense. Then $A \times B \subset \bigcup(F_n \times B)$. We need to show that $F_n \times B$ is nowhere dense. $X \setminus F_n$ open dense $\implies$ $(X \setminus F_n) \times Y$ open dense in $X \times Y$ $\implies F_n \times Y$ is nowhere dense $\implies$ $F_n \times B$ is nowhere dense.
	\end{dukazin}
\end{lemma}

% 26. 03. 2024

\begin{dukazin}[Kuratowski–Ulam remaining 2. $\impliedby$]
	$A \subset X \times Y$ has Baire property, $\forall^* x: A_x$ is meager. $A = U \triangle M$ (open $\triangle$ meager). For contradiction we assume that $A$ is not meager ($U$ is not meager). $\implies$ $\exists G \in Σ_1^0(X), H \in Σ_1^0(Y): G \times H \subset U$, $G \times H$ is not meager ($\impliedby$ $X, Y$ have countable base).

	$\overset{lemma}\implies$ $G, H$ non-meager $\implies$ $\exists x \in G: A_x$ is meager, $M_x$ is meager ($\impliedby \forall^* x : M_x$ is meager). Clearly non-meager $H \setminus M_x \subseteq U_x \setminus M_x \subset U_x \triangle M_x = A_x$ meager. \lightning.
\end{dukazin}

\begin{priklady}
	$\exists A \subset [0, 1]^2$, $A$ non-meager and there are no three points in $A$ on a straight line.

	\begin{dukazin}
		$\{F_α, α < 2^ω\}$ meager $F_ς$ sets. We will construct $\{x_α, α < 2^ω\}$ such that $x_α \notin F_α$ and there are no 3 points on the same line. By induction: 1) $α = 0: x_0 \in [0, 1]^2 \setminus F_0$.

		2) We already have $\{x_β, β < α\} \subset [0, 1]^2$, $α < 2^ω$ such that $\forall β < α: x_β \notin F_β$ and there are no 3 points on the same line. $©M := \{p \text{ line} | \exists β, γ < α: x_β ≠ x_γ \land x_β, x_γ \in p\}$. Clearly $\# < 2^ω$. From Kuratowski–Ulam: $\forall^* t \in [0, 1]$: $(F_α)_t$ is meager. We find $t \in [0, 1] \setminus ∏_1(\{x_β, β < α\})$ such that $(F_α)_t$ is meager.

		$\implies$ line $\{[l, y], y \in ®R\} \notin ©M \implies \forall p \in ©M: \#\{y \in [0, 1] | [t, y] \in p\} ≤ 1$. So $\exists y \in [0, 1]: [t, y] \notin \bigcup ©M \cup F_α$. $(F_α)_t$ is meager and $\# (\bigcup ©M) \cap \{[t, y], y \in ®R\} ≤ \# ©M < 2^ω$. Put $x_α := [t, y]$.
	\end{dukazin}
\end{priklady}

\section{Measurable selections}
\begin{definice}[Uniformization, selection]
	Let $X, Y$ sets, $C \subset X \times Y$ and $F: x \mapsto C_x$ is mapping from $X$ to $©P(Y)$. $U \subset C$ is uniformization of $C$ if $|U_x| = 1$ for $C_x ≠ \O$ ($x \in ∏_X(C)$), ($U$ is a graph of mapping $X \rightarrow Y$).

	Mapping $f: D_f \rightarrow Y$ ($D_f = ∏_x(C) = \{x \in X | F(x) ≠ \O\}$) is selection of $F$, if $f(x) \in F(x)$, $x \in D_F$.
\end{definice}

\begin{poznamka}[Kondo–Norikov]
	$X, Y$ Polish topological spaces, $C \in ∏_1^1(X \times Y)$. Then there exists $B \in ∏_1^1(X \times Y)$ unifomization of $C$.
\end{poznamka}

\begin{poznamka}
	The theorem above implies if $A \subset M \times \{0, 1\}$ ($M$ metric space) is cosouslin then there exists cosouslin uniformization.

	$A_i := ∏_x(A \cap M \times \{i\})$. $B_0 \cup B_1 = M \implies B_0 \times \{0\} \cup B_1 \{1\}$ is uniformization. $B_0 \subset A_0$, $B_1 \subset A_1$, $B_i \in ∏_1^1$, $B_0 \cap B_1 = \O$. Similarly, we can do reduction for countable collections.
\end{poznamka}

TODO!!! (Kondo–Norikov is not true in $Σ_1^1$.)

\begin{priklad}
	There exists $F \in ∏_1()$

	TODO!!!
\end{priklad}

TODO!!!

\end{document}
