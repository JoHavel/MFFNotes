\documentclass[12pt]{article}					% Začátek dokumentu
\usepackage{../../MFFStyle}					    % Import stylu



\begin{document}

% 02. 10. 2023
\begin{poznamka}
	At least 1 from (3-)4 homework (flexible deadlines -- last lecture).
\end{poznamka}

\begin{poznamka}
	In this lecture, there was also the revision of topology. (Topological space, topology, basis of topology, continuous map, quotient space, product topology, Hausdorff spaces).
\end{poznamka}

\begin{poznamka}
	World Homotopy comes from homós (= same, simiar) and topos (place).
\end{poznamka}

\begin{definice}[Homotopic functions]
	Given two topological spaces $X$ and $Y$ and two continuous functions $f, g: X \rightarrow Y$, we say that $f$ is homotopic to $g$ ($f \sim g$) if there is a $1$-parametric family $f_t: X \rightarrow Y$: $f_0 = f$, $f_1 = g$ and the map $F: [0, 1] \times X \rightarrow Y$ defined by $(t, x) \mapsto f_t(x)$ is continuous.
\end{definice}

\begin{definice}[Homotopy equivalent spaces]
	Given two topological spaces $X$ and $Y$ we say that $X$ and $Y$ are homotopy equivalent if there is a pair of continuous maps $(f, g)$ such that $f: X \rightarrow Y$ and $g: Y \rightarrow X$ and $X \overset{f}\rightarrow Y$ and $Y \overset{g}\rightarrow X$, $g ∘ f \sim \id_X$, $f ∘ g \sim \id_Y$.
\end{definice}

\begin{priklad}
	Given $®R$, $®R^2$ with the standard Euclidean topology and two maps $f: ®R \rightarrow ®R^2$, $x \mapsto f(x) = (x, x^3)$, $g: ®R \rightarrow ®R^2$, $x \mapsto g(x) = (x, e^x)$.

	Are $f$ and $g$ homotopic? (Show that by constructing homotopy.)

	\begin{reseni}
		$$ F(t, x) = (1 - t)(x, x^3) + t(x, e^x) = (x, (1-t)x^3 + te^x). $$
	\end{reseni}
\end{priklad}

\begin{priklad}
	Given three topological spaces $(X, τ_X), (Y, τ_Y), (Z, τ_Z)$ and two pairs of continuous maps $f_1, g_1: (X, τ_X) \rightarrow (Y, τ_Y)$ and $f_2, g_2: (Y, τ_Y) \rightarrow (Z, τ_Z)$. Assume that $f_1$ is homotopic to $g_1$ and $f_2$ is homotopic to $g_2$. Show that $f_2 ∘ f_1$ is homotopic to $g_2 ∘ g_1$.

	\begin{reseni}
		$$ F(t, x) = F_2(t, F_1(t, x)). $$
	\end{reseni}
\end{priklad}

\begin{priklad}
	Take $B^n := \{x, …, x_n | \sqrt{x_1^2 + … + x_n^2} ≤ 1\} \subseteq ®R^n$. And take a map $f: B^n \rightarrow B^n$: $f(x) = (0, …, 0) \in B^n$ for all $x \in B^n$. Shows that there is a homotopy from $\id$ to $f$.

	\begin{reseni}
		$$ F: [0, 1] \times B^n \rightarrow B^n, \qquad (t, x) \mapsto (1 - t)x. $$
	\end{reseni}
\end{priklad}

\begin{priklad}
	Take a 2-ball $B^2$. $B^2$ is homotopy equivalent to its center by previous problem, but it is not homeomorphic to $(0, 0)$.
\end{priklad}

% 09. 10. 2023

\begin{definice}[Deformation retraction]
	A deformation retraction of a topological space $X$ onto a subspace $A$ is a family of maps $f_t: X \rightarrow X$, $t \in [0, 1]$: $f_0 = \id_X$, $f_1(X) = A$ and $f_t|_A = \id_A$. And family $f_t$ is continuous in the following sense:
	$$ F: [0, 1] \times X \rightarrow X, (t, x) \rightarrow f_t(x), \text{ is continuous}. $$
\end{definice}

\begin{tvrzeni}
	Given a deformation retraction $f_t: X \rightarrow X$, there is a pair $(f, g): X\overset{f}\rightarrow A\overset{g}\rightarrow X: g ∘ f \sim \id_X, f ∘ g \sim \id_A$.

	\begin{poznamka}[Suggestion]
		$f = f_1$, $g = f_i \circ i_A$ ($A \overset{i_A}\hookrightarrow X$), tj. $f∘g: A \overset{i_A}\hookrightarrow X \overset{f_1}\rightarrow X \overset{f_1}\rightarrow X$, $a \mapsto a \mapsto a \mapsto a$ (or $A$) $\implies$ $f \circ g = \id_A$. $g∘f: X \overset{f_i}\rightarrow A \overset{i_A}\rightarrow X$ $\implies f_1(x) \sim \id_X$.
	\end{poznamka}
\end{tvrzeni}

\begin{definice}
	Given two topological spaces $X$ and $Y$ and a continuous map $f: X \rightarrow Y$, the mapping cylinder $M_f$ is defined to be the quotient space of $X \times [0, 1] \coprod Y$ and $\sim$: $(x, 1) \sim f(x)$. $M_f = X \times [0, 1]\coprod Y / \sim$.
\end{definice}

\begin{tvrzeni}
	Given $X$, $Y$ and $f$, $M_f$ deformation retracts to $Y$.

	\begin{dukazin}[/ Idea of proof]
		The way to construct $f_t = F(·, t): M_f \rightarrow M_f$ is to slide each point $(x, t)$ along the segment $\{x\}\times[0, 1]$ to $f(x)$:
		$$ F: (x, t) \mapsto f_t(x), \qquad \forall y \in Y: y = F(y, t) \mapsto \{f_1 = \id Y \rightarrow Y\} $$
		In your HW you will check that $F(x, t)$ is continuous.
	\end{dukazin}
\end{tvrzeni}

\begin{definice}[Cell complex (CW complex)]
	\begin{poznamkain}
		Cell complex (CW complex) is a topological space with a nice decomposition into small pieces.
	\end{poznamkain}

	1. Start with a discrete set $X^0$, whose points are called $0$-cells.

	2. We form the $n$-skeleton $X^n$ from $X^{n - 1}$ by attaching cells $e_α^n = I^n = [0, 1]^n$. By the attachment we mean ($e_a^n = B_α^n, \partial e_a^n = S_α^n$) $φ_α: \partial e_α^n \rightarrow X^{n-1}$. Hence we can view $X^n = X^{n-1} \coprod \coprod B_α^n / \sim$, where $x \sim φ_α(x)$ for $x \in \partial \partial B_α^n$.

	3. We can either stop this inductive process at a certain finite steps or take an infinite number of steps. In the first case $X = X^n$ for some $n$, in the second one $X = \bigcup_{n \in ®N_0} X^n$ with the weak topology ($A \subset X$ is open $\leftrightarrow$ $A \cap X^n$ is open for all $n$).

	\begin{prikladyin}
		Example of $1$-skeleton is graph.
	\end{prikladyin}
\end{definice}

\begin{definice}
	Given a cell complex $X$. Each cell $e_α^n$ has a characteristic map $Φ_α: e_α^n = B_α^n \rightarrow X$ which extends the attaching map $φ_α: \partial B_α^n \rightarrow X^n$, it is homeomorphism from the interior of $B_α^n$ onto $e_α^n$. Namely
	$$ B_α^n \hookrightarrow X^{n-1} \coprod \coprod_β B_β^n \overset{quotient}\rightarrow X^n \rightarrow X, \qquad B_α^n \rightarrow X $$
\end{definice}

\begin{definice}
	A subcomplex of CW complex is a closed subspace $A \subset X$ that is a union of cells with the corresponding attachments.
\end{definice}

\begin{priklad}
	Construct two different CW structures on $S^2$.
	
	\begin{reseni}
		$S^2 = e^0 \cup e^2$, $S^2 = e^0 \cup e^1 \cup \{e^2_1, e^2_2\}$. (See practicals.)
	\end{reseni}
\end{priklad}

\begin{priklad}
	We define $®RP^n$ to be the quotient of $S^n / \sim$, where $V \sim $ the antipodal point to $V$. TODO?
\end{priklad}

\begin{definice}
	Consider a pair $(X, A)$ where $X$ is a CW complex and $A$ is subcomplex. Then we define the quotient complex $X / A$ to be the CW complex with the structure: There are all the cells of $X \setminus A$ with the corresponding attaching maps, and there is a extra $0$-cell which is $A$ in $X \setminus A$. For a cell $e_α^n$ of $X \setminus A$ attached by $φ_α: S^{n-1} \rightarrow X^{n - 1}$, the attaching map in the corresponding cell in $X \setminus A$ is the composition $S^{n-1} \rightarrow X^{n-1} \rightarrow X^{n-1} / A^{n-1}$.
\end{definice}

\begin{priklad}
	Show that $S^n = e^0 \cup e^n$ is $B^n / S^{n-1} = TODO / e^0 \cup e^{n-1}$.
\end{priklad}

% 16. 10. 2023

TODO!!!

% 23. 10. 2023

\begin{tvrzeni}
	There is an isomorphism $π_1(X, x_1) \rightarrow π_1(X, x_0)$ for $x_0$ and $x_1$ in the same path connected component.

	\begin{dukazin}
		Since $x_0$, $x_1$ are in one path connected component $\tilde X$, $\exists$ path $h: [0, 1] \rightarrow X$: $h$ is in $\tilde X$ and $h(0) = x_0, h(1) = x_1$. $\overline{h}(s) := h^{-1}(s) := h(1 - s)$, $s \in [0, 1]$.

		To each loop $f$ based at $x_1$ we associate a loop $h·f·h^{-1}$. $h·f·h^{-1}$ is based at $x_0$. $β_h : π_1(x, x_1) \rightarrow π_1(x, x_0), [f] \mapsto [h·f·h^{-1}]$. We claim, that $β_h$ is an isomorphism. „$β_h$ is homomorhism“:
		$$ β_h([f·h]) = [hfgh^{-1}] = [hfh^{-1} hgh^{-1}] = [hfh^{-1}]·[hgh^{-1}] = β_h([f])·β_h([g]). $$
		„$β_h$ is isomorphism“: „the inverse of $β_h$ is $β_{h^{-1}}$“ (which is homomorphism too by the argument we used for $β_h$):
		$$ β_{h^{-1}}(β_h([f])) = β_{h^{-1}}([hfh^{-1}]) = [h^{-1}hfh^{-1}h] = [f]. $$
	\end{dukazin}
\end{tvrzeni}

\begin{veta}[Fundamental group of $S^1$]
	$S^1$ is path connected, thus $π_1(S^1, x_0) = π_1(S^1)$.
	
	$π_1(S^1) \simeq ®Z$.

	\begin{dukazin}
		We claim that $π_1(S^1) \simeq \<[ω]\>$, where $ω: [0, 1] \rightarrow S^1$, $s \mapsto (\cos(2πs), \sin(2πs)) \in ®R^2$, $s \in [0, 1]$. $ω_n(s) := (\cos(2π n s), \sin(2π n s)) \sim ω^n$, so $[ω]^n = [ω_n]$.

		Now our theorem is equivalent to the statement that every loop in $S^1$ based at $(1, 0)$ is homotopic to the unique $ω_n$. We use the following two facts:

		Fact 1: For every path $f: I \rightarrow X$ starting at $x_0 \in X$ and each $\tilde x_0 \in p^{-1}(x_0)$ there is a unique lift $\tilde f: I \rightarrow \tilde X$ starting at $x_0$.

		Fact 2: For each homotopy $f_x: I \rightarrow X$ of paths starting at $x_0$ and each $\tilde x_0 \in p^{-1}(x_0)$ $\exists$ unique lifted homotopy $\tilde f_t: I \rightarrow \tilde X$ of paths starting at $\tilde x_0$.

		$p$ that we need: $p: ®R \rightarrow S^1$; $p(s) = (\cos 2π s, \sin 2π s)$. If we define $\tilde ω_n(s) = n·s$. We will apply Facts 1 and 2 to $p: ®R \rightarrow S^1$, $\tilde ω_n$: Given $f: [0, 1] \rightarrow S^1$ based at $(0, 1)$ representing some element of $π_1(S^1)$. We take $\tilde f$. Since $p \tilde f(1) = f(1) = (1, 0)$ (and $p^{-1}(1) \in ®Z$), we can argument that if $\tilde f$ ends at $u$ (i.e. $\tilde f(1) = f$), it is homotopoc to $\tilde ω_n$ by the homotopy $\tilde F = (1 - t)\tilde f + t\tilde ω_n$.

		From fact 1 exists $\tilde f$ starting at 0 and ending at $p^{-1}(1) \in ®Z$.

		Theorem: Exists homotopy $\tilde F$ from $\tilde ω_k$ to $\tilde f$ denoted by $(*)$.

		So we define homotopy $F$ from $ω_n$ to $f$ by $F = p∘\tilde F$, homotopy from $ω_n$ to $f$. Since $[ω_n] = n·[ω]$, $π_1(S^1) \simeq ®Z$.

		Now we would like to show that $[f]$ is uniformly determined. Assume that $f \sim ω_n$ and $f \sim ω_m$, then using Facts 1 and 2 we have $[ω_n] = [ω_m]$ which lends to contradiction since they have different endpoints on ®R.
	\end{dukazin}
\end{veta}

\begin{definice}
	Given a topological space $X$, a covering space of $X$ consists of a topological space $\tilde X$ and a continuous map $p: \tilde X \rightarrow X$ satisfying that $\forall x \in X$ $\exists$ open neighbourhood $U$ of $x$ in $X$ such that $p^{-1}(U)$ is a disjoint union of open subsets $U_α$ each of which is homeomorphically mapped to $U$.
\end{definice}

\begin{definice}
	Given a map $[0, 1] \overset{f}\rightarrow X$ and $p: \tilde X \rightarrow X$ we say that $\tilde f:[0, 1] \rightarrow \tilde X$ is a lift of $f$ if $p ∘ \tilde f = f$.

	The same construction can be defined for homotopy.
\end{definice}

\begin{tvrzeni}[*]
	Given a map $F: Y \times [0, 1] \rightarrow X$ and a map $\tilde F: Y \times \{¦o\} \rightarrow \tilde X$, where $p: \tilde X \rightarrow X$ is a covering space, and $\tilde F$ lifts $F|_{Y \times \{¦o\}}$; there restricting to $\tilde F$ on $Y \times \{¦o\}$.

	\begin{poznamka}[Corollary: Fact 1 and Fact 2 from the previous proof]
		Fact 1 is free, it comes when $Y = \{\text{point}\}$, Fact 2 also follows.
	\end{poznamka}
\end{tvrzeni}

% 23. 10. 2023

\begin{priklad}
	We say that a topological (path-connected) space is simply connected $\Leftrightarrow$ $π_1(X) = \{e\}$. Examples of simply connected topological spaces: $®R, ®R^2, …$. $S^1$ is not simply connected.
\end{priklad}

\begin{priklad}
	Given $X, Y$ path-connected and $x_0 \in X$, $y_0 \in Y$. Show that $π_1(X\times Y, (x_0, y_0)) \simeq π_1(X, x_0) \times π_1(Y, y_0)$.

	\begin{reseni}
		Product topology is defined to be such that a map $f: Z \rightarrow X \times Y$ is continuous $\Leftrightarrow$ ($p_x: X\times Y \rightarrow X$, $p_y: X \times Y \rightarrow Y$) $p_x ∘ f$ and $p_y ∘ f$ are continuous.

		A loop $γ: [0, 1] \rightarrow X \times Y$ based at $(x_0, y_0)$ splits at two loops $γ_1: [0, 1] \rightarrow X$, $γ_2: [0, 1] \rightarrow Y$. The same holds for homotopy, i.e. $F$ from $γ$ to $\tilde γ$ splits into $(F_1, F_2)$, where $F_1$ is a homotopy on $X$ from $γ_1$ to $\tilde γ_1$ and $F_2$ is a homotopy on $Y$ from $γ_2$ to $\tilde γ_2$.
	\end{reseni}
\end{priklad}

\begin{dusledek}
	$π_1(T^n) := π_1(S^1 \times S^1 \times … \times S^1) = ®Z^n$.
\end{dusledek}

\begin{priklad}
	Show that TODO!!! is a covering space for $S^1 V S^1$.
\end{priklad}

TODO!!!

% 30. 10. 2023

\begin{dukaz}[Proposition *]
	To prove our proposition we need to construct $\tilde F: N \times I \rightarrow \tilde X$, where $N$ is an open neighbourhood in $Y$ of a given point $y_0 \in Y$. Since $F$ is continuous, every point $(y_0, t) \in Y \times I$ has a product neighbourhood $N_t \times (a_t, b_t)$ such that $F(N_t \times (a_t, b_t))$ is contained in an evenly covered neighbourhood of $F(y_0, t)$.

	By compactness of $\{y_0\} \times I$, finitely many such products $N_t \times (a_t, b_t)$ cover $\{y_0\} \times I$. This implies that we can choose a single neighbourhood $N$ of $y_0$, and a partition of $[0, 1]$ $0 = t_0 < t_1 < t_2 < … < t_n = 1$ such that $F|_{N \times [t_i, t_{i+1}]}$ is contained in an evenly covered neighbourhood $U_i$.

	Assume inductively that $\tilde F$ has been constructed for $N \times [0, t_i]$ starting at a given $\tilde F$ on $N \times \{0\}$. We have that $F(N \times [t_i, t_{i+1}]) \subset U_i$, so since $U_i$ is evenly covered, there is an open set $\tilde U_i \subset \tilde X$ projecting homeomorphically to $U_i$ via $p$ and $\tilde F((y_0, t_i)) \in \tilde U_i$. After replacing $N$ by a smaller neighbourhood of $y_0$. (We replace $N \times \{t_i\}$ with the intersection with ?) we may assume that $\tilde F(N \times \{t_0\}) \in \tilde U_i$. Now we define $\tilde F$ on $N \times [t_i, t_{i+1}]$ to be the composition of $F$ with $p^{-1}: U_i \rightarrow \tilde U_i$. After a finite number of steps, we eventually get a lift $\tilde F: N \times I \rightarrow \tilde X$ for $N$ (some neighbourhood of $y_0$).

	Next we show the uniqueness for $Y = \{\text{point}\}$. In this case we ?
	Suppose there are two lifts $\tilde F: I \rightarrow \tilde X$, $\tilde F': I \rightarrow \tilde X$. As before we choose a partition $0 = t_0 < t_1 < … < t_n = 1$ of $[0, 1]$ so that $\forall i: F([t_i, t_{i+1}])$ is contained in some evenly covered neighbourhood $U_i$. Assume inductively that $\tilde F = \tilde F'$ on $[0, t_i]$. Since $[t_i, t_{i+1}]$ is connected, co is $\tilde F([t_i, t_{i+1}])$, which must therefore lie in one of the disjoint open sets $\tilde U_i$ projecting homeomorphically to $U_i$. By the same token, $\tilde F'([t_i, t_{i+1}])$ lies in a single $\tilde U_i$, in fact in the same containing $\tilde F([t_i, t_{i+1}])$ (by the assumption of induction). Since $p$ is injective on $\tilde U_i$ and $p \tilde F = p \tilde F'$, it follows that $\tilde F = \tilde F'$ on $[t_i, t_{i+1}]$ and the induction step follows.

	The last step of the prove is to observe that since $\tilde F$, $\tilde F'$ are constructed on the sets of form $N \times I$ and are unique when we restrict to each segment $\{y\} \times I$, they must agree whenever two such sets $N \times I$ overlap, so we get in fact a well-defined lift $\tilde F$ on $Y \times I$. This $\tilde F$ is continuous since it is continuous on each segment $\{y\} \times I$.
\end{dukaz}

\begin{poznamka}
	We would like to see $π$, as a functor $π_1: Top \rightarrow Grp$. In order for $π_1$ to be a functor, we want for $φ: (X, x_0) \overset{\text{cont.}}\rightarrow (Y, y_0)$ associate $φ_*: π_1(X, x_0) \rightarrow π_1(Y, y_0)$. How to get $φ_*$? Given a loop $γ$ on $X$ based at $x_0$. We have a loop $φ ∘ γ$ on $Y$ based at $y_0$. $φ_*([γ]) := [φ ∘ γ]$. Is $φ_*$ a homomorphism? $φ_*([γ; γ_2]) = [φ ∘ (γ_1·γ_2)] = [φ ∘ γ_1 · φ · γ_2] = [φ ∘ γ_1]·[φ ∘ γ_2]$. Hence $φ_*$ is a homomorphism.

	$$ \forall (X, x_0) \in Fb(Top) \exists φ=1 \in Hom((X, x_0), (X, x_0)). $$

	We want $π_1(1) = \id_{π_1(X, x_0)}$.  This because of the definition of $π_1(1)$, which maps $[γ] \rightarrow [γ]$ $\implies$ $π_1(1): [γ] \mapsto [γ]$, so it is $\id_{π_1(X, x_0)}$.

	In order for $π_1$ to be a functor we also need: given $(X, x_0) \overset{φ}\rightarrow (Y, y_0) \overset{ψ}\rightarrow (Z, z_0)$ then $i(ψ ∘ φ)_* = π_1(ψ ∘ φ) = π_1(ψ) ∘ π_1(φ) = (ψ)_* · (φ)_*$. This is because
	$$ π_1(ψ ∘ φ)([γ]) = (ψ ∘ φ)_* ([γ]) = [ψ ∘ φ ∘ γ] = ψ_*([φ ∘ γ]) = π_1(ψ)(φ_*([γ])) = π_1(ψ) ∘ π_1(φ)([γ]). $$
	Hence it holds and $π_1$ is functor $TOP \rightarrow GRP$.
\end{poznamka}

\begin{tvrzeni}
	Given a topological space $X$. If $X = \bigcup A_α$, where each of $A_α$ is a path connected subspace of $X$ and $A_α \cap A_β$ is path connected $\forall α, β$, then each loop in $X$ based at $x_0$ can be decomposed as a product of loops each of which is in some $A_α$.

	\begin{dukazin}
		Given $f: I \rightarrow X$ with the basepoint $x_0$, we claim that there is a partition $0 = s_0 < s_1 < … < s_m = 1$ of $I$ such that $[s_{i-1}, s_i]$ is mapped by $f$ to a single $A_α$ (that by call $A_i$). Since $f$ is continuous, each $s \in I$ has an open neighbourhood $V_i$ in $I$ mapped by $f$ to some $A_α$. We may in fact take $V_s$ to be an interval whose closure is mapped to a single $A_α$. By compactness of $I$, we see that a finite number of such intervals cover $I$. The endpoints of these intervals form a partition of $I$: $0 = s_0 < s_1 < … < s_m = 1$. Again $A_i$ we call $A_α: f([s_{i-1}, s_i]) \subset A_α$. Let $f_i$ be denoted as $f|_{[s_{i-1}, s_i]}$. Then $f = f_i, …, f_m$, where $f_i$ is a path in $A_i$.

		Since $f([s_{i-1}, s_i]) \subset A_i$ $\land$ $f([s_i, s_{i+1}]) \subset A_{i+1}$ $\implies$ $f(s_i) \in A_i \cap A_{i-1}$. Since $A_i \cap A_{i+1}$ is path connected, we can choose path $g_i \subset A_i \cap A_{i+1}$, $g_i$ starts at $x_0$ and ends at $f(s_i)$. Hence $[f][f_1 g_1^{-1}][g_1 f_2 g_2^{-1}]…[g_{n-1} f_n g_n^{-1}]$, loops in $A_1, A_2, …, A_n$.
	\end{dukazin}
\end{tvrzeni}

% 30. 10. 2023

TODO?

% 06. 11. 2023

TODO!!!

% 13. 11. 2023

\begin{tvrzeni}[*, should be somewhere above]
	If $X = \bigcup_{α \in I} A_α$, $x_0 \in \bigcap_{α \in I} A_α$, $A_α \cap A_β$ is path connected for all $α$, $β$. Then for each loop $γ$ in $X$ based at $x_0$, there is a sequence of loops $γ_{α_1}, … γ_{α_k}: [γ] = [γ_{α_1}]·…·[γ_{α_k}]$, where $γ_{α_i}$ is a loop in $A_{α_i}$.
\end{tvrzeni}

\begin{veta}[Van Kampen]
	Given a topological space $X$ such that $X = \bigcup A_α$, where $A_α$'s are path connected subspaces of $X$, $x_0 \in \bigcap A_α ≠ \O$. Besides that we assume $A_α \cap A_β$ is connected $\forall α, β$. Then $\exists Φ: *_{α \in I} π_1(A_α, x_0) \rightarrow π_1(X, x_0)$, which is a surjective homomorphism. If in addition $\forall α, β, γ: A_α \cap A_β \cap A_γ$ is path connected, then $\Ker Φ = \<i_{αβ}(ω) i_{βα}(ω)^{-1}\>$.
	$$ *_{α \in I} π_1(A_α, x_0) / <i_{αβ}(ω)i_{βα}(ω)^{-1}> \simeq π_1(X, x_0). $$
\end{veta}

	\begin{dukaz}
		The proof uses Proposition *. Same terminology: Given a loop $f$ in $X$ based at $x_0$, by a factorization of $[f] \in π_1(X, x_0)$ we mean a formal product $[f_1]·…·[f_k]$, where each $f_i$ is a loop in some $A_α$ based at $x_0$ and $[f_i] \in π_1(A_α, x_0)$, and where $f$ is homotopic to $f_1·…·f_k$. (Factorization is a presentation from Proposition *. Factorization can be seen as a word in $*_{α \in I} π_1(A_α, x_0)$, but not necessary reduced word.)

		We construct $Φ$ by sending $[f]$ to $[f_1]·…·[f_k]$ coming from Proposition * with the potential reduction. We will be concerned with the uniqueness of such factorizations. We define an equivalence relation on factorizations. We say that two sequences are equivalent iff they are related by a sequence of moves: combine adjacent terms: $[f_i]·[f_j] = [f_if_j]$ if $[f_i], [f_j] \in π_1(A_α, x_0)$ for some $α$; and regard $[f_i] \in π_1(A_α)$ as $[f_i] \in π_1(A_β)$ if $f_i$ is a loop in $A_β$; and we do not write constant loops in the decomposition.

		This equivalence relation mimics, what is necessary for words in $* π_1(A_α, x_0)$. We modify this equivalence relation by substituting second condition (regard …) by: For $f_i$, $f_j$: $[f_i] \in π_1(A_α, x_0)$, $[f_j] \in π_1(A_β, x_0)$, we view $[f_i]$, $[f_j]$ as elements of $π_1(A_α \cap A_β, x_0)$. So first and third condition do not change the elements in $*_{α \in I} π_1(A_α, x_0)$ and the new second one is such that it does not change elements $*_{α \in I} π_1(A_α, x_0) / N$.

		So equivalent factorizations supposed to give the same element $*_{α \in I} π_1(A_α, x_0)$. Hence we can show that any two factorizations of $f$ are equivalent, this will imply $*_{α \in I} π_1(A_α, x_0) / N \rightarrow π_1(X, x_0)$ is injective, and therefore $*_{α \in I} π_1(A_α, x_0) / N \rightarrow *_{α \in I} π_1(A_α, x_0)$ is trivial, and this map is an isomorphism.

		„Equivalence of two factorizations“: We take two factorizations $[f_1]·…·[f_k]$ and $[f_1']·…·[f_l']$ of $[f] \in π_1(X, x_0)$. We would like to show that $[f_1]·…·[f_k] \sim [f_1']·…·[f_l']$. Since they represent $[f]$, they are homotopic. Let $F: I \times I \rightarrow X$ is a homotopy from $f_1·…·f_k$ to $f_1'·…·f_l'$. Then we say that $\exists$ decompositions $0 = s_0 < s_1 < … < s_m = 1$, $0 = t_0 < t_1 < … < t_n = 1$ such that each $R_{ij} := [s_{i-1}, s_i] \times [t_{j-1}, t_j]$ is mapped by $F$ to a single $A_α$, which we label by $A_{ij}$. These partitions can be obtained from $I \times I$ by finitely many rectangles $[a, b] \times [c, d]$ mapping to a single $A_α$ by compactness of $I \times I$.

		With this type of description, we can assume that $R_{ij}$'s are such that they are to the subdivision giving homotopies between $f_i$'s and $f_j$'s.

		Since $F$ maps a neighbourhood of $R_{ij}$ into $A_{ij}$, we may perturb the vertical sides of $R_{ij}$ so that each point of $I \times I$ lies in at most three $R_{ij}$'s. After this modification we do the relabelling by the following scheme: label row by row starting from the first one, and label from left to right.

		If $γ$ is path in $I \times I$ from the left edge (i.e. $\{0\} \times [0, 1]$) to the right edge (i.e. $\{1\} \times [0, 1]$), then $F|_γ$ is a loop from $x_0$ to $x_0$. We write $γ_r$ to be the path (through edges of rectangles) separating the first $r$ rectangles $R_1, …, R_r$ from the remaining rectangles. $γ_0$ is bottom edge, $γ_m$ is the top edge.

		Let us call the corners of $R_i$'s vertices. So for each vertex $v$ with $F(v) ≠ x_0$, there is a path $g_v$ from $x_0$ to $F(v)$ that lies in the intersection of the two or three $A_{ij}$'s corresponding to $R_i$'s containing $v$. (This is by the choice of $R_i$'s and the fact that $A_α$'s are path connected and contains $x_0$ and $F(v)$. And by path connectedness of disjoint of $A$'s.)

		Then we obtain a factorization $[F|_{γ_r}]$ by inserting the appropriate path connection $x_0$ with $F|_{\text{vertices corresponds of $R_i$'s}}$ then we can say that $[F|_{γ_r}]$ has a decomposition that depends on the choice of $A_{ij}$ that corresponds to the edge of $R_i$ on which the component of the decomposition lies.

		$[F|_{γ_r}]$ has a decomposition by inserting $g_v^{-1} g_v$. Different choices of $A_{ij}$'s will change the factorization of $[F|_{γ_r}]$. (Our equivalence relation $\sim$ is designed to make the choice of $A_{ij}$'s for such paths irrelevant. I.e. the different factorizations coming from different choices are equivalent.)

		Factorizations for two consecutive paths $γ_r$ and $γ_{r+1}$ are equivalent since pushing along $R_{r+1}$ from $γ_r$ to $γ_{r+1}$ changing $F|_{γ_r}$ to $F|_{γ_{r+1}}$ by a homotopy within $A_{ij}$ corresponding to $R_{r+1}$, so we can choose this $A_{ij}$ for all segments of $γ_r$.

		We can arrange that the factorization associated to $γ_0$ is equivalent to $[f_1]·…·[f_k]$ by choosing $g_r$'s for each vertex $v$ along the lower edge of $I \times I$ to lie not just in the two $A_{ij}$'s corresponding to $R_s$'s containing $v$, but also to lie in the $A_α$ for $f_i$ containing $v$ in its domain. In the case when $v$ is the common point of two domains for two consecutive $f_i$ and $f_{i+1}$ we have $F(v) = x_0$, so there is no need to choose $g_v$'s for such $v$'s. In this fashion we can assume that the factorization for $[f_1]·…·[f_k]$ and $[f_1']·…·[f_l']$ are equivalent.
	\end{dukaz}

% 20. 11. 2023

TODO? (Examples of covering space)

\begin{tvrzeni}
	Given a covering space $p: \tilde X \rightarrow X$, a homotopy $f_t: Y \rightarrow X$ and a map $\tilde f_ς: Y \rightarrow \tilde X$ lifting $f_0$, then there is a unique homotopy $\tilde f_t: Y \rightarrow \tilde X$ of $\tilde f_0$ that lifts $f_t$.

	\begin{poznamka}
		This statement in a more general form has already appeared on practicals.
	\end{poznamka}

	\begin{poznamka}
		One can use this proposition in two ways: We have a lifting property for paths which says that $\forall$path $f: [0, 1] = I \rightarrow X$and each lift $\tilde x_0$ of $f(0) = x_0 \in X$ there is a unique $\tilde f: I \rightarrow \tilde X$ lifting $f$ starting at $\tilde x_0$.

		If $Y = I$, we get that every homotopy $f_t$ of a path $f_0$ in $X$ lifts to a homotopy $\tilde f_t$ of each lift $\tilde f_0$ of $f_0$.

		As a corollary of Proposition (*) we can write:
	\end{poznamka}
\end{tvrzeni}

\begin{dusledek}
	The map $p_*: π_1(\tilde X, \tilde x_0) \rightarrow π_1(X, x_0)$ induced by the covering space map $p: \tilde X \rightarrow X$ is injective. The image subgroup $p_*(π_1(\tilde X, \tilde x_0))$ in $π_1(X, x_0)$ consists of the homotopy classes of loops in $X$ based at $x_0$ whose lifts to $\tilde X$ starting at $\tilde x_0$ are loops.
	
	\begin{dukazin}[Of corollary]
		An element of the kernel of $p_*$ is represented by a loop $\tilde f_0: I \rightarrow \tilde X$ with a homotopy $f_t: I \rightarrow X$ of $f_0 = p·\tilde f_0$ to the trivial loop. By ? the applications of proposition (*) there is a lifted homotopy of loops $\tilde f_t$ starting at $f_0$ and ending with a constant loop. Hence $[\tilde f_0] = 0$ in $π_1(\tilde X, \tilde x_0)$ and $p_*$ is injective.
	\end{dukazin}
\end{dusledek}

\begin{tvrzeni}
	The number of sheets of a covering space $p: \tilde X \rightarrow X, \tilde x \mapsto x$ with $X$ and $\tilde X$ path connected equals the index of $p_*(π_1(\tilde X, \tilde x_0))$ in $π_1(X, x_0)$.

	\begin{dukazin}
		For a loop $g$ in $X$ based at $x_0$ let $\tilde g$ be its lift in $\tilde X$ based at $\tilde x_0$. A product $h·g$, $[h] \in H = p_*(π_1(\tilde X, \tilde x_0))$ has a lift $\tilde h·\tilde g$ ending at the same point as $\tilde g$ (since $\tilde h$ is a loop). Hence we can define a map $Φ$ from cosets $H[g]$ to $p^{-1}(x_0)$ implies that $Φ$ is surjective since $\tilde x_0$ can be lowed by a path $\tilde g$ to an arbitrary point in $p^{-1}(x_0)$ and $\tilde g$ projects to a loop $g$ at $x_0$.

		To show that $Φ$ is injective, observe that $Φ(H[g_1]) = Φ(H[g_2])$ $\implies$ $g_1·g_2^{-1}$ lifts to a loop in $\tilde X$ based at $\tilde x_0$, so $[g_1][g_2]^{-1} \in H$ $\implies$ $H[g_1] = H[g_2]$ $\implies$ injectivity + surjectivity $\implies$ bijectivity of $Φ$.
	\end{dukazin}
\end{tvrzeni}

% 20. 11. 2023

TODO? (Problems)

\end{document}
