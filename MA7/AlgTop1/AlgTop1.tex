\documentclass[12pt]{article}					% Začátek dokumentu
\usepackage{../../MFFStyle}					    % Import stylu



\begin{document}

% 02. 10. 2023
\begin{poznamka}
	At least 1 from (3-)4 homework (flexible deadlines -- last lecture).
\end{poznamka}

\begin{poznamka}
	In this lecture, there was also the revision of topology. (Topological space, topology, basis of topology, continuous map, quotient space, product topology, Hausdorff spaces).
\end{poznamka}

\begin{poznamka}
	World Homotopy comes from homós (= same, simiar) and topos (place).
\end{poznamka}

\begin{definice}[Homotopic functions]
	Given two topological spaces $X$ and $Y$ and two continuous functions $f, g: X \rightarrow Y$, we say that $f$ is homotopic to $g$ ($f \sim g$) if there is a $1$-parametric family $f_t: X \rightarrow Y$: $f_0 = f$, $f_1 = g$ and the map $F: [0, 1] \times X \rightarrow Y$ defined by $(t, x) \mapsto f_t(x)$ is continuous.
\end{definice}

\begin{definice}[Homotopy equivalent spaces]
	Given two topological spaces $X$ and $Y$ we say that $X$ and $Y$ are homotopy equivalent if there is a pair of continuous maps $(f, g)$ such that $f: X \rightarrow Y$ and $g: Y \rightarrow X$ and $X \overset{f}\rightarrow Y$ and $Y \overset{g}\rightarrow X$, $g ∘ f \sim \id_X$, $f ∘ g \sim \id_Y$.
\end{definice}

\begin{priklad}
	Given $®R$, $®R^2$ with the standard Euclidean topology and two maps $f: ®R \rightarrow ®R^2$, $x \mapsto f(x) = (x, x^3)$, $g: ®R \rightarrow ®R^2$, $x \mapsto g(x) = (x, e^x)$.

	Are $f$ and $g$ homotopic? (Show that by constructing homotopy.)

	\begin{reseni}
		$$ F(t, x) = (1 - t)(x, x^3) + t(x, e^x) = (x, (1-t)x^3 + te^x). $$
	\end{reseni}
\end{priklad}

\begin{priklad}
	Given three topological spaces $(X, τ_X), (Y, τ_Y), (Z, τ_Z)$ and two pairs of continuous maps $f_1, g_1: (X, τ_X) \rightarrow (Y, τ_Y)$ and $f_2, g_2: (Y, τ_Y) \rightarrow (Z, τ_Z)$. Assume that $f_1$ is homotopic to $g_1$ and $f_2$ is homotopic to $g_2$. Show that $f_2 ∘ f_1$ is homotopic to $g_2 ∘ g_1$.

	\begin{reseni}
		$$ F(t, x) = F_2(t, F_1(t, x)). $$
	\end{reseni}
\end{priklad}

\begin{priklad}
	Take $B^n := \{x, …, x_n | \sqrt{x_1^2 + … + x_n^2} ≤ 1\} \subseteq ®R^n$. And take a map $f: B^n \rightarrow B^n$: $f(x) = (0, …, 0) \in B^n$ for all $x \in B^n$. Shows that there is a homotopy from $\id$ to $f$.

	\begin{reseni}
		$$ F: [0, 1] \times B^n \rightarrow B^n, \qquad (t, x) \mapsto (1 - t)x. $$
	\end{reseni}
\end{priklad}

\begin{priklad}
	Take a 2-ball $B^2$. $B^2$ is homotopy equivalent to its center by previous problem, but it is not homeomorphic to $(0, 0)$.
\end{priklad}

% 09. 10. 2023

\begin{definice}[Deformation retraction]
	A deformation retraction of a topological space $X$ onto a subspace $A$ is a family of maps $f_t: X \rightarrow X$, $t \in [0, 1]$: $f_0 = \id_X$, $f_1(X) = A$ and $f_t|_A = \id_A$. And family $f_t$ is continuous in the following sense:
	$$ F: [0, 1] \times X \rightarrow X, (t, x) \rightarrow f_t(x), \text{ is continuous}. $$
\end{definice}

\begin{tvrzeni}
	Given a deformation retraction $f_t: X \rightarrow X$, there is a pair $(f, g): X\overset{f}\rightarrow A\overset{g}\rightarrow X: g ∘ f \sim \id_X, f ∘ g \sim \id_A$.

	\begin{poznamka}[Suggestion]
		$f = f_1$, $g = f_i \circ i_A$ ($A \overset{i_A}\hookrightarrow X$), tj. $f∘g: A \overset{i_A}\hookrightarrow X \overset{f_1}\rightarrow X \overset{f_1}\rightarrow X$, $a \mapsto a \mapsto a \mapsto a$ (or $A$) $\implies$ $f \circ g = \id_A$. $g∘f: X \overset{f_i}\rightarrow A \overset{i_A}\rightarrow X$ $\implies f_1(x) \sim \id_X$.
	\end{poznamka}
\end{tvrzeni}

\begin{definice}
	Given two topological spaces $X$ and $Y$ and a continuous map $f: X \rightarrow Y$, the mapping cylinder $M_f$ is defined to be the quotient space of $X \times [0, 1] \coprod Y$ and $\sim$: $(x, 1) \sim f(x)$. $M_f = X \times [0, 1]\times Y / \sim$.
\end{definice}

\begin{tvrzeni}
	Given $X$, $Y$ and $f$, $M_f$ deformation retracts to $Y$.

	\begin{dukazin}[/ Idea of proof]
		The way to construct $f_t = F(·, t): M_f \rightarrow M_f$ is to slide each point $(x, t)$ along the segment $\{x\}\times[0, 1]$ to $f(x)$:
		$$ F: (x, t) \mapsto f(x), \qquad \forall y \in Y: y = F \mapsto \{f_1 = \id Y \rightarrow Y\} $$
		In your HW you will check that $F(x, t)$ is continuous.
	\end{dukazin}
\end{tvrzeni}

\begin{definice}[Cell complex (CW complex)]
	\begin{poznamkain}
		Cell complex (CW complex) is a topological space with a nice decomposition into small pieces.
	\end{poznamkain}

	1. Start with a discrete set $X^0$, whose points are called $0$-cells.

	2. We form the $n$-skeleton $X^n$ from $X^{n - 1}$ by attaching cells $e_α^n = I^n = [0, 1]^n$. By the attachment we mean ($e_a^n = B_α^n, \partial e_a^n = S_α^n$) $φ_α: \partial e_α^n \rightarrow X^{n-1}$. Hence we can view $X^n = X^{n-1} \coprod \coprod B_α^n / \sim$, where $x \sim φ_α(x)$ for $x \in \partial \partial B_α^n$.

	3. We can either stop this inductive process at a certain finite steps or take an infinite number of steps. In the first case $X = X^n$ for some $n$, in the second one $X = \bigcup_{n \in ®N_0} X^n$ with the weak topology ($A \subset X$ is open $\leftrightarrow$ $A \cap X^n$ is open for all $n$).

	\begin{prikladyin}
		Example of $1$-skeleton is graph.
	\end{prikladyin}
\end{definice}

\begin{definice}
	Given a cell complex $X$. Each cell $e_α^n$ has a characteristic map $Φ_α: e_α^n = B_α^n \rightarrow X$ which extends the attaching map $φ_α: \partial B_α^n \rightarrow X^n$, it is homeomorphism from the interior of $B_α^n$ onto $e_α^n$. Namely
	$$ B_α^n \hookrightarrow X^{n-1} \coprod \coprod_β B_β^n \overset{quotient}\rightarrow X^n \rightarrow X, \qquad B_α^n \rightarrow X $$
\end{definice}

\begin{definice}
	A subcomplex of CW complex is a closed subspace $A \subset X$ that is a union of cells with the corresponding attachments.
\end{definice}

\begin{priklad}
	Construct two different CW structures on $S^2$.
	
	\begin{reseni}
		$S^2 = e^0 \cup e^2$, $S^2 = e^0 \cup e^1 \cup \{e^2_1, e^2_2\}$. (See practicals.)
	\end{reseni}
\end{priklad}

\begin{priklad}
	We define $®RP^n$ to be the quotient of $S^n / \sim$, where $V \sim $ the antipodal point to $V$. TODO?
\end{priklad}

\begin{definice}
	Consider a pair $(X, A)$ where $X$ is a CW complex and $A$ is subcomplex. Then we define the quotient complex $X / A$ to be the CW complex with the structure: There are all the cells of $X \setminus A$ with the corresponding attaching maps, and there is a extra $0$-cell which is $A$ in $X \setminus A$. For a cell $e_α^n$ of $X \setminus A$ attached by $φ_α: S^{n-1} \rightarrow X^{n - 1}$, the attaching map in the corresponding cell in $X \setminus A$ is the composition $S^{n-1} \rightarrow X^{n-1} \rightarrow X^{n-1} / A^{n-1}$.
\end{definice}

\begin{priklad}
	Show that $S^n = e^0 \cup e^n$ is $B^n / S^{n-1} = TODO / e^0 \cup e^{n-1}$.
\end{priklad}

% 16. 10. 2023

TODO!!!

% 23. 10. 2023

\begin{tvrzeni}
	There is an isomorphism $∏_1(X, x_1) \rightarrow ∏_1(X, x_0)$ for $x_0$ and $x_1$ in the same path connected component.

	\begin{dukazin}
		Since $x_0$, $x_1$ are in one path connected component $\tilde X$, $\exists$ path $h: [0, 1] \rightarrow X$: $h$ is in $\tilde X$ and $h(0) = x_0, h(1) = x_1$. $\overline{h}(s) := h^{-1}(s) := h(1 - s)$, $s \in [0, 1]$.

		To each loop $f$ based at $x_1$ we associate a loop $h∘f∘h^{-1}$. $h∘f∘h^{-1}$ is based at $x_0$. $β_h : ∏_1(x, x_1) \rightarrow ∏_1(x, x_0), [f] \mapsto [h∘f∘h^{-1}]$. We claim, that $β_h$ is an isomorphism. „$β_h$ is homomorhism“:
		$$ β_h([f·h]) = [hfgh^{-1}] = [hfh^{-1} hgh^{-1}] = [hfh^{-1}]·[hgh^{-1}] = β_h([f])·β_h([g]). $$
		„$β_h$ is isomorphism“: „the inverse of $β_h$ is $β_{h^{-1}}$“ (which is homomorphism too by the argument we used for $β_h$):
		$$ β_{h^{-1}}(β_h([f])) = β_{h^{-1}}([hfh^{-1}]) = [h^{-1}hfh^{-1}h] = [f]. $$
	\end{dukazin}
\end{tvrzeni}

\begin{veta}[Fundamental group of $S^1$]
	$S^1$ is path connected, thus $∏_1(S^1, x_0) = ∏_1(S^1)$.
	
	$∏_1(S^1) \simeq ®Z$.

	\begin{dukazin}
		We claim that $∏_1(S^1) \simeq \<[ω]\>$, where $ω: [0, 1] \rightarrow S^1$, $s \mapsto (\cos(2πs), \sin(2πs)) \in ®R^2$, $s \in [0, 1]$. $ω_n(s) := (\cos(2π n s), \sin(2π n s)) \sim ω^n$, so $[ω]^n = [ω_n]$.

		Now our theorem is equivalent to the statement that every loop in $S^1$ based at $(1, 0)$ is homotopic to the unique $ω_n$. We use the following two facts:

		Fact 1: For every path $f: I \rightarrow X$ starting at $x_0 \in X$ and each $\tilde x_0 \in p^{-1}(x_0)$ there is a unique lift $\tilde f: I \rightarrow \tilde X$ starting at $x_0$.

		Fact 2: For each homotopy $f_x: I \rightarrow X$ of paths starting at $x_0$ and each $\tilde x_0 \in p^{-1}(x_0)$ $\exists$ unique lifted homotopy $\tilde f_t: I \rightarrow \tilde X$ of paths starting at $\tilde x_0$.

		$p$ that we need: $p: ®R \rightarrow S^1$; $p(s) = (\cos 2π s, \sin 2π s)$. If we define $\tilde ω_n(s) = n·s$. We will apply Facts 1 and 2 to $p: ®R \rightarrow S^1$, $\tilde ω_n$: Given $f: [0, 1] \rightarrow S^1$ based at $(0, 1)$ representing some element of $∏_1(S^1)$. We take $\tilde f$. Since $p \tilde f(1) = f(1) = (1, 0)$ (and $p^{-1}(1) \in ®Z$), we can argument that if $\tilde f$ ends at $u$ (i.e. $\tilde f(1) = f$), it is homotopoc to $\tilde ω_n$ by the homotopy $\tilde F = (1 - t)\tilde f + t\tilde ω_n$.

		From fact 1 exists $\tilde f$ starting at 0 and ending at $p^{-1}(1) \in ®Z$.

		Theorem: Exists homotopy $\tilde F$ from $\tilde ω_k$ to $\tilde f$ denoted by $(*)$.

		So we define homotopy $F$ from $ω_n$ to $f$ by $F = p∘\tilde F$, homotopy from $ω_n$ to $f$. Since $[ω_n] = n·[ω]$, $∏_1(S^1) \simeq ®Z$.

		Now we would like to show that $[f]$ is uniformly determined. Assume that $f \sim ω_n$ and $f \sim ω_m$, then using Facts 1 and 2 we have $[ω_n] = [ω_m]$ which lends to contradiction since they have different endpoints on ®R.
	\end{dukazin}
\end{veta}

\begin{definice}
	Given a topological space $X$, a covering space of $X$ consists of a topological space $\tilde X$ and a continuous map $p: \tilde X \rightarrow X$ satisfying that $\forall x \in X$ $\exists$ open neighbourhood $U$ of $x$ in $X$ such that $p^{-1}(U)$ is a disjoint union of open subsets $U_α$ each of which is homeomorphically mapped to $U$.
\end{definice}

\begin{definice}
	Given a map $[0, 1] \overset{f}\rightarrow X$ and $p: \tilde X \rightarrow X$ we say that $\tilde f:[0, 1] \rightarrow \tilde X$ is a lift of $f$ if $p ∘ \tilde f = f$.

	The same construction can be defined for homotopy.
\end{definice}

\begin{tvrzeni}
	Given a map $F: Y \times [0, 1] \rightarrow X$ and a map $\tilde F: Y \times \{¦o\} \rightarrow \tilde X$, where $p: \tilde X \rightarrow X$ is a covering space, and $\tilde F$ lifts $F|_{Y \times \{¦o\}}$; there restricting to $\tilde F$ on $Y \times \{¦o\}$.

	\begin{poznamka}[Corollary: Fact 1 and Fact 2 from the previous proof]
		Fact 1 is free, it comes when $Y = \{\text{point}\}$, Fact 2 also follows.
	\end{poznamka}
\end{tvrzeni}

% 23. 10. 2023

\begin{priklad}
	We say that a topological (path-connected) space is simply connected $\Leftrightarrow$ $∏_1(X) = \{e\}$. Examples of simply connected topological spaces: $®R, ®R^2, …$. $S^1$ is not simply connected.
\end{priklad}

\begin{priklad}
	Given $X, Y$ path-connected and $x_0 \in X$, $y_0 \in Y$. Show that $∏_1(X\times Y, (x_0, y_0)) \simeq ∏_1(X, x_0) \times ∏_1(Y, y_0)$.

	\begin{reseni}
		Product topology is defined to be such that a map $f: Z \rightarrow X \times Y$ is continuous $\Leftrightarrow$ ($p_x: X\times Y \rightarrow X$, $p_y: X \times Y \rightarrow Y$) $p_x ∘ f$ and $p_y ∘ f$ are continuous.

		A loop $γ: [0, 1] \rightarrow X \times Y$ based at $(x_0, y_0)$ splits at two loops $γ_1: [0, 1] \rightarrow X$, $γ_2: [0, 1] \rightarrow Y$. The same holds for homotopy, i.e. $F$ from $γ$ to $\tilde γ$ splits into $(F_1, F_2)$, where $F_1$ is a homotopy on $X$ from $γ_1$ to $\tilde γ_1$ and $F_2$ is a homotopy on $Y$ from $γ_2$ to $\tilde γ_2$.
	\end{reseni}
\end{priklad}

\begin{dusledek}
	$∏_1(T^n) := ∏_1(S^1 \times S^1 \times … \times S^1) = ®Z^n$.
\end{dusledek}

\begin{priklad}
	Show that TODO!!! is a covering space for $S^1 V S^1$.
\end{priklad}

TODO!!!

\end{document}
