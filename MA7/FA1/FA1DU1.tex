\documentclass[12pt]{article}					% Začátek dokumentu
\usepackage{../../MFFStyle}					    % Import stylu



\begin{document}

\begin{priklad}[11]
	Let $X = C^∞([0,1])$ be the space of $C^∞$ functions on $[0,1]$ (i.e., $C^∞$ functions on $(0,1)$ such that all the derivatives can be continuously extended to $[0,1]$) equipped with the family of seminorms ©P:
	$$ p_n(f) = \sup_{t \in [0,1]}\left|f^{(n)}(t)\right|, \qquad f \in X \qquad \(n \in ®N_0\). $$

	1. Describe a base of neighborhoods of zero.

	\begin{reseni}
		Podle Věty 5 je báze okolí ¦o rovná
		$$ \{\{x \in X \middle| \tilde p_1(x) < c_1, …, \tilde p_k(x) < c_k\} \middle | \tilde p_1, …, \tilde p_k \in ©P, c_1, …, c_k > 0\}, $$
		tedy jsou to množiny těch funkcí, které mají vždy určité (konečný počet) derivace omezené určitými konstantami.
	\end{reseni}

	2. Is $X$ Hausdorff? Is $X$ metrizable? Is it a Fréchet space?

	\begin{reseni}
		Je Hausdorfův, protože pro každou nenulovou $f \in C^∞$ je už $p_0$ nenulová (a použijeme Větu 5 z přednášky). Z Věty 5 také plyne, že $X$ je (s kanonickou topologií) LCS, tedy podle Věty 22 (i a iv) je $X$ metrizovatelný.

		Aby to byl Fréchetův prostor, musíme ukázat, že je úplný: Mějme tedy cauchyovskou posloupnost (podle tvrzení 21. je tedy cauchyovská v každé $p_n$). To znamená, že každá $i$-tá derivace konverguje stejnoměrně, tedy konverguje k derivaci $(i-1)$-ní derivaci (o záměně limity a derivace) a zároveň k spojité funkci (stejnoměrná konvergence spojitých funkcí je spojitá), tedy limita posloupnosti má všechny derivace spojité.
	\end{reseni}

	\break

	3. Is $X$ normable?

	\begin{reseni}
		Podle Věty 23 nám stačí rozhodnout, zda existuje omezené (podle lemma 15 omezené v každé normě) okolí ¦o. Každé okolí ¦o musí obsahovat množinu z báze okolí ¦o. Ale taková množina je omezená vždy jen v konečně mnoha normách. Je-li $p_n$ poslední norma (to neznamená, že v nějaké menší není omezena), ve které je daná množina omezená, $ε$ minimum z konstant z definice báze okolí ¦o (BÚNO $ε < 1$), pak můžeme pro každé $1 < K \in ®R$ najít $f(x) = ε\frac{ε^n}{K^n} gon_n\(\frac{K}{ε}x\)$, kde ($gon_i$ je $\sin$ pro sudé $i$ a $\cos$ pro liché $i$), což je funkce, která je jistě $C^∞$, pro $k ≤ n$ je
		$$ p_k(f) = \sup_{x \in [0, 1]} \left|f^{(k)}(x)\right| = \sup_{x \in [0, 1]} \left|ε·\frac{ε^{n - k}}{K^{n - k}} gon_{n - k}\(\frac{K}{ε}x\)\right| ≤ \sup_{x \in [0, 1]} ε·1·1 = ε, $$
		ale
		$$ p_{n+1}(f) = \sup_{x \in [0, 1]} \left|f^{(n+1)}(x)\right| = \sup_{x \in [0, 1]} \left|K·\cos\(\frac{K}{ε}x\)\right| = K. $$
		Tedy $p_{n+1}$ na tomto okolí je neomezená. Tedy neexistuje okolí ¦o, které by bylo omezené, tedy prostor $X$ není normovatelný.
	\end{reseni}

	4. Describe all the continuous linear functionals on $X$.

	\begin{reseni}
		Podle Tvrzení 14 musí pro $φ \in X^*$ existovat $c > 0$, $\tilde p_1, …, \tilde p_n \in ©P$, že $\forall f \in X: |L(f)| ≤ c·\max\{\tilde p_1(f), …, \tilde p_n(f)\}$.

		A pokud v předchozí $n$-tici norem máme pouze jednu, $p_k$, pak se můžeme podívat, jak působí tento funkcionál na $k$-tou derivaci $f \in X$ jako prvek prostoru $C([0, 1])$, tj. $f^{(k)} \in C([0, 1])$. $|L(f)| ≤ c·p_k(f) = c·\|f^{(k)}\|_∞$, tedy $L$ je spojitá (a zřejmě lineární) i na $\{f^{(k)}\} \subset C([0, 1])$. Navíc z Weierstrassovy věty je $\{f^{(k)}\}$ (libovolné nekonečně hladké funkce, neboť „odderivovat“ můžeme vždy neurčitým Lebesgueovým integrálem, který je spojitý) hustá v $C([0, 1])$, tedy $L$ jednoznačně určuje prvek $C([0, 1])^* = ©M([0, 1])$ (regulární borelovské znaménkové míry).

		Naopak libovolný prvek $μ \in C([0, 1])^*$ lze aplikovat na $\{f^{(k)}\}$ a splňuje podmínku $\forall f \in X: |μ(f)| := |μ(f^{(k)})| ≤ \|μ\| · \|f^{(k)}\|_∞ = c·p_k(f)$. Tedy spojité funkcionály splňující $|L(f)| ≤ c·p_k$ odpovídají (pro každé $k$ zvlášť) jedna ku jedné $C([0, 1])^*$.

		Navíc si můžeme všimnout, že pokud nenulové spojité funkcionály $K, L$ odpovídají jiným $p_k, p_l$, $l < k$, pak jsou jiné, neboť 
	\end{reseni}
\end{priklad}

\end{document}
