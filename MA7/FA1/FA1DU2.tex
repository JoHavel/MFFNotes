\documentclass[12pt]{article}					% Začátek dokumentu
\usepackage{../../MFFStyle}					    % Import stylu



\begin{document}

\begin{priklad}[5]
	Find all distributions $U$ on ®R satisfying $\sin · U = 0$.

	\begin{reseni}
		Nejprve dokážeme, že „$\supp U \subseteq \{k·π | k \in ®Z\} =: π®Z$“: Máme-li $f \in ©D(®R)$ takovou, že $\supp φ \subset (a, b)$ pro nějaký interval $[a, b] \cap π®Z = \O$. Potom definujme
		$$ ψ(x) = \begin{cases}0, \qquad x \notin (a, b)\\ \frac{φ(x)}{\sin(x)}, \qquad x \in (a, b)\end{cases}. $$
		Potom $ψ$ je zřejmě dobře definované, neboť $\sin(x) ≠ 0$ pro žádné $x \in (a, b)$. Navíc podíl dvou $C^∞$ funkcí, kde jmenovatel nenabývá nuly, je $C^∞$ funkce, a zřejmě každá derivace bude mít support uvnitř $(a, b)$. Tedy $ψ \in C^∞$. Tudíž dostáváme, co jsme chtěli dokázat:
		$$ 0 = (\sin·U)(ψ) = U(\sin · ψ) = U(φ). $$

		Nyní mějme $ε < 1 / 4$ hladké jádro $h_ε$ a definujme $ξ_k = χ_{B(k·π, ε)} * h_ε$ a pro libovolné $φ \in ©D(®R)$ definujme $φ_k = ξ_k · φ$. Potom $\supp φ_k \subset B(k·π, 1 / 2)$ a $\supp \(φ - \sum_{k=-∞}^∞ φ_k\) \cap π®Z = \O$ (v sumě je pouze konečně mnoho nenulových prvků, protože $\supp φ$ je kompaktní). Tedy (s použitím linearity $U$):
		$$ U(φ) = U\(\(φ - \sum_{k=-∞}^∞ φ_k\) + \sum_{k=-∞}^∞ φ_k\) = U\(φ - \sum_{k=-∞}^∞ φ_k\) + \sum_{k=-∞}^∞ U(φ_k). $$
		První člen napravo je nulový z Větičky 12 bod c), druhý „člen“ pak má pro každé $k$ tvar $U(φ_k) = \sum_{α, |α| ≤ N_k} c_{k, α}D^α Λ_{δ_{k·π}}$ pro $c_{k, α} \in ®R$ a nějaké $N_k \in ®N_0$, neboť buď je $k·π \in \supp U$ a pak je $\supp U|_{B(k·π, 1 / 2)} = \{k·π\}$, tedy použijeme Větičku 12 bod e), nebo je $k·π \notin \supp U$ a pak je $U|_{B(k·π, 1 / 2)} = 0$ (to nám přidává všechna $c_{k, α} = 0$).

		Tedy všechna taková $U$ jsou přesně
		$$ U = \sum_{k=-∞}^∞ \sum_{α, |α| ≤ N_k} c_{k, α} D^α Λ_{δ_{k·π}}, $$
		kde pro všechna $k$ je $N_k \in ®N_0$ a pro všechna $|α| ≤ N_k$ pak $c_{k, α} \in ®R$. $U \in ©D'(®R)$ máme z toho, že $U|_{©D_K(®R)}$ je spojitá (Tvrzení 6 bod (3) a uzavřenost $©D'(®R)$ na sčítání).
	\end{reseni}

	Find all distributions $V$ on ®R satisfying $\sin · V = Λ_1$.

	\begin{reseni}
		Jsou-li $V_1$ a $V_2$ takové distribuce, pak $\sin·(V_1 - V_2) = \sin·V_1 - \sin·V_2 = Λ_1 - Λ_1 = 0$. Tedy všechna taková $V$ se liší právě o řešení předchozí části.

		Tedy hledáme jedno řešení. Ukážeme, že jedno takové řešení je
		$$ Λ(φ) = \lim_{n \rightarrow ∞} Λ_n(φ) = \lim_{n \rightarrow ∞} \int_{®R \setminus \(π®Z + B\(¦o, \frac{1}{n}\)\)} \frac{φ(x)}{\sin(x)} dx, \qquad φ \in ©D(®R). $$
		Z TODO(název věty) nám stačí ukázat, že $Λ_n$ jsou distribuce a že $\exists \lim_{n \rightarrow ∞} Λ_n(φ) ≠ ±∞$ pro každé $φ$. To, že je $Λ_n = Λ_{\frac{1}{\sin}}$ je distribuce je jasné, neboť $\frac{1}{\sin}$ je lokálně integrovatelná, neboť je dokonce omezená, na $®R \setminus \(π®Z + B\(¦o, \frac{1}{n}\)\)$.

		BÚNO $\supp φ \subset \(-\frac{3}{4} π, \frac{3}{4} π\)$, protože jinak použijeme na všechny nenulové funkce $φ_k = φ(χ_{\{k·π\} + \(\frac{1}{2}π, \frac{1}{2}π\)} * h_{\frac{1}{4}π})$ a na $φ - \sum_{k \in ®Z}φ_k$. Potom z definice derivace $φ'(0)$, která existuje z hladkosti $φ$, $\exists$ $ε > 0$ a $C_1 := 2φ'(0)$, že $φ(x)$ je mezi $0$ a $φ(0) + C_1·x$ na intervalu $(-ε, ε)$. Navíc můžeme z definice derivace $\sin'(0) = 1$ ($-1$ v případě lichých $k$) zvolit $ε$ takové, aby i $|\sin(x)| > C_2·|x| := \frac{1}{2}·|x|$ na intervalu $(-ε, ε)$. Nechť $m > n > \frac{1}{ε}$, potom
		$$ |Λ_n(φ) - Λ_m(φ)| = \left|\int_{®R \setminus \(π®Z + B\(¦o, \frac{1}{n}\)\)} \frac{φ(x)}{\sin(x)}dx - \int_{®R \setminus \(π®Z + B\(¦o, \frac{1}{m}\)\)} \frac{φ(x)}{\sin(x)}dx\right| = $$
		$$ = \left|\int_{-\frac{1}{n}}^{-\frac{1}{m}} \frac{φ(x)}{\sin(x)}dx + \int_{\frac{1}{m}}^{\frac{1}{n}} \frac{φ(x)}{\sin(x)}dx\right| = $$
		$$ = \left|\underbrace{\int_{-\frac{1}{n}}^{-\frac{1}{m}} \frac{φ(0)}{\sin(x)}dx + \int_{\frac{1}{m}}^{\frac{1}{n}} \frac{φ(0)}{\sin(x)}dx}_{=0} + \int_{-\frac{1}{n}}^{-\frac{1}{m}} \frac{φ(x) - φ(0)}{\sin(x)}dx + \int_{\frac{1}{m}}^{\frac{1}{n}} \frac{φ(x) - φ(0)}{\sin(x)}dx\right| ≤ $$
		$$ ≤ \left|\int_{-\frac{1}{n}}^{-\frac{1}{m}} \frac{C_1·|x|}{C_2·|x|}dx + \int_{\frac{1}{m}}^{\frac{1}{n}} \frac{C_1·|x|}{C_2·|x|}dx\right| = 2·\frac{C_1}{C_2} \(\frac{1}{n} - \frac{1}{m}\) ≤ 2·\frac{C_1}{C_2}·ε \overset{ε\rightarrow 0}\longrightarrow 0. $$
		Tedy dle B–C (pro každé $n > \frac{1}{ε}$ můžeme $ε$ zmenšit na $\frac{1}{n}$, čímž nám předpoklady pořád platí, $C_1, C_2$ jsou na $ε$ nezávislé) podmínky posloupnost konverguje.

		% Sporem dokážeme, že takové $V$ neexistuje: Mějme tedy takové $V$. Nechť $K = [-π, π]$ a $L = [0, l]$, kde $0 < l < π$. Z definice distribuce (Tvrzení 6 bod 4) máme (pro naše konkrétní $K$):
		% $$ \exists N \in ®N_0\ \exists C > 0\ \forall φ \in ©D_K(Ω): |V(φ)| ≤ C·\|φ\|_N. $$
		% Tyto $N$ a $C$ zafixujme. Nyní vezměme nezápornou nenulovou testovací funkci $ψ$ tak, že $\supp ψ = L$ a definujme $ψ_ε(x) := ψ(x + ε)$. Všimněme si, že $\|·\|_N$ je invariantní vůči aplikaci $τ_ε$, tedy $\|ψ_ε\|_N = \|ψ\|_N$ pro všechna $ε$. Navíc $\supp ψ_ε = L_ε := [ε , l+ε] \subset K$.
		%
		%Na druhou stranu, pro všechna $ε > 0$, že $ε + l < π$, je $V(ψ_ε) = \int_{-∞}^∞ \frac{ψ_ε}{\sin x} = \int_ε^{l + ε} \frac{ψ_ε}{\sin x}$, jelikož můžeme do zadání stejně jako v první části dosadit $ψ_ε$ „podělené“ $\sin$ (neboť $\supp ψ_ε \cap π®Z = \O$).
	\end{reseni}
\end{priklad}

\end{document}
