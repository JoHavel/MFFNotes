\documentclass[12pt]{article}					% Začátek dokumentu
\usepackage{../../MFFStyle}					    % Import stylu



\begin{document}

% 03. 10. 2022

\begin{poznamka}
	Topology…
\end{poznamka}

\vspace{-2em}

\section{Locally convex spaces}

\begin{definice}[Topological vector space (TVS)]
	A Topological vector space over ®F is a pair $(X, τ)$, where $X$ is a vector space over ®F and $τ$ is a topology on $X$ with the following two properties:
	\begin{enumerate}
		\item The mapping $(x, y) \mapsto x + y$ is a continuous mapping of $X \times X$ into $X$;
		\item The mapping $(t, x) \mapsto tx$ is a continuous mapping of $®F \times X$ into $X$;
	\end{enumerate}

	We also denote Hausdorff topological vector space by HTVS. And the symbol $τ(¦o)$ will denote the family of all the neighbourhoods of ¦o in $(X, τ)$.
\end{definice}

\begin{definice}[Locally convex (LCS, HLCS)]
	Let $(X, τ)$ be a TVS. The space $X$ is said to be locally convex, if there exists a base of neighbourhoods of zero consisting of convex sets.
\end{definice}

\begin{poznamka}[Credit]
	Two homework (in Moodle) and one presentation.
\end{poznamka}

%\begin{priklady}
%	Let $(X, \|·\|)$ be a normed linear space. Let $τ$ be the topology induced by $\|·\|$. The $(X, τ)$ is HLCS.
%
%	\begin{dukazin}
%		$ρ(x, y) = \|x - y\|$ metric induced by $\|·\|$. $τ$ induced by $ρ$. This $τ$ is Hausdorff. Continuity of the operations: (from Funkcionalka)
%		$$ x_n \rightarrow x, y_n \rightarrow y, t_n \rightarrow t \implies x_n + y_n \rightarrow x + y \land t_nx_n \rightarrow tx. $$
%
%		So, it is a HTVS. Base of neighbourhood of ¦o is e. g. $U(0, r), r > 0$, which is convex.
%	\end{dukazin}
%
%	Let $Γ$ be any nonempty set, $X = ®F^Γ$ (= all functions $Γ \rightarrow ®F$) with point-wise operations, so it is a vector space over ®F. It is a HLCS.
%
%	\begin{dukazin}
%		„Continuity of addition:“ $x, y \in ®F^Γ$, $U$ a neighbourhood of $x + y$ $\implies$ $\exists F \subset Γ$ finite $\exists ε > 0$ such that
%		$$ U_{¦o} = \{z \in ®F^Γ \middle|\ \forall γ \in F: |z(γ) - (x(γ) + y(γ))| < ε\} \subset U $$
%		$$ U_x = \{z \in ®F^Γ \middle|\ \forall γ \in F: |z(γ) - x(γ)| < \frac{ε}{2}\} $$
%		$$ U_y = \{z \in ®F^Γ \middle|\ \forall γ \in F: |z(γ) - y(γ)| < \frac{ε}{2}\} $$
%		$\implies$ $V_x$ is neighbourhood of $x$, and $V_y$ is neighbourhood of $y$, and $U_x + U_y \subset U_0 \subset U$. Thus $z_1 \in V_x$, $z_2 \in V_y$ $\implies$ $z_1 + z_2 \in U_0 \subset U$.
%
%		„Continuity of multiplication“: $λ \in ®F$, $x \in ®F^Γ$, $U$ a neighbourhood of $λ x$ $\implies$ $\exists F \subset Γ$ finite $\exists μ > 0$ such that
%		$$ U_0 = \{z \in ®F^Γ \middle| \forall γ \in F: |z(γ) - λx(γ)| < ε\} \subset U $$
%		$$ |μz(γ) - λx(γ)| ≤ |μ|·|z(γ) - x(γ)| + |μ - λ|·|x(f)|. $$
%		$$ M:= \max_{γ \in F}|x(γ)|. $$
%		$$ V = \{μ \in ®F \middle| |μ - λ| < \frac{ε}{2(M + 1)}\}, \qquad W = \{z \in ®F^Γ \middle| \forall γ \in F: |z(γ) - x(γ)| < \frac{ε}{2(|λ| + \frac{ε}{2(M+1)})}\}. $$
%		$$ μ \in V, z \in W \implies μ z \in U_0 \subset U. $$
%
%		„Local convexity“: Base of neighbourhoods of ¦o: $\{x \in ®F^Γ \middle| \forall γ \in F: |x(γ)| < ε\}$, $F \subset Γ$ finite, $ε > 0$, consists of convex sets.
%
%		„Hausdorff“: $x ≠ y \implies \exists γ \in Γ: x(γ) ≠ y(γ)$. Take $ε = \frac{|x(γ) - y(γ)|}{2}$.
%		$$ U = \{z \in ®F^Γ \middle| |z(γ) - x(γ)| < ε\}, V = \{z \in ®F^Γ \middle| |z(γ) - y(γ)| < ε\} \implies U \cap V = \O. $$
%	\end{dukazin}
%
%	$X = C(®R, ®F) = \{f: ®R \rightarrow ®F \text{ continuous}\}$,
%	$$ ρ(f, g) = \sum_{n=1}^∞ \frac{1}{2^n} \min\{1, \max_{t \in [-n, n]}\{|f(t) - g(t)|\}\} =: \sum_{N=1}^∞ \frac{1}{2^N} \min\{1, p_N(f - g)\} $$
%	is translation invariant (that implies addition is continuous, see lecture) metric.
%	
%	\begin{dukazin}
%		$f_n \rightarrow f$ in $ρ$ $\Leftrightarrow$ $\forall N: f_n \rightrightarrows f$ on $[-N, N]$.
%
% 06. 10. 2023
%
%		„$f_n \rightarrow f$, $λ_n \rightarrow λ$ $\implies$ $λ_n f_n \rightarrow λ f$“: Let $N \in ®N$. We will show $λ_n f_n \rightrightarrows λ f$ in $[-N, N]$. $x \in [-N, N]$:
%	$$ |λ_n f_n(x) - λ f(x)| ≤ |λ_n|·|f_n(x) - f(x)| + |λ_n - λ|·|f(x)| ≤ c·p_N(f_n - f) + |λ_n - λ|·p_N(f) \rightarrow 0. $$
%
%		Hence, $X$ is HTVS. „Local convexity“: $U_{N, ε} = \{f \in X | p_N(t) < ε\}$, clearly $U_{N, ε}$ is a convex set and $U_{N, ε}$ is neighbourhood of ¦o. If $ε<λ$, then $\{f | ρ(f, ¦o) < \frac{ε}{2^N}\} \subset U_{N, ε}$, because for $ρ(f, ¦o) < \frac{ε}{2^N}$ it is $\frac{1}{2^N}p_N(f) < \frac{ε}{2^N}$. „they form a base“: $f \in U_{N, ε} \implies ρ(f, ¦o) < ε + \frac{1}{2^N}$. Hence fix $r > 0$ and take $N \in ®N$ such that $\frac{1}{2^N} < \frac{r}{2}$. Then $U_{N, \frac{r}{2}} \subset \{f | ρ(f, ¦o) < r\}$
%	\end{dukazin}
%
%	$(Ω, Σ, μ)$ a measure space, $p \in (0, 1)$. $L^p(Ω, Σ, μ) = \{f: Ω \rightarrow ®F \text{ measurable} | \int |f|^p dμ < ∞\}$ (we identify functions equal almost everywhere). $ρ(f, g) = \int |f - g|^p dμ$ is a metric making $X = L^p(Ω, Σ, μ)$ a HTVS (but not locally convex).
%
%	\begin{dukazin}
%		„$ρ$ is a metric“: „$\triangle$-inequality“: $a, b \in [0, ∞): (a + b)^p ≤ a^p + b^p$. (Fix $a ≥ 0$, take $φ_a(b) = (a + b)^p - a^p - b^p$ $\implies$ $φ_a$ is continuous on $[0, ∞)$, $φ_a(0) = 0$. For $b > 0$: $φ_a(b) = p(a + b)^{p - 1} - pb^{p - 1} = p·\((a + b)^{p-1} - b^{p - 1}\) < 0$ as $p - 1 < 0$ $\implies$ $φ_a$ decreasing on $[0, ∞)$ and $φ_a ≤ 0$.)
%
%		$φ$ is translation invariant $\implies$ addition is continuous. „Multiplication“: We can see that $ρ(λf, ¦o) = |λ|^pρ(f, ¦o)$. $f_n \rightarrow f$, $λ_n \rightarrow λ$:
%		$$ ρ(λ_n f_n, λ f) ≤ ρ(λ_n f_n, λ_n f) + ρ(λ_n f, λ f) = |λ_n|^p ρ(f_n, f) + |λ_n - λ|^p ρ(f, ¦o) \rightarrow 0. $$
%		Hence, we have a HTVS.
%	\end{dukazin}
%\end{priklady}

\begin{tvrzeni}[Observation]
	If $(X, τ)$ is a LCS, then $τ$ is translation invariant ($U \subset X, x \in X \implies (U \in τ \Leftrightarrow x + U \in τ)$). Hence $τ$ is determined by $τ(¦o)$.
\end{tvrzeni}

\begin{definice}[convex, symmetric, balanced, absolutely convex, and absorbing set]
	$X$ is a vector space, $A \subset X$. Then $A$ is
	\begin{itemize}
		\item convex if $tx + (1 - t)y \in A$ for $x, y \in A$, $t \in [0, 1]$;
		\item symmetric if $A = -A$;
		\item balanced if $α A \subset A$ for $α \in ®F$, $|α| ≤ 1$;
		\item absolutely convex if it is convex and balanced;
		\item absorbing if $\forall x \in X\ \exists t > 0: \{s·x | s \in [0, t]\} \subset A$.
	\end{itemize}
\end{definice}

\begin{definice}
	$\co(A) = $ convex hull, $\b(A) = $ balanced hull, $\aco(A) = $ absolutely convex hull.
\end{definice}

\begin{tvrzeni}
	$X$ is a metric space over ®F, $A \subset X$. Then:

	(a) If $®F = ®R$, it holds $A$ is absolutely convex $\Leftrightarrow$ $A$ is convex and symmetric.

	(b) $\co A = \{t_1x_1 + … + t_kx_k | x_1 … x_k \in A, t_1 … t_k ≥ 0, t_1 + … + t_k = 1, k \in ®N\}$.

	(c) $\b(A) = \{αx | x \in A, α \in ®F, |α| ≤ 1\}$.

	(d) $\aco(A) = \co(\b(A))$.

	(e) $A$ is convex $\Leftrightarrow$ $(s + t) A = sA + tA$ for all $s, t > 0$.

	\begin{dukazin}[a]
		„$\implies$“: trivial (and it also holds for $®F = ®C$). „$\impliedby$“: Assume $A$ is convex and symmetric. We show that $A$ is balanced:
		$$ x \in A, α \in ®R, |α| ≤ 1 \implies α \in [-1, 1]. $$
		And $x \in A, -x \in A$, so the segment from $x$ to $-x$ is contained in $A$ ($α x = \frac{1 - α}{2}(-x) + \frac{(1 + α)}{2} x \in A$).
	\end{dukazin}

	\begin{dukazin}[b]
		„$\subseteq$“: by induction on $k$:\vspace{-0.6em}
		$$ t_1 x_1 + … + t_{k+1} x_{k+1} = (t_1 + … + t_k) \frac{t_1x_1 + … + t_k x_k}{t_1 + … + t_k} + t_{k+1}x_{k+1}. \vspace{-0.2em} $$
		„$\supseteq$“: the set on the RHS is convex and contain $A$.
	\end{dukazin}

	\begin{dukazin}[c]
		„$\supseteq$“: clear. „$\subseteq$“: RHS is a balanced set.
	\end{dukazin}

	\begin{dukazin}[d]
		„$\supseteq$“: clear. „$\subseteq$“ the set on the RHS is absolutely continuous (Clearly RHS is convex. „balanced“: using (b) and (c):\vspace{-0.6em}
		$$ \co(\b(A)) = \{t_1 α_1 x_1 + … + t_kα_kx_k | x_1, …, x_k \in A, |α_j| ≤ 1, t_j ≥ 0, t_1 + … + t_k = 1\} \vspace{-0.6em} $$
		is clearly balanced.)
	\end{dukazin}

	\begin{dukazin}[e]
		„$\implies$“: „$\subseteq$“: always, „$\supseteq$“: $sa_1 + ta_2 = (s + t)·\(\frac{s}{s + t} a_1 + \frac{t}{s + t} a_2\)$.

		„$\impliedby$“: in particular $\forall t \in (0, 1)$: $tA + (1 - t)A \subset A$, it is the definition of convexity.
	\end{dukazin}
\end{tvrzeni}

\begin{tvrzeni}
	Let $(X, τ)$ be a LCS, $U \in τ(¦o)$. Then
	
	(i) $U$ is absorbing.

	(ii) $\exists V \in τ(¦o): V + V \subset U$.

	(iii) $\exists V \in τ(¦o)$ absolutely convex, open: $V \subset U$.

	\begin{dukazin}[i]
		$$ x \in X \implies 0·x = ¦o \in U \implies \exists V \text{ a neighbourhood of $0$ in ®F}: V·x \subset U \implies $$
		$$ \implies \exists t > 0: [0, t] \subset V. $$
	\end{dukazin}

	\begin{dukazin}[ii]
		$$ ¦o + ¦o = ¦o \in U \implies \exists W_1, W_2 \text{ neighbourhoods of ¦o}: W_1 + W \subset U. $$
		Take $V = W_1 \cap W_2$.
	\end{dukazin}

	\begin{dukazin}
		$$ \exists U_0 \in τ(¦o) \text{ convex}, U_0 \subset U: 0·¦o = ¦o \in U_0 \implies \exists c > 0\ \exists W \in τ(¦o) \text{ open}: $$
		$$ \forall λ, |λ| < c: λW \subset U_0. $$
		$V_1 := \bigcup_{0 < |λ| < 1} λW$. Then $V_1 \in τ(0)$ open, balanced, $V_1 \subset U_0$. Let $V := \co V_1$. Then $V$ is absolutely convex (the previous proposition (d)), $V \subset U_0 \subset U$ (as $V_0$ is convex). $V \in τ(¦o)$ as $V \supset V_1$. „$V$ is open“:
		$$ V = \bigcup\{t_1x_1 + … + t_nx_n + t_{n+1}V_1 | t_1, …, t_{n+1} ≥ 0, t_1 + … + t_{n+1} = 1, x_1, …, x_n \in V_1\} $$
	\end{dukazin}
\end{tvrzeni}

% 10. 10. 2023

\begin{veta}
	1. Let $(X, τ)$ be a LCS. Then there is ©U, a base of neighbourhoods of ¦o with properties:
	\begin{itemize}
		\item the elements of ©U are absorbing, open, absolutely convex;
		\item $\forall U \in ©U\ \exists V \in ©U: 2V \subset U$.
	\end{itemize}
	If $X$ is Hausdorff, then $\bigcap ©U = \{¦o\}$.

	2. Let $X$ be a vector space, ©U a nonempty family of subsets of $X$ satisfying:
	\begin{itemize}
		\item the elements of ©U are absorbing and absolutely convex;
		\item $\forall U \in ©U\ \exists V \in ©U: 2V \subset U$;
		\item $\forall U, V \in ©U\ \exists W \in ©U: W \subset U \cap V$.
	\end{itemize}
	Then there is a unique topology $τ$ on $X$ such that $(X, τ)$ is LCS and ©U is a base of neighbourhoods of ¦o. Further, if $\bigcap ©U = \{¦o\}$, the $τ$ is Hausdorff.

	\begin{dukazin}[1.]
		Let ©U be the family of all open absolutely convex neighbourhoods of ¦o. The previous proposition (iii) gives us ©U is a base of neighbourhoods of ¦o, (1) gives us elements of ©U are absorbing, so the first item holds. (ii) gives us $U \in ©U \implies \frac{1}{2} U \in ©U$.

		Assume $X$ is Hausdorff: $x \in X \setminus \{¦o\} \overset{\text{Hausdorff}}\implies \exists U \in τ(¦o): x \notin U \implies \exists V \in ©U: V \subset U: x \notin V$.
	\end{dukazin}
	
	\begin{dukazin}[2.]
		Set $τ = \{G \subset X | \forall x \in G\ \exists U \in ©U: x + U \subset G\}$. This is a unique possibility so uniqueness is clear.

		„$τ$ is topology“: $\O, X \in τ$ and $τ$ is closed to arbitrary union (clear). $τ$ is closed to finite intersections by third item ($G_1, g_2 \in τ$, $x \in G_1 \cap G_2 \ldots U_1, U_2 \in τ$, $x + U_1 \subset G_1$, $x + U_2 \subset G_2$; $\exists V \in ©U: V \subset U_1 \cap U_2$, then $x + V \subset (x + U_1)\cap (x + U_2) \subset G_1 \cap G_2$ $\implies$ $G_1 \cap G_2 \in τ$).

		„Elements of ©U are neighbourhoods of ¦o“: $U \in ©U$. $V:=\{x \in U | \exists W \in ©U: x + W \subset U\}$. Then $V \subset U$, $0 \in V$ (take $W = U$). $V \in τ$ ($x \in V \implies \exists W \in ©U: x + W \subset U$; let $\tilde W \in ©U$ such that $2\tilde W \subset W$, then $x + \tilde W \subset V$, because $y \in \tilde W \implies X + y + \tilde W \subset x + \tilde W + \tilde W \subset x + W \subset U$).

		„©U is a base of neighbourhood of ¦o“: now clear.

		„$(X, τ)$ is a TVS“: $x + y \in G \in τ \implies \exists U \in ©U: x + y + U \subset G \implies \exists V \in ©U: 2V \subset U.$ Then $(x + V) + (y + V) \subset x + y + 2V \subset x + y + U \subset G$. $λx \in G \in τ \implies \exists U \in ©U: λx + U \subset G$; $\exists V \in ©U: 2V \subset U$; $V$ is absorbing $\implies$ $\exists c > 0\ \forall t \in [0, c]: tx \in V$; $V$ balanced $\implies$ $\forall μ \in ®F, |μ| ≤ c: μ x \in V$; assume $λ \in ®F, |μ - λ| < c, y \in x + \frac{1}{|λ| + 1}V$,
		$$ \implies μy - λx = \underbrace{(μ - λ)y}_{(μ - 1)·\(μ + \frac{1}{|λ| + 1}\)V} + \underbrace{λ(y - x)}_{\in \frac{λ}{|λ| + 1}V \subset V}. $$

		„Local convexity“: by first item: $\forall U \in ©U: U$ is convex.

		Assume $\bigcap ©U = \{¦o\}$. Take $x, y \in X, x ≠ y \implies x - y ≠ ¦o \implies \exists U \in ©U: x - y \notin U$. Take $V \in ©U: 2V \subset U$. Then if $(x + V) \cap (y + V) = \O$, $x + v_1 = y + v_2$, $x - y = v_2 - v_1 \in V + V = 2V \subset U$ \lightning.
	\end{dukazin}
\end{veta}

\begin{veta}[On the topology generated by a family of seminorms]
	Let $X$ be a vector space and let ©P be a family of seminorms on $X$. The there is a unique topology $τ$ on $X$ such that $(X, τ)$ is a LCS and
	$$ ©U = \{\{x \in X | p_1(x) < c_1, …, p_k(x) < c_k\} \middle| p_1, …, p_k \in ©P, c_1, …, c_k > 0\} $$
	is a base of neighbourhood of ¦o.

	$(X, τ)$ is Hausdorff $\Leftrightarrow$ $\forall x \in X \setminus \{¦o\}\ \exists p \in ©P, p(x) > 0$.

	\begin{dukazin}
		Use the previous theorem 2. on ©U: The sets are absolutely convex (by properties of seminorms). „Absorbing“: $U = \{x \in X | p_1(x) < c_1, …, p_k(x) < c_k\}$. Take $x \in X$ ?, $j \in [k]$. Then $p_j(x) \in (0, ∞)$ as for $t > 0: p_j(t·x) = t·p_j(x)$ and $\exists c > 0$ such that $c·p_j(x) < c_j$ for $j \in [k]$. Now for $t \in [0, c]: tx \in U$.

		$U\!=\!\{x \in X | p_1(x) < c_1, …, p_k(x) < c_k\}$. Take $V\!=\!\{x \in X | p_1(x) \subset \frac{c_1}{2}, …, p_k(x) < \frac{c_k}{2}\}$.

		$U, V \in ©U \implies U \cap V \in ©U$ trivially.

		„Hausdorffness“:
		$$ \bigcap U = \{x \in X | \forall p \in ©P: p(x) = 0\}. $$
		„$\supseteq$“ clear. „$\subseteq$“: Assume $y \in X$, $p \in ©P: p(y) > 0$: $U = \{x \in X | p(x) < p(y)\} in ©U \implies y \notin U$.
	\end{dukazin}
\end{veta}

\begin{priklady}
	$(X, \|·\|)$ is a normed space, then its topology is generated by $©P = \{\|·\|\}$.\\
	The topology on $®F^Γ$ is generated by seminorms $p_γ(f) = |f(γ)|$, $f \in ®F^Γ$ ($γ \in Γ$).\\
	$C(®R, ®F)$ + topology is generated by this sequence of seminorms: $p_N(f) = \max_{x \in [-N, N]} |f(x)|$.
\end{priklady}

\begin{definice}[Minkowski functional]
	$X$ vector space, $A \subset X$ convex absorbing. Then Minkowski functional of set $A$ is defined by the formula $p_A(x) := \inf \{λ > 0 | x \in λ·A\}$.
\end{definice}

\begin{lemma}
	Let $X$ be LCS, $A \subset X$ convex set.
	$$ x \in \overline{A}, y \in \Int A \implies \{tx + (1 - t)y | t \in [0, 1)\} \subset \Int A. $$

	\begin{dukazin}
		WLOG $y = 0$. $t = 0$ clear, $0 \in \Int A$. $t \in (0, 1)$:

		Fix $U$, an open absolutely convex neighbourhood of ¦o such that $U \subset A$. Then $x + \frac{1 - t}{t} U$ is a neighbourhood of $x$ $\implies$ $\exists z \in \(x + \frac{1 - t}{t}U\) \cap A$, i.e. $\exists m \in U$ $z = x + \frac{1 - t}{t}m$ $\implies$ $-m \in U \subset m + A$. Find $V$ an absolutely convex open neighbourhood of ¦o such that $-m + V \subset U \subset A$ $\implies$ $tz + (1 - t)(-m + V) \subset A$ (an open set containing $tx$). (Because $tx = tz - (1 - t)a$.) And that's it.
	\end{dukazin}
\end{lemma}

% 13. 10. 2023

%\begin{dukaz}[Continuity of multiplication? Theorem 4. TODO?]
%	„$U$ is a neighbourhood of ¦o in $τ$, $λ > 0$ $\implies$ $λU$ is neighbourhood of ¦o“: $λ ≥ 1$: $\exists V \in ©U: V \subset U \implies V \subset λV \subset λU$ ($V$ is absolutely convex) $\implies λU$ is neighbourhood of ¦o. $λ = \frac{1}{2}$: $\exists V \in ©U: V \subset U$, then $\exists W \in ©U: 2W \subset V$, then $W \subset \frac{1}{2} V \subset \frac{1}{2}U \implies \frac{1}{2} U$ is a neighbourhood of ¦o. Now by induction for $λ = \frac{1}{2^n}$. For $λ > 0$ find $n \in ®N$ such that $λ > \frac{1}{2^n}$.
%
%	$λx \in G$ ($λ \in ®F, x \in X, G \in τ$) $\implies \exists U \in ©U: λx + U \in G$. Find $V \in ©U: 2V \subset U$ such that $V$ is absorbing ($\implies \exists c > 0\ \forall t \in [0, c]: tx \in V$) and $V$ is balanced ($\implies \forall μ \in ®F, |μ| ≤ c: μ x \in V$). Let $μ \in F, y \in X$ such that
%	$$ |μ - λ| < c \land y \in x + \frac{1}{|λ| + c}V \text{ (a neighbourhood of ¦o)} $$
%	$$ \implies μy - λx = μ(y - x) + (μ - λ)x \in V + V = 2V \subset U \implies μy \in λx + U \subset G. $$
%\end{dukaz}

\begin{tvrzeni}[on the Minkowski functional of a convex neighborhood of zero]
	Let $X$ be LCS, $A \subset X$ a convex neighbourhood of ¦o. Then:

	Clearly: $[p_A < 1] \subset A \subset [p_A ≤ 1]$.
	
	$[p_a < 1] = \Int A$:

	\begin{dukazin}
		„$\subseteq$“: $p_A(x) < 1 \implies \exists c > 1$ such that $cx \in A$ $\implies x = \frac{1}{c} cx \in \Int A$. „$\supseteq$“: $x \in \Int A \implies \exists U \in τ(¦o): x + U \subset A$. $U$ absorbing $\implies$ $\exists α > 0: αx \in U$. Then $(1 + α)x \in A \implies p(x) ≤ \frac{1}{1 + α} < 1$.
	\begin{dukazin}

	$[p_A ≤ 1] = \overline{A}$:

	\end{dukazin}
		„$\subseteq$“: $p_A(x) ≤ 1 \implies \forall n \in ®N: p_x\(\(1 - \frac{1}{n}\)x\) = \(1 - \frac{1}{n}\) p_A(x) ≤ 1$. $\(1 - \frac{1}{n}\)x \in \Int A \implies x \in \overline{\Int A} \subset \overline{A}$. „$\supseteq$“: $x \in \overline{A} \implies \forall n \in ®N: \(1 - \frac{1}{n}\)x \in \Int A$, so, $p_A\(\(1 - \frac{1}{n}\)x\) < 1$ $\overset{n \rightarrow ∞}\implies$ $p_A(x) ≤ 1$.
	\end{dukazin}

	$p_A$ is continuous on $X$.

	\begin{dukazin}
		$[p_A < c] = \O$ if $c ≤ 0$ and $c·\Int A$ if $c > 0$. $[p_A > c] = X$ if $c < 0$, $X \setminus (c·\overline{A})$ if $c > 0$, and $\bigcup_{t > 0} X \setminus t \overline{A}$ if $c = 0$. All these sets are open.
	\end{dukazin}

	$p_A = p_{\overline{A}} = p_{\Int A}$.

	\begin{dukazin}
		$\Int A \subset A \subset \overline{A} \implies p_{\overline{A}} ≤ p_A ≤ p_{\Int A}$. „Conversely“: Assume that $p_{\overline{A}}(x) < c$ $\implies$ $\exists d < c: x \in d·\overline{A}$ $\implies$ $\forall n \in ®N: \(1 - \frac{1}{n}\)x \in d \Int A$ $\implies$ $\(1 - \frac{1}{n}\)p_{\Int A}(x) ≤ d$ $\implies$ $p_{\Int A}(x) ≤ d < c$.
	\end{dukazin}
\end{tvrzeni}

\begin{dusledek}
	Any LCS ($X$) is completely regular.

	\begin{dukazin}
		$x \in X$, $U$ an open neighbourhood of $x$. Take $V$ a convex neighbourhood of ¦o such that $x + V \in U$. $f(y) := \min\{1, p_V(y - x)\}$. Then $f$ is continuous by the previous proposition, $f(x) = 0$.
		$$ y \in X \setminus U \implies y - x \notin V \implies p_V(y - x) ≥ 1 \implies f(y) = 1. $$
	\end{dukazin}
\end{dusledek}

\begin{veta}[On generating of locally convex topologies]
	Let $(X, τ)$ be a LCS. Let $©P_τ$ be the family of all continuous seminorms on $(X, τ)$. Then topology generated by $©P_τ$ coincides with $τ$.

	\begin{dukazin}
		Let $τ_1$ be topology induced by $©P_τ$. $τ_1 \subset τ$ (seminorms from $©P_τ$ are $τ$-continuous, hence the sets from theorem 5? are $τ$-open). „$τ \subset τ_1$“: Let $U \in τ(¦o)$ $\implies$ $\exists V$ a neighbourhood of ¦o such that $V \subset U$. The $p_V \in ©P_τ$ (from the previous proposition is continuous) $\implies$ $[p_V < 1] = V \subset U \implies U \in τ_1(¦o)$.
	\end{dukazin}
\end{veta}

\begin{tvrzeni}
	$X$ a vector space.
	\begin{enumerate}
		\item $p$ is seminorm $\implies$ $[p < 1]$ is absolutely convex, absorbing, and $p_{[p < 1]} = p$.
		\item $p, q$ are seminorms, then $p ≤ q \Leftrightarrow [p < 1] \supset [q < 1]$.
		\item ©P a set of seminorms generated by a topology $τ$. $p$ a seminorm on $X$. Then $p$ is $τ$-continuous $\Leftrightarrow$ $\exists p_1, …, p_k \in ©P\ \exists c > 0: p ≤ c·\max\{p_1, …, p_k\}$.
	\end{enumerate}

	\begin{dukazin}[1.]
		Absolutely convex and absorbing is clear.
		$$ p_{[p < 1]}(x) = \inf \{λ > 0 | x \in λ[p < 1]\} = \inf \{λ > 0 | x \in [p < λ]\} = p(x). $$
	\end{dukazin}

	\begin{dukazin}[2.]
		„$\implies$“ trivial. „$\impliedby$“: $[p < 1] \supset [q < 1] \implies p = p_{[p < 1]} ≤ p_{[q < 1]} = q.$
	\end{dukazin}

	\begin{dukazin}[3.]
		„$\impliedby$“: $A := [p < 1]$ $\implies$ $A \supset [c·\max \{p_1, …, p_k\} < 1] = \[p_1 < \frac{1}{c}, …, p_k < \frac{1}{c}\]$, which is a $τ$-open set $\implies$ $A$ is a neighbourhood of ¦o $\implies$ $p = p_A$ is continuous (by 1. and the previous proposition).

		„$\implies$“: $p$ is continuous $\implies$ $[p < 1]$ is neighbourhood of ¦o ($p(¦o) = 0$) $\implies$ $\exists p_1, …, p_k \in ©P\ \exists c_1, …, c_k > 0$ such that $[p < 1] \supset [p_1 < c_1, …, p_k < c_k] \supset [p_1 < c, …, p_k < c] = [\frac{1}{c}\max\{p_1, …, p_k\} < 1]$ ($c = \min\{c_1, …, c_k\}$). Use 2. for seminorms $p, \frac{1}{2\max\{p_1, …, p_k\}}$ and get $p ≤ \frac{1}{c} \max\{p_1, …, p_k\}$.
	\end{dukazin}
\end{tvrzeni}

\subsection{Continuous and bounded linear mapping}
\begin{tvrzeni}
	$(X, τ), (Y, ©U)$ LCS, $L: X \rightarrow Y$ linear. Then the following assertions are equivalent:
	\begin{enumerate}
		\item $L$ is continuous;
		\item $L$ is continuous at ¦o;
		\item $L$ is uniformly continuous.
	\end{enumerate}

	\begin{dukazin}
		„$1. \implies 2.$“ trivial, „$2. \implies 3.$“ assume $L$ continuous at ¦o. Then, given $U \in ©U(¦o)$, there is $V \in τ(¦o)$ such that $L(V) \subset U$. Take $x, y \in X$ such that $x - y \in V$. Then $L(x) - L(y) = L(x - y) \in U$ and that's continuous. „$3. \implies 1.$“ trivial.
	\end{dukazin}
\end{tvrzeni}

\begin{tvrzeni}
	$L: X \rightarrow Y$ linear. $L$ is continuous $\Leftrightarrow$ $\forall q$ a continuous seminorm on $Y$ $\exists p$ a continuous seminorm on $X$: $\forall x \in X: q(L(x)) ≤ p(x)$.

	\begin{dukazin}
		„$\implies$“: $L$ continuous, $q$ a continuous seminorm on $Y$, the $p(x) = q(L(x))$ is a continuous seminorm on $X$. „$\impliedby$“: By the previous proposition it is enough „$L$ is continuous at ¦o“: $U$ neighbourhood of ¦o in $Y$, $\exists V \subset U$ an absolutely convex neighbourhood of ¦o. $q:=p_V$ is a continuous seminorm. Let $p$ be a continuous seminorm on $X$ such that $q ∘ L ≤ p$. $W := [p < 1]$ a neighbourhood of ¦o in $X$ and $L(W) \subset V \subset U$. $x \in W \implies p(x) < 1 \implies q(L(x)) < 1 \implies L(x) \in V \subset U$.
	\end{dukazin}
\end{tvrzeni}

% 17. 10. 2023 from notes of lecturer; and
% 20. 10. 2023 from notes of lecturer

\begin{tvrzeni}
	Let $(X, τ)$ be a LCS over ®F and let $L: X \rightarrow ®F$ be a linear mapping. Then the following assertions are equivalent:
	\begin{itemize}
		\item $L$ is continuous;
		\item $\Ker L$ is a closed subspace of $X$;
		\item there exists $U \in τ(¦o)$ such that $L(U)$ is a bounded subset of ®F.
	\end{itemize}

	If ©P is a family of seminorms generating the topology of $X$, the continuity of $L$ is also equivalent to
	$$ \exists p_1, …, p_k \in ©P\ \exists c > 0\ \forall x \in X: |L(x)| ≤ c·\max\{p_1(x), …, p_k(x)\}. $$

	If $L$ is discontinuous, then $\Ker L$ is a dense subspace of $X$

	\begin{dukazin}
		TODO!!!
	\end{dukazin}
\end{tvrzeni}

\begin{definice}
	Let $(X, τ)$ be a LCS and let $A \subset X$. The set $A$ is said to be bounded in $(X, τ)$, if for any $U \in τ(¦o)$, there exists $λ > 0$ such that $A \subset λ U$.
\end{definice}

\begin{lemma}
	Let $(X, τ)$ be a LCS and let $A \subset X$. Then the set $A$ is bounded in $X$ if and only if each continuous seminorm $p$ on $X$ is bounded on $A$. (It is enough to test it for a family of seminorms generating the topology of $X$.)

	\begin{dukazin}
		TODO? (The proof is not directly needed for the exam.)
	\end{dukazin}
\end{lemma}

\begin{tvrzeni}
	Let $(X, τ)$ and $(Y, ς)$ be LCS over ®F and let $L: X \rightarrow Y$ be a linear mapping. Consider the following two assertions: 1. $L$ is continuous. 2. for any bounded subset $A \subset X$ its image $L(A)$ is bounded in $Y$ (i.e., $L$ is a bounded mapping).

	Then $1. \implies 2.$. In case $τ$ is generated by a translation invariant metric on $X$, then $1. \Leftrightarrow 2.$.

	\begin{dukazin}
		TODO!!!
	\end{dukazin}
\end{tvrzeni}

\begin{definice}[Isomorphism, isomorphic LCS]
	Let $(X, τ)$ and $(Y, ς)$ be LCS over ®F and let $L: X \rightarrow Y$ be a linear mapping. Then mapping $L$ is said to be an isomorphism of $X$ into $Y$, if $L$ is continuous, one-to-one and $L^{-1}$ is continuous on $L(X)$; and isomorphism of $X$ onto $Y$ if $L$ is continuous, one-to-one, onto and $L^{-1}$ is continuous on $Y$.

	The spaces $X$ and $Y$ are said to be isomorphic if there is an isomorphism of $X$ onto $Y$.
\end{definice}

\subsection{Spaces of finite and infinite dimension}
\begin{tvrzeni}
	Let $X$ be a HLCS of finite dimension. If $Y$ is any LCS and $L : X \rightarrow Y$ is any linear mapping, then $L$ is continuous. The space $X$ is isomorphic to $®F^n$, where $n = \dim X$.

	\begin{dukazin}
		If $\dim X = 0$, i.e., $X = \{¦o\}$, it is trivial. Assume $n := \dim X \in ®N$. Fix a basis $x_1, …, x_n$ of $X$. Define $T: (F^n, \|·\|_2) \rightarrow X$ by $T(λ_1, …, λ_n) = λ_1x_1 + … + λ_nx_n$.

		$T$ is clearly a linear bijection of $®F^n$ onto $X$ (since $x_1, …, x_n$ is a basis). „$T$ is continuous“:

		1. The mapping $(λ_1, …, λ_n) \mapsto λ_j$ is continuous $®F^n \rightarrow ®F$ for each $j$. 2. $\forall x \in X$ the mapping $x \mapsto λ·x$ is continuous $®F \rightarrow X$. 3. By composition $(λ_1, …, λ_n) \mapsto λ_j·x_j$ is continuous $®F^n \rightarrow X$.

		Hence $T$ is the sum of $n$ continuous mappings $®F^n \rightarrow X$. It is enough to show that the sum of two continuous mapping is continuous and use mathematical induction: „$Ω$ topological space, $X$ LCS, $f_1, f_2: Ω \rightarrow X$ continuous $\implies$ $f_1 + f_2$ is continuous“: $t \in Ω$ arbitrary, let $G \subset X$ be open such that $f_1(t) + f_2(t) \in G$ $\implies$ $\exists U$ neighbourhood of ¦o: $V + V \subset U$. $f_i$ continuous at $t$ $\implies$ $\exists W_i$ open in $Ω$, $t \in W_i$: $f_i(W_i) \subset f_i(t) + V$. $W := W_1 \cap W_2$ open in $Ω$, $t \in W$. $s \in W \implies f_1(s) + f_2(s) \in f_1(t) + V + f_2(t) + V \subset f_1(t) + f_2(t) + U \subset G$.

		So $T$ is continuous. „$T^{-1}$ is continuous as well“: $S_{F^n}$ the sphere of $®F^n$ is compact in $®F^n$ $\implies$ $T(S_{F^n})$ is compact in $X$. $X$ Hausdorff $\implies$ $T(S_{F^n})$ is closed. Clearly $¦o \notin T(S_{®F^n})$ ($T$ is a linear bijection and $¦o \notin S_{F^n}$) $\implies$ $\exists U$ an absolutely convex neighbourhood of $¦o$ in $X$ such that $U \cap T(S_{®F^n}) = \O$. We clam that $U \subset T(U_{®F^n})$ (open unit ball). Assume $x \in U \setminus T(U_{®F^n})$ $\implies$ $z := T^{-1}(X)$ satisfies $\|z\|_2 ≥ 1$. Then $\frac{z}{\|z\|_2} \in S_{®F^n}$, $T\(\frac{z}{\|z\|_2}\) = \frac{1}{\|z\|_2}·T(z) = \frac{1}{\|z\|_2}·x \in U$ ($U$ is balanced) $\implies$ $\frac{1}{\|z\|_2}·x \in U \cap T(S_{®F^n})$ a contradiction. So $T^{-1}$ is continuous at ¦o ($(T^{-1})^{-1}(U_{®F})$ is a neighbourhood of ¦o and the same for all multiplies).

		$\implies$ $T^{-1}$ is continuous. So, $T$ is an isomorphism and second part is proven.

		By the second part WLOG $X = ®F^n$. Let $L: ®F^n \rightarrow Y$ be linear, $Y$ LCS. Let $e_1, …, e_n$ be the canonical basis of $®F^n$. Then $L(λ_1, …, λ_n) = λ_1 L(e_1) + … + λ_n L(e_n)$. This is continuous (by the same argument as in the second part for $T$).
	\end{dukazin}
\end{tvrzeni}

\begin{dusledek}
	Let $X$ be a HLCS. Then any its finite-dimensional subspace is closed.
\end{dusledek}

\begin{definice}
	Let $(X, τ)$ be a LCS and let $A \subset X$. Then set $A$ is said to be totally bounded (or precompact), if for any $U \in τ(¦o)$ there exists a finite set $F \subset X$ such that $A \subset F + U$.
\end{definice}

\begin{poznamka}
	Any compact set in any LCS is totally bounded. Any totally bounded set is bounded.
\end{poznamka}

\begin{lemma}
	Let $(X, τ)$ be a LCS and let $A \subset X$. Then following assertions are equivalent:
	\begin{enumerate}
		\item $A$ is totally bounded in $X$.
		\item $A$ is totally bounded in $(X, p)$, for any continuous seminorm $p$ on $X$.
		\item For any continuous seminorm $p$ on $X$ and any sequence $(x_n)$ in $A$ there is a subsequence $x_{n_k}$ which is Cauchy with respect to $p$, i.e. $\forall ε > 0\ \exists k_0\ \forall k, l ≥ k_0: p(x_{n_k} - x_{n_l}) < ε$.
	\end{enumerate}

	\begin{dukazin}
		The proof is not needed for the exam.
	\end{dukazin}
\end{lemma}

\begin{veta}
	Let $X$ be a HLCS. Then following assertions are equivalent:
	\begin{enumerate}
		\item $\dim X < ∞$.
		\item There exists a compact neighbourhood of zero in $X$.
		\item There exists a totally bounded neighbourhood of zero in $X$.
	\end{enumerate}

	\begin{dukazin}
		„$1. \implies 2.$“ by the proposition above. „$2. \implies 3.$“ trivial. „$3. \implies 1.$“: Let $U$ be an absolutely convex totally bounded open neighbourhood of $¦o$. Then $\frac{1}{2}U$ is also a neighbourhood of ¦o, so, there is $F \subset X$ finite with $U \subset F + \frac{1}{2}U$. Set $Y := \LO F$. We claim that $Y = X$.

		„$\forall n \in ®N: U \subset Y + 2^{-n}·U$“: By induction: $n = 1$ follows by the previous. $n \mapsto n+1:$ suppose $U \subset Y + 2^{-n}·U$. Then
		$$ U \subset Y + 2^{-n}·U = Y + 2^{-n + 1}·\(\frac{1}{2}U\) \subset Y + 2^{-n + 1}\(\frac{1}{2}\(Y + \frac{1}{2}U\)\) = $$
		$$ = Y + 2^{-n + 1}\(Y + \frac{1}{4}·U\) = Y + 2^{-n+1}·Y + 2^{-n-1}·U = Y + 2^{-n-1}·U. $$
		($Y$ is linear subspace.)

		If $Y ≠ X$, then $\exists x \in X \setminus Y$. Since $U$ is absorbing $\exists t > 0$ such that $tx \in U$. So $U \setminus Y ≠ \O$. Fix $x \in U \setminus Y$. $X$ Hausdorff, $\dim Y < ∞$ $\implies$ $Y$ is closed. Hence there is $V$, an absolutely convex neighbourhood of ¦o such that $x + V \subset U \setminus Y$.

		Since $U$ is totally bounded, it is also bounded, so $\exists n \in ®N: U \subset 2^n·V$, i.e. $\frac{1}{2^n}·U \subset V$. It follows that $x + \frac{1}{2^n}·U \cap Y = \O \implies x \notin Y + \frac{1}{2^n}·U$. So, $x \in U \setminus \(Y + \frac{1}{2^n}·U\)$, a contradiction.
	\end{dukazin}
\end{veta}

\subsection{Metrizability of locally convex spaces}

\begin{tvrzeni}
	1. Let $(X, τ)$ be a metrizable LCS. Then the topology $τ$ is generated by a sequence of seminorms $(p_n)$ satisfying $p_1 ≤ p_2 ≤ p_3 ≤ …$

	2. Let $X$ be a vector space and let $(p_n)$ be a sequence of seminorms on $X$ satisfying conditions: $p_1 ≤ p_2 ≤ p_3 ≤ …$ and $\forall x \in X \setminus \{¦o\}\ \exists n: p_n(x) > 0$. Then
	$$ ρ(x, y) := \sum_{n=1}^∞ \frac{1}{2^n} \min\{1, p_n(x - y)\}, \qquad x, y \in X $$
	is a translation invariant metric on $X$ which generates the locally convex topology on $X$ generated by the sequence of seminorms $(p_n)$. Moreover, given a sequence $(x_k)$ in $X$ we have $ρ(x_k, x) \rightarrow 0 \Leftrightarrow \forall n \in ®N: p_n(x_n - x) \rightarrow 0$; and the sequence $(x_k)$ is Cauchy in $ρ$ if and only if it is Cauchy in each of the seminorms $p_n$.

	Let $(X, τ)$ be a HLCS whose topology is generated by a sequence $(p_n)_{n=1}^∞$ of seminorms.
	\begin{dukazin}[1. WLOG $p_1 ≤ p_2 ≤ p_3 ≤ …$]
		$q_n(x) := \max\{p_1(x), …, p_n(x)\}$ is also a seminorm and the family $(p_n)$ generates the same topology as $(q_n)$.
	\end{dukazin}

	\begin{dukazin}[2. $ρ$ is translation invariant metric]
		$ρ(x, x) = 0$ clear. $x + y \implies \exists n: p_n(x - y) > 0$ as $X$ is Hausdorff. Hence $ρ(x, y) > 0$.

		$ρ(x, y) = ρ(y, x)$ clear, as $p_n(x - y) = p_n(y - x)$. „$ρ(x, z) ≤ ρ(x, y) + ρ(y, z)$“: for each $n \in ®N$: $p_n(x - z) ≤ p_n(x - y) + p_n(y - z)$ and hence also
		$$ \min\(1, p_n(x - z)\) ≤ \min\(1, p_n(x - y)\) + \min\(1, p_n(y - z)\) $$
		$ρ$ translation invariant is clear.
	\end{dukazin}

	\begin{dukazin}[3.]
		„For each $n \in ®N$ and $ε > 0$ we have: $\{x | p_n(x) < ε\} \subset \{x | ρ(x, ¦o) < ε + 2^{-n}\}$“:
		$$ p_n(x) < ε \implies \forall k ≤ n: p_k(x) ≤ p_n(x) < ε, \text{ so} $$
		$$ ρ(x, ¦o) = \sum_{k=1}^∞ \frac{1}{2^k} \min\(1, p_k(x)\) = $$
		$$ = \sum_{k=1}^n \frac{1}{2^k} \max\(1, p_k(x)\) + \sum_{k=n + 1}^∞ \frac{1}{2^k} \min\(1, p_k(x)\) < \sum_{k=1}^n \frac{1}{2^k}·ε + \sum_{k=n+1}^∞ \frac{1}{2^k} < ε + \frac{1}{2^n}. $$
	\end{dukazin}

	\begin{dukazin}[4.]
		„$\forall ε \in (0, 1)\ \forall n \in ®N: \{x | ρ(x, ¦o) < \frac{ε}{2^n}\} \subset \{x | p_n(x) < ε\}$“:
		$$ ρ(x, ¦o) < \frac{ε}{2^n} \implies \sum_{k=1}^∞ \frac{1}{2^k} \min \{1, p_k(x)\} < \frac{ε}{2^n} \implies $$
		$$ \implies \frac{1}{2^n}·\min\(1, p_n(x)\) < \frac{ε}{2^n} \implies \min\(1, p_n(x)\) < ε. $$
		Since $ε < 1$, it follows $p_n(x) < ε$
	\end{dukazin}

	\begin{dukazin}[$ρ$ generates the topology $τ$]
		3. $\implies$ $\forall r > 0$ $\{x | ρ(x, ¦o) < r\}$ is a neighbourhood of ¦o ($r > 0$, fix $ε > 0$ and $n \in ®N$ such that $ε + \frac{1}{2^n} < r$. Then $\{x | ρ(x, ¦o) < r\} \supset \{x | ρ(x, ¦o) < ε + \frac{1}{2^n}\} \supset \{x | p_n(x) < ε\}$).

		4. $\implies$ $\{x | ρ(x, ¦o) < r\}$, $r > 0$, is a base of neighbourhoods of ¦o.

		Hence, the topology generated by $ρ$ has the same neighbourhoods of ¦o as $τ$, so it coincides with $τ$.
	\end{dukazin}

	\begin{dukazin}[6. $ρ(x_k, x) \rightarrow 0 \Leftrightarrow \forall n \in ®N: p_k(x_k - x) \rightarrow 0$]
		„$\implies$“: Assume $ρ(x_k, x) \rightarrow 0$. Fix $n \in ®N$. $ε \in (0, 1)$, $\exists k_0\ \forall k ≥ k_0: ρ(x_k, x) < \frac{ε}{2^n}$ $\overset{4.} \implies \forall k ≥ k_0: p_n(x_n - x) < ε$.

		„$\impliedby$“: Fix $ε > 0$. Find $n \in ®N$ such that $\frac{1}{2^n} < \frac{ε}{2}$. $\exists k_0\ \forall k ≥ k_0: p_n(x_k - x) < \frac{ε}{2}$. By 3. we have for $k ≥ k_0$: $ρ(x_k, x) < \frac{ε}{2} + \frac{1}{2^n} < ε$.
	\end{dukazin}

	\begin{dukazin}[7. $(x_n)$ is $ρ$-Cauchy $\Leftrightarrow$ $\forall n$: $(x_n)$ is $p_n$-Cauchy]
		The proof is the same as in 6. Only we work with $ρ(x_k, x_l)$ for $k, l ≥ k_0$ or $p_n(x_k - x_l)$ for $k, l ≥ k_0$.
	\end{dukazin}
\end{tvrzeni}

\begin{veta}[On metrizability of LCS]
	Let $(X, τ)$ be a HLCS. Then the following assertions are equivalent:

	\begin{enumerate}
		\item $X$ is metrizable (i.e., the topology $τ$ is generated by a metric on $X$).
		\item There exists a translation invariant metric on $X$ generating the topology $τ$.
		\item There exists a countable base of neighbourhoods of ¦o in $(X, τ)$.
		\item The topology $τ$ is generated by a countable family of seminorms.
	\end{enumerate}

% 24. 10. 2023

	\begin{dukazin}
		„$2. \implies 1.$“ trivial. „$1. \implies 3.$“ if $ρ$ a metric generating $τ$, then $U_n = \{x \in X | ρ(x, 0) < \frac{1}{n}\}$ $\implies$ $(U_n)_n$ is a base of neighbourhoods of ¦o. „$3. \implies 4.$“: (see the proof of the previous proposition, 1.) $(U_n)$ base of neighbourhood of ¦o, take $V_n \subset U_n$ absolutely convex neighbourhood of ¦o, $p_n = p_{V_n} \implies (p_n)$ generate $τ$. „$4. \implies 2.$“: the previous proposition 2.
	\end{dukazin}
\end{veta}

\begin{veta}[A characterization of normable LCS]
	$(X, τ)$ is HLCS. $X$ is normable $\Leftrightarrow$ $\exists U$, a bounded neighbourhood of ¦o.

	\begin{dukazin}
		„$\implies$“: $τ$ generated by $\|·\|$, $U := \{x \in X \middle| \|x\| < 1\}$ is a bounded neighbourhood of ¦o.

		„$\impliedby$“: $U$ bounded neighbourhood of ¦o. WLOG $U$ is absolutely convex. Then $\frac{1}{n}U$, $n \in ®N$ is a base of neighbourhoods of ¦o ($V$ neighbourhood of ¦o, $W \subset V$ an absolutely convex neighbourhood of ¦o $\implies$ $\exists λ > 0: U \subset λW$ Take $n \in ®N$ such that $n > λ$. Then $U \subset n·W$ so $\frac{1}{n}U \subset W \subset V$). Finally, $p_U$ is a norm generating the topology ($U$ absolutely convex neighbourhood of ¦o $\implies$ $p_U$ is a continuous seminorm. $\frac{1}{n}U = [p_U < \frac{1}{n}], n \in ®N$ is a base of neighbourhood of ¦o $\implies$ $p_U$ generated topology of $X$. From $X$ Hausdorff, $p_U$ is a norm.)
	\end{dukazin}
\end{veta}

\subsection{Fréchet spaces}
\begin{definice}[Fréchet space]
	A LCS whose topology is generated by a complete translation invariant metric is called Fréchet space.
\end{definice}

%\begin{priklady}
%	$X$ Banach space $\implies$ $X$ Fréchet space. $®F^{®N}, C(®R, ®F), H(Ω)$ are Fréchet spaces.
%
%	\begin{dukazin}[$®F^{®N}$]
%		$$ p_n((x_k)) = \max \{|x_k| \middle| k \in [n]\} $$
%		seminorms generating the topology, $p_1 ≤ p_2 ≤ …$
%		$$ ρ(x, y) = \sum_{n=1}^∞ \frac{1}{2^n} \min\{1, p_n(x - y)\} $$
%		is translation invariant metric generating the topology. It is complete: $\(\(x_k^m\)_k\)_{m=1}^∞$ a $ρ$-Cauchy sequence $\implies$ $\forall n \in ®N: \(\(x_k^m\)\)_m$ is $p_n$-Cauchy $\implies$ it is $\|·\|_∞$-Cauchy in $®F^{®N}$ $\implies$ (because $®F^{®N}$ is complete) $\forall n \in ®N: \(x_k^m\)_{k=1}^n \overset{m \rightarrow ∞}\longrightarrow (y_1^n, …, y_n^n) \in ®F^n$.
%
%		Moreover, if $i ≤ n_1 ≤ n_2$, then $y_i^{n_1} = y_i^{n_2}$. So, we have $y = (y_k)_{k=1}^∞ \in ®F^{®N}$, such that $\forall n \in ®N: (x_k^m)_{k=1}^n \overset{m}\rightarrow (y_k)_{k=1}^n$
%		$$ \implies \forall n \in ®N: p_n(x^n - y) \overset{m}\rightarrow 0 \implies ρ(x^n, y) \rightarrow 0, \text{ i.e. } x^n \rightarrow y \text{ in } X. $$
%	\end{dukazin}
%
%	\begin{dukazin}[$®C(®R, ®F)$]
%		$$ p_n(f) = \max_{x \in [-n, n]} |f(x)|. $$
%		$(f_k)$ $ρ$-Cauchy $\implies$ $\forall n: (f_k)$ is $p_n$-Cauchy $\implies$ $\forall n: (f_k|_{[-n, n]})$ is $\|·\|_∞$-Cauchy in $C([-n, n)$ $\implies$ $\forall n\ \exists g_n \in C([-n, n])$ such that $f_k |_{[-n, n]} \overset{k}\rightarrow g_n$ in $C([-n, n])$.
%
%		$\forall n_1 ≤ n_2: g_{n_2} |_{[-n_1, n_1]} = g_{n_1}$ so, we have one function $g: ®R \rightarrow ®F$ such that $\forall n \in ®N: g|_{[-n, n] = g_n}$. Then $g$ is continuous, i.e. $g \in C(®R, ®F)$ and $\forall n \in ®N: p_n(f_k - g) \overset{k}\rightarrow 0$. So $p_n(f_k, g) \rightarrow 0$ $\implies$ $f_n \rightarrow g$.
%	\end{dukazin}
%\end{priklady}

\begin{tvrzeni}
	$(X, τ)$ is a Fréchet space, $ρ$ any translation invariant metric on $X$ generating $τ$ $\implies$ $ρ$ is complete.

	\begin{dukazin}
		$ρ, d$ two translation invariant metrics generating by $τ$. Idea: convergent sequences with respect to $ρ$ and $d$ coincide, Cauchy sequences with respect to $ρ$ and $d$ coincide. $(x_n)$ $ρ$-Cauchy: $ε > 0 \implies \{x | d(x, ¦o) < ε\}$ is a neighbourhood of ¦o $\implies$ $\exists δ > 0: \{x | ρ(x, ¦o) < δ\} \subset \{x | d(x, ¦o) < ε\}$. $\exists n_0$ $\forall m, n > n_0$:
		$$ ρ(x_m - x_n, ¦o) = ρ(x_m, x_n) < δ \implies d(x_m - x_n, 0) = d(x_m, x_n) < ε \implies (x_n) \text{ is $d$-Cauchy.} $$
	\end{dukazin}
\end{tvrzeni}

\begin{tvrzeni}
	$X$ Fréchet, $A \subset X$. $A$ is compact $\Leftrightarrow$ $A$ is closed and totally bounded.

	\begin{dukazin}
		Let $ρ$ be a complete translation invariant metric generating the topology. $A$ is compact $\Leftrightarrow$ $A$ is closed and $ρ$-totally bounded. But $ρ$-totally boundedness = total boundedness in $X$. $A$ is totally bounded in $X$ $\Leftrightarrow$ $\forall U$ neighbourhood of ¦o $\exists F \subset X$ finite $A \subset F + U$. $A$ is totally bounded in $ρ$ $\Leftrightarrow$ $\forall ε > 0\ \exists F \subset X$ finite such that $A \subset \bigcup_{x \in F} U_ρ(x, ε) = F + U_ρ(0, ε)$.
	\end{dukazin}
\end{tvrzeni}

\begin{tvrzeni}
	$X$ LCS, $A \subset X$ totally bounded $\implies$ $\aco A$ is totally bounded.

	\begin{dukazin}
		Let $U$ be a neighbourhood of ¦o. Let $V$ be an absolutely convex neighbourhood of ¦o such that $2V \subset U$ $\implies$ $\exists F \subset X$ finite such that $A \subset F + V$. Then clearly $\aco A \subset (\aco F) + V$. $\aco F$ is compact, $F = \{x_1, …, x_k\} \implies \aco(F) = \co(\b(F)) =$
		$$ = \co\{λ x_j | j \in [k], |λ| ≤ 1\} = \{t_1λ_1x_1 + … t_n λ_n x_n \middle| |λ_j| ≤ 1, t_j ≥ 0, \sum t_j = 1\}. $$

		$$ B = \{(λ_1, …, λ_n, t_1, …, t_n) \middle| |λ_j| ≤ 1, t_j ≥ 0, \sum t_j = 1\} $$
		a compact set in $®F^n \times ®R^n$. $(λ_1, …, λ_n, t_1, …, t_n) \mapsto t_1λ_1x_1 + … + t_nλ_nx_n$ is a continuous map and maps $B$ onto $\aco F$.

		$\aco F$ compact $\implies$ totally bounded $\implies$ $\exists H \subset X$ finite, $\aco F \subset H + V$ So $\aco A \subset \aco F + V \subset H + V + V = H + 2V < H + U$.
	\end{dukazin}
\end{tvrzeni}

\begin{dusledek}
	$X$ Fréchet space, $A \subset X$ compact $\implies$ $\overline{\aco A}$ is compact.

	\begin{dukazin}
		$A$ compact $\implies$ $A$ is totally bounded $\implies$ $\aco A$ is totally bounded $\implies$ (because $M \subset X$ any set $\implies$ $\overline{M} \subset M + U$) $\overline{\aco A}$ is totally bounded $\implies$ $\overline{\aco A}$ is compact.

		($M$ totally bounded, for any $U \in τ(¦o)$: $U$ is neighbourhood of ¦o, $x \in \overline{M}$, $U$ absolutely convex convex neighbourhood of ¦o, then $V \subseteq U$ absolutely convex such that $2V \subset U$ $\implies$ $(x + U) \cap M ≠ 0 \implies x \in M + U$.)

		Find $F$ finite such that $M \subset F + V$ $\implies$ $\overline{M} \subset M + V \subset F + V + V \subset F + U$.
	\end{dukazin}
\end{dusledek}

\begin{veta}[Banach–Steinhaus]
	Let $X$ be a Fréchet space and let $Y$ be LCS. Let $(T_n)$ be a sequence of continuous linear mappings $T_n: X \rightarrow Y$ such that $\forall x \in X: \lim_{n \rightarrow ∞} T_n x$ exists in $Y$. Then $Tx := \lim_{n \rightarrow ∞} T_n x$ define a continuous linear map $X \rightarrow Y$.

	\begin{dukazin}
		Clear: $T$ is a linear map $X \rightarrow Y$. „Continuous“: Fix $q$ any continuous sequence on $Y$.
		$$ A_m = \{x \in X | \forall n \in ®N: q(T_n x) ≤ m\}. $$
		Then $A_m$ is closed, absolutely convex and $\bigcup_{m=1}^∞ A_m = X$.

TODO?

% 27. 10. 2023

	Baire category theorem $\implies$ $\exists m \in ®N: \Int A_m ≠ \O$ $\implies$ $\exists x \in A_m\ \exists U$ an absolutely convex neighbourhood of ¦o such that $x + U \subset A_m$ $\implies$ $-(x + U) \subset A_m$ $\implies$ ($A_m$ convex) $U \subset A_m$ ($y \in U \implies y = \frac{1}{2}(x + y + (-x + y))$) „$\implies$ $q(Ty) ≤ m p_U(y)$“:

	$$ p_U(y) < c \implies \frac{y}{c} \in U \subset A_m \implies \forall n \in ®N: q\(T_n \frac{y}{c}\) ≤ m \implies $$
	$$ \implies q\(T \frac{y}{c}\) ≤ m \implies q\(T y\) ≤ c·m. $$
	\end{dukazin}
\end{veta}

\begin{veta}[Open mapping theorem]
	$X$, $Y$ Fréchet, $T: X \rightarrow Y$ linear continuous surjective mapping. Then $T$ is an open mapping

	\begin{dukazin}
		1. It is enough to show that $\forall U$ neighbourhood of ¦o in $X$: $T(U)$ is a neighbourhood of ¦o in $Y$.

		2. „$\forall U$ a neighbourhood of ¦o in $X$, $\overline{TU}$ is neighbourhood of ¦o in $Y$“: $U$ an neighbourhood of ¦o in $X$. $\exists V \subset U$ an absolutely convex neighbourhood of ¦o. $V$ absorbing $\implies$
		$$ \implies X = \bigcup_{n=1}^∞ n V \implies Y = T(X) = T\(\bigcup_{n=1}^∞ n·V\) = \bigcup_{n=1}^∞ n·T(V). $$
		$Y$ Fréchet $\implies$ by Baire category theorem
		$$ \exists n \in ®N: \O ≠ \Int \overline{n·T(V)} = \Int n·\overline{T(V)} = n·\Int \overline{T(V)} \implies \Int \overline{T(V)} ≠ \O \implies $$
		$\implies$ $\exists y \in Y\ \exists W$ an absolutely convex neighbourhood of ¦o in $Y$ such that $y + W \subset \overline{T(V)}$. $\overline{T(V)}$ is absolutely convex $\implies$ $-(y + w) \subset \overline{T(V)} \implies W \subset \overline{T(V)} \subset \overline{T(U)}$.

		3. „$\forall U$ neighbourhood of ¦o in $X$, $TU$ is a neighbourhood of ¦o in $Y$“: $ρ$ a translation invariant metric on $X$, complete, generating topology. $U_n = \{x \in X | ρ(0, x) < \frac{1}{2^n}\}$. The $U_n$ is a base of neighbourhoods of ¦o. It is enough to prove that $T(U_n)$ is a neighbourhood of ¦o for each $n \in ®N$. We know from 2. that $\forall n: \overline{TU_n}$ is a neighbourhood of ¦o in $Y$. We will be done if we show that $TU_{n-1} \supset \overline{TU_n}$ for each $n \in ®N$.

		We will prove it for $n = 1$: So we will ? $TU_1 \subset TU_0$. Fix $y \in \overline{TU_1}$. Since $\overline{TU_2}$ is a neighbourhood of ¦o $(y - \overline{TU_2}) \cap TU_1 ≠ \O$. So there is $x_1 \in U_1$ such that $y - Tx_1 \in \overline{TU_2}$. $\overline{TU_3}$ is a neighbourhood of ¦o in $Y$ $\implies$ $y - Tx_1 - \overline{TU_3} \subset ap TU_2 = \O$ so, there is $x_2 \in U_2$ such that $y - Tx_1 - Tx_2 \in \overline{TU_3}$.

		By induction we find $x_n \in U_n$ such that $y - Tx_1 - Tx_2 - … - Tx_n \in \overline{T U_{n+1}} \quad (n \in ®N)$.
		$$ x := \sum_{n=1}^∞ x_n = \lim_{N \rightarrow ∞} \sum_{n=1}^N x_n: $$
		$$ M > N \implies ρ\(\sum_{n=1}^M x_n, \sum_{n=1}^N x_n\) = ρ\(\sum_{n = N+1}^M x_n, ¦o\) ≤ $$
		$$ ≤ \underbrace{ρ\(\sum_{n=N+1}^M x_n, \sum_{n=N+2}^M\)}_{ρ(x_{N+1}, ¦o)} + \underbrace{ρ\(\sum_{n=N+2}^M x_n, \sum_{n=N+3}^M\)}_{ρ(x_{N+2}, ¦o)} + … + ρ(x_{M}, ¦o). $$

		$$ Tx = y: y - Tx = \lim_{n \rightarrow ∞} (y - Tx_1 - … - Tx_n) $$
		$$ y - Tx_1 - … - Tx_n \in \overline{TU_{N+1}} \subset \overline{TU_k} \qquad \text{for $n+1 > k$} $$
		so, $y - Tx \in \overline{TU_k}$ for each $k \in ®N$, so $y - Tx \in \bigcap_{k=1}^∞ \overline{TU_k} = \{¦o\}$. „Last equality“: $y \in Y \setminus \{¦o\}$ $\implies$ $\exists V$ neighbourhood of ¦o in $Y$ such that $y \notin \overline{B}$. $T$ continuous $\implies$ $\exists k \in ®N$ such that $T(U_k) \subset V$ $\implies$ $\overline{T(U_1)} \subset \overline{V}$ $\implies$ $y \notin \overline{T(U_k)}$.
	\end{dukazin}
\end{veta}

\subsection{Extension and separation theorems}
\begin{definice}[Dual space (the dual)]
	$X$ LCS, $X^*$ is the vector space of continuous linear functions on $X$.
\end{definice}

\begin{veta}[Hahn–Banach extension theorem]
	$X$ LCS, $Y \Subset X$, $f \in Y^*$. Then $\exists g \in X^*$ such that $g|_Y = f$.

	\begin{poznamkain}
		If topology of $X$ is generated by ©P a topology of seminorms TODO?
	\end{poznamkain}

	\begin{dukazin}
		1. Topology of $Y$: $U \subset Y$ is open $\Leftrightarrow$ $\exists \tilde U \subset X$ open such that $U = \tilde U \cap Y$. $U$ is a neighbourhood of ¦o in $Y$ $\Leftrightarrow$ $\exists \tilde U$ a neighbourhood of ¦o in $X$ such that $U = \tilde U \cap Y$. Lz.pat. $Y$ is also a LSC.

		2. $f \in Y^*$ $\implies$ $\exists p$ a continuous seminorm on $Y$ such that $|f(y)| \subseteq p(y), y \in Y$. $U = [p < 1]$ a neighbourhood of ¦o in $Y$ $\implies$ $\exists \tilde U$ a neighbourhood of ¦o in $X$ such that $\tilde U \cap Y = U$ $\implies$ $\exists \tilde V \subset \tilde U$ an absolutely convex neighbourhood of ¦o in $X$. The $p_{\tilde V}$ is a continuous seminorm on $X$. Moreover, $p_{\tilde V}|_Y ≥ p$. ($[p_{\tilde V}|_Y < 1] \subset \tilde V \cap Y \subset U = [p < 1]$). So, for $y \in Y: |f(y)| ≤ p(y) ≤ p_{\tilde V}(y)$ $\implies$ (algebraic H–B for seminorms) $\exists g: X \rightarrow ®F$ linear, $g|_Y = f$, $|g(x)| ≤ p_{\tilde V}(x)$ for $x \in X$ $\implies$ $g$ is continuous by the proposition above.
	\end{dukazin}
\end{veta}

\begin{dusledek}[Separation from a subspace]
	$X$ LCS, $Y \Subset X$ closed, $x \in X \setminus Y$. Then $\exists f \in X^*: f|_Y = 0, f(x) = 1$.

	\begin{dukazin}
		Set $\tilde Y = \LO(Y \cup \{x\})$. Define $g(y  + λx) = λ$, $y \in Y$, $λ \in ®F$ $\implies$ $g$ is linear functional on $\tilde Y$, $g|_Y = 0$, $g(x) = 1$. $\Ker g = Y$ is closed $\implies$ $g$ is continuous $\implies$ $g$ can be extended to $f \in X^*$.
	\end{dukazin}
\end{dusledek}

\begin{dusledek}[A proof of density using Hahn–Banach theorem]
	$X$ LCS, $Z \Subset Y \Subset X$.
	$$ \overline{Z} \supset Y \Leftrightarrow \forall f \in X^*: f|_Z = 0 \implies f|_Y = 0. $$

	\begin{dukazin}
		„$\implies$“: clear. „$\impliedby$“: $y \in Y \setminus \overline{Z}$ $\implies$ $\exists f \in X^*: f(y) = 1, f|_Z = 0$.
	\end{dukazin}
\end{dusledek}

\begin{dusledek}[The dual separates the points]
	$X$ HLCS, $x \in X \setminus \{¦o\} \implies \exists f \in X^*: f(x) ≠ 0$.

	\begin{dukazin}
		$Y = \{¦o\}$ is closed linear subspace and use the first corollary.
	\end{dukazin}
\end{dusledek}

% 31. 10. 2023

\begin{veta}[Hahn–Banach separation theorem]
	$X$ LCS, $A, B \subset X$ nonempty convex, $A \cap B = \O$.

	\begin{itemize}
		\item $\Int A ≠ \O \implies \exists f \in X^* \setminus \{0\}\ \exists c \in ®R\ \forall a \in A\ \forall b \in B: \Re f(a) ≤ c < \Re f(s)$.
		\item $A$ compact, $B$ closed $\implies$ $\exists f \in X^*\ \exists c, d \in ®R\ \forall a \in A\ \forall b \in B: \Re f(a) ≤ c < d ≤ \Re f(b)$.
	\end{itemize}

	\begin{dukazin}
		Analogous to the theorem above. Assume $X$ is a real space $(®F = ®R)$. „First item“: $\Int A ≠ \O$ $\implies$ $\Int(B - A) ≠ \O$ and $¦o \notin B - A$. Fix $z \in \Int(B - A)$, set $U := z - (B - A)$. The $U$ is a convex neighbourhood of ¦o, $z \notin U \implies p_U(z) ≥ 1$. Define $g_0: \LO \{z\} \rightarrow ®R$ by $g_0(t·z) = t·p_U(z)$ $\implies$ $g_0$ is a linear functional, $g_0 ≤ p_U$ on $\LO\{z\}$ ($t ≥ 0 \implies g_0(t·z) = t·p_U(z) = p_U(t·z)$, $t < 0 \implies g_0(t·z) =t·p_U(z) < 0 ≤ p_U(t·z)$).

		From algebraic Hahn–Banach $\exists g: X \rightarrow ®R$ linear, $g|_{\LO\{z\}} = g_0$, $g ≤ p_U$ on $X$. $g$ is continuous ($g ≤ 1$ on $U$ $\implies$ $g ≥ -1$ on $-U$, so $|g| ≤ 1$ on $U \cap (-U)$, a neighbourhood of ¦o). $a \in A$, $b \in B$ $\implies$
		$$ \implies g(z) - g(b) + g(a) = g(z - (b - a)) ≤ p_U(z - (b - a)) ≤ 1, \quad g(a) ≤ g(b) + \underbrace{1 - \overbrace{g(z)}^{= p_U(z) ≥ 1}}_{≤ 0}. $$
		So, $\forall a \in A\ \forall b \in B: g(a) ≤ g(b)$, $c := \sup g(A)$.

		„Second item“: $A$ compact, $B$ closed. For $x \in A$ $\exists U_x$ an absolutely convex open neighbourhood of ¦o such that $(x + U_x) \cap B = \O$. The $(x + \frac{1}{2} U_x)_{x \in A}$, is an open cover of $A$. $A$ is compact $\implies$ $\exists x_1, …, x_n \in A: A \subset \(x_1 + \frac{1}{2}U_{x_1}\) \cup … \cup \(x_n + \frac{1}{2}U_{x_n}\)$. Set $V := \frac{1}{2}U_{x_1} \cap … \cap \frac{1}{2}U_{x_n}$ an absolutely convex open neighbourhood of ¦o. Then $(A + V) \cap B = \O$
		$$ \(a \in A \implies \exists j: a \in x_j + \frac{1}{2}U_{x_j} \implies a + V \subset x_j + \frac{1}{2}U_{x_j} + V \subset x_j + U_{x_j}\). $$
		Apply first item to $A + V$ (open convex), $B$ (convex) $\implies$ $\exists f \in X^* \setminus \{0\}$ such that
		$$ \sup f(A) + \sup f(V) = \sup(f(A) + f(V)) = \sup f(A + V) ≤ \inf f(B), $$
		observe that $\sup f(V) > 0$ ($f ≠ 0$, $V$ is neighbourhood of ¦o, hence absorbing).
		$$ c := \sup f(A), \qquad d := \sup f(A) + \sup f(V). \vspace{-1em} $$

		„$X$ complex“: look at $X$ as a real space, $f: X \rightarrow ®R$ real-linear such that. Define $f_c(x) = f(x) - if(ix), x \in X$.
	\end{dukazin}
\end{veta}

\begin{dusledek}
	$X$ LCS, $\O ≠ A \subset X$, $x \in X$.\vspace{-1.5em}
	\begin{itemize}
		\item $x \in X \setminus \overline{\co} A \Leftrightarrow \exists f \in X^*: \Re f(x) > \sup\{\Re f(a) | a \in A\}$. („$\impliedby$“: is clear because $\{y \in X, \Re f(y) ≤ \sup \{\Re f(x) | a \in A\}\}$ is closed convex set containing $A$. „$\implies$“: Apply the previous theorem to $\{x\}$ and $\overline{\co} A$, get $f$ and take $-f$.)
		\item $x \in X \setminus \overline{\aco} A \Leftrightarrow \exists f \in X^*: |f(x)| > \sup \{|f(a)| \middle| a \in A\}$ („$\impliedby$“: Clear. „$\implies$“: Apply the previous theorem to $\{x\}$ and $\overline{\aco} A$ (and multiply by $-1$), $f \in X^*$: $|f(x)| ≥ \Re f(x) > \sup \{\Re f(y) | y \in \overline{\aco} A\} = \sup\{|f(y)| \middle| y \in \overline{\aco} A\}$. „$≤$“ clear, „$≥$“: $y \in \overline{\aco A} \implies \exists α \in ®F, |α| = 1: |f(y)| = αf(y)$, then $|f(y)| = λ f(y) = \Re α f(y) = \Re f(α y)$, $α y \in \overline{\aco A}$).
	\end{itemize}
\end{dusledek}

\section{Weak topologies}
\subsection{General weak topologies and duality}
\begin{definice}[Algebraic dual, general weak topology]
	$X$ vector space. $X^{\#}$ is the algebraic dual of $X$ (it means set of all linear functionals on $X$). $\O ≠ M \subset X^{\#}$, then $ς(X, M)$ is the topology on $X$ generated by seminorms $X \mapsto |f(x)|$, $f \in M$.
\end{definice}

\begin{tvrzeni}
	Properties:
	\begin{enumerate}
		\item $(X, ς(X, M))$ is LCS (by the theorem above).
		\item $(X, ς(X, M))$ is Hausdorff $\Leftrightarrow$ $\forall x \in X \setminus \{0\}\ \exists f \in M: f(x) ≠ 0$ (i.e. $M$ separates points) (by the theorem above).
		\item $f \in M$ $\implies$ $f$ is continuous on $(X, ς(X, M))$ (fix $f \in M$, $p(x) = |f(x)|$, $x \in X$ is a continuous seminorm and $|f(x)| = p(x) ≤ p(x)$).
		\item $ς(X, M)$ is the weakest topology on $X$ making all $f \in M$ continuous.
		\item $ς(X, M) = ς(X, \LO(M))$.
		\item $T$ a topological space, $F: T \rightarrow X$ mapping. Then $F$ is continuous $T \rightarrow ς(X, M)$ $\Leftrightarrow$ $\forall f \in M: f ∘ F$ is continuous ($T \rightarrow ®F$).
	\end{enumerate}

	\begin{dukazin}[4.]
		Assume $τ$ is any topology on $X$ such that all $f \in M$ are $τ$-continuous $\implies$
		$$ \forall x \in X\ \forall f_1, …, f_n \in M\ \forall c_1, …, c_n > 0: \{y \in X \middle| |f_j(y) - f_j(x)| < c_j\ \forall j \in [n]\} \text{ is $τ$-open} $$
		but these sets form a base of $ς(X, M)$ $\implies$ $ς(X, M) \subset τ$.
	\end{dukazin}

	\begin{dukazin}[5.]
		„$\subseteq$“: Clear. „$\supseteq$“: $f \in \LO M$ $\implies$ $f$ is $ς(X, M)$-continuous (the linear combination of continuous linear functionals is continuous) $f = α_1 f_1 + … + α_n f_n$, $f_1, …, f_n \in M$, $x_1, …, x_n \in ®F$.
		$$ |f(x)| ≤ |α_1|·|f_1(x)| + … + |α_n|·|f_n(x)| ≤ (|α_1| + … + |α_n|)·\max\{|f_1(x)|, …, |f_n(x)|\}. $$
		So by the previous point we get $ς(X, \LO M) \subset ς(X, M)$.
	\end{dukazin}

	\begin{dukazin}[6.]
		„$\implies$“: $f \in M$ $\implies$ $f$ is $ς(X, M)$-continuous, so $f ∘ F$ is continuous. „$\impliedby$“: $t \in T$, $U$ neighbourhood of $F(t)$ in $ς(X, M)$ $\implies$ $\exists f_1, …, f_n \in M\ \exists c_1, …, c_n > 0$ such that
		$$ F(t) \in \{y \in X \middle| \forall j \in [n] |f_j(y) - f_j(F(t)) < c_j\} \subset U. $$
		Let $G = \{u \in T\middle| \forall j \in [n]: |(f_j ∘ F)(u) - (f_j ∘ F)(t)| < c_j\}$. Then $G$ is an open neighbourhood of $t$ (by continuity of $f_j ∘ F$ and $F(G) \subset U$).
	\end{dukazin}
\end{tvrzeni}

\begin{priklad}
	$X$ LCS. Then $X^* \Subset X^{\#}$. So, we may consider $ς(X, X^*)$ „the weak topology of $X$“. $ς(X, X^*)$ is Hausdorff when $X$ is HLCS.

	TODO?
\end{priklad}

% ?????????? (many lectures) from notes of lecturer

\begin{lemma}
	Let $X$ be a vector space and $f, f_1, …, f_k \in X^{\#}$. The following assertions are equivalent:
	\begin{enumerate}
		\item $f \in \LO\{f_1, …, f_k\}$;
		\item $\exists C > 0\ \forall x \in X: |f(x)| ≤ C · \max\{|f_1(x)|, …, |f_k(x)|\}$;
		\item $\bigcap_{j=1}^k \Ker f_j \subset \Ker f$.
	\end{enumerate}

	\begin{dukazin}
		TODO!!!
	\end{dukazin}
\end{lemma}

\begin{veta}
	Let $X$ be a vector space and let $M \subset X^{\#}$ be a nonempty set. Then $(X, ς(X, M))^* = \LO M$.

	\begin{dukazin}
		TODO!!!
	\end{dukazin}
\end{veta}

\begin{dusledek}
	\ \vspace{-2em}
	\begin{enumerate}
		\item Let $X$ be a LCS and let $f \in X^{\#}$. Then $f$ is continuous on $X$ (i.e., $f \in X^*$), if and only if it is weakly continuous (i.e., $ς(X, X^*)$-continuous) on $X$.
		\item Let $X$ be a LCS. Then $(X^*, ς(X^*, X))^* = κ(X)$.
		\item Let $X$ be a normed linear space and let $f \in X^{**}$. Then $f \in κ(X)$, if and only if $f$ is weak$^*$ continuous (i.e., $ς(X^*, X)$ continuous) on $X^*$.
	\end{enumerate}

	\begin{dukazin}
		The proof is not needed for the exam.
	\end{dukazin}
\end{dusledek}

\subsection{Weak topologies on locally convex spaces}
\begin{veta}[Mazur theorem]
	Let $X$ be a LCS and let $A \subset X$ be a convex set. Then a) $\overline{A}^w = \overline{A}$ b) $A$ is closed if and only if it is weakly closed.

	\begin{dukazin}
		TODO!!!
	\end{dukazin}
\end{veta}

\begin{dusledek}
	Let $X$ be a metrizable LCS and let $(x_n)$ be a sequence in $X$ weakly converging to a point $x \in X$. Then there is a sequence $(y_n)$ in $X$ such that $y_n \in \co\{x_k, k ≥ n\}$ for each $n \in ®N$ and $y_n \rightarrow x$ in (the original topology of) $X$.

	\begin{dukazin}
		TODO!!!
	\end{dukazin}
\end{dusledek}

\begin{veta}[Boundedness and weak boundedness]
	Let $X$ be a LCS and let $A \subset X$. Then $A$ is bounded in $X$ if and only if it is bounded in $ς(X, X^*)$.

	\begin{dukazin}
		„$\implies$“: trivial. „$\impliedby$“: 1. $A$ is $ς(X, X^*)$-bounded $\implies$ $\forall f \in X^*$: $f$ is bounded on $A$. Therefore, in case $X$ is a normed space, this theorem is a corollary to uniform boundedness principle.

		2. Let $A \subset X$ be $ς(X, X^*)$-bounded. To prove $A$ is bounded, it is enough to show that any continuous seminorm on $X$ is bounded on $A$ (lemma above). So, let $p$ be any continuous seminorm on $X$.

		Set $Y := \{x \in X | p(x) = 0\}$. Then $Y$ is a closed linear subspace of $Y$ ($Y$ is closed as $p$ is continuous, $¦o \in Y$ as $p(¦o) = 0$, $x \in Y, λ \in ®F \implies p(λx) = |x|·p(x) = 0 \implies λ x \in Y$, $x, y \in Y \implies 0 ≤ p(x + y) ≤ p(x) + p(y) = 0 \implies x + y \in Y$).

		Let $X / Y$ be the quotient in the linearly-algebraic sense and $q: X \rightarrow X / Y$ be the canonical quotient map.

		Define a norm $\|·\|$ on $X / Y$ by $\|q(x)\| = p(x)$, $x \in X$. „Well defined“: $q(x) = q(y) \implies q(x - y) = 0 \implies x - y \in Y$, hence $p(x - y) = 0$, so,
		$$ p(x) ≤ p(x - y) + p(y) = p(y) \land p(y) ≤ p(y - x) + p(x) = p(x) \implies p(x) = p(y). $$
		„It is a norm“:
		$$ \|¦o\| = \|q(¦o)\| = p(¦o) = 0; $$
		$$ \|q(x)\| = 0 \implies p(x) = 0 \implies x \in Y \implies q(x) = ¦o; $$
		$$ \|λ·q(x)\| = \|q(λ x)\| = p(λ x) = |λ| p(x) = |λ|·\|q(x)\|; $$
		$$ \|q(x) + q(y)\| = \|q(x + y)\| = p(x + y) ≤ p(x) + p(y) = \|q(x)\| + \|q(y)\|. $$

		$q$ is continuous $X \rightarrow (X / Y, \|·\|)$ as $\|q(x)\| ≤ p(x)$ (in fact equal) and $p$ is a continuous seminorm on $X$. Further „the set $q(A)$ is weakly bounded in $(X / Y, \|·\|)$“:
		$$ f \in (X / Y)^* \implies f ∘ q \in X^* \implies (f ∘ q)(A) \text{ is bounded}, $$
		and, finally, $f(q(A)) = (f ∘ q)(A)$.

		So, by q. (the case of normed spaces), $q(A)$ is norm-bounded on $X / Y$, i.e., $\exists c > 0$ $\forall x \in A: \|q(x)\| ≤ c$. But $\|q(x)\| = p(x)$. Hence, $p ≤ C$ on $A$ and the proof is finished.
	\end{dukazin}
\end{veta}

\begin{tvrzeni}[Weak topology on a subspace]
	Let $X$ be a LCS and let $Y \Subset X$. Then the weak topology $ς(Y, Y^*)$ coincides with the restriction of the weak topology $ς(X, X^*)$ to $Y$.

	\begin{dukazin}
		TODO? (The proof is not directly needed for the exam.)
	\end{dukazin}
\end{tvrzeni}

\subsection{Polars and their applications}
\begin{definice}[Polars, absolute polars and anihilators]
	Let $X$ be a LCS. Let $A \subset X$ and $B \subset X^*$ be nonempty sets. We define
	\begin{alignat*}{4}
		A^\triangleright &:= \{f \in X^* | \forall x \in A: \Re f(x) ≤ 1\}, \qquad & B_\triangleright &:= \{x \in X | \forall f \in B: \Re f(x) ≤ 1\},\\
		A^∘ &:= \{f \in X^* | \forall x \in A: |f(x)| ≤ 1\}, \qquad & B_∘ &:= \{x \in X | \forall f \in B: |f(x)| ≤ 1\},\\
		A^\perp &:= \{f \in X^* | \forall x \in A: f(x) = 0\}, \qquad & B_\perp &:= \{x \in X | \forall f \in B: f(x) = 0\}.
	\end{alignat*}
\end{definice}

\begin{priklad}
	Let $X$ be a normed linear space. Then $(B_X)^\triangleright = (B_X)^∘ = B_{X^*}$, $(B_{X^*})_\triangleright = (B_{X^*})_∘ = B_X$.
\end{priklad}

\begin{tvrzeni}[Polar calculus]
	Let $X$ be a LCS and let $A \subset X$ be a nonempty set.
	\begin{enumerate}
		\item The set $A^\triangleright$ is convex and contains the zero functional, $A^∘$ is absolutely convex and $A^\perp$ is a subspace of $X^*$. All the three sets are moreover weak$^*$ closed.
		\item $A^\perp \subset A^∘ \subset A^\triangleright$.
		\item If $A$ is balanced, then $A^\triangleright = A^∘$. If $A \Subset X$, then $A^\triangleright = A^∘ = A^\perp$.
		\item $\{¦o\}^\triangleright = \{¦o\}^∘ = \{¦o\}^\perp = X^*$, $X^\triangleright = X^∘ = X^\perp = \{¦o\}$.
		\item $(c·A)^\triangleright = \frac{1}{c}·A^\triangleright$ and $(c·A)^∘ = \frac{1}{c}·A^∘$ whenever $c > 0$.
		\item Let $(A_i)_{i \in I}$ be a nonempty family of nonempty subsets of $X$. Then $\(\bigcup_{i \in I} A_i\)^∘ = \bigcap_{i \in I} A_i^∘$. The analogous formulas hold for polars and anihilators too.
	\end{enumerate}

	\begin{poznamkain}
		Analogous statements hold for $B \subset X^*$. There are just two differences: The sets $B_\triangleright$, $B_∘$ and $B_\perp$ are weakly closed and for the validity of the second statement in 4. one needs to assume that $X$ is Hausdorff.
	\end{poznamkain}

	\begin{dukazin}
		TODO? (The proof is not directly needed for the exam.)
	\end{dukazin}
\end{tvrzeni}

\begin{veta}[Bipolar theorem]
	Let $X$ be a LCS and let $A \subset X$ and $B \subset X^*$ be nonempty sets. Then
	\begin{alignat*}{4}
		(A^\triangleright)_\triangleright &= \overline{\co}(A \cup \{¦o\}) \(= \overline{\co}^w(A \cup \{¦o\})\), \quad & (B_\triangleright)^\triangleright &= \overline{\co}^w(B \cup \{¦o\}),\\
		(A^∘)_∘ &= \overline{\aco} A \(= \overline{\aco}^w A\), \qquad & (B_∘)^∘ &= \overline{\aco}^w B,\\
		(A^\perp)_\perp &= \overline{\LO} A \(= \overline{\LO}^w A\), \qquad & (B_\perp)^\perp &= \overline{\LO}^w B.
	\end{alignat*}

	\begin{dukazin}
		TODO!!!
	\end{dukazin}
\end{veta}

\begin{dusledek}
	Let $X$ and $Y$ be normed linear spaces and let $T \in L(X, Y)$. Then $(\Ker T)^\perp = \overline{T^*(Y^*)}^{w^*}$.

	\begin{dukazin}
		The proof is not needed for the exam.
	\end{dukazin}
\end{dusledek}

\begin{veta}[Goldstine]
	Let $X$ be a normed linear space and let $κ: X \rightarrow X^{**}$ be the canonical embedding. Then $B_{X^{**}} = \overline{κ(B_X)}^{ς(X^{**}, X^*)}$.

	\begin{dukazin}
		TODO!!!
	\end{dukazin}
\end{veta}

\begin{veta}[Banach–Alaoglu]
	Let $X$ be a LCS and let $U \subset X$ be a neighbourhood of ¦o. Then
	\begin{enumerate}
		\item $U^∘$ is a weak$^*$ compact subset of $X^*$.
		\item If $X$ is moreover separable, $U^∘$ is metrizable in the topology $ς(X^*, X)$.
	\end{enumerate}

	\begin{dukazin}[1.]
		Consider $T: X^* \rightarrow ®F^U$ defined by $T(f)(x) = f(x)$, $f \in X^*$, $x \in U$, i.e. $T(f) = f|_U$. Then $T$ is a homeomorphism of $(X^*, w^*)$ into $®F^U$: „$T$ is one-to-one“: $T(x) = T(y) \implies f|_U = g|_U$. Since $f, g$ are linear and $U$ is absorbing, necessarily $f = g$.

		„$T$ is continuous (on $®F^U$ we consider the topology of point-wise convergence)“: $x \in U$ fixed $\implies$ $f \mapsto T(f)(x) = f(x)$ is $w^*$-continuous by definition of the $w^*$-topology. 
		„$T^{-1}$ is continuous on $T(X^*)$“: Fix $x \in X$. Since $U$ is absorbing, there is $t > 0$ with $t·x \in U$. If $g = T(f) \in T(X^*)$, then
		$$ T^{-1}(g)(x) = f(x) = \frac{1}{t} f(t·x) = \frac{1}{t} g(t·x), $$
		so $g \mapsto T^{-1}(g)(x)$ is continuous.

		„Moreover $T(U^∘) =$
		$$ \!\!\!\!\!\!\! \{\!F \in ®F^U\! \middle| \forall x\!\in\!U\!\!: |F(x)| ≤ 1 \!\land\! \forall α, β\!\in\!®F\ \forall x,\!y\!\in\!U\!\!: αx \!+\! βy \in U \!\!\implies\!\! F(αx \!+\! βy) = αF(x) \!+\! βF(y)\!\} \!\!\!\!\!\!\! $$
		“: „$\subset$“ is clear, „$\supset$“: Let $F$ be in the set on the RHS. We will define $f: X \rightarrow ®F$ as follows: Let $x \in X$. Find $α > 0$ such that $α·x \in U$ and set $f(x) = \frac{1}{α} F(α·x)$. $F(¦o) = 0$ ($F(¦o + ¦o) = F(¦o) + F(¦o)$).

		„$f$ is well defined“: $x \in X$, $α, β > 0$, $α·x, β·x \in U$. Then $\frac{1}{α}(α·x) - \frac{1}{β}(β·x) = ¦o \in U$. So,
		$$ 0 = F(0) = F\(\frac{1}{α}(α·x) - \frac{1}{β}(β·x)\) = \frac{1}{α}F(α·x) - \frac{1}{β}F(β·x), $$
		hence $\frac{1}{α} F(α·x) = \frac{1}{β} F(β·x)$.

		„$f$ is linear“: $x, y \in X$, $α, β \in ®F$. $U$ absorbing $\implies$ $\exists t > 0: t·x, t·y, t·(α·x + β·y) \in U$. Then
		$$ f(α·x + β·y) = \frac{1}{t} F(t·(α·x + β·y)) = \frac{1}{t} F(α·(t·x) + β·(t·y)) = $$
		$$ = \frac{1}{t} \(α·F(t·x) + β·F(t·y)\) = α·\frac{1}{t}·F(t·x) + β·\frac{1}{t}·F(t·y) = α·f(x) + β·f(y). $$

		„$f$ is continuous“: as $\forall x \in U: |f(x)| ≤ 1$ (and $U$ is a neighbourhood of ¦o), hence also $f \in U^∘$, so $F = T(f) \in T(U^∘)$.

		So, $T(U^∘)$ is a closed subset of $\{λ \in ®F \middle| |λ| ≤ 1\}^U$, which is compact by Tychonoff theorem. So, $U^∘$ is $ω^*$-compact.
	\end{dukazin}

	\begin{dukazin}[2.]
		Let $X$ be moreover separable. Let $D \subset X$ be a countable dense set. Then $ς(X^*, D)$ is Hausdorff ($D$ separates points of $X^*$: $f \in X^*$, $f|_D = 0$ $\implies$ $f = 0$ as $D$ is dense) and $ς(X^*)$ is metrizable ($D$ countable $\implies$ $ς(X^*, D)$ generated by a countable family of seminorms and use the theorem above) on $U^∘: ς(X^*, D)$ is a weaker Hausdorff topology than $ς(X^*, X)$.  $U^∘$ $ς(X^*, X)$ compact $\implies$ $ς(X^*, X) = ς(X^*, D)$ on $U^∘$.
	\end{dukazin}
\end{veta}

\begin{dusledek}[Banach–Alaoglu for normed spaces]
	Let $X$ be a normed linear space. Then $(B_{X^*}, w^*)$ is compact. If $X$ is separable $(B_{X^*}, w^*)$ is moreover metrizable.
\end{dusledek}

\begin{dusledek}[Reflexivity and weak compactness]
	Let $X$ be a Banach space. Then $X$ is reflexive if and only if $B_X$ is weakly compact. If $X$ is reflexive and separable, $(B_X, w)$ is moreover metrizable.
\end{dusledek}

\begin{dusledek}
	Let $X$ be a reflexive Banach space and let $f: X \rightarrow ®R$ be a function with the following properties:
	\begin{itemize}
		\item $f$ is weakly sequentially lower semi-continuous, i.e.
			$$ \forall x \in X\ \forall (x_n) \subset X: x_n \overset{w}\rightarrow x \implies f(x) ≤ \liminf f(x_n); $$
		\item $\lim_{\|x\| \rightarrow ∞} f(x) = +∞$.
	\end{itemize}
	Then $f$ attains its minimum at some point of $X$.

	\begin{dukazin}
		TODO!!!
	\end{dukazin}
\end{dusledek}

\section{Elements of the theory of distributions}
\subsection{Space of test functions and weak derivatives}

\begin{definice}[Support, test functions, the space of test functions, locally integrable, approximate unit (smoothing kernel)]
	Let $d \in ®N$ and let $Ω \subset ®R^d$ be an open set.
	\begin{itemize}
		\item If $f: Ω \rightarrow ®F$ is continuous, its support is the set
			$$ \supp f = \overline{\{x \in Ω | f(x) ≠ 0\}}, $$
			where the closure is taken in $Ω$.
		\item Let $©D(Ω, ®F) := \{f \in C^∞(Ω, ®F) | \supp f \text{ is compact subset of } Ω\}$. Elements of $©D(Ω, ®F)$ are called test functions, the space $©D(Ω, ®F)$ is called the space of test functions.
		\item A measurable function $f: Ω \rightarrow ®F$ is called locally integrable in $Ω$, if for any $x \in Ω$ there exists $r > 0$ such that $f$ is Lebesgue integrable on $U(¦x, r)$ (i.e. $\int_{U(¦x, r)} |f| < ∞$). The space of all locally integrable functions in $Ω$ is denoted by $L^1_{loc}(Ω, ®F)$. (More precisely, the space of equivalence classes.)
		\item Choose a non-negative $h \in ©D(®R^d)$ such that $\supp h \subset U(0, 1)$ and $\int_{®R^d} h = 1$. For $j \in ®N$ we define a function $h_j$ by $h_j(¦x) := j^d · h(j·¦x)$ for $¦x \in ®R^d$. The sequence $(h_j)$ obtained in this way is called an approximate unit in $©D(®R^d)$ or a smoothing kernel.
	\end{itemize}
\end{definice}

\begin{lemma}
	Let $Ω \subseteq ®R^d$ be open. Then $©D(Ω)$ is a dense subspace of $L^p(Ω)$ for any $p \in [1, ∞)$.

	\begin{dukazin}
		TODO? (notes)
	\end{dukazin}
\end{lemma}

\begin{lemma}
	Let $Ω \subset ®R^d$ be an open set.
	\begin{enumerate}
		\item Let $μ$ be a (finite) signed or complex regular Borel measure on $Ω$. If $\int_Ω φ dμ = 0$ for any $φ \in ©D(Ω)$, then $μ = 0$.
		\item Let $f \in L^1_{loc}(Ω)$ and $\int_Ω f φ = 0$ for any $φ \in ©D(Ω)$. Then $f = 0$ almost everywhere on $Ω$.
	\end{enumerate}

	\begin{dukazin}
		TODO? (notes)
	\end{dukazin}
\end{lemma}

\begin{definice}[Weak derivative]
	Let $(a, b) \subset ®R$ be an open interval an let $f \in L^1_{loc}((a, b))$.
	\begin{itemize}
		\item A function $g \in L^1_{loc}((a, b))$ is called a weak derivative of a function $f$, if for any $φ \in ©D((a, b))$ we have $\int_a^b f φ' = -\int_a^b g φ$.
		\item Let $μ$ be a finite regular Borel measure on $(a, b)$ (signed or complex). The measure $μ$ is said to be a weak derivative of a function $f$ if for any $φ \in ©D((a, b))$ we have $\int_a^b f φ' = - \int_{(a, b)} φ dμ$.
	\end{itemize}
\end{definice}

\begin{tvrzeni}
	Let $(a, b) \subset ®R$ be an open interval and let $f \in L^1_{loc}((a, b))$. If $\int_a^b f φ' = 0$ for any $φ \in ©D((a, b))$, the function $f$ is constant.

	\begin{dukazin}
		TODO? (The proof is not directly needed for the exam.)
	\end{dukazin}
\end{tvrzeni}

\begin{veta}
	Let $f \in L^1_{loc}((a, b))$.

	\begin{enumerate}
		\item The weak derivative of f is uniquely determined.
		\item If $f$ is absolutely continuous on $[a, b]$, it has a finite derivative almost everywhere, $f' \in L^1((a, b))$ and $f'$ is the weak derivative of $f$. Conversely, if a function $f$ has a weak derivative $g \in L^1((a, b))$, there exists a function $f_0$ absolutely continuous on $[a, b]$, equal to $f$ almost everywhere on $(a, b)$. In this case $g = f_0'$ almost everywhere.

			More generally, a function $f$ has a weak derivative in $L^1_{loc}((a, b))$ if and only if there exists function $f_0$ locally absolutely continuous on $(a, b)$ such that $f_0 = f$ almost everywhere.
		\item There exists a finite measure $μ$, which is a weak derivative of function $f$ if and only if there exists a function $f_0$ of bounded variation on $[a, b]$ such that $f_0 = f$ almost everywhere on $(a, b)$. In this case for each subinterval $(c, d) \subset (a, b)$ we have $μ((c, d)) = \lim_{x \rightarrow d_-} f_0(x) - \lim_{x \rightarrow c_+} f_0(x).$

			Moreover, $μ$ is real-valued if and only if $f_0$ may be real-valued and $μ$ is non-negative if and only if $f_0$ may be non-increasing.
	\end{enumerate}

	\begin{dukazin}[1. and 2.]
		TODO? (notes)
	\end{dukazin}
\end{veta}

\subsection{Distributions – basic properties and operations}

\begin{definice}[Convergence in ©D]
	Let $Ω \subset ®R^d$ be an open set, $(φ_n)$ a sequence in $©D(Ω)$ and $φ \in ©D(Ω)$. We say that the sequence $(φ_n)$ converges to $φ$ in $©D(Ω)$, if the following two conditions are fulfilled:
	\begin{itemize}
		\item There exists $K \subset Ω$ compact such that $\supp φ_n \subset K$ for each $n \in ®N$.
		\item $D^α φ_n \rightrightarrows D^α φ$ on $K$ for each multiindex $α \in ®N_0^d$.
	\end{itemize}
	This is expressed by writing $φ_n \rightarrow φ$ in $©D(Ω)$.
\end{definice}

% 21. 11. 2023

\begin{lemma}
	Let $Ω \subset ®R^d$ be an open set.\vspace{-1em}
	\begin{itemize}
		\item[a)] $\|·\|_N$ is a norm on $©D(Ω)$;
		\item[b)] $©D_K(Ω)$ is a Fréchet space when equipped with $(\|·\|_N)_{N \in ®N_0}$.
	\end{itemize}

	\begin{dukazin}[a)]
		TODO!!!
	\end{dukazin}

	\begin{dukazin}[b)]
		$\|·\|_0 ≤ \|·\|_1 ≤ \|·\|_2 ≤ … $ $\implies$ $©D_K(Ω)$ is a metrizable LCS (by translation invarinat metric $ρ$ from the proposition above).

		$(φ_n) \subset ©D_k(Ω)$ $ρ$-cauchy, then $\forall N \in ®N_0$: $(φ_n)$ is $\|·\|_N$-cauchy $\implies$ $\forall α$: $(D^α φ_n)$ is $\|·\|_∞$-cauchy $\implies$ $\forall α\ \exists ψ_n$ such that $D^α φ_n \rightrightarrows ψ_α$ on $Ω$. The $ψ_α$ is continuous, $φ_α = 0$ on $Ω \setminus K$. Fix $α \in ®N_0^d$ and $j \in [d]$. Then
		$$ D^α φ_n \rightrightarrows ψ_α \land \frac{\partial}{\partial x_j} D^α φ_n = D^{α+e_j}φ_n \rightrightarrows ψ_{α + e_j} \implies ψ_{α + e_j} = \frac{\partial}{\partial x_j} ψ_α. $$
		$\implies ψ_α = D^α ψ_0$.

		$\implies ψ_0 \in ©D_K(Ω)$, $\forall α: D^α φ_n \rightrightarrows D^α ψ_0$, i.e. $\forall N: φ_n \rightarrow ψ_0$ in $\|·\|_N$. $\implies φ_n \rightarrow ψ_0$ in $ρ$.
	\end{dukazin}
\end{lemma}

\begin{tvrzeni}
	$Λ: ©D(Ω) \rightarrow ®F$ linear then following assertions are equivalent:
	\begin{enumerate}
		\item $φ_n \rightarrow φ$ in $©D(Ω)$ $\implies$ $Λ(φ_n) \rightarrow Λ(φ)$;
		\item $φ_n \rightarrow 0$ in $©D(Ω)$ $\implies$ $Λ(φ_n) \rightarrow 0$;
		\item $\forall K \subset Ω$ compact and $Λ|_{©D_K(Ω)}$ is continuous;
		\item $\forall K \subset Ω$ compact $\exists N \in ®N_0$ $\exists C > 0$ such that
			$$ |Λ(φ)| ≤ C·\|φ\|_N, \qquad φ \in ©D_K(Ω). $$
	\end{enumerate}

	\begin{dukazin}
		„$1. \implies 2.$“ is trivial. „$2. \implies 3.$“: Fix $K \subset Ω$ compact. $φ_n \rightarrow 0$ on $©D_K(Ω)$ $\implies$ $φ_n \rightarrow 0$ in $©D(Ω)$ $\overset{2.} \implies$ $Λ(φ_n) \rightarrow 0$. Thus $Λ|_{©D_K(Ω)}$ is continuous at ¦o, so it is continuous.

		„$3. \implies 1.$“ $φ_n \rightarrow φ$ in $©D(Ω)$ $\implies$ $\exists K \subset Ω$ compact such that $\supp φ_n \subset K$ for each $n$. Then $(φ_n) \subset ©D_K(Ω)$ $\implies$ $φ_n \rightarrow φ$ in $©D_K(Ω)$ $\overset{3.}\implies$ $Λ(φ_n) \rightarrow φ(φ)$.

		„$3. \Leftrightarrow 4.$“. By the proposition above.
	\end{dukazin}
\end{tvrzeni}

\begin{definice}[Distribution, finite order]
	A distribution on $Ω$ is a linear functional $Λ: ©D(Ω) \rightarrow ®F$ satisfying assertions from the previous proposition. We will denote distributions on $Ω$ by $©D'(Ω)$.

	$Λ \in ©D'(Ω)$ is of finite order, if $N \in ®N_0$ in 4. of the previous proposition can be chosen independently on $K$.
\end{definice}

\begin{priklady}
	$f \in L^1_{loc}(Ω)$. $Λ_f(φ) = \int_Ω f·φ$ ($φ \in ©D(Ω)$) $\implies$ $Λ_f$ is a distribution of order 0. Because $K \subset Ω$ compact $\implies$ $\int_K |f| < ∞$, $φ \in D_K(Ω)$:
	$$ |Λ_f(φ)| = \left|\int_Ω f·φ\right| = \left|\int_K f·φ\right| ≤ \int_K |fφ| ≤ \|φ\|_∞·\int_K|f| = \|φ\|_0·\int_K|f|. $$

	$μ ≥ 0$ regular Borel measure, finite on compacts. $Λ_μ(φ) = \int_Ω φ dμ$ is a distribution on $Ω$ of order 0. Because if $K \subset Ω$, $φ \in ©D_K(Ω)$, then
	$$ |Λ_μ(φ)| = \left|\int_Ω φ dμ\right| = \left|\int_K φ dμ\right| ≤ \|φ\|_∞ μ(K). $$

	$μ$ is a signed (or complex) finite measure $Λ_μ(φ) = \int_Ω φ dμ$ is a distribution of order 0:
	$$ \left|\int_K φ dμ\right| ≤ \int_K |φ| d|μ| ≤ |μ|(K) · \|φ\|_∞ ≤ \|μ\|·\|φ\|_∞. $$

	$Λ(φ) = φ'(0)$, $φ \in ©D(®R)$ is a distribution of order 1. (Clearly $|Λ(φ)| ≤ \|φ'\|_∞ ≤ \|φ\|_1$.) $Λ$ not of order 0: Find $φ \in ©D(®R)$ such that $φ'(0) = 1$, $\supp φ \subset [-c, c]$ for some $c > 0$. $φ_n(x) = φ(nx)$, $x \in ®R$, $n \in ®N$, $\implies$ $φ_n \in ©D(®R)$. $\supp φ_n \subset [-c / n, c / n] \subset [-c, c]$. $\|φ_n\|_0 = \|φ\|_0$. $Λ(φ_n) = φ_n'(0) = φ'(0)·n = n$.

	$Λ(φ) = \sum_{n=0}^∞ φ^{(n)}(n)$, $φ \in ©D(®R)$ $\implies$ $Λ$ is a distribution on ®R, not of finite order ($\supp φ \subset [-k, k], k \in ®N, \implies |Λ(φ)| ≤ (k + 1)\|φ\|_K$.)
\end{priklady}

\begin{poznamka}
	If $f, g \in L^1_{loc}(Ω)$, $Λ_f = Λ_g$, then $f = g$ almost everywhere. If $μ, ν$ measures, $Λ_μ = Λ_ν$, then $μ = ν$.

	If $f \in L^1(Ω)$, $μ$ finite measure, $Λ_f = Λ_μ$, then $μ(A) = \int_A f$, for each $A \subset Ω$ Borel.
\end{poznamka}

\begin{definice}
	$Λ \in ©D'(Ω)$.

	\begin{itemize}
		\item For $α \in ®N_0^d$ define $D^αΛ(φ) = (-1)^{|α|} Λ(D^α φ)$. (For any $φ \in ©D(Ω)$.)
		\item For $f \in C^∞(Ω)$ define $(f Λ)(φ) = Λ(f φ)$. (For any $φ \in ©D(Ω)$.)
	\end{itemize}
\end{definice}

\begin{tvrzeni}
	a) $Λ \in ©D'(Ω)$, $α \in ®N_0^d$ $\implies$ $D^αΛ \in ©D'(Ω)$.

	\begin{dukazin}
		Clear: $D^α Λ: ©D(Ω) \rightarrow ®F$ linear, $K \subset Ω$ compact $\implies$ $\exists N \in ®N_0, C > 0: |Λ(φ)| ≤ C·\|φ\|_N, φ \in ©D_K(Ω)$. Then $\forall φ \in ©D_k(Ω)$:
		$$ |D^α Λ(φ)| = |Λ (D^α φ)| ≤ C·\|D^α φ\|_N ≤ C·\|φ\|_{|α| + N} $$.
	\end{dukazin}

	b) $f \in C^∞(Ω) \implies D^α Λ_f = Λ_{D^α f}$

	\begin{dukazin}[For $\partial / \partial x_1$]
		$$ \frac{\partial}{\partial x_1} Λ_f(φ) = -Λ_f\(\frac{\partial φ}{\partial x_1}\) = ? = -\int_Ω f·\frac{\partial φ}{\partial x_1} $$
		TODO!?
		$$ = \int_{\bigcup_j (a_j, b_j)} f \frac{\partial φ}{\partial x_1} dx_1 = $$
		$$ = \sum_{j=1}^n\([f·φ]_{a_j}^{b_j} - \int_{a_j}^{b_j} \frac{\partial f}{\partial x_1} φ dx_1\) = \int_Ω \frac{\partial f}{\partial x_1} φ = Λ_{\frac{\partial f}{\partial x_1}}(φ). $$
	\end{dukazin}

	c) $d = 1$, $Ω = (a, b)$, $f \in L^1_{loc}(Ω)$. Then $(Λ_f)' = Λ_g \Leftrightarrow g$ is the weak derivative of $f$. And $(Λ_f)' = Λμ \Leftrightarrow μ$ is the weak derivative of $f$.

	\begin{dukazin}
		By definitions.
	\end{dukazin}

	d) $Λ \in ©D'(Ω)$, $f \in C^∞(Ω)$ $\implies$ $f Λ \in ©D'(Ω)$.

	\begin{dukazin}
		clear: $f Λ: ©D(Ω) \implies$ IF linear
	\end{dukazin}
\end{tvrzeni}

% 24. 11. 2023

\begin{tvrzeni}
	a) $Λ \in ©D'((a, b)), Λ' = 0 \implies \exists c \in ®F: Λ = Λ_c$.

	\begin{dukazin}
		We will prove $\Ker Λ_1 \subset \Ker Λ$. Then $\exists c: Λ = c·Λ_1 = Λc$.

		$$ φ \in \Ker Λ_1 \implies Λ_1(φ) = 0, i.e. \int_a^b φ = 0. $$
		Define $φ(t) = \int_a^t φ$, $t \in (a, b)$. Then $ψ \in ©D((a, b))$, $ψ' = φ$ ($ψ' = φ$ … differentiation of indefinite integral $\implies$ $ψ \in C^∞((a, b))$, $ψ = 0$ on $(a, \min\supp φ)$ and $(\max\supp φ, b)$ $\implies$ $ψ \in ©D((a, b))$). Hence $Λ(φ) = Λ(ψ') = -Λ'(ψ) = 0$, so $φ \in \Ker Λ$.
	\end{dukazin}

	b) $Ω \subset ®R^d$ open connected, $Λ \in ©D'(Ω)$, $D^αΛ = 0$ for $|α| = 1$ $\implies$ $\exists c \in ®F: Λ = Λ_c$.

	\begin{dukazin}
		„Step 1: $Ω = \prod_{j=1}^d (a_j, b_j)$“: Induction on $d$. For $d = 1$ use a). Assume it holds for $d - 1$, denote $Ω' = \prod_{j=1}^{d - 1}(a_j, b_j)$, $x \in Ω \implies x = (x', x_d)$ ($x' \in ®R^{d - 1}$, $x_d \in ®R$), $α \in N_0^d \implies α = (α', α_d)$.

		$Λ \in ©D'(Ω)$, $D^αΛ = 0$ for $|α| = 1$. It means: $\forall φ \in ©D(Ω)\ \forall j \in [d]: Λ\(\frac{\partial φ}{\partial x_j}\) = 0$.

		Claim: $ψ \in ©D(Ω)$. Then $\exists φ \in ©D(Ω): \frac{\partial φ}{\partial x_d} = ψ \Leftrightarrow \forall x' \in Ω': \int_{a_d}^{b_d} ψ(x', x_d) dx_d = 0$. („$\implies$“ clear, „$\impliedby$“: define $φ(x', x_d) = \int_{a_d}^{x_d} ψ(x', t) dt$). Define
		$$ T: ©D(Ω) \rightarrow ©D(Ω'), \qquad Tφ(x') = \int_{a_d}^{b_d}φ(x', x_d) d x_d, \quad φ \in ©D(Ω). $$
		$T$ is linear, $\Ker T \subset \Ker Λ$ ($Tφ = 0 \implies \exists ψ \in ©D(Ω): φ = \frac{\partial ψ}{\partial x_d}$, thus $Λ(φ) = 0$). Fix $η \in ©D((a_d, b_d))$, $\int_{a_d}^{b_d} η = 1$. For $φ \in ©D(Ω')$ define $(φη)(x) = φ(x')η(x_d)$. Then $φη \in ©D(Ω)$. $\tilde Λ(φ) = Λ(φη)$, $φ \in ©D(Ω')$. Then $\tilde Λ \in ©D'(Ω')$.

		Moreover, $\forall α'$ with $|α'| = 1: D^{α'}\tilde Λ = 0$.
		$$ \(\forall j \in [d-1]: \frac{\partial}{\partial x_j} \tilde Λ(φ) = -\tilde Λ\(\frac{\partial φ}{\partial x_j}\) = - Λ\(\frac{\partial φ}{\partial x_j}η\) = -Λ\(\frac{\partial}{\partial x_j}(φη)\) = 0.\) $$
		$\implies \exists c \in ®F: \tilde Λ = Λ_c$ in $©D'(Ω')$. Then $Λ = Λ_c$ (in $©D(Ω)$) cause
		$$ φ \in ©D(Ω) \implies φ - (Tφ)η \in ©D(Ω), φ - (Tφ)η \in \Ker T \subset \Ker Λ, \text{ so, } Λ(φ) = Λ((Tφ)η) = $$
		$$ = \tilde Λ(Tφ) = Λ_c(Tφ) = \int_{Ω'} c·Tφ = \int_{Ω'}c·\int_{a_d}^{b_d} φ(x', x_d) dx_d dx' \overset{\text{FUBINI}}= \int_Ω c·φ = Λ_c(φ). $$

		„Step 2: $Ω$ is open connected, $Λ \in ©D'(Ω)$, $D^αΛ = 0$, $|α| = 1$.“: Step 1 $\implies$ $\forall Q \subset Ω$ cuboid $\exists c: Λ|_{©D(Q)} = Λ_c$. Fix one cuboid $Q_0 \subset Ω$ and the respective $c$.
		$$ A := \{x \in Ω | \exists Q \subset Ω \text{ cuboid}, x \in Q, Λ|_{©D(Q)} = Λ_c\}. $$
		Fix $A ≠ \O$ ($Q_0 \subset A$), $A$ is open, $A$ is closed in $Ω$ ($x \in \overline{A} \cap Ω$, $Q \cap A ≠ \O$, $Λ|_{©D(Q)} = Λ_d$, $y \in Q \cap A \implies Λ|_{©D(Q_y)} = Λ_c$ $\implies$ on $©D(Q \cap Q_y): Λ = Λ_c = Λ_d \implies c = d$ $\implies$ $x \in A$.). So $A = Ω$ as $Ω$ is connected. The $Λ = Λ_c$ in $©D'(Ω)$. (Proof of this was skipped, it remains that for every $φ \in ©D(Ω)$, not only for every $φ \in ©D(Q)$, it holds $Λ(φ) = Λ_c(φ)$.)
	\end{dukazin}
\end{tvrzeni}

\subsection{A bit more on distributions}
\begin{definice}[Convergence in distributions (in $©D'$)]
	$Λ_n \rightarrow Λ$ in $©D(Ω)$ $≡$ $\forall φ \in ©D(Ω): \lim_{n \rightarrow ∞} Λ_n(φ) = Λ(φ)$.
\end{definice}

\begin{tvrzeni}[On the convergence of distributions]
	a) $Λ_n \rightarrow Λ$ in $©D(Ω)$, then:
	\begin{itemize}
		\item $\forall α: D^αΛ_n \rightarrow D^αΛ$;
			\begin{dukazin}
				$$ D^αΛ_n(φ) = (-1)^{|α|}Λ_n(D^α φ) \rightarrow (-1)^{|α|}Λ(D^αφ) = D^αΛ(φ). $$
			\end{dukazin}
		\item $f \in C^∞(Ω): fΛ_n \rightarrow fΛ$.
			\begin{dukazin}
				$$ fΛ_n(φ) = Λ_n(fφ) \rightarrow Λ(fφ) = fΛ(φ). $$
			\end{dukazin}
	\end{itemize}

	b) $f_n \rightarrow f$ in $L^1_{loc}(Ω)$ ($\forall K \subset Ω$ compact: $\int_K |f_n - f| \rightarrow 0$). Then $Λ_{f_n} \rightarrow Λ_f$ in $©D'(Ω)$.

	\begin{dukazin}
		$$ φ \in ©D(Ω): |Λ_{f_n}(φ) - Λ_f(φ)| = \left|\int_Ω f_n φ - \int_Ω f φ\right| ≤ \int_Ω |f_n - f|·|φ| = $$
		$$ = \int_{\supp φ} |f_n - f|·|φ| ≤ \|φ\|_∞ \int_{\supp φ} |f_n - f| \rightarrow 0. $$
	\end{dukazin}

	c) $f_n \rightarrow f$ in $L^p(Ω)$ for some $p \in [1, ∞]$. Then $Λ_{f_n} \rightarrow Λ_f$.

	\begin{dukazin}
		Let $K \subset Ω$ be compact, $q$ the dual exponent. Then use b) with
		$$ \int_K |f_n - f| ≤ \|f_n - f\|_{L^p(K)}·\|1\|_{L^q(K)} \rightarrow 0. $$
	\end{dukazin}

	d) $φ_n \rightarrow φ$ in $©D(Ω)$. Then $Λ_{φ_n} \rightarrow Λ_φ$ in $©D'(Ω)$.

	\begin{dukazin}
		$$ φ_n \rightarrow φ \text{ in } ©D(Ω) \implies φ_n \rightarrow φ \text{ in } C^∞(Ω), \text{ and use c)}. $$
	\end{dukazin}
\end{tvrzeni}

\begin{veta}[Banach—Steinhaus for distributions]
	$(Λ_n) \subset ©D'(Ω)$ and $\forall φ \in ©D(Ω): (Λ_n(φ))$ converges in ®F. Then $Λ(φ) = \lim_{n \rightarrow ∞} Λ_n(φ)$ is a distribution on $Ω$.

	\begin{dukazin}
		Clearly $Λ$ is a linear functional on $©D(Ω)$. Further: $K \subset Ω$ compact $\implies$ $\forall n: Λ_n|_{©D_K(Ω)}$ is continuous. $©D_K(Ω)$ is a Fréchet space $\overset{\text{the lemma above, b)}} \implies$ $Λ|_{©D_K(Ω)}$ continuous $\implies$ $Λ \in ©D'(Ω)$.
	\end{dukazin}
\end{veta}

\begin{definice}[Vanishing of distributions, support of distribution, distribution with compact support]
	$Λ \in ©D'(Ω)$.
	\begin{itemize}
		\item $G \subset Ω$ open. $Λ$ vanishes on $G$ if $Λ(φ) = 0$ whenever $φ \in ©D(Ω), \supp φ \subset G$.
		\item $\supp Λ = Ω \setminus \{G \subset Ω \text{ open }| Λ \text{ vanishes on $G$}\} = $
			$$ = \{x \in Ω | \forall ε > 0 \exists φ \in ©D(Ω): \supp φ \subset U(x, ε) \land Λ(φ) ≠ \O\}. $$
		\item $Λ$ has compact support if $\supp Λ$ is a compact subset of $Ω$.
	\end{itemize}
\end{definice}

\begin{tvrzeni}[On the support of a distribution]
	a) $Λ = Λ_f$ for some $f \in L^1_{loc}(Ω)$. Then
	$$ \supp Λ_f = \supp f = \{x \in Ω | \forall ε > 0: λ^d\(\{y \in U(x, ε) \cap Ω | f(y) ≠ 0\}\) > 0\} $$

	\begin{dukazin}
		„$\subseteq$“: $X \notin \supp f$ $\implies$ $\exists ε > 0: f = 0$ almost everywhere on $U(x, ε) \cap Ω$ $\implies$ $Λ_f$ vanishes on $U(x, ε) \cap Ω$ $\implies$ $x \notin \supp Λ_f$.

		„$\supseteq$“: $x \in \supp$. Let $ε > 0$. Then $f$ is not 0 almost everywhere on $U(x, ε) \cap Ω \implies \exists φ \in ©D(U(x, ε) \cap Ω)$ 
	\end{dukazin}

	b) $Λ = Λ_μ$. Then $\supp Λ = \supp μ = Ω \setminus \bigcup \{G \subset Ω \text{ open } | \forall B \subset G \text{ Borel } μ(B) = 0\}$.

	\begin{dukazin}
		$G \subset Ω$ open the $\forall B \subset G$ Borel $μ(B) = 0$ $\Leftrightarrow$ $\forall φ \in ©D(G): \int φ dμ = 0$ $\Leftrightarrow$ $Λ_μ$ vanishes on $G$.
	\end{dukazin}

	\begin{poznamkain}
		$f$ is continuous $\implies$ $\supp f = \overline{\{x | f(x) ≠ 0\}} \cap Ω$.
	\end{poznamkain}

% 28. 11. 2023

	c) $φ \in ©D(Ω)$, $\supp φ \cap \supp Λ = \O$ $\implies$ $Λ(φ) = 0$.

	\begin{dukazin}
		Máme $\supp φ \cap \supp Λ = \O \implies \supp φ \subset \bigcup \{G \subset Ω \text{ open } | Λ \text{ vanishes on $G$}\}$ $\implies$\break $\exists G_1, G_2, …, G_n \subset Ω$ open such that $Λ$ vanishes on each $G_j$ and $\supp φ \subset G_1 \cup … \cup G_n$. We will be done if we show that $Λ$ vanishes on $G_1 \cup … \cup G_n$.
	\end{dukazin}

	\begin{dukazin}[$Λ$ vanishes on $G_1, G_2$ $\implies$ vanishes on $G_1 \cup G_2$]
		$ψ \in ©D(Ω)$, $\supp ψ \subset G_1 \cup G_2$. If $\supp ψ \subset G_1$ or $\supp ψ \subset G_2$, then $Λ(ψ) = 0$. Assume $\supp ψ \not\subset G_1$ and $\supp ψ \not\subset G_2$. Then $L:=\supp φ \setminus G_2$ $\implies$ $L$ is compact, nonempty, $L \subset G_1$. Fix $δ > 0$ such that $3δ < \dist(L, ®R^d \setminus G_1)$, $h_k$ smooth kernel.

		Fix $k \in ®N$ such that $\frac{1}{k} < δ$, $ξ := h_k * χ_{L + B(0, 2δ)} \implies ξ \in C^∞(®R^d)$. $\supp ξ \subset L + B(0, 2δ) + U(0, 1 / k) \subset L + U(0, 3δ) \subset G_1$, $ξ = 1$ on $L + B(0, δ)$. Set $ψ_1 = ξ·ψ$, $ψ_2 = (1 - ξ)ψ$ $\implies$ $ψ_1, ψ_2 \in ©D(Ω)$, $\supp ψ_1 \subset ξ \subset G_1$, $\supp ψ_2 \subset \overline{\supp ψ \setminus (L + B(0, δ))} \subset \supp ψ \setminus (L + U(0, δ)) \subset \supp ψ \setminus L \subset G_2$ $\implies$ $Λ(ψ_1) = Λ(ψ_2) = 0$. $ψ = ψ_1 + ψ_2$ $\implies$ $Λ(ψ) = Λ(ψ_1) + Λ(ψ_2) = 0$.
	\end{dukazin}

	d) $Λ$ has compact support $\implies$ $\exists N \in ®N_0\ \exists C > 0: |Λ(φ)| ≤ C·\|φ\|_N$ for $φ \in ©D(Ω)$. In particular, $Λ$ has finite order.

	\begin{dukazin}
		$\supp Λ$ is a compact subset of $Ω$ $\implies$ $\exists δ > 0: K := \supp Λ + B(0, 3δ) \subset Ω$ $\implies$ $K \subset Ω$ is compact $\implies$
		$$ \exists N \in ®N_0\ \exists C > 0: |Λ(φ)| ≤ C·\|φ\|_N, φ \in ©D_K(Ω). $$
		$ξ := h_k * χ_{\supp Λ + B(0, 2δ)}$. ($1 / k < δ$.) $ξ \in C^∞(®R^d), \supp ξ \subset \supp Λ + B(0, 2δ) + U(0, 1 / k) \subset K$. $ξ = 1$ on $\supp Λ + B(0, δ)$.

		$\forall φ \in ©D(Ω): Λ(φ) = Λ(φ ξ)$. $(1 - ξ)φ \in ©D(Ω) = 0$ on $\supp Λ + B(0, δ)$ $\implies$ $\supp (1 - ξ)φ \cap \supp Λ = \O$. $\implies Λ((1 - ξ)φ) = 0 \implies Λ(φ) = Λ(ξ φ)$.

		Then
		$$ |Λ(φ)| = |Λ(φ ξ)| ≤ C·\|ξ·φ\|_N ≤ C·2^N·\|ξ\|_N·\|φ\|_N. $$
	\end{dukazin}

	e) $\supp Λ = \{p\} \Leftrightarrow \exists N \in ®N_0, C_α \in ®F, |α| ≤ N, Λ = \sum_{|α| ≤ N} C_α D^α Λ_{δ_p}$.

	\begin{dukazin}
		„$\impliedby$“: trivial. „$\implies$“: $\{p\}$ is compact $\implies$ $\exists N, C: |Λ(φ)| ≤ C·\|φ\|_M$, $φ \in D(Ω)$. The $Λ$ is a linear combination of $D^α Λ_{δ_p}$, $|α| ≤ N$. To prove this, we use lemma above and show
		$$ \bigcap_{|α| ≤ N} \Ker D^α Λ_{δ_p} \subset \Ker Λ, $$
		i.e. $\forall φ \in ©D(Ω): D^α φ(p) = 0$ for each $|α| ≤ N$ $\implies$ $Λ(φ) = 0$.
	\end{dukazin}
\end{tvrzeni}

\subsection{Convolution of distribution}
\begin{definice}[Notation: translate, reflexion and derivative in direction]
	$M \subset ®R^d$, $f: M \rightarrow ®F$
	\begin{itemize}
		\item $y \in ®R^d$, $τ_y f(x) = f(x - y)$, $x \in y + M$;
		\item $\hat{f}(x) = f(-x)$, $x \in -M$;
		\item $a, e \in ®R^d$: $\partial_e f(a) = \lim_{r \rightarrow 0}: \frac{f(a + re) - f(a)}{r}$.
	\end{itemize}
\end{definice}

\begin{lemma}
	$φ \in ©D(®R^d)$.

	a) $x_n \rightarrow x$ in $®R^d$ $\implies$ $τ_{x_n} φ \rightarrow τ_x φ$ in $©D(®R^d)$.

	\begin{dukazin}
		$\supp φ \subset U(0, r_1)$ for some $r_1 > 0$, $\{x_n, n \in ®N\} \subset U(0, r_2)$ for some $r_2 > 0$. $K := \overline{U(0, r_1 + r_2)}$ $\implies$ $K$ is compact and $\supp τ_{x_n} φ \subset K$ for each $n$. $α \in ®N_0^d$:
		$$ \|D^α τ_{x_n} φ - D^α τ_x φ\|_∞ = \sup_{y \in ®R^d} |D^α φ(y - x_n) - D^αφ(y - x)| = \sup_{y \in K} |D^α φ(y - x_n) - D^α φ(y - x)|. $$
		Thus $D^α φ$ is continuous, so it is uniformly continuous on $\overline{U(2r_2 + r_1)}$.
		$$ ε > 0 \implies \exists δ > 0\ \forall y_1, y_2 \in \overline{U(2r_2 + r_1)}: \(\|y_1 - y_2\| < δ \implies |D^αφ(y_1) - D^αφ(y_2)| < ε\). $$
		$$ x_n \rightarrow x \implies \exists n_0\ \forall n ≥ n_0: \|x_n - x\| < δ. $$
		$$ n ≥ n_0, y \in K \implies y - x_n, y - x \in \overline{U(2r_2 + r_1)}, \|(y - x_n) - (y - x)\| = \|x_n - x\| < δ \implies $$
		$$ \implies |D^α φ(y - x_n) - D^αφ(y - x)| < ε \implies D^α τ_{x_n}φ \rightrightarrows D^α τ_x φ. $$
	\end{dukazin}

	b) $e \in ®R^d \implies \partial_e φ \in ©D(®R^d)$. Moreover, set
	$$ φ_r(x) := \frac{1}{r}(φ(x + re) - φ(x)), \qquad x \in ®R^d, $$
	then $φ_r \overset{r \rightarrow 0}\longrightarrow \partial_e φ$ in $©D(®R^d)$.

	\begin{dukazin}[$e \in ®R^d \implies \partial_e φ \in ©D(®R^d)$]
		$x \in ®R^d$. $g_x(t) := φ(x + te)$, $t \in ®R$. Then $g_x \in C^∞(®R)$.
		$$ \partial_e φ(x) = g_x'(0) = \sum_{j=1}^d \frac{\partial φ}{\partial x_j}(x + te)·e_j |_{t = 0} = $$
		$$ = \sum_{j=1}^d \frac{\partial φ}{\partial x_j}(x)e_j \implies \partial_e φ = \sum_{j=1}^d e_j \frac{\partial φ}{\partial x_j} \in ©D(®R^d). $$
	\end{dukazin}

	\begin{dukazin}[Moreover part]
		Fix $c > 0$, such that $\supp φ \subset U(0, c)$, and $0 < |r| < 1$. Then $\supp φ_r \subset \overline{U(0, c + \|e\|)}$.
		$$ |φ_r(x) - \partial_e φ(x)| = \left| \frac{1}{r}(g_x(r) - g_x(0)) - g'_x(0) \right| = \left| \frac{1}{r} \int_0^r g_x' - g_x'(0)\right| = $$
		$$ = \left| \frac{1}{r} \int_0^r (g_x'(t) - g_x'(0)) dt \right| = \left| \frac{1}{r} \int_0^r \sum_{j=1}^d e_j \(\frac{\partial φ}{\partial x_j}(x + te) - \frac{\partial φ}{\partial x_j}(x)\) dt \right| ≤ $$
		$$ ≤ \left| \frac{1}{r} \int_0^r \|e\|\(\sum_{j=1}^d \left\|\frac{\partial φ}{\partial x_j}(x + te) - \frac{\partial φ}{\partial x_j}(x)\right\|^2\)^{1 / 2} dt \right| ≤ $$
		$$ ≤ \left| \frac{1}{r} \int_0^r \|e\|\(\sum_{j=1}^d \left\|τ_{-te} \frac{\partial φ}{\partial x_j} - \frac{\partial φ}{\partial x}\right\|_∞^2\)^{1 / 2} dt \right|. $$

		$$ ε > 0 \implies \exists δ\ \forall y, \|y\| < δ: \left\|τ_{-te} \frac{\partial φ}{\partial x_j} - \frac{\partial φ}{\partial x}\right\|_∞ < ε. $$
		If $0 < |t|·\|e\|·c$, then
		$$ \|e\|\(\sum_{j=1}^d \left\|τ_{-te} \frac{\partial φ}{\partial x_j} - \frac{\partial φ}{\partial x}\right\|_∞^2\)^{1 / 2} ≤ \|e\|·\sqrt{d}·ε. $$
		So $φ_r \rightrightarrows \partial_e φ$, $D^α φ_r = (D^αφ)_r \rightrightarrows \partial_e(D^α φ) = D^α(\partial_e φ)$.
	\end{dukazin}
\end{lemma}

% 01. 12. 2023

\begin{tvrzeni}
	$φ \in ©D(®R^{d_1} \times ®R^{d_2})$.

	a) $Λ \in ©D'(®R^{d_1})$. Define $ψ(y) = Λ(x \mapsto φ(x, y))$ ($y \in ®R^{d_2}$). Then $ψ \in ©D(®R^{d_2})$.

	\begin{dukazin}
		Fix $c > 0$ such that $\supp φ \subset \overline{U(¦o, c)}$. 1. „$ψ$ is well defined“: given $y \in ®R^{d_2}$, $x \mapsto φ(x, y)$ belongs to $©D(®R^{d_1})$, i.e. it is $C^∞$ and $\supp \subset \overline{U(0, c)}$. 2. $\supp ψ \subset \overline{U(¦o, c)}$, so it is compact.

		3. $y \in ®R^{d_2}$, $φ_y(x) = φ(x, y)$ ($x \in ®R^{d_1}$). Then „$y_n \rightarrow y$ in $®R^{d_2}$ $\implies$ $φ_{y_n} \rightarrow φ_y$ in $©D(®R^{d_2})$“: Assume $y_n \rightarrow y$ in $®R^{d_2}$. WLOG $\|y_n\| ≤ c$ for each $n$. $\forall n: \supp φ_{y_n} \subset \overline{U(¦o, c)}$. Fix $α \in ®N_0^{d_1}$. Then „$D^α φ_{y_n} \rightrightarrows D^α φ_y$“:

		$D^α φ_{y_n}(x) = D^{(α, 0)} φ(x, y_n)$. $D^{(α, 0)}φ$ is continuous, hence uniformly continuous on $\overline{U(¦o, c)}$. So, give $ε .> 0$ $\exists δ > 0$ $\forall(u_1, u_2), (v_1, v_2) \in \overline{U(¦o, c)}$:
		$$ \|(u_1, v_1) - (u_2, v_2)\| < δ \implies |D^{(α, 0)} φ(u_1, v_1) - D^{(α, 0)}φ(u_2, v_2) < ε. $$
		Fix $n_0 \in ®N$ such that $\forall n ≥ n_0: \|y - y_n\| < d$. If $n ≥ n_0$ and $x \in \overline{U_{®R^{d_1}}(¦o, c)}$, then
		$$ |D^{(α, 0)}φ(x, y_n) - D^{(α, 0)}φ(x, y)| < ε \qquad \impliedby \|(x, y_n) - (x, y)\| < δ. $$
		Hence $\|D^αφ_{y_n} - D^αφ_y\| ≤ ε$ for $n ≥ n_0$.

		4. „$ψ$ is continuous“:
		$$ y_n \rightarrow y \overset{3.}\implies φ_{y_n} \rightarrow φ_y \text{ in } ©D(®R^{d_1}) \implies ψ(y_n) = Λ(φ_{y_n}) \rightarrow Λ(φ_y) = ψ(y). $$

		5. „$\frac{\partial ψ}{\partial y_j}(y) = Λ(x \mapsto \frac{\partial φ}{\partial y_j}(x, y))$“:
		$$ \frac{\partial ψ}{\partial y_j}(y) = \lim_{t \rightarrow 0} \frac{ψ(y + t e_j) - ψ(y)}{τ} \overset{Λ \text{ linear}}= \lim_{t \rightarrow 0} Λ\(x \mapsto \frac{φ(x, y+te_j) - φ(x, y)}{t}\) = $$
		$$ = \lim_{t \rightarrow 0} Λ(x \mapsto φ_t(x, y)). $$
		We know $φ_t \rightarrow \partial_{(0, y_j)}φ$ in $©D(®R^{d_1} \times ®R^{d_2})$. So we have $φ_t \rightarrow \frac{\partial φ}{\partial y_j}$ in $©D(®R^{d_1} \times ®R^{d_2})$. Hence, for each $y \in ®R^{d_2}$: $(φ_t)_y \rightarrow \(\frac{\partial φ}{\partial y_j}\)_y$ in $©D(®R^{d_1})$ $\implies$ $Λ((φ_t)_y) \rightarrow Λ\(\(\frac{\partial φ}{\partial y_j}\)_y\)$.

		$$ (*) = Λ\(\(\frac{\partial φ}{\partial y_j}\)_y\) = Λ(x \mapsto \frac{\partial φ}{\partial y_j}(x, y)). $$

		6. „$ψ \in C^∞(®R^{d_2})$ and $\forall α: D^α ψ(y) = Λ(x \mapsto D^{(0, α)} φ((x, y)))$“: 5. $\implies$ for $|α| = 1$. 4. applied to $\frac{\partial φ}{\partial y_j}$ implies $ψ \in C^1(®R^{d_2})$. Induction: Assume it holds for $|α| ≤ k$, take $|α| = k+1$. Then $α = β + e_j$, $|β| = k$, $j \in [d]$.
		$$ D^α ψ(y) = \frac{\partial }{y_j} (D^βψ)(y) = \frac{\partial}{\partial y_j} \(y \mapsto Λ\(x \mapsto D^{(0, β)}φ(x, y)\)\) \overset{5.}= $$
		$$ = Λ(x \mapsto \frac{\partial}{\partial y_j} D^{(0, β)} φ(x, y)) = Λ(x \mapsto D^{(0, α)}φ(x, y)). $$
	\end{dukazin}

	\begin{lemmain}
		$Ω \subset ®R^d$ open, $Λ \in ©D(Ω)$, $K \subset Ω$ compact. Then $\exists N \in ®N_0$, $\exists μ_α$, $|α| ≤ N$, finite (signed or complex) Borel measure on $K$ such that
		$$ Λ(φ) = \sum_{|α| ≤ N} \int_K D^α φ dμ_α, \qquad φ \in ©D_K(Ω). $$
	\end{lemmain}

	\begin{dukazin}[of lemma, sketch]
		From the proposition above $\exists N, C$ such that
		$$ |Λ(φ)| ≤ C·\|φ\|_N, φ \in ©D_K(Ω). $$
		$X := (C(K))^{\{α \middle| |α| ≤ N\}}$. $T: ©D_K(Ω) \rightarrow X$ by $T φ = (D^α φ)_{|α| ≤ N}$ $\implies$ $Λ ∘ T^{-1}$ is continuous on $T(©D_K(Ω))$ $\implies$ extend to $X$ $\implies$ (by Riesz) find $μ_α, |α| ≤ N$.
	\end{dukazin}

	b) $Λ_1 \in ©D'(®R^{d_1})$, $Λ_2 \in ©D'(®R^{d_2})$. Then
	$$ Λ_2(y \mapsto Λ_1(x \mapsto φ(x, y))) = Λ_1(x \mapsto Λ_2(y \mapsto φ(x, y))). $$

	\begin{dukazin}
		By a) both sides are well defined. $\supp φ \subset \overline{U(¦o, c)}$. From the previous lemma: $Λ_1$ (resp. $Λ_2$) on $\overline{U(¦o, c)}$ is equal to $μ_α$ (resp. $ν_α$) for some $|α| ≤ N_1$ (resp. $|α| ≤ N_2$).
		$$ Λ_2(y \mapsto Λ_1(x \mapsto φ(x, y))) = \sum_{|β| ≤ N_2} \int D^β λ_1(x \mapsto φ(x, y)) dν_β(y) = $$
		$$ = \sum_{|β| ≤ N_2} \int Λ_1(x \mapsto D^{(0, β)}φ(x, y)) dν_β(y) = $$
		$$ = \sum_{|β| ≤ N_2} \sum_{|α| ≤ N_1} \int \int D^{(α, β)} φ(x, y) dμ_α(x)dν_β(y) \overset{\text{FUBINI}}= $$
		$$ = \sum_{|β| ≤ N_2} \sum_{|α| ≤ N_1} \int \int D^{(α, β)} φ(x, y) dν_β(y)dμ_α(x) … $$
	\end{dukazin}
\end{tvrzeni}

\begin{definice}[Convolution in distributions]
	$U \in ©D'(®R^d)$, $φ \in ©D(®R^d)$, $U * φ(x) = U(τ_x \check φ) = U(y \mapsto φ(x - y))$ ($x \in ®R^d$).
\end{definice}

\begin{veta}[On the convolution of a distribution and a test function]
	a) $f \in L^1_{loc} \implies Λ_f * φ = f*φ$.

	\begin{dukazin}
		$$ Λ_f * φ(x) = Λ_f(y \mapsto φ(x - y)) = \int_{®R^d} f(y)φ(x - y) dy = f*φ(x). $$
	\end{dukazin}

	b) $U * φ \in C^∞(®R^d)$, $D^α(U * φ) = D^α U * φ = U * D^α φ$.

	\begin{dukazin}
		„$U * φ$ is continuous“:
		$$ x_n \rightarrow x \text{ in } ®R^d \implies τ_{x_n} \check φ \rightarrow τ_x \check φ \text{ in } ©D(®R^d) \implies U*φ(x_n) = U(τ_{x_n}\check φ) \rightarrow U(τ_x \check φ) = U*φ(x). $$

		$$ \frac{\partial}{\partial x_j}(U * φ)(x) = \lim_{t \rightarrow 0} \frac{U*φ(x + te_j) - U*φ(x)}{t} = $$
		$$ = \lim_{t\rightarrow 0} U\(\frac{τ_{x + te_j} \check φ - τ_x \check φ}{t}\) \overset{ψ := τ_x \check φ}= \lim_{t \rightarrow 0} U\(\frac{τ_{te_j}ψ - ψ}{t}\) = U(\partial_{-e_j} ψ) = $$
		$$ = U\(τ_x\(\frac{\partial φ}{\partial x_j}\)\) = U*\frac{\partial φ}{\partial x_j}(x). $$
		$$ \partial_{-e_j}ψ = -\partial_{e_j} ψ = - \frac{\partial ψ}{\partial y_j} = - \frac{\partial}{\partial y_j}(τ_x \check φ) = τ_x \(\frac{\partial φ}{\partial y_j}\)^v. $$
		$$ \frac{\partial}{\partial x_j}(U * φ) = U * \frac{\partial φ}{\partial x_j}. $$
		$$ \frac{\partial U}{\partial x_j} * φ(x) = \frac{\partial U}{\partial x_j}τ_x \check φ = -U\(\frac{\partial τ_x \check φ}{\partial x}\) = U*\frac{\partial φ}{\partial x_j}(x). $$

		So, we have it for $|α| = 1$. The general case by induction.
	\end{dukazin}

	c) $\supp (U*φ) \subset \supp U + \supp φ$.

	\begin{dukazin}
		$$ U*φ(x) ≠ 0 \implies U(τ_x \check φ) ≠ 0 \implies \supp (τ_x \check φ) \cap \supp U ≠ \O \implies x \in \supp φ + \supp U. $$
	\end{dukazin}

	\begin{dusledekin}
		So $U$ has compact support $\implies$ $U*φ$ has compact support.
	\end{dusledekin}

	d) $h_j$ smoothing kernel. Then $Λ_{U * h_j} \rightarrow U$ in $©D'(®R^d)$.

	\begin{dukazin}
		$$ Λ_{U * h_j}(φ) = \int(U * h_j) (x) φ(x) dx = \int U(y \mapsto h_j(x - y))φ(x) dx = $$
		$$ = \int U(y \mapsto φ(x) h_j(x - y))dx = Λ_1(y \mapsto φ(x)h_j(x - y)) = $$
		$$ = U(y \mapsto Λ_1(x \mapsto φ(x) h_j(x - y))) = U(y \mapsto \int φ(x) h_j(x - y) dx) = U(φ*\check h_j) \rightarrow Λ(φ). $$
		Because $φ*\check h_j \rightarrow φ$ in $©D(®R^d)$ and
		$$ \supp (φ * \check h_j) \subset \supp φ + U(0, 1 / j) \subset φ + \overline{U(0, 1)}, $$
		$$ D^α(φ * \check h_j) = (D^α φ) * h_j \rightrightarrows D^α φ. $$
	\end{dukazin}

% 05. 12. 2023

	e) $τ_x(U * φ) = τ_x U * φ = U*τ_x φ$

	\begin{dukazin}
		$$ τ_x(U*φ)(z) = (U * φ)(z - x) = U(τ_{z - x} \check φ) = U(τ_{-x} τ_z \check φ) = τ_xU(τ_z \check φ) = τ_x U * φ(z). $$
		$$ τ_x(U*φ)(z) = (U * φ)(z - x) = U(τ_{z - x} \check φ) = U(τ_z(τ_{-x} \check φ)) = U(τ_z(\widecheck{τ_x φ})) = U * τ_x φ(z). $$
		$$ \(τ_{-x}\check φ(y) = \check φ(y + x) = φ(-y - x) = τ_x φ(-y) = (\widecheck{τ_x φ})(y).\) $$
	\end{dukazin}

	f) $U * (φ * ψ) = (U * φ) * ψ$ ($U \in ©D'(®R^d), φ, ψ \in ©D(®R^d)$).

	\begin{dukazin}
		$$ U * (φ * ψ) (x) = U(y \mapsto (φ*ψ)(x - y)) = U(y \mapsto \int_{®R^d} φ(x - y - z) ψ(z) dz) = $$
		$$ = U(y \mapsto Λ_1(z \mapsto φ(x - y - z)ψ(z))) = Λ_1(z \mapsto U(y \mapsto φ(x - y - z)ψ(z))) = $$
		$$ = Λ_1(z \mapsto ψ(z)·U(y \mapsto φ(x - y - z))) = Λ_1(z \mapsto ψ(z)·(U * φ)(x - z)) = $$
		$$ = \int ψ(z)·(U*φ(x - z)) dz = (U * f) * ψ(x). $$
	\end{dukazin}
\end{veta}

\begin{poznamka}
	$$ \check U(φ) = U(\check φ), φ \in ©D(®R^d). $$

	$τ_x U$ and $\check U$ are distributions, $τ_x Λ_f = Λ_{τ_x f}$, $\check Λ_f = Λ_{\check f}$, $f \in L^1_{loc}(®R^d)$ (standard one page of computations or less).
\end{poznamka}

\begin{poznamka}
	$U, V$ distributions, $U*V(φ) = U(\check V * φ)$, $φ \in ©D(®R^d)$:
	
	\begin{itemize}
		\item It is natural formula:
			$$ V = Λ_ψ, ψ \in ©D(®R^d) \implies Λ_{U*ψ}(φ) = U(\check ψ * φ). $$
			\begin{dukazin}
				$$ Λ_{U*ψ}(φ) = \int_{®R^d} U * ψ(x)φ(x) dx = \int_{®R^d} U(y \mapsto ψ(x - y)) φ(x) dx = $$
				$$ = \int_{®R^d} U(y \mapsto ψ(x - y)φ(x)) dx = U(y \mapsto \int_{®R^d} ψ(x - y) φ(x) dx) = U(y \mapsto \check ψ * φ(y)). $$
			\end{dukazin}
		\item This formula does not work in general because $\check V * φ$ is a $C^∞$-function but it need not have compact support.
	\end{itemize}
\end{poznamka}

\begin{poznamka}[1.]
	$\supp V$ is compact, then $V*φ \in ©D(®R^n)$ for each $φ \in ©D(®R^d)$ ($\supp \check V * φ \subset \supp \check V + \supp φ$, so it is compact). Then $U*V$ is linear functional on $©D(®R^d)$. Moreover, „it is a distribution“:

	Fix $K \subset ®R^d$ compact. Set $L := \supp \check V + K$ $\implies$
	$$ \implies \exists C>0, N \in ®N_0: |V(ψ)| ≤ C·\|ψ\|, \qquad \forall ψ \in ©D_L(®R^d). $$
	$$ φ \in ©D_K(®R^d)\!\!\implies\!\!\check V * φ \in ©D_L(®R^d)\!\!\implies\!\!|(U * V)(φ)| = |U(\check V * φ)| ≤ C·\|\check V * φ\|_N ≤ C·D·\|φ\|_{N+M}. $$
	($\check V * φ(x) = V(y \mapsto φ(x + y))$, $V$ has compact support $\implies$ $\exists D, M: |V(η)| ≤ D·\|η\|_M$, $\forall η \in ©D(®R^d)$.)
\end{poznamka}

\begin{poznamka}[2.]
	$\supp U$ is compact $\implies$ $\exists ψ \in ©D(®R^d)$ such that $U(φ) = U(ψ·φ), φ \in ©D(®R^d)$. (Proof of the theorem above item d.) So, define $(U * V)(φ) = U(ψ·(\check V * φ))$. Again $U * V \in ©D'(®R^d)$. (Proof skipped.)
\end{poznamka}

\begin{poznamka}[3.]
	$\forall r > 0: (\overline{U(¦o, r)} - \supp V) \cap \supp U$ is compact. For $r > 0$ let $ψ_r \in ©D(®R^d)$, $ψ_r = 1$ on a neighbourhood of this set. Then $U$ may be extended to
	$$ Y = \{f \in C^∞(®R^d) \middle| \supp f \subset \overline{U(¦o, r)} - \supp V \text{ for some } r > 0\}. $$
	$\tilde U(f) = U(ψ_r · f)$ if $\supp f \subset \overline{U(¦o, r)} - \supp V$.

	Then define $U * V(φ) = \tilde U(\check V * φ)$ ($\supp \check V * φ \subset \supp φ - \supp V$).
\end{poznamka}

\begin{poznamka}[4.]
	Assume $\exists m, n \in ®N_0$, $c, d > 0$:
	$$ |U(φ)| ≤ c·\|φ\|_n \land |V(φ)| ≤ d·\|φ\|_m, \qquad \forall φ \in ©D(®R^d). $$
	$\implies$ $μ_α, |α| ≤ n$ measures (finite …):
	$$ U(φ) = \sum_{|α| ≤ n} \int_{®R^d} D^α φ dμ_α, φ \in ©D(®R^d) \implies $$
	$$ \implies (U * V)(φ) = \sum_{|α| ≤ n} \int_{®R^d} D^α(\check V * φ) dμ_α. $$
	$$ |(U*V)(φ)| ≤ c·d·\|φ\|_{n+m}. $$
\end{poznamka}

\subsection{Tempered distributions}
\begin{definice}[Schwartz space]
	$$ ©S(®R^d) = \{f \in C^∞(®R^d) \middle| \forall α \in ®N_0^d\ \forall N \in ®N: x \mapsto (1 + \|x\|^2)^N D^αf(x) \text{ is bounded on } ®R^d\}. $$
	$$ f \in ©S(®R^d), \quad N \in ®N_0, \quad p_N(f) := \max_{|α| ≤ N} \|x \mapsto (1 + \|x\|^2)^N D^α f(x)\|_∞. $$
	Then $(p_N)_{N=0}^∞$ is sequence of norms on $©S(®R^d)$, $p_0 ≤ p_1 ≤ p_2 ≤ …$. ($p_0(f) = \|f\|_∞$)
\end{definice}

\begin{tvrzeni}
	a) $©S(®R^d)$ is a Fréchet space when equipped with $(p_N)_{N=0}^∞$.

	\begin{dukazin}
		$©S(®R^d)$ is a metrizable LCS. Let $ρ$ be the respective translation invariant metric. „Completeness“: Assume $(f_n)$ is $ρ$-Cauchy $\implies$ $\forall N$: $(f_n)$ is $p_N$-Cauchy $\implies$ $\forall N\ \forall α, |α| ≤ N: (x \mapsto (1 + \|x\|^2)D^α f_k(x))_{k=1}^∞$ is $\|·\|_∞$-Cauchy $\implies$ $\forall N, α, |α| ≤ N\ \exists g_{N, α}$ such that $(1 + \|x\|^2)^N D^α f_n(x) \rightrightarrows g_{N, α}(x)$ on $®R^d$. $D^α f_k(x) \rightrightarrows \frac{g_{N, α}(x)}{(1 + \|x\|^2)^N}$. $\implies$ $\forall α\ \exists h_α$ continuous such that $g_{N, α}(x) = (1 + \|x\|^2)^N h_α(x)$ if $N ≥ |α|$. $D^α f_k \rightrightarrows h_α \implies h_α = D^α h_α \implies h_α \in C^∞(®R^d)$.

		„$h_0 \in ©S(®R^d)$“:
		$$ (1 + \|x\|^2)^N D^αh_0(x) = g_{N, α}(x), $$
		which is bounded (uniform limit of bounded functions). Moreover $f_k \rightarrow h_0$ in $p_N$, hence by the theorem above $f_n \rightarrow h_0$ in $©S(®R^d)$ (in $ρ$).
	\end{dukazin}

% 08. 12. 2023

	b) $©D(®R^d)$ is a dense subset of $©S(®R^d)$.

	\begin{dukazin}
		Clearly $©D(®R^d) \Subset ©S(®R^d)$. „Density“: Fix $φ \in ©D(®R^d)$ such that $0 ≤ φ ≤ 1$, $φ = 1$ na $U(¦o, 1)$. Let $f \in ©S(®R^d)$. Let $f_n(x) = f(x) · φ(x / n)$, $x \in ®R^d$. Then $f_n \in ©D(®R^d)$. Moreover, „$f_n \rightarrow f$ in $©S(®R^d)$“: Let $N \in ®N_0$, $d \in ®N_0^d$, $|α| ≤ N$:
		$$ \left|(1 + \|x\|^2)^N (D^α f(x) - D^αf_n(x))\right| = (1 + \|x\|^2)^N \left|D^α((1 - φ(x / n)))f(x)\right| = $$
		$$ = (1 + \|x\|^2)^N \left|(1 - φ(x / n))D^αf(x) +\!\!\sum_{0 ≠ β ≤ α}\!\binom{α_1}{β_1}·…·\binom{α_d}{β_d} (-1) \frac{1}{n^{|β|}} D^β φ(x / n) D^{α - β} f(x)\right| $$
		$$ \begin{cases}=0, \quad & \|x\| ≤ n\\ ≤ \sup_{\|x\| ≥ n, |γ| ≤ N} \frac{(1 + \|x\|^2)^{N+1}|D^γ f(x)|}{1 + \|x\|^2}, \quad & \|x\| > n\end{cases} $$
		$$ \(\sup_{\|x\| ≥ n} \(1 + \sum_{0 ≠ β ≤ α} \binom{α_1}{β_1}·…·\binom{α_d}{β_d}·\underbrace{\frac{1}{n^{|β|}}}_{≤1} \underbrace{|D^β φ(x / n)|}_{≤ \|φ\|_N}\)\) ≤ 1 + 2^N \|φ\|_N. $$
		$$ ≤ (1 + 2^N·\|φ\|_n)·\frac{p_{N+1}(f)}{1 + n^2} \rightarrow 0. $$
	\end{dukazin}

	c) $φ_n \rightarrow φ$ in $©D(®R^d)$ $\implies$ $φ_n \rightarrow φ$ in $©S(®R^d)$.

	\begin{dukazin}
		Assume $φ_n \rightarrow φ$ in $©D(®R^d)$ $\implies$ $\exists R > 0$ such that $\supp φ_n \subset \overline{U(¦o, R)}$. Then
		$$ p_n(φ_n - φ) = \max_{|α| ≤ N} \left\|x \mapsto (1 + \|x\|^2)^N(D^α φ_n(x) - D^αφ(x))\right\|_∞ ≤ (1 + R^2)^N·\|φ_n - φ\|_N \rightarrow 0. $$
	\end{dukazin}
\end{tvrzeni}

\begin{definice}[A tempered distribution on $®R^d$]
	A tempered distribution on $®R^d$ is a continuous linear functional on $©S(®R^d)$. Notation: $©S'(®R^d)$.
\end{definice}

\begin{poznamka}
	$Λ \in ©S'(®R^d) \implies Λ|_{©D(®R^d)} \in ©D'(®R^d)$. (By the previous theorem item c.)

	$©D'(®R^d) \Subset ©D'(®R^d)$. (By item a. and b.)

	We say that distribution is tempered, if it can be extended to $©S(®R^d)$.
\end{poznamka}

\begin{tvrzeni}[A characterization of tempered distributions]
	a) $Λ: ©S(®R^d) \rightarrow ®F$ linear. Then
	$$ Λ \in ©S'(®R^d) \Leftrightarrow \exists N \in ®N_0\ \exists C > 0: |Λ(φ)| ≤ C·p_N(φ), \qquad φ \in ©S(®R^d). $$

	\begin{dukazin}
		By the proposition above.
	\end{dukazin}

	b) Assume $Λ \in ©D'(®R^d)$. Then $Λ$ is tempered iff
	$$ \exists N \in ®N_0\ \exists c > 0: |Λ(φ)| ≤ C·p_N(φ), \qquad φ \in ©D(®R^d). $$

	\begin{dukazin}
		„$\implies$“: a). „$\impliedby$“: For example by Hahn–Banach and a).
	\end{dukazin}
\end{tvrzeni}

\begin{definice}[Convergence in $©S'$]
	$Λ_n \rightarrow Λ$ in $©S'(®R^d)$ $≡$ $\forall φ \in ©S(®R^d): Λ_n(φ) \rightarrow Λ(φ)$, i.e. $Λ_n \overset{w^*}\rightarrow Λ$.
\end{definice}

\begin{veta}[Banach–Steinhaus theorem for tempered distribution]
	$(Λ_n) \subset ©S'(®R^d)$, $\forall φ \in ©S(®R^d)$: $(Λ_n(φ))$ converges in ®F. Then $Λ(φ) = \lim_{n \rightarrow ∞} Λ_n(φ)$, $φ \in ©S(®R^d)$ is tempered distribution.
	
	\begin{dukazin}
		Use the previous proposition item a) and the theorem above.
	\end{dukazin}
\end{veta}

\begin{tvrzeni}[Examples of tempered distributions]
	a) $Λ \in ©D'(®R^d)$, $\supp Λ$ is compact $\implies$ $Λ$ is tempered.

	\begin{dukazin}
		$Λ$ has compact support $\implies$ $\exists C > 0\ \exists N \in ®N_0: |Λ(φ)| ≤ C·\|φ\|_N ≤ C·p_N(φ)$, $φ \in ©D(®R^d)$.
	\end{dukazin}

	b) $f \in L^p(®R^d)$ for some $p \in [1, ∞]$. Then $Λ_f \in ©S(®R^d)$ and, moreover, $L_f(φ) = \int_{®R^d} fφ, φ \in ©S(®R^d)$.

	\begin{dukazin}
		Theorem IV.11(a) $\implies$ $©S(®R^d) \subset \bigcap_{p \in [1, ∞]} L^p(®R^d)$. (It was stated and almost proven at chapter IV, but full proof is not easy.) So, fix $p \in [1, ∞]$ and $f \in L^p(®R^d)$. Let $p'$ be the dual exponent. Then $\forall φ \in ©S(®R^d): φ \in L^{p'}(®R^d)$, hence $f φ \in L^1(®R^d)$.

		So $\tilde Λ(φ) = \int_{®R^d} f φ$, $φ \in ©S(®R^d)$ is a well-defined linear functional on $©S(®R^d)$: „continuity“:
		$$ p = 1: |\tilde Λ(φ)| = |\int_{®R^d} f φ| ≤ \|f\|_1·\|φ\|_∞ = \|f\|_1·p_0(φ); $$
		$$ p > 1: \forall n \in ®N: f·χ_{U(¦o, n)} \in L^1(®R^d) \implies Λ_{f·χ_{U(¦o, n)}} \in ©S(®R^d) \text{ by the first case} \implies $$
		$$ \implies \tilde Λ(φ) = \int_{®R^d} fφ = \lim_{n \rightarrow ∞} \int_{®R^d} f·χ_{U(¦o, n)} φ = \lim_{n \rightarrow ∞} Λ_{f·χ_{U(¦o, n)}}(φ) = Λ(φ). $$
	\end{dukazin}

	c) $f$ measurable on $®R^d$, $|f| ≤ |p|$ for some polynomial $p$ on $®R^d$. Then $Λ_f \in ©S'(®R^d)$ and $Λ_f(φ) = \int_{®R^d} fφ, f \in ©S(®R^d)$.

	\begin{dukazin}
		$p$ polynomial $\implies$ $p(x) = \sum_{|α| ≤ N} c_α x^α$ ($c_α \in ®F, x^α = x_1^{α_1}·…·x_d^{α_d}$).
		$$ \implies |p(x)| ≤ c·(\sqrt{2})^{dN} (1 + \|x\|^2)^{N·\frac{d}{2}}, \qquad c = \max_α |c_α|. $$
		So, if $|f| ≤ |p|$, then $\frac{|f(x)|}{(1 + \|x\|^2)^m} ≤ c·(\sqrt{2})^{d·N}·(1 + \|x\|^2)^{N·\frac{d}{2} - m}$. If $m$ is large enough (such that $N·\frac{d}{2} - m < -\frac{d}{2}$), then $f(x) / (1 + \|x\|^2)^m$ is integrable in $®R^d$. ($1 / (1 + \|x\|^2)^k$ is integrable for $k > \frac{d}{2}$ see the comment before theorem IV.11). Then:
		$$ \left|\int_{®R^d} f·φ\right| = \left|\int_{®R^d} \frac{f(x)·(1 + \|x\|^2)^m}{(1 + \|x\|^2)^m}\right| ≤ \(\int_{®R^d} \frac{|f(x)|}{(1 + \|x\|^2)^m}\)·p_m(f). $$
	\end{dukazin}

	d) $μ$ is a finite measure $\implies$ $Λ_μ \in ©S'(®R^d)$, $Λ_m(φ) = \int_{®R^d} φ dμ$, $φ \in ©S(®R^d)$.

	\begin{dukazin}
		$φ \in ©S(®R^d)$ $\implies$ $φ$ is continuous and bounded.
		$$ \left| \int_{®R^d} φ dμ\right| ≤ \int_{®R^d} |φ| d|μ| ≤ \|f\|_∞·\|μ\| = p_0(φ)·\|μ\|. $$
	\end{dukazin}
\end{tvrzeni}

\begin{lemma}[Continuity of operations on the Schwartz space]
	%Let $L: ©S(®R^d) \rightarrow ©S(®R^d)$ is linear. Then $L$ is continuous $\Leftrightarrow$ $\forall N \in ®N_0\ \exists c > 0\ \exists M \in ®N_0: p_N(c·(f)) ≤ c·p_M(f)$, $f \in ©S(®R^d)$, $α \in ®N_0^d$.

	$f \mapsto D^α f$ is continuous $©S(®R^d) \rightarrow ©S(®R^d)$.

	\begin{dukazin}
		$f \in ©S(®R_6d)$, $α \in ®N_0^d$ $\implies$ $D^α f \in L^∞()®R^d$. Fix $N \in ®N_0$ and $β$, $|β| ≤ N$:
		$$ |(1 + \|x\|^2)^N D^β(D^α f)(x)| = (1 + \|x\|^2)^N |D^{β+α} f(x)| ≤ p_{N + |α|}(f) \implies $$
		$\implies$ $p_N(D^α f) ≤ p_{N+|α|}(f)$.
	\end{dukazin}
	
	$p$ is polynomial $\implies$ $f \mapsto p·f$ is continuous $©S(®R^d) \rightarrow ©S(®R^d)$.

	\begin{dukazin}
		Clearly $p·f \in C^∞(®R^d)$. Fix $N \in ®N_0$. Then $\exists c > 0, m \in ®N$ such that
		$$ \forall α, |α| ≤ N,\ \forall x \in ®R^dL |D^α p(x)| ≤ c·(1 + \|x\|^2)^m. $$

% 12. 12. 2023

		Fix $α$, $|α| ≤ N$, $x \in ®R^d$:
		$$ |(1 + \|x\|^2)^N D^α(p·f)(x)| = (1 + \|x\|^2)^N |\sum_{β ≤ α} \binom{α}{β} D^β p(x) D^{α - β}f(x)| ≤ $$
		$$ ≤ c·(1 + \|x\|^2)^{N+M} \sum_{β ≤ α} \binom{α}{β} |D^{α - β} f(x)| ≤ c·\sum_{β ≤ α} \binom{α}{β} p_{N+M}(f) ≤ c·2^N p_{N+M}(f) \implies $$
		$$ \implies p_N(p·f) ≤ c·2^N p_{N+M}(f). $$
	\end{dukazin}

	$g \in ©S(®R^d)$ $\implies$ $f \mapsto f·g$ is continuous $©S(®R^d) \rightarrow ©S(®R^d)$.

	\begin{dukazin}
		$g \in ©S(®R^d)$ $\implies$ $\forall α: D^α g$ is bounded on $®R^d$. Fix $N \in ®N_0$. Set $C := \max_{|α| ≤ N} \|D^α g\|_∞$. Fix $α$, $|α| ≤ N$, $x \in ®R^d$.
		$$ \left|(1 + \|x\|^2)^N D^α(f·g) (x)\right| = (1 + \|x\|^2)^N \left|\sum_{β ≤ α} \binom{α}{β} D^β g(x) D^{α - β}f(x)\right| ≤ $$
		$$ C·\sum_{β ≤ α} p_N(f) ≤ C·2^N ·p_N(f) \implies p_N(g·f) ≤ C·2^N p_N(f). $$
	\end{dukazin}

	\begin{poznamkain}
		Similarly one may probe that: $g \in C^∞(®R^d)$, $\forall α$ $\exists P_α$ polynomial: $|D^α g| ≤ |P_α|$ on $®R^d$ $\implies$ $f \mapsto f·g$ is a continuous mapping $©S(®R^d) \rightarrow ©S(®R^d)$.
	\end{poznamkain}
\end{lemma}

\begin{tvrzeni}[Operations with tempered distributions]
	Let $Λ \in ©S'(®R^d)$.

	a) $\forall α: D^α Λ \in ©S'(®R^d)$ and $D^α Λ(φ) = (-1)^{|α|}Λ(D^α φ)$, $φ \in ©S(®R^d)$.

	\begin{dukazin}
		$φ \in ©S(®R^d) \implies D^αφ \in ©S(®R^d)$. So, $\tilde Λ(φ) = (-1)^{|α|}Λ(D^α φ)$, $φ \in ©S(®R^d)$, is well-defined linear functional on $©S(®R^d)$ whose restriction to $©D(®R^d)$ is $D^αΛ$. „Continuity:“ $φ_n \rightarrow φ$ in $©S(®R^d)$ $\implies$ (by the previous lemma) $D^α φ_n \rightarrow D^α φ$ in $©S(®R^d)$, so $\tilde Λ(φ_n) = (-1)^{|α|}Λ(D^αφ_n) \rightarrow (-1)^{|α|} Λ(D^α φ) = \tilde Λ(φ)$.
	\end{dukazin}

	b) $f \in ©S(®R^d)$ and $f$ is a polynomial $\implies$ $f·Λ \in ©S'(®R^d)$ and $fΛ(φ) = Λ(fφ)$, $φ \in ©S(®R^d)$.

	\begin{dukazin}[Skipped on lecture]
		Completely analogous to a).
	\end{dukazin}

	c) $y \in ®R^d \implies τ_y Λ \in ©S'(®R^d), τ_yΛ(φ) = Λ(τ_{-y} φ), φ \in ©S'(®R^d)$.

	\begin{dukazin}
		„$φ \in ©S(®R^d) \implies τ_{-y}φ \in ©S(®R^d)$, $τ_{-y}φ(x) = φ(x - y)$“: Clearly $τ_{-y} φ \in C^∞(®R^d)$.
		$$ |α| ≤ N: (1 + \|x\|^2)^N D^α τ_{-y}φ(x) = (1 + \|α\|^2)^N D^α φ(x + y) = $$
		$$ = \(\frac{1 + \|x\|^2}{1 + \|x + y\|^2}\)^N·(1 + \|x + y\|^2)^N D^α φ(x + y) ≤ \(\frac{1 + \|x\|^2}{1 + \|x + y\|^2}\)^N·p_N(φ) ≤ M^N, $$
		where $M = \sup_{t \in [0, ∞)} \frac{1 + t^2}{1 + (t - \|y\|)^2} < ∞$. $\implies$ $τ_{-y}φ \in ©S(®R^d)$ and $p_N(τ_{-y}φ) ≤ M^N p_N(φ)$. So $φ \mapsto τ_{-y} φ$ is continuous and then continue as in a).
	\end{dukazin}

	d) $\check Λ \in ©S'(®R^d)$, $\check Λ(φ) = Λ(\check φ)$, $φ \in ©S(®R^d)$.

	\begin{dukazin}
		Observe that $φ \in ©S(®R^d) \implies \check φ \in ©S(®R^d)$ ($\check φ(x) = φ(-x)$) and $p_N(\check φ) = p_N(φ)$.
	\end{dukazin}
\end{tvrzeni}

\begin{tvrzeni}
	$Λ_n \rightarrow Λ$ in $©S'(®R^d)$. a) $\forall α: D^α Λ_n \rightarrow D^α Λ$, b) $f \in ©S(®R^d)$ and $f$ is polynomial $\implies$ $f Λ_n \rightarrow fΛ$.

	\begin{dukazin}
		„a)“: $φ \in ©S(®R^d)$:
		$$ D^α Λ_n(φ) = (-1)^{|α|} Λ_n(D^α φ) \rightarrow (-1)^{|α|}Λ(D^α φ) = D^α Λ(φ). $$
		„b)“ similarly.
	\end{dukazin}
\end{tvrzeni}

\subsection{Convolution and the Fourier transform of tempered distributions}
\begin{poznamka}[Recall]
	$f \in L^1(®R^d) \implies \hat{f}(t) = \int_{®R^d} f(x) e^{-i\<t, x\>} d m_d(x)$.

	Fourier transform maps $L^1(®R^d)$ into $C_0(®R^d)$ and $©S(®R^d)$ onto $©S(®R^d)$.

	$$ \hat{\hat{f}} = \check f, \qquad f \in ©S(®R^d), \qquad \(\hat{\hat{\hat{\hat{f}}}} = f\). $$
\end{poznamka}

\begin{lemma}
	Fourier transform is an isomorphism of $©S(®R^d)$ onto $©S(®R^d)$.

	\begin{dukazin}
		1. The theorem above $\implies$ Fourier transform is a linear bijection $©S(®R^d)$ and $©S(®R^d)$.

		2. $m := \left\lfloor\frac{d}{2}\right\rfloor + 1$. Then $C:= \int_{®R^d} \frac{1}{(1 + \|x\|^2)^m} d m_d(x) ≤ ∞$. $f \in ©S(®R^d)$ $\implies$
		$$ \implies \|\hat{f}\|_∞ ≤ \|f\|_{L^1} = \int_{®R^d} |f(x)| d m_d(x) ≤ \int_{®R^d} \frac{(1 + \|x\|^2)^m |f(x)|}{(1 + \|x\|^2)^m} d m_d(x) ≤ C·p_m(f). $$
		TODO!?

		3. Fix $N \in ®N_0$, $α$, $|α| ≤ N$.
		$$ f \in ©S(®R^d): (1 + \|x\|^2)^N D^α \hat{f}(x) = (1 + \|x\|^2)^N \widehat{(y \mapsto (-1)^{|α|} y^α f(y))}(x) = $$
		$$ = (-i)^{|α|} \widehat{(y \mapsto \breve{\breve{p}}(D) (y^α(f(y))))}(x) = (-i)^{|α|} \widehat{(y \mapsto \sum_{|β| ≤ 2N} a_β D^β(y^α f(y)))}(x), $$
		where $\breve p(x) = p(ix)$, $p(D)f = \sum_{c_α}D^α f$ if $p(x) = \sum c_α x^α$. $\breve{\breve p}(x) = (1 - \sum_{j=1}^d x_j^2)^N$ a polynom of degree $2N$.

		So, $\|x \mapsto (1 + \|x\|^2)^N D^α \hat{f}(x)\|_∞ ≤ c·p_m(y \mapsto \sum_{|β| ≤ 2N} a_β D^β(y^α f(y)))$. From the previous lemma $f \mapsto \sum_{|β| ≤ N} a_β D^β (y^α f(y))$ is continuous.

		So, $\exists M = M_{N, α} > 0$, $\exists m = m_{N, α} \in ®N_0$:
		$$ p_m(y \mapsto \sum_{|β| ≤ 2N} a_β D^β (y^α f(y))) ≤ M·p_n(f) \implies $$
		$$ \implies \|x\| \mapsto (1 + \|x\|^2)^N D^α \hat{f}(x)\|_∞ ≤ C·M·p_m(f). $$

		4. So, $p_N(\hat{f}) ≤ C·\tilde M· p_{\tilde m}(t)$, where $\tilde M = \max_{|α| ≤ N} M_{N, α}$, $\tilde m = \max_{|α| ≤ N} m_{N, α}$.
	\end{dukazin}
\end{lemma}

\begin{definice}[Fourier transform of tempered distribution]
	$Λ \in ©S'(®R^d)$: $\hat{Λ}(φ) = Λ(\hat{φ})$, $φ \in ©S(®R^d)$.

	\begin{poznamkain}
		$\hat{Λ} \in ©S'(®R^d)$: $φ_n \rightarrow φ$ in $©S(®R^d)$ $\implies$ $\hat{φ_n} \rightarrow \hat{φ}$ in $©S(®R^d)$ $\implies$ $\hat{Λ}(φ_n) = Λ(\hat{φ_n}) \rightarrow Λ(\hat{φ}) = \hat{Λ}(φ)$.
	\end{poznamkain}
\end{definice}

\begin{veta}[Properties of Fourier transform]
	a) Fourier transform is a linear bijection of $©S'(®R^d)$ onto $©S'(®R^d)$.

	\begin{dukazin}
		$\hat{\hat{Λ}} = \check Λ$, $\hat{\hat{\hat{\hat{Λ}}}} = Λ$ for $Λ \in ©S'(®R^d)$, $\check(Λ)(φ) = Λ(\check φ)$, $\hat{Λ_1} = \hat{Λ_2} \implies \hat{\hat{\hat{\hat{Λ_1}}}} = \hat{\hat{\hat{\hat{Λ_2}}}} \implies Λ_1 = Λ_2$.
	\end{dukazin}

% 15. 12. 2023

	b) $Λ_n \rightarrow Λ$ in $©S'(®R^d)$ $\implies$ $\hat{Λ_n} \rightarrow \hat{Λ}$ in $©S'(®R^d)$.

	\begin{dukazin}
		$$ \hat{Λ_n}(φ) = Λ_n(\hat{φ}) \rightarrow Λ(\hat{φ}) = \hat{Λ}(φ). $$
	\end{dukazin}

	c) $f \in C^1(®R^d) \implies \hat{Λ_f} = Λ_{\hat{f}}$.

	\begin{dukazin}
		$$ \hat{Λ_f}(φ) = Λ_f(\hat{φ}) = \int f \hat{φ} d m_d = \int \hat{f} φ d m_d = Λ_{\hat{f}}(φ). $$
	\end{dukazin}

	d) $f \in L^2(®R^d) \implies \hat{Λ_f} = Λ_{©P(f)}$, where ©P is the Plancherel transform.

	\begin{dukazin}
		$f_n := f·χ_{U(¦o, n)}$. Then $f_n \in L^1(®R^d) \cap L^2(®R^d)$, $f_n \rightarrow f$ in $L^2(®R^d)$, and, moreover, $\hat{f_n} \rightarrow ©P(f)$ in $L^2(®R^d)$. So, $\hat{Λ_f}(φ) = Λ_f (\hat{φ}) =$
		$$ = \int_{®R^d} f \hat{φ} d m_d = \lim_{n \rightarrow ∞} \int_{®R^d} f_n \hat{φ} d m_d = \lim_{n \rightarrow ∞} \int_{®R^d} \hat{f_n} φ = \int_{®R^d} ©P(f) φ d m_d = L_{©P(f)}(φ). $$
	\end{dukazin}

	e) $p$ polynomial $\implies$ $\widehat{p(D) Λ} = \breve p \hat{Λ}$, $\widehat{p·Λ} = \breve p(D) \hat{Λ}$.
	$$ \(\breve p(t) = p(it), \qquad \check p(t) = p(-t), \qquad p(t) = \sum c_α t^α \implies p(D) f = \sum c_α D^α f.\) $$

	\begin{dukazin}
		$$ \widehat{p(D) Λ}(φ) = p(D) Λ(\hat{φ}) = Λ(\check(D) \hat{φ}) = Λ\(\widehat{\breve p φ}\) = \hat{Λ}(\breve p φ) = \breve p \hat{Λ}(φ). $$
		$$ \widehat{p·Λ}(φ) = p·Λ(\hat{φ}) = Λ(p \hat{φ}) = Λ\(\widehat{\breve{\check{p}}(D)φ}\) = \hat{Λ}(\breve{\check{p}}(D)φ) = \breve p(D)\hat{Λ}(φ). $$
	\end{dukazin}

	\begin{poznamkain}
		In particular
		$$ \widehat{D^α Λ} = (x \mapsto c^{|α|}x^α) \hat{Λ}, \qquad \widehat{(x \mapsto x^α) L} = c^{|α|}D^α \hat{Λ}. $$
	\end{poznamkain}
\end{veta}

\begin{poznamka}
	Next two lemmata are analogues of Lemmata above.
\end{poznamka}

\begin{lemma}
	a) $φ \in ©S(®R^d)$, $x_n \rightarrow x$ in $®R^d$ $\implies$ $τ_{x_n} φ \rightarrow τ_x φ$ in $©S(®R^d)$.

	\begin{dukazin}
		Fix $N \in ®N_0$, $α \in ®N_0^d$, $|α| ≤ N$. $x, z \in ®R^d$:
		$$ \left|(1 + \|x\|^2)^N D^α τ_z φ(y) - (1 + \|y\|^2)^N D^α τ_x φ(y)\right| = $$
		$$ = (1 + \|y\|^2)^N \left| \int_0^1 \frac{d}{dt} D^α φ(y - x - t(z - x)) dt \right| ≤ $$
		$$ ≤ (1 + \|y\|^2)^N \int_0^1 \left| \sum_{j = 1}^d D^{α + e_j} φ(y - x - t(z - x))(x_j - z_j)\right| dt ≤ $$
		$$ ≤ (1 + \|y\|^2)^N \int_0^1 \left| \sum_{j = 1}^d D^{α + e_j} φ(y - x - t(z - x))^2\right|^{1 / 2} \|x - z\| dt ≤ $$
		$$ ≤ \|x - z\| p_{N+1}(φ) (1 + \|y\|^2)^N \int_0^1 \frac{\sqrt{d} dt}{(1 + \|y - x - t(z - x)\|^2)^{N+1}} ≤ $$
		$$ ≤ \|x - z\| p_{N+1}(φ)·\sqrt{d} \frac{(1 + \|y\|^2)^N}{(1 + \|y - x\|^2 - \|y - x\|)^{N+1}}. $$
		$$ \(1 + \|y - x\| - t\|z - x\|^2 ≥ 1 + \|y - x\|^2 - \|y - x\|, \text{ if } \|z - x\| ≤ \frac{1}{2}.\) $$
	\end{dukazin}

	b) skipped and proof skipped too.
\end{lemma}

\begin{lemma}[RT]
	$$ Λ \in ©S'(®R^d) \Leftrightarrow \exists N \in ®N_0, μ_α, |α| ≤ N_0: Λ(φ) = \!\!\sum_{|Λ| ≤ N} \int_{®R^d} (1 + \|x\|^2)^N D^α φ(x) d μ_α(x), φ \in ©S(®R^d). $$
	(Finite signed/complex measure on $®R^d$.)
\end{lemma}

\begin{definice}[Convolution of function and tempered distribution]
	$$ U \in ©S'(®R^d), φ \in ©S(®R^d) \implies U * φ(x) = U(τ_x \check φ) = U(y \mapsto φ(x - y)). $$
\end{definice}

\begin{veta}[Analogues to the theorem above]
	TODO? (Fubini; It is not directly needed for the exam.)

	a) $U*φ \in C^∞(®R^d)$ and $D^α(U*φ) = (D^αU) * φ = U*D^αφ$ for each multi-index $α$.

	\begin{dukazin}
		Skipped.
	\end{dukazin}

	b) $Λ_{U*φ}$ is a tempered distribution.

	\begin{dukazin}
		The proposition above $\implies$ $\exists N, C$ such that $|U(ψ)| ≤ c·p_N(ψ)$
		$$ \implies |(U*φ)(x)| = |U(τ_x \check φ)| ≤ C·p_N(τ_x \check φ). $$
		$$ |α| ≤ N: |(1 + \|y\|^2)^N D^α φ(x - y)| ≤ p_N(φ)·\(\frac{1 + \|y\|^2}{1 + \|x - y\|^2}\)^N \!\!≤ p_N(φ) (1 + \|y\| + \|x\|^2)^N \!\implies $$ 
		$\implies$ $Λ_{U*φ}'$ is tempered.
		$$ \frac{1 + \|y\|^2}{1 + \|x - y\|^2} = \frac{1 + \|y - x\|^2 + 2\<y - x, x\> + \|x\|^2}{1 + \|x - y\|^2} = $$
		$$ = 1 + \frac{2·\|y - x\|·\|x\|}{1 + \|x - y\|^2} + \frac{\|x\|^2}{1 + \|x - y\|^2} ≤ 1 + \|x\| + \|x\|^2. $$
	\end{dukazin}

	c) If $f \in L^p(®R^d)$ for some $p \in [1, ∞]$, then $Λ_f * φ = f * φ$.

	\begin{dukazin}
		Skipped.
	\end{dukazin}

	d) $\widehat{Λ_{U*φ}} = \hat{φ}·\hat{U}$, $\widehat{φ·U} = Λ_{\hat{φ}*\hat{U}}$.

	\begin{dukazin}
		$$ \widehat{Λ_{U*φ}}(ψ) = Λ_{U*φ}(\hat{ψ}) = \int_{®R^d} (U*φ)(x) \hat{ψ}(x) f m_d(x) = \int_{®R^d} U(y \mapsto φ(x - y)) \hat{ψ}(x) d m_d(x) = $$
		$$ = U(y \mapsto \int_{®R^d} φ(x - y)\hat{ψ}(x) d m_d(x)) = U(\check φ * \hat{ψ}) = U(\hat{\hat{φ}} + \hat{ψ}) = U(\widehat{\hat{φ}·ψ}) = \hat{U}(\hat{φ}·ψ) = \hat{φ}·\hat{U}(ψ). $$
		$$ Λ_{\hat{φ}*\hat{U}} = \widehat{\widehat{\widehat{\widehat{Λ_{\hat{φ} + \hat{U}}}}}} = \widehat{\widehat{\widehat{\hat{\hat{φ}}·\hat{\hat{U}}}}} = \widehat{\widehat{\widehat{\check φ·\check U}}} = \widehat{\widecheck{\check φ · \check U}} = \widehat{φ U}. $$
	\end{dukazin}

	e) $U * (φ * ψ) = (U * φ) * ψ$.

	\begin{dukazin}
		Skipped.
	\end{dukazin}
\end{veta}

\section{Elements of vector integration}
\begin{poznamka}
	$(M, ©A)$ is measure space, $(Ω, Σ, μ)$ is a complete measure space ($μ≥0$), $X$ is a Banach space.
\end{poznamka}

\subsection{Measurability}
\begin{definice}[Simple, simple measurable, (strongly) ©A-measurable, Borel ©A-measurable, weakly ©A-measurable]
	$f: M \rightarrow X$.

	\begin{itemize}
		\item $f$ is simple, if $f(M)$ is finite , i.e. $f = \sum_{j=1}^k x_j χ_{A_j}$, where $x_j \in X$, $A_j \subset M$ pairwise disjoint;
		\item $f$ is simple measurable, if $f$ is a simple and, moreover, $A_j \in ©A$;
		\item $f$ is (strongly) ©A-measurable if $\exists (u_n)$ simple measurable: $u_n \rightarrow f$ point-wise, i.e. $\forall x \in M: u_n(x) \rightarrow f(x)$ in $(X, \|·\|)$;
		\item $f$ is Borel ©A-measurable, if $\forall U \subset X$ open: $f^{-1}(U) \in ©A$;
		\item $f$ is weakly ©A-measurable if $\forall φ \in X^*: φ ∘ f$ is ©A-measurable.
	\end{itemize}
\end{definice}

\begin{tvrzeni}
	a) Simple functions, simple measurable functions, strongly ©A-measurable functions, and weakly ©A-measurable functions form vector spaces.
	\begin{dukazin}
		$f, g: M \rightarrow X$, $α, β \in ®F$.
		\begin{itemize}
			\item „$f, g$ simple $\implies$ $α f + β g$ is simple“: $(αf + βg)(M) \subset αf(M) + βg(M)$.
			\item „$f, g$ simple measurable $α f + βg$ is simple measurable“:
				$$ f = \sum_{j=1}^k x_j χ_{A_j}, \quad g = \sum_{l=1}^m y_l χ_{B_l}, \qquad αf + βg = \sum_{j=1}^k \sum_{l=1}^m (αx_j + β y_l)·χ_{A_j \cap B_l}, $$
				$A_j, B_l \in ©A \implies A_j \cap B_l \in ©A$.
			\item „$f, g$ strongly ©A-measurable $\implies$ $α f + β g$ is strongly ©A-measurable“: $f = \lim u_n$, $g = \lim v_n$, $u_n, v_n$ simple measurable, $α f + β g = \lim (α u_n + β v_n)$.
			\item „$f, g$ weakly ©A-measurable $\implies$ $α f + β g$ weakly ©A-measurable“: $\forall φ \in X^*: φ∘(α f + β g) = α φ ∘ f + β φ ∘ g$ (measurable by the scakercah?).
		\end{itemize}
	\end{dukazin}

	b) $f_n \rightarrow f$ point-wise, $f_n$ Borel ©A-measure (resp. weakly ©A-measurable) $\implies$ $f$ is Borel ©A-measurable (resp. weakly ©A-measurable).

	\begin{dukazin}
		Assume that $\forall n$: $f_n$ is Borel ©A-measurable. $U \subset X$ open:
		$$ f^{-1}(U) = \bigcup_{n \in ®N} \bigcup_{m \in ®N} \bigcap_{k=m}^∞ f_k^{-1}(\{x \in X | \dist(x, X \setminus U) > \frac{1}{n}\}), $$
		$$ f(x) \in U \Leftrightarrow \exists n \in ®N\ \exists m \in ®N\ \forall k ≥ m: \dist(f_k(x), X \setminus U) > \frac{1}{n}. $$

		$f_n$ are wakly ©A-measurable, $φ \in X^*$ $\implies$ $\forall n: φ ∘ f_n$ is Borel ©A-measurable and $φ ∘ f_n \rightarrow φ ∘ f$, so, $φ ∘ f$ is Borel ©A-measurable.
	\end{dukazin}

	c) $f$ is strongly ©A-measurable $\implies$ $f$ is Borel ©A-measurable $\implies$ $f$ is weakly ©A-measurable.

	\begin{dukazin}
		$f$ is simple $\implies$ ($f$ is simple measurable $\Leftrightarrow$ $f$ is Borel ©A-measurable).

		„$f$ strongly ©A-measurable $\implies$ $f$ Boreal ©A measurable“: $f = \lim u_n$, $u_n$ simple measurable, then $u_n$ are Borel ©A-measurable, so by b), $f$ is Borel ©A-measurable.

		„$f$ Borel ©A-measurable $\implies$ $f$ is weakly ©A-measurable“: $φ \in X^*$, $U \subset ®F$ open $\implies$ $(φ ∘ f)^{-1}(U) = f^{-1}(\underbrace{φ^{-1}(U)}_{\text{open}}) \in ©A$.

		„$f$ simple, weakly ©A-measurable $\implies$ $f$ is simple measurable“: $f(M) = \{x_1, …, x_k\}$, distinct points. „$f^{-1}(x_1) \in ©A$“: for $j \in \{2, …, k\}$ find $φ_j \in X^*$, $φ_j(x_1) ≠ φ_j(x_j)$. Then
		$$ f^{-1}(x_1) = \bigcap_{j=2}^k \{t \in M | φ_j(f(t) - x_j) ≠ 0\} = \bigcap_{j=2}^k \overbrace{(φ_j ∘ f)^{-1}(\underbrace{®F \setminus \{φ_j(x_j)\}}_{\text{open}})}^{\in ©A}. $$
	\end{dukazin}

	d) $f: M \rightarrow X$ strongly ©A-measurable $\implies$ $f(M)$ is separable.

	\begin{dukazin}
		$f = \lim u_n$, $u_n$ simple measurable. $f(M) \subset \overline{\bigcup_n u_n(M)}$.
	\end{dukazin}

	e) $f$ Borel ©A-measurable $\implies$ $t \mapsto \|f(t)\|$ measurable.

	\begin{dukazin}
		$h(x) = \|x\|$, $x \in X$, is continuous, hence $h ∘ f$ is measurable:
		$$ U \text{ open}: (h∘f)^{-1}(U) = f^{-1}(\underbrace{h^{-1}}_{\text{open}}) \in ©A. $$
	\end{dukazin}
\end{tvrzeni}

\begin{lemma}
	$(f_n)$ strongly ©A-measurable, $f_n \rightarrow f$ point-wise $\implies$ $f$ is strongly ©A-measurable.

	\begin{dukazin}
		$u_{m, n}$ simple measurable, $u_{m, n} \overset{m}\rightarrow f_n$. $C = \underset{{m, n}}\bigcup u(m, n)(M)$ is countable, so, $C = \{x_k, k \in ®N\}$. For $k \in ®N$ define $g_k: M \rightarrow X$ by $g_k(x) = $ the point from $\{x_1, …, x_k\}$ nearest to $f(x)$ (the first such point). Then $g_k$ is simple, $g_k \rightarrow f$ point-wise ($t \in M$, $ε > 0$ $\implies$ $\exists n_0\ \forall n ≥ n_0: \|f_n(t) - f(t)\| < ε/2$. Fix one $n ≥ n_0$ $\implies \exists m_9\ \forall m ≥ m_0: \|u_{m, n}(t) - f_n(t)\| < ε/2$. Fix one $m ≥ m_0$ $\implies \|u_{m, n}(t) - f(t)\| < ε$, and there is $k_0$ such that $u_{m, n}(t) = x_{k_0}$. Then for $k ≥ k_0$: $\|f(t) - g_k(t)\| ≤ \|f(x) - x_{k_0}\| < ε$).

		„$g_k$ are also simple measurable“: $f$ is Borel ©A–measurable $\implies$ $\forall x \in X$: $f - x$ is Borel ©A-measurable $\implies$ $\forall x \in X$: $t \mapsto \|f(t) - x\|$ is measurable, $g_k(t) = x_j$
		$$ \Leftrightarrow \forall i \in [k]: \|x_j - f(x)\| ≤ \|x_i - f(x)\| \land \forall i < j: \|x_j - f(x)\| < \|x_i - f(x)\| \Leftrightarrow \forall i \in [k]\ \forall g \in ®Q:$$
		$$ \|x_j - f(x)\| ≤ q \lor \|x_i - f(x)\| ≥ q \land \forall i < j\ \exists q \in ®Q: \|x_j - f(x)\| < q \land \|x_i - f(x)\| > q. $$
	\end{dukazin}
\end{lemma}

\begin{veta}[Pettis]
	$f: M \rightarrow X$. Then following assertions are equivalent:\vspace{-0.8em}
	\begin{enumerate}
		\item $f$ is strongly ©A-measurable;
		\item $f$ is Borel ©A-measurable and $f(M)$ is separable;
		\item $f$ is weakly ©A-measurable and $f(M)$ is separable.
	\end{enumerate}

	\begin{dukazin}
		„$1. \implies 2. \implies 3.$“ from the previous proposition. „$3. \implies 1.$“: Firstly WLOG  $X$ is separable (replace $X$ by $\overline{\LO f(M)}$). Secondly let $(x_n)$ be a dense sequence in $X$.
		$$ \forall n: \text{ fix } φ_n \in X^*, \|φ_n\| = 1, φ_n(x_n) = \|x_n\|. $$
		Thirdly $\forall x \in X: \|x\| = \sup_n |φ_n(x)|$ („$≥$“: clear as $\|φ_n\| = 1$, „$≤$“: it holds for $x = x_n$, so on dense set, LHS is continuous, RHS is continuous (supremum of 1-Lipschitz functions), so it holds on $X$). Fourthly $\forall x \in X: t \mapsto \|f(t) - x\|$ is measurable ($\|f(t) - x\| = \sup_n |(φ_n ∘ f)(t) - φ_n(x)|$, so supremum from ©A-measurable functions).

		Fifthly $k, n \in ®N$: $A_n^k := f^{-1}(U(x_n, 1 / k)) = \{t \in M\middle| \|f(t) - x_n\| < 1 / k\} \in ©A$ by fourthly.
		$$ \bigcup_n A_n^k = M, \qquad B_n^k = A_n^k \setminus \bigcup_{j < n} A_j^k \in ©A, \qquad \bigcup_n B_n^k = M, $$
		and $\{B_n^k, n \in ®N\}$ is pair-wise disjoint.

		Define $g_k(t) = x_n$, $t \in B_n^k$. Then $\|g_k(t) - f(t)\| < \frac{1}{k}$. So $g_k \rightrightarrows f$ on $M$. $g_n$ is strongly measurable. $g_k = \lim_{n \rightarrow ∞} \sum_{j=1}^n x_j χ_{B_j^k}$, so, by the previous lemma $f$ is strongly ©A-measurable.
	\end{dukazin}
\end{veta}

\begin{definice}[Strongly $μ$-measurable]
	$(Ω, Σ, μ)$ complete measure space, $f: Ω \rightarrow X$ is strongly $μ$-measurable if $\exists (u_n)$ simple measurable such that $u_n \rightarrow f$ point-wise $μ$-almost everywhere.
\end{definice}

\begin{poznamka}
	$f$ is strongly $μ$-measurable $\Leftrightarrow$ $\exists g$ strongly $Σ$-measurable: $f = g$ almost everywhere.

	\begin{dukazin}
		„$\impliedby$“ obvious. „$\implies$“: $u_n$ simple measurable, $u_n \rightarrow f$ almost everywhere. $\exists N, μ(N) = 0: u_n \rightarrow f$ on $Ω \setminus N$. Modify $u_n, f$: $v_n = 0$ on $N$ and $u_n$ on $Ω \setminus N$, $g = 0$ on $N$ and $f$ on $Ω \setminus N$. $v_n$ simple measurable, $v_n \rightarrow g$.
	\end{dukazin}
\end{poznamka}

% 05. 01. 2024

\begin{definice}[Essentially separably valued]
	(If $f: Ω \rightarrow X$ is (strongly) $μ$-measurable, then)
	$$ \exists Y ≤ X \text{ separable } \exists N \in Σ: μ(N) = 0 \land f(Ω \setminus N) \subset Y. $$
\end{definice}

\begin{lemma}
	Let $(f_n)$ be a sequence of strongly $μ$-measurable functions $f_n: M \rightarrow X$ almost everywhere converging to a function $f: M \rightarrow X$. Then $f$ is strongly $μ$-measurable as well.

	\begin{dukazin}
		TODO!!! (notes)
	\end{dukazin}
\end{lemma}

\begin{veta}[Pettis]
	Let $f: Ω \rightarrow X$ be a function. Then following assertions are equivalent
	\begin{enumerate}
		\item $f$ is strongly $μ$-measurable.
		\item $f$ is Borel $μ$-measurable and essentially separably valued.
		\item $f$ is weakly $μ$-measurable and essentially separably valued.
	\end{enumerate}

	\begin{dukazin}
		TODO!!! (Notes)
	\end{dukazin}
\end{veta}

\subsection{Integrability of vector-valued functions}
\begin{definice}[Integrable over set, integral over set, integrable, Bochner integrable, Bochner integral, weakly integrable]
	Let $f: Ω \rightarrow X$ be a simple measurable function of the form $f = \sum_{j=1}^k x_j χ_{E_j}$. Let $E \in Σ$. We say that $f$ is integrable over $E$, if for each $j \in [k]$ one has either $μ(E \cap E_j) < ∞$ or $x_j = ¦o$. By the integral of $f$ over $E$ we mean the element of $X$ defined by the formula
	$$ \int_E f dμ = \sum_{j=1}^k μ(E \cap E_j) x_j, \qquad (∞·¦o = ¦o). $$
	If $f$ is integrable over $Ω$, it is called integrable.

	Let $f: Ω \rightarrow X$ be strongly $μ$-measurable. Then function $f$ is said to be Bochner integrable if there exists a sequence $(f_n)$ of simple integrable functions such that
	$$ \lim_{n \rightarrow ∞} \int_Ω \|f_n(ω) - f(ω)\| dμ(ω) = 0. $$
	By Bochner integral of $f$ we then mean the element of $X$ defined by
	$$ (B) \int_Ω f dμ = \lim_{n \rightarrow ∞} \int_Ω f_n dμ. $$

	A function $f: Ω \rightarrow X$ is said to be weakly integrable if $φ ∘ f$ is integrable (i.e., $φ ∘ f \in L^1(μ)$) for each $φ \in X^*$.
\end{definice}

\begin{tvrzeni}[Basic properties of the Bochner integral]
	a) Integrable functions form a vector space, and the mapping assigning to a simple integrable function $f$ its integral $\int_Ω f dμ$ is linear.

	b) Let $f$ be a simple measurable function. Then $f$ is integrable if and only if the function $ω \mapsto \|f(ω)\|$ is integrable. In this case $\left\|\int_Ω f dμ\right\| ≤ \int_Ω \|f(ω)\| d μ(ω)$.

	c) The limit defining the Bochner integral does exist and does not depend on the choice of the sequence $(f_n)$.

	d) Bochner integrable functions form a vector space and the mapping assigning to a Bochner integrable function its Bochner integral is linear.

	e) If $f: Ω \rightarrow X$ is Bochner integrable, then the function $ω \mapsto \|f(ω)\|$ is integrable and $\left\|(B)\int_Ω f dμ\right\| ≤ \int_Ω \|f(ω)\| d μ(ω)$.

	f) If $f: Ω \rightarrow X$ Bochner integrable, then $χ_E·f$ is Bochner integrable for each $E \in Σ$. (The value $(B) \int_Ω χ_E · f dμ =: (B) \int_E f dμ$ is called the Bochner integral of $f$ over $E$.)

	\begin{dukazin}
		TODO!!!
	\end{dukazin}
\end{tvrzeni}

\begin{veta}[A characterization of Bochner integrability]
	Let $f: Ω \rightarrow X$ be a strongly $μ$-measurable function. Then $f$ is Bochner integrable if and only if $\int_Ω \|f(ω)\| dμ(ω) < ∞$.

	\begin{dukazin}
		TODO!!! (notes)
	\end{dukazin}
\end{veta}

\begin{veta}[Lebesgue dominated convergence theorem for Bochner integral]
	Let $(f_n)$ be a sequence of Bochner integrable functions $f_n: Ω \rightarrow X$ almost everywhere converging to a function $f: Ω \rightarrow X$. Let $g: Ω \rightarrow ®R$ be an integrable function such that for each $n \in ®N$ one has $\|f_n(Ω)\| ≤ g(ω)$ for almost all $ω \in Ω$. Then $f$ is Bochner integrable and $(B) \int_Ω f dμ = \lim_{n \rightarrow ∞} (B) \int_Ω f_n dμ$.

	\begin{dukazin}
		TODO!!! (notes)
	\end{dukazin}
\end{veta}

\begin{tvrzeni}[Absolute continuity of Bochner integral]
	Let $f: Ω \rightarrow X$ be Bochner integrable. Then
	$$ \forall ε > 0\ \exists δ > 0\ \forall E \in Σ: μ(E) < δ \implies \left\|\int_E f dμ\right\| < ε. $$

	\begin{dukazin}
		The proof is not needed for the exam.
	\end{dukazin}
\end{tvrzeni}

% 09. 01. 2024

\begin{poznamka}
	$(Ω, Σ, μ)$ complete measure space, $X$ Banach space, $f: Ω \rightarrow X$ is weakly integrable $≡$ $\forall φ \in X^*: φ ∘ f$ is integrable.
\end{poznamka}

\begin{tvrzeni}[Weak integral]
	$f$ weakly integrable $\implies$ $F(φ) = \int_Ω φ ∘ f dμ$, $φ \in X^*$ belongs to $X^{**}$.

	\begin{poznamkain}
	Then
	\begin{itemize}
		\item $(D) \int_Ω f dμ := F$;
		\item $E \in Σ$, then $(D) \int_E f dμ := (D) \int_Ω χ_E · f dμ$;
		\item $f$ is Pettits-integrable if $\forall E \in Σ: (D) \int_E f dμ \in κ(X)$.
	\end{itemize}
	\end{poznamkain}

	\begin{dukazin}
		TODO!!!
	\end{dukazin}
\end{tvrzeni}

\begin{tvrzeni}
	Let $f: Ω \rightarrow X$ be Bochner integrable. Let $L: X \rightarrow Y$ be bounded linear operator. Then $L ∘ f$ is Bochner integrable and $(B) \int_Ω L ∘ f dμ = L((B) \int_Ω f dμ)$.

	\begin{dukazin}
		„1. $L ∘ f$ is strongly $μ$-measurable.“: $f$ strongly $μ$-measurable $\implies$ $\exists s_n$ simple measurable $s_n \rightarrow f$ almost everywhere. The $L ∘ s_n$ are simple measurable and $L ∘ s_n \rightarrow L ∘ f$ almost everywhere.

		2. $f$ Bochner integrable $\implies$ $\exists (s_n)$ simple integrable such that $\int_Ω \|s_n - f\| dμ \rightarrow 0$. The $L∘s_n$ are simple integrable
		$$ \int \|(L ∘ s_n)(ω) - (L ∘ f)(ω)\| dμ(ω) = \int \|L(s_n(ω) - f(ω))\| dμ(ω) ≤ $$
		$$ ≤ \int \|L\|·\|s_n(ω) - f(ω)\| dμ(ω) \rightarrow 0 \implies $$
		$\implies$ $L ∘ f$ is Bochner integrable.

		$$ 3. (B) \int_Ω L ∘ f dμ = \lim_{n \rightarrow ∞} \overbrace{\int_Ω L ∘ s_n dμ}^{\sum L(x_j) μ(E_j)} = \lim_{n \rightarrow ∞} L\overbrace{\underbrace{\(\int_Ω s_n dμ\)}_{\rightarrow (B) \int_Ω f dμ}}^{\sum x_j μ(E_j)} = L\((B) \int_Ω f dμ\), $$
		where $s_n = \sum_{j=1}^k x_j χ_{E_j}$, $L∘s_n = \sum_{j=1}^n L(x_j), χ_{E_j}$.
	\end{dukazin}
\end{tvrzeni}

\begin{dusledek}
	$f$ Bochner integrable $\implies$ $f$ is Pettis integrable.
\end{dusledek}

\begin{priklad}
	$Ω = ®N$, $Σ = ©P(®N)$, $μ$ the contg measure. $f : Ω \rightarrow X$

	a) $f$ is Bochner integrable $\sum f(n)$ is absolutely convergent.
	
	\begin{dukazin}
		$\sum f(n)$ absolutely convergent $\Leftrightarrow$ $\int_Ω \|f(n)\| dμ(n) = \sum \|f(n)\| < ∞$ $\Leftrightarrow$ $f$ Bochner integrable.
	\end{dukazin}

	b) If $\sum f(n)$ is unconditionally convergent then $f$ is Pettis integrable

	\begin{dukazin}
		$\sum_{n \in ®N} f(n)$ is unconditionally convergent $\implies$ $\forall A \subset ®N: \sum_{n \in A} f(n)$ unconditionally convergent $\implies$
		$$ \implies \forall A \subset ®N\ \forall φ \in X^*\!: \sum_{n \in A} φ(f(n)) \text{ unconditionally convergent} \implies \!\! \sum_{n \in A} |φ(f(n))| < ∞. $$
		$\implies$ $f$ is weakly integrable. Moreover $\sum_{n \in A} φ(f(n)) = φ\(\sum_{n \in A} f(n)\)$ $\implies$ $f$ is Pettis integrable.
	\end{dukazin}
\end{priklad}

\begin{priklad}
	$f: ®N \rightarrow l_2$, $f(n) = \frac{e_n}{n}$. Then $f$ is Pettis integrable, not Bochner integrable ($f$ strongly measurable).

	$g: ®N \rightarrow C_0$, $g(n) = e_n$. Then $g$ is weakly integrable not Pettis integrable. The weak integral is $(1)_{n=1}^∞ \in l^∞$.
\end{priklad}

\subsection{Lebesgue–Bochner spaces}
\begin{definice}
	$f: Ω \rightarrow X$ strongly $μ$-measurable.

	$p \in [1, ∞)$: $f \in L^p(μ; X)$ if $\int_{Ω} \|f(ω)\|^p dμ(ω) < ∞$, $\|f\|_p = \(\int_Ω \|f(ω)\|^p dμ(ω)\)^{1 / p}$.
	
	$p = ∞$: $f \in L^∞(μ; X)$ if $ω \mapsto \|f(ω)\|^p$ is essentially bounded, $\|f\|_∞ = \esssup_{ω \in Ω} \|f(ω)\|^p$.
\end{definice}

\begin{poznamka}
	1. Simple integrable functions belong to $L^p(μ; X)$, $p \in [1, ∞)$.
	$$ \left\|\sum_{j=1}^n x_i χ_{e_i}\right\|_p = \(\sum_{j=1}^n \|x_j\|^p μ(E_j)\)^{1 / p}. $$

	2. Simple measurable functions belong to $L^∞(μ; X)$.
	$$ \|\sum_{j=1}^n x_j χ_{E_j}\|_∞ = \max \{\|x_j\|, μ(E_j) > 0\}. $$

	3. $p \in [1, ∞]$, $h \in C^p(μ)$, $x \in X$. Then $f(ω) = h(ω)·x$, $ω \in Ω$ ($f = h·x$), $f \in L^p(μ; X)$, $\|f\|_p = \|h\|_p·\|x\|$.
\end{poznamka}

\begin{veta}
	a) $(L^p(μ; X), \|·\|_p)$ is a Banach space.
	\begin{dukazin}
		It is a NLS by the Minkowski inequality. Completeness: $p = ∞$ easy, $p \in [1, ∞)$: Assume $(f_n) \subset L^p(μ; X)$, $\sum_{n=1}^∞ \|f_n\|_p < ∞$. Define $g_n(ω) = \|f_n(ω)\|$, $ω \in Ω$. Then $g_n \in L^p(μ)$, $\|g_n\|_p = \|f_n\|_p$. So, $\sum_{n=1}^∞ \|g_n\|_p < ∞$, $L^p(μ)$ complete $\implies$ $g := \sum_{n=1}^∞ g_n \in L^p(μ)$ $\implies$ $\exists$ a subsequence of the sequence of partial sums which converges almost everywhere. But $g_n ≥ 0$, so $g(ω) = \sum_{n=1}^∞ g_n(ω)$ almost everywhere. So, for almost all $ω \in Ω$. We have $\sum g_n (ω) < ∞$, hence $\sum \|f_n(ω)\| < ∞$, so $f(n) = \sum_{n=1}^∞ f_n(ω)$ is defined almost everywhere.

		Then $f$ is strongly $μ$-measurable. $\|f(ω)\| ≤ \sum_{n=1}^∞ \|f_n(ω)\| = g(ω)$. Since $g \in L^p(ω)$, we deduce $f \in L^p(μ; X)$.
		$$ \|f(ω) - \sum_{k=1}^n f_k(ω)\| = \|\sum_{k > n} f_k(ω)\| ≤ \sum_{k > n} \|f_k(ω)\| \text{ almost everywhere } \implies $$
		$$ \implies \|f - \sum_{k=1}^n f_n\|_p ≤ \|ω \mapsto \sum_{k > n} \|f_k(ω)\| \|_p ≤ $$
		$$ ≤ \sum_{k > n} \|ω \mapsto \|f_n(ω)\| \|_p = \sum_{k > n} \|f_n\|_p \overset{n \rightarrow ∞}\longrightarrow 0. $$
		So $f = \sum_{k=1}^∞ f_k$ is $L^p(μ; X)$.
	\end{dukazin}

	b) $L^1(μ; X) =$ Bochner-integrable functions. (se the definition and a theorem above.)

	c) If $X$ is a Hilbert space. Then $L^2(μ; X)$ is also a Hilbert space.
	$$ \<f, g\> := \int_Ω \<f(ω), g(ω)\> dμ(ω). $$

	\begin{dukazin}
		$ω \mapsto \<f(ω), g(ω)\>$ is (strongly) measurable. $ω \mapsto \<f(ω), g(ω)\>$ is integrable
		$$ \int |\<f(ω), g(ω)\>| ≤ \int \|f(ω)\|·\|g(ω)\| ≤ \(\int \|f(ω)\|^2\)^{\frac{1}{2}} · \(\int \|g(ω)\|^2\)^{\frac{1}{2}} = \|f\|_2·\|g\|_2 < ∞. $$
	\end{dukazin}
\end{veta}

\begin{veta}
	$p \in [1, ∞)$.

	a) Simple integrable function are dense in $L^p(μ; X)$.

	\begin{dukazin}
		$f \in L^p(μ; X)$ $\implies$ $f$ is strongly measurable. Find $(s_n)$ simple measurable, $s_n \rightarrow f$ almost everywhere. Define $f_n : Ω \rightarrow X$:
		$$ f_n(ω) := \begin{cases}s_n(ω) & \text{if } \|f(ω) - s_n(ω)\| < 2·\|f(ω)\|, \\ 0 & \text{otherwise}.\end{cases} $$
		Then $f_n$ are simple measurable, $f_n \rightarrow f$ almost everywhere (a proof above). $\|f_n(ω) - f(ω)\| ≤ 2·\|f(ω)\|$. So,
		$$ f_n - f \in L^p(μ; X) \implies f_n \in L^p(μ; X) \implies f_n \text{ simple integrable}. $$
		$\|f_n(ω) - f(ω)\|^p \rightarrow 0$ almost everywhere, $2·\|f(ω)\|^p$ is a majorant $\implies$
		$$ \int \|f_n(ω) - f(ω)\|^p dμ(ω) \rightarrow 0 \qquad f_n \rightarrow f \text{ in } L^p(p; X). $$
	\end{dukazin}

% 09. 01. 2024

	b) $L^p(μ)$, $X$ separable $\implies$ $L^p(μ; X)$ is separable.

	\begin{dukazin}
		$\{z_n, n \in ®N\}$ dense in $X$, $\{k_n, n \in ®N\}$ dense in $L^p(μ)$.
		$$ A = \{\sum_{j=1}^k z_{n_k}·k_{m_j}, n_1, …, n_k \in ®N, m_1, …, m_k \in ®N, k \in p\} $$
		is a countable dense set of $L^p(μ; X)$

		1. $A$ is countable subset of $L^p(μ; X)$.

		2. $f \in L^p(μ; X)$, $ε > 0$ by a): $\exists g_1$ simple integrable $\|f - g_1\|_p < \frac{ε}{3}$. Then $g_1 = \sum_{j=1}^k x_j χ_{E_j}$ ($E_j \in Σ$, disjoint, $x_j \in X$, $μ(E_j) < ∞$).

		3. Find $n_1, …, n_k \in ®N$ such that $\|x_j - z_{n_j}\|$ is so small, such that
		$$ \sum_{j=1}^k \|x_j - z_{n_j}\|^p μ(E_j) < \(\frac{ε}{3}\)^p. $$
		Set $g_2 := \sum_{j=1}^k z_{n_j} χ_{E_j}$. Then $\|g_2 - g_1\|_p < \frac{ε}{3}$.

		4. Find $m_1, …, m_k \in ®N$ such that $\|χ_{E_j} - k·m_j\|_p$ is so small that
		$$ \sum_{j=1}^k \|z_{n_j}\|·\|χ_{E_j} - k_{m_j}\|_p ≤ \frac{ε}{3}. $$
		Then $g_3 := \sum_{j=1}^k z_{n_j}·k_{m_j} \in A$, $\|g_3 - g_2\|_p < \frac{ε}{3}$.

		To summarize: $\|g_3 - f\| < ε$.
	\end{dukazin}
\end{veta}

\begin{priklad}
	1. $G \subset ®R^n$ Lebesgue measurable. $μ := λ^n|_G$ for some $μ(G) > 0$. $L^p(G; X) := L^p(μ; X)$. $p \in [1, ∞)$, $X$ separable $\implies$ $L^p(G; X)$ is separable.

	2. $μ = $ the counting measure on ®N $\implies$ $L^p(μ, X) = l^p(X)$. $X$ separable, $p \in [1, ∞)$ $\implies$ $l^p(X)$ separable.
\end{priklad}

\begin{poznamka}[Representation of duals]
	$p$, $p^*$ dual exponents, $p \in [1, ∞)$ ($p^* \in (1, ∞]$).
	$$ \(l^p(X)\)^* ≈ l^{p^*}\(X^*\). $$
	$X$ reflexive, $μ$ $ς$-finite:
	$$ \(L^p(μ; X)\)^* = L^{p^*}\(μ; X^*\). $$

	$X$ reflexive, $μ$ $ς$-finite, $p \in (1, ∞)$: $L^p(μ; X)$ is also reflexive
\end{poznamka}

% 12. 01. 2024 from notes of lecturer

\section{Compact convex sets}
\begin{poznamka}[Convention]
	In this chapter we consider only vector spaces over ®R. It causes no harm, as all the definitions and results can be used for complex spaces as well, because only the structure of real version of the space in question is used.
\end{poznamka}

\begin{definice}[Extreme point]
	Let $X$ be a vector space and let $A \subset X$ be a convex set. A point $x \in A$ is said to be an extreme point of $A$ if it is not an interior point of any segment in $A$, i.e. if
	$$ \forall a, b \in A\ \forall t \in (0, 1): x = t·a + (1 - t)·b \implies a = b = x. $$

	The set of all extreme point of $A$ is denoted $\ext A$.
\end{definice}

TODO?

\begin{definice}[Face]
	Let $X$ be a vector space and let $A \subset X$ be a convex set. A subset $F \subset A$ is said to be a face of $A$ if the following two conditions are fulfilled: 1. $F$ is a nonempty convex subset of $A$; 2. $\forall a, b \in A: \frac{1}{2}(a + b) \in F \implies a \in F \land b \in F$.
\end{definice}

\begin{lemma}[Properties of faces]
	Let $X$ be a vector space and let $A \subset X$ be a convex set
	\begin{enumerate}
		\item $x \in A$ is an extreme point of $A$ if and only if $\{x\}$ is a face of $A$.
		\item If $F_1 \subset A$ is a face of $A$ and $F_2 \subset F_1$ is a face of $F_1$, then $F_2$ is a face of $A$.
		\item If, moreover, $X$ is a HLCS and $A$ is a compact set containing at least two points, then there is a closed face $F \subsetneq A$.
	\end{enumerate}

	\begin{dukazin}
		„1.“ clear from definitions. „2.“: clearly $F_2$ is nonempty and convex, $x, y \in A \land \frac{x + y}{2} \in F_2 \implies x, y \in F_1$ (as $F_2 \subset F_1$ and $F_1$ is a face of $A$.) Hence $x, y \in F_2$ (as $F_2$ is a face of $F_1$).

		„3.“: $x, y \in A$, $x ≠ y$ by Hahn–Banach: $\exists f \in X^*$: $f(x) ≠ f(y)$. Since $A$ is compact and $f$ continuous, if attains maximum on $A$. Let $F = \{a \in A | f(a) = \max f(A)\}$. Then
		\begin{itemize}
			\item $0 ≠ F \subsetneq K$ (as $f(x) ≠ f(y)$, $f$ is not constant on $A$);
			\item $F$ is closed and convex (as $f$ is continuous and linear);
			\item $\frac{x + y}{2} \in F$, $x, y \in A$ $\implies$
				$$ \max f(A) = f\(\frac{x + y}{2}\) = \frac{1}{2}(f(x) + f(y)) ≤ \frac{1}{2}(\max f(A) \max f(A)) = \max f(A) $$
				$\implies$ there are equalities, i.e. $x, y \in F$.
		\end{itemize}
		So, $F$ is a closed face.
	\end{dukazin}
\end{lemma}

\begin{veta}[Krein–Milman]
	Let $X$ be a HLCS and let $K \subset X$ be a convex compact set. Then $K = \overline{\co \ext K}$. (In particular $\ext K ≠ \O$ whenever $K$ is nonempty.)

	\begin{dukazin}
		Suppose $K ≠ \O$. Denote by ©F the family of closed faces in $K$. If $©R \subset ©F$ is linearly ordered by „$\subset$“, then $\bigcap ©R \in ©F$. (The intersection if compact nonempty, convex, the second property of a face is clear.) Then by Zorn's lemma, there is a minimal $F \in ©F$.

		From the previous lemma $F$ is a singleton, i.e. $F = \{x\}$ for some $x \in K$ and $x \in \ext K$.

		Thus $\ext K ≠ \O$. If $K \setminus \overline{\co \ext K} ≠ \O$, fix $x \in K \setminus \overline{\co \ext K}$. By Hahn–Banach separation theorem $\exists f \in X^*: f(x) > \sup f(\ext K)$.

		Let $F = \{y \in K | f(y) = \max f(K)\}$. The $F$ is a closed face. By the first part, we know that $\ext F ≠ \O$, thus there is $y \in \ext F$.

		Since $F \cap \ext K = \O$, we have $y \notin \ext K$. But by the previous lemma, we deduce $y \in \ext K$. \lightning.
	\end{dukazin}
\end{veta}

\begin{tvrzeni}[Minkoski–Carathéodory]
	TODO?
\end{tvrzeni}

\begin{tvrzeni}[Milman]
	Let $X$ be a HLCS and $K \subset X$ a convex compact set. If $A \subset K$ is such that $K = \overline{\co A}$, then $\ext K \subset \overline{A}$.

	\begin{dukazin}
		1. If $A \subset K$ is such that $K = \co A$, then $\ext K \subset A$ by the definition of extreme points.

		2. Let $U$ be an absolutely convex open neighbourhood of ¦o in $X$. Then there is a finite set $F \subset \overline{A}$ with $F + U \supset \overline{A}$ (using compactness of $\overline{A}$).

		$$ \implies K = \overline{\co A} \subset \overline{\co((F + U) \cap K)} = $$
		$$ = \overline{\co \(\bigcup_{x \in F}(x + \overline{U}) \cap K\)} = \co (\bigcup_{x \in F} (x + \overline{U}) \cap K). $$
		($(x + \overline{U}) \cap K$, $x \in F$, are compact convex sets, they are finitely many, so the convex hull of their union is compact.)

		Then by 1. we get $\ext K \subset \bigcup_{x \in F} (x + \overline{U}) \cap K \subset \overline{A} + \overline{U}$. Since $U$ is arbitrary, we get $\ext K \subset \overline{A}$ and we are done. ($x \notin \overline{A}$ $\implies$ $\exists U$ absolutely convex open neighbourhood of ¦o such that $(x + U) \cap \overline{A} = \O$, then $\(x + \frac{1}{2} \overline{U}\) \cap \overline{A} = \O$, so $x \notin \overline{A} + \frac{1}{2}\overline{U}$.)
	\end{dukazin}

	\begin{dukazin}[Union of compact convex sets is compact]
		The set $H_1 \times … \times H_n \times \{(t_1, …, t_n) \in [0, 1]^n | t_1 + … + t_n = 1\}$ is compact and the mapping $(x_1, …, x_n, (t_1, …, t_n)) \mapsto t_1x_1 + … + t_n x_n$ is continuous, its range is $\co(H_1 \cup … \cup H_n)$.
	\end{dukazin}
\end{tvrzeni}

\begin{tvrzeni}[On the barycenter of a measure]
	TODO?
\end{tvrzeni}

\begin{veta}[Krein–Milman theorem on integral representation]
	TODO?
\end{veta}

\begin{tvrzeni}
	TODO?
\end{tvrzeni}

\end{document}
