\documentclass[12pt]{article}					% Začátek dokumentu
\usepackage{../../MFFStyle}					    % Import stylu



\begin{document}

% 03. 10. 2022

\begin{poznamka}
	Topology…
\end{poznamka}

\begin{definice}[Topological vector space (TVS)]
	A Topological vector space over ®F is a pair $(X, τ)$, where $X$ is a vector space over ®F and $τ$ is a topology on $X$ with the following two properties:
	\begin{enumerate}
		\item The mapping $(x, y) \mapsto x + y$ is a continuous mapping of $X \times X$ into $X$;
		\item The mapping $(t, x) \mapsto tx$ is a continuous mapping of $®F \times X$ into $X$;
	\end{enumerate}

	We also denote Hausdorff topological vector space by HTVS. And the symbol $τ(¦o)$ will denote the family of all the neighbourhoods of ¦o in $(X, τ)$.
\end{definice}

\begin{definice}[Locally convex (LCS, HLCS)]
	Let $(X, τ)$ be a TVS. The space $X$ is said to be locally convex, if there exists a base of neighbourhoods of zero consisting of convex sets.
\end{definice}

\begin{poznamka}
	Two homework (in Moodle) and one presentation.
\end{poznamka}

\begin{priklady}
	Let $(X, \|·\|)$ be a normed linear space. Let $τ$ be the topology induced by $\|·\|$. The $(X, τ)$ is HLCS.

	\begin{dukazin}
		$ρ(x, y) = \|x - y\|$ metric induced by $\|·\|$. $τ$ induced by $ρ$. This $τ$ is Hausdorff. Continuity of the operations: (from Funkcionalka)
		$$ x_n \rightarrow x, y_n \rightarrow y, t_n \rightarrow t \implies x_n + y_n \rightarrow x + y \land t_nx_n \rightarrow tx. $$

		So, it is a HTVS. Base of neighbourhood of ¦o is e. g. $U(0, r), r > 0$, which is convex.
	\end{dukazin}

	Let $Γ$ be any nonempty set, $X = ®F^Γ$ (= all functions $Γ \rightarrow ®F$) with point-wise operations, so it is a vector space over ®F. It is a HLCS.

	\begin{dukazin}
		„Continuity of addition:“ $x, y \in ®F^Γ$, $U$ a neighbourhood of $x + y$ $\implies$ $\exists F \subset Γ$ finite $\exists ε > 0$ such that
		$$ U_{¦o} = \{z \in ®F^Γ \middle|\ \forall γ \in F: |z(γ) - (x(γ) + y(γ))| < ε\} \subset U $$
		$$ U_x = \{z \in ®F^Γ \middle|\ \forall γ \in F: |z(γ) - x(γ)| < \frac{ε}{2}\} $$
		$$ U_y = \{z \in ®F^Γ \middle|\ \forall γ \in F: |z(γ) - y(γ)| < \frac{ε}{2}\} $$
		$\implies$ $V_x$ is neighbourhood of $x$, and $V_y$ is neighbourhood of $y$, and $U_x + U_y \subset U_0 \subset U$. Thus $z_1 \in V_x$, $z_2 \in V_y$ $\implies$ $z_1 + z_2 \in U_0 \subset U$.

		„Continuity of multiplication“: $λ \in ®F$, $x \in ®F^Γ$, $U$ a neighbourhood of $λ x$ $\implies$ $\exists F \subset Γ$ finite $\exists μ > 0$ such that
		$$ U_0 = \{z \in ®F^Γ \middle| \forall γ \in F: |z(γ) - λx(γ)| < ε\} \subset U $$
		$$ |μz(γ) - λx(γ)| ≤ |μ|·|z(γ) - x(γ)| + |μ - λ|·|x(f)|. $$
		$$ M:= \max_{γ \in F}|x(γ)|. $$
		$$ V = \{μ \in ®F \middle| |μ - λ| < \frac{ε}{2(M + 1)}\}, \qquad W = \{z \in ®F^Γ \middle| \forall γ \in F: |z(γ) - x(γ)| < \frac{ε}{2(|λ| + \frac{ε}{2(M+1)})}\}. $$
		$$ μ \in V, z \in W \implies μ z \in U_0 \subset U. $$

		„Local convexity“: Base of neighbourhoods of ¦o: $\{x \in ®F^Γ \middle| \forall γ \in F: |x(γ)| < ε\}$, $F \subset Γ$ finite, $ε > 0$, consists of convex sets.

		„Hausdorff“: $x ≠ y \implies \exists γ \in Γ: x(γ) ≠ y(γ)$. Take $ε = \frac{|x(γ) - y(γ)|}{2}$.
		$$ U = \{z \in ®F^Γ \middle| |z(γ) - x(γ)| < ε\}, V = \{z \in ®F^Γ \middle| |z(γ) - y(γ)| < ε\} \implies U \cap V = \O. $$
	\end{dukazin}

	$X = C(®R, ®F) = \{f: ®R \rightarrow ®F \text{ continuous}\}$,
	$$ ρ(f, g) = \sum_{n=1}^∞ \frac{1}{2^n} \min\{1, \max_{t \in [-n, n]}\{|f(t) - g(t)|\}\} =: \sum_{N=1}^∞ \frac{1}{2^N} \min\{1, p_N(f - g)\} $$
	is translation invariant (that implies addition is continuous, see lecture) metric.
	
	\begin{dukazin}
		$f_n \rightarrow f$ in $ρ$ $\Leftrightarrow$ $\forall N: f_n \rightrightarrows f$ on $[-N, N]$.

% 06. 10. 2023

		„$f_n \rightarrow f$, $λ_n \rightarrow λ$ $\implies$ $λ_n f_n \rightarrow λ f$“: Let $N \in ®N$. We will show $λ_n f_n \rightrightarrows λ f$ in $[-N, N]$. $x \in [-N, N]$:
	$$ |λ_n f_n(x) - λ f(x)| ≤ |λ_n|·|f_n(x) - f(x)| + |λ_n - λ|·|f(x)| ≤ c·p_N(f_n - f) + |λ_n - λ|·p_N(f) \rightarrow 0. $$

		Hence, $X$ is HTVS. „Local convexity“: $U_{N, ε} = \{f \in X | p_N(t) < ε\}$, clearly $U_{N, ε}$ is a convex set and $U_{N, ε}$ is neighbourhood of ¦o. If $ε<λ$, then $\{f | ρ(f, ¦o) < \frac{ε}{2^N}\} \subset U_{N, ε}$, because for $ρ(f, ¦o) < \frac{ε}{2^N}$ it is $\frac{1}{2^N}p_N(f) < \frac{ε}{2^N}$. „they form a base“: $f \in U_{N, ε} \implies ρ(f, ¦o) < ε + \frac{1}{2^N}$. Hence fix $r > 0$ and take $N \in ®N$ such that $\frac{1}{2^N} < \frac{r}{2}$. Then $U_{N, \frac{r}{2}} \subset \{f | ρ(f, ¦o) < r\}$
	\end{dukazin}

	$(Ω, Σ, μ)$ a measure space, $p \in (0, 1)$. $L^p(Ω, Σ, μ) = \{f: Ω \rightarrow ®F \text{ measurable} | \int |f|^p dμ < ∞\}$ (we identify functions equal almost everywhere). $ρ(f, g) = \int |f - g|^p dμ$ is a metric making $X = L^p(Ω, Σ, μ)$ a HTVS (but not locally convex).

	\begin{dukazin}
		„$ρ$ is a metric“: „$\triangle$-inequality“: $a, b \in [0, ∞): (a + b)^p ≤ a^p + b^p$. (Fix $a ≥ 0$, take $φ_a(b) = (a + b)^p - a^p - b^p$ $\implies$ $φ_a$ is continuous on $[0, ∞)$, $φ_a(0) = 0$. For $b > 0$: $φ_a(b) = p(a + b)^{p - 1} - pb^{p - 1} = p·\((a + b)^{p-1} - b^{p - 1}\) < 0$ as $p - 1 < 0$ $\implies$ $φ_a$ decreasing on $[0, ∞)$ and $φ_a ≤ 0$.)

		$φ$ is translation invariant $\implies$ addition is continuous. „Multiplication“: We can see that $ρ(λf, ¦o) = |λ|^pρ(f, ¦o)$. $f_n \rightarrow f$, $λ_n \rightarrow λ$:
		$$ ρ(λ_n f_n, λ f) ≤ ρ(λ_n f_n, λ_n f) + ρ(λ_n f, λ f) = |λ_n|^p ρ(f_n, f) + |λ_n - λ|^p ρ(f, ¦o) \rightarrow 0. $$
		Hence, we have a HTVS.
	\end{dukazin}
\end{priklady}

\begin{tvrzeni}[Observation]
	If $(X, τ)$ is a LCS, then $τ$ is translation invariant ($U \subset X, x \in X \implies (U \in τ \Leftrightarrow x + U \in τ)$). Hence $τ$ is determined by $τ(¦o)$.
\end{tvrzeni}

\begin{definice}[convex, symmetric, balanced, absolutely convex, and absorbing set]
	$X$ is a vector space, $A \subset X$. Then $A$ is
	\begin{itemize}
		\item convex if $tx + (1 - t)y \in A$ for $x, y \in A$, $t \in [0, 1]$;
		\item symmetric if $A = -A$;
		\item balanced if $α A \subset A$ for $α \in ®F$, $|α| ≤ 1$;
		\item absolutely convex if it is convex and balanced;
		\item absorbing if $\forall x \in X\ \exists t > 0: \{s X | s \in [0, t]\} \subset A$.
	\end{itemize}
\end{definice}

\begin{definice}
	$\co(A) = $ convex hull, $\b(A) = $ balanced hull, $\aco(A) = $ absolutely convex hull.
\end{definice}

\begin{tvrzeni}
	$X$ is a metric space over ®F, $A \subset X$. Then:

	(a) If $®F = ®R$, it holds $A$ is absolutely convex $\Leftrightarrow$ $A$ is convex and symmetric.

	(b) $\co A = \{t_1x_1 + … + t_kx_k | x_1 … x_k \in A, t_1 … t_k ≥ 0, t_1 + … + t_k = 1, k \in ®N\}$.

	(c) $\b(A) = \{αx | x \in A, α \in ®F, |α| ≤ 1\}$.

	(d) $\aco(A) = \co(\b(A))$.

	(e) $A$ is convex $\Leftrightarrow$ $(s + t) A = sA + tA$ for all $s, t > 0$.

	\begin{dukazin}[a]
		„$\implies$“: trivial (and it also holds for $®F = ®C$). „$\impliedby$“: Assume $A$ is convex and symmetric. We show that $A$ is balanced:
		$$ x \in A, α \in ®R, |α| ≤ 1 \implies α \in [-1, 1]. $$
		And $x \in A, -x \in A$, so the segment from $x$ to $-x$ is contained in $A$ ($α x = \frac{1 - α}{2}(-x) + \frac{(1 + α)}{2} x \in A$).
	\end{dukazin}

	\begin{dukazin}[b]
		„$\subseteq$“: by induction on $k$:
		$$ t_1 x_1 + … + t_{k+1} x_{k+1} = (t_1 + … + t_k) \frac{t_1x_1 + … + t_k x_k}{t_1 + … + t_k} + t_{k+1}x_{k+1}. $$

		„$\supseteq$“: the set on the RHS is convex and contain $A$.
	\end{dukazin}

	\begin{dukazin}[c]
		„$\supseteq$“: clear. „$\subseteq$“: RHS is a balanced set.
	\end{dukazin}

	\begin{dukazin}[d]
		„$\supseteq$“: clear. „$\subseteq$“ the set on the RHS is absolutely continuous (Clearly RHS is convex. „balanced“: using (b) and (c): $\co(\b(A)) = \{t_1 α_1 x_1 + … + t_kα_kx_k | x_1, …, x_k \in A, |α_j| ≤ 1, t_j ≥ 0, t_1 + … + t_k = 1\}$ is clearly balanced.)
	\end{dukazin}

	\begin{dukazin}[e]
		„$\implies$“: „$\subseteq$“: always, „$\supseteq$“: $sa_1 + ta_2 = (s + t)·\(\frac{s}{s + t} a_1 + \frac{t}{s + t} a_2\)$.

		„$\impliedby$“: in particular $\forall t \in (0, 1)$: $tA + (1 - t)A \subset A$, it is the definition of convexity.
	\end{dukazin}
\end{tvrzeni}

\begin{tvrzeni}
	Let $(X, τ)$ be a LCS, $U \in τ(¦o)$. Then
	
	(i) $U$ is absorbing.

	(ii) $\exists V \in T(0): V + V \subset U$.

	(iii) $\exists V \in τ(¦o)$ absolutely convex, open: $V \subset U$.

	\begin{dukazin}[i]
		$$ x \in X \implies 0·x = ¦o \in U \implies \exists V \text{ a neighbourhood of $0$ in ®F}: V·x \subset U \implies \exists t > 0: [0, t] \subset V $$
	\end{dukazin}

	\begin{dukazin}[ii]
		$$ ¦o + ¦o = ¦o \in U \implies \exists W_1, W_2 \text{ neighbourhoods of ¦o}: W_1 + W \subset U. $$
		Take $V = W_1 \cap W_2$.
	\end{dukazin}

	\begin{dukazin}
		$$ \exists U_0 \in τ(¦o) \text{ convex}, U_0 \subset U: ¦o·¦o = ¦o \in U_0 \implies \exists c > 0\ \exists W \in τ(¦o) \text{ open}: $$
		$$ \forall λ, |λ| < c: λW \subset U_0. $$
		$V_1 := \bigcup_{0 < |λ| < 1} λW$. Then $V_1 \in τ(0)$ open, balanced, $V_1 \subset U_0$. Let $V := \co V_1$. Then $V$ is absolutely convex (the previous proposition (d)), $V \subset U_0 \subset U$ (as $V_0$ is convex). $V \in τ(¦o)$ as $V \supset V_1$. „$V$ is open“:
		$$ V = \bigcup\{t_1x_1 + … + t_nx_n + t_{n+1}V_1 | t_1, …, t_{n+1} ≥ 0, t_1 + … + t_{n+1} = 1, x_1, …, x_n \in V_1\} $$
	\end{dukazin}
\end{tvrzeni}

% 10. 10. 2023

\begin{veta}
	1. Let $(X, τ)$ be a LCS. Then there is ©U, a base of neighbourhoods of ¦o with properties:
	\begin{itemize}
		\item the elements of ©U are absorbing, open, absolutely convex;
		\item $\forall U \in ©U\ \exists V \in ©U: 2V \subset U$.
	\end{itemize}
	If $X$ is Hausdorff, then $\bigcap ©U = \{¦o\}$.

	2. Let $X$ be a vector space, ©U a nonempty family of subsets of $X$ satisfying:
	\begin{itemize}
		\item the elements of ©U are absorbing and absolutely convex;
		\item $\forall U \in ©U\ \exists V \in ©U: 2V \subset U$;
		\item $\forall U, V \in ©U\ \exists W \in ©U: W \subset U \cap V$.
	\end{itemize}
	Then there is a unique topology $τ$ on $X$ such that $(X, τ)$ is LCS and ©U is a base of neighbourhoods of ¦o. Further, if $\bigcap ©U = \{¦o\}$, the $τ$ is Hausdorff.

	\begin{dukazin}[1.]
		Let ©U be the family of all open absolutely convex neighbourhoods of ¦o. The previous proposition (iii) gives us ©U is a base of neighbourhoods of ¦o, (1) gives us elements of ©U are absorbing, so the first item holds. (ii) gives us $U \in ©U \implies \frac{1}{2} U \in ©U$.

		Assume $X$ i Hausdorff: $x \in X \setminus \{¦o\} \overset{\text{Hausdorff}}\implies \exists U \in τ(¦o): x \notin U \implies \exists V \in ©U: V \subset U: x \notin V$.
	\end{dukazin}
	
	\begin{dukazin}[2.]
		Set $τ = \{G \subset X | \forall x \in G\ \exists U \in ©U: x + U \subset G\}$. This is a unique possibility so uniqueness is clear.

		„$τ$ is topology“: $\O, X \in τ$ and $τ$ is closed to arbitrary union (clear). $τ$ is closed to finite intersections by third item ($G_1, g_2 \in τ$, $x \in G_1 \cap G_2 \ldots U_1, U_2 \in τ$, $x + U_1 \subset G_1$, $x + U_2 \subset G_2$; $\exists V \in ©U: V \subset U_1 \cap U_2$, then $x + V \subset (x + U_1)\cap (x + U_2) \subset G_1 \cap G_2$ $\implies$ $G_1 \cap G_2 \in τ$).

		„Elements of ©U are neighbourhoods of ¦o“: $U \in ©U$. $V:=\{x \in U | \exists W \in ©U: x + W \subset U\}$. Then $V \subset U$, $0 \in V$ (take $W = U$). $V \in τ$ ($x \in V \implies \exists W \in ©U: x + W \subset U$; let $\tilde W \in ©U$ such that $2\tilde W \subset W$, then $x + \tilde W \subset V$, because $y \in \tilde W \implies X + y + \tilde W \subset x + \tilde W + \tilde W \subset x + W \subset U$).

		„©U is a base of neighbourhood of ¦o“: now clear.

		„$(X, τ)$ is a TVS“: $x + y \in G \in τ \implies \exists U \in ©U: x + y + U \subset G \implies \exists V \in ©U: 2V \subset U.$ Then $(x + V) + (y + V) \subset x + y + 2V \subset x + y + U \subset G$. $λx \in G \in τ \implies \exists U \in ©U: λx + U \subset G$; $\exists V \in ©U: 2V \subset U$; $V$ is absorbing $\implies$ $\exists c > 0\ \forall t \in [0, c]: tx \in V$; $V$ balanced $\implies$ $\forall μ \in ®F, |μ| ≤ c: μ x \in V$; assume $λ \in ®F, |μ - λ| < c, y \in x + \frac{1}{|λ| + 1}V$,
		$$ \implies μy - λx = \underbrace{(μ - λ)y}_{(μ - 1)·\(μ + \frac{1}{|λ| + 1}\)V} + \underbrace{λ(y - x)}_{\in \frac{λ}{|λ| + 1}V \subset V}. $$

		„Local convexity“: by first item: $\forall U \in ©U: U$ is convex.

		Assume $\bigcap ©U = \{¦o\}$. Take $x, y \in X, x ≠ y \implies x - y ≠ ¦o \implies \exists U \in ©U: x - y \notin U$. Take $V \in ©U: 2V \subset U$. Then if $(x + V) \cap (y + V) = \O$, $x + v_1 = y + v_2$, $x - y = v_2 - v_1 \in V + V = 2V \subset U$ \lightning.
	\end{dukazin}
\end{veta}

\begin{veta}
	Let $X$ be a vector space and let ©P be a family of seminorms on $X$. The there is a unique topology $τ$ on $X$ such that $(X, τ)$ is a LCS and $©U = \{\{x \in X | p_1(x) < c_1, …, p_k(x) < c_k\} \middle| p_1, …, p_k \in ©P, c_1, …, c_k > 0\}$ is a base of neighbourhood of ¦o.

	$(X, τ)$ is Hausdorff $\Leftrightarrow$ $\forall x \in X \setminus \{¦o\}\ \exists p \in ©P, p(x) > 0$.

	\begin{dukazin}
		Use the previous theorem (2.) on ©U: The sets are absolutely convex (by properties of seminorms). „Absorbing“: $U = \{x \in X | p_1(x) < c_1, …, p_k(x) < c_k\}$. Take $x \in X$ ?, $j \in [k]$. Then $p_j(x) \in (0, ∞)$ as for $t > 0: p_j(t·x) = t·p_j)_x$ and $\exists c > 0$ such that $c·p_j(x) < c_j$ for $j \in [k]$. Now for $t \in [0, c]: tx \in U$.

		$U = \{x \in X | p_1(x) < c_1, …, p_k(x) < c_k\}$. Take $V = \{x \in X | p_1(x) \subset \frac{c_1}{2}, …, p_k(x) < \frac{c_k}{2}\}$.

		$U, V \in ©U \implies U \cap V \in ©U$ trivially.

		„Hausdorffness“:
		$$ \bigcap U = \{x \in X | \forall p \in ©P: p(x) = 0\}. $$
		„$\supseteq$“ clear. „$\subseteq$“: Assume $y \in X$, $p \in ©P: p(y) > 0$: $U = \{x \in X | p(x) < p(y)\} in ©U \implies y \notin U$.
	\end{dukazin}
\end{veta}

\begin{priklady}
	$(X, \|·\|)$ is a normed space, then its topology is generated by $©P = \{\|·\|\}$.

	The topology on $®F^Γ$ is generated by seminorms $p_γ(f) = |f(γ)|$, $f \in ®F^Γ$ ($γ \in Γ$).

	$C(®R, ®F)$ the topology is generated by this sequence of seminorms: $p_N(f) = \max_{x \in [-N, N]} |f(x)|$.
\end{priklady}

\begin{definice}[Minkowski functional]
	$X$ vector space, $A \subset X$ convex absorbing. Then
	$$ p_A(x) := \inf \{λ > 0 | x \in λ·A\}. $$
\end{definice}

\begin{lemma}
	Let $X$ be LCS, $A \subset X$ convex set.
	$$ x \in \overline{A}, y \in \Int A \implies \{tx + (1 - t)y | t \in [0, 1)\} \subset \Int A. $$

	\begin{dukazin}
		WLOG $y = 0$. $t = 0$ clear, $0 \in \Int A$. $t \in (0, 1)$:

		Fix $U$, an open absolutely convex neighbourhood of ¦o such that $U \subset A$. Then $x + \frac{1 - t}{t} U$ is a neighbourhood of $x$ $\implies$ $\exists $

		TODO!!!
	\end{dukazin}
\end{lemma}

TODO!!!

% 13. 10. 2023

\begin{dukaz}[Continuity of multiplication? Theorem 4. TODO?]
	„$U$ is a neighbourhood of ¦o in $τ$, $λ > 0$ $\implies$ $λU$ is neighbourhood of ¦o“: $λ ≥ 1$: $\exists V \in ©U: V \subset U \implies V \subset λV \subset λU$ ($V$ is absolutely convex) $\implies λU$ is neighbourhood of ¦o. $λ = \frac{1}{2}$: $\exists V \in ©U: V \subset U$, then $\exists W \in ©U: 2W \subset V$, then $W \subset \frac{1}{2} V \subset \frac{1}{2}U \implies \frac{1}{2} U$ is a neighbourhood of ¦o. Now by induction for $λ = \frac{1}{2^n}$. For $λ > 0$ find $n \in ®N$ such that $λ > \frac{1}{2^n}$.

	$λx \in G$ ($λ \in ®F, x \in X, G \in τ$) $\implies \exists U \in ©U: λx + U \in G$. Find $V \in ©U: 2V \subset U$ such that $V$ is absorbing ($\implies \exists c > 0\ \forall t \in [0, c]: tx \in V$) and $V$ is balanced ($\implies \forall μ \in ®F, |μ| ≤ c: μ x \in V$). Let $μ \in F, y \in X$ such that
	$$ |μ - λ| < c \land y \in x + \frac{1}{|λ| + c}V \text{ (a neighbourhood of ¦o)} $$
	$$ \implies μy - λx = μ(y - x) + (μ - λ)x \in V + V = 2V \subset U \implies μy \in λx + U \subset G. $$
\end{dukaz}

\begin{tvrzeni}[8. see notes of lecturer]
	Let $X$ be LCS, $A \subset X$ a convex neighbourhood of ¦o.

	Clearly: $[p_A \subset 1] \subset A \subset [p_A ≤ 1]$.
	\begin{dukazin}
		„$[p_a < 1] = \Int A$“: „$\subseteq$“: $p_A(x) < 1 \implies \exists c > 1$ such that $cx \in A$ $\implies x = \frac{1}{c} cx \in \Int A$. „$\supseteq$“: $x \in \Int A \implies \exists U \in τ(¦o): x + U \subset A$. $U$ absorbing $\implies$ $\exists α > 0: αx \in U$. Then $(1 + α)x \in A \implies p(x) ≤ \frac{1}{1 + α} < 1$.

		„$[p_A ≤ 1] = \overline{A}$“: „$\subseteq$“: $p_A(x) ≤ 1 \implies \forall n \in ®N: p_x\(\(1 - \frac{1}{n}\)x\) = \(1 - \frac{1}{n}\) p_A(x) ≤ 1$. $\(1 - \frac{1}{n}\)x \in \Int A \implies x \in \overline{\Int A} \subset \overline{A}$. „$\supseteq$“: $x \in \overline{A} \implies \forall n \in ®N: \(1 - \frac{1}{n}\)x \in \Int A$, so, $p_A\(\(1 - \frac{1}{n}\)x\) < 1$ $\overset{n \rightarrow ∞}\implies$ $p_A(x) ≤ 1$.
	\end{dukazin}

	$p_A$ is continuous on $X$.
	\begin{dukazin}
		$[p_A < c] = \O$ if $c ≤ 0$ and $c·\Int A$ if $c > 0$. $[p_A > c] = X$ if $c < 0$, $X \setminus (c·\overline{A})$ if $c > 0$, and $\bigcup_{t > 0} X \setminus t \overline{A}$ if $c = 0$. All these sets are open.
	\end{dukazin}

	$p_A = p_{\overline{A}} = p_{\Int A}$.
	\begin{dukazin}
		$\Int A \subset A \subset \overline{A} \implies p_{\overline{A}} ≤ p_A ≤ p_{\Int A}$. „Conversely“: Assume that $p_{\overline{A}}(x) < c$ $\implies$ $\exists d < c: x \in d·\overline{A}$ $\implies$ $\forall n \in ®N: \(1 - \frac{1}{n}\)x \in d \Int A$ $\implies$ $\(1 - \frac{1}{n}\)p_{\Int A}(x) ≤ d$ $\implies$ $p_{\Int A}(x) ≤ d < c$.
	\end{dukazin}
\end{tvrzeni}

\begin{dusledek}
	Any LCS ($X$) is completely regular.

	\begin{dukazin}
		$x \in X$, $U$ an open neighbourhood of $x$. Take $V$ a convex neighbourhood of ¦o such that $x + V \in U$. $f(y) := \min\{1, p_V(y - x)\}$. The $f$ is continuous by the previous proposition, $f(x) = 0$.
		$$ y \in X \setminus U \implies y - x \notin V \implies p_V(y - x) ≥ 1 \implies f(y) = 1. $$
	\end{dukazin}
\end{dusledek}

\begin{veta}
	TODO!!! The topology generated by $©P_τ$ coincides with $τ$.

	\begin{dukazin}
		Let $τ_1$ be topology induced by $©P_τ$. $τ_1 \subset τ$ (seminorms from $©P_τ$ are $τ$-continuous, hence the sets from theorem 5? are $τ$-open). „$τ \subset τ_1$“: Let $U \in τ(¦o)$ $\implies$ $\exists V$ a neighbourhood of ¦o such that $V \subset U$. The $p_V \in ©P_τ$ (from the previous proposition is continuous) $\implies$ $[p_V < 1] = V \subset U \implies U \in τ_1(¦o)$.
	\end{dukazin}
\end{veta}

\begin{tvrzeni}
	$X$ a vector space.
	\begin{enumerate}
		\item $p$ is seminorm $\implies$ $[p < 1]$ is absolutely convex, absorbing, and $p_{[p < 1]} = p$.
		\item $p, q$ are seminorms, then $p ≤ q \Leftrightarrow [p < 1] \supset [q < 1]$.
		\item ©P a set of seminorms generated by a topology $τ$. $p$ a seminorm on $X$. Then $p$ is $τ$-continuous $\Leftrightarrow$ $\exists p_1, …, p_k \in ©P\ \exists c > 0: p ≤ c·\max\{p_1, …, p_k\}$.
	\end{enumerate}

	\begin{dukazin}[1.]
		Absolutely convex and absorbing is clear.
		$$ p_{[p < 1]}(x) = \inf \{λ > 0 | x \in λ[p < 1]\} = \inf \{λ > 0 | x \in [p < λ]\} = p(x). $$
	\end{dukazin}

	\begin{dukazin}[2.]
		„$\implies$“ trivial. „$\impliedby$“: $[p < 1] \supset [q < 1] \implies p = p_{[p < 1]} ≤ p_{[q < 1]} = q.$
	\end{dukazin}

	\begin{dukazin}[3.]
		„$\impliedby$“: $A := [p < 1]$ $\implies$ $A \supset [c·\max \{p_1, …, p_k\} < 1] = \[p_1 < \frac{1}{c}, …, p_k < \frac{1}{c}\]$, which is a $τ$-open set $\implies$ $A$ is a neighbourhood of ¦o $\implies$ $p = p_A$ is continuous (by 1. and the previous proposition).

		„$\implies$“: $p$ is continuous $\implies$ $[p < 1]$ is neighbourhood of ¦o ($p(¦o) = 0$) $\implies$ $\exists p_1, …, p_k \in ©P\ \exists c_1, …, c_k > 0$ such that $[p < 1] \supset [p_1 < c_1, …, p_k < c_k] \supset [p_1 < c, …, p_k < c] = [\frac{1}{c}\max\{p_1, …, p_k\} < 1]$ ($c = \min\{c_1, …, c_k\}$). Use 2. for seminorms $p, \frac{1}{2\max\{p_1, …, p_k\}}$ and get $p ≤ \frac{1}{c} \max\{p_1, …, p_k\}$.
	\end{dukazin}
\end{tvrzeni}

\section{Continuous and bounded linear mapping}
\begin{tvrzeni}
	$(X, τ), (Y, ©U)$ LCS, $L: X \rightarrow Y$ linear. Then the following assertions are equivalent:
	\begin{enumerate}
		\item $L$ is continuous;
		\item $L$ is continuous at ¦o;
		\item $L$ is uniformly continuous.
	\end{enumerate}

	\begin{dukazin}
		„$1. \implies 2.$“ trivial, „$2. \implies 3.$“ assume $L$ continuous at ¦o. Then, given $U \in ©U(¦o)$, there is $V \in τ(¦o)$ such that $L(V) \subset U$. Take $x, y \in X$ such that $x - y \in V$. Then $L(x) - L(y) = L(x - y) \in U$ and that's continuous. „$3. \implies 1.$“ trivial.
	\end{dukazin}
\end{tvrzeni}

\begin{tvrzeni}
	$L: X \rightarrow Y$ linear. $L$ is continuous $\Leftrightarrow$ $\forall q$ a continuous seminorm on $Y$ $\exists p$ a continuous seminorm on $X$: $\forall x \in X: q(L(x)) ≤ p(x)$.

	\begin{dukazin}
		„$\implies$“: $L$ continuous, $q$ a continuous seminorm on $Y$, the $p(x) = q(L(x))$ is a continuous seminorm on $X$. „$\impliedby$“: By the previous proposition it is enough „$L$ is continuous at ¦o“: $U$ neighbourhood of ¦o in $Y$, $\exists V \subset U$ an absolutely convex neighbourhood of ¦o. $q:=p_V$ is a continuous seminorm. Let $p$ be a continuous seminorm on $X$ such that $q ∘ L ≤ p$. $W := [p < 1]$ a neighbourhood of ¦o in $X$ and $L(W) \subset V \subset U$. $x \in W \implies p(x) < 1 \implies q(L(x)) < 1 \implies L(x) \in V \subset U$.
	\end{dukazin}
\end{tvrzeni}

% 17. 10. 2023

TODO!!!

% 20. 10. 2023

TODO!!!

\begin{veta}
	TODO[Theorem 22]!!!

% 24. 10. 2023

	\begin{dukazin}
		„$2. \implies 1.$“ trivial. „$1. \implies 3.$“ if $ρ$ a metric generating $τ$, then $U_n = \{x \in X | ρ(x, 0) < \frac{1}{n}\}$ $\implies$ $(U_n)_n$ is a base of neighbourhoods of ¦o. „$3. \implies 4.$“: (see the proof of the previous proposition, 1.) $(U_n)$ base of neighbourhood of ¦o, take $V_n \subset U_n$ absolutely convex neighbourhood of ¦o, $p_n = p_{V_n} \implies (p_n)$ generate $τ$. „$4. \implies 2.$“: the previous proposition 2.
	\end{dukazin}
\end{veta}

\begin{veta}
	$(X, τ)$ is HLCS. $X$ is normable $\Leftrightarrow$ $\exists U$, a bounded neighbourhood of ¦o.

	\begin{dukazin}
		„$\implies$“: $τ$ generated by $\|·\|$, $U := \{x \in X \middle| \|x\| < 1\}$ is a bounded neighbourhood of ¦o.

		„$\impliedby$“: $U$ bounded neighbourhood of ¦o. WLOG $U$ is absolutely convex. Then $\frac{1}{n}U$, $n \in ®N$ is a base of neighbourhoods of ¦o ($V$ neighbourhood of ¦o, $W \subset V$ an absolutely convex neighbourhood of ¦o $\implies$ $\exists λ > 0: U \subset λW$ Take $n \in ®N$ such that $n > λ$. Then $U \subset n·W$ so $\frac{1}{n}U \subset W \subset V$). Finally, $p_U$ is a norm generating the topology ($U$ absolutely convex neighbourhood of ¦o $\implies$ $p_U$ is a continuous seminorm. $\frac{1}{n}U = [p_U < \frac{1}{n}], n \in ®N$ is a base of neighbourhood of ¦o $\implies$ $p_U$ generated topology of $X$. From $X$ Hausdorff, $p_U$ is a norm.)
	\end{dukazin}
\end{veta}

\section{Fréchet spaces}
\begin{definice}[Fréchet space]
	A LCS whose topology is generated by a complete translation invariant metric is called Fréchet space.
\end{definice}

\begin{priklady}
	$X$ Banach space $\implies$ $X$ Fréchet space. $®F^{®N}, C(®R, ®F), H(Ω)$ are Fréchet spaces.

	\begin{dukazin}[$®F^{®N}$]
		$$ p_n((x_k)) = \max \{|x_k| \middle| k \in [n]\} $$
		seminorms generating the topology, $p_1 ≤ p_2 ≤ …$
		$$ ρ(x, y) = \sum_{n=1}^∞ \frac{1}{2^n} \min\{1, p_n(x - y)\} $$
		is translation invariant metric generating the topology. It is complete: $\(\(x_k^m\)_k\)_{m=1}^∞$ a $ρ$-Cauchy sequence $\implies$ $\forall n \in ®N: \(\(x_k^m\)\)_m$ is $p_n$-Cauchy $\implies$ it is $\|·\|_∞$-Cauchy in $®F^{®N}$ $\implies$ (because $®F^{®N}$ is complete) $\forall n \in ®N: \(x_k^m\)_{k=1}^n \overset{m \rightarrow ∞}\longrightarrow (y_1^n, …, y_n^n) \in ®F^n$.

		Moreover, if $i ≤ n_1 ≤ n_2$, then $y_i^{n_1} = y_i^{n_2}$. So, we have $y = (y_k)_{k=1}^∞ \in ®F^{®N}$, such that $\forall n \in ®N: (x_k^m)_{k=1}^n \overset{m}\rightarrow (y_k)_{k=1}^n$
		$$ \implies \forall n \in ®N: p_n(x^n - y) \overset{m}\rightarrow 0 \implies ρ(x^n, y) \rightarrow 0, \text{ i.e. } x^n \rightarrow y \text{ in } X. $$
	\end{dukazin}

	\begin{dukazin}[$®C(®R, ®F)$]
		$$ p_n(f) = \max_{x \in [-n, n]} |f(x)|. $$
		$(f_k)$ $ρ$-Cauchy $\implies$ $\forall n: (f_k)$ is $p_n$-Cauchy $\implies$ $\forall n: (f_k|_{[-n, n]})$ is $\|·\|_∞$-Cauchy in $C([-n, n)$ $\implies$ $\forall n\ \exists g_n \in C([-n, n])$ such that $f_k |_{[-n, n]} \overset{k}\rightarrow g_n$ in $C([-n, n])$.

		$\forall n_1 ≤ n_2: g_{n_2} |_{[-n_1, n_1]} = g_{n_1}$ so, we have one function $g: ®R \rightarrow ®F$ such that $\forall n \in ®N: g|_{[-n, n] = g_n}$. Then $g$ is continuous, i.e. $g \in C(®R, ®F)$ and $\forall n \in ®N: p_n(f_k - g) \overset{k}\rightarrow 0$. So $p_n(f_k, g) \rightarrow 0$ $\implies$ $f_n \rightarrow g$.
	\end{dukazin}
\end{priklady}

\begin{tvrzeni}
	$(X, τ)$ is a Fréchet space, $ρ$ any translation invariant metric on $X$ generating $τ$ $\implies$ $ρ$ is complete.

	\begin{dukazin}
		$ρ, d$ two translation invariant metrics generating by $τ$. Idea: convergent sequences with respect to $ρ$ and $d$ coincide, Cauchy sequences with respect to $ρ$ and $d$ coincide. $(x_n)$ $ρ$-Cauchy: $ε > 0 \implies \{x | d(x, ¦o) < ε\}$ is a neighbourhood of ¦o $\implies$ $\exists δ > 0: \{x | ρ(x, ¦o) < δ\} \subset \{x | d(x, ¦o) < ε\}$.
		$$ \exists n_0\ \forall m, n > n_0: ρ(x_m - x_n, ¦o) = ρ(x_m, x_n) < δ \implies d(x_m - x_n, 0) = d(x_m, x_n) < ε \implies (x_n) \text{ is $d$-Cauchy.} $$
	\end{dukazin}
\end{tvrzeni}

\begin{tvrzeni}
	$X$ Fréchet, $A \subset X$. $A$ is compact $\Leftrightarrow$ $A$ is closed and totally bounded.

	\begin{dukazin}
		Let $ρ$ be a complete translation invariant metric generating the topology. $A$ is compact $\Leftrightarrow$ $A$ is closed and $ρ$-totally bounded. But $ρ$-totally boundedness = total boundedness in $X$. $A$ is totally bounded in $X$ $\Leftrightarrow$ $\forall U$ neighbourhood of ¦o $\exists F \subset X$ finite $A \subset F + U$. $A$ is totally bounded in $ρ$ $\Leftrightarrow$ $\forall ε > 0\ \exists F \subset X$ finite such that $A \subset \bigcup_{x \in F} U_ρ(x, ε) = F + U_ρ(0, ε)$.
	\end{dukazin}
\end{tvrzeni}

\begin{tvrzeni}
	$X$ LCS, $A \subset X$ totally bounded $\implies$ $\aco A$ is totally bounded.

	\begin{dukazin}
		Let $U$ be a neighbourhood of ¦o. Let $V$ be an absolutely convex neighbourhood of ¦o such that $2V \subset U$ $\implies$ $\exists F \subset X$ finite such that $A \subset F + V$. Then clearly $\aco A \subset (\aco F) + V$. $\aco F$ is compact,
		$$ F = \{x_1, …, x_k\} \implies \aco(F) = \co(\b(F)) = \co\{λ x_j | j \in [k], |λ| ≤ 1\} = \{t_1λ_1x_1 + … t_n λ_n x_n \middle| |λ_j| ≤ 1, t_j ≥ 0, \sum t_j = 1\}. $$

		$$ B = \{(λ_1, …, λ_n, t_1, …, t_n) \middle| |λ_j| ≤ 1, t_j ≥ 0, \sum t_j = 1\} $$
		a compact set in $®F^n \times ®R^n$. $(λ_1, …, λ_n, t_1, …, t_n) \mapsto t_1λ_1x_1 + … + t_nλ_nx_n$ is a continuous map and maps $B$ onto $\aco F$.

		$\aco F$ compact $\implies$ totally bounded $\implies$ $\exists H \subset X$ finite, $\aco F \subset H + V$ So $\aco A \subset \aco F + V \subset H + V + V = H + 2V < H + U$.
	\end{dukazin}
\end{tvrzeni}

\begin{dusledek}
	$X$ Fréchet space, $A \subset X$ compact $\implies$ $\overline{\aco A}$ is compact.

	\begin{dukazin}
		$A$ compact $\implies$ $A$ is totally bounded $\implies$ $\aco A$ is totally bounded $\implies$ (because $M \subset X$ any set $\implies$ $\overline{M} \subset M + U$) $\overline{\aco A}$ is totally bounded $\implies$ $\overline{\aco A}$ is compact.

		($M$ totally bounded, for any $U \in τ(¦o)$: $U$ is neighbourhood of ¦o, $x \in \overline{M}$, $U$ absolutely convex convex neighbourhood of ¦o, then $V \subseteq U$ absolutely convex such that $2V \subset U$ $\implies$ $(x + U) \cap M ≠ 0 \implies x \in M + U$.)

		Find $F$ finite such that $M \subset F + V$ $\implies$ $\overline{M} \subset M + V \subset F + V + V \subset F + U$.
	\end{dukazin}
\end{dusledek}

\begin{veta}[Banach–Steinhaus]
	Let $X$ be a Fréchet space and let $Y$ be LCS. Let $(T_n)$ be a sequence of continuous linear mappings $T_n: X \rightarrow Y$ such that $\forall x \in X: \lim_{n \rightarrow ∞} T_n x$ exists in $Y$. Then $Tx := \lim_{n \rightarrow ∞} T_n x$ define a continuous linear map $X \rightarrow Y$.

	\begin{dukazin}
		Clear: $T$ is a linear map $X \rightarrow Y$. „Continuous“: Fix $q$ any continuous sequence on $Y$.
		$$ A_m = \{x \in X | \forall n \in ®N: q(T_n x) ≤ m\}. $$
		Then $A_m$ is closed, absolutely convex and $\bigcup_{m=1}^∞ A_m = X$.

TODO?

% 27. 10. 2023

	Baire category theorem $\implies$ $\exists m \in ®N: \Int A_m ≠ \O$ $\implies$ $\exists x \in A_m\ \exists U$ an absolutely convex neighbourhood of ¦o such that $x + U \subset A_m$ $\implies$ $-(x + U) \subset A_m$ $\implies$ ($A_m$ convex) $U \subset A_m$ ($y \in U \implies y = \frac{1}{2}(x + y + (-x + y))$) „$\implies$ $q(Ty) ≤ m p_U(y)$“:

	$$ p_U(y) < c \implies \frac{y}{c} \in U \subset A_m \implies \forall n \in ®N q(T_n \frac{y}{c}) ≤ m \implies q(T \frac{y}{c}) ≤ m \implies q(T y) ≤ cm. $$
	\end{dukazin}
\end{veta}

\begin{veta}[Open mapping theorem]
	$X$, $Y$ Fréchet, $T: X \rightarrow Y$ linear continuous surjective mapping. Then $T$ is an open mapping

	\begin{dukazin}
		1. It is enough to show that $\forall U$ neighbourhood of ¦o in $X$: $T(U)$ is a neighbourhood of ¦o in $Y$.

		2. „$\forall U$ a neighbourhood of ¦o in $X$, $\overline{TU}$ is neighbourhood of ¦o in $Y$“: $U$ an neighbourhood of ¦o in $X$. $\exists V \subset U$ an absolutely convex neighbourhood of ¦o. $V$ absorbing $\implies$
		$$ \implies X = \bigcup_{n=1}^∞ n V \implies Y = T(X) = T\(\bigcup_{n=1}^∞ n·V\) = \bigcup_{n=1}^∞ n·T(V). $$
		$Y$ Fréchet $\implies$ by Baire category theorem
		$$ \exists n \in ®N: \O ≠ \Int \overline{n·T(V)} = \Int n·\overline{T(V)} = n·\Int \overline{T(V)} \implies \Int \overline{T(V)} ≠ \O \implies $$
		$\implies$ $\exists y \in Y\ \exists W$ an absolutely convex neighbourhood of ¦o in $Y$ such that $y + W \subset \overline{T(V)}$. $\overline{T(V)}$ is absolutely convex $\implies$ $-(y + w) \subset \overline{T(V)} \implies W \subset \overline{T(V)} \subset \overline{T(U)}$.

		3. „$\forall U$ neighbourhood of ¦o in $X$, $TU$ is a neighbourhood of ¦o in $Y$“: $ρ$ a translation invariant metric on $X$, complete, generating topology. $U_n = \{x \in X | ρ(0, x) < \frac{1}{2^n}\}$. The $U_n$ is a base of neighbourhoods of ¦o. It is enough to prove that $T(U_n)$ is a neighbourhood of ¦o for each $n \in ®N$. We know from 2. that $\forall n: \overline{TU_n}$ is a neighbourhood of ¦o in $Y$. We will be done if we show that $TU_{n-1} \supset \overline{TU_n}$ for each $n \in ®N$.

		We will prove it for $n = 1$: So we will ? $TU_1 \subset TU_0$. Fix $y \in \overline{TU_1}$. Since $\overline{TU_2}$ is a neighbourhood of ¦o $(y - \overline{TU_2}) \cap TU_1 ≠ \O$. So there is $x_1 \in U_1$ such that $y - Tx_1 \in \overline{TU_2}$. $\overline{TU_3}$ is a neighbourhood of ¦o in $Y$ $\implies$ $y - Tx_1 - \overline{TU_3} \subset ap TU_2 = \O$ so, there is $x_2 \in U_2$ such that $y - Tx_1 - Tx_2 \in \overline{TU_3}$.

		By induction we find $x_n \in U_n$ such that
		$$ y - Tx_1 - Tx_2 - … - Tx_n \in \overline{T U_{n+1}} \quad (n \in ®N). $$
		$$ x := \sum_{n=1}^∞ x_n = \lim_{N \rightarrow ∞} \sum_{n=1}^N x_n: $$
		$$ M > N \implies ρ\(\sum_{n=1}^M x_n, \sum_{n=1}^N x_n\) = ρ\(\sum_{n = N+1}^M x_n, ¦o\) ≤ \underbrace{ρ\(\sum_{n=N+1}^M x_n, \sum_{n=N+2}^M\)}_{ρ(x_{N+1}, ¦o)} + \underbrace{ρ\(\sum_{n=N+2}^M x_n, \sum_{n=N+3}^M\)}_{ρ(x_{N+2}, ¦o)} + … + ρ(x_{M}, ¦o). $$

		$$ Tx = y: y - Tx = \lim_{n \rightarrow ∞} (y - Tx_1 - … - Tx_n) $$
		$$ y - Tx_1 - … - Tx_n \in \overline{TU_{N+1}} \subset \overline{TU_k} \qquad \text{for $n+1 > k$} $$
		so, $y - Tx \in \overline{TU_k}$ for each $k \in ®N$, so $y - Tx \in \bigcap_{k=1}^∞ \overline{TU_k} = \{¦o\}$. „Last equality“: $y \in Y \setminus \{¦o\}$ $\implies$ $\exists V$ neighbourhood of ¦o in $Y$ such that $y \notin \overline{B}$. $T$ continuous $\implies$ $\exists k \in ®N$ such that $T(U_k) \subset V$ $\implies$ $\overline{T(U_1)} \subset \overline{V}$ $\implies$ $y \notin \overline{T(U_k)}$.
	\end{dukazin}
\end{veta}

\section{Extension and separation theorems}
\begin{definice}
	$X$ LCS, $X^*$ is the vector space of continuous linear functions on $X$.
\end{definice}

\begin{veta}
	$X$ LCS, $Y \Subset X$, $f \in Y^*$. Then $\exists g \in X^*$ such that $g|_Y = f$.

	\begin{poznamkain}
		If topology of $X$ is generated by ©P a topology of seminorms TODO!!!
	\end{poznamkain}

	\begin{dukazin}
		1. Topology of $Y$: $U \subset Y$ is open $\Leftrightarrow$ $\exists \tilde U \subset X$ open such that $U = \tilde U \cap Y$. $U$ is a neighbourhood of ¦o in $Y$ $\Leftrightarrow$ $\exists \tilde U$ a neighbourhood of ¦o in $X$ such that $U = \tilde U \cap Y$. Lz.pat. $Y$ is also a LSC.

		2. $f \in Y^*$ $\implies$ $\exists p$ a continuous seminorm on $Y$ such that $|f(y)| \subseteq p(y), y \in Y$. $U = [p < 1]$ a neighbourhood of ¦o in $Y$ $\implies$ $\exists \tilde U$ a neighbourhood of ¦o in $X$ such that $\tilde U \cap Y = U$ $\implies$ $\exists \tilde V \subset \tilde U$ an absolutely convex neighbourhood of ¦o in $X$. The $p_{\tilde V}$ is a continuous seminorm on $X$. Moreover, $p_{\tilde V}|_Y ≥ p$. ($[p_{\tilde V}|_Y < 1] \subset \tilde V \cap Y \subset U = [p < 1]$). So, for $y \in Y: |f(y)| ≤ p(y) ≤ p_{\tilde V}(y)$ $\implies$ (algebraic H–B for seminorms) $\exists g: X \rightarrow ®F$ linear, $g|_Y = f$, $|g(x)| ≤ p_{\tilde V}(x)$ for $x \in X$ $\implies$ $g$ is continuous by the proposition above.
	\end{dukazin}
\end{veta}

\begin{dusledek}
	$X$ LCS, $Y \Subset X$ closed, $x \in X \setminus Y$. Then $\exists f \in X^*: f|_Y = 0, f(x) = 1$.

	\begin{dukazin}
		Set $\tilde Y = \LO(Y \cup \{x\})$. Define $g(y  + λx) = λ$, $y \in Y$, $λ \in ®F$ $\implies$ $g$ is linear functional on $\tilde Y$, $g|_Y = 0$, $g(x) = 1$. $\Ker g = Y$ is closed $\implies$ $g$ is continuous $\implies$ $g$ can be extended to $f \in X^*$.
	\end{dukazin}
\end{dusledek}

\begin{dusledek}
	$X$ LCS, $Z \Subset Y \Subset X$.
	$$ \overline{Z} \supset Y \Leftrightarrow \forall f \in X^*: f|_Z = 0 \implies f|_Y = 0. $$

	\begin{dukazin}
		„$\implies$“: clear. „$\impliedby$“: $y \in Y \setminus \overline{Z}$ $\implies$ $\exists f \in X^*: f(y) = 1, f|_Z = 0$.
	\end{dukazin}
\end{dusledek}

\begin{dusledek}
	$X$ HLCS, $x \in X \setminus \{¦o\} \implies \exists f \in X^*: f(x) ≠ 0$.

	\begin{dukazin}
		$Y = \{¦o\}$ is closed linear subspace and use the first corollary.
	\end{dukazin}
\end{dusledek}

\end{document}
