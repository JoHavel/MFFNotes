\documentclass[12pt]{article}					% Začátek dokumentu
\usepackage{../../MFFStyle}					    % Import stylu



\begin{document}

% 04. 10. 2022

\begin{poznamka}[Literature]
	Kechris.
\end{poznamka}

\begin{definice}[Polish space]
	We say TS $(X, τ)$ is polish (PTS) if $X$ is separable and completely metrizable.
\end{definice}

\begin{poznamka}
	% PTS is $(X, τ)$.
	Complete compatible metric is not unique: $\tilde ρ = \min\{1, ρ\}$.
\end{poznamka}

\begin{priklady}
	®R, ®C, $®R^n$, $®C^n$, $2 := \{0, 1\}$, $ω := \{0, 1, 2, …\}$ with discrete topology, Separable Banach space (SBS), metrizable compacts, $2^ω$, $ω^ω$ (both with product topology).
\end{priklady}

\begin{veta}[Baire]
	$X$ TS metrizable with complete metric. Then countable intersection of open dense subsets of $X$ is dense in $X$.

	\begin{dukazin}
		Without proof. (We should know it already.)
	\end{dukazin}
\end{veta}

\begin{veta}
	$X$ complete metric space, $\{F_n\}$ is decreasing sequence of closed subsets of $X$, such that $\diam(F_n) \rightarrow 0$. Then $|\bigcap F_n| = 1$.

	\begin{dukazin}
		Without proof. (We should know it already.)
	\end{dukazin}
\end{veta}

\begin{veta}
	(i) If $X_n$ are PTS, $n \in ω$. Then $\prod_{n \in ω} X_n$ is PTS.

	(ii) $X$ PTS, $H \subset X$. Then $H$ is PTS $\Leftrightarrow$ $H \in ©G_δ(X)$

	\begin{dukazin}[(i)]
		Let $d_n$ be CCM (complete compatible metric) on $X_n$, $n \in ω$. Then
		$$ d(x, y) := \sum_{n=0}^∞ \min\{2^{-n}, d_n(x_n, y_n)\} $$
		is CCM on $X = \prod_{n \in ω} X_n$, where $x = (x_n)$, $y = (y_n)$. („Definition is correct“ is trivial, „$d$ is metric“ straightforward, „$d$ is complete“ also easy, compatibility too).
	\end{dukazin}

	\begin{dukazin}[(ii)]
		$H = \O$, $H = X$ trivial. Assume $H ≠ \O, X$.

		„$\implies$“: Fix CCM $ρ$ on $H$. $V_n := \bigcup\{V \subset X | V \text{ open in $X$} \land V \cap H ≠ \O \land \diam_ρ(V \cap H) < 2^{-n}\}$, $n \in ω$. We want to show $H \overset{?}= \bigcap_{n \in ω}(V_n \cap \overline{H}) \in ©G_δ$:
		$$ \text{„$\subseteq$“: } x \in H, n \in ω, x \in B_ρ(x, 2^{-n-2}) \subset V_n. $$
		„$\supseteq$“: $x \in V_n \cap \overline{H}$ for every $n \in ω$ $\implies$ $\exists$ open sets $G_n$: $x \in G_n$, $G \cap H ≠ \O$, $\diam(G_n \cap H) < 2^{-n}$. We can assume: $G_{n+1} \supset G_n$ (we can use intersection: $G_{n+1} \cap G_n \cap H \overset{?}≠ \O$ $\impliedby$ $x \in G_n \cap G_{n+1} \cap \overline{H} ≠ \O$).

		$\{y\} := \bigcap_{n \in ω} \overline{G_n \cap H}^H \in H$. For contradiction: $x ≠ y$ $\implies$ $\exists O \subset X$ open: $x \notin \overline{O}$, $y \in O$, $G_n \cap H \subset B(y, 2^{-n})$, $n \in ω$. $\implies \exists n \in ω G_n \cap H \subset O$, $x \in G_n \cap (X \setminus \overline{O}) \cap \overline{H}$ $\implies$ $G_n \cap (X \setminus \overline{O}) \cap H ≠ \O$.

		„$\impliedby$“: fix CCM $d$ on $X$, $H = \bigcap_{n \in ω} U_n$, $\O = U_n ≠ X$. $F_n := X \setminus U_n$, $\tilde d(x, y) = d(x, y) + \sum_{n=0}^∞ \min\{2^{-n}, \left|\frac{1}{\dist(x, F_n)} - \frac{1}{\dist(y, F_n)}\right|\}$, $x, y \in H$. Next we verified that $\tilde d$ is metric, that $\tilde d$ is equivalent with $d$ on $H$ (by convergence), and that $(H, \tilde d)$ is complete metric space and separable. TODO?
	\end{dukazin}
\end{veta}

\begin{definice}[Notation]
	$A ≠ 0$:

	\begin{itemize}
		\item $A^{< ω} :=$ finite sequence of elements of $A$ = $\bigcup_{n \in ω} A^n$;
		\item $s \in A^k$, $t \in A^{< ω} \cup A^ω$: $s^\wedge t := (s_0, s_1, …, s_{k-1}, t_0, t_1, …)$, where $s = (s_0, …, s_{k-1})$, $t = (t_0, t_1, …)$;
		\item $s \in A^{<ω} \cup A^ω$: $|s|$ is the number of elements of sequence $s$ ($|s| \in ω \cup \{∞\}$);
		\item $s \in A^{<ω} \cup A^ω$, $k \in ω$, $|s| ≥ k$, then we denote restriction of $s$ on first $k$ elements as $s / k$;
		\item $s \prec t$ iff $|t| ≥ |s|$ and $s = t / |s|$ ($s \in A^{<ω}$, $t \in A^{<ω} \cup A^ω$).
	\end{itemize}
\end{definice}

\section{Baire space $ω^ω$}
\begin{definice}
	For $s \in ω^{<ω}$ we define Baire interval of $s$ as $©N(s) := \{ν \in ω^ω | s \prec ν\}$.

	$©N(s)$ are clopen ($©N(s) = ω^ω \setminus \bigcup\{©N(t) \middle| |t| = |s|, t ≠ s, t \in ω^{<ω}\}$).

	$\{©N | s \in ω^{<ω}\}$ is base of topology of $ω^ω$.
\end{definice}

\begin{veta}[Alexandrov–Urysohn]
	$ω^ω$ is up to homeomorphism unique nonempty multi-dimension PTS such that every compact has empty interior.

	\begin{dukazin}
		Bez důkazu.
	\end{dukazin}
\end{veta}

\begin{dusledek}
	$ω^ω$ is homeomorphic to $®R \setminus ®Q$.
\end{dusledek}

% 11. 10. 2023

\begin{veta}
	Let $X ≠ \O$, PTS. Then $X$ is continuous image of $ω^ω$.

	\begin{poznamkain}
		$X ≠ \O$ PTS. Then there $\exists F \subset ω^ω$, $F$ closed, and continuous injection $φ: F \rightarrow X$.
	\end{poznamkain}

	\begin{dukazin}
		Find CCM on $X$ such that $\diam X ≤ 1$. We inductively construct closed $\O ≠ A_s \subset X$ for every $s \in ω^{< ω}$ such that 1. $A_{\O} = X$; 2. $\diam (A_s) ≤ 2^{-|s|}$; 3. $A_s = \bigcup_{i \in ω} A_{s^\wedge i}$.

		Empty set is trivial. Assume we already have $A_s$. Find $\{x_i | i \in ω\} \subset A_s$ dense in $A_s$. $A_{s^\wedge i} := A_s \cap \overline{B(x_i, 2^{-|s|-2})} ≠ \O$ closed.

		Fix $\forall ν \in ω^ω: f(ν) := x$, where $\{x\} = \bigcap_{k \in ω} A_{ν / k} ≠ \O$ (intersection of closed nonempty non-increasing sequence of sets). „$f$ is surjection“: $x \in A_s \overset{3.} \implies \exists n \in ω: x \in A_{s^\wedge n} \overset{1.} \implies \forall x \in X\ \exists α \in ω^ω\ \forall k \in ω: x \in A_{α / k} \implies x = f(α)$.

		„$f$ continuous“: $f(©N_{ν / k}) \subset A_{ν / k}$ for every $ν \in ω^ω$, $k \in ω$, $\diam A_{ν / k} ≤ 2^{-k}$.
	\end{dukazin}
\end{veta}

\subsection{Cantor set $2^ω$}
\begin{tvrzeni}
	$2^ω$ is up to homeomorphism unique nonempty nuldimensional compact metrizable space without isolated points (without isolated points is called perfect space).
\end{tvrzeni}

\begin{tvrzeni}
	Let $X ≠ \O$ metrizable, compact. Then $X$ is continuous image of $2^ω$.

	\begin{dukazin}
		Without proof, but it is similar to the previous one.
	\end{dukazin}
\end{tvrzeni}

\subsection{Hilbert cube $[0, 1]^ω$}
\begin{tvrzeni}
	Let $X$ be PTS. Then $X$ is homeomorphic to $G_δ$ subset of $[0, 1]^ω$.

	\begin{dukazin}
		$X$ PTS, case $\O$ is trivial, so assume $X ≠ \O$, $ρ$ is CCM on $X$, $ρ ≤ 1$. Let $\{x_n, n \in ω\}$ be dense in $X$. Define $f: [0, 1]^ω: f(x) = (ρ(x, x_n))_{n \in ω}$. $ρ ≤ 1 \implies f(x) \in [0, 1]^ω$.

		„Continuity of $f$“: $f^{-1}(U) = \bigcap_{i = 1}^nB(x_i, b_i) \setminus \overline{B(x_i, a_i)}$ open.

		„Injective“: $x ≠ y \implies \exists n \in ω: ρ(x, x_n) < ρ(y, x_n) \implies f(x) ≠ f(y)$.

		„Continuity of $f^{-1}$“ $f(y^n) \rightarrow f(y) \overset{?} \implies y^n \rightarrow y$.
		$$ f(y^n) \rightarrow f(y) \overset{?} \Leftrightarrow \forall k \in ω: ρ(y^n, x_k) \rightarrow ρ(y, x_k). $$
		Let $ε > 0$ be arbitrary:
		$$ \exists k \in ω: ρ(y, x_k) < \frac{ε}{3}.\ \exists n_0\ \forall n ≥ n_0: ρ(y^n, x_k) < \frac{2ε}{3}. $$
		Then
		$$ \forall n ≥ n_0: ρ(y^n, y) ≤ ρ(y^n, x_k) + ρ(x_k, y) < ε. $$

		So $f(X)$ is homeomorphism to $X$ $\implies$ $f(X)$ is PTS $\implies$ $f(X) \in ©G_δ([0, 1]^ω)$.
	\end{dukazin}
\end{tvrzeni}

\begin{dusledek}
	Let $X$ be compact metrizable space. Then $X$ is homeomorphic to some closed subset of $[0, 1]^ω$.

	\begin{dukazin}
		Compact metrizable space is Polish. And compact subset must be closed.
	\end{dukazin}
\end{dusledek}

\subsection{$©K(X)$: Hyperspace of compact subsets of $X$}
\begin{definice}
	Let $X$ be PTS, denote $©K(X) := \{K \subset X | K \text{ is compact}\}$. Vietoris topology on $©K(X)$ is generated by $\{K \in ©K(X) | K \subset V\}$ for $V$ open and $\{K \cap ©K(X) | K \cap V ≠ \O\} = ©K(X) \setminus \{K \in ®K(X) | K \subset X \setminus V\}$ for $V$ open.
\end{definice}

\begin{tvrzeni}
	Let $X$ be PTS, $ρ$ CCM on $X$, $ρ ≤ 1$. Then mapping $h: ©K(X) \times ©K(X) \mapsto [0, +∞)$ defined as:
	$$ h(K, L) = \begin{cases}0, & K = L = \O,\\ \max\{\sup_{x \in K} ρ(x, L), \sup_{y \in L} ρ(y, K)\}, & K, L ≠ \O,\\1, &\text{other cases},\end{cases} $$
	is CCM on ©K(X) with Vietoris topology. $h$ is known as Hausdorff metric.

	\begin{poznamkain}
		$©K(X)$ is separable if $X$ is PTS. $X$ is compact metrizable $\implies$ $©K(X)$ is compact (totally bounded).

		$X$ is separable $\implies$ $\exists D \subset X: \overline{D} = X, |D| = ω$.
		$$ M = \{K \subset D \middle| |K| < ω\} \implies |M| = ω. $$
		$\overline{M} = ©K(X)$. $K \in ©K(X)$ arbitrary, $ε > 0$ arbitrary. Then $\exists \frac{ε}{2}$ net $P \subset K$, $|P| < ω$. We find $\{\tilde x_0, …, \tilde x_n\} \subset D: ρ(x_i, \tilde x_i) < \frac{ε}{2} \land h(K, \{\tilde x_0, …, \tilde x_n\}) < ε$.

		$X$ is compact, $P$ is $ε$-net in $X$, $|P| < ω$ $\implies$ $2^P$ is finite $ε$-net in $©K(X)$.
	\end{poznamkain}

	\begin{dukazin}
		($\O ≠ K, L, P \in ©K(X)$.) $h$ is metric, definition is correct, $h ≥ 0$ trivial, $h(K, L) = h(L, K)$ trivial, $h(K, L) = 0 \implies K = L$ ($x \notin L \implies ρ(x, L) > 0 \implies K \subset L \land L \subset K$).

		„$\triangle$“ aka „$h(K, L) ≤ h(K, P) + h(P, L)$“: Let $x \in K, y \in L, p \in P$. Then
		$$ ρ(x, L) ≤ ρ(x, y) ≤ ρ(x, p) + ρ(p, y) \qquad \inf y \in L $$
		$$ ρ(x, L) ≤ ρ(x, p) + ρ(p, L) \qquad \sup p \in P $$
		$$ ρ(x, L) ≤ ρ(x, p) + h(P, L) \qquad \inf p \in P $$
		$$ ρ(x, L) ≤ ρ(x, P) + h(P, L) \qquad \inf p \in P $$
		$$ \sup_{x \in K} ρ(x, L) ≤ h(K, P) + h(P, L). $$
		Similarly $\sup_{y \in L} ρ(y, K) ≤ h(K, P) + h(P, L)$.
	\end{dukazin}
\end{tvrzeni}

% 18. 10. 2023

TODO!!!

% 25. 10. 2023

\begin{definice}
	$X$ is metrizable space, $1 ≤ α < ω_1$. We define $Σ_α^0(X)$, $∏_α^0(X)$, and $Δ_α^0(X)$ by induction:
	$$ Σ_1^0(X) := \{U \subset X | U \text{ open}\}, $$
	$$ ∏_α^0(X) := \{A \subset X | X \setminus A \in Σ_α^0(X)\}, $$
	$$ Σ_α^0(X) := \{\bigcup_{n \in ω} A_n | A_n \in ∏_{α_n}^0(X), α_n < α, n \in ω\}, $$
	$$ Δ_α^0(X) := Σ_α^0 \cap ∏_α^0(X). $$

	\begin{poznamkain}[By induction it can be prooven]
		$$ Σ_α^0(X) \subset Σ_β^0(X), ∏_α^0(X) \subseteq ∏_β^0(X), \qquad 1 ≤ α < β < ω_1. $$
	\end{poznamkain}

	\begin{poznamkain}
		$$ \forall α, β: 1 ≤ α < β < ω_1: Σ_α^0(X) \subset ∏_β^0(X). $$
	\end{poznamkain}

	\begin{poznamkain}
		If $X$ contains homeomorphic copy of $2^ω$ then all inclusions are strict.
	\end{poznamkain}

	We denote $Borel(X)$ as $ς$-algebra of Borel sets ($ς$-algebra generated by $Σ_1^0(X)$).
\end{definice}

\begin{poznamka}[Also non-trivial theorem]
	$$ Borel(X) = \bigcup_{1 ≤ α < ω_1} Σ_α^0(X) = \bigcup_{1 ≤ α < ω_1}(X) = \bigcup_{1 ≤ α < ω_1} Δ_α^0(X). $$

	$$ A_n \in \bigcup_{1 ≤ α < ω_1} Σ_α^0(X) \implies \exists 1 ≤ α_n < ω_1: A_n \in Σ_{α_n}^0 (X) \implies A_n \in Σ_{\sup \{α_n | n \in ω\}}^0 \implies \bigcup_{n \in ω} A_n \in Σ_{\sup\{α_n, n \in ω\}}^0 \implies \bigcup A_n \in \bigcup_{1 ≤ α < ω_1} Σ_α^0(X). $$
\end{poznamka}

\begin{poznamka}
	$F_ς = Σ_2^0$, $G_δ = ∏_2^0$, $F_{ςδ} = ∏_3^0$, $G_{δς} = Σ_3^0$.

	$Σ_α^0(X)$ is closed under countable union and $∏_α^0(X)$ under countable intersection.
\end{poznamka}

\begin{veta}
	$X$ be metrizable, $1 ≤ α < ω_1$. Then
	\begin{enumerate}
		\item $Σ_α^0(X)$ is closed under finite intersection;
		\item $∏_α^0(X)$ is closed under finite union.
	\end{enumerate}

	\begin{dukazin}
		„1.“ Firstly for $α = 1$, it is trivial. Then let $A, B \in Σ_α^0(X)$, $α > 1$. Then $A = \bigcup_{n \in ω} A_n$, $A_n \in ∏_{α_n}^0(X)$, $α_n < α$, $B = \bigcup_{m \in ω} B_m$, $B_m \in ∏_{β_m}^0(X)$, $β_n < α$. $A \cap B = \bigcup_{(m, n) \in ω^2} A_n \cap B_m$, $A_n \cap B_m \in ∏_{\max\{α_n, β_n\}}^0(X) \implies A \cap B \in Σ_α^0(X)$. „2.“ $\impliedby$ de Morgan and 1.
	\end{dukazin}
\end{veta}

\begin{veta}
	$X$ be metrizable, $A \subset Z \subset X$, $1 ≤ α < ω_1$. Then $A \in Σ_α^0(Z)$ $\Leftrightarrow$ there exists $\tilde A \in Σ_α^0(X): A = \tilde A \cap Z$. Similarly for $∏_α^0, Δ_α^0$.

	\begin{dukazin}
		Firstly $α = 1$ from definition of subspace. Then assume that it is all true for all $β < α$. We want to prove it for $α$. „$\implies$“:
		$$ A \in Σ_α^0(Z) \implies A = \bigcup A_n, A_n \in ∏_{β_n}^0(Z), β_n < α \implies \exists \tilde A_n \in ∏_{β_n}^0(X): \tilde A_n \cap Z = A_n. $$
		$$ \tilde A = \bigcup \tilde A_n \in Σ_α^0(X), \tilde A \cap Z = Z \cap \bigcup \tilde A_n = \bigcup(Z \cap \tilde A_n) = \bigcup A_n = A. $$
		„$\impliedby$“:
		$$ \tilde A \in Σ_α^0(X), A = \tilde A \cap Z \implies \exists \tilde A_n \in ∏_{β_n}^0(X), β_n < α, \bigcup \tilde A_n = \tilde A. $$
		$$ \tilde A \cap Z \in ∏_{β_n}^0(Z) \implies A = \tilde A \cap Z = \(\bigcup \tilde A_n\) \cap Z = \bigcup\(\tilde A_n \cap Z\) = \bigcup A_n \in Σ_α^0(Z). $$
	\end{dukazin}
\end{veta}

\begin{veta}
	$X$, $Y$ be metric spaces, $f: X \rightarrow Y$ is continuous. If $A \in Σ_α^0(Y)$ ($∏_α^0(Y)$, $Δ_α^0(Y)$) then $f^{-1}(A) \in Σ_α^0(X)$ ($∏_α^0(X)$, $Δ_α^0(Y)$).

	\begin{dukazin}
		$α = 1$ trivial. Assume it holds true for $Σ_β^0(Y)$, $∏_β^0(Y)$, $β < α$, and we want to show for $Σ_α^0(Y)$ ($∏_α^0(Y)$). Let $A \in Σ_α^0(Y)$, $α > 1$ $\implies$ $A = \bigcup_{n \in ω} A_n$, $A_n \in ∏_{β_n}^0(Y)$, $β_n < α$.
		$$ f^{-1}(A) = f^{-1}(\bigcup A_n) = \bigcup \underbrace{f^{-1}(A_n)}_{∏_{β^n}^0(X)} \in Σ_α^0(X), $$
		$$ f^{-1}(Y \setminus A) = f^{-1}(Y) \setminus f^{-1}(A) = X \setminus f^{-1}(A). $$
	\end{dukazin}
\end{veta}

\begin{veta}[Borel classes in PTS]
	$X, Y$ be PTS, $A \in Σ_α^0(X)$, $α ≥ 3$ (resp. $A \in ∏_α^0(X)$, $α ≥ 2$), $B \subset Y$. If $B$ and $A$ are homeomorphic then $B \in Σ_α^0(Y)$ (resp. $∏_α^0$).

	\begin{dukazin}
		$f: A \rightarrow B$ is homeomorphism $A$ onto $B$. The theorem above (name ?) there is extension $\tilde f$ of $f$, $\tilde f$ is homeomorphism $\tilde A$ onto $\tilde B$, $A \subset \tilde A$, $B \subset \tilde B$, $\tilde A \in ∏_2^0(X)$, $\tilde B \in ∏_2^0(Y)$. Then $B \in Σ_α^0(\tilde B)$ (because $B = (f^{-1})^{-1}(A)$). From the theorem above, $\exists \hat{B} \in Σ_α^0(Y): B = \hat{B} \cap \tilde B \in Σ_α^0(Y) \impliedby α ≥ 3$.
	\end{dukazin}
\end{veta}

\subsection{Analytic sets}
\begin{definice}
	$X$ PTS, $A \subset X$. We say that $A$ is analytic set in $X$ if there exists PTS $Y$ and continuous mapping $φ: Y \rightarrow X$ such that $φ(Y) = A$.

	We denote collection of analytic subsets of $X$ as $Σ_1^1(X)$. We say that $A$ is coanalytic in $X$ if $X \setminus A \in Σ_1^1(X)$ and we denote this collection as $∏_1^1(X)$. $Δ_1^1(X) = Σ_1^1(X) \cap ∏_1^1(X)$.
\end{definice}

\begin{priklady}
	$$ Q = \{α \in 2^ω | \exists n \in ω\ \forall j ≥ n: α_j = 0\} = 2^{<ω} \in Σ_2^0(2^ω) \setminus ∏_2^0(2^ω) $$

	TODO?
\end{priklady}

\begin{poznamka}
	$X$ PTS, $F: X \rightarrow ©K(X)$ by $F(x) = \{x\}$. Then $F$ is continuous, $F^{-1}(©K(A)) = A$ $\implies$ if $©K(A) \in Σ_α^0 (©K(X))$ ($∏_α^0$, $Δ_α^0$) then $A \in Σ_α^0(X)$ ($∏_α^0$, $Δ_α^0$). $A$ open $\implies$ $©K(A)$ is open, $A$ is closed $\implies$ $©K(A)$ is closed. $©K(\bigcap A_n) = \bigcap ©K(A_n)$. Thus for $A \in ∏_2^0(X) : ©K(A) \in ∏_2^0(©K(X))$. $A \in Σ_1^0(X)$ ($∏_1^0(X)$, $∏_2^0(X)$) $\Leftrightarrow$ $©K(A) \in Σ_1^0(©K(X))$ ($∏_1^0(©K(X))$, $∏_2^0(©K(X))$).
\end{poznamka}

\begin{veta}
	$X$ PTS, $|X| > ω$. Assume $I \subset ©K(X)$, $I$ is $ς$-ideal ($K \in I, L \subset K \implies L \in I$; $K_n \in I, \bigcup K_n \in ©K(X) \implies \bigcup K_n \in I$). If $I \in ∏_2(©K(X))$, then $I \in Σ_1^1(©K(X))$.
\end{veta}

\begin{dusledek}
	$A \notin ∏_2^0(X) \implies ©K(A) \notin Σ_1^1(©K(X))$.
\end{dusledek}

\begin{poznamka}
	$A \in ∏_1^1(X)$, $©K(A) = ©K(X) \setminus \{K \in ©K(X) | \exists x \in (X \setminus A) \cap K\}$ % =: ©K(X) \setminus ©M$
	$\{(K, x) \in ©K(X) \times X | x \in K\}$ is closed.
\end{poznamka}

% 01. 11. 2023

\begin{definice}
	$$ Σ_1^1(X) := \{A \subset X | \exists Y \text{ PTS}, f: Y \rightarrow X \text{ continuous}: f(Y) = A\}. $$
\end{definice}

\begin{poznamka}
	\begin{itemize}
		\item $\O \in Σ_1^1$;
		\item $∏_2^0(X) \subset Σ_1^1(X)$, $f = \id$;
		\item $X$, $Z$ PTS, $ψ: X \rightarrow Z$ continuous, $A \in Σ_1^1(X) \implies ψ(A) \in Σ_1^1(Z)$;
		\item $Σ_{n+1}^1(X) = \{A \subset X | \exists Y \text{ PTS}, ψ: Y \rightarrow X \text{ continuous}, B \in ∏_n^1(X), A = ψ(B)\}$, $n \in ω \setminus \{0\}$;
		\item $∏_n^1(X) = \{A \subset X | X \setminus A \in Σ_n^1(X)\}$, $Δ_n^1(X) = \sum_n^1(X) \cap ∏_n^1(X)$;
		\item $\bigcup_{n \in ®N} Σ_n^1(X) = \bigcup_{n \in ®N} ∏_n^1 = \bigcup_{n \in ®N}Δ_n^1(x) = ®P(X)$;
		\item $\# ®P(X) ≤ 2^ω$, $®P(X)$ is closed under continuous images and inverse images;
		\item $Σ_1^1(X) = \{A \subset X | \exists ψ: ω^ω \rightarrow X \text{ continuous}: ψ(ω^ω) = A\}$; $Y$ PTS, $f: Y \rightarrow X: f(Y) = A$, $g: ω^ω \rightarrow Y: g(ω^ω) = Y$, $g, f$ are constant. So $ψ = f ∘ g$.
	\end{itemize}
\end{poznamka}

\begin{veta}
	$X$ PTS, $A_n \in Σ_1^1(X)$, $n \in ω$. Then $\bigcup_{n \in ω} A_n, \bigcap_{n \in ω} A_n \in Σ_1^1(X)$.

	\begin{dusledekin}
		Similar for $∏_1^1(X)$.
	\end{dusledekin}

	\begin{dukazin}
		„Union“: Assume $A_n ≠ \O$, $n \in ω$ $\implies$ $φ_n: ω^ω \rightarrow X: φ_n(ω^ω) = A_n$ continuous. Define $φ: ω^ω \rightarrow X$ by $φ(ν_0, ν_1, …) = φ_{ν_0}(ν_1, ν_2, …)$. „$φ$ is continuous“: $ν^j \rightarrow ν \implies \exists n_0 \in ω\ \forall j ≥ n_9: ν^j_0 = ν_0$.
		$$ \lim_{j \rightarrow ∞} φ(ν^j) = \lim_{j \rightarrow ∞} φ_{ν^j_0}(ν^j_1, ν^j_2, …) = \lim_{j \rightarrow ∞} φ_{ν_0}(ν^j_1, …) = φ_{ν_0}(ν_1, …) = φ(ν). $$
		„$φ(ω^ω) = \bigcup_{n \in ω} A_n$“:
		$$ x \in \bigcup A_n \implies \exists n \in ω: x \in A_n \implies \exists ν \in ω^ω: φ_n(ν) = x \implies φ(n^{\wedge}ν) = x. $$
		$$ x \in φ(ω^ω) \implies \exists \tilde ν \in ω^ω: φ(\tilde ν) = x \implies x = φ_{\tilde ν_0}(\tilde ν_1, …) \implies z \in A_{\tilde ν_0} \implies x \in \bigcup A_n. $$
	\end{dukazin}
\end{veta}

\begin{poznamka}[Intersection]
	WLOG: $A_n ≠ \O$, $n \in ω$. $Y := (ω^ω)^ω$, $Y$ PTS by the theorem above (first item). $φ_n: ω^ω \rightarrow X$, meh that $φ_n(ω^ω) = A_n$.
	$$ F:= \{y = (y_0, y_1, …) \in Y | \forall n, m \in ω: φ_n(y_n) = φ_m(y_m)\} = \bigcap_{n, m \in ω}\{y \in Y | φ_n(y_n) = φ_m(y_m)\} = \bigcap))n, m \in ω \{y \in Y | φ_n ∘ π_n(y) = φ_m ∘ π_m(y)\} $$
	intersection of closed, so $F$ is closed and is PTS.

	„$φ_0 ∘ π_0(F) = \bigcap_{n \in ω} A_n$“:
	$$ x \in φ_0 ∘ π_0(F) \implies \exists y \in F: x = φ_0(y_0) = φ_1(y_1) = φ_2(y_2) = … \implies x \in \bigcap_{n \in ω} A_n. $$
	$$ x \in \bigcap A_n \implies \exists y_0, y_1, … \in ω^ω: φ_0(y_0) = x, φ_1(y_1) = x, … \implies y = (y_0, y_1, …) \in F, φ_0 ∘ π_0(y) = x \implies x \in φ_0 ∘ π_0(F). $$
\end{poznamka}

\begin{poznamka}
	$Σ_1^1(X)$ is not closed under complement: $ς(Σ_1^1(X)) \supset Σ_1^1(X) \cup ∏_1^1(x)$.

	$Borel(X) \subset Σ_1^1(X) \cap ∏_1^1(X) = Δ_1^1(X)$.
\end{poznamka}

\begin{veta}
	$X$, $Y$ PTS, $A \in Σ_1^1(X)$ (respective $∏_1^1(X)$), $B \subset Y$, $A$ and $B$ are homeomorphism. Then $B \in Σ_1^1(Y)$ (resp. $∏_1^1(Y)$).

	\begin{dukazin}
		For $Σ_1^1$ trivial. $A \in ∏_1^1(X)$, $φ: A \rightarrow B$ homeomorphism. Then from the theorem above, $\exists \tilde A \in ∏_2^0(X), \tilde B \in ∏_2^0(Y)$ and $\tilde φ: \tilde A \rightarrow \tilde B$ homeomorphism extending $φ$, $A \subset \tilde A$, $B \subset \tilde B$. Then $\tilde A \setminus A = (X \setminus A) \cap \tilde A \in Σ_1^1(X) \implies \tilde B \setminus B \in Σ_1^1(Y)$. $B = Y \setminus (\tilde B \setminus B \cup Y \setminus \tilde B) \in ∏_1^1(Y)$.
	\end{dukazin}
\end{veta}

\begin{veta}
	$X$ PTS. Then $Borel(X) \subset Δ_1^1(X)$.
	
	\begin{dukazin}
		Trivial.
	\end{dukazin}
\end{veta}

\subsection{Luzin theorem}
\begin{veta}[Luzin]
	$X$ PTS, $A_1, A_2 \in Σ_1^1(X)$, $A_1 \cap A_2 = \O$. Then there exists $B \in Borel(X)$, such that $A_1 \subset B \subset X \setminus A_2$.
\end{veta}

\begin{dusledek}
	$X$ PTS. $Δ_1^1(X) = Borel(X)$.

	\begin{dukazin}
		$Δ_1^1(X) \subseteq Borel(X)$ we already have.
		$$ A \in Δ_1^1(X) \implies A \in Σ_1^1(X), X \setminus A \in Σ_1^1 \implies \exists B \in Borel(X): A \subset B \subset X \setminus (X \setminus A) = A \implies A = B \implies A \in Borel(X). $$
	\end{dukazin}
\end{dusledek}

\begin{lemma}
	$C_n, D_n \subset X$, $n, m \in ω$ and $\forall n, m \in ω$ we can separate $C_n, D_m$ by some Borel set. Then we can separate $\bigcup_{n \in ω}C_n$ and $\bigcup_{m \in ω} D_m$ by Borel set.

	\begin{dukazin}
		Let $B_{n, m} \in Borel(X)$ separating $C_n$ from $D_m$ ($C_n \subset B_{n, m} \subset X \setminus D_m$). Put $B := \bigcup_{n \in ω} \bigcap_{m \in ω} B_{n, m}$.
	\end{dukazin}
\end{lemma}

\begin{dukaz}[Luzin theorem]
	Assume $A_1, A_2 ≠ \O$. Then  exists $φ_1, φ_2: ω^ω \rightarrow X$ $φ_i(ω^ω) = A_i$. We assume $A_1$ can't be separated from $A_2$ by any Borel set.
	$$ A_i = φ_i(ω^ω) \implies A_i = \bigcup_{n \in ω} φ_i(©N(n)) \implies \exists ν_0, μ_0 \in ω: φ_i(©N(μ_0)) \text{ can't be separated from } φ_2(©N(ν_0)). $$
	We use lemma again and obtain $μ, ν \in ω^ω$ such that $\forall k \in ω: φ_1(©N(μ / k))$ can't be separated from $φ_2(©N(ν / k))$
	$$ φ_1(μ) \in A_1, φ_2(ν) \in A_2 \implies φ_1(μ) ≠ φ_2(ν) \implies \exists G_1, G_2 \text{ open }, G_1 \cap G_2 = \O $$
	such that $φ_1(μ) \in G_1$, $φ_2(ν) \in G_2$, $φ_1, φ_2$ are continuous $\implies$ $\exists k \in ω: φ_1(©N(μ / k)) \subset G_1$, $φ_2(©N(ν / k)) \subset G_2$ which is continuous.
\end{dukaz}

\begin{priklady}
	$\{f \in C([0, 1]) | \forall x \in [0, 1]: f'(x) \in ®R\} \in ∏_1^1 \setminus Δ_1^1$.

	$\{f \in C([0, 2π) | \text{ Fourier series converges to $f$ for every } x \in [0, 2π]\} \in ∏_1^1 \setminus Δ_1^1$.

	$\{K \in ©K([0, 1]) \middle| |K| ≤ ω\}, \{K \in ©K(®R) | K \subset ®Q\} \in ∏_1^1 \setminus Δ_1^1$.
\end{priklady}

\begin{priklady}
	$\{x \in X | \exists y \in Y: (x, y) \in B\} \in Σ_1^1(X)$.
\end{priklady}

% 08. 11. 2023

TODO!!!

% 15. 11. 2023

\begin{lemma}
	$(X, τ)$ PTS, $F \in ∏_1^0(X)$. Let $τ_F$ be topology generated by $τ \cup \{F\}$. Then $τ_F$ is Polish, $F \in Δ_1^0(X, τ_F)$, $Δ_1^1(X, τ_F) = Δ_1^1(X, τ)$.

	\begin{dukazin}
		$(X, τ_F)$ is homeomorphic with $((X \setminus F) \times \{0\}) \cup (F \times \{1\}) \subset X \times \{0, 1\}$ which is PTS and those two subsets are $G_δ$ in $X \times \{0, 1\}$, so, $(X, τ_F)$ is Polish.

		$$ Δ_1^1(((X \setminus F) \times \{0\}) \cup (F \times \{1\})) \leftrightarrow Δ_1^1(τ_F) = \{A \cup B | A \in Δ_1^1(X \setminus F, τ), B \in Δ_1^1(F, τ)\} \subset Δ_1^1(τ) \subset Δ_1^1(τ_F). $$
	\end{dukazin}
\end{lemma}

\begin{lemma}
	$(X, τ)$ PTS, $(τ_n)_{n \in ω}$ Polish topology, $τ \subset τ_n$, $n \in ω$. Then topology $τ_∞$ generated by $\bigcup_{n \in ω} τ_n$ is polish. If $\forall n \in ω: τ_n \subset Δ_1^1(τ)$, then $Δ_1^1(τ) = Δ_1^1(τ_∞)$.

	\begin{dukazin}
		Set $X_n := (X, τ_n)$, $φ: X \rightarrow \prod_{n \in ω} X_n$, $φ(x) = (x, x, x, x, …)$. $φ$ is homomorphism $(X, τ_∞)$ on $φ(X)$. ($U \in$ base of $τ_∞$ $\implies$ $\exists n \in ω: U \in τ_n, φ(U) = x_1 \times x_2 \times … \times x_{n-1} \times U \times x_{n+1} \times … \cap φ(X)$ is open. $φ(X) \in ∏_1^0(\prod X_n)$ $\implies$ $φ(X)$ PTS $\implies$ $(X, τ_∞)$ PTS.)

		$Δ_1^1(τ) = Δ_1^1(τ_∞) \impliedby ς(ς(M)) = ς(M)$. ($τ_∞ \subset Δ_1^1(τ) = Δ_1^1(τ_n)$.) $τ_∞ \subset \bigcup Δ_1^1(τ_n)$.
	\end{dukazin}
\end{lemma}

\begin{veta}
	$(X, τ)$ PTS, $A \in Δ_1^1(X, τ)$. There exists polish topology $τ_A$ such that $τ \subset τ_A$, $Δ_1^1(τ_A) = Δ_1^1(τ)$ and $A \in Δ_1^0(X, τ_A)$.

	\begin{dukazin}
		$©S := \{D \in Δ_1^1(X) | \text{ exists polish topology $τ_D \supset τ$ and $Δ_1^1(τ_D) = Δ_1^1(τ)$}, D \in Δ_1^0(X, τ_D)\}$. We know that $τ \subset ©S$ and that ©S is closed under complements. Moreover, ©S is closed under countable union ($A_n \in ©S \rightarrow τ_{A_n} \rightarrow τ_∞ = τ_{\bigcup A_n}$). So $©S = Δ_1^1(X, τ)$.
	\end{dukazin}
\end{veta}

\begin{lemma}
	$X$, $Y$ PTS. $f: X \rightarrow Y$ Borel. Then $\graph(f) \in Δ_1^1(X \times Y)$.

	\begin{dukazin}
		Fix compatible complete metric $ρ$ on $Y$. $U_n$, $n \in ω$, countable colection of open balls with $\diam < 2^{-n}$ covering $Y$.
		$$ \graph f \overset?= \bigcap_{n \in ω} \bigcup_{U \in U_n} f^{-1}(U) \times U \in Δ_1^1(X \times Y). $$
		„$\subseteq$“: $(x, y) \in \graph(f) \Leftrightarrow f(x) = y \implies \forall n \in ω\ \exists U \in U_n: y \in U \land x \in f^{-1}(U) \implies (x, y) \in \bigcap_{n \in ω} \bigcup_{U \in U_n} f^{-1}(U) \times U$.

		„$\supseteq$“: $(x, y) \notin \graph(f) \Leftrightarrow f(x) ≠ y \implies \exists n \in ω: ρ(f(x), y) > \frac{1}{n} \implies \exists n \in ω \forall U \in U_n \neg(x \in f^{-1}(U) \cap y \in U) \implies (x, y) \notin \bigcap_{n \in ω} \bigcup_{U \in U_n} f^{-1}(U) \times U$.
	\end{dukazin}
\end{lemma}

\begin{poznamka}[Notation]
	If $f$ is Borel, we write $f \in Δ_1^1$.
\end{poznamka}

\begin{veta}
	$X$, $Y$ PTS, $f \in Δ_1^1(X \times Y)$. If $A \in Δ_1^1(X)$ and $f|_A$ is injective, then $f(A) \in Δ_1^1(Y)$.

	\begin{dukazin}
		If $f: X \rightarrow Y$ is injective, then $f(A) = \prod_Y(\graph(f) \cap A \times Y) \in Σ_1^1(Y)$.
		$$ Y \setminus F(A) = \prod_Y(\graph(f) \cap (X \setminus A) \times Y) \in Σ_1^1(Y) \implies f(A) \in Δ_1^1(Y). $$

		Assume $f$ is continuous, $A \in ∏_1^0(X)$. From the theorem above $A \subset ω^ω$, $B_s := f(©N(s) cap A)$. $\forall s \in ω^{<ω}\ \forall i, j, i ≠ j: B_{s^\wedge i} \cap B_{s^\wedge j} = \O$ $\impliedby$ $f$ is injection. $\forall s \in ω^{<ω}: B_s = \bigcup_{i \in ω} B_{s^\wedge i}$.

		From Luzin separation theorem, there exists (by induction) $(B'_s)_{s \in ω^{<ω}}$ of Borel sets:
		$$ \forall s \in ω^{<ω}\ \forall i, j \in ω, i ≠ j B'_{s^\wedge i} \cap B'_{s^\wedge j} = \O. $$
		(separation $B_{s^\wedge i}$, $\bigcup_{j < i} B_{s^\wedge j} \cup \bigcup_{l > i} B_{s^\wedge l}$) $\forall s \in ω^{<ω}: B_s \subset B'_s$.

		Put: $B^*_{\O} = Y$, $B^*_{s^\wedge j} = B_{s^\wedge j} \cap \overline{B_{s^\wedge j}} \cap B^*_s$. $\forall s \in ω^{<ω}: B^*_s \in Δ_1^1(Y), B_s \subset B^*_s \subset \overline{B_s}$, $B_{s^\wedge j}^* \subset B_s^*$, $B_{s^\wedge j}^* \cap B_{s^\wedge i}^* = \O$, $s \in ω^{<ω}$, $i, j \in ω, i ≠ j$. We proof: $f(A) \overset?= \bigcup_{s \in ω^{<ω}} \bigcap_{k \in ω} B^*_{s / k} = \bigcap_{k \in ω} \bigcup_{s \in ω^k} B_s^* \in Δ_1^1(Y)$.

% 22. 11. 2023

		$B_s^*, s \in ω^{<ω}, B_s^* \in Δ_1^1(Y)$. $f(A) = \bigcap_{k \in ω} \bigcup_{s \in ω^k} B_s^*$:

		„$\subseteq$“: $x \in f(A) \implies \exists ν \in A: f(ν) = x$. Then $x \in f(©N_{ν / k} \cap A) = B_{ν / k} \subset B_{ν / k}^*$, $k \in ω$ $\implies$ $x \in \bigcap_{k \in ω} \bigcup_{s \in ω^k} B_s^*$.

		„$\supseteq$“: $x \in \bigcap_{k \in ω} \bigcup_{s \in ω^k} B_s^* \implies \forall k \in ω\ \exists ν^k \in ω^ω: x \in B_{ν / k}^*$. $\exists ν \in ω^ω: ν^k = ν$, $k \in ω$. $\implies \forall k \in ω\ \exists ν \in ω^ω: f(©N(ν / k) \cap A) ≠ \O$. $\implies$ $\exists ν^k \in ©N(ν / k) \cap A, ν^k \rightarrow ν \implies ν \in A$ ($A$ is closed). $f(ν) = x$? Assume $f(ν) ≠ x$ $\implies$ $\exists U$ neighbourhood of $f(ν)$, such that $x \notin \overline{U}$ $\implies$ ($f$ is continuous) $\exists k_0 \in ω: x \in B_{ν / k_0}^* \subset \overline{f(©N_{ν / k_0}) \cap A} = \overline{B_{ν / k_0}} \subset \overline{U}$ which is contradiction.

		a) Let $f$ is continuous and $A \in Δ_1^1(X)$. On $X$ we find Polish topology $τ_A$ such that $A \in Δ_1^0(τ_A)$, $τ \subset τ_A$ (so $f$ is continuous with respect to $τ_A$), $Δ_1^1(τ) = Δ_1^1(τ_A)$.

		b) Let $f \in Δ_1^1$. Then $f(A) = π_Y(\graph(f) \cap A \times Y)$. Observe that $π_Y$ is injective on $(\graph(f) \cap A \times Y$ if $f$ is injective on $A$.
	\end{dukazin}
\end{veta}

\begin{veta}
	$X$, $Y$ PTS, $f \in Δ_1^1(X \times Y)$.
	\begin{enumerate}
		\item $A \in Σ_1^1(X) \implies f(A) \in Σ_1^1(Y)$;
		\item $B \in Σ_1^1(Y) \implies f^{-1}(B) \in Σ_1^1(X)$;
		\item $B \in ∏_1^1(Y) \implies f^{-1}(B) \in ∏_1^1(X)$.
	\end{enumerate}

	\begin{dukazin}
		„1.“: $f(A) = π_Y((\graph(f) \cap A \times Y)$ is continuous image of $Σ_1^1$ set.
		
		„2.“: $f^{-1}(B) = π_X((\graph(f) \cap X \times B)$ is continuous image of $Σ_1^1$ set.

		„3.“: $f^{-1}(B) = f^{-1}(Y) \setminus f^{-1}(Y \setminus B)$.
	\end{dukazin}
\end{veta}

\subsection{Standard Borel spaces (SBS)}
\begin{definice}[Standard Borel space (SBS)]
	Measurable space $(X, ©S)$ is called standard Borel space (SBS) if there exists Polish topology $τ$ on $X$ such that $Δ_1^1(X, τ) = ©S$.
\end{definice}

\begin{definice}[Effros Borel space]
	Let $X$ be PTS and $©F(X) := ∏_1^0(X)$. Let ©S be $ς$-algebra generated by sets of form $\{F \in ©F(X) | F \cap U ≠ \O\} =: M_U$, where $U \in Σ_1^0(X)$. $(©F(X), ©S)$ is called Effros Borel space.
\end{definice}

\begin{veta}
	$X$ PTS. Then $(©F(X), ©S)$ is SBS.

	\begin{dukazin}
		Without proof.
	\end{dukazin}
\end{veta}

\begin{poznamka}
	$X$ be measurable compact. Then $©F(X)$ can be equipped by Vietoris topology.
\end{poznamka}

\begin{priklad}
	$SB := \{Y \in ©F(C([0, 1])) \middle| Y \text{ is Banach subspace of $C([0, 1])$} \}$. If we restrict Effros $ς$-algebra on SB then SB is SBS.

	$$ SD = \{Y \in SB \middle| Y \text{ has separable dual}\}, $$
	$$ NU = \{Y \in SB \middle| Y \text{ is not universal}\}, $$
	$$ REFL = \{Y \in SB \middle| Y \text{ is reflexive}\}, $$
	$$ NL_1 = \{Y \in SB \middle| Y \text{ does not contain $l_1$}\}. $$
\end{priklad}

\section{Regularity of $Σ_1^1$ sets}
\subsection{Sets with Baire property (BP)}
\begin{definice}[Baire property (BP)]
	$X$ TS, $A \subset X$ has Baire property (BP) in $X$ if there exists open $U \subset X$ and set of 1. category $M \subset X$ such that $A = U \triangle M := (U \setminus M) \cup (M \setminus U)$. Collection of all sets with BP we denote as $Baire(X)$.
\end{definice}

\begin{veta}
	$X$ TS. Then $Baire(X)$ is $ς$-algebra and $Baire(X) \supset Δ_1^1(X)$.

	\begin{dukazin}
		1. „$Baire(X) \supset Σ_1^0(X)$“ trivial. 2. „$Baire(X)$ is $ς$-algebra“: a) „$A \in Baire(X) \overset?\implies X \setminus A \in Baire(X)$“: $A \in Baire(X) \implies \exists G \in Σ_1^0(X)$ and $M$ meagre such that $A = G \triangle M$.
		$$ X \setminus A = X \setminus (G \triangle M) = (X \setminus G) \triangle M = (\Int(X \setminus G) \cup (X \setminus G) \setminus \Int(X \setminus G)) \triangle M = $$
		$$ = (V \cup M_1) \triangle M_2 = V \triangle M \qquad (M = M_1 \triangle M_2). $$
		b) „$A_n \in Baire(X) \overset?\implies \bigcup A_n \in Baire(X)$“: $A_n = G_n \triangle M_n$, $G_n \in Σ_1^0(X)$, $M_n$ meager. $M_n' = G_n \cap M_n$ (meager), $M_n'' = M_n \setminus G_n$ (meager).
		$$ \bigcup A_n = \bigcup((G_n \setminus M_n') \cup M_n'') = ((\bigcup G_n) \setminus M''') \cup \bigcup M_n'', $$
		where $M''' \subset \bigcup_{n \in ω} M_n'$.
	\end{dukazin}
\end{veta}

\begin{lemma}
	$X$ TS, $A \subset X$. Then $A$ is meager iff $\forall x \in A\ \exists V \in Σ_1^0(X)$ such that $x \in V$ and $A \cap V$ is meager.

	\begin{dukazin}
		„$\implies$“ trivial. „$\impliedby$“ ©U denote as maximal collection of disjoint $Σ_1^0$ sets such that $U \cap A$ is meager for $U \in ©U$. We show that $A \cap \bigcup ©U$ is meager, $X \setminus \bigcup ©U$ is nowhere dense, so meager.

		„$X \setminus \bigcup ©U$ is nowhere dense“: By contradiction we assume that there exists $\O ≠ V \in Σ_1^0(X)$, $V \subset X \setminus \bigcup ©U$. Now we have 2 cases: $A \cap V = \O \implies V \in ©U$ contradiction, or $A \cap V ≠ \O \implies \exists x \in A \cap V \implies \exists W \in Σ_1^0(X): x \in W, W \cap A$ is meager $\implies x\in W \cap V ≠ \O, W \cap V \cap A$ is meager $\implies$ $W \cap V \in ©U$ contradiction.

		„$\bigcup ©U \cap A$ is meager“: $©U := \{U_α | α \in I\}$, $U_α \cap A$ meager $\implies$ exist? $F_n^α \in ∏_1^0(X)$ nowhere dense: $U_α \cap A \subset \bigcup F_n^α \subset \overline{U_α}$. We show that $\bigcup_{α \in I} F_n^α$ is nowhere dense:
		$$ a) \bigcup_{α \in I} U_α \setminus F_n^α \in Σ_1^0(X), \quad (\bigcup_{α \in I} U_α \setminus F_n^α) \cap (\bigcup_{α \in I} F_n^α) = \O \impliedby F_n^α \subset \overline{U_α}, \quad \overline{U_α} \cap U_β = \O, α ≠ β $$
		So ©U is disjoint collection, so $\bigcup_{α \in I} U_α F_n^α \cap \overline{\bigcup_{α \in I} F_n^α} = \O$.
		$$ \implies \overline{\bigcup_{α \in I} F_n^α} \subset (\bigcup_{α \in I} (U_α \cap F_n^α)) \cup (X \setminus \bigcup ©U). $$

		b) We assume $\exists V \in Σ_1^0(X)$, $V ≠ \O$, $V \subset \overline{\bigcup_{α \in I} F_n^α}$.
		$$ ? \implies V \not \subset X \setminus \bigcup ©U \overset{a)}\implies V \cap \bigcup_{α \in I} (U_α \cap F_n^α) ≠ \O \implies \exists α \in I: V \cap U_α ≠ \O. $$
		$$ a) \implies V \cap U_α \subset \bigcup_{α \in I}(U_α \cap F_n^α) \overset{\text{©U disjoint}}\implies V \cap U_α \subset F_n^α \text{ \lightning}. $$
	\end{dukazin}
\end{lemma}

% 29. 11. 2023

TODO!!!

% 06. 12. 2023

\subsection{Solecky theorem}
\begin{poznamka}[Notation]
	$X$ PTS, $©I \subset ∏_1^0(X)$.
	$$ ©I^{ext} := \{A \subset X | \exists ©F \subset ©I, |©F| = ω, A \subset \bigcup ©F\}. $$

	\begin{priklady}
		$©I = \{A \subset X \middle| |A| < ω\}$, $©I = \{A \subset X | A \text{ nowhere dense}\}$.
	\end{priklady}

	$$ ©I^{perf} = \{A \subset X | A ≠ \O, \forall U \in Σ_1^0(X): U \cap A ≠ \O \implies U \cap A \notin ©I^{ext}\}. $$
	$$ \Ker A := A \setminus \bigcup \{U \subset X | U \in Σ_1^0(X), U \cap A \in ©I^{ext}\} = $$
	= max perfect subset of $A$ $\impliedby$ $X$ has countable base.
	$$ MGR(A) = \{Z \subset A | Z \text{ be meager in } A\}, \qquad A \subset X. $$
\end{poznamka}

\begin{veta}[Solecki]
	$X$ PTS, $A \in Σ_1^1(A)$, $©I \subset ∏_1^0(X)$. $A \notin ©I^{ext} \implies \exists H \in ∏_2^0(X), H \subset A, H \notin ©I^{ext}$
\end{veta}

\begin{lemma}[For proof of Solecki]
	$A \in Σ_1^1(X) \setminus ©I^{ext}$. Then there exists Suslin scheme $(A_s)_{s \in ω^{<ω}}$ of closed subsets of $X$ such that:
	$$ A_{\O} = \O, \qquad a_s A_s \subset A, \qquad A_s ≠ \O \implies A \cap A_s \in ©I^{perf}, \overline{A \cap A_s} = A_s, \qquad \overline{\bigcup_{n \in ω} A_{s^\wedge n}} = A_s. $$

	\begin{dukazin}
		$(H_s)_{s \in ω^{<ω}}$ closed subsets of $X$, decreasing ($H_s \supset H_{s^\wedge n}, n \in ω$), $A = a_sH_s \impliedby A \in Σ_1^1(X)$. For $s \in ω^{<ω}: L_s := a_t H_{s^\wedge t}, A_s := \overline{\Ker(L_s)}$.
		\begin{enumerate}
			\item $A_\O = \overline{\Ker(L_\O)} = \overline{\Ker(A)} ≠ \O \impliedby A \notin ©I^{ext}$ ($X$ has countable base).
			\item $H_s \searrow \implies L_s \subset H_s \implies \Ker(L_s) \subset H_s \overset{H_s \in ∏_1^0(X)}\implies A_s \subset H_s \implies a_s A_s \subset a_s H_s = A$.
			\item $\Ker(L_s) \subset A_s, L_s \subset A: (A = \bigcup_{|s| = k} L_s, k \in ω \impliedby H_s \searrow) \implies \Ker(L_s) \subset A_s \cap A$, $\overline{\Ker(L_s)} = A_s$.
				$$ A_s = \overline{\Ker(L_s)} \subset \overline{A_s \cap A} \subset \overline{A_s} = A_s. $$
				Assume $A_s ≠ \O \implies A \cap A_s ≠ \O$. $U \in Σ_1^0(X)$, $U \cap A \cap A_s ≠ \O$ $\implies$ $U \cap \Ker(L_s) ≠ \O$ $\implies$ $U \cap \Ker(L_s) \notin ©I^{ext}$. $\implies U \cap A \cap A_s \notin ©I^{ext}$.
			\item $\bigcup_{n \in ω} A_{s^\wedge n} \subset A_s \impliedby (H_s\searrow \implies L_s \searrow \implies A_s \searrow)$. Let $U \in Σ_1^0(X)$, $U \cap A_s ≠ \O$ $\implies$ $U \cap \Ker(L_s) ≠ \O \implies U \cap L_s \notin ©I^{ext}$.
				$$ L_s = \bigcup_{n \in ω} L_{s^\wedge n} \implies \exists n_0 \in ω: U \cap L_{s^\wedge n_0} \notin ©I^{ext} \implies U \cap \Ker(L_{s^\wedge n_0}) \notin ©I^{ext} \implies U \cap A_{s^\wedge n_0} ≠ \O. $$
		\end{enumerate}
	\end{dukazin}
\end{lemma}

\begin{dukaz}[Solecki theorem, not in exam]
	$A \in Σ_1^1(X) \setminus ©I^{ext}$, $(A_s)_{s \in ω^{<ω}}$ from the previous lemma. There are 2 cases:

	„1st case $\exists s \in ω^{<ω}\ \exists U \in Σ_1^0(X): A_s \cap U ≠ \O \land MGR(A_s \cap U) \subset ©I^{ext}$“: Put $\tilde A := A \cap A_s \cap U$. Then from the third item of the previous lemma $\tilde A \in ©I^{perf}$, $\tilde A \in Σ_1^1(X)$. $A_s ≠ \O$, $A \cap A_s \in ©I^{perf}$, $U \cap A_s ≠ \O \implies U \cap A \cap A_s ≠ \O \impliedby \overline{A \cap A_s} = A_s$.
	$$ \implies \tilde A \in Baire(A_s \cap U) \impliedby (A_s \cap U \in ∏_2^0(X)), A_s \cap U \text{ PTS}. $$
	$$ \tilde A = H \cup M, H \in ∏_2^0(A_s \cup U), M \in MGR(A_s \cap U) \subset ©I^{ext} \implies H \notin ©I^{ext}, H \subset A. $$

	„2nd case $\forall s \in ω^{<ω}\ \forall U \in Σ_1^0(X), U \cap A_s ≠ \O: MGR(A_s \cap U) \setminus ©I^{ext} ≠ \O$“: Notation: $©F \subset 2^X: ©F^d := \overline{\bigcup ©F} \setminus \bigcup\{\overline{F} | F \in ©F\}$. Choose CCM $≤ 1$ on $X$. We will inductively construct $φ: ω^{<ω} \rightarrow ω^{<ω}$, $U_s \subset X$, $s \in ω^{<ω}$ such that:
	\begin{enumerate}
		\item $|φ(s)| = |s|$ TODO
		\item $U_s \in Σ_1^0(X)$;
		\item $\diam U_s ≤ 2^{-|s|}$;
		\item $\lim_{n \rightarrow ∞} \diam(U_{s^\wedge n}) = 0$;
		\item $\forall t, s \in ω^{<ω}, t < s, t ≠ s: \overline{U_s} \subset U_t$;
		\item $\forall s \in ω^{<ω}\ \forall m, n \in ω, m ≠ n: U_{s^\wedge m \cap U_{s^\wedge n}} = \O$;
		\item $U_a \cap A_{φ(s)} ≠ \O$;
		\item $\{U_{s^\wedge n} | n \in ω\}^d \notin ©I^{ext}$;
		\item $\{U_{s^\wedge n} | n \in ω\}^d \subset U_s$;
		\item (9. + 5.) $\overline{\bigcup_{n \in ω} U_{s^\wedge n}} \subset U_s$.
	\end{enumerate}
	Construction: $φ(\O) = \O$, $U_\O$ be arbitrary open subset of $X$: $U_{\O} \cap A_{\O} ≠ \O$. Then all items are satisfied. We assume that $U_s$, $φ_s$ are constructed for all $s \in ω^{<ω}$, $|s| ≤ N \in ω$. Let $s \in ω^{<ω}$, $|s| ≤ N$ be arbitrary. From 7th item $U_s \cap A_{φ(s)} ≠ \O$, $MGR(A_{φ(s)} \cap U_s) \notin ©I^{ext}$ $\implies$ $\exists K \subset A_{φ(s)} \cap U_s, K \in ∏_1^0(X)$, nowhere dense in $A_{φ(s)} \cap U_s$, $K \notin ©I^{ext}$. Because
	$$ \exists L \in MGR(A_{φ(s)} \cap U_s) \setminus ©I^{ext} \implies \exists H \in Σ_2^0(X), H \supset L, H \in Σ_2^0(A_{φ(s)} \cap U_s), H \notin ©I^{ext}, $$
	so $H = \bigcup F_n$, $F_n \in ∏_1^0(X)$, nowhere dense in $A_{φ(s)} \cap U_s$ $\implies$ $\exists n_0 \in ω: F_{n_0} = K \notin ©I^{ext}$.

	Find $D \subset A_{φ(s)} \cap U_s$: $D$ is discrete in $X \setminus K$. $D \cap K = \O$. $\overline{D} = K \cup D$. Let $\{y_n\} \subset K$, $\overline{\{y_n\}} = K$, and every element of $\{y_n\}$ repeats infinitely many times. Find $x_n \in (A_{φ(s)} \cap U_s) \setminus K$ such that $ρ(x_n, y_n) < \frac{1}{n}$ (it exists $\impliedby$ $K$ is nowhere dense in $A_{φ(s)} \cap U_s$). Then $D = \{x_n | n \in ω\}$, $D \cap K = \O$, $\overline{D} \supset \overline{D \cup \{y_n | n \in ω\}} \supset D \cup K$, $x \notin K \cup D \implies \exists n \in ω \setminus \{0\}: ρ(x, K) > \frac{1}{n}$ $\implies$ $\#(B(x, 1 / 2n) \cap D) ≤ 2n \implies x \notin \overline{D}$. $\implies$ $\overline{D} = D \cup K$, $D$ is discrete in $X \setminus K$. Assume $x_n ≠ x_m$, $n ≠ m$.

	Define $U_{s^\wedge n}$ as open ball with center $x_n$: $\overline{U_{s^\wedge n}} \subset U_s$. $U_{s^\wedge n} \cap U_{s^\wedge m} = \O$ ($D$ is discrete), $\diam U_{s^\wedge n}≤ 2^{-|s|-1}$, $\lim_{n \rightarrow ∞} \diam U_{s^\wedge n} = 0$, $\overline{\bigcup_{n \in ω} U_{s^\wedge n}} \setminus \bigcup_{n \in ω} \overline{U_{s^\wedge n}} = \{U_{s^\wedge n} | n \in ω\} = K \impliedby \overline{U_{s^\wedge n}} \cap K = \O$, $\overline{D} = K \cup D$. $x_n \in A_{φ(s)} \implies U_{s^\wedge n} \cap A_{φ(s)} ≠ \O$, $\overline{\bigcup_{k \in ω} A_{φ(s)^\wedge k}} = A_{φ(s)} \implies \exists k \in ω: U_{s^\wedge n} \cap A_{φ(s)^\wedge k} ≠ \O$.

	Put $φ(s^\wedge n) = φ(s)^\wedge k$. And then all items are satisfied. $H = \bigcap_{n \in ω} \bigcup_{|s| = n, s \in ω^{<ω}} U_s \in ∏_2^0(X)$, $H \subset A$, $H \notin ©I^{ext}$.
\end{dukaz}

\end{document}
