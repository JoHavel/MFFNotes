\documentclass[12pt]{article}					% Začátek dokumentu
\usepackage{../../MFFStyle}					    % Import stylu



\begin{document}

% 04. 10. 2022

\begin{poznamka}[Literature]
	Kechris.
\end{poznamka}

\begin{definice}[Polish space]
	We say TS $(X, τ)$ is polish (PTS) if $X$ is separable and completely metrizable.
\end{definice}

\begin{poznamka}
	% PTS is $(X, τ)$.
	Complete compatible metric is not unique: $\tilde ρ = \min\{1, ρ\}$.
\end{poznamka}

\begin{priklady}
	®R, ®C, $®R^n$, $®C^n$, $2 := \{0, 1\}$, $ω := \{0, 1, 2, …\}$ with discrete topology, Separable Banach space (SBS), metrizable compacts, $2^ω$, $ω^ω$ (both with product topology).
\end{priklady}

\begin{veta}[Baire]
	$X$ TS metrizable with complete metric. Then countable intersection of open dense subsets of $X$ is dense in $X$.

	\begin{dukazin}
		Without proof. (We should know it already.)
	\end{dukazin}
\end{veta}

\begin{veta}
	$X$ complete metric space, $\{F_n\}$ is decreasing sequence of closed subsets of $X$, such that $\diam(F_n) \rightarrow 0$. Then $|\bigcap F_n| = 1$.

	\begin{dukazin}
		Without proof. (We should know it already.)
	\end{dukazin}
\end{veta}

\begin{veta}
	(i) If $X_n$ are PTS, $n \in ω$. Then $\prod_{n \in ω} X_n$ is PTS.

	(ii) $X$ PTS, $H \subset X$. Then $H$ is PTS $\Leftrightarrow$ $H \in ©G_δ(X)$

	\begin{dukazin}[(i)]
		Let $d_n$ be CCM (complete compatible metric) on $X_n$, $n \in ω$. Then
		$$ d(x, y) := \sum_{n=0}^∞ \min\{2^{-n}, d_n(x_n, y_n)\} $$
		is CCM on $X = \prod_{n \in ω} X_n$, where $x = (x_n)$, $y = (y_n)$. („Definition is correct“ is trivial, „$d$ is metric“ straightforward, „$d$ is complete“ also easy, compatibility too).
	\end{dukazin}

	\begin{dukazin}[(ii)]
		$H = \O$, $H = X$ trivial. Assume $H ≠ \O, X$.

		„$\implies$“: Fix CCM $ρ$ on $H$. $V_n := \bigcup\{V \subset X | V \text{ open in $X$} \land V \cap H ≠ \O \land \diam_ρ(V \cap H) < 2^{-n}\}$, $n \in ω$. We want to show $H \overset{?}= \bigcap_{n \in ω}(V_n \cap \overline{H}) \in ©G_δ$:
		$$ \text{„$\subseteq$“: } x \in H, n \in ω, x \in B_ρ(x, 2^{-n-2}) \subset V_n. $$
		„$\supseteq$“: $x \in V_n \cap \overline{H}$ for every $n \in ω$ $\implies$ $\exists$ open sets $G_n$: $x \in G_n$, $G \cap H ≠ \O$, $\diam(G_n \cap H) < 2^{-n}$. We can assume: $G_{n+1} \supset G_n$ (we can use intersection: $G_{n+1} \cap G_n \cap H \overset{?}≠ \O$ $\impliedby$ $x \in G_n \cap G_{n+1} \cap \overline{H} ≠ \O$).

		$\{y\} := \bigcap_{n \in ω} \overline{G_n \cap H}^H \in H$. For contradiction: $x ≠ y$ $\implies$ $\exists O \subset X$ open: $x \notin \overline{O}$, $y \in O$, $G_n \cap H \subset B(y, 2^{-n})$, $n \in ω$. $\implies \exists n \in ω G_n \cap H \subset O$, $x \in G_n \cap (X \setminus \overline{O}) \cap \overline{H}$ $\implies$ $G_n \cap (X \setminus \overline{O}) \cap H ≠ \O$.

		„$\impliedby$“: fix CCM $d$ on $X$, $H = \bigcap_{n \in ω} U_n$, $\O = U_n ≠ X$. $F_n := X \setminus U_n$, $\tilde d(x, y) = d(x, y) + \sum_{n=0}^∞ \min\{2^{-n}, \left|\frac{1}{\dist(x, F_n)} - \frac{1}{\dist(y, F_n)}\right|\}$, $x, y \in H$. Next we verified that $\tilde d$ is metric, that $\tilde d$ is equivalent with $d$ on $H$ (by convergence), and that $(H, \tilde d)$ is complete metric space and separable. TODO?
	\end{dukazin}
\end{veta}

\begin{definice}[Notation]
	$A ≠ 0$:

	\begin{itemize}
		\item $A^{< ω} :=$ finite sequence of elements of $A$ = $\bigcup_{n \in ω} A^n$;
		\item $s \in A^k$, $t \in A^{< ω} \cup A^ω$: $s^\wedge t := (s_0, s_1, …, s_{k-1}, t_0, t_1, …)$, where $s = (s_0, …, s_{k-1})$, $t = (t_0, t_1, …)$;
		\item $s \in A^{<ω} \cup A^ω$: $|s|$ is the number of elements of sequence $s$ ($|s| \in ω \cup \{∞\}$);
		\item $s \in A^{<ω} \cup A^ω$, $k \in ω$, $|s| ≥ k$, then we denote restriction of $s$ on first $k$ elements as $s / k$;
		\item $s \prec t$ iff $|t| ≥ |s|$ and $s = t / |s|$ ($s \in A^{<ω}$, $t \in A^{<ω} \cup A^ω$).
	\end{itemize}
\end{definice}

\section{Baire space $ω^ω$}
\begin{definice}
	For $s \in ω^{<ω}$ we define Baire interval of $s$ as $©N(s) := \{ν \in ω^ω | s \prec ν\}$.

	$©N(s)$ are clopen ($©N(s) = ω^ω \setminus \bigcup\{©N(t) \middle| |t| = |s|, t ≠ s, t \in ω^{<ω}\}$).

	$\{©N | s \in ω^{<ω}\}$ is base of topology of $ω^ω$.
\end{definice}

\begin{veta}[Alexandrov–Urysohn]
	$ω^ω$ is up to homeomorphism unique nonempty multi-dimension PTS such that every compact has empty interior.

	\begin{dukazin}
		Bez důkazu.
	\end{dukazin}
\end{veta}

\begin{dusledek}
	$ω^ω$ is homeomorphic to $®R \setminus ®Q$.
\end{dusledek}

% 11. 10. 2023

\begin{veta}
	Let $X ≠ \O$, PTS. Then $X$ is continuous image of $ω^ω$.

	\begin{poznamkain}
		$X ≠ \O$ PTS. Then there $\exists F \subset ω^ω$, $F$ closed, and continuous injection $φ: F \rightarrow X$.
	\end{poznamkain}

	\begin{dukazin}
		Find CCM on $X$ such that $\diam X ≤ 1$. We inductively construct closed $\O ≠ A_s \subset X$ for every $s \in ω^{< ω}$ such that 1. $A_{\O} = X$; 2. $\diam (A_s) ≤ 2^{-|s|}$; 3. $A_s = \bigcup_{i \in ω} A_{s^\wedge i}$.

		Empty set is trivial. Assume we already have $A_s$. Find $\{x_i | i \in ω\} \subset A_s$ dense in $A_s$. $A_{s^\wedge i} := A_s \cap \overline{B(x_i, 2^{-|s|-2})} ≠ \O$ closed.

		Fix $\forall ν \in ω^ω: f(ν) := x$, where $\{x\} = \bigcap_{k \in ω} A_{ν / k} ≠ \O$ (intersection of closed nonempty non-increasing sequence of sets). „$f$ is surjection“: $x \in A_s \overset{3.} \implies \exists n \in ω: x \in A_{s^\wedge n} \overset{1.} \implies \forall x \in X\ \exists α \in ω^ω\ \forall k \in ω: x \in A_{α / k} \implies x = f(α)$.

		„$f$ continuous“: $f(©N_{ν / k}) \subset A_{ν / k}$ for every $ν \in ω^ω$, $k \in ω$, $\diam A_{ν / k} ≤ 2^{-k}$.
	\end{dukazin}
\end{veta}

\subsection{Cantor set $2^ω$}
\begin{tvrzeni}
	$2^ω$ is up to homeomorphism unique nonempty nuldimensional compact metrizable space without isolated points (without isolated points is called perfect space).
\end{tvrzeni}

\begin{tvrzeni}
	Let $X ≠ \O$ metrizable, compact. Then $X$ is continuous image of $2^ω$.

	\begin{dukazin}
		Without proof, but it is similar to the previous one.
	\end{dukazin}
\end{tvrzeni}

\subsection{Hilbert cube $[0, 1]^ω$}
\begin{tvrzeni}
	Let $X$ be PTS. Then $X$ is homeomorphic to $G_δ$ subset of $[0, 1]^ω$.

	\begin{dukazin}
		$X$ PTS, case $\O$ is trivial, so assume $X ≠ \O$, $ρ$ is CCM on $X$, $ρ ≤ 1$. Let $\{x_n, n \in ω\}$ be dense in $X$. Define $f: [0, 1]^ω: f(x) = (ρ(x, x_n))_{n \in ω}$. $ρ ≤ 1 \implies f(x) \in [0, 1]^ω$.

		„Continuity of $f$“: $f^{-1}(U) = \bigcap_{i = 1}^nB(x_i, b_i) \setminus \overline{B(x_i, a_i)}$ open.

		„Injective“: $x ≠ y \implies \exists n \in ω: ρ(x, x_n) < ρ(y, x_n) \implies f(x) ≠ f(y)$.

		„Continuity of $f^{-1}$“ $f(y^n) \rightarrow f(y) \overset{?} \implies y^n \rightarrow y$.
		$$ f(y^n) \rightarrow f(y) \overset{?} \Leftrightarrow \forall k \in ω: ρ(y^n, x_k) \rightarrow ρ(y, x_k). $$
		Let $ε > 0$ be arbitrary:
		$$ \exists k \in ω: ρ(y, x_k) < \frac{ε}{3}.\ \exists n_0\ \forall n ≥ n_0: ρ(y^n, x_k) < \frac{2ε}{3}. $$
		Then
		$$ \forall n ≥ n_0: ρ(y^n, y) ≤ ρ(y^n, x_k) + ρ(x_k, y) < ε. $$

		So $f(X)$ is homeomorphism to $X$ $\implies$ $f(X)$ is PTS $\implies$ $f(X) \in ©G_δ([0, 1]^ω)$.
	\end{dukazin}
\end{tvrzeni}

\begin{dusledek}
	Let $X$ be compact metrizable space. Then $X$ is homeomorphic to some closed subset of $[0, 1]^ω$.

	\begin{dukazin}
		Compact metrizable space is Polish. And compact subset must be closed.
	\end{dukazin}
\end{dusledek}

\subsection{$©K(X)$: Hyperspace of compact subsets of $X$}
\begin{definice}
	Let $X$ be PTS, denote $©K(X) := \{K \subset X | K \text{ is compact}\}$. Vietoris topology on $©K(X)$ is generated by $\{K \in ©K(X) | K \subset V\}$ for $V$ open and $\{K \cap ©K(X) | K \cap V ≠ \O\} = ©K(X) \setminus \{K \in ®K(X) | K \subset X \setminus V\}$ for $V$ open.
\end{definice}

\begin{tvrzeni}
	Let $X$ be PTS, $ρ$ CCM on $X$, $ρ ≤ 1$. Then mapping $h: ©K(X) \times ©K(X) \mapsto [0, +∞)$ defined as:
	$$ h(K, L) = \begin{cases}0, & K = L = \O,\\ \max\{\sup_{x \in K} ρ(x, L), \sup_{y \in L} ρ(y, K)\}, & K, L ≠ \O,\\1, &\text{other cases},\end{cases} $$
	is CCM on ©K(X) with Vietoris topology. $h$ is known as Hausdorff metric.

	\begin{poznamkain}
		$©K(X)$ is separable if $X$ is PTS. $X$ is compact metrizable $\implies$ $©K(X)$ is compact (totally bounded).

		$X$ is separable $\implies$ $\exists D \subset X: \overline{D} = X, |D| = ω$.
		$$ M = \{K \subset D \middle| |K| < ω\} \implies |M| = ω. $$
		$\overline{M} = ©K(X)$. $K \in ©K(X)$ arbitrary, $ε > 0$ arbitrary. Then $\exists \frac{ε}{2}$ net $P \subset K$, $|P| < ω$. We find $\{\tilde x_0, …, \tilde x_n\} \subset D: ρ(x_i, \tilde x_i) < \frac{ε}{2} \land h(K, \{\tilde x_0, …, \tilde x_n\}) < ε$.

		$X$ is compact, $P$ is $ε$-net in $X$, $|P| < ω$ $\implies$ $2^P$ is finite $ε$-net in $©K(X)$.
	\end{poznamkain}

	\begin{dukazin}
		($\O ≠ K, L, P \in ©K(X)$.) $h$ is metric, definition is correct, $h ≥ 0$ trivial, $h(K, L) = h(L, K)$ trivial, $h(K, L) = 0 \implies K = L$ ($x \notin L \implies ρ(x, L) > 0 \implies K \subset L \land L \subset K$).

		„$\triangle$“ aka „$h(K, L) ≤ h(K, P) + h(P, L)$“: Let $x \in K, y \in L, p \in P$. Then
		$$ ρ(x, L) ≤ ρ(x, y) ≤ ρ(x, p) + ρ(p, y) \qquad \inf y \in L $$
		$$ ρ(x, L) ≤ ρ(x, p) + ρ(p, L) \qquad \sup p \in P $$
		$$ ρ(x, L) ≤ ρ(x, p) + h(P, L) \qquad \inf p \in P $$
		$$ ρ(x, L) ≤ ρ(x, P) + h(P, L) \qquad \inf p \in P $$
		$$ \sup_{x \in K} ρ(x, L) ≤ h(K, P) + h(P, L). $$
		Similarly $\sup_{y \in L} ρ(y, K) ≤ h(K, P) + h(P, L)$.
	\end{dukazin}
\end{tvrzeni}

% 18. 10. 2023

TODO!!!

% 25. 10. 2023

\begin{definice}
	$X$ is metrizable space, $1 ≤ α < ω_1$. We define $Σ_α^0(X)$, $∏_α^0(X)$, and $Δ_α^0(X)$ by induction:
	$$ Σ_1^0(X) := \{U \subset X | U \text{ open}\}, $$
	$$ ∏_α^0(X) := \{A \subset X | X \setminus A \in Σ_α^0(X)\}, $$
	$$ Σ_α^0(X) := \{\bigcup_{n \in ω} A_n | A_n \in ∏_{α_n}^0(X), α_n < α, n \in ω\}, $$
	$$ Δ_α^0(X) := Σ_α^0 \cap ∏_α^0(X). $$

	\begin{poznamkain}[By induction it can be prooven]
		$$ Σ_α^0(X) \subset Σ_β^0(X), ∏_α^0(X) \subseteq ∏_β^0(X), \qquad 1 ≤ α < β < ω_1. $$
	\end{poznamkain}

	\begin{poznamkain}
		$$ \forall α, β: 1 ≤ α < β < ω_1: Σ_α^0(X) \subset ∏_β^0(X). $$
	\end{poznamkain}

	\begin{poznamkain}
		If $X$ contains homeomorphic copy of $2^ω$ then all inclusions are strict.
	\end{poznamkain}

	We denote $Borel(X)$ as $ς$-algebra of Borel sets ($ς$-algebra generated by $Σ_1^0(X)$).
\end{definice}

\begin{poznamka}[Also non-trivial theorem]
	$$ Borel(X) = \bigcup_{1 ≤ α < ω_1} Σ_α^0(X) = \bigcup_{1 ≤ α < ω_1}(X) = \bigcup_{1 ≤ α < ω_1} Δ_α^0(X). $$

	$$ A_n \in \bigcup_{1 ≤ α < ω_1} Σ_α^0(X) \implies \exists 1 ≤ α_n < ω_1: A_n \in Σ_{α_n}^0 (X) \implies A_n \in Σ_{\sup \{α_n | n \in ω\}}^0 \implies \bigcup_{n \in ω} A_n \in Σ_{\sup\{α_n, n \in ω\}}^0 \implies \bigcup A_n \in \bigcup_{1 ≤ α < ω_1} Σ_α^0(X). $$
\end{poznamka}

\begin{poznamka}
	$F_ς = Σ_2^0$, $G_δ = ∏_2^0$, $F_{ςδ} = ∏_3^0$, $G_{δς} = Σ_3^0$.

	$Σ_α^0(X)$ is closed under countable union and $∏_α^0(X)$ under countable intersection.
\end{poznamka}

\begin{veta}
	$X$ be metrizable, $1 ≤ α < ω_1$. Then
	\begin{enumerate}
		\item $Σ_α^0(X)$ is closed under finite intersection;
		\item $∏_α^0(X)$ is closed under finite union.
	\end{enumerate}

	\begin{dukazin}
		„1.“ Firstly for $α = 1$, it is trivial. Then let $A, B \in Σ_α^0(X)$, $α > 1$. Then $A = \bigcup_{n \in ω} A_n$, $A_n \in ∏_{α_n}^0(X)$, $α_n < α$, $B = \bigcup_{m \in ω} B_m$, $B_m \in ∏_{β_m}^0(X)$, $β_n < α$. $A \cap B = \bigcup_{(m, n) \in ω^2} A_n \cap B_m$, $A_n \cap B_m \in ∏_{\max\{α_n, β_n\}}^0(X) \implies A \cap B \in Σ_α^0(X)$. „2.“ $\impliedby$ de Morgan and 1.
	\end{dukazin}
\end{veta}

\begin{veta}
	$X$ be metrizable, $A \subset Z \subset X$, $1 ≤ α < ω_1$. Then $A \in Σ_α^0(Z)$ $\Leftrightarrow$ there exists $\tilde A \in Σ_α^0(X): A = \tilde A \cap Z$. Similarly for $∏_α^0, Δ_α^0$.

	\begin{dukazin}
		Firstly $α = 1$ from definition of subspace. Then assume that it is all true for all $β < α$. We want to prove it for $α$. „$\implies$“:
		$$ A \in Σ_α^0(Z) \implies A = \bigcup A_n, A_n \in ∏_{β_n}^0(Z), β_n < α \implies \exists \tilde A_n \in ∏_{β_n}^0(X): \tilde A_n \cap Z = A_n. $$
		$$ \tilde A = \bigcup \tilde A_n \in Σ_α^0(X), \tilde A \cap Z = Z \cap \bigcup \tilde A_n = \bigcup(Z \cap \tilde A_n) = \bigcup A_n = A. $$
		„$\impliedby$“:
		$$ \tilde A \in Σ_α^0(X), A = \tilde A \cap Z \implies \exists \tilde A_n \in ∏_{β_n}^0(X), β_n < α, \bigcup \tilde A_n = \tilde A. $$
		$$ \tilde A \cap Z \in ∏_{β_n}^0(Z) \implies A = \tilde A \cap Z = \(\bigcup \tilde A_n\) \cap Z = \bigcup\(\tilde A_n \cap Z\) = \bigcup A_n \in Σ_α^0(Z). $$
	\end{dukazin}
\end{veta}

\begin{veta}
	$X$, $Y$ be metric spaces, $f: X \rightarrow Y$ is continuous. If $A \in Σ_α^0(Y)$ ($∏_α^0(Y)$, $Δ_α^0(Y)$) then $f^{-1}(A) \in Σ_α^0(X)$ ($∏_α^0(X)$, $Δ_α^0(Y)$).

	\begin{dukazin}
		$α = 1$ trivial. Assume it holds true for $Σ_β^0(Y)$, $∏_β^0(Y)$, $β < α$, and we want to show for $Σ_α^0(Y)$ ($∏_α^0(Y)$). Let $A \in Σ_α^0(Y)$, $α > 1$ $\implies$ $A = \bigcup_{n \in ω} A_n$, $A_n \in ∏_{β_n}^0(Y)$, $β_n < α$.
		$$ f^{-1}(A) = f^{-1}(\bigcup A_n) = \bigcup \underbrace{f^{-1}(A_n)}_{∏_{β^n}^0(X)} \in Σ_α^0(X), $$
		$$ f^{-1}(Y \setminus A) = f^{-1}(Y) \setminus f^{-1}(A) = X \setminus f^{-1}(A). $$
	\end{dukazin}
\end{veta}

\begin{veta}[Borel classes in PTS]
	$X, Y$ be PTS, $A \in Σ_α^0(X)$, $α ≥ 3$ (resp. $A \in ∏_α^0(X)$, $α ≥ 2$), $B \subset Y$. If $B$ and $A$ are homeomorphic then $B \in Σ_α^0(Y)$ (resp. $∏_α^0$).

	\begin{dukazin}
		$f: A \rightarrow B$ is homeomorphism $A$ onto $B$. The theorem above (name ?) there is extension $\tilde f$ of $f$, $\tilde f$ is homeomorphism $\tilde A$ onto $\tilde B$, $A \subset \tilde A$, $B \subset \tilde B$, $\tilde A \in ∏_2^0(X)$, $\tilde B \in ∏_2^0(Y)$. Then $B \in Σ_α^0(\tilde B)$ (because $B = (f^{-1})^{-1}(A)$). From the theorem above, $\exists \hat{B} \in Σ_α^0(Y): B = \hat{B} \cap \tilde B \in Σ_α^0(Y) \impliedby α ≥ 3$.
	\end{dukazin}
\end{veta}

\subsection{Analytic sets}
\begin{definice}
	$X$ PTS, $A \subset X$. We say that $A$ is analytic set in $X$ if there exists PTS $Y$ and continuous mapping $φ: Y \rightarrow X$ such that $φ(Y) = A$.

	We denote collection of analytic subsets of $X$ as $Σ_1^1(X)$. We say that $A$ is coanalytic in $X$ if $X \setminus A \in Σ_1^1(X)$ and we denote this collection as $∏_1^1(X)$. $Δ_1^1(X) = Σ_1^1(X) \cap ∏_1^1(X)$.
\end{definice}

\begin{priklady}
	$$ Q = \{α \in 2^ω | \exists n \in ω\ \forall j ≥ n: α_j = 0\} = 2^{<ω} \in Σ_2^0(2^ω) \setminus ∏_2^0(2^ω) $$

	TODO?
\end{priklady}

\begin{poznamka}
	$X$ PTS, $F: X \rightarrow ©K(X)$ by $F(x) = \{x\}$. Then $F$ is continuous, $F^{-1}(©K(A)) = A$ $\implies$ if $©K(A) \in Σ_α^0 (©K(X))$ ($∏_α^0$, $Δ_α^0$) then $A \in Σ_α^0(X)$ ($∏_α^0$, $Δ_α^0$). $A$ open $\implies$ $©K(A)$ is open, $A$ is closed $\implies$ $©K(A)$ is closed. $©K(\bigcap A_n) = \bigcap ©K(A_n)$. Thus for $A \in ∏_2^0(X) : ©K(A) \in ∏_2^0(©K(X))$. $A \in Σ_1^0(X)$ ($∏_1^0(X)$, $∏_2^0(X)$) $\Leftrightarrow$ $©K(A) \in Σ_1^0(©K(X))$ ($∏_1^0(©K(X))$, $∏_2^0(©K(X))$).
\end{poznamka}

\begin{veta}
	$X$ PTS, $|X| > ω$. Assume $I \subset ©K(X)$, $I$ is $ς$-ideal ($K \in I, L \subset K \implies L \in I$; $K_n \in I, \bigcup K_n \in ©K(X) \implies \bigcup K_n \in I$). If $I \in ∏_2(©K(X))$, then $I \in Σ_1^1(©K(X))$.
\end{veta}

\begin{dusledek}
	$A \notin ∏_2^0(X) \implies ©K(A) \notin Σ_1^1(©K(X))$.
\end{dusledek}

\begin{poznamka}
	$A \in ∏_1^1(X)$, $©K(A) = ©K(X) \setminus \{K \in ©K(X) | \exists x \in (X \setminus A) \cap K\}$ % =: ©K(X) \setminus ©M$
	$\{(K, x) \in ©K(X) \times X | x \in K\}$ is closed.
\end{poznamka}

\end{document}
