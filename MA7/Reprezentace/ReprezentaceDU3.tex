\documentclass[12pt]{article}					% Začátek dokumentu
\usepackage{../../MFFStyle}					    % Import stylu



\begin{document}

% z 18. 10. 2023 na 01. 11. 2023

\begin{priklad}
	Nechť ¦V je vektorový prostor nad komutativním tělesem ®T, ¦V má spočetnou (nekonečnou) bázi $B = \{B_i | i \in ®N\}$. Popište ideály (oboustranné) okruhu $R = \End_{®T}(¦V)$.
	
	\begin{reseni}
		%Samozřejmě nevlastní ideál (celý $R$) je ideál.
		%
		Předpokládejme, že $I$ je ideál, který obsahuje endomorfismus $f$ zobrazující na podprostor s bází $C = \(C_i\)_{i=1}^N$, kde $N \in ®N_0 \cup \{∞\}$. Je-li $N = ∞$, pak vezměme $D = \(D_i\)_{i=1}^∞ = \(f^{-1}(C_i)\)_{i=1}^∞$ posloupnost nějakých vzorů $C$. Nyní vezmeme endomorfismy $g, h$, že $\forall i \in ®N: g(B_i) = D_i \land h(C_i) = B_i$. Složení všech našich endomorfismů: $h∘f∘g$, musí být též v $I$ a zároveň se na bázi $B$ chová jako identita, tj. je identita na celém ¦V. A cokoliv složeno s identitou je ono samo, tedy $I = R$, ať bylo $f$ jakékoliv s nekonečněrozměrným obrazem.

		Nyní tedy $N \in ®N$ (pro všechny prvky $I$). Potom můžeme pro každý vektor $¦v \in ¦V$ a pro každý endomorfismus $f_2: ¦V \rightarrow \LO ¦v$ najít endomorfismus $g_{¦v}$ zobrazující ¦v na $D_1$ a zbytek (kromě lineárního obalu ¦v) na ¦o a endomorfismus $h_{¦v}$ zobrazující $C_i$ na $¦v$. Potom $f_2 ∘ g_{¦v} ∘ f ∘ h_{¦v} = f_2$. Tedy $f_2 \in I$. Tudíž $I$ obsahuje všechny endomorfismy zobrazující na libovolný 1D podprostor. A (konečným) součtem (na něž musí být $I$ také uzavřený) takových endomorfismů dostaneme libovolný endomorfismus, který má konečněrozměrný obraz.

		Zbývá ještě ukázat, že zobrazení s konečněrozměrnými obrazy tvoří ideál (tj. že jsou uzavřené na $∘$ s libovolným endomorfismem na ¦V a že tvoří vzhledem ke sčítání podgrupu ¦V). „Složení“: z jedné strany je to jasné, protože endomorfismus zobrazí V na V nebo jeho podmnožinu, tedy složené zobrazení bude mít $\im$, který je podmnožinou $\im$ zobrazení z našeho ideálu, tedy konečně dimenzionální; u složení z druhé strany pak můžeme vidět, že endomorfismus „nezvětší dimenzi“, neboť bázi zobrazí na generátor, tedy dimenze $\im$ složeného zobrazení bude menší rovna dimenzi našeho zobrazení z ideálu, tedy konečná). „Sčítání“: $\im(f+g)$ je podmnožina $\im(f)+\im(g)$, neboť pro každý vzor jen sečteme obrazy, z nichž jeden je v $\im(f)$ a druhý v $\im(g)$, což má dimenzi rovnou součtu dimenzí, tedy, pro f a g z našeho ideálu, dimenzi konečnou. \footnote{Ještě by se mělo ukázat, že obsahuje „jednotku“ vzhledem ke sčítání, ale to je zřejmě konstantní ¦o, která má dimenzi obrazu $0 < ∞$.}

		Nakonec $N = 0$ (pro všechny prvky $I$) je ten nezajímavý případ na závěr, kdy $I = \{¦o\}$, jelikož endomorfismus vracející počátek tvoří sám o sobě grupu vůči sčítání a zároveň je invariantní vůči složení (pravému i levému) s jiným endomorfismem.

		Tedy ideály jsou 3: $\End_{®T}(¦V)$, $\{\text{konečné obrazy}\}$, $\{¦o\}$.
		%
		%Mějme ideál $I$ obsahující endomorfismus $f$. Potom $I$ musí obsahovat i $R ∘ f ∘ R$, tedy pokud je $f$ prosté, pak $I = R$, jelikož existuje $f_p^{-1}: f(¦V) \rightarrow ¦V$, které je zřejmě lineární a na a můžeme ho rozšířit na celé ¦V (třeba nulou), tedy $f^{-1} \in R$, a $\id ∘ f ∘ f_l^{-1}$. Pokud je naopak na, 
	\end{reseni}
\end{priklad}

\end{document}

