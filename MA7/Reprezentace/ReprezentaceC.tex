\documentclass[12pt]{article}					% Začátek dokumentu
\usepackage{../../MFFStyle}					    % Import stylu



\begin{document}

% 04. 10. 2022
\section{Monoidové okruhy}
\begin{definice}[Monoid]
	Množina $M$ s binární asociativní operací $·$ a neutrálním prvkem $1$ se nazývá monoid. Značíme $M(·, 1) := (M, ·, 1)$.
\end{definice}

\begin{priklady}
	$R$ okruh $(R, +, -, 0, ·, 1)$ $\implies$ $(R, ·, 1)$ monoid. $(®N_0, +, 0)$ komutativní monoid.
\end{priklady}

\begin{poznamka}[Řád]
	Podobně jako pro grupy definujeme pro monoid $M$ a $a \in M$ řád prvku $a$ jako $\ord_M(a) = |\<a\>|$, kde $\<a\>$ je nejmenší podmonoid obsahující $a$.

	Pokud existuje $n \in ®N$, že $a^n = 1$, pak nejmenší takové $n$ je rovno $\ord_M(a)$. (Pozor, opačně to neplatí, viz $®Z \mod 2$, kde $\ord(0) = 2$ nebo $®Z \mod 8$, kde $\ord(4) = 3$).
\end{poznamka}

\begin{definice}[RM]
	Nechť $R$ okruh, $M$ monoid. Definujme $RM = R[M] := R^{(M)} := \{f: M \rightarrow R | f(m) = 0 \text{ pro skoro všechna, tj. až na konečně mnoho, }, m \in M\}$.

	Operace na $R[M]$:

	\begin{itemize}
		\item $0_{R[M]} =$ nulové zobrazení;
		\item $1_{R[M]} = f$ takové, že $f(1) = 1$ a $f(m) = 0$ pro všechna $m \in M \setminus \{1\}$;
		\item $(f ± g)(m) = f(m) ± g(m)$ ($\forall m \in M$);
		\item $(f·g)(m) = \sum_{k, l \in M, k·l=m} f(k)·g(l)$.
	\end{itemize}

	Pak $R[M]$ je okruh.

	\begin{poznamkain}
		Prvek $f \in R[M]$ se často zapisují jako formální suma $\sum_{m \in M} f_m·m$, kde $f_m := f(m)$.
	\end{poznamkain}
\end{definice}

\begin{tvrzeni}
	Pokud existuje $a \in M$ a $n \in ®N$ takové, že $a^n = 1$, ale $a ≠ 1$, pak $R[M]$ není obor, tj. existují v $R[M]$ netriviální dělitelé 0.

	\begin{dukazin}
		$$ (a - 1)·(a^{n-1} + a^{n - 2} + … + a + 1) = a^n - 1^n = a^n - 1 = 0. $$
		Ale jen tak to není definované, jelikož $a$ je z $M$ ale sčítání ne. Tedy počítáme nad $R[M]$, kde $a = f$ takové, že $f(a) = a$ a $f(b) = 1$ pro všechna $b \in M \setminus \{a\}$.
	\end{dukazin}
\end{tvrzeni}

\begin{definice}[Kanonické vnoření do RM]
	Kanonická vnoření $R$ a $M$ do $R[M]$ definujeme jako:
	$$ α: R \rightarrow R[M], r \mapsto f_r, \qquad f_r(1) = r, f_r(k) = 0\ \forall k \in M \setminus \{1\}, $$
	$$ β: M \rightarrow R[M], m \mapsto f_m, \qquad f_m(m) = 1, f_r(k) = 0\ \forall k \in M \setminus \{m\}, $$
	kde $f_r$ značíme často jen $r$ a $f_m$ často jen $m$. $α$ je okruhový monomorfismus (tj. injektivní okruhový honomorfismus) a $β$ je injektivní homomorfismus monoidů.
\end{definice}

\begin{poznamka}[Pozorování]
	$(\forall r \in R)(\forall m \in M) α(r)·β(m) = β(m)·α(r)$.
\end{poznamka}

\end{document}
