\documentclass[12pt]{article}					% Začátek dokumentu
\usepackage{../../MFFStyle}					    % Import stylu

\newcommand{\op}{^{\text{op}}}

\begin{document}

% 05. 10. 2022

\begin{definice}[Category, map (arrow, morphism), composition, domain, codomain]
	A category $©A$ consists of: a collection $\ob(©A)$ of objects, and for each $A, B \in ©A$, a collection $©A(A, B)$ of maps, arrows, or morphisms from $A$ to $B$. Such that for each $A, B, C \in \ob(©A)$ a function (named composition) $∘: ©A(B, C) \times ©A(A, B) \rightarrow ©A(A, C)$, $(g, f) \mapsto g∘f$ meets following:

	For each $f \in ©A(A, B), g \in ©A(B, C), h \in ©S(C, D): (h ∘ g) ∘ f = h ∘ (g ∘ f)$ (asociativity). For each $A \in \ob(©A)$ $\exists 1_A \in ©A(A, A)$, called the identity, such that, for each $f \in ©A(A, B): f ∘ 1_A = f = 1_B ∘ f$.

	\begin{poznamka}[Notation]
		$$ A \in \ob(©A) \Leftrightarrow A \in ©A. $$
		$$ f \in ©A(A, B) \Leftrightarrow A \overset{f}\rightarrow B \Leftrightarrow f: A \rightarrow B. $$
	\end{poznamka}

	For $f \in ©A(A, B)$: $\dom(f) := A$ and $\codom(f) := B$.
\end{definice}

\begin{priklady}[of categories]
	Category of:
	\begin{itemize}
		\item sets (SET): $\ob(SET) :=$ sets, $SET(A, B) :=$ functions from $A$ to $B$, $∘$ is composition;
		\item groups (GRP): $\ob(GRP) :=$ groups, $GRP(G, H) :=$ group homomorphisms, $∘$ is composition;
		\item rings (RING): $\ob(RING) :=$ rings, $RING(A, B) :=$ ring homomorphisms, $∘$ is composition;
		\item vector spaces (VECT$_{®K}$): $\ob(VECT_{®K}) :=$ vector spaces over ®K, $RING(A, B) :=$ ®K linear maps, $∘$ is composition;
		\item topological spaces (TOP): $\ob(TOP) :=$ topological spaces, $RING(A, B) :=$ continuous maps, $∘$ is composition.
	\end{itemize}
\end{priklady}

\begin{definice}[Isomorphism, inverse]
	$f: A \rightarrow B$ in a category ©A is an isomorphism if exists a map $g: B \rightarrow A$ in ©A such that $g ∘ f = 1_A$ and $f ∘ g = 1_B$. Then we call $g$ the inverse of $f$.
\end{definice}

\begin{priklady}
	In SET isomorphisms are bijections.
\end{priklady}

\begin{priklad}
	Show that inverses are unique (justifying the use of the determine article in the previous definition).
\end{priklad}

\begin{poznamka}
	0-morphisms are called morphisms (between objects), 1-morphisms are called functors (between categories), 2-morphisms are called natural transformations (between functors).
\end{poznamka}

\begin{definice}[Functor]
	Let ©A and ©B be categories. A functor $F: ©A \rightarrow ©B$ consists of: a function $F: \ob(©A) \rightarrow \ob(©B)$, and for each $A, A' \in ©A$ a function $F: ©A(A, A') \rightarrow ©B(F(A), F(A'))$. Such that
	$$ F(f' ∘ f) = F(f) ∘ F(f'), \qquad \forall A \overset{f} A' \overset{f'} A'' \in ©A, $$
	$$ F(1_A) = 1_{F(A)} \qquad \forall A \in ©A. $$
\end{definice}

\begin{priklady}[Forgetful functors]
	$U: GRP \rightarrow SET$, for any group $(G, ·)$, $U((G, ·)) := G$, and for any morphism $f$, $U(f: (G, ·) \rightarrow (H, *)) := f: G \rightarrow H$. (Exercise: Convince yourself that this is a well-defined functors.)
	
	We can do the same for rings, vector spaces and topological spaces.
\end{priklady}

% 05. 10. 2023

\begin{priklady}
	Let ©A be the following category: $\ob(©A) = \{·\}$, $©A(·, ·) = 1_·$, and $1_· ∘ 1_· = 1_·$. It is called discrete category with one object.

	$\ob(©B) = \{·, *\}$, $©B(·, ·) = 1_·$, $©B(·, *) = \O$

	Directed transitive graph (with all loops) with concatenation of edges.

	From group $(G, +)$ we construct category ©G by putting: $\ob(©G) := {·}$, $©G(·, ·) := G$ and $∘ := +$. We can generalize to a monoid $(M, +)$.

	Now, let ©A be a category with one object $\{·\}$ (and assume that $©S(·, ·)$ is a set). Then homomorphism with composition are monoid. And isomorphisms with composition are groups (so one-object category with all homomorphism isomorphic represents group).

	(Category, where $©A(·, ·)$ is a set, is often called locally small.)

	Let $G$ and $H$ be groups and ©G, ©H their associated one-object categories. What is a functor from ©G to ©H? For $F: \ob(©G) \rightarrow \ob(©H)$ we have no other choice than $F(·) := *$. For $F: ©G(·, ·) \rightarrow ©H(*, *) = ©H(F(·), F(·))$ we demonstrated (see lecture) that $F$ needs to be group homomorphism (and every group homomorphism $G \rightarrow H$ is functor). (All this work for monoids too.)

	Let $AB$ be the category of $\ob(AB) :=$ Abelian groups and $AB(A, B) :=$ group homomorphism. Then $U: AB \rightarrow GRP$ as „forgetful functor“ is „identity“. The same for commutative rings. Also we have forgetful functor $U: RING \rightarrow AB$, $(R, +, ·) \mapsto (R, +)$ and functor $U: RING \rightarrow MONOIDS$, $(R, +, ·) \mapsto (R, ·)$.

	$U: SET \rightarrow VECT_{®K}$ we can define by $F(X) = (X \rightarrow F)$ (functions from $X$ to $F$) (free vector space).
\end{priklady}

% 12. 10. 2023

\begin{definice}[Functor composition]
	When we have functor $F: ©A \rightarrow ©B$ and $F': ©B \rightarrow ©C$. We want to $F' ∘ F$ to be functor, so it has function on objects and functions on morphism classes. Function on object is simply composition $F' ∘ F$. Functions on morphism classes is also composition:
	$$ ©A(A, A') \overset{F}\rightarrow B(F(A), F(A')) \overset{F'}\rightarrow ©C(F'∘F(A), F'∘F(A')) \implies F'∘F: ©A(A, A') \rightarrow ©C(F'∘F(A), F'∘F(A')). $$

	\begin{dukazin}
		1. $(F'∘F)(1_A) = F'(F(1_A)) = F'(1_{F(A)}) = 1_{F'∘F(A)}$. (For $A \in ©A$.)

		2. $(F'∘F)(f'∘f) = F'(F(f'∘f)) = F'((F(f')) ∘ (F(f))) = (F'∘F(f')) ∘ (F'∘F(f))$. (For $A \overset{f}\rightarrow A'\overset{f'}\rightarrow A'' \in ©A$.)
	\end{dukazin}

	So $F' ∘ F$ is a functor. We call it the composition of $F$ and $F'$.
\end{definice}

\begin{definice}[CAT]
	The category of categories (CAT) has categories as objects and functors as morphisms (with its composition from the previous definition).

	\begin{dukazin}
		We need: 1. An identity functor $1_{©A} \in CAT(©A, ©A)$ (function on objects is identity, function on $CAT(©A, ©B)$ is identity too), we can easily see that it fulfills condition from category definition.

		2. Associativity of composition: composition of functions is associative, so we see this from the definition of the functor composition.
	\end{dukazin}
\end{definice}

\begin{definice}[Dual category (opposite category)]
	For a category ©A, its dual category (or opposite category) $©A\op$ is defined by: $\ob(©A\op) = \ob(©A)$, $©A\op(B, A) = ©A(A, B)$ ($\forall A, B \in \ob(©A)$), composition in $©A\op$ is the composition in ©A.
\end{definice}

\begin{priklad}[Excercise]
	$$ (©A\op)\op = ©A. $$
\end{priklad}

\begin{definice}[Contravariant functor]
	For two cats $©A, ©B$ a contravariant functor: $©A \rightarrow ©B$ is a functor $F: ©A\op \rightarrow ©B$ ($F(f' ∘ f) = (F(f)) ∘ (F(f'))$).
\end{definice}

\begin{priklad}
	Functor $C: TOP \rightarrow ALG_{®K}$ is $X \in TOP \mapsto C(X) \in ALG_{®K}$, where $C(X)$ is the collection of all continuous functions $X \rightarrow ®K$ with addition, multiplication and scalar multiplication. But when we try to define $C$ for morphisms, we find that it cannot be done this way. ($C(X \overset{f}\rightarrow Y) = C(X) \overset{C(f)}\rightarrow C(Y)$, so $C(f)(φ) = φ ∘ f$ $\implies$ this does not define a functor.)

	So we „fix it“ by taking contravariant functor.
\end{priklad}

\begin{definice}[Presheaf]
	Let ©A be a category a presheaf on ©A is a functor $©A\op \rightarrow SET$.
\end{definice}

\begin{priklad}
	Let $X$ be a topological space. Write $O(X)$ for ordered subsets of $X$ ordered by inclusion $\rightarrow$ category $©O(X)$: objects are open subsets, morphisms are inclusion and $∘$ is composition of inclusions.
\end{priklad}

\begin{definice}[Faithful functor, full functor]
	A functor $F: ©A \rightarrow ©B$ is faithful (resp. full) if for each $A, A' \in ©A$ the function
	$$ ©A(A, A') \rightarrow ©B(F(A), F(A')), \qquad f \mapsto F(f), $$
	is injective (resp. surjective) $\forall A, A' \in ©A$.
\end{definice}

\begin{upozorneni}
	If $F$ is faithful, we do not have $F(f_1) ≠ F(f_2)$ $\forall$ distinct morphisms $f_1, f_2$. ($F(A)$ still can be equal to $F(A')$, so it can be $f_1: A \rightarrow A$, $f_2: A' \rightarrow A'$.)
\end{upozorneni}

\begin{definice}[Subcategory]
	Let ©A be a category. A subcategory $©S \subset ©A$ consists of a subclass $\ob(©S) \subseteq \ob(©A)$ together with, for $S, S' \in \ob(©S)$, a subclass $©S(S, S') \subseteq ©A(S, S')$ such that ©S is closed under composition.
\end{definice}

\begin{definice}[Full subcategory]
	We say that subcategory ©S is full if $©S(S, S') = ©A(S, S')$, $\forall S, S' \in \ob(©S)$.

	\begin{poznamkain}
		A full subcategory is identified by its objects.
	\end{poznamkain}
\end{definice}

\begin{priklady}
	$AB$ is the full subcategory of $GRP$.
\end{priklady}

\begin{priklad}
	For any subcategory $©S \subset ©A$, we have an inclusion functor $I: ©S \rightarrow ©A$.

	$I$ is faithful, and it is full $\Leftrightarrow$ $©S$ is full.
\end{priklad}

\begin{definice}
	$F: ©A \rightarrow ©B$, $\im(F)$ has objects $F(A)$ and morphisms $F(f)$.
\end{definice}

\begin{upozorneni}
	$\im(F)$ nemusí být kategorie. (Mohou vzniknout „možnosti složení“, které v původní kategorii nebyly.)
\end{upozorneni}

\subsection{2-morphism and natural transformations}
\begin{definice}[Natural transformation]
	Let ©A and ©B be categories and $©A \overset{F}{\underset{G}\rightrightarrows} ©B$ two functors. A natural transformation between $F$ and $G$ ia s family of morphisms in ©B: $(F(A) \overset{α_A}\rightarrow G(A))_{A \in ©A}$ such that $F(f)∘α_B = α_A G(f)$ for every $A \overset{f}\rightarrow B \in ©A$.

	We call the morphisms $α_A$ the components of the natural transformation.
\end{definice}

\begin{priklad}
	Define a composition of natural transformations and use it to define the functor category of $©A$ and $©B$ (objects functors $F: ©A \rightarrow ©B$ and morphisms natural transformations $α$).
\end{priklad}

\begin{priklad}
	For two graphs $H, K$, functors between their 1-object cats $\leftrightarrow$ group homomorphism. What is a natural transformations between two functors?
\end{priklad}

\subsection{Free functors}
\begin{poznamka}
	Recall forgetfull functors. What about functors in the other direction?
\end{poznamka}

\begin{priklady}
	$F: SET \rightarrow VECT_{®K}$, $X \mapsto F(X)$. $F(X)$ (the free ®K-vector space) is is functions $f: X \rightarrow ®K$ endowed with the vector space structure (addition and scalar multiplication). (Alternatively $F(X)$ is the vector space with a basis $\{e_x^X | x \in X\}$).

	Morphisms: $F(f)(e_x^X) := e_{f(x)}^X$.
\end{priklady}

\begin{priklady}
	$U: GRP \rightarrow SET$, so free functor should look like $F: SET \rightarrow GRP$. $S \mapsto F(S)$, where $F(S)$ (the free group) is a sets for which $\exists i: S \rightarrow F(S)$ inclusion of sets to $F(S)$, that for every $f: S \rightarrow ©G$ function between sets and groups, $\exists! φ_i$ such that $i ∘ φ_i$ commutes.

	Think about / look up: this defines $F(S)$ uniquely up to group isomorphism.
\end{priklady}

\begin{priklad}
	Take the set $©S^{-1} = \{S^{-1} | S \in ©S\}$. Take all words in the alphabet $©S\cup ©S^{-1}$ that are reduced, i.e. we remove pairs of the form $SS^{-1}$, $S^{-1}S$ and ? is concatenation of words with reduction.
\end{priklad}

\begin{priklad}
	How does act on morphisms.
\end{priklad}

% 19. 10. 2023

TODO!!!

% 26. 10. 2023

\section{Adjunction}
\begin{definice}
	Let $©A \overset{F}{\underset{G}\rightleftarrows} ©B$ be categories and functors. We say that $F$ is left adjoint to $G$, and $G$ is right adjoint to $F$, and write $F -| G$ if $B(F(A), B) \cong ©A(A, G(B))$ „naturally“ in $A \in ©A$, and $B \in ©B$.

	\begin{poznamkain}
		Naturally: $\overline{\ }: ©B(F(A), B) \rightarrow ©A(A, G(B))$ and $\overline{\ }: ©A(A, G(B)) \rightarrow ©B(F(A), B)$.

		1. $\overline{F(A) \overset{g}\rightarrow B \overset{q}\rightarrow B'} = A \overset{\overline{g}}\rightarrow G(B) \overset{F(q)}\rightarrow G(B') \in ©A$.
		2. $\overline{A' \overset{p}\rightarrow A \overset{f}\rightarrow G(B)} = F(A') \overset{F(p)}\rightarrow F(A) \overset{\overline{f}}\rightarrow B \in ©B$.

		An adjunction between $F$ and $G$ is a choice of such isomorphism in $B(F(A), B) \cong ©A(A, G(B))$.
	\end{poznamkain}
\end{definice}

\begin{priklad}[Think about this]
	Adjoints may not exist. But if an adjunction does exist, then it is unique up to unique isomorphism.
\end{priklad}

\begin{definice}[Initial, terminal and zero object]
	Let ©A be a category. An object $I \in ©A$ is initial if for every $A \in ©A$, $\exists !$ map $I \rightarrow A$. An object $T \in ©A$ is terminal if for every $A \in ©A$ $\exists!$ map $A \rightarrow T$. If object is both initial and terminal, we say that it is a zero object.
\end{definice}

\begin{priklady}
	In SET, we have an initial object. It is empty set.

	In GRP we have an initial object $\{e\}$. And it is also a terminal object.

	What object is a terminal object in SET? $T =$ the set with one element.

	The terminal object in CAT is 1, the discrete category with one object.
\end{priklady}

\begin{lemma}
	Let $I$ and $I'$ be two initial objects in a category ©A. Then there is a unique isomorphism $I \rightarrow I'$, i.e. $I \cong I'$.

	\begin{dukazin}
		Since $I$ and $I'$ are both initial objects, $\exists!$ morphisms $\id_I: I \rightarrow I$, $f: I \rightarrow I'$, $g: I' \rightarrow I$ and $\id_{I'}: I' \rightarrow I'$. Because $g ∘ f = \id_I$ and $f ∘ g = \id_{I'}$, $f$ and $g$ give an isomorphism between $I$ and $I'$. Moreover we see that it is unique.
	\end{dukazin}
\end{lemma}

\begin{priklady}
	$VECT_{®K}$: initial object and terminal object is zero vector space (this is part of the „abelian category structure“ of $VECT_{®K}$).

	Let $R$ be a ring. Then we denote by $MOD_R$ the category of $R$-modules with $R$-linear maps. This has zero object 0 – the zero module.
\end{priklady}

\begin{priklad}
	Initial and terminal objects can be described via adjunctions: Let ©A be a category, then $\exists!$ functor $©A \rightarrow 1$ (the discrete category with one element). What bout a functor $1 \rightarrow ©A$? We see that such functor $F$ $\leftrightarrow$ objects $A \in ©A$.

	TODO?
\end{priklad}

TODO!!!

% 01. 11. 2023

TODO!!!

% 16. 11. 2023

TODO!!!

\begin{veta}
	Take cats and functors $©A \overset{F}\rightarrow ©B \overset{G}\rightarrow ©A$. There is a bijective correspondence between
	\begin{enumerate}
		\item (hom-class) adjunctions $F -| G$;
		\item pairs ($1 \overset{?}\rightarrow GF$, $FG \oversetε\rightarrow 1_B$) of natural transformations, satisfying the triangle identities;
		\item "initial objects in certain comma categories".
	\end{enumerate}
\end{veta}

TODO!!!

\begin{lemma}
	Take an adjunction $©A \overset{F}\rightarrow ©B \overset{G}\rightarrow ©B$, $F -| G$, and $A \in ©A$. Then $(F(A), η_A: A \rightarrow GF(A))$ is an initial object in the category ($A \Rightarrow G$)

	TODO!!!
\end{lemma}

TODO!!!

% 23. 11. 2023

\section{Representable functors}
\begin{poznamka}
	From now on all categories are assumed to be locally small (i.e. $©A(A, B)$ is a set).
\end{poznamka}

\begin{definice}
	Let ©A be a locally small category and let $A \in ©A$. We define a functor
	$$ H^A(·) := ©A(A, ·): ©A \rightarrow SET $$
	as follows:
	\begin{itemize}
		\item objects: $B \in ©A$, $H^A(B) := ©A(A, B)$;
		\item morphisms: for $B \overset{g}\rightarrow B' \in ©A$ the map $H^A(g) := ©A(A, g): ©A(A, B) \rightarrow ©A(A, B')$ is defined by $p \mapsto g ∘ p$.
	\end{itemize}
\end{definice}

\begin{poznamka}
	Let $(V, \<·, ·\>)$ be an inner product space. Define $f_V := \<v, ·\>: V \rightarrow ®K$, $w \mapsto \<v, w\>$. For some $v \in V$.
\end{poznamka}

\begin{definice}
	Let ©A be a locally small category, a functor $X: ©A \rightarrow SET$ is called representable if, for some $A \in ©A$, we have $X \simeq H^A$. A representation is a choice of isomorphism: $X \rightarrow H^A$.
\end{definice}

\begin{priklady}
	Let $G$ be a group and let $©G$ be the associated one object category. Recall that (functors $©G \rightarrow SET$ $\Leftrightarrow$ $G$-sets.)

	Since a representable functor is a functor, it must correspond to a $G$-set. The corresponding $G$-set is $G$ itself, i.e. the left regular representation. (Since we only have one object, we only have one representable functor $H^·: ©G \rightarrow SET, · \mapsto ©G(·, ·)$.)
\end{priklady}

\begin{tvrzeni}
	Any SET valued with a left adjoint is representable.

	\begin{dukazin}
		$G: ©A \rightarrow SET \implies ©G(A) \simeq SET(1, G(A))$, where $1$ is 1-element set.
	\end{dukazin}
\end{tvrzeni}

\begin{priklad}
	The forgetful functor $U: VECT_{®K} \rightarrow SET$ is representable, since it admits a left adjoint, i.e. the free functor.

	($f_v: V \rightarrow ®K$, $w \mapsto \<v, w\>$, $v \in V$, $\implies$ $V \rightarrow V^* = LIN_{®K}(V, ®K)$, $v \mapsto f_v = \<v, ·\>$.)
\end{priklad}

\begin{poznamka}
	A morphism $A' \overset{f}\rightarrow A$ induces a natural transformation $H^A \overset{H^f}\Rightarrow H^{A'}$, defined by $H^A(B) = ©A(A, B) \overset{H_B^f}\rightarrow H^{A'}(B) = ©A(A', B)$, $p \mapsto p∘f$.
\end{poznamka}

\begin{definice}
	Let ©A be a locally small cat, the functor $H^·: ©A\op \rightarrow [©A, SET]$ (functor category: objects are $F: ©A \rightarrow SET$, morphisms are natural transformations) is defined on objects $H^·(A) = H^A$ and on morphisms $H^·(f) = H^f$.
\end{definice}

\begin{poznamka}[Moral]
	This is a „representation“ of $©A\op$ in $[©A, SET]$. (Functor categories „nicer“ than general categories.)
\end{poznamka}

\begin{definice}
	Let ©A be a locally small category and $A \in ©A$. We define a functor $H_A: ©A(·, A): ©A \rightarrow SET$, as following:
	\begin{itemize}
		\item objects: $H_A(B) = ©A(B, A)$, $B \in ©A$;
		\item morphism: $B' \overset{g}\rightarrow B$ define $H_A(g) := ©A(g, A): ©A(B, A) \rightarrow ©A(B', A)$, $p \mapsto p∘g$.
	\end{itemize}
\end{definice}

\begin{poznamka}
	This now gives the definition of representable functor for functors $X: ©A\op \rightarrow SET$.
\end{poznamka}

\begin{definice}[Recall]
	The functor category $[©A\op, SET]$ is called the category of pre-sheaves on ©A.
\end{definice}

\begin{definice}
	Let ©A be a locally small category. The Yoneda embedding of ©A is the functor $H_·: ©A \rightarrow [©A\op, SET]$ (defined in analogy with $H^·$).

	\begin{poznamkain}
		Embedding for categories is defined at the level of homomorphism sets $©A(A, B) \overset{F}\rightarrow ©B(F(A), F(B))$ is injective, $\forall A, B \in ©A$, i.e. $F$ is a faithfull functor.
	\end{poznamkain}
\end{definice}

\begin{priklady}
	Recall the functor $C: TOP\op \rightarrow RING$, $X \mapsto C(X)$ (continuous functions from $X$ to $®C$ or $®R$, ring with respect to point-wise operations). The functor $TOP\op \overset{C}\rightarrow RING \underset{U}\rightarrow SET$ is representable since
	$$ U(C(X)) = TOP(X, ®R) \text{ or } TOP(X, ®C) = H_{®R}(X) \text{ or } H_{®C}(X). $$
\end{priklady}

\begin{veta}[Yoneda lemma]
	Let ©A be locally small small category. Then $[©A\op, SET](H_A, X) \simeq X(A)$. Naturally in $A \in ©A$, and $X \in [©A\op, SET]$ (pre-sheaf), where naturality means that the composite functor
	$$ ©A\op \times [©A\op, SET] \overset{H_·\op \times 1}\rightarrow [©A\op, SET]\op \times [©A\op, SET] \overset{Hom_{[©A\op, SET]}} SET, $$
	$$ (A, X) \mapsto (H_A, X) \mapsto [©A\op, SET](H_A, X) $$
	is naturally isomorphic to the evaluation functor
	$$ ©A\op \times [©A\op, SET] \overset{en}\rightarrow SET, \qquad (A, X) \mapsto X(A). $$
\end{veta}

\begin{priklad}
	Confirm how the two functors act on morphisms.
\end{priklad}

\begin{dukaz}[Yoneda]
	Strategy for the proof: we want a natural isomorphism between our two functors:
	\begin{itemize}
		\item components are isomorphisms in $SET$ labelled by the objects of $©A\op \times [©A\op, SET]$; (This we need to look up.)
		\item naturality conditions labelled by morphisms in $©A\op \times [©A\op, SET]$. (This is what we need to check.)
	\end{itemize}

	Let's focus this week on the first point. So for every $A \in ©A\op$, and every $X \in [©A\op, SET]$ we want an isomorphism of sets:
	$$ [©A\op, SET](H_A, X) \overset{\hat{\ }_{(A, X)}}\rightarrow X(A) \overset{\tilde{\ }_{(A, X)}}\rightarrow [©A\op, SET](H_A, X). $$
	$$ F(A) \overset{α_A}\rightarrow G(A) \overset{α_A^{-1}} F(A). $$

	\begin{lemmain}[Observation]
		A function is defined by $[©A\op, SET](H_A, X) \overset{\hat{\ }}\rightarrow X(A)$, $(α: H_A \rightarrow X) \mapsto \hat{α} := α_A(1_A)$.

		Rough work: $α: H_A \rightarrow X$ $\Leftrightarrow$ $α_B: ©A(B, A) \rightarrow X(B)$ ($B \in ©A$). Let's look at the case $B = A: α_A: ©A(A, A) \rightarrow X(A)$, $1_A \mapsto α_A(1_A)$.
	\end{lemmain}

	\begin{lemmain}
		A function is defined by
		$$ [©A\op, SET](H_A, X) \overset{\tilde{\ }}\leftarrow X(A), \qquad \tilde x \leftarrow| x. $$
	\end{lemmain}

	\begin{dukazin}[The previous lemma]
		$x \in X(A)$, we need natural transformation $\tilde X: H_A \rightarrow X$, that is, for each $B \in ©A\op$ a function $\tilde x_B: H_A(B) = ©A(B, A) \rightarrow X(V)$, which is natural in $B$.

		TODO!!!
	\end{dukazin}

	TODO!!!
\end{dukaz}

\end{document}
