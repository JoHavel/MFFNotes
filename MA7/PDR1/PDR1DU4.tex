\documentclass[12pt]{article}					% Začátek dokumentu
\usepackage{../../MFFStyle}					    % Import stylu



\begin{document}

TODO OPRAVIT!!! (Jsou tam chyby.)

\begin{priklad}[4. – Fredholm alternative vs Lax-Milgram lemma vs minimum principe]
	Consider $\Omega \subset ®R^d$ a Lipschitz domain. Let $®A: \Omega \rightarrow ®R^d$ be an elliptic matrix. Assume that $¦c \in L^∞(\Omega, ®R^d)$ and $b ≥ 0$. Consider the problem
	$$ -\Div(®A \nabla u) + bu + ¦c·\nabla u = f \text{ in } \Omega, \qquad u = u_0 \text{ on } \partial \Omega. $$

	\begin{itemize}
		\item[a)] Consider the case $b = 0$, $¦c = ¦o$ and $f \in L^2(\Omega)$ fulfilling $f ≥ 0$. Let $u_0 \in W^{1, 2}(\Omega)$ and denote $m := \essinf_{\partial \Omega} u_0$. Show that the unique weak solution $u$ satisfies $u ≥ m$ almost everywhere in $\Omega$.
	\end{itemize}

	\begin{dukazin}
		Jak nám napovídá hint, definujeme $\phi(x) := (u(x) - m)_-$. Jelikož $\Omega$ je omezené a $u \in W^{1, 2}$, tak
		$$ ||\phi||_2^2 = \int_\Omega (u(x) - m)_-^2 dx ≤ \int_\Omega (u(x) - m)^2 dx = ||u(x) - m||_2^2 ≤ \(||u(x)||_2 + ||m||_2\)^2 < ∞ $$
		
		Zároveň $u_0 = u ≥ m$ na $\partial \Omega$, tedy $\phi$ je na $\partial \Omega$ nulové.

		$$ \forall \psi \in C_0^∞(\Omega): \int_\Omega (\nabla \phi(x)) \psi(x) dx = \int_\Omega \nabla ((u(x) - m)_-)\psi(x) dx = $$
		$$ = \int_{u(x) > m} \nabla (u(x) - m) \psi(x) dx + \int_{u(x) ≤ m} 0 \psi(x) dx = \int_{u(x) > m} (\nabla u(x)) \psi(x) dx = $$
		$$ = \int_\Omega (\nabla \chi_{u(x) > m} u(x))·\psi(x) dx. $$
		Tedy $\nabla \phi = \nabla u \chi_{u(x) > m}$, tedy $||\nabla \phi||_2 < ||\nabla u||_2 < ∞$, tj. $\phi \in W_0^{1, 2}$.

		Nyní použijeme $\phi$ jako testovací funkci:
		$$ \int_\Omega -\Div(®A \nabla u) \phi = \int_\Omega f \phi $$
		$$ \underbrace{\int_\Omega ®A \nabla u \nabla \phi}_{≥ C_1 |\nabla u|^2 ≥ 0} = \int_\Omega \underbrace{f}_{>0} \underbrace{\phi}_{≤ 0} $$
		Tedy levá strana $≥0$, pravá $≤0$, tudíž se rovnají nule. Aby se pravá strana rovnala nule ($f$ je nenulové), tak musí být $\phi = 0$ skoro všude, tedy $u ≥ m$ skoro všude na $\Omega$.
	\end{dukazin}

\newpage

	\begin{itemize}
		\item[b)] Consider $b > 0$ and ¦c arbitrary. Prove that for any $u_0 \in W^{1, 2}(\Omega)$ and any $f \in L^2(\Omega)$ there exists a weak solution.
	\end{itemize}

	\begin{dukazin}
		Nejprve si podle hintu převedeme úlohu na důkaz tvrzení, že
		$$ -\Div(®A \nabla u) + b u + ¦c·\nabla u = 0 \text{ v } \Omega $$
		má pouze jedno řešení $u \in W_0^{1, 2}(\Omega)$, $u = 0$.

		Převedení bych moc rád udělal tak, že místo $f$ na pravé straně použiji $f - bu_0 - ¦c·\nabla u_0 + \Div(®A \nabla u)$, protože to k tomu hrozně nabádá, navíc mě nenapadá nic jiného, co by šlo použít, než Fredholmova alternativa a nenapadá mě žádný jiný postup, jak se dostat z FA na boundary value problém. Jenže $\Div(®A \nabla u)$ prostě nemusí být v $L^2$. Asi mi něco jednoduchého uniká, ale bohužel už nemám moc času do deadlinu. Takže řekněme, že nová pravá strana je v $L^2$.

		Potom z Fredholmovy alternativy a z tvrzení (pokud tedy platí, což si dokážeme dále) plyne, že problém
		$$ -\Div(®A \nabla u) + b u + ¦c·\nabla u = f - bu_0 - ¦c·\nabla u_0 + \Div(®A \nabla u) \text{ v } \Omega, \qquad u = 0 \text{ na } \partial \Omega, $$
		má (právě jedno) řešení $u \in W_0^{1, 2}(\Omega)$. Pokud tedy zvolíme $\tilde u = u + u_0$, pak $\tilde u$ je slabé řešení problému
		$$ -\Div(®A \nabla \tilde u) + b \tilde u + ¦c·\nabla \tilde u = f \text{ v } \Omega, \qquad \tilde{u} = u_0 \text{ na } \partial \Omega, $$
		neboť „všechno“ je zde lineární, takže „přičtením“ $u_0$ k $u$ na levé straně se přičtou odpovídající členy na pravé.
	\end{dukazin}

	\begin{dukazin}
		Mějme $u$ řešící $-\Div(®A \nabla u) + b u + ¦c·\nabla u = 0 \text{ v } \Omega$.

		Nyní dokážeme, že pro nějaké $M$ je $|u| < M$ skoro všude, tedy $u \in L^∞(\Omega)$ a $\|u\|_{L^∞} ≤ M$. Pokud $d = 1$, tak je z věty o vnoření $u$ spojité, takže se omezenost může „rozbíjet“ pouze na hranici $\Omega$, ale my víme, že $\tr u = 0$. Pro tuto část důkazu tedy předpokládejme $d > 1$.

		Ať $M > 0$ a $\phi_M := (u - M)_+$. Protože je $u \in W_0^{1, 2}(\Omega)$, tak $\phi_M \in W^{1, 2}(\Omega)$ ze stejných důvodů jako v a), $\nabla \phi_M = \nabla u · \chi_{u ≥ M}$, a navíc $\phi_M \in W_0^{1, 2}$, neboť $u$ zůstává 0 tam, kde 0 bylo.

		Tedy ho můžeme použít jako testovací funkci: $\int ®A \nabla u · \nabla \phi_M + b u \phi_M + ¦c·\nabla u \phi_M = \int 0 · \phi_M$.
		První a třetí člen už je na $u < M$ stejně nulový, tedy můžeme psát
		$$ \int ®A \nabla \phi_M · \nabla \phi_M + b u \phi_M = - \int ¦c·\nabla \phi_M \phi_M. $$
		Levou stranu můžeme zezdola odhadnout pomocí toho, že $b>0$, $\phi_M ≥ 0$ a tam, kde $\phi_M ≠ 0$, $u ≥ M > 0$. Navíc $®A$ je eliptické, takže
		$$ c_1\|\nabla \phi_M\|_2^2 = \int c_1 \|\nabla \phi_M\|_{®R^d}^2 ≤ \int ®A \nabla \phi_M · \nabla \phi_M + b u \phi_M = - \int ¦c·\nabla \phi_M \phi_M. $$
		
		Levou část můžeme shora odhadnout pomocí dvakrát použité Hölderovy nerovnosti:
		$$ - \int ¦c·\nabla \phi_M \phi_M ≤ \|c\|_∞ · \|\nabla \phi_M\|_2 · \|\phi_M\|_2. $$
		Nyní znovu použijeme Hölderovu nerovnost, tentokrát na $\|\phi_M\|_2$. Protože $\psi$ je na $u < M$ nulové, můžeme psát (jak bylo na přednášce)
		$$ \|\phi_M\|_2 = \sqrt{\int \phi_M^2} = \sqrt{\int \phi_M^2 \chi_{u ≥ M}} ≤ \sqrt{\(\int \phi_M^{2p}\)^{\frac{1}{p}} · \(\int \chi^q_{{u ≥ M}}\)^{\frac{1}{q}}} = \|\phi_M\|_{2p} · \(\int \chi_{{u ≥ M}}\)^{\frac{1}{2q}}, $$
		kde $\frac{1}{p} + \frac{1}{q} = 1$, avšak musíme použít správné $p ≠ 1$ ($p = 1$ nám nedává nic nového), aby $\phi_M \in L^{2p}$. To můžeme z věty o vnoření Sobolevových prostorů: pokud $d = 2$, tak $W^{1, 2}(\Omega) \hookrightarrow L^r$ pro $r$ jakékoliv, takže není co řešit. Pokud $d > 2$, tak můžeme vybrat $2p = r = \frac{d·2}{d - 2} = \frac{2}{1 - (2 / d)} > 2$ ($p > 1$).

		Nakonec $∞ > \int u ≥ \int_{u > M} u ≥ \int_{u > M} M$, tedy míra $\{u > M\}$ se musí pro rostoucí $M$ zmenšovat k nule. Takže můžeme zvolit libovolně malé $\(\int \chi_{{u ≥ M}}\)^{\frac{1}{2q}}$ v nerovnosti:
		$$ c_1·C·\|\nabla \phi_M\|_2·\|\phi_M\|_{2p} \overset{\text{Sob.}}{\underset{nerov}≤} c_1\|\nabla \phi_M\|_2^2 ≤ - \int ¦c·\nabla \phi_M \phi_M ≤ \|c\|_∞ · \|\nabla \phi_M\|_2 · \|\phi_M\|_{2p}\(\int \chi_{u ≥ M}\)^{\frac{1}{q}}, $$
		tedy $\|\nabla \phi_M\|_2 = 0$ (nebo $\|\phi_M\|_2 = 0$, ale to bychom byli hotovi). Tudíž se nám celá rovnost s testovací funkcí $\phi_M$ stala $\int b · u · \phi_M = 0$, ale $b > 0$, $u > 0$ (kde $\phi_M ≠ 0$), takže musí být $\phi_M = 0$ skoro všude, tedy $u ≤ M$ skoro všude.
	\end{dukazin}

	\begin{dukazin}
		Úplně stejně dostaneme $u ≥ -M'$ pro nějaké $M' > 0$ z $\phi_{M'} = (u + M)_-$, jelikož pak 
		$$ \int ®A \nabla \phi_{M'} · \nabla \phi_{M'} + b (-u) (-\phi_{M'}) = - \int ¦c·\nabla \phi_{M'} \phi_{M'}. $$
		má úplně stejné vlastnosti jako rovnice výše, jelikož v prvním členu je druhá mocnina, v druhém je to zase kladné a vpravo omezujeme vlastně absolutní hodnotu (víme, že pravá strana je nezáporná, takže i levá musí být) normami, takže na znamínkách nezáleží.
	\end{dukazin}

	\begin{dukazin}
		Nyní máme tedy dokázáno, že $u$ je „omezená skoro všude“, tedy $u \in L^∞$. Tedy i $u^k \in L^∞$ pro $k \in ®N$, navíc $\nabla u^k = k·u^{k-1}\nabla u$, protože $\nabla (u·…·u) = u \nabla (u·…·u) + (\nabla u) (u·…·u)$ a $u^{k-1} \in L^∞$, tedy $u^k \in W^{1, 2}(\Omega)$. Nakonec $\tr u^k|_{\partial \Omega} = 0$, neboť
		$$ \tr u^k|_{\partial \Omega} = u^k|_{\partial \Omega} = (u|_{\partial \Omega})^k = (\tr u|_{\partial \Omega})^k = 0^k =0. $$
		Tedy $u^k \in W_0^{1, 2}$.

		Použijme $u^k$ pro $k$ liché jako testovací funkci:
		$$ \int ®A \nabla u · \nabla u^k = \int k·u^{k-1}®A \nabla u · \nabla u = \int - b u·u^k - u^k¦c·\nabla u. $$
		Na levou stranu můžeme použít elipticitu ®A ($u^{k-1}$ je sudé), napravo je $-b u^{k+1}$ určitě záporné, tedy ji můžeme zvětšit přidáním absolutní hodnoty do části s ¦c:
		$$ \int c_1 (\nabla u)^2·k·u^{k-1} ≤ - \int b u^{k+1} + \int |¦c·\nabla u|·|u^{(k-1)/2}|·|u^{(k+1)/2}|. $$
		Chtěli bychom se zbavit integrálu s ¦c, tedy rozdělíme výraz jako výše a použijeme Yangovu nerovnost pro koeficienty $p=q=2$, tj. $\(|¦c·\nabla u|·|u^{(k-1)/2}|\)·|u^{(k+1)/2}| ≤ \frac{|¦c·\nabla u|^2·u^{k-1}}{2} + \frac{u^{k+1}}{2}$:
		$$ \int (c_1·k - |c|/2) (\nabla u)^2·u^{k-1} ≤ \int (-b+|c|/2) u^{k+1}. $$
		Hölderovou nerovností (pro $1, ∞$) (a tím, že na levé straně vytýkáme kladnou a zmenšujeme zápornou část a na pravé vytýkáme zápornou a zvětšujeme kladnou)
		$$ (c_1·k - \|c\|_∞/2) \int (\nabla u)^2·u^{k-1} ≤ (-b+\|c\|_∞/2) \int u^{k+1}, $$
		$$ \int (\nabla u)^2·u^{k-1} ≤ \frac{-b+\|c\|_∞/2}{c_1·k - \|c\|_∞/2} \|u^{\frac{k+1}{2}}\|_2^2. $$
		Teď už stačí jen $\int (\nabla u)^2·u^{k-1} ≥ \konst \|u\|_{k+1}$. Takže konstantu na pravé straně můžeme libovolně zmenšit, takže $\|u\|_{k+1} = 0$, tedy $u = 0$ skoro všude.
	\end{dukazin}
\end{priklad}

\end{document}
