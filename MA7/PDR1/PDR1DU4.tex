\documentclass[12pt]{article}					% Začátek dokumentu
\usepackage{../../MFFStyle}					    % Import stylu



\begin{document}

\begin{priklad}[4. – Fredholm alternative vs Lax-Milgram lemma vs minimum principe]
	Consider $Ω \subset ®R^d$ a Lipschitz domain. Let $®A: Ω \rightarrow ®R^d$ be an elliptic matrix. Assume that $¦c \in L^∞(Ω, ®R^d)$ and $b ≥ 0$. Consider the problem
	$$ -\Div(®A \nabla u) + bu + ¦c·\nabla u = f \text{ in } Ω, \qquad u = u_0 \text{ on } \partial Ω. $$

	\begin{itemize}
		\item[a)] Consider the case $b = 0$, $¦c = ¦o$ and $f \in L^2(Ω)$ fulfilling $f ≥ 0$. Let $u_0 \in W^{1, 2}(Ω)$ and denote $m := \essinf_{\partial Ω} u_0$. Show that the unique weak solution $u$ satisfies $u ≥ m$ almost everywhere in $Ω$.
	\end{itemize}

	\begin{dukazin}
		Jak nám napovídá hint, definujeme $φ(x) := (u(x) - m)_- = \min(u(x) - m, 0)$. Chceme $φ \in W_0^{1, 2}$. Víme, že $u \in W^{1, 2}$. Nejprve ukážeme „$φ \in L_2$“: ($φ$ je zřejmě měřitelná, neboť $u$ je měřitelná a $\min(\text{měřitelná}, \text{měřitelná})$ je měřitelná)
		$$ \int_Ω |φ|^2 = \int_Ω |(u - m)_-|^2 ≤ \int_Ω |u - m|^2 ≤ \int_Ω |u|^2 + |m|^2 = \|u\|_2^2 + |m|^2·λ^d(Ω) ≤ ∞, $$
		jelikož $u \in L^2$ a $Ω$ je omezená (protože je lipschitzovská).

		Na další stráně ukážeme $\nabla φ = χ_{\{u < m\}}\nabla u$. Tedy $\|\nabla φ\|_2 = \|χ\nabla u\|_2 ≤ \|\nabla u\|_2$ a $φ \in W^{1, 2}$. Následně „$\tr((u - m)_-) = 0$“: můžeme si všimnout, že $\tr(\min(a, b)) = \min(\tr(a), \tr(b))$ (platí pro $W^{1, 2} \cap C(\overline{Ω})$, a spojitostí $\tr$ a $\min$ rozšíříme na $W^{1, 2}$), tedy
		$$ \!\!\tr((u - m)_-) {=} \min(\tr(u - m), \tr(0)) {=} \min(\tr(u) - \tr(m), 0) {=} \min(\tr(u_0) - \tr(m), 0) \overset{\tr (u_0) ≥ m}= 0. $$

		Nyní můžeme použít $\phi$ jako testovací funkci:
		$$ 0 ≥ \int_Ω \underbrace{f}_{>0} \underbrace{\phi}_{≤ 0} = \int_Ω ®A \nabla u \nabla \phi = \int_Ω ®A \nabla u \(χ_{\{u < m\}}\nabla u\) = \int_Ω ®A \nabla u \(χ_{\{u < m\}}^2\nabla u\) = $$
		$$ = \int_Ω ®A \(χ_{\{u < m\}}\nabla u\) \(χ_{\{u < m\}}\nabla u\) = \int_Ω ®A \nabla φ \nabla φ ≥ \int_Ω C_1 |\nabla φ|^2 = C_1 \|\nabla φ\|_2^2 \overset{\text{Poincaré s $α_i = 0$}}≥ $$
		$$ ≥ c\|φ\|_{1, 2} ≥ 0. $$
		Tedy $\|φ\|_{1, 2} = 0$, tudíž $\|φ\|_2 = 0$, tj. $φ = 0$ skoro všude, a proto $u ≥ m$ skoro všude.
	\end{dukazin}

	\begin{dukazin}[$\nabla φ = χ_{\{u < m\}}\nabla u$]
		Z charakterizace sobolevovských funkcí víme $\exists u_n \in C^∞(Ω): u_n \overset{W^{1, 2}}\rightarrow u$. Tedy $\nabla u_n \overset{L^2}\rightarrow \nabla u$ a $u_n \overset{L^2}\rightarrow u$. Navíc z omezenosti $Ω$ a Hölderovi nerovnosti můžeme konvergenci v $L^2$ nahradit konvergencí v $L^1$.

		Označme $φ_n = (u_n - m)_-$. Potom
		$$ |φ_n - φ| = |(u_n - m)_- - (u - m)_-| ≤ |(u_n - m) - (u - m)| = |u_n - u| $$
		(třeba rozebráním všech čtyř možností), tedy $φ_n \overset{L_1}\rightarrow 0$. Tudíž
		$$ \forall ψ \in C^∞_0(Ω): \int_Ω φ_n \nabla ψ \rightarrow \int_Ω φ \nabla ψ \impliedby \left|\int_Ω (φ_n - φ) \nabla ψ\right| ≤ \|φ_n - φ\|_1 · \|\nabla ψ\|_∞. $$

		Nyní použijeme na tento integrál per partes (ze spojitosti $u_n$ jsou $\{u_n < m\}$ otevřené a $\partial\{u_n < m\} \subseteq \{u_n = m\}$):
		$$ \int_Ω φ_n \nabla ψ \leftarrow \int_Ω φ_n \nabla ψ = \int_Ω (u_n - m)_- \nabla ψ = \int_{\{u_n < m\}} (u_n - m) \nabla ψ + 0 \overset{\text{pp}}= $$
		$$ = \int_{\partial\{u_n < m\}} (u_n - m) ψ - \int_{\{u_n < m\}} \nabla u_n ψ - \int_Ω \nabla m ψ = 0 - \int_Ω χ_{\{u_n < m\}} \nabla u_n ψ - 0 = $$
		$$ = - \int_Ω χ_{u < m} \nabla u_n ψ + \int_Ω χ_{\{u < m \land u_n ≥ m\}}\nabla u ψ - \int_Ω χ_{\{u < m \land u_n ≥ m\}}(\nabla u - \nabla u_n) ψ. $$
		Poslední integrál konverguje k 0, neboť
		$$ \left|\int_Ω χ_{\{u < m \land u_n ≥ m\}}(\nabla u - \nabla u_n) ψ\right| ≤ \int_Ω |(\nabla u - \nabla u_n)ψ| ≤ \|\nabla u - \nabla u_n\|_1 · \|ψ\|_∞ \rightarrow 0, $$
		druhý jde k 0, protože $u_n \overset{L^1}\rightarrow u$ nám dává, že pro každé $ε$ je $λ^d(\{u_n ≥ m \land u < m - ε\}) \rightarrow 0$. Ale protože $λ^d(\{u_n ≥ m \land u < m - ε\}) \overset{ε \rightarrow 0}\rightarrow λ^d(\{u_n ≥ m \land u < m\})$, tak dostáváme, že $λ^d(\{u < m \land u_n ≥ m\}) \rightarrow 0$. Tedy integrujeme přes „mizející“ množinu.

		Nakonec třetí z těch integrálů konverguje k $- \int_Ω χ_{u < m} \nabla u ψ$, tedy $\nabla φ = χ_{u < m} \nabla u$, protože
		$$ \left|\int_Ω χ_{u < m} (\nabla u - \nabla u_n) ψ\right| ≤ \int_Ω |(\nabla u - \nabla u_n) ψ| ≤ \|\nabla u - \nabla u_n\|_1·\|ψ\|_∞ $$
	\end{dukazin}

\newpage

	\begin{itemize}
		\item[b)] Consider $b > 0$ and ¦c arbitrary. Prove that for any $u_0 \in W^{1, 2}(Ω)$ and any $f \in L^2(Ω)$ there exists a weak solution.
	\end{itemize}

	\begin{dukazin}
		Nejprve si podle hintu převedeme úlohu na důkaz tvrzení, že
		$$ -\Div(®A \nabla u) + b u + ¦c·\nabla u = 0 \text{ v } Ω $$
		má pouze jedno řešení $u \in W_0^{1, 2}(Ω)$, $u = 0$.

		Doplňme si do zadání $a_{ij}, b \in L^∞$. Pak $f - bu_0 - ¦c·\nabla u_0 + \Div(®A \nabla u) \in L^2$ (z Höldera a $u_0 \in W^{1, 2}$, tj. $u_0 \in L^2$ a $\nabla u_0 \in L^2$).

		Potom z Fredholmovy alternativy a z tvrzení (pokud tedy platí, což si dokážeme dále) plyne, že problém
		$$ -\Div(®A \nabla u) + b u + ¦c·\nabla u = f - bu_0 - ¦c·\nabla u_0 + \Div(®A \nabla u) \text{ v } Ω, \qquad u = 0 \text{ na } \partial Ω, $$
		má (právě jedno) řešení $u \in W_0^{1, 2}(Ω)$. Pokud tedy zvolíme $\tilde u = u + u_0$, pak $\tilde u$ je slabé řešení problému
		$$ -\Div(®A \nabla \tilde u) + b \tilde u + ¦c·\nabla \tilde u = f \text{ v } Ω, \qquad \tilde{u} = u_0 \text{ na } \partial Ω, $$
		neboť „všechno“ je zde lineární, takže „přičtením“ $u_0$ k $u$ na levé straně se přičtou odpovídající členy na pravé.
	\end{dukazin}

	\begin{dukazin}
		Mějme $u$ řešící $-\Div(®A \nabla u) + b u + ¦c·\nabla u = 0 \text{ v } Ω$.

		Nyní dokážeme, že pro nějaké $M$ je $|u| < M$ skoro všude, tedy $u \in L^∞(Ω)$ a $\|u\|_{L^∞} ≤ M$. Pokud $d = 1$, tak je z věty o vnoření $u$ spojité, takže se omezenost může „rozbíjet“ pouze na hranici $Ω$, ale my víme, že $\tr u = 0$. Pro tuto část důkazu tedy předpokládejme $d > 1$.

		Ať $M > 0$ a $\phi_M := (u - M)_+$. Protože je $u \in W_0^{1, 2}(Ω)$, tak $\phi_M \in W^{1, 2}(Ω)$ ze stejných důvodů jako v a), $\nabla \phi_M = \nabla u · \chi_{u ≥ M}$, a navíc $\phi_M \in W_0^{1, 2}$, neboť $u$ zůstává 0 tam, kde 0 bylo.

		Tedy ho můžeme použít jako testovací funkci: $\int ®A \nabla u · \nabla \phi_M + b u \phi_M + ¦c·\nabla u \phi_M = \int 0 · \phi_M$.
		První a třetí člen už je na $u < M$ stejně nulový, tedy můžeme psát
		$$ \int ®A \nabla \phi_M · \nabla \phi_M + b u \phi_M = - \int ¦c·\nabla \phi_M \phi_M. $$
		Levou stranu můžeme zezdola odhadnout pomocí toho, že $b>0$, $\phi_M ≥ 0$ a tam, kde $\phi_M ≠ 0$, $u ≥ M > 0$. Navíc $®A$ je eliptické, takže
		$$ c_1\|\nabla \phi_M\|_2^2 = \int c_1 \|\nabla \phi_M\|_{®R^d}^2 ≤ \int ®A \nabla \phi_M · \nabla \phi_M + b u \phi_M = - \int ¦c·\nabla \phi_M \phi_M. $$	
		Levou část můžeme shora odhadnout pomocí dvakrát použité Hölderovy nerovnosti:
		$$ - \int ¦c·\nabla \phi_M \phi_M ≤ \|¦c\|_∞ · \|\nabla \phi_M\|_2 · \|\phi_M\|_2. $$
		Nyní znovu použijeme Hölderovu nerovnost, tentokrát na $\|\phi_M\|_2$. Protože $\psi$ je na $u < M$ nulové, můžeme psát (jak bylo na přednášce)
		$$ \|\phi_M\|_2 = \sqrt{\int \phi_M^2} = \sqrt{\int \phi_M^2 \chi_{u ≥ M}} ≤ \sqrt{\(\int \phi_M^{2p}\)^{\frac{1}{p}} · \(\int \chi^q_{{u ≥ M}}\)^{\frac{1}{q}}} = \|\phi_M\|_{2p} · \(\int \chi_{{u ≥ M}}\)^{\frac{1}{2q}}, $$
		kde $\frac{1}{p} + \frac{1}{q} = 1$, avšak musíme použít správné $p ≠ 1$ ($p = 1$ nám nedává nic nového), aby $\phi_M \in L^{2p}$. To můžeme z věty o vnoření Sobolevových prostorů: pokud $d = 2$, tak $W^{1, 2}(Ω) \hookrightarrow L^r$ pro $r$ jakékoliv, takže není co řešit. Pokud $d > 2$, tak můžeme vybrat $2p = r = \frac{d·2}{d - 2} = \frac{2}{1 - (2 / d)} > 2$ ($p > 1$).

		Nakonec $∞ > \int u ≥ \int_{u > M} u ≥ \int_{u > M} M$, tedy míra $\{u > M\}$ se musí pro rostoucí $M$ zmenšovat k nule. Takže můžeme zvolit libovolně malé $\(\int \chi_{{u ≥ M}}\)^{\frac{1}{2q}}$ v nerovnosti:\vspace{-0.5em}
		$$ \!\!c_1·C·\|\nabla \phi_M\|_2·\|\phi_M\|_{2p} \!\!\overset{\text{Sob.}}{\underset{nerov}≤}\!\! c_1\|\nabla \phi_M\|_2^2 ≤ - \int ¦c·\nabla \phi_M \phi_M ≤ \|¦c\|_∞ · \|\nabla \phi_M\|_2 · \|\phi_M\|_{2p}\(\int \chi_{u ≥ M}\)^{\!\!\frac{1}{q}}\!\!\!, $$
		tedy $\|\nabla \phi_M\|_2 = 0$ (nebo $\|\phi_M\|_2 = 0$, ale to bychom byli hotovi). Tudíž se nám celá rovnost s testovací funkcí $\phi_M$ stala $\int b · u · \phi_M = 0$, ale $b > 0$, $u > 0$ (kde $\phi_M ≠ 0$), takže musí být $\phi_M = 0$ skoro všude, tedy $u ≤ M$ skoro všude.
	\end{dukazin}

	\begin{dukazin}
		Úplně stejně dostaneme $u ≥ -M'$ pro nějaké $M' > 0$ z $\phi_{M'} = (u + M)_-$, jelikož pak 
		$$ \int ®A \nabla \phi_{M'} · \nabla \phi_{M'} + b (-u) (-\phi_{M'}) = - \int ¦c·\nabla \phi_{M'} \phi_{M'}. $$
		má úplně stejné vlastnosti jako rovnice výše, jelikož v prvním členu je druhá mocnina, v druhém je to zase kladné a vpravo omezujeme vlastně absolutní hodnotu (víme, že pravá strana je nezáporná, takže i levá musí být) normami, takže na znamínkách nezáleží.
	\end{dukazin}

	\begin{dukazin}
		Nyní máme tedy dokázáno, že $u$ je „omezená skoro všude“, tedy $u \in L^∞$. Tedy i $u^k \in L^∞$ pro $k \in ®N$, navíc $\nabla u^k = k·u^{k-1}\nabla u$, protože $\nabla (u·…·u) = u \nabla (u·…·u) + (\nabla u) (u·…·u)$ a $u^{k-1} \in L^∞$, tedy $u^k \in W^{1, 2}(Ω)$. Nakonec $\tr u^k|_{\partial Ω} = 0$, neboť
		$$ \tr u^k|_{\partial Ω} = u^k|_{\partial Ω} = (u|_{\partial Ω})^k = (\tr u|_{\partial Ω})^k = 0^k =0. $$
		
		Tedy $u^k \in W_0^{1, 2}$. Použijme $u^k$ pro $k$ liché jako testovací funkci:
		$$ \int ®A \nabla u · \nabla u^k = \int k·u^{k-1}®A \nabla u · \nabla u = \int - b u·u^k - u^k¦c·\nabla u. $$
		Na levou stranu můžeme použít elipticitu ®A (existuje $c_1 > 0$, že $®A ¦v ¦v ≥ c_1 |¦v|$), napravo je $-b u^{k+1}$ určitě záporné, tedy ji můžeme zvětšit přidáním absolutní hodnoty do části s ¦c (absolutní hodnotu si hned rozdělíme pro použití v Youngově nerovnosti, použitím faktu $k+1$ a $k - 1$ jsou sudá):
		$$ \int c_1 (\nabla u)^2·k·u^{k-1} = \int k·u^{k-1}®A \nabla u · \nabla u ≤ - \int b u^{k+1} + \int |¦c·\nabla u|·|u^{(k-1)/2}|·|u^{(k+1)/2}|. $$
		Chtěli bychom se zbavit integrálu s ¦c, tedy použijeme Youngovu nerovnost pro koeficienty $p=q=2$, tj. $\(|¦c·\nabla u|·|u^{(k-1)/2}| / \sqrt{K}\)·|u^{(k+1)/2}|·\sqrt{K} ≤ \frac{|¦c·\nabla u|^2·u^{k-1}}{2K} + \frac{K·u^{k+1}}{2}$, kde $K > 0$ libovolné: (na $¦c·\nabla u$ navíc použijeme Cauchy-Schwarze)
		$$ \int c_1 (\nabla u)^2·k·u^{k-1} ≤ - \int b u^{k+1} + \int |…| ≤ -\int b u^{k+1} + \int \frac{|¦c·\nabla u|^2·u^{k-1}}{2K} + \frac{u^{k+1}·K}{2}, $$
		$$ \int (c_1·k - |¦c|^2/2K) (\nabla u)^2·u^{k-1} ≤ \int (-b+K/2) u^{k+1}. $$
		Tím, že $-|¦c| ≥ -\|¦c\|_∞$ a $-b ≤ -\essinf b$ skoro všude, a Hölderovou nerovností (pro $1, ∞$):
		$$ (c_1·k - \|¦c\|_∞^2/2K) \int (\nabla u)^2·u^{k-1} ≤ (-\essinf b+K/2) \int u^{k+1}. $$
		Volbou $K = \frac{1}{\sqrt{k}}$ dostaneme $\sqrt{k}(c_1·\sqrt{k} - \|¦c\|_∞^2/2) \int (\nabla u)^2·u^{k-1} ≤ (-\essinf b+1/2\sqrt{k}) \int u^{k+1}$. Pro $k > \|¦c\|_∞^4 / 4c_1^2$ je levá strana nezáporná. A je-li $\essinf b > 0$, pak pro $k > 4 / (\essinf b)^2$ je koeficient vpravo záporný, tedy $\int u^{k+1} = 0$, takže $\|u\|_{k+1} = 0$,
		tedy $u = 0$.
	\end{dukazin}

	%\begin{dukazin}
	%	Pokud $\essinf b = 0$, pak zvolíme $K = \frac{\|¦c\|_∞^2}{2\(c_1·k + (k+1)^2 / 4C\)}$, kde $C$ je z Poincarého nerovnosti v:
	%	$$ \int (\nabla u)^2 · u^{k - 1} = \int \(\frac{1}{\frac{k+1}{2}}\)^2 · (\nabla u^{\frac{k+1}{2}})^2 \overset{\text{Poincaré}}≥ \frac{4C}{(k+1)^2}·\|u^{\frac{k+1}{2}}\|_{1, 2}^2 ≥ \frac{4C}{(k+1)^2}·\|u^{\frac{k+1}{2}}\|_2^2. $$
	%	Potom $K \rightarrow 0$ a $\|u^{\frac{k+1}{2}}\|_2^2 ≤ \frac{K}{2} \int u^{k+1} = \frac{K}{2} \|u^{\frac{k+1}{2}}\|_2^2$, tedy $\|u^{\frac{k+1}{2}}\|_2 = 0$, tj. $u^{\frac{k+1}{2}} = 0$, takže $u = 0$.
	%\end{dukazin}
\end{priklad}

\end{document}
