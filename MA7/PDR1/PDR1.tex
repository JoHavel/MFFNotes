\documentclass[12pt]{article}					% Začátek dokumentu
\usepackage{../../MFFStyle}					    % Import stylu



\begin{document}

% 02. 10. 2022
TODO(you should know).

TODO(motivation)

\section{Sobolev spaces}
\begin{definice}[Multiindex]
	$α$ je multi-index $≡$ $α = (α_1, …, α_d)$, $α_i \in ®N$. Length of multi-index $α$ is $|α| := α_1 + … + α_d$. If $u \in C^k(Ω)$ then $D^α := \frac{\partial^{|α|}u}{\partial x_1^{α_1} … \partial x_d^{α_d}}$, $α ≤ k$.
\end{definice}

\begin{definice}[Weak derivative]
	Let $u, v_α \in L^1_{loc}(Ω)$ and $α$ be a multi-index. We say that $v_α$ is the $α$-th weak derivative of $u$ in $Ω$ iff $\forall φ \in C_0^∞(Ω): \int_Ω uD^α φ = (-1)^{|α|} \int v_α φ$.
\end{definice}

\begin{lemma}
	Weak derivative is unique. If the classical derivative exists then it is also the weak derivative.

	\begin{dukazin}
		Let $v_α^1$ and $v_α^2$ be two weak derivatives. Then
		$$ \int_Ω (v_α^1 - v_α^2) φ = 0 \qquad \forall φ \in C_0^∞(Ω) $$
		$\implies v_α^1 = v_α^2$ almost everywhere in $Ω$.

		If classical $D^α u$ exists, then
		$$ \int_Ω \underbrace{D^α u}_{v_α} φ \overset{\text{BP}} = (-1)^{|α|} \int_Ω u D^α φ. $$
	\end{dukazin}
\end{lemma}

\begin{poznamka}[Notation for this course]
	$D^α$ always means the weak derivative.
\end{poznamka}

\begin{definice}[Sobolev space]
	Let $Ω \subseteq ®R^d$ be open, $k \in ®N$, $p \in [1, ∞]$. We define $W^{k, p}(Ω) = \{u \in L^p(Ω) | \forall α, |α| ≤ k: D^αu \in L^p(Ω)\}$.
	$$ \|u\|_{W^{k, p}(Ω)} = \(\sum_{α, |α| ≤ k} \|D^α u\|_{L^p(Ω)}^p\)^{\frac{1}{p}} $$
	$$ \|u\|_{W^{k, ∞}(Ω)} = \sup_{α, |α| ≤ k} \|D^α u\|_{L^∞(Ω)} $$
\end{definice}

\begin{lemma}[Base properties of Sobolev spaces]
	Let $u, v \in W^{k, p}(Ω)$, $k \in ®N$ and $α$ is multi-index. Then
	\begin{itemize}
		\item $D^α u \in W^{k - |α|, p}(Ω)$, if $|α| ≤ k$;
		\item $λu + μv \in W^{k, p}(Ω)$ $\forall λ, μ \in ®R$ ($D^α(λu + μv) = λD^α u + μD^α v$);
		\item $\tildeΩ \subset Ω$ open, $u \in W^{k, p}(\tildeΩ)$;
		\item $\forall η \in C^∞(Ω): η·u \in W^{k, p}(Ω)$.
	\end{itemize}
\end{lemma}

% 09. 10. 2023

TODO?

\begin{veta}[Properties of Sobolev spaces]
	$Ω \subseteq ®R^d$, $p \in [1, ∞]$, $k \in ®N$:
	\begin{enumerate}
		\item $W^{k, p}(Ω)$ is a Banach space;
		\item if $p < ∞$, then $W^{k, p}(Ω)$ is separable;
		\item if $p \in (1, ∞)$, then $W^{k, p}(Ω)$ is reflexive.
	\end{enumerate}

	\begin{dukazin}[1.]
		„Linear space“ is from Minkowski inequality. „Completeness“: $u^n$ is Cauchy in $W^{k, p}(Ω)$ $\implies$ $\exists u \in W^{k, p}(Ω)$ $\implies$ $u^n \rightarrow u$ in $L^p(Ω)$, $D^α u^n \rightarrow v_α$ in $L^p(Ω)$ $\forall |α| ≤ k$. We must check that „$v_α = D^α u$“:
		$$ \int_Ω v_α η dx = \int_Ω (v_α - D^α u^n)η + \int_Ω D^α u^n η = $$
		$$ \overset{IBP}= \int_Ω (v_α - D^α u^n)η + (-1)^{|α|} \int_Ω u^n D^αη = TODO? $$
		
		$$ |\int_Ω (v_α - D^α u^n)η| ≤ \|η\|_∞·\|v_α - D^α u^n\|_{L^p} \rightarrow 0. $$
	\end{dukazin}
	
	\begin{dukazin}[2. + 3. for $W^{1, p}(Ω)$]
		$W^{1, p}(Ω) = X \subseteq L^p(Ω) \times … L^p(Ω)$ ($d + 1$ times) and it is closed.
	\end{dukazin}
\end{veta}

\subsection{Approximation of Sobolev functions}
\begin{veta}
	$Ω \subseteq ®R^d$ open, bounded. $k \in ®N$, $p \in [1, ∞)$. Then
	$$ \overline{©C^∞(Ω)}^{\|·\|_{k, p}} = W^{k, p}(Ω). $$

	\begin{dukazin}
		TODO!!!
	\end{dukazin}
\end{veta}

\begin{upozorneni}
	$$ \overline{©C^∞(\overline{Ω})}^{\|·\|_{k, p}} ≠ W^{k, p}(Ω). $$
\end{upozorneni}

\begin{poznamka}
	If $Ω \subset ®R^d$ open, connected, then $u = \const \Leftrightarrow \frac{\partial u}{\partial x_i} = 0\ \forall_i = 1, …, d$.

	$W^{1, 1}(I)$, $I$ interval. Then $W^{1, 1}(I) \hookrightarrow C(\overline{I})$.
	\begin{dukazin}
		„$\implies$“: easy. „$\impliedby$“: $u_ε = u * η_ε$, $Ω_ε := \{x \in Ω, \dist(x, \partial Ω) > ε\}$.
		$$ x \in Ω_ε: \frac{\partial u_ε}{\partial x_1}(x) = \(\frac{\partial u}{\partial x_i}\)_ε(x) = 0 \implies u_ε ≡ \const \text{ in } Ω_ε. $$
		Fix $ε_0 > 0$: $ε ≤ ε_0$: $u_ε \rightarrow u$ in $W^{1, 1}(Ω_{ε_0})$ $\implies$ $u ≡ \const$ in $Ω_{ε_0}$. $u \in W^{1, 1}(I)$.
		$$ \tilde u(x) := \int_0^x \frac{\partial u(y)}{\partial y} dy, \qquad \|\tilde u(x)\|_∞ ≤ \int_0^1 |\nabla u| dx. $$

		Aim $\frac{\partial \tilde u}{\partial x} = \frac{\partial u}{\partial x}$. $η \in C_0^∞(0, 1)$:
		$$ \int_0^1 \tilde u(x) \frac{\partial η}{\partial x}(x) dx = \int_0^1 \int_0^1 \frac{\partial u(y)}{\partial y} \frac{\partial η(x)}{\partial x} χ_{\{0 ≤ y ≤ x\}} dx dy = $$
		$$ = \int_0^1 \int_y^1 \frac{\partial u}{\partial y}(y) \frac{\partial η(x)}{\partial x_i} dx dy = $$
		$$ = -\int_0^1 \frac{\partial u(y)}{\partial y}η(y) dy. $$
		$\implies \tilde u - u = \const =: c$.
		$$ |u(x_1) - u(x_2)| = |\tilde(x_1) - c - \tilde u(x_2) + c| = |\tilde u(x_1) - \tilde u(x_2)| ≤ \int_{x_1}^{x_2}\left|\frac{\partial u}{\partial y}\right| dy \rightarrow 0. $$
		$\implies C(I)$.

		„$\|u\|_∞ ≤ K·\|u\|_1$“:
		$$ |c| = \int_0^1 |c| = \int_0^1 |\tilde u(x) - u(x)| ≤ \|\tilde u\|_∞ + \|u\|_1 ≤ \|u\|_{1, 1}. $$
	\end{dukazin}

	$W^{d, 1}(Ω) \hookrightarrow C(\overline{Ω})$ (for Lipschitz domain $Ω$).
\end{poznamka}

\subsection{Characterization of Sobolev functions}
\begin{veta}
	Let $Ω \subset ®R^d$, $p \in [1, ∞]$, $Ω_d := \{x \in Ω | \dist(x, \partial Ω) > δ\}$. 1. Then
	$$ u \in W^{1, p}(Ω): \|\triangle_i^n u\|_{L^p(Ω_δ)} ≤ \left\|\frac{\partial u}{\partial x_i}\right\|_{L^p(Ω)}, $$
	where $\triangle_i^n u(x) = \frac{u(x + h e_i) - u(x)}{h}$.

	2. If $\forall h, i, δ: \|\triangle_i^h u\|_{L^p(Ω_δ)} ≤ c_i$ ($p > 1$). Then
	$$ \exists \frac{\partial u}{\partial x_i}, \qquad \left\|\frac{\partial u}{\partial x_i}\right\|_{L^p(Ω)} ≤ c_i. $$
\end{veta}

\begin{upozorneni}
	This works only for $(p > 1)$.
\end{upozorneni}

\begin{definice}[Domains of class $C^{k, α}$]
	Let $\Omega \subseteq ®R^d$ open bounded set. We say that $\Omega \in C^{k, \mu}$ $(\partial\Omega \in C^{k, \mu})$ iff:
	\begin{itemize}
		\item there exist $M$ coordinate systems $¦x = \(x_{r_1}, …, x_{r_d}\) = \(x_r', x_{r_d}\)$ and functions $a_r: \Delta_r \rightarrow ®R$ where $\Delta_r = \{x_r' \in ®R^{d - 1} |\ |x_{r_i}| ≤ \alpha\}$ such that $a_r \in C^{k, \mu}(\Delta_r)$,
		\item denoting $Tr$ the orthogonal transformation from $(x_r', x_{r_d})$ to $(x', x_d)$, then $\forall x \in \partial \Omega$ $\exists r \in \{1, …, M\}$ such that $x = Tr\(x_{r_1}', a(x_{r_d})\)$,
		\item $\exists \beta > 0$, if we define
			$$ V_r^+ := \{(x_r', x_{r_d}) \in ®R^d | x_r' \in \Delta_r, a(x_r') < x_{r_d} < a(x_r') + \beta\} $$
			$$ V_r^- := \{(x_r', x_{r_d}) \in ®R^d | x_r' \in \Delta_r, a(x_r') - \beta < x_{r_d} < a(x_r')\} $$
			$$ \Lambda_r := \{(x_r', x_{r_d}) \in ®R^d | x_r' \in \Delta_r, a(x_r') = x_{r_d}\} $$
			Then $Tr (V_r^+) \subset \Omega, Tr(V_r^-) \subset ®R^d \setminus \overline{\Omega}, Tr(\Lambda_r) \subseteq \partial \Omega$ and $\bigcup_{r=1}^M Tr(\Lambda_r) = \partial \Omega$.
	\end{itemize}
\end{definice}

\begin{veta}[Density]
	Let $Ω \in C^{0, 1}$ and $p ≠ ∞$, then $W^{k, p}(Ω) = \overline{C^∞(\overline{Ω})}^{\|·\|_{k, p}}$.
\end{veta}

\begin{veta}[Extension]
	Let $Ω \in C^{0, 1}$, $k \in ®N$, $p \in [1, ∞]$. Then $\exists $ continuous bounded operator $E: W^{k, p}(Ω) \rightarrow W^{k, p}(®R^d)$ such that
	\begin{enumerate}
		\item $\|Eu\|_{W^{k, p}}(®R^d) ≤ c·\|u\|_{W^{k, p}(Ω)}$ ($E u$ has compact support);
		\item $Eu = u$ almost everywhere in $Ω$.
	\end{enumerate}
\end{veta}

\begin{veta}[Trace]
	Let $Ω \in C^{0, 1}$, $p \in [1, ∞]$. Then $\exists $ continuous bounded operator $\tr: W^{1, p}(Ω) \rightarrow L^p(\partial Ω)$ such that:
	\begin{enumerate}
		\item $\|\tr u\|_{L^p(\partial Ω)} ≤ c·\|u\|_{W^{1, p}(Ω)}$;
		\item $u \in W^{1, p}(Ω) \cap C(\overline{Ω}) \implies \tr u = u |_{\partial Ω}$.
	\end{enumerate}
\end{veta}

\begin{definice}
	$W^{k, p}_0(Ω) = \overline{u \in C^∞_0(Ω)}^{\|·\|_{k, p}}$.

	\begin{poznamkain}
		$W^{1, p}_0(Ω) = \{u \in W^{1, p}(Ω) | \tr u = 0\}$.
	\end{poznamkain}
\end{definice}

% 16. 10. 2023

TODO!!!

% 23. 10. 2023

\subsection{Existence theory via Lax–Milgram}

\begin{definice}[Eliptic forms]
	Let $B: V \times V \rightarrow ®R$ a linear form and $V$ be a Hilbert space. We say that $B$ is elliptic iff
	\begin{enumerate}
		\item $B$ is $V$-bounded, it is $\exists c_2\ \forall u, φ \in V: |B(u, φ)| ≤ c_2 \|u\|_V·\|φ\|_V$;
		\item $B$ is $V$-coercive, it is $\exists c_1 > 0\ \forall u \in V: B(u, u) ≥ c_1 \|u\|_V^2$.
	\end{enumerate}
\end{definice}

\begin{veta}[Lax–Milgram]
	Let $B$ be linear and satisfying two previous properties. Then $\forall F \in V^*\ \exists! u \in V: \forall φ \in V: B(u, φ) = \<F, φ\>$.
\end{veta}

\begin{definice}[Lipschitz, unifomly monototne]
	Let $B: V \rightarrow V^*$. We say that $B$ is
	\begin{enumerate}
		\item Lipschitz iff $\forall u, v \in V: \|B(u) - B(v)\|_{V^*} ≤ \overline{c_2} \|u - v\|_V$;
		\item uniformly monotone iff $\forall u, v \in V: \<B(u) - B(v), u - v\> ≥ \overline{c_1} \|u - v\|_V^2$.
	\end{enumerate}
\end{definice}

\begin{veta}[Non-linear Lax–Milgram]
	Let $V$ be a Hilbert space, $B: V \rightarrow V^*$ be Lipschitz and uniformly monotone. Then $\forall F \in V^*\ \exists! u \in V: \forall φ \in V: \<B(u), φ\> = \<F, φ\>$. ($B(u) = F$.)
\end{veta}

\begin{dukaz}[Lax–Milgram by using non-linear version]
	Define $B: V \rightarrow V^*: \<B(u), φ\> =: B(u, φ)$. We show that $B$ is Lipschitz and uniformly monotone:
	$$ \|B(u) - B(v)\|_{V^*} = \sup_{φ \in V, \|φ\| ≤ 1} \<B(u) - B(v), φ\> = \sup_φ (B(u, φ) - B(v, φ)) \overset{\text{linear}}= \sup_φ B(u - v, φ) \overset{\text{bounded}}≤ \sup_φ c_2 \|u - v\|_V·\|φ\|_V ≤ c_2 \|u - v\|_V. $$
	$$ \<B(u) - B(v), u - v\> = B(u, u - v) - B(v, u - v) = B(u - v, u - v) ≥ c_1 \|u - v\|_V^2. $$

	So $\forall F^*\ \exists! u \in V: B(u, φ) = \<B(u), φ\> = \<F, φ\>$ ($\forall φ \in V$).
\end{dukaz}

\begin{dukazin}[Non-linear Lax–Milgram]
	„Uniqueness“: $u_1 ≠ u_2$:
	$$ \forall φ: \<B(u_1), φ\> = \<B(u_2), φ\> = \<F\> \implies \forall φ \in V: \<B(u_1) - B(u_2), φ\> = 0. $$
	$$ φ = u_1 - u_2 \implies 0 = \<B(u_1) - B(u_2), u_1 - u_2\> ≥ \overline{c_1}\|u_1 - u_2\|_V^2 \implies u_1 = u_2. $$

	„Existence“: $\forall \<B(u), φ\> = \<F, φ\>$ $\Leftrightarrow$ $\exists ε > 0: (u, φ)_V = (u, φ)_V - ε\(\<B(u), φ\> - \<F, φ\>\)$. Desire $M: V \rightarrow V, v \mapsto u$:
	$$ (u, v)_V = (v, φ)_V - ε\(\<B(v), φ\> - \<F, φ\>\). $$
	If $M$ is well-defined and if it has a fixed point then we find solution.

	„$M$ well defined“:
	$$ \forall v \in V\ \exists \tilde F \in V^* \<\tilde F, φ\>_V = (v, φ)_V - ε\(\<B(v), φ\> - \<F, φ\>\). $$
	Riesz $\implies$ $\exists! u \in V (u, φ) = \<\tilde F, φ\>$.

	„$M$ contraction“: We want bound $\|M(u) - M(v)\|_V^2$.
	$$ M(u) = \overline{u}, \qquad (\overline{u}, φ) = (u, φ) - ε\(\<B(u), φ\> - \<F, φ\>\). $$
	$$ M(v) = \overline{v}, \qquad (\overline{v}, φ) = (v, φ) - ε\(\<B(v), φ\> - \<F, φ\>\). $$
	$$ (\overline{u} - \overline{v}, φ) = (u - v, φ) - ε\<B(u) - B(v), φ\>. $$
	Riesz: $\exists! w_1: \<B(u), φ\> = (w_1, φ)$ and $\exists! w_2: \<B(v), φ\> = (w_2, φ)$.

	$$ (*) \|\overline{u} - \overline{v}\|_V^2 = \|u - v - ε(w_1 - w_2)\|_V^2 = \|u - v\|_V^2 + ε^2 \|w_1 - w_2\|_V^2 - 2ε(w_1 - w_2, u - v). $$

	$$ (w_1 - w_2, w_1 - w_2) = \|w_1 - w_2\|_V^2 = \<B(u) - B(v), w_1 - w_2\> ≤ \|B(u) - B(v)\|_{V^*}·\|w_1 - w_2\|_V \implies \|w_1 - w_2\|_V ≤ \|B(u) - B(v)\|_{V^*} ≤ c_2·\|u - v\|_V. $$
	$$ (w_1 - w_2, u - v) = \<B(u) - B(v), u - v\> \overset{\text{Uniformly monotone}}≥ c_1 \|u - v\|_V^2 $$
	By (*):
	$$ \|M(u) - M(v)\|_V^2 ≤ \|u - v\|_V^2 + ε^2 \overline{c}_2^2 \|u - v\|_V^2 - 2ε \overline{c}_1 \|u - v\|_V^2 = $$
	$$ = (1 + ε^2 \overline{c}_2 - 2ε\overline{c}_1) \|u - v\|_V^2. $$
	$\implies$ $M$ is contraction and has a fixed point (for $ε$ such that this constant is less than 1, so $ε(ε c_2^2 - 2c_1) < 0$, so $0 < ε < \frac{2c_1}{c_2^2}$).
\end{dukazin}

\begin{veta}
	Let $B_{L, ς}$ be bilinear, $V$-bounded and $V$-elliptic. Then $\exists! u$ weak solution:
	$$ u_0 \in W^{1, 2}, \qquad u - 0 \in V, \qquad B_{L, ς}(u, φ) = \<f, φ\> + \int_{Γ_2 \cup Γ_3} g φ. $$

	\begin{dukazin}
		„Uniqueness“: $u_1, u_2 \implies u_1 - u_0 \in V, u_2 - u_0 \in V$ TODO!!!

		„Existence“: $w \in V$ ($u - u_0 = w$):
		$$ B_{L, ς}(w, φ) = \<f, φ\> + \int_{Γ_2 \cup Γ_3} g φ - B_{L, ς}(u_0, φ) := \<\overline{F}, φ\>. $$
		Find $w \in V$.
		$$ \<\overline{F}, φ\> := \<f, φ\> + \int_{Γ_2 \cup Γ_3} g φ + \int_Ω -A \nabla u·\nabla φ - bu_0φ - ¦c·\nabla u_0φ + ¦d·\nabla φ u_0 - \int_{Γ_2} ς u_0 φ. $$
		Is $\overline{F}$ in $V^*$? ($φ \in V \subseteq W^{1, 2}(Ω)$)
		$$ |\<\overline{F}, φ\>| ≤ \|φ\|_V c\(\|f\|_{V^*} + \|g\|_{L^2(\partial Ω)} + \|A\|_∞·\|u_0\|_{1, 2} + \|b\|_∞ \|u_0\|_2 + \|c\|_∞\|u_0\|_{1, 2} + …\). $$
	\end{dukazin}
\end{veta}

TODO!!!

\subsection{Existence theory via Fredholm alternative}
\begin{lemma}[Fredholm alternative]
	Let $H$ be a Hilbert space and $K: H \rightarrow H$ be linear compact.
	\begin{itemize}
		\item[F1] $\Ker(I - K)$ has finite dimension ($u \in \Ker(I - K) \Leftrightarrow (I - K)(u) = 0$);
		\item[F2] $\Rang(I - K)$ is closed ($u \in \Rang(I - K) \Leftrightarrow \exists w \in H (I - K)w = u$);
		\item[F3] $\Rang(I - K) = (\Ker(I - K^*))^\perp$ ($u \in \Rang(I - K), w \in \Ker(I - K^*) \Leftrightarrow (u, w) = 0$);
		\item[F4!] $\Ker(I - K) = \{¦o\} \Leftrightarrow \Rang(I - K) = H$;
		\item[F5] $\dim(\Ker(I - K)) = \dim(\Ker(I - K^*)) < ∞$;
		\item[F6] spectrum of $K$ is at most countable and if it is infinite then zero is the only attracting point.
	\end{itemize}
\end{lemma}

\begin{veta}[Fredholm alternative for PDR]
	Let $Ω \in C^{0, 1}$, $u_0 = 0$ and $Γ_1 = \partial Ω$ and $L$ be an elliptic operator.
	\begin{enumerate}
		\item Either $\forall f \in L^2(Ω)\ \exists! u \in W_0^{1, 2}(Ω): Lu = f$ in $Ω$ and $u = 0$ on $\partial Ω$, or $\exists u ≠ 0: Lu = 0$ in $Ω$ and $u = 0$ on $\partial Ω$.
		\item $N_L := \{u \in V | Lu = ¦o\}$: $B_L(u, φ) = 0$ $\forall φ \in W_0^{1, 2}(Ω)$, $N_{L^*} := \{φ \in V | L^* φ = ¦o\}$. Then $N_L$ and $N_{L^*}$ are closed subspaces of $W_0^{1, 2}(Ω)$, $\dim N_L = \dim N_{L^*} < ∞$.
		\item For $f \in L^2(Ω)$: $(\exists u \in W_0^{1, 2}: Lu = f) \Leftrightarrow (\forall φ \in N_{L^*}: \int_Ω fφ = 0)$.
	\end{enumerate}

	Where
	$$ Lu = -\Div(A \nabla u) + ¦c·\nabla u + \Div(¦du) + bu, $$
	$$ L^*φ = -\Div(A^T \nabla φ) - ¦d·\nabla φ - \Div(¦c φ) + bφ. $$
	$$ Lu = f \Leftrightarrow \forall φ: B_L(u, φ) = \int_Ω fφ, \qquad L^*φ = g \Leftrightarrow \forall u: B_L(φ, u) = \int_Ω g u. $$
	
	\begin{dukazin}
		From Lax–Milgram $\exists p > 0$:
		$$ \forall f \in L^2\ \exists! u \in W_0^{1, 2}(Ω) L_pu := Lu + pu = f. $$
		$$ B_{L, p}(u, φ) = B_L(u, φ) + p·\int_Ω u φ. $$
	\end{dukazin}
\end{veta}

\end{document}
