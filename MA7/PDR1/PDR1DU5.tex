\documentclass[12pt]{article}					% Začátek dokumentu
\usepackage{../../MFFStyle}					    % Import stylu



\begin{document}

\begin{priklad}[5. – Lax-Milgram lemma vs Fredholm alternative II]
	Consider $\Omega \subset ®R^d$ a Lipschitz domain. Let $a, b \in ®R$. Consider the problem: For given $¦f = (f_1, f_2) \in \(L^2(\Omega)\)^2$ find $¦u = (u_1, u_2) \in \(W_0^{1, 2} (\Omega)\)^2$ solving
	\begin{alignat*}{2}
		-Δ u_1 - aΔu_2 + u_1 &= f_1\quad &&\text{ in }\Omega, \\
		-Δ u_2 - bΔu_1 + u_2 &= f_2\quad &&\text{ in }\Omega, \\
		u_1 = u_2 = u_3 &= 0 \quad &&\text{ on } \partial \Omega. \\
	\end{alignat*}
	Under which conditions on $a, b$ the system has for any ¦f a weak solution?

	\begin{poznamkain}[Prerekvizita řešení: vlastní vektory $Δ$ jako ortogonálni báze $W^{1, 2}_0$ a ortonormální množina v $L^2$]
		Definujme operátor $Δ: W^{1, 2}_0 \rightarrow W^{1, 2}_0$ tak, že $Δu$ je taková pravá strana $f \in W^{1, 2}_0$, že $u$ je slabým řešením $Δu = f$, neboli $-\int \nabla u · \nabla v = \int f v$ ($= \int Δu\,v$).

		% $Δ$ je zjevně lineární a spojité v $W^{1, 2}_0$. $Δ$ je na, jelikož $Δu = f$ má slabé řešení $u \in W^{1, 2}_0$ pro každé $f \in L^2 \supset W^{1, 2}_0$.
		%
		%Tedy inverzní operátor k $Δ$ je dobře definován, navíc kompaktní, neboť z Poincarého je obraz omezené množiny omezený a z kompaktního vnoření $W^{1, 2}$ do $L^p$ nebo $C^{0, α}$ je obraz prekompakt. Zároveň protože žijeme na ®R, tak jsou všechny operátory samoadjungované. Tedy podle spektrálního rozkladu samoadjungovaného kompaktního operátoru existuje ortonormální báze $(ω_i)$ prostoru $W^{1, 2}_0$ (inverze je na) tvořená vlastními vektory vektory operátoru $Δ^{-1}$. ($ω_i$ přísluší vlastní číslo $-\frac{1}{λ_i}$. Nula není vlastní číslo, neboť $Δ^{-1}$ je prostý.) A jelikož pokud $Δ^{-1} ω_i = -\frac{1}{λ_i}ω_i$, tak $Δ ω_i = -λ_i ω_i$, $ω_i$ jsou vlastní vektory $Δ$ (příslušné $-λ_i$).
		%
		%Následně ortogonalizace (a vzápětí ortonormalizace) v $L^2$:
		%$$ \<ω_i, ω_j\> = \int ω_iω_j + \int \nabla ω_i\nabla ω_j = \begin{cases}\<ω_i, ω_j\>_{L^2} + \int (λ_i ω_i) ω_j = (1 + λ_i)\<ω_i, ω_j\>_{L^2},\\ \<ω_i, ω_j\>_{L^2} + \int ω_i (λ_j ω_j) = (1 + λ_j)\<ω_i, ω_j\>_{L^2}.\end{cases} $$
		%Tj. pokud $λ_i ≠ λ_j$, tak $ω_i \perp ω_j$ v $L^2$ (jinak by se nemohly řádky rovnat). $λ_i = λ_j$ vždy jen pro konečný počet (spektrální rozklad říká „nějakou limitu“), takže na nich můžeme klidně v $L^2$ provést ortogonalizaci (třeba Gramovu–Schmidtovu), což nám nerozhází chování v $W^{1,2}_0$ (řekněme, že jsme ortogonalizaci dělali opatrně vzhledem k normě v $W^{1, 2}_0$, a dokonce díky rovnosti výše zůstanou ortogonální i v $W^{1, 2}_0$).
		%
		%Z této rovnosti navíc můžeme získat $\|ω_i\|_2 = \frac{\|ω_i\|_{1, 2}}{\sqrt{1 + λ_i}}$.

		Víme, že existuje ortogonální báze $W^{1, 2}$ složená z vlastních vektorů $Δ$ (tj. slabých řešení $-Δ u = λ_k u$), která je zároveň ortonormální množina v prostoru $L^2$. Navíc nenulová vlastní čísla $\rightarrow ∞$.
	\end{poznamkain}

	\begin{reseni}
		$¦u \in \(W_0^{1, 2} (\Omega)\)^2$ je slabé řešení, když $\forall φ \in W^{1, 2}_0$:
		\begin{alignat*}{1}
			\int (-Δ u_1 - aΔu_2)φ + \int u_1 φ &= \int f_1 φ, \\
			\int (-Δ u_2 - bΔu_1) φ + \int u_2 φ &= \int f_2 φ.
		\end{alignat*}
		Jelikož $ω_i$ je báze $W^{1, 2}_0$ a rovnice jsou lineární a „uzavřené na limity“, stačí nám, když budou splněné pro $(ω_i) \subset W^{1, 2}_0$:
		\begin{alignat*}{3}
			λ_i (u_1)_i + a·λ_i·(u_2)_i + (u_1)_i &=& \overbrace{\int -Δ(u_1 + au_2)ω_i}^{=\int ((u_1)_i + a(u_2)_i)ω_i λ_i ω_i + 0} + \overbrace{\int u_1 ω_i}^{=\int \((u_1)_i ω_i\) ω_i + 0} &= \int f_1 ω_i &=& (f_1)_i, \\
			λ_i (u_2)_i + b·λ_i·(u_1)_i + (u_2)_i &=& \underbrace{\int -Δ(u_2 + bu_1)ω_i}_{=\int ((u_2)_i + b(u_1)_i)ω_i λ_i ω_i + 0} + \underbrace{\int u_2 ω_i}_{=\int \((u_2)_i ω_i\) ω_i + 0} &= \int f_2 ω_i &=& (f_2)_i,
		\end{alignat*}
		kde $u_j = \sum_{i = 1}^∞ (u_j)_i·ω_i$ a $(f_j)_i = \int f_j ω_j$. Tedy $(u_1)_i$ a $(u_2)_i$ odpovídá řešení
		$$ \begin{pmatrix} λ_i + 1 & a·λ_i \\ b·λ_i & λ_i + 1 \end{pmatrix} \begin{pmatrix} (u_1)_i \\ (u_2)_i \end{pmatrix} = \begin{pmatrix} (f_1)_i \\ (f_2)_i \end{pmatrix}. $$
		Tj. řešení existuje právě tehdy, když všechny ($\forall i$) tyto soustavy mají řešení (jelikož $λ_i \rightarrow ∞$, tak konvergence $\sum (f_j)_i ω_i$ nám garantuje konvergenci $\sum_{i=1}^∞ (u_j)_i ω_i$). To je pro libovolné ¦f tehdy, když ($\forall i$)
		$$ \det \begin{pmatrix} λ_i + 1 & a·λ_i \\ b·λ_i & λ_i + 1 \end{pmatrix} = (λ_i + 1)^2 - a·b·λ_i^2 ≠ 0, $$
		tedy například, když $a·b ≤ 0$.
	\end{reseni}

	Under which condition on ¦f, the system has a solution?

	\begin{reseni}
		V případě, že $(λ_i + 1)^2 - a·b·λ_i^2 = 0$ pro některá $i$, pak je druhý řádek matice násobkem prvního, tedy i $(f_2)_i$ musí být stejným násobkem $(f_1)_i$ (pro všechna taková $i$).
	\end{reseni}
\end{priklad}

\end{document}
