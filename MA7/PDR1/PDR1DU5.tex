\documentclass[12pt]{article}					% Začátek dokumentu
\usepackage{../../MFFStyle}					    % Import stylu



\begin{document}

TODO JE TO BLBĚ A JEŠTĚ ČÁST CHYBÍ

\begin{priklad}[5. – Lax-Milgram lemma vs Fredholm alternative II]
	Consider $\Omega \subset ®R^d$ a Lipschitz domain. Let $a, b, c, d \in ®R$. Consider the problem: For given $¦f = (f_1, f_2, f_3) \in \(L^2(\Omega)\)^3$ find $¦u = (u_1, u_2, u_3) \in \(W_0^{1, 2} (\Omega)\)^3$ solving
	\begin{align*}
	-\Delta u_1 + au_3 &= f_1\quad \text{ in }\Omega, \\
	-\Delta u_2 + bu_3 &= f_2\quad \text{ in }\Omega, \\
	-\Delta u_3 + cu_1 + du_2 &= f_3\quad \text{ in }\Omega, \\
	u_1 = u_2 = u_3 &= 0 \quad &\text{ on } \partial \Omega. \\
	\end{align*}
	Under which conditions on $a, b, c, d$ the system has for any ¦f a weak solution?

	\begin{reseni}
		Chtěli bychom použít Lax-Milgram. K tomu potřebujeme, aby na levé straně byl eliptický bilineární operátor. Operátor si definujeme přímočaře (tak aby odpovídal představě slabého řešení). Pro $V = (W_0^{1, 2}(\Omega))^3$ definujme $B: V \times V \rightarrow ®R$ jako
		$$ B(¦u, ¦v) = \int_{\Omega} \nabla u_1 · \nabla v_1 + au_3 v_1 + \nabla u_2 · \nabla v_2 + b u_3 v_2 + \nabla u_3 · \nabla v_3 + c u_1v_3 + du_2v_3. $$
		Zřejmě je lineární. Také je „$V$-bounded“, neboť ($\|·\|_2 = \|·\|_{L^2(\Omega)}$, $\|·\|_{1, 2} = \|·\|_{\(W^{1, 2}(\Omega)\)^3} = \|·\|_V$)
		$$ B(¦u, ¦v) = \int_{\Omega} \nabla u_1 · \nabla v_1 + au_3 v_1 + \nabla u_2 · \nabla v_2 + b u_3 v_2 + \nabla u_3 · \nabla v_3 + c u_1v_3 + du_2v_3 ≤ $$
		$$ ≤ \sum_{ij} \left|\int_\Omega \partial_j u_i \partial_j v_i\right| + |a|·\left|\int_\Omega u_3v_1\right|+ |b|·\left|\int_\Omega u_3v_2\right|+ |c|·\left|\int_\Omega u_1v_3\right| + |d|·\left|\int_\Omega u_2v_3\right| \overset{\text{Hölder}}≤ $$
		$$ ≤ \sum_{ij} \|\partial_j u_i\|_2 \|\partial_j v_i\|_2 + |a|·\|u_3\|_2\|v_1\|_2 + |b|·\|u_3\|_2\|v_2\| + |c|·\|u_1\|_2\|v_3\|_2 + |d|·\|u_2\|_2\|v_3\|_2 ≤ $$
		$$ = \max(1, |a|, |b|, |c|, |d|) \(\sum_i \(\sum_j\|\partial_j u_i\|_2 + \|u_i\|_2\)\)·\(\sum_i \(\sum_j\|\partial_j v_i\|_2 + \|v_i\|_2\)\) \overset{*}{≤} $$
		$$ ≤ (3d + 1)^2·\max(1, |a|, |b|, |c|, |d|)\|¦v\|_{1, 2}·\|¦u\|_{1, 2}. $$
		$$ *: \qquad \sum_i \alpha_i ≤ n·\max_i \alpha_i = n·\sqrt{\(\max_i \alpha_i\)^2} ≤ n·\sqrt{\sum_i \alpha_i^2}, \qquad i \in [n] = \{1, …, n\}, \alpha_i ≥ 0. $$

		Aby byl „$V$-coercive“, potřebujeme, aby $B(¦u, ¦u) ≥ c_1\|u\|_V^2$:
		$$ B(¦u, ¦u) = \int_{\Omega} \nabla u_1 · \nabla u_1 + au_3 u_1 + \nabla u_2 · \nabla u_2 + b u_3 u_2 + \nabla u_3 · \nabla u_3 + c u_1u_3 + d u_2u_3 = $$
		$$ = \int_{\Omega} \sum_{ij} (\partial_j u_i)^2  + (a + c) u_3 u_1 + (b + d) u_3 u_2. $$
		Kdyby $(a + c)$ nebylo 0, pak můžeme zvolit např. $±u_1 = u_3 = e^{\sqrt{\frac{a + c}{2}}t}$, $u_2=0$ (znamínko podle toho, zda je $(a + c)$ záporné nebo kladné), pak $B(¦u, ¦u) = 0 \not≥ c_1 \|u\|_V^2$. Obdobně pro $(b + d)$. Tedy dostáváme podmínky $a + c = 0$, $b + d = 0$. Jakmile ale máme tuto podmínku, tak
		$$ B(¦u, ¦u) = \int_\Omega \sum_{ij} (\partial_j u_i)^2 = \sum_{ij} \int_\Omega (\partial_j u_i)^2 = \|\nabla u\|_{(L^2(\Omega))^{3d}}^2 $$
		Z Poincarého nerovnosti máme $\|u\|_2 ≤ C·\|\nabla u\|_2$, tj. $\frac{1}{C^2 + 1}\(\|u\|^2 + \|\nabla u\|^2\) ≤ \|\nabla u\|_{1, 2}$, tedy $B(¦u, ¦u) ≥ c_1\|u\|_V$.

		Tedy pokud $a + c = 0 = b + d$, tak je operátor $B$ eliptický bilineární operátor a tedy podle Lax-Milgram existuje pro každé $¦f \in \(L^2{\Omega}\)^3$ právě jedno slabé řešení problému ze zadání.
	\end{reseni}

	%Under which condition on ¦f, the system has a solution?
\end{priklad}

\end{document}
