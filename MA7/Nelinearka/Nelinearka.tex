\documentclass[12pt]{article}					% Začátek dokumentu
\usepackage{../../MFFStyle}					    % Import stylu



\begin{document}

% 06. 10. 2022

\begin{poznamka}
	There will be homework. We will discus it on practicals (particular solutions are good).
\end{poznamka}

\begin{poznamka}[What it is about]
	Functional analysis generalizes Linear Algebra. This lecture generalizes (real) Analysis in $®R^n$ ($Df(x_0): ®R^n \rightarrow ®R^m$ is linear) by replacing $®R^n$ with Banach spaces.
\end{poznamka}

\begin{priklad}[Calculus of variations]
	Know things: $f: ®R \rightarrow ®R$, differentiable has minimizer at $x_0 \in ®R$ $\implies$ $f'(x_0) = 0$ (in $®R^n$: $Df(x_0) = 0$). Generalize it:

	\begin{reseni}
		Trick: For example $F: u \mapsto \int_Ω \frac{1}{2} |\nabla u|^2 - f u dx$, $W^{1, 2}_g(Ω) \rightarrow ®R$ ($g$ means bounded values). For any $φ \in W^{1, 2}_0(Ω)$ consider $ε \mapsto F(u + εφ)$, $®R \rightarrow ®R$.
		$$ 0 = \frac{d}{dε}|_{ε = 0} F(u + εφ) = \frac{d}{dε}|_{ε = 0} \int_Ω \frac{1}{2} |\nabla u + ε\nabla φ|^2 - f·(u + εφ) dx = $$
		$$ = \frac{d}{dε}|_{ε = 0} \[\int_Ω \frac{1}{2} |\nabla u|^2 - f u dx + ε\int_Ω \nabla u \nabla φ - fφ dx + ε^2 \int_Ω \frac{1}{2} |\nabla φ|^2 dx\] = $$
		$$ = \int_Ω \nabla u \nabla φ - fφ. $$
		Assume $u \in W^{2, 2}(Ω)$:
		$$ \overset{\text{P.I.}} \int_{\partial Ω} \frac{\partial u}{\partial n} φ dx - \int_Ω (\triangle u + f)φ dx \qquad \forall φ \in W^{1, 2}_0(Ω). $$
		$$ \overset{\text{Fundamental lemma}} \triangle u + f = 0. $$
	\end{reseni}
\end{priklad}

\begin{priklad}[Mapping degree]
	Consider $f \in ©C([-1, 1]; ®R)$. How many zeroes does $f$ have? Let assume $f(-1) < 0 < f(1)$. Let assume $f \in ®C^1$. And 0 is a regular value ($f(x_0) = 0 \implies f'(x_0) ≠ 0$).

	\begin{reseni}
		From 0 to $∞$. After assumption: by intermediate value theorem at least 1. After second assumption: odd and finitely many. Moreover, the number of zeros with positive derivative minus the number of zeros with the negative one is 1, which is called degree of $f$.

		Observation: In one dimension $\deg(f) \in \{-1, 0, 1\}$. $\deg(f)$ is invariant under perturbations. $\deg f$ depends on boundary values. Can be extended from $©C^1$ to $©C$ (we take smooth perturbation).

		Ad second observation: homotopy: $h: [0, 1] \times [-1, 1] \rightarrow ®R$, $(s, x) \mapsto h_s(x)$ continuous $h_0 = f$, $h_1 = g$.. And it is admissible if $h_s(-1) ≠ 0$ and $h_s(1) ≠ 0$ for all $s$.
	\end{reseni}

	There is generalization to $®R^n$, to Manifolds, and to Banach spaces. And we get „corollaries“: Fix point theorems, topological statements,inability to comb a hedgehog, 
\end{priklad}

\section{Derivatives in Banach spaces}
\subsection{The notion of a derivative}
\begin{poznamka}[In $®R^n$]
	Partial derivative, directional derivative, total derivative.
\end{poznamka}

\begin{definice}[Directional and Gateaux derivative]
	Let $X, Y$ be Banach spaces, $A \subset X$ open, $f: A \rightarrow Y$. For any $x_0 \in A$, $v \in X$ if
	$$ \frac{\partial f}{\partial v}(x_0) := \lim_{h \rightarrow 0} \frac{f(x_0 + hv) - f(x_0)}{h} $$
	exists, we call it directional derivative (at $x_0$, in direction $v$).

	If $v \mapsto \frac{\partial f}{\partial v}(x_0)$ is a continuous linear operator from $X$ to $Y$, we denote it by $\partial f(x_0)$ and call it the Gateaux derivative (at $x_0$).
\end{definice}

\begin{poznamka}[Notation]
	Some authors omit continuous and linear, i.e. for them directional $\Leftrightarrow$ Gateaux.

	Some use $df$ or $Df$ instead of $\partial f$.

	We will write $\frac{\partial f}{\partial v}(x_0) = \partial f(x_0)\<v\>$. ($\<·\>$ for linear arguments.)
\end{poznamka}

\begin{priklady}
	Consider $F: L^2([0, 1]) \rightarrow L^2([0, 1])$, $u \mapsto F(u)$, $F(u)(x) := \sin(u(x))$. It is continuous ($\|F(u) - F(v)\|_{L^2}^2 = \int|\sin(u(x)) - \sin(v(x))|^2 ≤ \int |u(x) - v(x)|^2$). Fix $φ \in L^2([0, 1])$ and calculate:
	$$ \frac{\partial F}{\partial φ}(u) = \lim_{h \rightarrow 0} \frac{\sin(u(·) + h φ(·)) - \sin(u(·))}{h} = \cos(u(·))·φ(·) $$
	point-wise almost everywhere and by domain convergence everywhere.

	$\frac{\partial F}{\partial φ}$ is linear in $φ$ and bounded $\implies$ $F$ is Gateaux differentiable. Consider $u \mapsto \frac{\partial F}{\partial φ}(u)$ for fixed $φ$. It is continuous.

	Is $\partial F$ a good linear approximation? I.e. $\|F(u + φ) - F(u) - \partial F(u)\<φ\>\|_{L^2} \overset? = o(\|φ\|_{L^2})$. No: Pick $u = 0$ $φ_k = πχ_[0, \frac{1}{k}]$, then $\|φ_k\|_2 = \sqrt{\frac{1}{k} π^2} \rightarrow 0$.
	$$ F(u + φ_k)(x) = \begin{cases}\sin(0), & x > \frac{1}{k},\\ \sin(π), & x ≤ \frac{1}{k}.\end{cases} = 0. $$
	$$ \|…\| = \|0 - 0 - \partial F(0)\<φ_k\>\|_{L^2} = \|φ_k\|_{L^2} \notin o(\|φ_k\|_{L^2}). $$
\end{priklady}

\begin{definice}[Fréchet derivative]
	Let $X, Y$ be Banach, $A \subset X$ open $f: A \rightarrow Y$. For any $x_0 \in A$ if there exists $Df(x_0) \in ©L(X, Y)$ such that
	$$ \lim_{v \rightarrow ¦o} \frac{\|f(x_0 + v) - f(x_0)\|_Y}{\|v\|_X} = 0 $$
	then $Df(x_0)$  is called Fréchet derivative.
\end{definice}

\begin{lemma}[Fréchet $\implies$ Gateaux]
	$X$, $Y$ Banach spaces, $A \subset X$ open, $f: A \rightarrow Y$. If $F$ is Fréchet differentiable at $x_0$, it is also Gateaux differentiable with $\partial f(x_0) = D f(x_0)$.

	\begin{dukazin}
		Trivial.
	\end{dukazin}
\end{lemma}

\begin{definice}[Gradient]
	Let $H$ be a Hilbert space, $A \subset H$ open $f: A \rightarrow ®R$. If $f$ is Gateaux differentiable at $x_0 \in A$, then the unique $\nabla  f(x_0) \in H$ such that $\<\nabla f(x_0), v\>_H = \partial f(x_0) \<v\>\quad\forall v \in H$ is called the gradient of $f$ at $x_0$.
\end{definice}

\begin{poznamka}[Gradients in different spaces]
	Derivatives are „independent“ of the space used: $X_1 \hookrightarrow X_2$, $Y_1 \hookrightarrow Y_2$ Banach, $f_1: X_1 \rightarrow Y_1$, $f_2: X_2 \rightarrow Y_2$ such that $f_2|_{X_1} = f_1$. Then $Df_2(x_0)|_{X_1} = Df_1(x_0)$, if both exist.

	For Hilbert spaces $H_1 \hookrightarrow H_2$:
	$$ \<a, v\>_{H_1} = \<b, v\>_{H_2} \forall v \in H_1 \nRightarrow a = b. $$
	$\implies$ $\nabla f$ depends on the space! Notation $\nabla_H f(x_0)$.

	One can define „formal gradients“: Let $X$ Banach, $H$ Hilbert, $X \hookrightarrow H$. $f: A \subset X \rightarrow ®R$ Gateaux differentiable. Then there might be $\nabla f(x_0) \in H$ such that
	$$ \<v, \nabla f(x_0)\>_H = Df(x_0)(v) \qquad \forall v \in X. $$
	If $X$ is dense in $H$, then $\nabla f(x_0)$ is unique.

	Classically gradients are associate inner product, but principle works with dual pairings, ($\<·, ·\>_{L^p \times L^q}$, $\frac{1}{p} + \frac{1}{q} = 1$).
\end{poznamka}

\subsection{Calculation rules}
\begin{tvrzeni}[Chain rule]
	Let $X, Y, Z$ be Banach, $A \subset X$, $B \subset Y$ open, $f: B \rightarrow Z$, $g: A \rightarrow B$, $x_0 \in A$, $y_0 := g(x_0)$.

	\begin{enumerate}
		\item If $f$ is Fréchet differentiable at $y_0$ and $g$ is Gateaux differentiable at $x_0$, then $f ∘ g$ is Gateaux differentiable at $x_0$ with
			$$ \partial(f ∘ g)(x_0)\<v\> = Df(x_0)\<\partial g(x_0)\<v\>\> \qquad \forall v \in X. $$
		\item If $g$ is additionally Fréchet differentiable, then so is $f ∘ g$.
	\end{enumerate}

	\begin{dukazin}[1.]
		$$ \lim_{h \rightarrow 0} \left\|\frac{f(g(x_0 + hv)) - f(g(x_0))}{h} - Df(y_0)\<\partial g(x_0)\<v\>\>\right\|_Z ≤ $$
		$$ ≤ \lim_{h \rightarrow 0} \left\|\frac{f(g(x_0 + hv) + y_0 - g(x_0)) - f(y_0) - Df(y_0)\<g(x_0 + hv) - g(x_0)\>}{h}\right\|_Z + $$
		$$ + \lim_{h \rightarrow 0} \underbrace{\left\|D f(y_0) \<\partial g(x_0)\<v\> - \frac{g(x_0 + hv) - g(x_0)}{h}\>\right\|_Z}_{\rightarrow 0} = $$
		$$ = \lim_{h \rightarrow 0} \frac{\|f(x_0 + g(x_0 + hv) - g(x_0)) - f(y_0) - Df(x_0)\<g(x_0 + hv) - g(x_0)\>\|_Z}{\|g(x_0 + hv) - g(x_0)\|_Y}·\frac{\|g(x_0 + hv) - g(x_0)\|_Y}{h} = 0·\|\partial g(x_0)\<v\>\|. $$
	\end{dukazin}

	\begin{dukazin}[2.]
		Last convergence in 1. is independent of $v$.
	\end{dukazin}
\end{tvrzeni}

\begin{lemma}[Mean value]
	Let $I \subset ®R$ be an interval, $Y$ Banach, $f: I \rightarrow Y$ differentiable, $a \in Y$. Then $\forall x, y \in I$, $x > y$, $\exists ξ \in [y, x]$ such that
	$$ \left\|\frac{f(x) - f(y)}{x - y} - a\right\|_Y ≤ \|f'(ξ) - a\|_Y. $$

	\begin{dukazin}
		By Hahn–Banach $\exists φ \in Y^*$ such that
		$$ * := \left\|\frac{f(x) - f(y)}{x - y} - a\right\|_Y = φ\<\frac{f(x) - f(y)}{x - y} - a\> \land \|φ\|_{Y^*} = 1. $$
		Define $Ψ: [y, x] \rightarrow ®R$, $s \mapsto φ\<f(s) - s·a\>$. Then
		$$ * = \frac{φ\<f(x)\> - φ\<f(y)\>}{x - y} - \frac{x - y}{x - y}φ\<a\> = \frac{ψ(x) - ψ(y)}{x - y} \overset{\text{Mean value theorem}} ψ'(ξ) \overset{\text{Chain rule}} = φ\<f'(ξ) - a\> ≤ \|f'(ξ) - a\|_Y. $$
	\end{dukazin}
\end{lemma}

\begin{tvrzeni}[Product spaces]
	Let $X_1$, $X_2$, $Y$ be Banach, $f: X_1 \times X_2 \rightarrow Y$. Let $x_1 \in X_1$, $x_2 \in X_2$, and denote by $\partial_1 f(x_1, x_2)$ the Gateaux derivative of $x \mapsto f(x, x_2)$ at $x_1$, by $\partial_2 f(x_1, x_2)$ the Gateaux derivative of $x \mapsto f(x_1, x)$ and similarly $D_1f(x_1, x_2)$ and $D_2f(x_1, x_2)$.

	\begin{enumerate}
		\item If $f$ is Gateaux differentiable at $(x_1, x_2)$ then $\partial_1 f(x_1, x_2)$, $\partial_2 f(x_1, x_2)$ exists and we have
			$$ \forall v_1 \in X_1, v_2 \in X_2: \partial f(x_1, x_2)\<(v_1, v_2)\> = \partial_1 f(x_1, x_2)\<v_1\> + \partial_2 f(x_1, x_2)\<v_2\>. $$
		\item If $\partial_1 f$ and $\partial_2 f$ exists at $(x_1, x_2)$ and one of them is continuous (as a function $X_1 \times X_2 \mapsto ©L(X_i; Y)$) then $f$ is Gateaux differentiable.
		\item The previous points hold also for Fréchet derivation.
	\end{enumerate}

	\begin{dukazin}[1.]
		From definition:
		$$ \partial_1 f(x_1, x_2) = \partial f(x_1, x_2)\<(v_1, 0)\> = \lim_{h \rightarrow 0} \frac{f(x_1 + h v_1, x_2) - f(x_1, x_2)}{h}. $$
	\end{dukazin}

	\begin{dukazin}[2.]
		WLOG $\partial_2 f$ is continuous.
		$$ \lim_{h \rightarrow 0} \left\| \frac{f(x_1 + hv_1, x_2 + hv_2) - f(x_1, x_2)}{h} - \partial_1 f(x_1, x_2)\<v_1\> - \partial_2 f(x_1, x_2)\<v_2\> \right\|_Y ≤ $$
		$$ ≤ \lim_{h \rightarrow 0} \underbrace{\left\|\frac{f(x_1 + hv_1, x_2) - f(x_1, x_2)}{h} - \partial_1 f(x_1, x_2) \<v_1\>\right\|_Y}_{\rightarrow 0} + $$
		$$ + \lim_{h \rightarrow 0} \underbrace{\left\|\frac{f(x_1 + hv_1, x_2 + hv_2) - f(x_1 + hv_1, x_2)}{h} - \partial_2 f(x_1 + hv_1, x_2) \<v_2\>\right\|_Y}_{*} + $$
		$$ + \underbrace{\lim_{h \rightarrow 0} \|\partial_2 f(x_1 + hv_1, x_2)\<v_2\> - \partial_2 f(x_1, x_2)\<v_2\>\|_Y}_{\rightarrow 0} = 0 $$

		Consider $ψ: s \mapsto f(x_1 + hv_1, x_2 + sv_2)$.
		$$ * ≤ \sup_{ξ \in [0, h]} \|\partial_2 f(x_1 + hv_1, x_2 + ξv_2)\<v_2\> - \partial_2 f(x_1 + hv_1, x_2)\<v_2\>\| \rightarrow 0 $$
		by continuous of $\partial_2 f$.
	\end{dukazin}

	\begin{dukazin}[3.]
		Similarly.
	\end{dukazin}
\end{tvrzeni}

% 11. 10. 2023

\subsection{Inverse and implicit function theorem}
\begin{veta}[Inverse function theorem]
	Let $X$, $Y$, $A \subset X$ open, $f: A \rightarrow Y$ continuously Fréchet differentiable. If $x_0 \in A$ such that $Df(x_0): X \rightarrow Y$ is an isomorphism then there exists $U \subset A$, $V \subset Y$ such that $f|_U: U \rightarrow V$ is bijection and $(f|_U)^{-1}$ is Fréchet differentiable with
	$$ D(f^{-1})(y_0) = \(D f(x_0)\)^{-1}, \qquad y_0 := f(x_0). $$

	\begin{dukazin}
		Given $\hat{y}$ close to $f(x_0)$ find $\hat{x}$ such that $f(\hat{x}) = \hat{y}$. Idea: fix $\hat{y}$ try $x$: error in $y$ is $f(x) - y$ and error in x is $\(D f(x_0)\)^{-1}\<f(x) - y\>$. Therefore try iteration:
		$$ F_{\hat{y}}(x) := x - (Df(x_0))<f(x) - y>. $$
		If $F_{\hat{y}}$ has fix point $\hat{x}$ then $\hat{x} = F_{\hat{y}}(\hat{x}) = \hat{x} - (D f(x_0))\<f(\hat{x} - y)\>$ $\implies$ $f(\hat{x}) = \hat{y}$. So we use Banach fixed point theorem: „$F_{\hat{y}}$ is contraction“: ($x_1, x_2 \in B_δ(x_0)$)
		$$ \|F_{\hat{y}}(x_1) - F_{\hat{y}}(x_2)\|_X = \|x_1 - x_2 - (Df(x_0))^{-1}\<f(x_1) - f(x_2)\>\|_X = $$
		$$ = \|(Df(x_0))^{-1}\<Df(x_0)\<x_1, x_2\> + f(x_1) - f(x_2)\>\|_X ≤ $$
		$$ ≤ \|(Df(x_0))^{-1}\|_{©L(Y, X)}·\|Df(x_0) \<x_1 - x_2\> + f(x_1) - f(x_2)\|_Y = * $$
		Consider $a := Df(x_0)\<x_1 - x_2\>$. $ψ: [0, 1] \rightarrow Y$, $f(1 - ξ)x_1 + ξx_2)$ and apply Mennroltz? lemma.
		$$ * ≤ \|(D f(x_0))^{-1}\|_{©L(Y, X)}·\|Df(x_0)<x_1 - x_2> - Df((1 - ξ)x_1 + ξx_2)\<x_2 - x_1\>\|_Y ≤ $$
		$$ ≤ \|(D f(x_0))^{-1}\|_{©L(Y, X)}·\sup_{x \in B_0(x_0)}\|Df(x_0) - Df(x)\|_{©L(X, Y)}·\|x_1 - x_2\|_X \ll 1 $$

		$$ \|F_{\hat{y}}(x) - x_0\|_X = \|F_{\hat{y}}(x) - F_{\hat{y}}(x_j)\|_X + \|F_{\hat{y}}(x_0) - x_0\|_X ≤ $$
		$$ ≤ \frac{1}{2}\|x - x_0\|_X + \|(D f(x_0))^{-1}\|·\|\hat{y} - x_0\| $$
		$\|\hat{y} - x_0\|$ can chosen to be small $\implies$ $F_{\hat{y}}$ maps $\overline{B_δ(x_0)}$ to $\overline{B_δ(x_0)}$ $\implies$ $F_{\hat{y}}$ has unique fix point.

		Next „regularity“: ($y_1 := f(x_1)$, $y_2 := f(x_2)$)
		$$ \|f^{-1}(y_1) - f^{-1}(y_2)\|_X = \|F_{y_1}(x_1) - F_{y_2}(x_2)\|_X ≤ $$
		$$ ≤ \|F_{y_1}(x_1) - F_{y_1}(x_2)\|_X + \|F_{y_1}(x_2) - F_{y_2}(x_2)\|_X ≤ $$
		$$ ≤ \frac{1}{2} \|x_1 - x_2\|_X + \|(Df(x_0))^{-1}\<y_1 - y_2\>\|_X ≤ \frac{1}{2}\underbrace{\|x_1 + x_2\|_X}_{=\|f^{-1}(y_1) - f^{-1}(y_2)\|} + c·TODO!!! $$
		$$ \implies \frac{1}{2}\|f^{-1}(x_1) - f^{-1}(x_2)\|_X ≤ c·\|y_1 - y_2\|_Y \implies f^{-1} \text{ is Lipschitz}. $$

		Pick $δ$ so small that
		$$ \|Df(x) - Df(x_0)\| ≤ \frac{1}{2}·\frac{1}{\|(Df(x_0))^{-1}\|}\qquad \forall x \in B_δ(x_0). $$
		$\implies$ $(Df(x))^{-1}$ exists and is uniformly bounded (from functional analysis).

		$$ \|\underbrace{f^{-1}(y + w) - f^{-1}(y)}_{=: v} - (Df(x))^{-1}\<w\>\| $$
		($f(x + v) + f(x) = f(f^{-1}(y + w)) - y = w$)
		$$ \|v - (Df(x))\<f(x + v) - f(x)\>\| = \|(Df(x))^{-1}\<Df(x)\<v\> - f(x + v) + f(x)\> ≤ \|(Df(x))^{-1}\|·ς(\|v\|) ≤ ς(\|w\|) $$
		because $f^{-1}$ is Lipschitz.

		„Continuity of $Df^{-1}$“ follows from continuity of $f^{-1}$, $Df(·)$ and $(·)^{-1}$.
	\end{dukazin}
\end{veta}

\begin{veta}[Global inverse function theorem]
	Let $X, Y$ Banach, $f: X \rightarrow Y$ continuously Fréchet differentiable and $(D f(x))^{-1}$ exists, depends continuously on $X$ and $c > 0$ such that $\|(Df(x))^{-1}\| < c$ $\forall x \in X$. Then $f: X \rightarrow Y$ is a diffeomorphism.

	\begin{dukazin}
		Last theorem $\implies$ $f$ is a local diffeomorphism. Left to show: $f$ is bijective. „Surjectivity“: Fix $x_0 \in X$, $y_0 \in Y$, $ $. Let $y \in Y$, $φ(t) = y_0 + t(y - y_0)$, $t \in [0, 1]$. Goal: find $ψ(t)$ continuous, such that $φ(t) = f(ψ(t))$ (then $y = f(φ(t))$) (so called lifting). Local diffeomorphism implies $ψ$ exists on $[0, δ]$, in fact if $Y$ is defined on $[0, t_0]$, it can be extended to $[0, t_0 + δ]$. Similarly, if $ψ$ is defined on $[0, t_0]$, per chain rule:
		$$ \|ψ'(t)\| = \|Df^{-1}(φ(t))\<φ'(t)\>\| < c. $$
		$ψ$ is Lipschitz, $\lim_{t \nearrow t_0} ψ(t)$ is well defined and $ψ$ can be extended to $[0, t_0]$. From Zorn lemma $Ψ$ is defined on $[0, 1]$.

		„Injectivity“: Assume $f(x_1) = f(x_2) = y$. Pick $ψ_1(t) := x_1 + t(x_2 - x_1)$. $φ_1(t) = f(ψ_1(t))$. Define $φ_s(t) = sφ_1(t) + (1 - s)y$ ($t, s \in [0, 1]$). Similar to before (homework) $\exists ψ_s(t)$ continuous in $s$ and $t$, such that $f(ψ_s(t)) = φ_s(t)$. But then
		$$ x_1 = ψ_1(0) = ψ_s(0) = ψ_0(0) = ψ_0(t) = ψ_0(1) = ψ_s(1) = ψ_1(1) = x_2. $$
	\end{dukazin}
\end{veta}

\begin{veta}[Implicit function theorem]
	Let $X_1, X_2, Y$ Banach, $A_1 \subset X_1$, $A_2 \subset X_2$ open, $f: A_1 \times A_2 \rightarrow Y$ continuously Fréchet differentiable and exists $\hat{x}_1 \in A_1$ and $\hat{x}_2 A_2$ with $f(x_1, x_2) = 0$. If $D_2f(\hat{x}_1, \hat{x}_2)$ is an isomorphism (between $X_2$ and $Y$), then are neighbourhoods $U_1, U_2$ of $x_1, x_2$ such that $\forall \hat{x}_1 \in U_1\ \exists! \hat{x}_2 \in U_2$ with $f(\hat{x}_1, \hat{x}_2) = 0$.

	If we call $\hat{x}_2 = g(x_1)$, then $g$ is continuously Fréchet differentiable with $Dg(x) = - (D_2f(x, g(x)))^{-1} ∘ D_1f(x, g(x))$.

	\begin{dukazin}
		Apply the inverse function theorem to
		$$ F(x_1, x_2) := (x_1, (D(f(\hat{x}_1, \hat{x}_2)))^{-1})\<f(x_1, x_2)\> $$
	\end{dukazin}
\end{veta}

\end{document}
