\documentclass[12pt]{article}					% Začátek dokumentu
\usepackage{../../MFFStyle}					    % Import stylu



\begin{document}

% 04. 10. 2022

\section{Preliminaries}
\begin{definice}[Slabá derivace]
	Nechť $f \in L^1_{loc}(®R^n)$. Říkáme, že $g \in L^1_{loc}(®R^n)$ je slabou derivací $f$ podle $i$-té proměnné, pokud platí
	$$ \int_{®R^n} f \partial_i φ dλ^n = - \int_{®R^n} g φ dλ^n \qquad φ \in ©D(®R^n) = ®C^∞_0(®R^n). $$
\end{definice}

\begin{definice}[Značení]
	$$ \partial_i f(x) = \lim_{h \rightarrow ¦o} \frac{f(x + e_i h) - f(x)}{h}, \qquad \nabla f(x) = \begin{pmatrix} \partial_1f(x) \\ \vdots\\ \partial_n f(x) \end{pmatrix}, $$
	$$ D_if \text{ slabá derivace dle $i$-té proměnné}, \qquad \nabla f(x) = \begin{pmatrix} \partial_1f(x) \\ \vdots\\ Dn f(x) \end{pmatrix}, $$
	$D·$ bude také značit derivaci distribuce? (Distribuční derivaci?)

	$f \in Lip(X, Y)$ jsou všechny Lipschitzovská zobrazení (tj. $ρ_Y(f(a), f(b)) ≤ lip(f)·ρ_X(a, b)$) z $X$ do $Y$.

	$A \triangle B := (A \setminus B) \cup (B \setminus A)$ (symetrický rozdíl množin).
\end{definice}

\begin{definice}[Lebesgueova–Stieltjesova míra]
	$μ$ míra vytvořená $M: I(®R) \rightarrow [0, ∞)$ pomocí Caratheodorovy konstrukce se nazývá Lebesgueova–Stieltjesova míra.
\end{definice}

\begin{definice}[Radonova míra]
	$©M^+_{loc}(Ω)$ je prostorem všech Borelovských měr na $Ω \subset ®R^n$, které jsou vnitřně regulární ($μ(E) = \sup\{\mu(K) | K \subset E\}$), lokálně kompaktní.

	Pokud navíc $|μ| < ∞$, pak je to prostor $©M^+$. $©M_{loc}(Ω) = μ^+ - μ^-$.
\end{definice}

\begin{definice}[?]
	$$ ψ(x) = \begin{cases}e^{-\frac{1}{1 - |x|^2}}, & |x| < 1,\\ 0, & |x| ≥ 1.\end{cases} $$

	$$ ψ_k(x) = k^n ψ(kx) $$

	\begin{poznamkain}
		$$\int |Ψ| = 1, \qquad ψ(x) = ψ(x'), |x| = |x'|, \qquad ψ \in ©D(®R^n) $$
		$$ ψ_k(x) > 0 \implies |x| < \frac{1}{k} $$
	\end{poznamkain}
\end{definice}

\begin{veta}[Lebesgueova o derivaci 1]
	Nechť $1 ≤ p < ∞$. Nechť $Ω \subset ®R^n$ otevřená. A nechť $f \in L^p_{loc}(Ω)$. Potom pro skoro všechna $x \in Ω$:
	$$ \lim_{r \rightarrow 0} \fint_{B(x, r)} |f(y) - f(x)|^p dλ^n(x) = 0. $$
	\vspace{-1em}

	\begin{dukazin}
		Bez důkazu.
	\end{dukazin}
\end{veta}

\begin{definice}[Lebesgueův bod]
	Každý takový bod se nazývá ($p$) Lebesgueův bod.
\end{definice}

\begin{definice}[Konvoluce]
	$$ f*g(x) = \int_{®R^n} f(x - y)g(y) dy. $$
	Za podmínek, kdy pravá strana existuje. $g$ může být i míra.

	\begin{poznamka}
		Je-li $f * g \in L^1$ pak $f*g=g*f$. (Z Fubiniovy věty.)
	\end{poznamka}
\end{definice}

\begin{tvrzeni}
	Nechť $u \in L^1_{loc}(®R^n)$. Pak $ψ_k*u(x)$ je definováno $\forall x \in ®R^n$ a $\forall k \in ®N$.

	\begin{dukazin}
		Nebyl. Viz Funkcionalka.
	\end{dukazin}
\end{tvrzeni}

\begin{veta}
	$u \in L^1_{loc}(®R^n)$. Pak $ψ_k * u \in ®C^∞(®R^n)$ a $\partial_i(ψ_k*u) = \partial_iψ_k*u$.

	\begin{dukazin}
		Nebyl. Viz Funkcionalka.
	\end{dukazin}
\end{veta}

\begin{lemma}
	Nechť $f \in L^p$. Potom $ψ_k * f \in L^p$ $p \in [1, ∞]$. Navíc $\|ψ_k*f\|_p ≤ \|f\|_p$.

	\begin{dukazin}
		Nebyl. Viz Funkcionalka.
	\end{dukazin}
\end{lemma}

\begin{veta}
	$f \in L^1_{loc}(®R^n)$ nechť $x$ je Lebesgueův bod $f$ (a $f(x) = \lim_{r \rightarrow 0} \fint_{B(x, r)} f$) pak $ψ_k * f(x) \overset{k} \rightarrow f(x)$.

	\begin{dukazin}
		Nebyl.
	\end{dukazin}
\end{veta}

\begin{veta}
	Nechť $f \in ©C^∞_0(®R^n)$. Potom $ψ_k * f \rightrightarrows f$ na $®R^n$.

	\begin{dukazin}
		Nebyl.
	\end{dukazin}
\end{veta}

\begin{lemma}
	Pro $p \in [1, ∞)$ platí $\overline{©C^∞_0(®R^n)}^{L^p} = L^p(®R^n)$.

	\begin{dukaz}
		Nebyl. (Docela jednoduchý.)
	\end{dukaz}
\end{lemma}

\begin{veta}
	$1 ≤ p < ∞$: $f \in L^p(®R^n)$ $\implies$ $ψ_k * f \rightarrow f$ v $L^p(®R^n)$.

	\begin{dukazin}
		Nebyl.
	\end{dukazin}

	\begin{poznamkain}
		($ψ_1 * f \overset{w}\rightarrow f$ v $L^∞$)
	\end{poznamkain}
\end{veta}

% \pagebreak

\begin{veta}
	Nechť $u \in Lip(®R^n, ®R)$. Pak $u$ je slabě diferencovatelná na $®R^n$ a $\|Du\|_{L^∞} ≤ lip(u)$.

	\begin{dukazin}
		Nechť $x, z \in ®R^n$.
		$$ |ψ_k * u(z) - ψ_k * u(x)| = \left|\int(u(z - y) - u(x - y))ψ_k(y)dλ^n\right| ≤ lip(u) |z - x|. $$
		$lip(u_k) := lip(ψ_k * u) ≤ lip(u)$.
		Nechť $B$ je koule v $®R^n$. $\{\nabla u_k\}$ je omezená v $L^2(B)$ slabě konverguje k $g \in L^2(B, ®R^n)$.

		$\{f \in L^2(B): \|f\|_∞ ≤ c\}$ konvexní a uzavřená $\implies$ slabě uzavřená $\implies$ $\|g\|_∞ ≤ lip(u)$. Tedy
		$$ \int_B u\nabla φ \leftarrow \int_B u_k \nabla φ = -\int_B \nabla u_k φ \rightarrow -\int_B g φ. $$
	\end{dukazin}
\end{veta}

\begin{lemma}
	Nechť $E \subset Ω$ a pro nějaké $r > 0$: $E + B(¦o, r) \subset Ω$. Potom $\exists η \in ©D(Ω)$, že $η = 1$ na $E$.

	\begin{dukazin}
		$E + B\(0, \frac{r}{2}\) \subset\subset Ω$. Najdeme $k$, že $\frac{1}{k} < \frac{r}{2}$. Potom $ψ_k * χ_{E + B\(0, \frac{r}{2}\)}$ je hledaná funkce.
	\end{dukazin}
\end{lemma}

\section{Absolutně spojité funkce}
\begin{poznamka}[V této kapitole vždy]
	$I = (a_0, b_0)$ je interval. $®D(I)$ bude množina všech konečných dělení ($a_0 < x_0 < … < x_n < b_0$) intervalu.
\end{poznamka}

\begin{definice}[Variace funkce]
	Nechť $D = \{x_0 < x_1 < … < x_m\} \in ®D(I)$ a $u: I \rightarrow ®R$. Potom variace $u$ podle dělení $D$ je $V(u, D) = \sum_{i=1}^n |u(x_i) - u(x_{i-1})|$.

	Variace $u$ je $V(u, I) = \sup_{D \in ®D(I)} V(u, D)$.

	Je-li $V(u, I) < ∞$ pak říkáme, že $u$ má konečnou variaci na $I$.
\end{definice}

\begin{definice}[Absolutně spojité funkce]
	Nechť $u: I \rightarrow ®R$. Říkáme, že $u$ je (klasicky) absolutně spojité na $I$, jestliže ke každému $ε > 0$ existuje $δ > 0$ takové, že pro všechny $\{[a_i, b_i]\}_{i=1}^m$ po dvou disjunktní $\sum_{i=1}^m b_i - a_i < δ$ je $\sum_{i=1}^n |u(b_i) - u(a_i)| < ε$.
\end{definice}

% 11. 10. 2023

\begin{definice}
	Nechť $u: (a_0, b_0) \rightarrow ®R$. Říkáme, že $u \in W^{1, 1}(I)$ $\Leftrightarrow$ $u \in L^1(I)$ a $\exists Du \in L^1(I)$. ($Du = f dλ^1, f \in L^1$.)
\end{definice}

\begin{veta}
	Nechť $T \in ©D^*(I)$ a $\<T, φ'\> = 0$ $\forall φ \in ©D(I)$. Then $\exists c \in ®R$, $T = c(dλ^1)$ (tj. $\<T, φ\> = \int c φ$).

	\begin{dukazin}
		Nechť $η \in ©D(I): \int_I η = 1$. Nechť $φ \in ©D(I)$ a $λ = \<1, φ\> = \int_I φ$. Označme $c := \<T, η\>$. Zadefinujeme $Φ(x) = \int_{a_0}^x φ - λ η$, $Φ(b_0) = 0$, $Φ \in ©C_c^∞(I)$.
		$$ 0 = \<T, Φ'\> = \<T, φ\> - λ\<T, η\> = \<T, φ\> - λc = \<T, φ\> - \int c φ \implies $$
		$$ \implies \<T, φ\> = \int c φ. $$
	\end{dukazin}
\end{veta}

\begin{veta}
	Nechť $f: I = (a_0, b_0) \rightarrow ®R$, $f \in L^1(I)$. Potom
	\begin{enumerate}
		\item $\exists!$(až na aditivní $c$) $u$: $u(b) - u(a) = \int_a^b f$ (pro $a_0 < a < b < b_0$);
		\item $u$ má slabou derivaci a $Du = f$;
		\item $\exists!T \in ©D^*(I)$ (až na aditivní c), že $T' = f$;
		\item $T = udλ^1 + c dλ^1$;
		\item $u$ je absolutně spojitá;
	\end{enumerate}

	\begin{dukazin}
		„1.“ $u(x) = \int_{a_0}^x f(t) dt$.

		„2.“ $\int_I u(x) φ'(x) dx = \int_I φ'(x) \int_{a_0}^x f(t) dt dx \overset{\text{Fubini}}= \int_{a_0}^x f(t) \int_I φ'(x) dx dt = - \int_I φ(t) f(t) dt$. Tedy $Du = f$ na $I$.

		„3.“ a „4.“ jednoduché.

		„5.“: $f \in L^1 \implies \forall ε > 0\ \exists δ > 0\ \forall A \subset I$ měřitelná a $©L^1(A) < δ$: $\int_A |f| < ε$. Nechť $[a_i, b_i]$ po dvou disjunktní, $i \in [n]$, $\sum b_i - a_i < δ \implies \sum |u(b_i) - u(a_i)| ≤ \int_{\bigcup (a_i, b_i)} |f| < ε$.
	\end{dukazin}
\end{veta}

\begin{veta}
	Nechť $u$ je absolutně spojitá na $I = (a_0, b_0)$. Potom
	\begin{enumerate}
		\item $u$ je spojitá a lze ji spojitě dodefinovat na $\overline{I}$;
		\item $V(u, I) < ∞$ a $V(u, (a_0, x])$ je absolutně spojitá;
		\item $u$ je rozdílem 2 neklesajících funkcí;
		\item $\exists! f \in L^1: u(b) - u(a) = \int_a^b f$;
		\item $\exists u'$ skoro všude, $u'(x) = f(x)$ skoro všude;
		\item $Du = u' dλ^1$ na $I$;
		\item $u(b) - u(a) = \int_a^b u'(x)$.
	\end{enumerate}

	\begin{dukazin}
		„1.“ $\forall ε > 0 \exists δ: ε_k = 2^{-k} … δ_k$. $x_k \in (a_0, a_0 + δ_k)$.
		$$ \sum_{k=n}^∞ |u(x_{k + 1}) - u(x_k)| < 2^{-n + 1}. $$
		Značme $u(a_0)$ jakoukoliv limitu $u(x_k)$. Potom $|u(a_0) - u(x)| < 2ε_k$ jakmile $x \in (a_0, a_0 + δ_k)$.

		„2.“ $ε = 1$, $\exists δ > 0$. $λ^1(I) = b_0 - a_0$. Najdeme $N \in ®N$, že $N ≥ \frac{b_0 - a_0}{δ}$. $D$ je dělení $I$. $v(u, D) ≤ N$. $ε > 0$, $\exists δ > 0$: $V(u, (a_0, x]) := g(x)$ mějme konečné intervaly $λ^1(\bigcup [a_i, b_i]) < δ$ $\implies$ $\sum |u(b_i) - u(a_i)| < ε$.

		„3.“: $v(x) = V(u, [a_0, x])$ a $v(x) - u(x)$ jsou hledané funkce (jsou omezené a snadno se dokáže, že jsou neklesající).

		„4.“: (z 3. předpokládejme, že $u$ je neklesající) Caratheodorovou konstrukcí nalezneme míru: $M((a, b)) = u(b) - u(a)$ a ukážeme o ní, že je spojitá (pak je to Lebesgue-Stieltjesova míra, tedy platí $M((a, b)) = \int_a^b f$). Nechť $λ^1(N) = 0$. $\forall δ > 0$ najdu $G \supset N$ $λ^1(G) < δ$, $G$ otevřená, tedy $G = \bigcup_{i=1}^∞(a_i, b_i)$. $μ(G) = \sum_{i=1}^m |u(b_i) - u(a_i)| < ε$.
		$$ \frac{1}{2δ} \int_{x - δ}^{x + δ} f(y) dy \overset{δ \rightarrow 0^+} f(x). $$

	\end{dukazin}
\end{veta}

\break

\begin{veta}
	Nechť $u$ je spojitá na $I$. Pak NÁPOJE:
	\begin{itemize}
		\item $u$ je absolutně spojitá na $I$;
		\item $u \in W^{1, 1}(I)$;
		\item $\exists f \in L^1(I): Du = fdλ^1$;
		\item $Du$ má $L^1$ reprezentanta $u(b) - u(a) = \int_a^b Du$;
		\item $\exists u'$ skoro všude, $u' \in L^1$ a $u(b) - u(a) = \int_a^b u'$;
		\item $\exists f \in L^1: u(b) - u(a) = \int_a^b$;
		\item $\exists g \in L^1: |u(b) - u(a)| ≤ \int_a^b g$.
	\end{itemize}

	\begin{dukazin}
		Máme vše kromě „poslední bod $\implies$ první“: $λ^1(\bigcup(a_i, b_i)) < δ \implies \sum |u(b_i) - u(a_i)| < ε$.
	\end{dukazin}
\end{veta}

% 25. 10. 2023

\begin{definice}
	Nechť $Ω \subseteq ®R^n$ otevřená:
	$$ W^{1, 1}_{loc}(Ω) := \{u \in L^1_{loc}(Ω) | \forall i \in [n]\ \exists D_i u \in L^1_{loc}(Ω)\}, $$
	$$ W^{1, p}(Ω) = \{u \in W^{1, 1}_{loc} \middle| \|u\|_{1, p} < ∞\}, \text{ kde } \|u\|_{1, p} = \sqrt[p]{\sum_{i=1}^n \int_Ω |D_i u|^p + \int_Ω |u|^p}, $$
	$$ W^{1, p}_c(Ω) = \{u \in W^{1, p}(Ω) \middle| \exists K \subset Ω: \{u ≠ 0\} \subset K\}, $$
	$$ p \in [1, ∞): W^{1, p}_0(Ω) = \overline{W_c^{1, p}(Ω)}^{\|·\|_{1, p}}, \qquad p = ∞: W^{1, p}_0(Ω) = \overline{W^{1, ∞}(Ω) \cap C_0(Ω)}^{\|·\|_{1, p}}. $$
\end{definice}

\begin{veta}
	$(W^{1, p}(Ω), \|·\|_{1, p})$ je Banachův prostor.

	\begin{dukazin}
		Linearita a funkčnost normy je zřejmá. Jak je to s úplností? $u_k$ cauchyovská v $\|·\|_{1, p}$. $W^{1, p} \hookrightarrow L^p$ $\implies$ $u_k$ cauchyovská v $L^p$ $\implies$ $\exists u \in L^p: u_k \rightarrow u$ v $L^p$.
		$$ φ \in ©D(Ω): \int_Ω u \partial_i φ \leftarrow \int_Ω u_k \partial_i φ = -\int_Ω D_i u_k φ \rightarrow -\int g φ. $$
		Poslední konvergence z $\exists g: D_i u_k \rightarrow g$ v $L^p$ a $D_i u = g \in L^p$.
	\end{dukazin}
\end{veta}

\begin{veta}[Rieszova pro $W^{1, p}$]
	Nechť $Ω \subset ®R^n$ otevřená. $1 ≤ p < ∞$, $p' = \frac{1}{1 - \frac{1}{p}}$. Pak pro každý $L \in \(W^{1, p}(Ω)\)^*$ existuje $f \in L^{p'}(Ω, ®R^{n+1})$ takové, že $L(u) = \sum_{i=1}^n \int_Ω f^i D_i u + \int_Ω f^{n+1} u$ a navíc $\|L\|_{(W^{1, p})^*} = \|f\|_{L^p}$.

	\begin{dukazin}
		Definujeme $T\!: W^{1, p}(Ω) \rightarrow L^p(Ω, ®R^{n+1})$, $u \mapsto (D_1 u, …, D_n u, u)$. $T W^{1, p}(Ω) \Subset L^p(Ω, ®R^{n+1})$. Díky HB větě každý $L \in \(W^{1, p}(Ω)\)^*$ lze rozšířit na $L^1 \in (L^p(Ω, ®R^{n+1}))^* = L^{p'}(Ω, ®R^{n+1})$.
	\end{dukazin}

	\begin{poznamkain}
		$f, h \in L^{p'}(Ω, ®R^{n+1})$ jsou takové, že
		$$ \sum_i \int f^i D_i u + \int f^{n+1} u = \sum_i \int h^i D_i u + \int h^{n+1} u. $$

		$\Div h = \Div f$ na $Ω$ a $f·ν = h·ν$ na $\partial Ω$. $Δu = f$.
	\end{poznamkain}
\end{veta}

\begin{poznamka}
	$u_k \rightarrow u$ v $W^{1, p}$ $\Leftrightarrow$ $\|u_k - \|_{1, p} \rightarrow 0$.

	Pro $p = ∞$
\end{poznamka}

\begin{veta}
	Nechť 1. $u \in W^{1, p}(®R^n)$, $1 ≤ p < ∞$, nebo 2. $u \in W^{1, ∞}(®R^n) ψ_k * u \overset{*}\rightarrow u$ v $W^{1, ∞}(®R^n)$, potom $\|ψ_k * u - u\|_{1, p} \overset{k \rightarrow ∞}\longrightarrow 0$.

	\begin{dukazin}
		Víme $\int |ψ_k * D_i u - D_i u|^p \overset{k \rightarrow ∞}\longrightarrow 0$. Stačí dokázat $ψ_k * D_i u = D_i(ψ * u) = \partial_i (ψ_k * u)$. Nechť $φ \in ©D(®R^n)$, pak
		$$ - \int_{®R^n} D_i(ψ_k * u)φ = \int_{®R^n} ψ_k * u \partial φ = \int_{®R^n} \int_{®R^n} ψ_k(x - y) u(y) dy \partial_i φ(x) dx = $$
		$$ = \int_{®R^n} \int_{®R^n} ψ_k(y - x) \partial_i φ(x) dx u(y) dy = $$
		$$ = \int_{®R^n} ψ_k * \partial_i φ(y) u(y) dy = \int_{®R^n} \partial(ψ_k * φ) u(y) = $$
		$$ = - \int_{®R^n}(ψ_k * φ)D_i u = \int_{®R^n} (ψ_k * D_i u) φ. $$
		Tedy $ψ_k * D_i u = D_i(ψ_k * u)$.
	\end{dukazin}
\end{veta}

\begin{veta}
	$$ \overline{©D(®R^n)}^{W^{1, p}} = W^{1, p}(®R^n). $$

	\begin{dukazin}
		Definujeme $η(x) = 1$ na $B(0, 1)$, $η(x) = 0$ na $®R^n \setminus B(0, 2)$, $η(x) \in [0, 1]$ in $®R^n$ a $η \in C^∞(®R^n)$.

		$u_k(x) = u(x) η(x / k)$. $u_k(x) = u(x)$ na $B(0, k)$, $u_k(x) = 0$ na $®R^n \setminus B(0, 2k)$.
		$$ \int_{®R^n} |u_k - u|^p ≤ \int_{®R^n \setminus B(0, k)} |2u|^p \overset{k \rightarrow ∞}\longrightarrow 0. $$
		$$ D_i u_k = D_i u(x)η\(\frac{x}{k}\) = (D_i u(x))η\(\frac{x}{k}\) + u(x) D_i η\(\frac{x}{k}\). $$
		$$ \int_{®R^n} |D_i u_k - D_i u|^p ≤ \int_{®R \setminus B(0, k)} \left|2D_i u + u(x) \frac{\|\nabla η\|_∞}{k}\right|^p ≤ $$
		$$ ≤ c·\int_{®R^n \setminus B(0, k)} |D_iu|^p + |u|^p \overset{k \rightarrow ∞}\longrightarrow 0. $$
	\end{dukazin}
\end{veta}

\begin{veta}
	$©D(®R^n)$ jsou slabě $*$ sekvenciálně husté v $W^{1, ∞}(®R^n)$. (Jinak řečeno, pro každé $u \in W^{1, ∞}$ najdeme $φ_k \subset ©D$, $φ_k \overset{*} \rightarrow u$ v $W^{1, ∞}(®R^n)$).

	\begin{dukazin}
		$u \in W^{1, ∞}$, $u_k(x) = u(x) η\(\frac{x}{k}\)$. Zvolme $f \in L^1$. $\int D_i u_k f =$
		$$ = \int_{®R^n} χ_{B(0, 2k) \setminus B(0, k)} \frac{\partial_i η(\frac{x}{k})}{k} u(x) f(x) + \int_{?} η\(\frac{x}{k}\) D_i u f(x) = \int_{B(0, k)} D_i u f(x) \rightarrow \int D_i u f. $$
%		$\int η D_i u f(x) \rightarrow $
	\end{dukazin}
\end{veta}

% 01. 11. 2023

\begin{veta}
	Nechť $1 ≤ p < ∞$, $Ω \subset ®R^n$ otevřená, $u \in W^{-1, p}(Ω)$. Potom $\exists u_k \in C^∞(Ω) \cap W^{1, p}(Ω): u_k \rightarrow u$ v $W^{1, p}(Ω)$.

	\begin{dukazin}
		$Ω_1 \Subset Ω_2 \Subset … \Subset Ω$, $\overline{Ω_k}$ kompaktní, $Ω_k$ Otevřené, $Ω = \bigcup_{j=1}^∞ Ω_j$. Najdeme rozklad jednotky $ω_j$ (tj. $ω_j \in ©D(Ω)$, $ω_j(x) \in [0, 1]$, $ω_j ≥ χ_{Ω_j}$, $\forall x \in Ω_n: \sum_{j=1}^∞ ω_j(x) = 1$).

		Mějme $ω_ju \in W^{1, p}_0(Ω)$ $\exists v_{k, j} \in ©D(Ω)$: $v_{k, j} \rightarrow ω_j u$ v $W^{1, p}$. Takže najdeme $v_j \in ©D(Ω): \int_Ω |v_j - ω_j u|^p + |Dv_j - Dω_j u| < 2^{-j} ε$, $v_j \in W^{1, p}$. $\sum_{j=1}^∞ ω_j u = u \in W^{1, p}$.

		Položme $v = \sum_{j=1}^∞ v_j$. Chceme dokázat $\|v - u\|_{1, p} < ε$. Nejprve na $Ω_n$. Máme $u|_{Ω_n} = \sum_{j=1}^n (ω_j u) |_{Ω_n} v|_{Ω_n}  = \sum_{j=1}^n v_j|_{Ω_n}$.
		$$ \!\!\!\int_Ω \|v - u\|^p + \|Dv - Du\|^p ≤ \int_{Ω_n} \sum_{j=1}^n |v_j - u_j|^p + |Dv_j - Du_j|^2 ≤ \sum_{j=1}^n \|v_j - u_j\|_{W^{1, p}(Ω_n)}^p ≤ \sum_{j=1}^n 2^j ε ≤ ε.\!\! $$
		Pošleme $n \rightarrow ∞$ a zjistíme, že $\|u - v\|_{W^{1, p}(Ω)}^p ≤ ε$.
	\end{dukazin}

	\begin{poznamkain}[Konstrukce rozkladu jednotky]
		Mějme nějaké $η_j: η_j \in ©D(Ω)$ tak, že $0 ≤ η_1 ≤ η_2 ≤ …$ a $η_j = 1$ na $Ω_j$. Označme $ω_1 = η_1$ a $ω_j = η_j - η_{n-1}$ pro $j ≥ 2$. Potom $ω_j$ mají kompaktní nosič a tvoří rozklad jednotky.
	\end{poznamkain}
\end{veta}

\section{Absolutní spojitost po přímkách}
\begin{veta}
	Nechť $Ω \subset ®R^n$ otevřená a $a \in L^1$. Nechť platí následující: Pro každé $i \in [n]$ a pro $λ^n$-skoro všechna $x \in Ω$ je funkce $t \mapsto u(x + te_i)$ lokálně absolutně spojitá a $\partial_i u \in L^1(Ω)$. Pak $u \in W^{1, 1}(Ω)$ a $D_i u = \partial_i u dλ^n$.

	\begin{dukazin}
		$u \in L^1 \implies u$ měřitelná. $φ_x: t \mapsto u(x + te_i)$ absolutně spojitá $\implies$ $\exists φ_x'(t)$ pro skoro všechna $t$ a
		$$ φ_x'(t) = \lim_{y \rightarrow t, y \in Q} \frac{φ_x(y) - φ_x(t)}{y - t} \implies φ'(x, t) = \partial_i u(x + te_i) \text{ je $λ^n$-měřitelná}. $$
		BÚNO $Ω = (a_1, b_1) \times (a_2, b_2) \times … \times (a_n, b_n)$. Označme $\tilde Ω := (a_2, b_2) \times … \times (a_n, b_n)$. Nechť $φ \in ©D(Ω)$ pro $λ^n$-skoro všechna $\tilde x \in \tilde Ω$ máme ($u$ absolutně spojitá v $L^1$ $\implies$ $Du = u'$):
		$$ \int_{\tilde Ω} \int_{a_1}^{b_1} \partial_1 u(t, \tilde x) φ(t, x) dt d\tilde x = \int_{\tilde Ω} \int_{a_1}^{b_1} u(t, \tilde x) \partial_1 φ(t, x) dt d\tilde x. $$
		$$ \(\forall φ \in ©D(Ω): \int_Ω \partial u φ = -\int_Ω u \partial_1 φ\) $$
	\end{dukazin}
\end{veta}

\begin{veta}
	Nechť $Ω \subset ®R^n$ otevřená, $u \in W^{1, 1}(Ω)$. Pak existuje vhodný reprezentant funkce $u$ takový, že platí následující: Pro $λ^n$-skoro všechny $x \in ®R^n$ je funkce $t \mapsto u(x + te_i)$ lokálně absolutně spojitá na $\{t \in ®R | x + te_i \in Ω\}$. Funkce $u$ má všechny parciální derivace v $λ^n$ skoro každém bodě a $D_i u = \partial_i u λ^n$.

	\begin{dukazin}
		Jako v předchozím důkazu můžeme předpokládat, že $Ω$ je interval a označíme $\tilde Ω = (a_2, b_2) \times \cdots \times (a_n, b_n)$. Podle předpředchozí věty existuje posloupnost $\{u_k\}$ $©C^1$-funkcí tak, že $\|u_k - u\|_{1, 1} \rightarrow 0$. Dokonce můžeme předpokládat, že $\|u_k - u_{k+1}\|_{1, 1} < 2^{-k}$.

		Potom řada
		$$ \lim_{k \rightarrow ∞} u_{k + 1} = u_1 + (u_2 - u_1) + (u_3 - u_2) + … = u_1 + \sum_{k=1}^∞ \(u_{k+1} - u_k\) $$
		konverguje skoro všude a její součet je skoro všude původní funkce $u$. Můžeme tedy předpokládat, že funkce $u$ je reprezentována tímto součtem. Máme:
		$$ \!\!\!\int_Ω \sum_{k=1}^∞ \(|u_{k+1} - u_k| + |\nabla u_{k+1} - \nabla u_k|\)dλ^n = \sum_{k=1}^∞ \int_Ω \(|u_{k+1} - u_k| + |\nabla u_{k+1} - \nabla u_k|\) dλ^n < ∞,\!\! $$
		tedy pro $λ^{n-1}$-skoro každý $\tilde x \in \tilde Ω$ je integrál
		$$ \int_{a_1}^{b_1} \sum_{k=1}^∞ \(|u_{k+1}(t, \tilde x) - u_k(t, \tilde x)| + |\nabla u_{k+1}(t, \tilde x) - \nabla u_k(t, \tilde x)|\) dλ^1(t) < ∞. $$

		Pro takový $\tilde x$ je $u|_{(a_1, b_1) \times \{\tilde x\}} \in W^{1, 1}((a_1, b_1))$ a dle věty výše je $u|_{(a_1, b_1) \times \{\tilde x\}}$ absolutně spojitá funkce na $(a_1, b_1)$. Dále je $\partial_1 u_k \rightarrow \partial_1 u$ v $L^1((a_1, b_1))$. Provedením limitního přechodu diferenčních podílů přes racionální čísla, dostaneme, množina bodů, kde parciální derivace existuje je měřitelná a tudíž derivace existuje skoro všude.

		Pro každé $φ \in ©D(Ω)$ je $u|_{(a_1, b_1) \times \{\tilde x\}} \in ©D((a_1, b_1))$ a dle věty výše (bod 6.) a dle Fubiniovy věty je $D_1 u = \partial_1 u \in L^1(Ω)$. Podobně pro ostatní parciální derivace.
	\end{dukazin}
\end{veta}

\begin{poznamka}
	Předchozí dvě věty dávají dohromady tzv. Beppo Leviho charakterizaci Sobolevových prostorů.
\end{poznamka}

\begin{veta}[Svazové vlastnosti Sobolevových prostorů]
	Nechť $u, v \in W^{1, p}(Ω)$. Potom též funkce $w := \max \{u, v\} \in W^{1, p}$ a $\nabla w = \nabla u$ na $\{w = u\}$ a $\nabla w = \nabla v$ na $\{w = v\}$.

	\begin{dukazin}
		Pro případ $n = 1$ si stačí uvědomit, že $Duχ_{\{u ≥ v\}} + Dvχ_{\{v > u\}} \in L^1$ ($|u(b) - u(a)| ≤ \int_a^b f$ a $|v(b) - v(a)| ≤ \int_a^b g$ $\implies$ $|w(b) - w(a)| ≤ \int_a^b f + g$). Pro $n ≥ 2$ je tvrzení zřejmým důsledkem předchozích vět a jednorozměrného případu.
	\end{dukazin}
\end{veta}

\begin{dusledek}
	$u \in W^{1, p} \implies |u| \in W^{1, p} \land \nabla |u| = \sgn u \nabla u$.
\end{dusledek}

% 08. 11. 2023 Odpadlo.

% 15. 11. 2023

\begin{definice}
	Nechť $U$ je konvexní otevřená omezená a $x \in U$. Značíme $U_t := \{x + t(y - x) | y \in U\} = \{z | x + \frac{1}{t}(z - x) \in U\}$. Dále značíme $\overline{u}_U = \fint_{U} u dy$.

	Minkowského funkcionál množiny $U$ je $p(y) = \inf \{t | y \in U_t\}$. Platí, že $U_t = \{y | p(y) < t\}$.
\end{definice}

\begin{veta}[O odhadu potenciálem]
	Nechť $U$ je konvexní otevřená omezená, $x \in U$, $u \in W^{1, 1}(U)$. Dále nechť $x$ je Lebesgueův bod $u$, pak
	$$ |u(x) - \overline{u}_U| ≤ \frac{\diam U}{λ^n(U)} \int_0^1 \frac{1}{t^n} \int_{U_t} |Du(y)| dλ^n(y) dλ^1(t). $$
	Je-li navíc ? konečný, pak
	$$ u(x) - \overline{u}_U = \int_0^1 \frac{1}{t} \fint_{U_t} |Du(y)| (y - x) dλ^n(y) dλ^1(t). $$

	\begin{dukazin}
		Nechť $u \in ©C^1(U)$ ($\forall u \in W^{1, 1}\ \exists u_k \rightarrow u$ v $W^{1, 1}$, $u_k \in ©C^1$).
		$$ ξ(t) = \fint_U u(x + t(y - x)) dλ^n(y) = \fint_{U_t} u dλ^n. $$
		$ξ(1) = \fint_U u = \overline{u}_U$. $\lim_{s \rightarrow 0_+} ξ(s) = u(x)$ Lebesgueův bod. $0 < a < b < 1$:
		$$ ξ(b) - ξ(a) = \int_a^b \partial_t\(\fint_U u(x + t(y - x)) dy\) dt = \int_a^b \fint_U \nabla u(x + t(y - x))·(y - x) dy dt = $$
		$$ = \int_a^b \frac{1}{λ^n(U) t^n} \int_{U_t} \nabla u(z)\frac{(z - x)}{t} dz dt. $$
		$$ ω_a^b(z) = \begin{cases}\frac{1}{λ^n(u)} \int_{\max\{a, p(z)\}}^b \frac{1}{t^{n+1}} dt, & p(z) < b, \\ 0, & p(z) ≥ b.\end{cases} \qquad ω_a^b \in ®C_c(U) $$
		$$ ξ(b) - ξ(a) = \frac{1}{λ^n(U)} \int_a^b \int_{U_t} \nabla u(z) (z - x) dλ^n(z) \frac{dt}{t^{n+1}} =: ** = $$
		$$ = \frac{1}{λ^n(U)} \int_a^b \int_{\{p(z) < t\}} \nabla u(z) (z - x) dλ^n(z) \frac{dt}{t^{n+1}} = $$
		$$ = \frac{1}{λ^n(U)} \int_{U_b} \int_{(a, b) \cap \{p(z) < t\}} \nabla \frac{u(z) (z - x)}{t^{n+1}} dλ^n(z) dt = $$
		$$ = \int_U ω_a^b(z) \nabla u(z) (z - x) dz =: * $$
		Pomocí silné aproximace hladkými funkcemi v $W^{1, 1}(U)$ dostaneme $*$ i pro $\forall u \in W^{1, 1}(U)$.
		$$ ξ(b) - ξ(a) \overset{\text{Fubini}} \int_a^b \frac{1}{t} \(\fint_{U_t} D_u(z)(z - x) dz\) dt. $$
		Použijeme odhad $|z - x| ≤ \diam U_t = t \diam U$:
		$$ |ξ(b) - ξ(a)| ≤ \diam U \int_a^b \fint_{u_t} |D_u(z)| dz dt = \frac{\diam U}{λ^n(U)} \int_a^b \int_{U_t} |Du(z)| dz \frac{dt}{t^n}. $$
		$$ |ξ(1) - ξ(0)| ≤ \frac{\diam U}{λ^n(U)} \int_0^1 \int_{U_t} Du(z) dz \frac{dt}{t^n}. $$

		$$ ** \overset{b \rightarrow 1_-, a \rightarrow 0_+}\longrightarrow \int_0^1 \frac{1}{t} \fint_{U_t} Du(z) (z - x) dz dt $$
	\end{dukazin}
\end{veta}

\break

\begin{dusledek}
	Nechť $U \in ®R^n$ otevřená konvexní omezená, $u \in W^{1, 1}(U)$ a nechť $x \in U$ je Lebesgueův. Pak
	$$ \int_U |u(y) - u(x)| dλ^n(y) ≤ c_U \int_U \frac{|Du(y)|}{|y - x|^{n-1}} dλ^n(y), $$
	kde $C_U$ závisí pouze na tvaru $U$.

	\begin{dukazin}
		Označme $R = \diam U$. Potom $p(y) ≥ \frac{|y - x|}{R}$, $y \in ®R^n$. Víme, že 
		$$ |ξ(b) - ξ(a)| ≤ \int_U ω_a^b(y) |Du(y)|·|y - z| dλ^n(y). $$
		$$ ω_a^b(y) ≤ \frac{1}{λ^n(U)} \int_{p(y)}^∞ \frac{1}{t^{n+1}} = \frac{1}{n·λ^n(U)} [p(y)]^{-n} ≤ \frac{1}{n·λ^n(u)} \frac{R^n}{|y - x|^n}. $$
		Potom
		$$ |ξ(b) - ξ(a)| ≤ C_U · \frac{R^n}{n·λ^n(U)} \int_U \frac{|Du(y)|}{|y - x|^{n-1}}. $$
		$$ |\fint_U u(y) - u(x) dy| ≤ C_U \int_U \frac{|Du(y)|}{|y - x|^{n-1}}. $$

		$v(y) = |u(y) - u(x)| \in W^{1, 1}$.
		$$ \int_{B(x, r)} | v(y) - v(x) | \rightarrow 0, \qquad \int \left| |u(y) - u(x)| - 0\right| = \int |u(y) - u(x)|. $$
		$|Dv| = |Du|$ skoro všude.
		$$ \left| \fint |u(y) - u(x)| \right| ≤ Cu \int \frac{|Du|}{|y - x|^{n-1}}. $$
	\end{dukazin}
\end{dusledek}

\begin{lemma}
	Nechť $r > 0$ a $x \in ®R^n$. Potom
	$$ \int_{B(¦o, r)} |x - y|^{1 - n} dλ^n(y) ≤ Cr. $$
	Odhad platí také, když vyměníme $B(¦o, r)$ za $Q(¦o, r)$

	\begin{dukazin}
		Nechť $x \in ®R^n \setminus B(¦o, 2r)$, pak $|y - x| > r$. $|y - x|^{1 - n} ≤ r^{1 - n}$. $λ^n (B(¦o, 1)) =: ω_n$, $λ^n(B(¦o, r)) = r^n ω_n$.
		$$ \int_{B(¦o, r)} |y - x|^n ≤ \int_{B(¦o, r)} r^{1 - n} = λ^n(B(¦o, r)) r^{1 - n} = ω_n r. $$

		$x \in B(0, 2r)$, $B(x, 3r) \supset B(0, r)$.
		$$ \int_{B(0, r)} |y - x|^{1 - n} ≤ \int_{B(x, 3r)} |y - x|^{1 - n} = \int_{B(0, r)} |y|^{1 - n}·\int_0^{3r} ©H^{n - 1}(S_t) t^{1 - n} = 3 ©H^{n - 1} (S_n) r. $$
	\end{dukazin}
\end{lemma}

\begin{lemma}[Symetrizace Rieszova integrálu s jádrem]
	Nechť $E \subseteq ®R^n$ měřitelná, pak $\int_E |x|^{1 - n} dλ^n(x) ≤ c·λ^n(E)^{\frac{1}{n}}$.

	\begin{dukazin}
		Nechť $R: λ^n(B(0, R)) = λ^n(E)$. Pak $λ^n(B \setminus E) = λ^n(E \setminus B)$.
		$$ \int_{E \setminus B} \frac{1}{|x|^{n-1}} ≤ \int_{E \setminus B} \frac{1}{R^{n+1}} = \int_{B \setminus E} R^{1 - n} ≤ \int_{B \setminus E} |x|^{1 - n}. $$
		$$ \int_E |x|^{1 - n} ≤ \int_B |x|^{1 - n} ≤ C R = C λ^n(B(0, R))^{\frac{1}{n}} = c(λ(E))^{\frac{1}{n}}. $$
	\end{dukazin}
\end{lemma}

\begin{veta}
	$Ω$ otevřená v $®R^n$, $u \in W^{1, 1}(Ω)$. Potom $\forall B \subset Ω$ platí
	$$ \int_B |u - \overline{u}_B| dλ^n ≤ c·r·\int_B |Du| dλ^n,  $$
	kde $C$ závisí pouze na $n$ a $r = \frac{\diam B}{2}$.

	\begin{dukazin}
		Je-li $x$ Lebesgueův bod pro $u$, pak
		$$ |u(x) - \overline{u}_B| ≤ C_B \int_B \frac{|Du(y)|}{|y - x|^{n - 1}} dy $$
		$$ \int_B |u(x) - \overline{u}_B| ≤ \int_B C(n) \int_B \frac{|D u(y)|}{|y - x|^{n - 1}} dy dx ≤ C(n) r \int_B |Du(y)|. $$
	\end{dukazin}
\end{veta}

% 22. 11. 2023
\begin{poznamka}
	Aproximace $u \in W^{1, p}$ … $\exists u_k \in ©C^∞$, $u_k \rightarrow u$ v $W^{1, p}$. ($\int |u_k - u|^p < ε$.) $u_k \rightarrow u$ v $W^{1, 1}$ ($λ^n(\{u_k ≠ u\}) < \frac{1}{k}$), $u_k \in Lip$.
\end{poznamka}

\begin{veta}
	Nechť $u \in W^{1, 1}(B)$. Pak $\exists E_m: B \supset E_1 \supset E_2 \supset …$ takové, že $m·λ^n(E_m) \overset{m \rightarrow ∞}\longrightarrow 0$ a $\exists g_m \in Lip(B)$:
	\begin{itemize}
		\item $lip g_m ≤ C·m$;
		\item $\{g_n = u\} \subseteq B \setminus E_m$;
		\item $\|u - g_m\|_{1,1} \rightarrow 0$.
	\end{itemize}

	\begin{poznamkain}
		$Mf(x) = \sup_{r > 0} \fint_{B(x, r)} |f|$ ($f \in L^1_{loc}(®R^n)$). Slabý odhad:
		$$ λ^n(Mf > α) ≤ \frac{C}{α} \int_{|f| > \frac{α}{2}} |f| dλ^n \overset{α \rightarrow ∞}\longrightarrow 0. $$
	\end{poznamkain}

	\begin{dukazin}
		Definujeme $h(x) = |Du(x)|$ pro $x \in B$ a $h(x) = 0$ jinak. $h \in L^1(®R)$. Dále definujme $E_m = \{Mh > m\}$. Díky slabému odhadu je $λ^n(E_m) ≤ \frac{C}{m} \int_{\{|h| > \frac{m}{2}\}} |h| = ς\(\frac{1}{m}\)$. Chceme dokázat, že $x, y \in B \setminus E_m$ je $|u(x) - u(y)| ≤ C·m·|x - y|$. Definujme $B_j = B(x, 2^{-j} |x + y|)$, $B_{-j} = B(y, 2^{-j} |x + y|)$, $B_0 = B\(\frac{x}{2} + \frac{y}{2}, 2|x - y|\)$, $j \in ®N$.

		Protože je $x, y \in B \setminus E_m$ je $\fint_{B_j} h ≤ m$ $\forall j \in ®Z$ (pro $j = 0$, protože to tak vyjde). Předefinujeme $E_m$ jako $\{Mh > m\} \cup \{x \text{ má Lebesgueův bod pro $u$}\}$. Tím pádem máme $\lim_{j \rightarrow ∞} \fint_{B_j} u =: \lim_{j \rightarrow ∞} \overline{u_{B_j}} = u(x)$. Potom
		$$ |u(x) - u(y)| ≤ \left|\overline u_{B_1} - \overline u_{B_{-1}}\right| + \sum_{j=1}^∞ \left|\overline u_{B_{j+1}} - \overline u_{B_j}\right| + \left|\overline u_{B_{-j-1}} - \overline u_{B_{-j}}\right|. $$
		$$ |\overline u_{B_{j+1}} - \overline u_{B_j}| = \left| \fint_{B_{j+1}} u - \overline u_{B_j} \right| ≤ \fint_{B_{j+1}} |u - \overline u_{B_j} ≤ 2^n \fint_{B_j} |u - \overline u_{B_j}| ≤ C·|x - y|·2^{-j} \fint_{B_j} |Du|. $$
		$$ |\overline u_{B_1} - \overline u_{B_{-1}}| = \left|\fint_{B_0} \overline u_{B_1} - u + \fint_{B_0} u - \overline u_{B_{-1}}\right| = \left|\fint_{B_1} u - \overline u_{B_0} + \fint_{B_{-1}} \overline u_{B_0} - u\right| ≤ $$
		$$ ≤ \fint_{B_1} |u - \overline{u}_{B_0} + \fint_{B_{-1}} |u - \overline{u}_{B_0}| ≤ C \fint_{B_0} |u - \overline{u}_{B_0}| ≤ C·|x - y|·\fint_{B_0} |Du|. $$
		$$ |u(x) - u(y)| ≤ C·|x - y|·\fint_{B_0} |Du| + C·|x - y|·\sum_{j=1}^∞ 2^{-j} \fint_{B_j}|Du| + 2^{-j} \fint_{B_{-j}} |Du| ≤ C·m·|x - y|. $$
		Použijeme McShanovu rozšiřovací větu.

		$g_m \in W^{1, 1}(B)$. „$Dg_m = Du$ skoro všude na $B \setminus E_m$“: stačí $D_1g_m = D_1u$. Když vybereme správného reprezentanta $u$: $\exists \partial_1 u(·, \tilde x)$, $\exists \partial_1 g_m(·, \tilde x)$, $u = g_m$ skoro všude $\implies$ $\partial_1 u = \partial_1 g_m$ skoro všude (na $(a, b) \cap B \setminus E_m$).

		$\int_B |u - g_n| ≤ \int_{B \setminus E_m} |u| + \int_{E_m} |g_m| \rightarrow 0 + \lim_{m \rightarrow ∞} \int_{E_m} |g_m|$. A $|g_n| ≤ \sup |u|_{B \setminus E_m}|$.
		$$ \int_B |Du - Dg_m| ≤ \int_{E_m} |Du| dλ^n + c·m λ^n(E_m) \rightarrow 0, $$
		neboť $λ^n(E_m) \rightarrow 0$ a slabý odhad.
		$$ \int_B |g_m - \overline{g_m}_B| ≤ C·r·\int_B |D g_m| ≤ C·\int_B |Du|. $$
		$g_m - \overline{g_m}_B \in L^1$.
		$$ |\overline u - \overline{g_m}_B| ≤ \fint_B |u - \overline u_B| + |g_m - \overline{g_m}_B| ≤ c·\int_B |Du|. $$
		Tedy $\int_{E_m} g_m \rightarrow 0$ (neboť $g_m$ je omezená).
	\end{dukazin}
\end{veta}

\begin{veta}[Franchi–Hajlasz–Koskela]
	Nechť $u \in L^1(Ω)$. Pak $u \in W^{1, 1}(Ω)$ $\Leftrightarrow$ $\exists f \in L^1(Ω)\ \forall B(x, r) \subset\subset Ω: \int_B |u - \overline u_B| ≤ r·\int_B f$.

	\begin{dukazin}
		$u \in W^{1, 1} \implies \int |u - \overline u_B| ≤ C·r·\int_B |Du|$.

		„$\implies$“: Viz věta výše. „$\impliedby$“: Nechť $Ω' \subset \subset Ω: Ω' + B(0, \frac{1}{k}) \subset Ω$, pak $ψ_k * u \in ©C^1(Ω')$ ($ψ_k(x) = k^n ψ(kx)$).
		$$ \forall i \in [n]\ \forall x \in Ω': \int_{B(x, k^{-1})} \partial_i ψ_k(x - y) \overline u_{B(x, k^{-1})} = \overline u_{B(x, k^{-1})} \int_{B(0, k^{-1})} \partial_i ψ_k(y) dy = 0. $$
		$$ \int |\partial_i ψ_k * u| = \left| \int_{B(x, k^{-1})} \partial_i ψ_k(x - y)(u(y) - \overline u_{B(x, k^{-1})}) \right| ≤ $$
		$$ ≤ C k^{n+1} \int_{B(x, k^{-1})} |u - \overline u_{B(x, k^{-1})}| ≤ C k^n \int_{B(x, k^{-1})} f ≤ C \|f\|_1. $$

% 29. 11. 2023

		Z toho vidíme, že pro $u_k = ψ_k * u$ je posloupnost $\{\partial_i u_k\}$ omezená v $L^1(Ω')$ a tudíž existuje $μ_i \in ©M(Ω)$ tak, že (podposloupnost) $\partial_i u_k χ_{Ω'}$ konverguje slabě s * k $μ_i$. Tedy pro každou $φ \in ©D(Ω')$ je
		$$ \int_{Ω'} u(\partial_i φ) dλ^n = \lim_{k \rightarrow ∞} \int_{Ω'} u_k(\partial_i φ) dλ^n = - \lim_{k \rightarrow ∞} \int_{Ω'} (\partial_i u_k) φ dλ^n = - \int_{Ω'} φ dμ_i. $$
		Tím jsme ověřili, že $μ$ je distributivní derivací $u$ v $Ω'$. Distributivní derivace v celém $Ω$ je tedy lokálně znaménková míra a díky nezávislosti odhadů na $Ω'$ je to globálně znaménková míra.

		Chceme $μ_i \ll λ^n$ ($φ \in C_c^∞(Ω')$):
		$$ \int_Ω u \partial_i φ \overset{u_k \rightarrow u \text{ v } L^2}\leftarrow \int_{Ω'} u_k \partial_i φ = -\int_{Ω'} \partial_i u_k φ \overset{(w^*)}\rightarrow -\intφ dμ_i. $$
		Tj. $μ_i = D_i u$ ve smyslu distribucí.

		$$ \int_G |\partial_i u_k φ| ≤ c·k^k \int_G \int_{B\(x, \frac{1}{k}\)} f(y) dy φ(x) dx ≤ c·k^n \int_G \int_{B\(x, \frac{1}{k}\)} f(y) φ(x) ≤ $$
		$$ ≤ c·k^n·\int_{G + B\(0, \frac{1}{k}\) \cap \supp φ} f(x) λ^n\(B\(x, \frac{1}{k}\)\) ≤ C·\int_G |f|. $$
		Limitním přechodem $k \rightarrow ∞$ a supremum přes $φ$:
		$$ \left| \int_G φ dμ_i \right| ≤ c·\int_G |f|, \qquad |dμ_i|(G) ≤ |f| dλ^n(G), \qquad μ_i \ll λ^n. $$
	\end{dukazin}
\end{veta}

\begin{veta}[Gagliardo–Nirembergoum]
	Nechť $U$ je otevřená omezená konvexní. Nechť $u \in W^{1, 1}(U)$ s nulovým mediánem v $U$. Pak
	$$ \|u\|_{L^{\frac{n}{n - 1}}} = \(\int_U |u|^{\frac{n}{n-1}}\)^{\frac{n-1}{n}} ≤ C_U \int_U |Du|, $$
	$$ \|u\|_{L^{p^*}} = \(\int_U |u|^{p^*}\)^{\frac{1}{p^*}} ≤ C_U \(\int_U |Du|^p\)^{\frac{1}{p}}, \qquad p^* = \frac{np}{n - p}, \qquad p \in [1, ∞]. $$

	\begin{dukazin}
		Níže.
	\end{dukazin}
\end{veta}

\begin{definice}
	Nechť $M \subset ®R^n$ a $0 < λ^n(M) < ∞$ a $u: M \rightarrow ®R$ je měřitelná. Říkáme, že $c \in ®R$ je mediánem $u$ na $M$, pokud:
	$$ λ^n(\{u > c\} \cap M) ≤ \frac{1}{2} λ^n(M), \qquad λ^n(\{u < c\} \cap M) ≤ \frac{1}{2} λ^n(M). $$
\end{definice}

\begin{tvrzeni}[Odhad slabého typu]
	Nechť $U \subset ®R^n$ je otevřená omezená konvexní, $u \in W^{1, 1}(U)$ s nulovým mediánem na $U$. Pak
	$$ s\(λ^n(\{u > s\})\)^{1 - \frac{1}{n}} ≤ c· \int_U |Du| dλ^n. $$

	\begin{dukazin}
		Označme $G_s := \{|u| > s\}$, $s λ^n(G_s) ≤ C· \int_{G_s} |u|$.
		$$ sλ^n(G_s) = \fint_U sλ^n(G_s) ≤ c·\fint_U \int_{G_s} |u(x)| dx dy = $$
		$$ = c· \int_{G_s} \fint_{U} |u(x)| dy dx ≤ $$
		$$ \(|u(x)| ≤ |u(x) - u(y)| \qquad \forall y: \sgn(u(y)) ≠ \sgn(u(x)).\) $$
		$$ ≤ c·\int_{G_s} \fint_{\{\sgn(u(y)) ≠ \sgn(u(x))\}} |u(x) - u(y)| \overset{\text{0 je med}}≤ c·\fint_{G_s} \fint_U |u(x) - u(y)| dy dx ≤ $$
		$$ ≤ c·\int_{G_s} \int_U \frac{|Du(y)Z|}{|x - y|^{n-1}} dy dx \overset{\text{Fubini}}= c·\int_U |Du(y)| \int_{G_s} \frac{1}{|x - y|^{n-1}} dx dy ≤ c·\int_U |Du(y)| λ^n(G_s)^{\frac{1}{n}}. $$
	\end{dukazin}
\end{tvrzeni}

\begin{veta}
	Nechť $U$ je otevřená omezená konvexní a $u \in W^{1, 1}(U)$ s nulovým mediánem v $U$. Pak
	$$ \int_0^∞ λ^n(\{|u| > s\})^{1 - \frac{1}{n}} ds ≤ c \int_u |Du|. $$

	\begin{dukazin}
		Zvolme $a > 0$, $k \in ®N$. Nechť $u ≥ 0$ (BÚNO, aplikujeme na $u_+ := \max\{u, 0\}$ a $u_- := \max\{-u, 0\}$). Zadefinujeme
		$$ v(x) := v_{a, k}(x) := \begin{cases}u(x) - (k - 1)a, & u(x) \in [(k-1)a, k·a] \\ 0, & u(x) ≤ (k-1)a\end{cases}. $$
		$v$ má nulový medián v $U$ a $Du = Dv$ skoro všude v $\{x | a(k - 1) ≤ u(x) ≤ ak\}$.
		$$ s·λ^n(\{|v| > s\})^{1 - \frac{1}{n}} ≤ c·\int_U Dv. $$
		$$ \int_{(k-1)a}^k·a \(λ^n(\{u > s\})\)^{1 - \frac{1}{n}} ds ≤ a(λ^n(\{u > (k - 1)a\}))^{1 - \frac{1}{n}} ≤ c·\int_{U \cap V_{a, k}} |Du|. $$
		Sečteme přes $k \in ®N$:
		$$ \int_a^∞ λ^n(\{u > s\})^{1 - \frac{1}{n}} ds ≤ c·\int_U |Du|, \qquad \forall a > 0. $$
		$a \rightarrow 0_+$.
	\end{dukazin}
\end{veta}

\begin{poznamka}
	$M$ měřitelná, $u: M \rightarrow ®R$ měřitelná. Pak
	$$ \int_M |u|^p = \int_0^∞ ps^{p - 1} λ^n(M \cap \{|u| > s\})ds. $$
\end{poznamka}

\break

\begin{lemma}
	$f: (0, ∞) \rightarrow [0, ∞)$ nerostoucí a $q ≥ 1$, $r > 1$. Pak
	$$ \int_0^∞ s^{qr - 1} f(s) ds ≤ c·\(\int_0^∞ s^{q - 1} f(s)^{\frac{1}{r}}ds\)^r. $$

	\begin{dukazin}
		$$ I := \int_0^∞ s^{q - 1} f(s)^{\frac{1}{r}}. $$
		$$ \forall t > 0: t^q f(t)^{\frac{1}{r}} = \int_0^t q s^{q - 1} (f(t))^{\frac{1}{r}} ds ≤ \int_0^t q s^{q - 1} (f(s))^{\frac{1}{r}} ≤ c·I. $$

		$$ s^{qr - q} f(s)^{1 - \frac{1}{r}} ≤ c·I^{r - 1}, \qquad t \mapsto t^{r - 1} \text{ rostoucí pro } r > 1. $$
		$$ \int_0^∞ s^{qr - 1} f(s) ≤ \int_0^∞ s^{qr - q} f(s)^{1 - \frac{1}{r}} ≤ c·I^{r - 1} \int_0^∞ s^{q - 1} f(s)^{\frac{1}{r}} = c·I^r. $$
	\end{dukazin}
\end{lemma}

% 13. 12. 2023

\begin{veta}[Gagliardova–Nirembergonova]
	Nechť $U$ je otevřená konvexní omezená a nechť $u \in W^{1, 1}(U)$ s nulovým mediánem v $U$. Pak
	$$ \(\int_U |u|^{\frac{n}{n - 1}}\)^{\frac{n-1}{n}} ≤ C_U \int_U |Du|, $$
	kde $C_U$ závisí pouze na tvaru $U$.

	\begin{dukazin}
		$$ \int_M |u|^p = \int_0^∞ ps^{p - 1} λ^n(\{|u| > s\}), \qquad p = \frac{n}{n - 1}. $$
		$$ \int_0^∞ s^{qr - 1} f(s) ≤ c·\(\int_0^∞ s^{q - 1} f^{\frac{1}{r}}\)^r, \qquad \frac{n}{n-1} - 1 = \frac{1}{n-1}. $$
		$$ \int_U |u|^{\frac{n}{n-1}} \int_0^∞ \frac{n}{n-1} s^{\frac{1}{n-1}} λ^n(U \cap \{|u| > s\}) ≤, \qquad q = 1, r = \frac{n}{n-1}. $$
		$$ ≤ c_n\(\int_0^∞ λ^n(U \cap\{|u| > s\})^{\frac{n-1}{n}}\)^{\frac{n}{n-1}} ≤ \(c·\int_U |Du|\)^{\frac{n}{n-1}}. $$
	\end{dukazin}
\end{veta}

\break

\begin{veta}[Sobolevova nerovnost]
	$U$ omezená otevřená konvexní, $1 ≤ p < n$ a nechť $u \in W^{1, p}(U)$ s nulovým mediánem v $U$. Pak
	$$ \(\int |u|^{p^*}\)^{\frac{1}{p^*}} ≤ C_U \frac{pn - p}{n - p} \(\int_U |Du|^p\)^{\frac{1}{p}}. $$

	\begin{dukazin}
		„$p = 1$“ triviální. „$p > 1$“: $u = u^+ - u^-$, tedy BÚNO $u ≥ 0$. Nejdřív předpokládejme navíc, že $u \in L^∞(U)$. Definujme $v = u^{\frac{pn - p}{n - p}}$. (?$\frac{p}{p - n}(n - 1) v$ je ACL?.)
		$$ \nabla v = \frac{pn - p}{n - p} u^{\frac{pn - n}{n - p}} \nabla u \implies v \in W^{1, 1}(U). $$
		$$ \(\int_U |u|^{p^*}\)^{\frac{n-1}{n}} = \(\int_U |v|^{\frac{n}{n - 1}}\)^{\frac{n - 1}{n}} ≤ C_U \int_U |Dv| = C_U \frac{pn - p}{n - p} \int_U u^{\frac{np - n}{n - p}} |Du| ≤ $$
		$$ ≤ C_U \frac{pn - p}{n - p} \(\int_U |Du|^p\)^{\frac{1}{p}}·\(\int_U u^{p^*}\)^{\frac{p - 1}{p}}. $$
		$$ \(\frac{n - 1}{n} - \frac{p - 1}{p} = \frac{np - p - np + n}{np} = \frac{n - p}{np} = \frac{1}{p^*}.\) $$

		Tím je důkaz hotov pro $u \in W^{1, p} \cap L^∞$. Nechť $u \in W^{1, p}$. Zadefinujeme $u_k = \min\{u, k\}$. $\exists Du_k$ a $|Du_k| ≤ |Du|$.
		$$ \(\int |u|^{p^*}\)^{\frac{1}{p^*}} \leftarrow \(\int |u_k|^{p^*}\)^{\frac{1}{p^*}} ≤ c·\(\int |Du_k|^p\)^{\frac{1}{p}} ≤ c·\(\int |Du|^p\)^{\frac{1}{p}}. $$
	\end{dukazin}
\end{veta}

\begin{veta}[Sobolevova nerovnost na celém $®R^n$]
	Nechť $1 ≤ p < n$ a nechť $u \in W^{1, p}$. Pak
	$$ \|u\|_{L^{p^*}(®R^n)} ≤ C_β \frac{p n - p}{n - p} \|Du\|_{L^p(®R^d)}. $$

	\begin{dukazin}
		Nechť $φ \in ©D(®R^n)$, $\supp φ$ je kompaktní, $φ$ má nulový medián v $B(¦o, R)$ pro $R$ dostatečně velké. Tím pádem je
		$$ \|φ\|_{L^{p^*}} ≤ C_β \frac{pn - p}{n - p} \|Dφ\|_{L^p(®R^n)}. $$
		Mějme $u \in W^{1, p}(®R^n)$, najdeme $φ_k \rightarrow u$ v $W^{1, p}$ a tudíž platí odhad i pro $u$.
	\end{dukazin}
\end{veta}

\begin{veta}[Sobolevova–Poincarého nerovnost]
	$U$ je omezená otevřená kompaktní, $p·q ≥ 1$. Nechť $u \in W^{1, p}(U)$. Pokud $\frac{1}{q} ≥ \frac{1}{p} - \frac{1}{n}$, pak
	$$ \(\fint_U |u - \overline{u}_U|^q\)^{\frac{1}{q}} ≤ C_{p, q, U} (\diam U)·(\fint |Du|^p)^{\frac{1}{p}}. $$

	\begin{dukazin}
		BÚNO $u$ má nulový medián v $U$. Za prvé nechť „$\frac{1}{q} = \frac{1}{p} - \frac{1}{n}$“: $q = \frac{np}{n - p}$:
		$$ \(\fint_U |u|^q\)^{\frac{1}{q}} ≤ c·(\diam U)^{\frac{-n}{q}} \(\int_u |u|^q\)^{\frac{1}{q}} ≤ c·(\diam U)^{\frac{-n}{q}} \(\int_U |Du|^p\)^{\frac{1}{p}} ≤ $$
		$$ ≤ c·(\diam U)^{\frac{-n}{q} + \frac{n}{p}} \(\fint_U |Du|^p\)^{\frac{1}{p}} = c·(\diam U)·\(\fint_U |Du|^p\)^{\frac{1}{p}}. $$

		„$q < \frac{1}{p} - \frac{1}{n}$“: Najdeme $s \in [1, n)$, $\(\fint |u|^q\)^{\frac{1}{q}} ≤ \(\fint |u|^{s^*}\)^{\frac{1}{s^*}}$, $\frac{1}{q} - \frac{1}{s^*} = \frac{1}{t}$.
	\end{dukazin}
\end{veta}

\begin{veta}[O vnoření $W^{1, p}(®R^n) \hookrightarrow L^q(Ω)$]
	Nechť $Ω \subset ®R^n$ otevřená a $λ^n(Ω) < ∞$, nechť $\frac{1}{q} ≥ \frac{1}{p} - \frac{1}{n}$ a nechť $u \in W^{1, p}(®R^n)$. Pak
	$$ (λ^n(Ω))^{-\frac{1}{q}} \|u\|_{L^q(Ω)} ≤ C_{p, n}(λ^n(Ω))^{-\frac{1}{p}} \|u\|_{L^p(Ω)} + C_{p, n} (λ^n(Ω))^{\frac{1}{n} - \frac{1}{p}} \|Du\|_{L^p(®R^n)}. $$

	\begin{dukazin}
		„$\frac{1}{q} = \frac{1}{p} - \frac{1}{n}$ ($q = p^*$)“:
		$$ \(\fint_Ω |u|^q\)^{\frac{1}{q}} ≤ (λ^n(Ω))^{-\frac{1}{q}} \(\int_{®R^n} |u|^{p^*}\)^{\frac{1}{p^*}} ≤ (λ^n(Ω))^{-\frac{1}{q}} C·\frac{pn - p}{n - p} \(\int_{®R^n} |Du|^p\)^{\frac{1}{p}}. $$
		
		„$\frac{1}{q} > \frac{1}{p} - \frac{1}{n}$“: BÚNO $λ^n(Ω) = 1$ (jinak rozšíříme $W^{1, p}$ na celé $®R^n$ za pomoci věty z PDR1 a pak položíme $v(y) = r·u(y / r)$, kde $r = (λ^n(Ω))^{1 / n}$). $\forall t ≥ s ≥ 1: \(\fint_Ω |u|^s\)^{\frac{1}{s}} = \(\fint_Ω |u|^t\)^{\frac{1}{t}}$. $q > p$, jinak Hölder a stačí první člen na pravé straně.
		$$ p < q < p^* \implies \(\fint_Ω |u|^q\)^{\frac{1}{q}} \overset{\text{Hölder}}≤ \(\fint_Ω |u|^{p^*}\)^{\frac{1}{p^*}} ≤ C_β \frac{p·n - p}{n - p} \(\int_{®R^n} |Du|^p\)^{\frac{1}{p}}. $$
	\end{dukazin}
\end{veta}

% 20. 12. 2023

\break

\begin{definice}
	Říkáme, že $x$ je bodem hustoty měřitelné množiny $E$, pokud $\lim_{r \rightarrow 0} \frac{λ^n(E \cap B(x, r))}{λ^n(B(x, r))} = 1$.

	Nechť $u: ®R^n \rightarrow ®R$ a nechť $x \in ®R^n$. Říkáme, že $L = \aplim_{y \rightarrow x} u(y)$ je aproximativní limita $u$ v bodě $x$, pokud $\exists E \subseteq ®R^n$ měřitelná: $x$ je bodem hustoty $E$ a $\lim_{y \rightarrow x, y \in E} u(y) = L$.

	\begin{poznamkain}
		$u$ měřitelná:
		$$ L = \aplim_{y \rightarrow x} u(y) \Leftrightarrow \forall ε, ς > 0\ \exists r_0: \forall r \in (0, r_0): \frac{1}{r^n} λ^n(\{|u(y) - L| ≥ ε\}) ≤ ς \Leftrightarrow $$
		$\Leftrightarrow$ $x$ je bodem hustoty $M_ε = \{y \in ®R^n \middle| |u(y) - u(x)| < ε\}$ $\forall ε$.
	\end{poznamkain}

	$A = \nabla_a u ≡ \aplim_{y \rightarrow x} \frac{u(y) - u(x) - A(y - x)}{|x - y|} = 0$.
\end{definice}

\begin{lemma}
	Nechť $Ω \subset ®R^n$ a $u \in W^{1, 1}_{loc}(Ω)$. Pak pro skoro všechna $x \in Ω$:
	$$ \lim_{r \rightarrow 0_+} \frac{1}{r} \fint_{B(x, r)} |u(y) - u(x) - Du(x) (y - x)| dλ^n(y) = 0. $$

	\begin{dukazin}
		BÚNO $x$ je Lebesgueův bod pro $u$ a $Du$. Definujeme $v(y) = u(y) - u(x) - Du(x)(u - x)$. $v(x) = 0$.
		$$ \frac{1}{R} \fint |v(y)| = \frac{1}{R} \fint_{B(x, R)} |v(y) - v(x)| ≤ $$
		$$ ≤ \frac{C}{R} \int_{B(x, R)} \frac{|Du(y)|}{|x - y|^{n-1}} = \frac{C}{R} \int_{B(x, R)} \(\int_{|x - y|}^∞ \frac{1}{r^n}dr\) |Du(y)| = $$
		$$ = \frac{C}{R} \int_0^R \frac{1}{r^n} \int_{B(x, r)} |Du(y) - Du(x)| dy dr + \frac{C}{R} \int_R^∞ \frac{1}{r^n} \int_{B(x, R)} |Du(y) - Du(x)| dy xr ≤ $$
		$$ ≤ c_1·\fint_0^R \underbrace{\fint_{B(x, r)} |Du(y) - Du(x)|}_{\overset{R \rightarrow 0_+}\longrightarrow 0} + c_2·\fint_{B(x, R)} \underbrace{|Du(y) - Du(x)|}_{\overset{R \rightarrow 0_+}\longrightarrow 0} \rightarrow 0. $$
	\end{dukazin}
\end{lemma}

\break

\begin{veta}
	Nechť $u \in W^{1, 1}(Ω)$. Potom $\exists \nabla_a u(x) = Du(x)$ pro skoro všechna $x \in Ω$.

	\begin{dukazin}
		Mějme $x$ takové, že $\lim_{r \rightarrow 0_+} \frac{1}{r} \fint_{B(x, r)} |u(y) - u(x) - Du(x)(y - x)| dy = 0$. Zvolme $ε > 0$ a $τ \in (0, ε)$ (např. $ε^2$). Označme $v(y) = u(y) - u(x) - Du(x)(y - x)$ a $ς = \(\frac{τ}{ε}\)^{\frac{1}{n+1}}$. Najděme $r_0 > 0$: $\fint_{B(x, ς)} |v| ≤ τς$ $\forall r \in (0, r_0)$.

		Máme
		$$ λ^n(\{|v(y)| ≥ ε |y - x|\} \cap B(x, r)) ≤ $$
		$$ ≤ λ^n(\{y \in B(x, r) \setminus B(x, ςr) \middle| |v(y)| ≥ ε·|y - x| ≥ ε·ς·r\}) + λ^n(B(x, δ·r)) ≤ $$
		$$ ≤ \int_{B(x, r) \cap \{|v| > ς·ε·r\}} \frac{|v|}{ς·ε·r} dy + ς^n λ^n(B(x, r)) ≤ \frac{τ·r}{ς·ε·r}·r^n + c·δ^n·r^n = \(\frac{τ}{ε·δ} + δ^n·c\)·r^n. $$
	\end{dukazin}
\end{veta}

\begin{lemma}
	Nechť $v \in Lip_{loc}(Ω, ®R)$ a nechť $\exists \nabla_a v(x)$. Potom $\exists \nabla v(x) = \nabla_a v(x)$.

	\begin{dukazin}
		BÚNO $u$ je lipschitzovská a $Lip u =: L$. Najděme $r_0 > 0$: $B(x, r_0) \subset Ω$ a $\forall r \in (0, r_0)$ platí
		$$ λ^n(B(x, r) \cap E) ≤ \(\frac{ε}{L}\)^n λ^n(B(x, r)), \qquad E := \{y\middle| |v(y)| > ε|y - x|\}, $$
		protože $\exists \nabla_a u(x)$. Nechť $E \cap B\(x, \frac{c}{4}\) \setminus B\(x, \frac{c}{8}\)$ ($|v(y)| ≤ C·ε|y - x|$) otevřená, tedy $\overline{B(x, r)} \setminus E$ je uzavřená $\implies$ $\exists z' \in \overline{B(x, r)} \setminus E$ nejbližší k $z$. $|z - z'| ≤ \frac{r}{4}$, protože $x \in B(x, r) \setminus E$.

		$z' \in B\(x, \frac{r}{2}\)$ a $B(z, |z - z'|) \subset B(x, r)$.
		$$ \(\frac{|z' - z|}{r}\)^n ≤ \frac{λ^n(E \cap B(x, r))}{λ^n(B(x, r))} ≤ \(\frac{ε}{L}\)^n. $$
		$|z' - z| ≤ \frac{ε}{L}r$, $|v(z') - v(z)| ≤ ε r$,
		$$ |v(z)| ≤ |v(z')| + |v(z) - v(z')| ≤ ε|z' - x| + ε r ≤ 2ε r ≤ c·ε·|z - x|. $$
	\end{dukazin}
\end{lemma}

\break

\begin{veta}[Rademacher]
	Nechť $u \in Lip_{loc}(Ω)$. Potom $\exists \nabla$ skoro všude na $Ω$, $u \in W^{1, 1}_{loc}(Ω)$ a $D u = \nabla u$.

	\begin{dukazin}
		$\implies$ $\exists \nabla_a u$ skoro všude $\implies$ $\exists \nabla u$ skoro všude. ? $\nabla u$ je měřitelná. $|\nabla u| < M \implies \nabla u \in L^1_{loc}(Ω)$. Funkce $u$ má $\partial_? u$ skoro všude na skoro všech přímkách $\implies$ (ACL charakterizace) $u \in W^{1, 1}_{loc}(Ω)$.

		$Lip(u, x) = \limsup_{y \rightarrow x} \frac{|u(y) - u(x)|}{|y - x|}$. $S(u) = \{x \middle| Lip(u, x) < ∞\}$.
	\end{dukazin}
\end{veta}

\begin{veta}[Stepanov]
	Nechť $u$ je reálná funkce na $Ω$. Potom $u$ je diferencovatelná skoro všude na $S(u)$.

	\begin{dukazin}
		Nechť $\{B_j\}$ – střed racionální poloměr ?. Pro každé $B_j$ definujeme lipschitsovské obálky:
		$$ w_j(x) = \inf\{w(x) | w(x) ≥ u(x) lip(u, B_j) ≤ j\}, $$
		$$ v_j(x) = \sup\{v(x) | v(x) ≤ u(x) lip(u, B_j) ≤ j\}. $$

		Dokažme, že pro $N := \bigcup \{x \in B | \nexists \nabla w_j(x) \lor \nexists v_j(x)\}$ je $λ^n(N) = 0$. Nechť $x \in S(u) \setminus N$. $\exists r > 0\ \exists l > 0: |u(x) - u(y)| ≤ l·|x - y|$. $lip(u, B_j) ≤ L ≤ j$ $\implies$ $w_j = u = v_j$ na $B_j$, $x \notin N$.
	\end{dukazin}
\end{veta}

% 03. 01. 2024

\begin{lemma}
	Nechť $U \subset ®R^n$ je omezená otevřená konvexní, nechť $p > r$ a nechť $u \in W^{1, p}(U)$. Pokud $x$ je Lebesgueův bod pro $u$, pak
	$$ |u(x) - \overline{u}_U| ≤ C·r^{1 - \frac{n}{p}} \(\int_U |Du|^p\)^{\frac{1}{p}}, $$
	kde $r = \diam U$ a $C = C(n, p, U)$.

	\begin{dukazin}
		$$ |u(x) - \overline{u}_U| ≤ c·\int_U \frac{|Du(y)|}{|x - y|^{n-1}} dy ≤ c·\(\int_U |Du|^p\)^{\frac{1}{p}}·\(\int_U \frac{1}{|x - y|^{\frac{np - p}{p - 1}}}\)^{1 - \frac{1}{p}} ≤ $$
		$$ ≤ c·\(\int_U |Du|^p\)^{\frac{1}{p}} · \(\int_{B(x, r)} |x - y|^{\frac{(1 - n)p}{p - 1}}\)^{1 - \frac{1}{p}} ≤ $$
		$$ ≤ c·(…)^{\frac{1}{p}} \(\int_0^r \underbrace{\overbrace{t^{\frac{1 - np}{p-1}}}^{\approx r^{1 - \frac{n - 1}{p - 1}}} · t^{\frac{(n - 1)·(p - 1)}{p - 1}}}_{=t^{\frac{-n-1}{p - 1}}}\)^{\frac{p - 1}{p}} ≤ c·(…)^{\frac{1}{p}}·\(r^{\frac{p - n}{p - 1}}\)^{\frac{p - 1}{p}}. $$
	\end{dukazin}
\end{lemma}

\begin{veta}[Morrey]
	Nechť $U$ je omezená otevřená konvexní a nechť $p > n$ a $u \in W^{1, p}(U)$. Pak $u$ má spojitého reprezentanta. Navíc
	$$ osc_U u := \sup\{|u(x) - u(y)|, x,y \in U\} ≤ c·r^{1 - \frac{n}{p}} \(\int_U |Du|^p\)^{\frac{1}{p}} = c·r^{1 - \frac{n}{p}}·\|Du\|_p, $$
	$$ \sup_U u ≤ C·r^{1 - \frac{n}{p}}\(\int_U |Du|^p + r^{-p}|u|^p\). $$

	\begin{dukazin}
		Pokud $u$ má spojitého reprezentanta, potom každý bod $x \in U$ je Lebesgueův pro $u$. Pak platí první (a z $\sup u ≤ \overline{u} + osc u$) i druhý odhad.

		% Z poznámek:
		Najdeme $u_k \in C^∞(U)$, $u_k \rightarrow u$ v $W^{1, p}(U)$. Druhý odhad $\implies$ konvergence v $W^{1, p}$ implikuje konvergenci v $L^∞$ a hledaná limita je náš reprezentant.

		% Na hodině: Ať $U^* := \{x \in U | x \text{ je Lebesgueův}\}$. $U^*$ je hustá v $U$, $u|_{U^*}$ je spojitá (použijeme tuto větu na $u|_U^*$, první odhad) a omezená (druhý odhad), a tudíž má spojité rozšíření na $U$.
	\end{dukazin}
\end{veta}

\begin{veta}
	Nechť $U$ je omezená otevřená kompaktní, $p > n$ a $u \in W^{1, p}(U)$.
	$$ \|u\|_{C^{0, 1 - \frac{n}{p}}} = \sup_{x ≠ y \in U} \frac{|u(y) - u(x)|}{|y - x|^{1 - \frac{n}{p}}} ≤ c·\(\int |Du|^p\)^{\frac{1}{p}} = c·\|Du\|_p. $$

	\begin{dukazin}
		$r = \diam U$. $\exists z \in U$ a $ρ > 0$: $\forall B(w, s) \subset U: s ≤ 2ρ$. Pak $\frac{r}{ρ}$ závisí pouze na tvaru $U$. Vybereme $x, y \in U$ a označme $t = \frac{|x - y|}{ρ}$ ($tρ = |x - y|$). Pokud $t ≥ 1$, pak stačí použít předchozí větu. Nechť tedy $t < 1$:
		$$ φ^x(w) = x + t(w - x), \quad φ^y(w) = y + t·(w - y), \qquad U^x = φ^x(U), \quad U^y = φ^y(U), \qquad x^* = φ^x(z), \quad y^* = φ^y(z). $$
		$$ |y^* - x^*| = |y + t(z - y) - x - t(z - x)| = |1 - t|·|y - x| < t·ρ. $$
		$$ \O ≠ B(x^*, t·ρ) \cap B(y^*, t·ρ) \subset U^x \cap U^y. $$
		$$ |u(y) - u(x)| ≤ c·(osc_{U_x}u + osc_{U^y}u) ≤ c·(\diam U^x + \diam U^y)^{1 - \frac{n}{p}}\|Du\|_p ≤ $$
		$$ ≤ c·(t·r)^{1 - \frac{n}{p}} \|Du\|_p ≤ C·|x - y|^{1 - \frac{n}{p}} \|Du\|_p. $$
	\end{dukazin}
\end{veta}

\begin{veta}
	Nechť $p > n$, $u \in W^{1, p}(®R^n)$. Pak
	$$ \sup_{x≠y} \frac{|u(x) - u(y)|}{|x - y|^{1 - \frac{n}{p}}} ≤ c·\|Du\|_{L^p(®R^n)} $$
	a
	$$ \sup_{®R^n} u ≤ c·\(\int_{®R^n} |Du|^p + |u|^p\)^{\frac{1}{p}}. $$
	Navíc $u \in C_0(®R^n)$.

	\begin{dukazin}
		$z \in ®R^n$, $u = B(z, 1)$, $u(z) ≤ c·\(\int_{B(z, 1)} |Du|^p + |u|^p\)^{\frac{1}{p}} ≤ c·\(\int_{®R^n}…\)^{\frac{1}{p}}$. $u \in C_0(®R^n)$ je důsledkem hustoty testovaček v $W^{1, p}$ a vnoření $W^{1, p} \hookrightarrow L^∞$.
	\end{dukazin}
\end{veta}

\begin{veta}
	$p > n$, $Ω \subset ®R^n$ omezená, rozšiřitelná. Pak $W^{1, p}(®R^n) \overset{\text{kompaktně}}\hookrightarrow ®C(\overline{Ω})$.

	\begin{dukazin}
		$$ \exists C: |u(x) - u(y)| ≤ c·|x - y|^{1 - \frac{n}{p}} \qquad \forall x, y \in Ω\ \forall u: \|u\|_{W^{1, p}(®R^n)} $$
		Totiž jsou to funkce stejně stejnoměrně spojité a $\sup u ≤ C$, tedy jsou stejně omezené. Použijeme Arzela-Ascoli (tj. $W^{1, p}(®R^n) \hookrightarrow C(\overline{Ω})$ kompaktně).

		$v(y):= u(y) - u(x) - Du(x)(y - x)$. $x$ je $p$-Lebesgeův bod $Du$. Potom $\frac{|v(y)|}{r} ≤ c·\(\fint |Dv|^p\)^{\frac{1}{p}} \rightarrow 0$.
	\end{dukazin}
\end{veta}

\end{document}
