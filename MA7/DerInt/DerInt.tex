\documentclass[12pt]{article}					% Začátek dokumentu
\usepackage{../../MFFStyle}					    % Import stylu



\begin{document}

% 04. 10. 2022

\section{Preliminaries}
\begin{definice}[Slabá derivace]
	Nechť $f \in L^1_{loc}(®R^n)$. Říkáme, že $g \in L^1_{loc}(®R^n)$ je slabou derivací $f$ podle $i$-té proměnné, pokud platí
	$$ \int_{®R^n} f \partial_i φ dλ^n = - \int_{®R^n} g φ dλ^n \qquad φ \in ©D(®R^n) = ®C^∞_0(®R^n). $$
\end{definice}

\begin{definice}[Značení]
	$$ \partial_i f(x) = \lim_{h \rightarrow ¦o} \frac{f(x + e_i h) - f(x)}{h}, \qquad \nabla f(x) = \begin{pmatrix} \partial_1f(x) \\ \vdots\\ \partial_n f(x) \end{pmatrix}, $$
	$$ D_if \text{ slabá derivace dle $i$-té proměnné}, \qquad \nabla f(x) = \begin{pmatrix} \partial_1f(x) \\ \vdots\\ Dn f(x) \end{pmatrix}, $$
	$D·$ bude také značit derivaci distribuce? (Distribuční derivaci?)

	$f \in Lip(X, Y)$ jsou všechny Lipschitzovská zobrazení (tj. $ρ_Y(f(a), f(b)) ≤ lip(f)·ρ_X(a, b)$) z $X$ do $Y$.

	$A \triangle B := (A \setminus B) \cup (B \setminus A)$ (symetrický rozdíl množin).
\end{definice}

\begin{definice}[Lebesgueova–Stieltjesova míra]
	$μ$ míra vytvořená $M: I(®R) \rightarrow [0, ∞)$ pomocí Caratheodorovy konstrukce se nazývá Lebesgueova–Stieltjesova míra.
\end{definice}

\begin{definice}[Radonova míra]
	$©M^+_{loc}(Ω)$ je prostorem všech Borelovských měr na $Ω \subset ®R^n$, které jsou vnitřně regulární ($μ(E) = \sup\{\mu(K) | K \subset E\}$), lokálně kompaktní.

	Pokud navíc $|μ| < ∞$, pak je to prostor $©M^+$. $©M_{loc}(Ω) = μ^+ - μ^-$.
\end{definice}

\begin{definice}[?]
	$$ ψ(x) = \begin{cases}e^{-\frac{1}{1 - |x|^2}}, & |x| < 1,\\ 0, & |x| ≥ 1.\end{cases} $$

	$$ ψ_k(x) = k^n ψ(kx) $$

	\begin{poznamkain}
		$$\int |Ψ| = 1, \qquad ψ(x) = ψ(x'), |x| = |x'|, \qquad ψ \in ©D(®R^n) $$
		$$ ψ_k(x) > 0 \implies |x| < \frac{1}{k} $$
	\end{poznamkain}
\end{definice}

\begin{veta}[Lebesgueova o derivaci 1]
	Nechť $1 ≤ p < ∞$. Nechť $Ω \subset ®R^n$ otevřená. A nechť $f \in L^p_{loc}(Ω)$. Potom pro skoro všechna $x \in Ω$:
	$$ \lim_{r \rightarrow 0} \fint_{B(x, r)} |f(y) - f(x)|^p dλ^n(x) = 0. $$

	\begin{dukazin}
		Bez důkazu.
	\end{dukazin}
\end{veta}

\begin{definice}[Lebesgueův bod]
	Každý takový bod se nazývá ($p$) Lebesgueův bod.
\end{definice}

\begin{definice}[Konvoluce]
	$$ f*g(x) = \int_{®R^n} f(x - y)g(y) dy. $$
	Za podmínek, kdy pravá strana existuje. $g$ může být i míra.

	\begin{poznamka}
		Je-li $f * g \in L^1$ pak $f*g=g*f$. (Z Fubiniovy věty.)
	\end{poznamka}
\end{definice}

\begin{tvrzeni}
	Nechť $u \in L^1_{loc}(®R^n)$. Pak $ψ_k*u(x)$ je definováno $\forall x \in ®R^n$ a $\forall k \in ®N$.

	\begin{dukazin}
		Nebyl. Viz Funkcionalka.
	\end{dukazin}
\end{tvrzeni}

\begin{veta}
	$u \in L^1_{loc}(®R^n)$. Pak $ψ_k * u \in ®C^∞(®R^n)$ a $\partial_i(ψ_k*u) = \partial_iψ_k*u$.

	\begin{dukazin}
		Nebyl. Viz Funkcionalka.
	\end{dukazin}
\end{veta}

\begin{lemma}
	Nechť $f \in L^p$. Potom $ψ_k * f \in L^p$ $p \in [1, ∞]$. Navíc $\|ψ_k*f\|_p ≤ \|f\|_p$.

	\begin{dukazin}
		Nebyl. Viz Funkcionalka.
	\end{dukazin}
\end{lemma}

\begin{veta}
	$f \in L^1_{loc}(®R^n)$ nechť $x$ je Lebesgueův bod $f$ (a $f(x) = \lim_{r \rightarrow 0} \fint_{B(x, r)} f$) pak $ψ_k * f(x) \overset{k} \rightarrow f(x)$.

	\begin{dukazin}
		Nebyl.
	\end{dukazin}
\end{veta}

\begin{veta}
	Nechť $f \in ©C^∞_0(®R^n)$. Potom $ψ_k * f \rightrightarrows f$ na $®R^n$.

	\begin{dukazin}
		Nebyl.
	\end{dukazin}
\end{veta}

\begin{lemma}
	Pro $p \in [1, ∞)$ platí $\overline{©C^∞_0(®R^n)}^{L^p} = L^p(®R^n)$.

	\begin{dukaz}
		Nebyl. (Docela jednoduchý.)
	\end{dukaz}
\end{lemma}

\begin{veta}
	$1 ≤ p < ∞$: $f \in L^p(®R^n)$ $\implies$ $ψ_k * f \rightarrow f$ v $L^p(®R^n)$.

	\begin{dukazin}
		Nebyl.
	\end{dukazin}

	\begin{poznamkain}
		($ψ_1 * f \overset{w}\rightarrow f$ v $L^∞$)
	\end{poznamkain}
\end{veta}

\begin{veta}
	Nechť $u \in Lip(®R^n, ®R)$. Pak $u$ je slabě diferencovatelná na $®R^n$ a $\|Du\|_{L^∞} ≤ lip(u)$.

	\begin{dukazin}
		Nechť $x, z \in ®R^n$.
		$$ |ψ_k * u(z) - ψ_k * u(x)| = \left|\int(u(z - y) - u(x - y))ψ_k(y)dλ^n\right| ≤ lip(u) |z - x|. $$
		$lip(u_k) := lip(ψ_k * u) ≤ lip(u)$.
		Nechť $B$ je koule v $®R^n$. $\{\nabla u_k\}$ je omezená v $L^2(B)$ slabě konverguje k $g \in L^2(B, ®R^n)$.

		$\{f \in L^2(B): \|f\|_∞ ≤ c\}$ konvexní a uzavřená $\implies$ slabě uzavřená $\implies$ $\|g\|_∞ ≤ lip(u)$. Tedy
		$$ \int_B u\nabla φ \leftarrow \int_B u_k \nabla φ = -\int_B \nabla u_k φ \rightarrow -\int_B g φ. $$
	\end{dukazin}
\end{veta}

\begin{lemma}
	Nechť $E \subset Ω$ a pro nějaké $r > 0$: $E + B(¦o, r) \subset Ω$. Potom $\exists η \in ©D(Ω)$, že $η = 1$ na $E$.

	\begin{dukazin}
		$E + B\(0, \frac{r}{2}\) \subset\subset Ω$. Najdeme $k$, že $\frac{1}{k} < \frac{r}{2}$. Potom $ψ_k * χ_{E + B\(0, \frac{r}{2}\)}$ je hledaná funkce.
	\end{dukazin}
\end{lemma}

\section{Absolutně spojité funkce}
\begin{poznamka}[V této kapitole vždy]
	$I = (a_0, b_0)$ je interval. $®D(I)$ bude množina všech konečných dělení ($a_0 < x_0 < … < x_n < b_0$) intervalu.
\end{poznamka}

\begin{definice}[Variace funkce]
	Nechť $D = \{x_0 < x_1 < … < x_m\} \in ®D(I)$ a $u: I \rightarrow ®R$. Potom variace $u$ podle dělení $D$ je $V(u, D) = \sum_{i=1}^n |u(x_i) - u(x_{i-1})|$.

	Variace $u$ je $V(u, I) = \sup_{D \in ®D(I)} V(u, D)$.

	Je-li $V(u, I) < ∞$ pak říkáme, že $u$ má konečnou variaci na $I$.
\end{definice}

\begin{definice}[Absolutně spojité funkce]
	Nechť $u: I \rightarrow ®R$. Říkáme, že $u$ je (klasicky) absolutně spojité na $I$, jestliže ke každému $ε > 0$ existuje $δ > 0$ takové, že pro všechny $\{[a_i, b_i]\}_{i=1}^m$ po dvou disjunktní $\sum_{i=1}^m b_i - a_i < δ$ je $\sum_{i=1}^n |u(b_i) - u(a_i)| < ε$.
\end{definice}

% 11. 10. 2023

\begin{definice}
	Nechť $u: (a_0, b_0) \rightarrow ®R$. Říkáme, že $u \in W^{1, 1}(I)$ $\Leftrightarrow$ $u \in L^1(I)$ a $\exists Du \in L^1(I)$. ($Du = f dλ^1, f \in L^1$.)
\end{definice}

\begin{veta}
	Nechť $T \in ©D^*(I)$ a $\<T, φ'\> = 0$ $\forall φ \in ©D(I)$. Then $\exists c \in ®R$, $T = c(dλ^1)$ (tj. $\<T, φ\> = \int c φ$).

	\begin{dukazin}
		Nechť $η \in ©D(I): \int_I η = 1$. Nechť $φ \in ©D(I)$ a $λ = \<1, φ\> = \int_I φ$. Označme $c := \<T, η\>$. Zadefinujeme $Φ(x) = \int_{a_0}^x φ - λ η$, $Φ(b_0) = 0$, $Φ \in ©C_c^∞(I)$.
		$$ 0 = \<T, Φ'\> = \<T, φ\> - λ\<T, η\> = \<T, φ\> - λc = \<T, φ\> - \int c φ \implies $$
		$$ \implies \<T, φ\> = \int c φ. $$
	\end{dukazin}
\end{veta}

\begin{veta}
	Nechť $f: I = (a_0, b_0) \rightarrow ®R$, $f \in L^1(I)$. Potom
	\begin{enumerate}
		\item $\exists!$(až na aditivní $c$) $u$: $u(b) - u(a) = \int_a^b f$ (pro $a_0 < a < b < b_0$);
		\item $u$ má slabou derivaci a $Du = f$;
		\item $\exists!T \in ©D^*(I)$ (až na aditivní c), že $T' = f$;
		\item $T = udλ^1 + c dλ^1$;
		\item $u$ je absolutně spojitá;
	\end{enumerate}

	\begin{dukazin}
		„1.“ $u(x) = \int_{a_0}^x f(t) dt$.

		„2.“ $\int_I u(x) φ'(x) dx = \int_I φ'(x) \int_{a_0}^x f(t) dt dx \overset{\text{Fubini}}= \int_{a_0}^x f(t) \int_I φ'(x) dx dt = - \int_I φ(t) f(t) dt$. Tedy $Du = f$ na $I$.

		„3.“ a „4.“ jednoduché.

		„5.“: $f \in L^1 \implies \forall ε > 0\ \exists δ > 0\ \forall A \subset I$ měřitelná a $©L^1(A) < δ$: $\int_A |f| < ε$. Nechť $[a_i, b_i]$ po dvou disjunktní, $i \in [n]$, $\sum b_i - a_i < δ \implies \sum |u(b_i) - u(a_i)| ≤ \int_{\bigcup (a_i, b_i)} |f| < ε$.
	\end{dukazin}
\end{veta}

\begin{veta}
	Nechť $u$ je absolutně spojitá na $I = (a_0, b_0)$. Potom
	\begin{enumerate}
		\item $u$ je spojitá a lze ji spojitě dodefinovat na $\overline{I}$;
		\item $V(u, I) < ∞$ a $V(u, (a_0, x])$ je absolutně spojitá;
		\item $u$ je rozdílem 2 neklesajících funkcí;
		\item $\exists! f \in L^1: u(b) - u(a) = \int_a^b f$;
		\item $\exists u'$ skoro všude, $u'(x) = f(x)$ skoro všude;
		\item $Du = u' dλ^1$ na $I$;
		\item $u(b) - u(a) = \int_a^b u'(x)$.
	\end{enumerate}

	\begin{dukazin}
		„1.“ $\forall ε > 0 \exists δ: ε_k = 2^{-k} … δ_k$. $x_k \in (a_0, a_0 + δ_k)$.
		$$ \sum_{k=n}^∞ |u(x_{k + 1}) - u(x_k)| < 2^{-n + 1}. $$
		Značme $u(a_0)$ jakoukoliv limitu $u(x_k)$. Potom $|u(a_0) - u(x)| < 2ε_k$ jakmile $x \in (a_0, a_0 + δ_k)$.

		„2.“ $ε = 1$, $\exists δ > 0$. $λ^1(I) = b_0 - a_0$. Najdeme $N \in ®N$, že $N ≥ \frac{b_0 - a_0}{δ}$. $D$ je dělení $I$. $v(u, D) ≤ N$. $ε > 0$, $\exists δ > 0$: $V(u, (a_0, x]) := g(x)$ mějme konečné intervaly $λ^1(\bigcup [a_i, b_i]) < δ$ $\implies$ $\sum |u(b_i) - u(a_i)| < ε$.

		„3.“: $v(x) = V(u, [a_0, x])$ a $v(x) - u(x)$ jsou hledané funkce. ($V(u, [a_0, x]) + V(u, (x, y)) = TODO$)

		„4.“: (z 3. předpokládejme, že $u$ je neklesající) Caratheodorovou konstrukcí nalezneme míru: $M((a, b)) = u(b) - u(a)$ a ukážeme o ní, že je spojitá (pak je to Lebesgue-Stieltjesova míra, tedy platí $M((a, b)) = \int_a^b f$). Nechť $λ^1(N) = 0$. $\forall δ > 0$ najdu $G \supset N$ $λ^1(G) < δ$, $G$ otevřená, tedy $G = \bigcup_{i=1}^∞(a_i, b_i)$. $μ(G) = \sum_{i=1}^m |u(b_i) - u(a_i)| < ε$.
		$$ \frac{1}{2δ} \int_{x - δ}^{x + δ} f(y) dy \overset{δ \rightarrow 0^+} f(x). $$

	\end{dukazin}
\end{veta}

\begin{veta}
	Nechť $u$ je spojitá na $I$. Pak NÁPOJE:
	\begin{itemize}
		\item $u$ je absolutně spojitá na $I$;
		\item $u \in W^{1, 1}(I)$;
		\item $\exists f \in L^1(I): Du = fdλ^1$;
		\item $Du$ má $L^1$ reprezentanta $u(b) - u(a) = \int_a^b Du$;
		\item $\exists u'$ skoro všude, $u' \in L^1$ a $u(b) - u(a) = \int_a^b u'$;
		\item $\exists f \in L^1: u(b) - u(a) = \int_a^b$;
		\item $\exists g \in L^1: |u(b) - u(a)| ≤ \int_a^b g$.
	\end{itemize}

	\begin{dukazin}
		Máme vše kromě „poslední bod $\implies$ první“: $λ^1(\bigcup(a_i, b_i)) < δ \implies \sum |u(b_i) - u(a_i)| < ε$.
	\end{dukazin}
\end{veta}

% 25. 10. 2023

\begin{definice}
	Nechť $Ω \subseteq ®R^n$ otevřená:
	$$ W^{1, 1}_{loc}(Ω) := \{u \in L^1_{loc}(Ω) | \forall i \in [n]\ \exists D_i u \in L^1_{loc}(Ω)\}, $$
	$$ W^{1, p}(Ω) = \{u \in W^{1, 1}_{loc} \middle| \|u\|_{1, p} < ∞\}, \text{ kde } \|u\|_{1, p} = \sqrt[p]{\sum_{i=1}^n \int_Ω |D_i u|^p + \int_Ω |u|^p}, $$
	$$ W^{1, p}_c(Ω) = \{u \in W^{1, p}(Ω) \middle| \exists K \subset Ω: \{u ≠ 0\} \subset K\}, $$
	$$ p \in [1, ∞): W^{1, p}_0(Ω) = \overline{W_c^{1, p}(Ω)}^{\|·\|_{1, p}}, \qquad p = ∞: W^{1, p}_0(Ω) = \overline{W^{1, ∞}(Ω) \cap C_0(Ω)}^{\|·\|_{1, p}}. $$
\end{definice}

\begin{veta}
	$(W^{1, p}(Ω), \|·\|_{1, p})$ je Banachův prostor.

	\begin{dukazin}
		Linearita a funkčnost normy je zřejmá. Jak je to s úplností? $u_k$ cauchyovská v $\|·\|_{1, p}$. $W^{1, p} \hookrightarrow L^p$ $\implies$ $u_k$ cauchyovská v $L^p$ $\implies$ $\exists u \in L^p: u_k \rightarrow u$ v $L^p$.
		$$ φ \in ©D(Ω): \int_Ω u \partial_i φ \leftarrow \int_Ω u_k \partial_i φ = -\int_Ω D_i u_k φ \rightarrow -\int g φ. $$
		Poslední konvergence z $\exists g: D_i u_k \rightarrow g$ v $L^p$ a $D_i u = g \in L^p$.
	\end{dukazin}
\end{veta}

\begin{veta}[Rieszova pro $W^{1, p}$]
	Nechť $Ω \subset ®R^n$ otevřená. $1 ≤ p < ∞$, $p' = \frac{1}{1 - \frac{1}{p}}$. Pak pro každý $L \in \(W^{1, p}(Ω)\)^*$ existuje $f \in L^{p'}(Ω, ®R^{n+1})$ takové, že $L(u) = \sum_{i=1}^n \int_Ω f^i D_i u + \int_Ω f^{n+1} u$ a navíc $\|L\|_{(W^{1, p})^*} = \|f\|_{L^p}$.

	\begin{dukazin}
		Definujeme $T: W^{1, p}(Ω) \rightarrow L^p(Ω, ®R^{n+1})$, $u \mapsto (D_1 u, …, D_n u, u)$. $T W^{1, p}(Ω) \Subset L^p(Ω, ®R^{n+1})$. Díky HB větě každý $L \in \(W^{1, p}(Ω)\)^*$ lze rozšířit na $L^1 \in (L^p(Ω, ®R^{n+1}))^* = L^{p'}(Ω, ®R^{n+1})$.
	\end{dukazin}

	\begin{poznamkain}
		$f, h \in L^{p'}(Ω, ®R^{n+1})$ jsou takové, že
		$$ \sum_i \int f^i D_i u + \int f^{n+1} u = \sum_i \int h^i D_i u + \int h^{n+1} u. $$

		$\Div h = \Div f$ na $Ω$ a $f·ν = h·ν$ na $\partial Ω$. $Δu = f$.
	\end{poznamkain}
\end{veta}

\begin{poznamka}
	$u_k \rightarrow u$ v $W^{1, p}$ $\Leftrightarrow$ $\|u_k - \|_{1, p} \rightarrow 0$.

	Pro $p = ∞$
\end{poznamka}

\begin{veta}
	Nechť 1. $u \in W^{1, p}(®R^n)$, $1 ≤ p < ∞$, nebo 2. $u \in W^{1, ∞}(®R^n) ψ_k * u \overset{*}\rightarrow u$ v $W^{1, ∞}(®R^n)$, potom $\|ψ_k * u - u\|_{1, p} \overset{k \rightarrow ∞}\longrightarrow 0$.

	\begin{dukazin}
		Víme $\int |ψ_k * D_i u - D_i u|^p \overset{k \rightarrow ∞}\longrightarrow 0$. Stačí dokázat $ψ_k * D_i u = D_i(ψ * u) = \partial_i (ψ_k * u)$. Nechť $φ \in ©D(®R^n)$, pak
		$$ - \int_{®R^n} D_i(ψ_k * u)φ = \int_{®R^n} ψ_k * u \partial φ = \int_{®R^n} \int_{®R^n} ψ_k(x - y) u(y) dy \partial_i φ(x) dx = $$
		$$ = \int_{®R^n} \int_{®R^n} ψ_k(y - x) \partial_i φ(x) dx u(y) dy = $$
		$$ = \int_{®R^n} ψ_k * \partial_i φ(y) u(y) dy = \int_{®R^n} \partial(ψ_k * φ) u(y) = $$
		$$ = - \int_{®R^n}(ψ_k * φ)D_i u = \int_{®R^n} (ψ_k * D_i u) φ. $$
		Tedy $ψ_k * D_i u = D_i(ψ_k * u)$.
	\end{dukazin}
\end{veta}

\begin{veta}
	$$ \overline{©D(®R^n)}^{W^{1, p}} = W^{1, p}(®R^n). $$

	\begin{dukazin}
		Definujeme $η(x) = 1$ na $B(0, 1)$, $η(x) = 0$ na $®R^n \setminus B(0, 2)$, $η(x) \in [0, 1]$ in $®R^n$ a $η \in C^∞(®R^n)$.

		$u_k(x) = u(x) η(x / k)$. $u_k(x) = u(x)$ na $B(0, k)$, $u_k(x) = 0$ na $®R^n \setminus B(0, 2k)$.
		$$ \int_{®R^n} |u_k - u|^p ≤ \int_{®R^n \setminus B(0, k)} |2u|^p \overset{k \rightarrow ∞}\longrightarrow 0. $$
		$$ D_i u_k = D_i u(x)η\(\frac{x}{k}\) = (D_i u(x))η\(\frac{x}{k}\) + u(x) D_i η\(\frac{x}{k}\). $$
		$$ \int_{®R^n} |D_i u_k - D_i u|^p ≤ \int_{®R \setminus B(0, k)} \left|2D_i u + u(x) \frac{\|\nabla η\|_∞}{k}\right|^p ≤ $$
		$$ ≤ c·\int_{®R^n \setminus B(0, k)} |D_iu|^p + |u|^p \overset{k \rightarrow ∞}\longrightarrow 0. $$
	\end{dukazin}
\end{veta}

\begin{veta}
	$©D(®R^n)$ jsou slabě $*$ sekvenciálně husté v $W^{1, ∞}(®R^n)$. (Jinak řečeno, pro každé $u \in W^{1, ∞}$ najdeme $φ_k \subset ©D$, $φ_k \overset{*} \rightarrow u$ v $W^{1, ∞}(®R^n)$).

	\begin{dukazin}
		$u \in W^{1, ∞}$, $u_k(x) = u(x) η\(\frac{x}{k}\)$. Zvolme $f \in L^1$.
		$$ \int D_i u_k f = \int_{®R^n} χ_{B(0, 2k) \setminus B(0, k)} \frac{\partial_i η(\frac{x}{k})}{k} u(x) f(x) + \int_{?} η\(\frac{x}{k}\) D_i u f(x) = \int_{B(0, k)} D_i u f(x) \rightarrow \int D_i u f. $$
%		$\int η D_i u f(x) \rightarrow $
	\end{dukazin}
\end{veta}

% 01. 11. 2023

\begin{veta}
	Nechť $1 ≤ p < ∞$, $Ω \subset ®R^n$ otevřená, $u \in W^{-1, p}(Ω)$. Potom $\exists u_k \in C^∞(Ω) \cap W^{1, p}(Ω): u_k \rightarrow u$ v $W^{1, p}(Ω)$.

	\begin{dukazin}
		$Ω_1 \Subset Ω_2 \Subset … \Subset Ω$, $\overline{Ω_k}$ kompaktní, $Ω_k$ Otevřené, $Ω = \bigcup_{j=1}^∞ Ω_j$. Najdeme rozklad jednotky $ω_j$ (tj. $ω_j \in ©D(Ω)$, $ω_j(x) \in [0, 1]$, $ω_j ≥ χ_{Ω_j}$, $\forall x \in Ω_n: \sum_{j=1}^∞ ω_j(x) = 1$).

		Mějme $ω_ju \in W^{1, p}_0(Ω)$ $\exists v_{k, j} \in ©D(Ω)$: $v_{k, j} \rightarrow ω_j u$ v $W^{1, p}$. Takže najdeme $v_j \in ©D(Ω): \int_Ω |v_j - ω_j u|^p + |Dv_j - Dω_j u| < 2^{-j} ε$, $v_j \in W^{1, p}$. $\sum_{j=1}^∞ ω_j u = u \in W^{1, p}$.

		Položme $v = \sum_{j=1}^∞ v_j$. Chceme dokázat $\|v - u\|_{1, p} < ε$. Nejprve na $Ω_n$. Máme $u|_{Ω_n} = \sum_{j=1}^n (ω_j u) |_{Ω_n} v|_{Ω_n}  = \sum_{j=1}^n v_j|_{Ω_n}$.
		$$ \int_Ω \|v - u\|^p + \|Dv - Du\|^p ≤ \int_{Ω_n} \sum_{j=1}^n |v_j - u_j|^p + |Dv_j - Du_j|^2 ≤ \sum_{j=1}^n \|v_j - u_j\|_{W^{1, p}(Ω_n)}^p ≤ \sum_{j=1}^n 2^j ε ≤ ε. $$
		Pošleme $n \rightarrow ∞$ a zjistíme, že $\|u - v\|_{W^{1, p}(Ω)}^p ≤ ε$.
	\end{dukazin}

	\begin{poznamkain}[Konstrukce rozkladu jednotky]
		Mějme nějaké $η_j: η_j \in ©D(Ω)$, $ TODO!!!$
	\end{poznamkain}
\end{veta}

\section{Absolutní spojitost po přímkách}
\begin{veta}
	Nechť $Ω \subset ®R^n$ otevřená a $a \in L^1$. Nechť platí následující: Pro každé $i \in [n]$ a pro $λ^n$-skoro všechna $x \in Ω$ je funkce $t \mapsto u(x + te_i)$ lokálně absolutně spojitá a $\partial_i u \in L^1(Ω)$. Pak $u \in W^{1, 1}(Ω)$ a $D_i u = \partial_i u dλ^n$.

	\begin{dukazin}
		$u \in L^1 \implies u$ měřitelná. $φ_x: t \mapsto u(x + te_i)$ absolutně spojitá $\implies$ $\exists φ_x'(t)$ pro skoro všechna $t$ a
		$$ φ_x'(t) = \lim_{y \rightarrow t, y \in Q} \frac{φ_x(y) - φ_x(t)}{y - t} \implies φ'(x, t) = \partial_i u(x + te_i) \text{ je $λ^n$-měřitelná}. $$
		BÚNO $Ω = (a_1, b_1) \times (a_2, b_2) \times … \times (a_n, b_n)$. Označme $\tilde Ω := (a_2, b_2) \times … \times (a_n, b_n)$. Nechť $φ \in ©D(Ω)$ pro $λ^n$-skoro všechna $\tilde x \in \tilde Ω$ máme ($u$ absolutně spojitá v $L^1$ $\implies$ $Du = u'$):
		$$ \int_{\tilde Ω} \int_{a_1}^{b_1} \partial_1 u(t, \tilde x) φ(t, x) dt d\tilde x = \int_{\tilde Ω} \int_{a_1}^{b_1} u(t, \tilde x) \partial_1 φ(t, x) dt d\tilde x. $$
		$$ \(\forall φ \in ©D(Ω): \int_Ω \partial u φ = -\int_Ω u \partial_1 φ\) $$
	\end{dukazin}
\end{veta}

\begin{veta}
	Nechť $Ω \subset ®R^n$ otevřená, $u \in W^{1, 1}(Ω)$. Pak existuje vhodný reprezentant funkce $u$ takový, že platí následující: Pro TODO!!!
\end{veta}

TODO!!!

\end{document}
