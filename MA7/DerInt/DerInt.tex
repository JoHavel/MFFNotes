\documentclass[12pt]{article}					% Začátek dokumentu
\usepackage{../../MFFStyle}					    % Import stylu



\begin{document}

% 04. 10. 2022

\section{Preliminaries}
\begin{definice}[Slabá derivace]
	Nechť $f \in L^1_{loc}(®R^n)$. Říkáme, že $g \in L^1_{loc}(®R^n)$ je slabou derivací $f$ podle $i$-té proměnné, pokud platí
	$$ \int_{®R^n} f \partial_i φ dλ^n = - \int_{®R^n} g φ dλ^n \qquad φ \in ©D(®R^n) = ®C^∞_0(®R^n). $$
\end{definice}

\begin{definice}[Značení]
	$$ \partial_i f(x) = \lim_{h \rightarrow ¦o} \frac{f(x + e_i h) - f(x)}{h}, \qquad \nabla f(x) = \begin{pmatrix} \partial_1f(x) \\ \vdots\\ \partial_n f(x) \end{pmatrix}, $$
	$$ D_if \text{ slabá derivace dle $i$-té proměnné}, \qquad \nabla f(x) = \begin{pmatrix} \partial_1f(x) \\ \vdots\\ Dn f(x) \end{pmatrix}, $$
	$D·$ bude také značit derivaci distribuce? (Distribuční derivaci?)

	$f \in Lip(X, Y)$ jsou všechny Lipschitzovská zobrazení (tj. $ρ_Y(f(a), f(b)) ≤ lip(f)·ρ_X(a, b)$) z $X$ do $Y$.

	$A \triangle B := (A \setminus B) \cup (B \setminus A)$ (symetrický rozdíl množin).
\end{definice}

\begin{definice}[Lebesgueova–Stieltjesova míra]
	$μ$ míra vytvořená $M: I(®R) \rightarrow [0, ∞)$ pomocí Caratheodorovy konstrukce se nazývá Lebesgueova–Stieltjesova míra.
\end{definice}

\begin{definice}[Radonova míra]
	$©M^+_{loc}(Ω)$ je prostorem všech Borelovských měr na $Ω \subset ®R^n$, které jsou vnitřně regulární ($μ(E) = \sup\{\mu(K) | K \subset E\}$), lokálně kompaktní.

	Pokud navíc $|μ| < ∞$, pak je to prostor $©M^+$. $©M_{loc}(Ω) = μ^+ - μ^-$.
\end{definice}

\begin{definice}[?]
	$$ ψ(x) = \begin{cases}e^{-\frac{1}{1 - |x|^2}}, & |x| < 1,\\ 0, & |x| ≥ 1.\end{cases} $$

	$$ ψ_k(x) = k^n ψ(kx) $$

	\begin{poznamkain}
		$$\int |Ψ| = 1, \qquad ψ(x) = ψ(x'), |x| = |x'|, \qquad ψ \in ©D(®R^n) $$
		$$ ψ_k(x) > 0 \implies |x| < \frac{1}{k} $$
	\end{poznamkain}
\end{definice}

\begin{veta}[Lebesgueova o derivaci 1]
	Nechť $1 ≤ p < ∞$. Nechť $Ω \subset ®R^n$ otevřená. A nechť $f \in L^p_{loc}(Ω)$. Potom pro skoro všechna $x \in Ω$:
	$$ \lim_{r \rightarrow 0} \fint_{B(x, r)} |f(y) - f(x)|^p dλ^n(x) = 0. $$

	\begin{dukazin}
		Bez důkazu.
	\end{dukazin}
\end{veta}

\begin{definice}[Lebesgueův bod]
	Každý takový bod se nazývá ($p$) Lebesgueův bod.
\end{definice}

\begin{definice}[Konvoluce]
	$$ f*g(x) = \int_{®R^n} f(x - y)g(y) dy. $$
	Za podmínek, kdy pravá strana existuje. $g$ může být i míra.

	\begin{poznamka}
		Je-li $f * g \in L^1$ pak $f*g=g*f$. (Z Fubiniovy věty.)
	\end{poznamka}
\end{definice}

\begin{tvrzeni}
	Nechť $u \in L^1_{loc}(®R^n)$. Pak $ψ_k*u(x)$ je definováno $\forall x \in ®R^n$ a $\forall k \in ®N$.

	\begin{dukazin}
		Nebyl. Viz Funkcionalka.
	\end{dukazin}
\end{tvrzeni}

\begin{veta}
	$u \in L^1_{loc}(®R^n)$. Pak $ψ_k * u \in ®C^∞(®R^n)$ a $\partial_i(ψ_k*u) = \partial_iψ_k*u$.

	\begin{dukazin}
		Nebyl. Viz Funkcionalka.
	\end{dukazin}
\end{veta}

\begin{lemma}
	Nechť $f \in L^p$. Potom $ψ_k * f \in L^p$ $p \in [1, ∞]$. Navíc $\|ψ_k*f\|_p ≤ \|f\|_p$.

	\begin{dukazin}
		Nebyl. Viz Funkcionalka.
	\end{dukazin}
\end{lemma}

\begin{veta}
	$f \in L^1_{loc}(®R^n)$ nechť $x$ je Lebesgueův bod $f$ (a $f(x) = \lim_{r \rightarrow 0} \fint_{B(x, r)} f$) pak $ψ_k * f(x) \overset{k} \rightarrow f(x)$.

	\begin{dukazin}
		Nebyl.
	\end{dukazin}
\end{veta}

\begin{veta}
	Nechť $f \in ©C^∞_0(®R^n)$. Potom $ψ_k * f \rightrightarrows f$ na $®R^n$.

	\begin{dukazin}
		Nebyl.
	\end{dukazin}
\end{veta}

\begin{lemma}
	Pro $p \in [1, ∞)$ platí $\overline{©C^∞_0(®R^n)}^{L^p} = L^p(®R^n)$.

	\begin{dukaz}
		Nebyl. (Docela jednoduchý.)
	\end{dukaz}
\end{lemma}

\begin{veta}
	$1 ≤ p < ∞$: $f \in L^p(®R^n)$ $\implies$ $ψ_k * f \rightarrow f$ v $L^p(®R^n)$.

	\begin{dukazin}
		Nebyl.
	\end{dukazin}

	\begin{poznamkain}
		($ψ_1 * f \overset{w}\rightarrow f$ v $L^∞$)
	\end{poznamkain}
\end{veta}

\begin{veta}
	Nechť $u \in Lip(®R^n, ®R)$. Pak $u$ je slabě diferencovatelná na $®R^n$ a $\|Du\|_{L^∞} ≤ lip(u)$.

	\begin{dukazin}
		Nechť $x, z \in ®R^n$.
		$$ |ψ_k * u(z) - ψ_k * u(x)| = \left|\int(u(z - y) - u(x - y))ψ_k(y)dλ^n\right| ≤ lip(u) |z - x|. $$
		$lip(u_k) := lip(ψ_k * u) ≤ lip(u)$.
		Nechť $B$ je koule v $®R^n$. $\{\nabla u_k\}$ je omezená v $L^2(B)$ slabě konverguje k $g \in L^2(B, ®R^n)$.

		$\{f \in L^2(B): \|f\|_∞ ≤ c\}$ konvexní a uzavřená $\implies$ slabě uzavřená $\implies$ $\|g\|_∞ ≤ lip(u)$. Tedy
		$$ \int_B u\nabla φ \leftarrow \int_B u_k \nabla φ = -\int_B \nabla u_k φ \rightarrow -\int_B g φ. $$
	\end{dukazin}
\end{veta}

\begin{lemma}
	Nechť $E \subset Ω$ a pro nějaké $r > 0$: $E + B(¦o, r) \subset Ω$. Potom $\exists η \in ©D(Ω)$, že $η = 1$ na $E$.

	\begin{dukazin}
		$E + B\(0, \frac{r}{2}\) \subset\subset Ω$. Najdeme $k$, že $\frac{1}{k} < \frac{r}{2}$. Potom $ψ_k * χ_{E + B\(0, \frac{r}{2}\)}$ je hledaná funkce.
	\end{dukazin}
\end{lemma}

\section{Absolutně spojité funkce}
\begin{poznamka}[V této kapitole vždy]
	$I = (a_0, b_0)$ je interval. $®D(I)$ bude množina všech konečných dělení ($a_0 < x_0 < … < x_n < b_0$) intervalu.
\end{poznamka}

\begin{definice}[Variace funkce]
	Nechť $D = \{x_0 < x_1 < … < x_m\} \in ®D(I)$ a $u: I \rightarrow ®R$. Potom variace $u$ podle dělení $D$ je $V(u, D) = \sum_{i=1}^n |u(x_i) - u(x_{i-1})|$.

	Variace $u$ je $V(u, I) = \sup_{D \in ®D(I)} V(u, D)$.

	Je-li $V(u, I) < ∞$ pak říkáme, že $u$ má konečnou variaci na $I$.
\end{definice}

\begin{definice}[Absolutně spojité funkce]
	Nechť $u: I \rightarrow ®R$. Říkáme, že $u$ je (klasicky) absolutně spojité na $I$, jestliže ke každému $ε > 0$ existuje $δ > 0$ takové, že pro všechny $\{[a_i, b_i]\}_{i=1}^m$ po dvou disjunktní $\sum_{i=1}^m b_i - a_i < δ$ je $\sum_{i=1}^n |u(b_i) - u(a_i)| < ε$.
\end{definice}

\end{document}
