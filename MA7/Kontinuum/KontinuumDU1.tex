\documentclass[12pt]{article}					% Začátek dokumentu
\usepackage{../../MFFStyle}					    % Import stylu



\begin{document}

\begin{priklad}[1.]
	We already know that
	$$ ε_{ijk} ε_{imn} = δ_{jm}δ_{kn} - δ_{jn}δ_{km}, $$
	and let us further assume that we also know that
	$$ ε_{ijk}ε_{lmn} = \det \begin{bmatrix}δ_{il} & δ_{im} & δ_{in}\\ δ_{jl} & δ_{jm} & δ_{jn}\\ δ_{kl} & δ_{km} & δ_{kn}\end{bmatrix}. $$
	Show that
	\begin{align*}
		ε_{ijk} ε_{ijn} &= 2δ_{kn},\\
		ε_{ijk} δ_{lm} &= ε_{jkm} δ_{il} + ε_{kim}δ_{jl} + ε_{ijm}δ_{kl}.
	\end{align*}

	\begin{dukazin}
		První rovnici vysčítáme přes $j = m$, čímž dostaneme (v menšenci víme, že $δ_{ac}δ_{cb} = δ_{ab}$, neboť ze součtu může být výraz nenulový jen pro $c = a$, a pak je jedna právě tehdy, když $b = c := a$):
		$$ ε_{ijk} ε_{ijn} = δ_{jj}δ_{kn} - δ_{jn}δ_{kj} = 3·1·δ{kn} - δ_{kn} = 2δ_{kn}. $$

		Využijeme první dokazovanou rovnost, poté druhou rovnost ze zadání (a nakonec $ε_{xyz} = -ε_{yxz}$):
		$$ ε_{ijk}δ_{lm} = \frac{ε_{ijk}ε_{abl}ε_{abm}}{2} = \frac{\det(…)ε_{abm}}{2} = $$
		$$ = \frac{ε_{abm}(δ_{ia}δ_{jb}δ_{kl} + δ_{ib}δ_{jl}δ_{ka} + δ_{il}δ_{ja}δ_{kb} - δ_{ia}δ_{jl}δ_{kb} - δ_{ib}δ_{ja}δ_{kl} - δ_{il}δ_{jb}δ_{ka})}{2} = $$
		$$ = \frac{ε_{ijm}δ_{kl} + ε_{kim}δ_{jl} + ε_{jkm}δ_{il} - ε_{ikm}δ_{jl} - ε_{jim}δ_{kl} - ε_{kjm}δ_{il}}{2} = ε_{ijm} δ_{kl} + ε_{kim}δ_{jl} + ε_{jkm}δ_{il}. $$
	\end{dukazin}
\end{priklad}

\begin{priklad}[2.]
	Let $®A, ®B \in ®R^{3 \times 3}$ be non-singular matrices such that $®A + ®B$ is non-singular matrix as well. Show that
	$$ \text{a)} \qquad \det(®A + ®B) = \det ®A + \tr(®A^T \cof ®B) + \tr(®B^T \cof ®A) + \det ®B $$
	$$ \text{b)} \qquad (®A + ®B)^{-1} = \frac{1}{\det(®A + ®B)}\,· $$
	$$·\,\(®A^2 + ®B^2 + ®A®B + ®B®A - (®A + ®B)\tr(®A + ®B) + \frac{1}{2}\(\(\tr(®A + ®B)\)^2 - \tr(®A + ®B)^2\)®I\) $$
	The Cayley-Hamilton theorem might be useful.

	\begin{dukazin}[a]
		Použijeme C–H ve tvaru $(\det ®A)®I = ®A^3 - (\tr ®A)®A^2 + (\tr \cof ®A)®A$, linearitu $\tr$ a $\tr (®A®B) = \tr (®B®A)$:
		$$ 3(\det (®A + ®B) - \det ®A - \det ®B) = $$
		$$ = \underbrace{\tr\((®A + ®B)^3 - ®A^3 - ®B^3\)}_{= 3\tr\(®A^2®B\) + 3\tr\(®B^2 ®A\)} + \underbrace{\tr\(-(\tr (®A + ®B))(®A + ®B)^2 + (\tr ®A)®A^2 + (\tr ®B)®B^2\)}_{=-(\tr ®B)\(\tr ®A^2\) - (\tr ®A)\(\tr ®B^2\) - 2(\tr (®A + ®B))(\tr ®A®B)} + $$
		$$ + \underbrace{\tr\((\tr \cof (®A + ®B))(®A + ®B) - (\tr \cof ®A)®A - (\tr \cof ®B)®B\)}_{=?} $$
		Část ? upravíme pomocí * příkladu a linearity $\tr$ na:
		$$ ? = \frac{1}{2}\(\(\tr (®A + ®B)\)^2 - \(\tr (®A + ®B)^2\)\)(\tr (®A + ®B)) - $$
		$$ - \frac{1}{2}\((\tr ®A)^2 - \tr ®A^2\)(\tr ®A) - \frac{1}{2}\((\tr ®B)^2 - \tr ®B^2\)(\tr ®B) = $$
		$$ = (\tr \cof ®A)(\tr ®B) + (\tr \cof ®B)(\tr ®A) + (\tr ®A)^2(\tr ®B) + (\tr ®A)(\tr ®B)^2 - \(\tr (®A®B)\)(\tr (®A + ®B)). $$

		Z příkladu * je $-(\tr ®B)\(\tr ®A^2\) + (\tr ®B)\(\tr ®A\)^2 = (\tr ®B)(\tr \cof ®A)$ (a s prohozeným $®A$ a $®B$). Tedy dohromady ($\tr ®A = \tr ®A^T$)
		$$ 3(\det (®A + ®B) - \det ®A - \det ®B) = 3\(\tr \(®A^2®B\) - (\tr ®A)(\tr ®A®B) + (\tr \cof ®A)(\tr ®B)\) + \overset{A \leftrightarrow B}… \overset\dagger= $$
		$$ \overset\dagger= \tr\((\cof ®A)^T®B\) + \tr\((\cof ®B)^T®A\) = \tr \(®B^T \cof ®A\) + \tr \(®A^T \cof ®B\). $$

		$\dagger$ z C–H: $(\cof ®A)^T = (\det ®A) ®A^{-1} = ®A^2 - (\tr ®A)®A + (\tr \cof ®A)®I$.
	\end{dukazin}

	\begin{dukazin}[b]
		Z C-H, kam dosadíme koeficienty odvozené na přednášce, velmi triviální úpravou rovnic (matice není singulární, tedy jí i jejím determinantem můžeme dělit) dostaneme
		$$ ®C^{-1} = \frac{1}{c_3}®C^2 - \frac{c_1}{c_3}®C + \frac{c_2}{c_3}®I = \frac{1}{\det ®C}®C^2 - \frac{\tr ®C}{\det ®C}®C + \frac{\tr \cof ®A}{\det ®A}®I. $$
		Tam můžeme dosadit $®C = ®A + ®B$:
		$$ \text{b)} \qquad (®A + ®B)^{-1} = \frac{1}{\det(®A + ®B)} · \((®A + ®B)^2 - (®A + ®B)\tr(®A + ®B) + \tr \cof (®A + ®B)\). $$
		To můžeme roznásobit a dosadit z * příkladu:
	$$ (®A + ®B)^{-1} = \frac{1}{\det(®A + ®B)}\,· $$
	$$·\,\(®A^2 + ®B^2 + ®A®B + ®B®A - (®A + ®B)\tr(®A + ®B) + \frac{1}{2}\(\(\tr(®A + ®B)\)^2 - \tr(®A + ®B)^2\)\) $$
	\end{dukazin}
\end{priklad}

\begin{priklad}[*]
	Let $®A \in ®R^{3 \times 3}$ be a non-singular matrix. Show that $\frac{1}{2}\((\tr ®A)^2 - \tr ®A^2\) = \tr(\cof(®A))$.

	\begin{dukazin}
		Vezměme C–H ve tvaru (®A není singulární, takže s ní můžeme vydělit)
		$$ -®A^2 + (\tr ®A)®A - (\tr \cof ®A)®I = (\det ®A) ®A^{-1} $$
		a vypusťme na to $\tr$ (ta je invariantní vůči transpozici a „$\tr - = - \tr$“):
		$$ -\tr ®A^2 + (\tr ®A)^2 - 3(\tr \cof ®A)®I = \tr \((\det ®A)A^{-1}\) = \tr \((\det ®A)A^{-T}\) =: \tr \cof ®A. $$
	\end{dukazin}
\end{priklad}

\end{document}
