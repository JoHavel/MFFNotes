\documentclass[12pt]{article}					% Začátek dokumentu
\usepackage{../../MFFStyle}					    % Import stylu



\begin{document}

% 06. 10. 2023

TODO!!!

% 13. 10. 2023

\begin{definice}[Dot product on the space of matrices]
	$$ ®A : ®B = \tr(®A ®B^T). $$
\end{definice}

\begin{definice}[Norm of matrix]
	$$ |®A| = (®A : ®A)^{\frac{1}{2}}. $$
\end{definice}

\begin{priklad}
	$$ (¦a \otimes ¦b)^T = ¦b \otimes ¦a. $$

	\begin{dukazin}
		$$ ¦u·(¦a \otimes ¦b)^T ¦v = (¦a\otimes¦b)¦u·¦v = (¦a(¦b·¦u))¦v = (¦b·¦u) (¦a·¦v) = ¦u·(¦b(¦a·¦v)) = ¦u·(¦b \otimes ¦a)¦v. $$
	\end{dukazin}
\end{priklad}

\begin{priklad}
	$$ \det(e^{®A}) = e^{\tr ®A}. $$

	\begin{dukazin}
		$$ e^{®A} = \lim\(®I + \frac{®A}{n}\)^n. $$
		$$ \det e^{®A} = \lim_{n \rightarrow ∞} \(\det \(®I + \frac{®A}{n}\)^n\) = \lim_{n \rightarrow ∞} \(\det\(®I + \frac{®A}{n}\)\)^n = ? $$
		Subtask: Is there an approximation for $\det(®I + ®S)$, where $®S$ is a „small“ matrix. Yes, we did it (KontinuumDU1.pdf) for $®S \in ®R^{3 \times 3}$:
		$$ \det(®I + ®S) = \det ®I + \tr(®I \cof ®S) + \tr(®S^T \cof ®I) + \det ®S \approx 1 + \tr(®S^T \cof ®I) + o(®S^2) = 1 + \tr(S) + o(®S^2). $$
		And for $®S \in ®R^{n \times n}$, one can see that:
		$$ \det(®I + ®S) = \begin{pmatrix} 1 + s_{11} & s_{12} & … & s_{1n} \\ s_{21} & 1 + s_{22} & … & s_{2n} \\ \vdots & \vdots & \ddots & \vdots \\ s_{n1} & s_{n2} & … & 1 + s_{nn} \end{pmatrix} = (1 + s_{11})(1 + s_{22})·…·(1 + s_{nn}) + o(®S^2) = $$
		$$ = 1 + s_{11} + s_{22} + … + s_{nn} + o(®S^2) = 1 + \tr ®S + o(®S^2). $$
		$$ ? = \lim_{n \rightarrow ∞} \(1 + \frac{\tr ®A}{n} + …\)^n = e^{\tr ®A}. $$
	\end{dukazin}
\end{priklad}

\begin{tvrzeni}
	$$ \det(®I + ®S) = 1 + \tr ®S + … $$
\end{tvrzeni}

\begin{definice}[Gateaux derivative]
	$$ D¦f(¦x)[¦y] := \frac{d}{dτ} ¦f(¦x + τ¦y)|_{τ=0}. $$
\end{definice}

\begin{definice}[Fréchet derivative]
	$¦f: U \rightarrow V$:
	$$ \lim_{\|¦y\|_U \rightarrow 0} \frac{\|¦f(¦x + ¦y) - ¦f(¦x) - D¦f(¦x)[¦y]\|_V}{\|¦y\|_V} = 0. $$

	\begin{poznamkain}
		Sometimes we write $\nabla f(¦x)·¦y$ instead of $Df(¦x)[¦y]$ (from Riesz representation theorem).
	\end{poznamkain}

	For matrices ($φ: ®A \in ®R^{3\times 3} \rightarrow ®R$):
	$$ \frac{\|φ(®A + ®B) - φ(®A) - Dφ(®A)[®B]\|_{®R}}{\|®B\|_{®R^{3 \times 3}}}. $$

	\begin{poznamkain}
		We write $\frac{\partial φ}{\partial ®A}(®A):®B$ instead of $Dφ(®A)[®B]$, where $\frac{\partial φ}{\partial ®A}(®A)$ is right matrix. Warning $\frac{\partial φ}{\partial ®A}(®A) ≠ Dφ(®A)$, because of transposition ($®A:®B = \tr(®A®B^T) = \tr(®A^T®B)$).
	\end{poznamkain}
\end{definice}

\begin{priklad}
	$$ \frac{\partial \tr ®A}{\partial ®A}(®A)[®B] = \frac{d}{dτ}(\tr(®A + τ®B))|_{τ = 0} = \frac{d}{dτ}\(\tr ®A + τ\tr ®B\)|_{τ = 0} = \tr ®B = ®I:®B. $$
	So $\frac{\partial \tr ®A}{\partial ®A} = ®I$.
\end{priklad}

\begin{priklad}
	$$ \frac{\partial \det ®A}{\partial ®A}(®A)[®B] = \frac{d}{dτ}(\det(®A + τ®B))|_{τ = 0} = \frac{d}{dτ}\(\det(®A)·\det\(®I + τ®A^{-1}®B\)\)|_{τ = 0} = $$
	$$ = \frac{d}{dτ}\((\det ®A)·\(1 + τ\tr(®A^{-1}®B) + o(τ^2)\)\)|_{τ = 0} = (\det ®A) \tr\(®A^{-1}®B\) = $$
	$$ = (\det ®A)\tr\(\(®A^{-T}\)^T®B\) = \((\det ®A)®A^{-T}\):®B. $$
	So $\frac{\partial \det ®A}{\partial ®A} = (\det ®A)®A^{-T} = \cof(®A)$.
\end{priklad}

\begin{priklad}
	$®A: ®R \rightarrow ®R^{3 \times 3}$.
	$$ \frac{d}{dt}(\det ®A(t)) = (\det ®A(t)) \tr\(®A(t)^{-1} \frac{d®A(t)}{dt}\). $$
\end{priklad}

\begin{priklad}
	$®F: ®A \in ®R^{3 \times 3} \rightarrow ®F(®A) \in ®R^{3 \times 3}$. $®F(®A) = ®A^{-1}$. (We know $\frac{1}{1 + x} = 1 - x + …$.)
	$$ \frac{\partial ®F(®A)}{\partial ®A}(®A)[®B] = \frac{d}{dτ}\((®A + τ®B)^{-1}\)|_{τ = 0} = \frac{d}{dτ}\(\(®A\(®I + τ®A^{-1}®B\)\)^{-1}\)|_{τ = 0} = $$
	$$ = \frac{d}{dτ} \(\(®I + τ®A^{-1}®B\)^{-1}®A^{-1}\)|_{τ = 0} = \frac{d}{dτ} \(\(®I - τ®A^{-1}®B + …\)®A^{-1}\)|_{τ = 0} = -®A^{-1}®B®A^{-1}. $$
	So we have $\frac{\partial (®A^{-1})_{ij}}{\partial(®A)_{kl}} (®B)_{kl}$.

	From chain rule (but this is easily solvable by differentiating $®A^{-1}(t)®A(t) = ®I$):
	$$ \frac{d}{dt}\(®A^{-1}\) = -®A^{-1} \frac{d®A}{dt} ®A^{-1}. $$
\end{priklad}

\begin{priklad}
	$®F(®A) = e^{®A}$
	$$ \frac{\partial e^{®A}}{\partial ®A}[®B] = \frac{d}{dτ}(e^{®A + τ®B})|_{τ = 0} = \frac{d}{dτ} \(®I + \frac{®A + τ®B}{1!} + \frac{(®A + τ®B)^2}{2!}\)|_{τ = 0}. $$
\end{priklad}

\begin{veta}[Daleckii–Krein]
	®A real symmetric matrix, $®A \in ®R^{k \times k}$, $®A = \sum_{i=1}^k λ_i ¦v_i \otimes ¦v_i$, where $λ_i$ are eigenvalues and $¦v_i$ are normalised orthogonal ($¦v_i·¦v_j = δ_{ij}$) eigenvectors.

	$f$ continuously differentiable real function defined on open set containing the spectrum of ®A
	$$ ®F(®A) := \sum_{i=1}^k f(λ_i) ¦v_i \otimes ¦v_i =: \sum_{i=1}^k f(λ_i) ®P_i. $$

	Then the formula for the Gateaux derivative of $f$ at point ®A in direction ®X reads
	$$ D®F(®A)[®X] = \frac{\partial ®F}{\partial ®A}[®X] = \sum_{i=1}^k \frac{df}{dλ}|_{λ = λ_i} ®P_i ®X ®P_i + \sum_{i=1}^k\sum_{j=1, j≠i}^k \frac{f(λ_i) - f(λ_j)}{λ_i - λ_j} ®P_i ®X ®P_j. $$

	Sometimes we write $D®F(®A)[®X] = f^{[1]}(®A) \ocircle ®X$ (Schur product of matrices, it is point-wise multiplication). Then
	$$ [f^{[1]}(®A)]_{ij} = \begin{cases}\frac{df}{dλ}|_{λ = λ_i}, & i = j,\\ \frac{f(λ_i) - f(λ_j)}{λ_i - λ_j}, & i ≠ j.\end{cases} $$

	\begin{dukazin}
		No summation conventions, all sums are stated explicitly!
		$$ ®F(®A) = \sum_{i=1}^k f(λ_i)¦v_i \otimes ¦v_i = $$
		$$ = \sum_{i=1}^k f(λ_i(a_{11}, a_{12}, …, a_{21}, …))¦v_i(a_{11}, a_{12}, …, a_{21}, …) \otimes ¦v_i(a_{11}, a_{12}, …, a_{21}, …). $$
		$$ \frac{\partial ®F(®A)}{\partial ®A} = \sum_{i=1}^k \(\frac{\partial f}{\partial λ_i} \frac{\partial λ_i}{\partial ®A} ¦v_i \otimes ¦v_i + f(λ_i) \frac{\partial ¦v_i}{\partial ®A} \otimes ¦v_i + f(λ_i)¦v_i \otimes \frac{\partial ¦v_i}{\partial ®A}\) = ?. $$

		We derivate $®A¦v_i = λ_i¦v_i$:
		$$ \frac{\partial ®A}{\partial ®A}¦v_i + ®A \frac{\partial ¦v_i}{\partial ®A} = \frac{\partial λ_i}{\partial ®A}¦v_i + λ_i \frac{\partial ¦v_i}{\partial ®A}. $$
		We multiply (with dot product) it by $¦v_i$:
		$$ ®P_i + \frac{\partial ¦v_i}{\partial ®A}·®A^T¦v_i = \frac{\partial λ_i}{\partial ®A}·1 + \frac{\partial ¦v_i}{\partial ®A}®A·¦v_i. $$
		$$ \frac{\partial λ_i}{\partial ®A} = ®P_i = ¦v_i \otimes ¦v_i. $$
		
		We again multiply derivative of $®A|¦v_i = λ¦v_i$, but this time by $¦v_j$:
		$$ ¦v_j \otimes ¦v_i + \frac{\partial ¦v_i}{\partial ®A}·λ_j¦v_j = 0 + λ_i \frac{\partial ¦v_i}{\partial ®A}·¦v_j. $$
		$$ (λ_j - λ_i) \frac{\partial ¦v_i}{\partial ®A}·¦v_j = -¦v_j \otimes ¦v_i. $$

		We also need $(¦v_j \otimes ¦v_i)®X_{ij} = … = ®P_i ®X ®P_j$:
		$$ … = (¦v_j \otimes ¦v_i)(¦v_i·®X¦v_j) = (¦v_j \otimes ¦v_i) ®X(¦v_j \otimes ¦v_j). $$
	\end{dukazin}
\end{veta}

% 20. 10. 2023

TODO!!!

% 27. 10. 2023

\section{Kinematics}
\begin{definice}
	We have some abstract body with point $P$. We can look at it in reference configuration (some point in past), where $K_0(P) = ¦X$ ($K_0$ = placer), $t = t_0$. Or in current configuration (how it is situated now), where $K_t(P) = ¦x$.

	The change of configuration, $¦χ$ in $¦x = ¦χ (¦X, t)$ is called deformation (but it contains translation and rotation too!).
\end{definice}

\begin{definice}
	Let us consider quantity $θ$ that describes the given material point. We can describe it by:
	\begin{itemize}
		\item $θ(P, t)$;
		\item $\hat{θ}(¦X, t)$ (referential/Lagrangian description, commonly used for solids because deformation is with respect to reference configuration);
		\item $\tilde θ(¦x, t)$ (spatial/Eulerian description, commonly used for fluids because velocity is time-local property).
	\end{itemize}
	But people write those functions without $\hat{\ }$ or $\tilde{\ }$

	\begin{poznamka}
		$$ \tilde θ(¦x, t)|_{¦x = ¦χ(¦X, t)} = \hat{θ}(¦X, t). $$
	\end{poznamka}
\end{definice}

\begin{definice}[Deformation gradient]
	$$ d¦x = ¦x_2 - ¦x_1 = ¦χ(¦X_2, t) - ¦χ(¦X_1, t) = $$
	$$ = ¦χ(¦X_1 + d¦X, t) - ¦χ(¦X_1, t) = ¦χ(¦X_1, t) + \frac{\partial ¦χ}{¦X}(¦X_1, t) d¦X + … - ¦χ(¦X_1, t) = \frac{\partial ¦χ}{¦X}(¦X_1, t) d¦X. $$

	$$ ®F(¦X, t) := \frac{\partial ¦χ}{¦X}(¦X_1, t) d¦X. \qquad d¦x = ®F d¦X $$

	\begin{poznamkain}
		It can be derived by derivatives on curves (see lecture).
	\end{poznamkain}
\end{definice}

\begin{dusledek}
	Transformation of infinitesimal line segment: $d¦x = ®F d¦X$.

	Transformation of infinitesimal surface elements: $d¦s = (\det ®F)®F^{-T} d¦S = \cof ®F d¦S$.

	Transformation of infinitesimal volume: $dv = (\det ®F) dV$.
\end{dusledek}

\begin{dusledek}[In tangent spaces]
	$$ F(¦X, t_0) = f(¦χ(¦X, t), t). $$

	Representation theorem:
	$$ (Grad F) ¦W = ¦U_{Grad F}·¦W $$
	$$ (Grad f) ¦w = ¦u_{Grad f} ·¦w $$
	$f(¦χ(¦X, t), t) = F(¦X, t_0)$
	$$ Grad f(¦x, t) |_{¦x = ¦χ(¦X, t)} = Grad F(¦X, t_0) $$
	$$ ¦U_{Grad F}·¦W = (Grad F)¦W = (Grad f)®F ¦W = (grad f)(®F ¦W) = ¦u_{grad f}·®F ¦W = ®F^T ¦u_{Grad f}·¦W. $$

	$¦u_{grad f} = ®F^{-T} ¦U_{Grad F}$.
\end{dusledek}

\begin{priklad}[Hollow cylinder]
	$r = f(R)$, $φ = Φ$, $z = Z$.

	\begin{reseni}
		$$ ®F = \frac{\partial χ_i}{\partial x_j} ¦e_i \otimes ¦E_j $$

		$$ X_1 = R \cos Φ, \qquad X_2 = R \sin Φ, x_1 = r \cos Φ, x_2 = r \sin Φ. $$
		$$ x_1 = χ_1(X_1, X_2, t), \qquad x_2 = ¦χ(x_1, x_2, t), x_i = χ_i(X_j, t). $$

		By chain rule:
		$$ \frac{\partial x_1}{\partial X_2} = \frac{\partial r \cos Φ}{\partial \partial X_2} = \frac{\partial}{\partial X_2} f(R) \cos Φ. $$

		$$ ®F = F_{rR}¦e_r \otimes ¦E_R + F_{rΦ} ¦e_r \otimes ¦E_Φ + … $$
	\end{reseni}

	\begin{reseni}
		From image:
		$$ ¦E_R \overset{®F}\rightarrow F_{rR} ¦e_r. $$
		$$ ¦E_{Φ} \overset{®F}\rightarrow F_{φΦ} ¦e_φ $$

		So $®F = \begin{pmatrix}F_{rR} & 0 \\ 0 & F_{φΦ} \end{pmatrix} $
	\end{reseni}

	TODO? (Solution by curve)
\end{priklad}

\begin{poznamka}
	How to differentiate in time tensorial quantities related to the current configuration?
	
	Upper convected derivative:
	$$ \dall{®A} (¦x, t) |_{¦x = ¦χ(¦X, t)} = \det ®F(¦X, t)\[\frac{d}{dt}\(®F^{-1}(¦X, t) ®A(¦χ(¦X, t), t) ®F^{-T}(¦X, t)\)\]®F^T(¦X, t). $$
\end{poznamka}

\subsection{Derivatives}
\begin{definice}[Lagragian velocity]
	$$ ¦V(¦X, t) = \frac{d¦χ(¦X, t)}{dt}. $$
	$$ ¦v(¦x, t)|_{¦x = ¦χ(¦X, t)} $$
\end{definice}

\begin{definice}[Eulerian velocity]
	$$ ¦v(¦x, t) = ¦V(¦X, t)|_{¦X = ¦χ^{-1}(¦x, t)}. $$
\end{definice}

\begin{definice}[Material time derivative]
	$\frac{d}{dt}=$ keep ¦X fixed, and differentiate with respect to time.
	$$ ψ(¦X, t) \rightarrow \frac{d}{dt} ψ(¦X, t) = \frac{\partial ψ}{\partial t}(¦X, t) $$
	$$ ψ(¦x, t) \rightarrow \frac{d}{dt} ψ(¦χ(¦X, t), t) = \frac{\partial ψ}{\partial t}|_{¦x = ¦χ(¦X, t)} + \frac{\partial ψ}{\partial x_i}(¦x, t)|_{¦x = ¦χ(¦X, t)} \frac{dχ_i}{dt}(¦X, t) = $$
	$$ = \(\frac{\partial ψ}{\partial t}(¦x, t)|_{¦x = ¦χ(¦X, t)} + V_i(¦X, t) \frac{\partial ψ}{\partial x_i}(¦x, t)|_{¦x = ¦χ(¦X, t)}\) = $$
	$$ = \(\frac{\partial ψ}{\partial t}(¦x, t) + v_i(¦x, t) \frac{\partial ψ}{\partial x_i}(¦x, t)\)|_{¦x = ¦χ(¦x, t)} $$

	$$ \frac{d}{dt}ψ(¦x, t) = \frac{\partial ψ}{\partial t}(¦x, t) + (¦v(¦x, t)·\nabla)ψ(¦x, t). $$
\end{definice}

\begin{definice}[Time derivative of deformation gradient ®F]
	$$ \frac{d}{dt} ®F (¦X, t) = \frac{d}{dt} \(\frac{\partial ¦χ(¦X, t)}{\partial ¦X}\) = \frac{\partial}{\partial ¦X} \frac{d ¦χ(¦X, t)}{dt} = \frac{\partial}{\partial ¦X}¦V(¦X, t) = $$
	$$ = \frac{\partial}{\partial ¦X} ¦v(¦χ(¦X, t), t) = \frac{\partial ¦v}{\partial ¦x}(¦x, t)|_{¦x = ¦χ(¦X, t)} \frac{\partial ¦χ}{\partial ¦X}(¦X, t) = \frac{\partial ¦v}{\partial x}|_{¦x = ¦χ(¦X, t)}®F(¦X, t). $$

	$$ ®L(¦x, t) := \nabla ¦V(¦x, t) = \frac{\partial ¦v}{\partial ¦x}(¦x, t). $$
\end{definice}

\begin{dusledek}
	$$ \frac{d®F}{dt} = ®L®F $$
\end{dusledek}

\begin{dusledek}
	$$ \dall{®A} = \frac{d®A}{dt} - ®L®A - ®A®L^T $$
\end{dusledek}

% 10. 11. 2023

TODO!!!

% 24. 11. 2023

\begin{poznamka}[Balance laws in Eulerian description (revision, the last lecture)]
	$$ \frac{dρ}{dt} + ρ \Div ¦v = 0; $$
	$$ ρ \frac{d¦v}{dt} = \Div ®T + ρ¦b, \qquad ®T = ®T^T; $$
	$$ ρ \frac{de}{dt} = ®T:®L - \Div ¦j_q; $$
	or
	$$ ρ \frac{d}{dt}(e + \frac{1}{2}¦v·¦v) = \Div(®T^T ¦v) + ρ¦b·¦v - \Div ¦j_q. $$
\end{poznamka}

\begin{poznamka}[Balance laws in Lagrangian description]
	Starting with $\frac{d}{dt}_{V(t)} ρ(¦x, t) dv = \frac{d}{dt} m_{V(t)} = 0$ i.e. mass remains same: $m_{V(t)} = m_{V(t_0)}$). We integrate over volume:
	$$ \int_{V(t_0)} ρ_R(¦X) dV = \int_{V(t)} ρ(¦x, t) dv = \int_{V(t_0)} ρ(¦x, t)|_{¦x = ¦χ(¦X, t)} \det ®F dV. $$
	Localization principle:
	$$ ρ(¦x, t)|_{¦x = ¦χ(¦X, t)} \det ®F = ρ_R(¦X). $$

	$$ \int_{V(t)} ρ \frac{d¦v}{dt} dv = \int_{V(t)} \Div ®T d¦v + \int_{V(t)} ρ¦b dv. $$
	$$ \int_{V(t)} \Div ®T d¦v = \int_{\partial V(t)} ®T ¦n\,ds = \int_{\partial V(t_0)} ®T(\det ®F)®F^{-T} ¦N\, dS \overset{\text{Stokes}} \int_{V(t_0)} \Div_{¦X}((det ®F) ®T ®F^{-T}) dV =: \int_{V(t_0)}(®T_R) dV. $$
	$$ ®T_R(¦X, t) := (\det ®F(¦X, t))®T(¦x, t)|_{¦x = ¦χ(¦X, t)} ®F^{-T}(¦X, t) $$
	is first Piola–Kirchhoff stress tensor. Cauchy ($®T$) is current $\rightarrow$ current. P–K ($®T_R$) is reference $\rightarrow$ current.
	$$ \int_{\partial V(t)} dv \rightarrow \int_{\partial V(t_0)} (\Div_{¦X} ®T_R) dV. $$
	$$ \int_{V(t)} ρ¦b dv = \int_{V(t_0)} \rho(¦x, t)|_{¦x = ¦χ(¦X, t)} ¦b(¦x, t)|_{¦x = ¦χ(¦X, t)} \det ®F dV = \int_{V(t_0)} ρ_R(¦X) ¦b dV. $$
	$$ \int_{V(t)} ρ \frac{d¦v}{dt} dv = \int_{V(t_0)} ρ \frac{\partial^2 ¦χ}{\partial t^2}(¦X, t) \det ®F dV = \int_{V(t_0)} ρ_R \frac{\partial^2 ¦χ}{\partial t^2} dV. $$
	Altogether:
	$$ ρ_R \frac{\partial^2 ¦χ}{\partial t^2} = \Div_{¦X} ®T_R + ρ_R¦b \qquad (\text{Solve for $¦χ$}). $$
	
	$®T = ®T^T \rightarrow ®T_R®F^T = ®F®T_R^T$ (P–K is not symmetric!).

	$$ ρ \frac{de}{dt} = ®T:®L - \Div ¦j_q \rightarrow \int_{V(t)} ρ \frac{de}{dt} dv = \int_{V(t)}®T:®L dv - \int_{V(t)}\Div ¦j_q dv. $$
	$$ \int_{V(t)} \Div ¦j_q dv = \int_{\partial V(t)} ¦j_q·¦n ds = \int_{\partial V(t_0)} ¦j_q(¦x, t)|_{¦x = ¦χ(¦X, t)} · \det ®F(¦X, t)®F^{-T}(¦X, t)¦N dS = $$
	$$ = \int_{\partial V(t_0)}(\det ®F(¦X, t)®F^{-1}(¦X, t)¦j_q(¦x, t))·¦N dS = $$
	$$ = \int_{V(t_0)} \Div((\det ®F)®F^{-1}¦j_q) dV. $$
	$¦J_q = (\det ®F)®F^{-1}¦j_q$ is called referential heat flux. (It cannot be given by Fouriers law ($¦j_q = k \nabla_{¦x} θ$, $\Div ¦j_q = \Div(k \nabla θ)$).)
	$$ \int_{V(t)} \underbrace{®T:\overbrace{®L}^{\nabla_{¦x} ¦v}}_{\tr(®T®L^T) = \tr(®L®T^T)} dv = \int_{V(t_0)} (\det ®F) ®T:®L dV = $$
	$$ = \int_{V(t_0)} \tr((\det ®F)®T®L^T) dV = \int_{V(t_0)} \tr\((\det ®F)®T®F^{-T}\(\frac{d®F}{dt}\)^T\)dV = \int_{V(t_0)} ®T_R : \dot{®F} dV. $$
	Altogether
	$$ ρ_R \frac{\partial e}{\partial t} = ®T_R:\dot{®F} - \Div_{¦X} ¦J_q. $$
\end{poznamka}

\section{Entropy}
\begin{poznamka}[Objective]
	Find quantity that is increasing/decreasing in time.
\end{poznamka}

\begin{poznamka}[With no interior]
	$$ ρ \frac{d}{dt}\(e + \frac{1}{2}¦v·¦v\) = \Div ®T + ρ¦b - \Div ¦j_q = \Div ®T + 0 + \Div(k \nabla θ). $$
	Let us work with\footnote{$®D_δ := ®D - \frac{1}{3} (\tr ®D)®I$. (Traceless part of ®D.)} $\Div ®T = -p_{th}®I + \tilde λ(\Div ¦v) + 2μ®D_δ$, and assume that $®T = -p_{th}(ρ, θ)®I + \tilde λ(\Div ¦v)®I + 2μ®D_δ$ (from $\frac{pV}{T} = \const$).

	$$ ρ \frac{d}{dt}(e + \frac{1}{2}¦v·¦v) = \Div(®T^T¦v) - \Div ¦j_q. $$
	$$ \frac{d}{dt} \int_{V(t)}ρ(e + \frac{1}{2}¦v·¦v) dv = \int_{\partial V(t)} ®T^T ¦v·¦n ds - \int_{\partial V(t)} ¦j_q·¦n ds. $$
	The first part is work and it is zero if we have boundary condition $¦v|_{\partial V} = 0$. The second part is heat exchange which is zero if we have boundary condition $¦j_q·¦n|_{\partial V} = 0$. Both boundary conditions together are math way to say system with \emph{no interactions}.

	$ρ, θ, ¦v \rightarrow ρ, e, θ$.

	Assume $η = η(ρ, e) \rightarrow e=e(η, ρ)$. (We will write $e = e(η, ρ) = e(η(¦x, t), ρ(¦x, t)) = e(¦x, t)$.)

	We have 1. Balance of internal energy
	$$ ρ\frac{de}{dt} = ®T:®D - \Div ¦j_q; $$
	2. Chain rule
	$$ ρ\frac{de}{dt} = \frac{\partial e}{\partial η}(η, ρ) \frac{dη}{dt} + \frac{\partial e}{\partial ρ}(η, ρ) \frac{dρ}{dt}. $$
	$$ ρ\frac{\partial e}{\partial η}(η, ρ) \frac{dη}{dt} = ®T:®D - \Div ¦j_q - \frac{\partial e}{\partial ρ}(η, ρ) \frac{dρ}{dt}, $$
	$$ ρ\frac{\partial e}{\partial η}(η, ρ) \frac{dη}{dt} = ®T:®D - \Div ¦j_q + \frac{\partial e}{\partial ρ}(η, ρ)ρ \Div ¦v, $$
	$$ ρ\frac{\partial e}{\partial η}(η, ρ) \frac{dη}{dt} = (-p_{th}®I + \tilde(\Div ¦v)®I + 2μ®D_δ):®D - \Div ¦j_q + \frac{\partial e}{\partial ρ}(η, ρ)ρ \Div ¦v, $$
	$$ ρ\frac{\partial e}{\partial η}(η, ρ) \frac{dη}{dt} = \(-p_{th} + ρ\frac{\partial e}{\partial ρ}(η, ρ)\) \Div ¦v + \tilde λ (\Div ¦v)^2 + 2μ®D_δ:®D_δ - \Div ¦j_q, $$
	$$ ρ \frac{dη}{dt} = \frac{\(-p_{th} + ρ\frac{\partial e}{\partial ρ}(η, ρ)\)}{\frac{\partial e}{\partial η}}\Div ¦v - \frac{\Div ¦j_q}{\frac{\partial e}{\partial η}} + \frac{\tilde λ (\Div ¦v)^2 + 2μ|®D_δ|^2}{\frac{\partial e}{\partial η}}. $$
	There is no chance that this could be positive. (Its obvious, because the value can flow, so point-wise $≥0$ is lost case.) But we can integrate over volume. Thus instead of $\frac{dη}{dt} ≥ 0$ we want just $\frac{d}{dt} \int_{V(t)} ρη dv ≥ 0$.
	$$ \frac{d}{dt} \int_{V(t)} ρη dv = \int_{V(t)}\frac{\(-p_{th} + ρ\frac{\partial e}{\partial ρ}(η, ρ)\)}{\frac{\partial e}{\partial η}}\Div ¦v dv - \int_{V(t)}\frac{\Div ¦j_q}{\frac{\partial e}{\partial η}}dv + \int_{V(t)} \frac{\tilde λ (\Div ¦v)^2 + 2μ|®D_δ|^2}{\frac{\partial e}{\partial η}} dv. $$
	The third integral OK, if $\frac{\partial e}{\partial η} > 0$.
	$$ \Div \(\frac{¦j_q}{\frac{\partial e}{\partial η}}\) = \frac{\Div ¦j_q}{\frac{\partial e}{\partial η}} + \nabla \(\frac{1}{\frac{\partial e}{\partial η}}\)·¦j_q. $$
	$$ \frac{d}{dt} \int_{V(t)} ρη dv = \int_{V(t)}\frac{\(-p_{th} + ρ\frac{\partial e}{\partial ρ}(η, ρ)\)}{\frac{\partial e}{\partial η}}\Div ¦v dv - \int_{V(t)}\Div \(\frac{¦j_q}{\frac{\partial e}{\partial η}}\)dv + \int_{V(t)} \nabla \(\frac{1}{\frac{\partial e}{\partial η}}\)·¦j_q dv + REST. $$
	The second integral is zero from Stokes and boundary condition $¦j_q·¦n|_{\partial V} = 0$. On the third integral, we can use derivative of inverse value:
	$$ \frac{d}{dt} \int_{V(t)} ρη dv = \int_{V(t)}\frac{\(-p_{th} + ρ\frac{\partial e}{\partial ρ}(η, ρ)\)}{\frac{\partial e}{\partial η}}\Div ¦v dv - \int_{V(t)} \frac{\nabla \(\frac{\partial e}{\partial η}\)·¦j_q}{\(\frac{\partial e}{\partial η}\)^2} dv + REST =  $$
	$$ = \int_{V(t)}\frac{\(-p_{th} + ρ\frac{\partial e}{\partial ρ}(η, ρ)\)}{\frac{\partial e}{\partial η}}\Div ¦v dv + k\int_{V(t)} \frac{\nabla \(\frac{\partial e}{\partial η}\)·\nabla θ}{\(\frac{\partial e}{\partial η}\)^2} dv + REST. $$
	If we set $\frac{\partial e}{\partial η}(η, ρ) = θ$, the second integral is non-negative. Moreover, for $θ ≥ 0$ we satisfy the assumption for the "first third integral". Moreover if we enforce $ρ^2 \frac{\partial e}{\partial ρ}(ρ, η) = p_{th}(θ, ρ)$, the first integral is zero, so we win.
\end{poznamka}

\begin{poznamka}
	Volíme si tedy $\tilde λ, μ > 0$.
\end{poznamka}

\begin{dusledek}
	$$ \frac{d}{dt} \int_{V(t)} ρ η dv ≥ 0 $$
	is granted for quantity that solves equations
	$$ e = e(η, ρ), \qquad \frac{\partial e}{\partial η} = θ, \qquad ρ^2 \frac{de}{dρ} = p_{th}(θ, ρ). $$
\end{dusledek}

\begin{priklad}
	$$ p_{th}(θ, ρ) = c_V(γ - 1)ρθ, \qquad e(θ, ρ) = c_V θ. $$
\end{priklad}

\begin{poznamka}[?]
	\ 
	\begin{enumerate}
		\item Energy is constant.
		\item Energy is function of entropy and volume.
		\item Entropy increases.
	\end{enumerate}
\end{poznamka}

\begin{poznamka}
	$e = e(η, ρ)$ is given $\rightarrow$ we know everything $θ = \frac{\partial e}{\partial η}(η, ρ)$, $p_{th} = ρ^2 \frac{\partial e}{\partial ρ}(η, ρ)$. (Warning: $e = e(ρ, θ)$ is not enough!)
\end{poznamka}

\begin{poznamka}
	Is there a better function that will allow us to do something like this?
\end{poznamka}

\begin{definice}[Helmholtz free energy density]
	$$ ψ(θ, ρ) := e(η, ρ)|_{η = η(θ, ρ)} - θη|_{η = η(θ, ρ)}. $$

	\begin{poznamka}
		This is le Legendre transformation of internal energy.
	\end{poznamka}
\end{definice}

\begin{dusledek}
	$$ \frac{\partial ψ}{\partial θ}(θ, ρ) = -η, \qquad \frac{\partial ψ}{\partial ρ}(θ, ρ) = \frac{\partial e}{\partial ρ}(η, ρ)|_{η = η(θ, ρ)} \quad \(= \frac{p_{th}}{ρ^2}\). $$
	
	\begin{dukazin}
		$$ \frac{\partial ψ}{\partial θ}(θ, ρ) = \frac{\partial e(η, ρ)}{\partial η}|_{η = η(θ, ρ)} \frac{\partial η}{\partial θ}(θ, ρ) - η|_{η = η(θ, ρ)} - θ \frac{\partial η}{\partial θ}(θ, ρ) = -η|_{η = η(θ, ρ)}. $$
		$$ \frac{\partial ψ}{\partial ρ}(θ, ρ) = \frac{\partial e}{\partial η}(η, θ) \frac{\partial η}{\partial ρ}(θ, ρ) + \frac{\partial e}{\partial ρ}(η, ρ)|_{η = η(θ, ρ)} - θ \frac{\partial η}{\partial ρ}(θ, ρ) = \frac{\partial e}{\partial ρ}(η, ρ)|_{η = η(θ, ρ)}. $$
	\end{dukazin}
\end{dusledek}

% 01. 12. 20

\begin{poznamka}[Why do we call $c_{V, ref}$ the specific heat at constant volume?]
	(Constant volume = constant density.) $¦j_q = -k\nabla θ$, $®T \approx -p_{th}®I$, $®D = \frac{1}{2}\((\nabla ¦v) + (\nabla ¦v)^T\)$.
	$$ ρ \frac{de}{dt} = ®T : ®D - \Div ¦j_q, $$
	$$ ρ \frac{\partial e}{\partial θ}(θ, ρ) \frac{dθ}{dt} + ρ \frac{\partial e}{\partial ρ} \frac{dρ}{dt} = -p_{th}(\underbrace{¦Div ¦v}_{\frac{dρ}{dt} + ρ\Div ¦v = 0}) + \Div(k \nabla θ), $$
	$$ ρ \frac{\partial e}{\partial θ}(θ, ρ) \frac{dθ}{dt} + 0 = 0 + \Div(k \nabla θ), $$
	$$ \int_V ρ \frac{\partial e}{\partial θ}(θ, ρ) \frac{dθ}{dt} dv = \int_{\partial V} (k \nabla θ)¦n ds. $$
	So in left we multiply $ρ$, difference of temperature and some $c_V(θ, ρ) := \frac{\partial e}{\partial θ}(θ, ρ)$. (On the right there is flow of heat, $¦j_q$, through boundary).
	
	For calorically perfect ideal gas: $e = e(ρ, θ) = c_{V, ref}·θ$.
\end{poznamka}

\begin{poznamka}[How to get specific heat at constant pressure?]
	$$ e = e(η(θ, p_{th}), ρ(θ, p_{th})), \qquad ρ \frac{de}{dt} = ®T:®D - \Div ¦j_q = -p_{th}·(\Div ¦v) + \Div(k \nabla θ). $$
	Chain rule:
	$$ \frac{de}{dt} = \frac{\partial e}{\partial η}(η, ρ) \frac{\partial η}{\partial θ}(θ, p_{th}) \frac{dθ}{dt} + \frac{\partial e}{\partial ρ}(η, ρ) \frac{dρ}{dt} + … \frac{d p_{th}}{dt} = θ \frac{\partial η}{\partial θ}(θ, p_{th}) \frac{dθ}{dt} + \frac{\partial e}{\partial ρ}(η, ρ) \frac{dρ}{dt}. $$
	$$ ρ θ \frac{\partial η}{\partial θ}(θ, p_{th}) \frac{dθ}{dt} + ρ\frac{\partial e}{\partial ρ}(η, ρ) \frac{dρ}{dt} = -p_{th}·(\Div ¦v) + \Div(k \nabla ¦θ), $$
	$$ ρ θ \frac{\partial η}{\partial θ}(θ, p_{th}) \frac{dθ}{dt} + ρ \frac{p_{th}}{ρ^2} (ρ·(\Div ¦v)) = -p_{th}·(\Div ¦v) + \Div(k \nabla ¦θ), $$
	$$ ρ θ \frac{\partial η}{\partial θ}(θ, p_{th}) \frac{dθ}{dt} = \Div(k \nabla ¦θ). $$
	So $c_p(θ, p_{th}) := θ \frac{\partial η}{\partial θ}(θ, p_{th})$ is specific heat at constant pressure.
\end{poznamka}

\begin{poznamka}[Alternative formula for the specific heat at constant volume]
	Chain rule:
	$$ θ \frac{\partial η}{\partial θ} (θ, ρ) = \frac{\partial e}{\partial η}(η, ρ) \frac{\partial η}{\partial θ}(θ, ρ) = \frac{\partial}{\partial θ} e(θ, ρ) = c_V(θ, ρ). $$
	$$ c_V(θ, ρ) := θ \frac{\partial η}{\partial θ}(θ, ρ). $$
\end{poznamka}

\begin{poznamka}[Another alternative formula for the specific heat at constant volume, for usage in practice]
	$$ c_V(θ, ρ) = θ \frac{\partial η}{\partial θ}(θ, ρ) = - θ \frac{\partial^2 ψ}{\partial θ^2} (θ, ρ), $$
	because $η(θ, ρ) = - \frac{\partial ψ}{\partial θ}(θ, ρ)$ (property of Helmholtz free energy).

	Conclusion: If $ψ(θ, ρ)$ is given, then
	$$ c_V(θ, ρ) = - θ \frac{\partial^2 ψ}{\partial θ^2} (θ, ρ), \qquad p_{th}(θ, ρ) = ρ^2 \frac{\partial ψ}{\partial ρ}(θ, ρ). $$
\end{poznamka}

\begin{poznamka}[Where are my evolution equations?]
	Unknowns: $¦v, ρ, θ$.
	$$ \frac{dρ}{dt} + ρ \Div ¦v = 0, $$
	$$ ρ \frac{d¦v}{dt} = \Div ®T + ρ¦b, \qquad ®T = -p_{th}(θ, ρ)®I + \tilde λ(\Div ¦v) ®I + 2μ ®D_δ. $$

	\begin{poznamkain}[Third equation]
	TODO!!!

		$$ ρ θ \frac{dη}{dt} = \Div(k \nabla θ) + \tilde λ(\Div ¦v)^2 + 2μ ®D_δ : ®D_δ. $$
		$$ \frac{dη}{dt} = - \frac{d}{dt}\(\frac{\partial ψ}{\partial θ}(θ, ρ)\) = - \frac{\partial^2 ψ}{\partial θ^2}(θ, ρ) \frac{dθ}{dt} - \frac{\partial^2 ψ}{\partial θ \partial ρ}(θ, ρ) \frac{dρ}{dt}. $$
		$$ θ c_V(θ, ρ) \frac{dθ}{dt} = - ρ^2 θ \frac{\partial^2 ψ}{\partial θ \partial ρ}(θ, ρ) \Div ¦v +  \Div(k \nabla θ) + \tilde λ(\Div ¦v)^2 + 2μ ®D_δ : ®D_δ, $$
		$$ θ c_V(θ, ρ) \frac{dθ}{dt} = - θ \frac{\partial}{\partial θ}\(ρ^2 \frac{\partial ψ}{\partial ρ}\) \Div ¦v +  \Div(k \nabla θ) + \tilde λ(\Div ¦v)^2 + 2μ ®D_δ : ®D_δ. $$
	\end{poznamkain}
	
	$$ θ c_V(θ, ρ) \frac{dθ}{dt} = - θ \frac{\partial}{\partial θ}\(ρ^2 \frac{\partial ψ}{\partial ρ}\) \Div ¦v +  \Div(k \nabla θ) + \tilde λ(\Div ¦v)^2 + 2μ ®D_δ : ®D_δ. $$

	And we can get $c_V(θ, ρ)$ and $p_{th}$ from $ψ = ψ(θ, ρ)$.
\end{poznamka}

\begin{poznamka}[Why do we call $γ$ the adiabatic exponent?]
	(This applies only to the ideal gas.)

	$$ p_{th} = c_{V, ref}(γ - 1) ρ θ, $$
	$$ ρ \frac{de}{dt} = ®T:®D - \Div ¦j_q = ®T:®D, \qquad ®T = -p_{th} ®I, $$
	$$ ρ \frac{de}{dt} = -p_{th} \Div ¦v, $$
	$$ e = c_{V, ref}·θ $$
	$$ ρ c_{V, ref} \frac{dθ}{dt} = \frac{p_{th}}{ρ} \frac{dρ}{dt} \impliedby \frac{dρ}{dt} + ρ \Div ¦v = 0. $$
	
	$$ \frac{dp_{th}}{dt} = c_{V, ref}(γ - 1) \frac{dρ}{dt} θ + c_{V, ref} (γ - 1) ρ \frac{dθ}{dt}, $$
	$$ \(ρ c_{V, ref} \frac{dθ}{dt} = \frac{1}{γ - 1} \frac{dp_{th}}{dt} - c_{V, ref} θ \frac{dρ}{dt}.\) $$
	$$ \frac{1}{γ - 1} \frac{dp_{th}}{dt} - c_{V, ref} θ \frac{dρ}{dt} = \frac{p_{th}}{ρ} \frac{dρ}{dt}, $$
	$$ \frac{1}{γ - 1} \frac{1}{p_{th}} \frac{dp_{th}}{dt} = \(\frac{c_{V, ref} θ}{p_{th}} + \frac{1}{ρ}\) \frac{dρ}{dt}, $$
	$$ \frac{1}{γ - 1} \frac{1}{p_{th}} \frac{dp_{th}}{dt} = \(\frac{1}{ρ}\(\frac{1}{γ - 1} + 1\)\) \frac{dρ}{dt}. $$
	$$ \(\frac{c_{V, ref} θ}{p_{th}} + \frac{1}{ρ} =? \frac{c_{V, ref}θ}{c_{V, ref} θ ρ (γ - 1)}.\) $$
	$$ \frac{d}{dt}(L_n p_{th}) = γ \frac{d}{dt} L_n ρ. $$
	$$ p_{th} = p_{th, ref} \(\frac{ρ}{ρ_{ref}}\)^γ. $$
\end{poznamka}

\begin{poznamka}[Where is my hyperbolic equation?]
	Assuming isentropic process ($η = \const$). $®T = -p_{th}®I = -p_{th}(ρ, η)®I$. ($\Div -p_{th}®I = -\nabla p_{th}$.)

	$ρ(¦x, t) = \hat{ρ} + \tilde ρ(¦x, t) =$ referential density + small perturbations, similarly $¦v(¦x, t) = \hat{¦v} + \tilde{¦v}(¦x, t) = \tilde{¦v}(¦x, t)$.
	$$ \(\frac{dρ}{dt} + ρ \Div ¦v = 0, \qquad ρ \frac{d¦v}{dt} = - \nabla p_{th}(ρ, η).\) $$
	$$ \frac{dρ}{dt} = \frac{\partial ρ}{\partial t} + (¦v · \nabla)ρ = \frac{\partial}{\partial t} (\hat{ρ} \tilde ρ) + (\hat{¦v} + \tilde{¦v})·\nabla (\hat{ρ} + \tilde ρ) \approx \frac{\partial \tilde ρ}{\partial t}. $$

	$$ \frac{d ρ}{dt} + ρ \Div ¦v = 0 \overset{\text{Linearize}}\longrightarrow \frac{\partial \tilde ρ}{\partial t} + \hat{ρ}\Div \tilde{¦v} = 0, $$
	$$ ρ\frac{d¦v}{dt} = -\nabla p_{th}(ρ, η) \overset{\text{Linearize}}\longrightarrow \hat{ρ} \frac{\partial \tilde{¦v}}{\partial t} = - \frac{\partial p_{th}}{\partial ρ}(ρ, η)|_{ρ = \hat{ρ}} \nabla \tilde ρ. $$

	Differentiate by time:
	$$ \frac{\partial^2 \tilde ρ}{\partial t} - \hat{ρ} \Div \(\frac{\partial \tilde{¦v}}{\partial t}\) = 0 $$
	$$ \frac{\partial^2 \tilde ρ}{\partial t^2} - \Div \(\frac{\partial p_{th}}{\partial ρ}\middle|_{ρ = \hat{ρ}} \nabla \tilde ρ\) = 0. $$
	$$ \frac{\partial^2 \tilde ρ}{\partial t^2} = \(\frac{\partial p_{th}}{\partial ρ}\middle|_{ρ = \hat{ρ}}\) Δ\tilde ρ. $$

	(Speed of sound: $c = \sqrt{\frac{\partial p_{th}}{\partial ρ}|_{ρ = \hat{ρ}}(ρ, η)}$.)
\end{poznamka}

\begin{poznamka}[Stability of test state]
	Set
	$$ ¦v = \hat{¦v} + \tilde{¦v}, \quad \hat{¦v} = 0, \qquad θ = \hat{θ} + \tilde θ, \quad \hat{θ} ≠ \hat{θ}(¦x, t), \qquad ρ = \hat{ρ} + \tilde ρ, \quad \hat{ρ} ≠ \hat{ρ}(¦x, t). $$
	Is it $\tilde{¦v}, \tilde ρ, \tilde θ \rightarrow 0$?

	$$ \frac{dρ}{dt} + ρ \Div ¦v = 0 \longrightarrow \frac{\partial \tilde ρ}{\partial t} + \hat{ρ} \Div \tilde{¦v} = 0, $$
	$$ ρ \frac{d¦v}{dt} = \Div ®T + ρ¦b = \Div ®T \longrightarrow \hat{ρ} \frac{\partial \tilde{¦v}}{\partial t} = - \frac{\partial p_{th}}{\partial ρ}|_{ρ = \hat{ρ}, θ = \hat{θ}} \nabla \tilde ρ - \frac{\partial p_{th}}{\partial θ}|_{ρ = \hat{ρ}, θ = \hat{θ}} \nabla \tilde θ + \Div \(\tilde λ (\Div \tilde{¦v})®I + 2μ \tilde{®D}_δ\), $$
	$$ ρ c_V \frac{dθ}{dt} = - θ \frac{\partial p_{th}}{\partial θ}(θ, ρ) \Div ¦v + \Div (k \nabla θ) + \tilde λ(\Div ¦v)^2 + 2μ ®D_δ:®D_δ = 0 \longrightarrow $$
	$$ \longrightarrow \hat{ρ} c_V|_{ρ=\hat{ρ}, θ = \hat{θ}} \frac{\partial \tilde θ}{\partial t} = \Div(k \nabla \tilde θ) - \hat{θ} \frac{\partial p_{th}}{\partial ρ} |_{ρ = \hat{ρ}, θ = \hat{θ}} \Div \tilde{¦v}. $$

	Test by $\tilde{¦v}$:
	$$ \int_V \hat{ρ} \frac{\partial \tilde{¦v}}{\partial t}·\tilde{¦v} = \int_V \(- \frac{\partial p_{th}}{\partial ρ} \nabla \tilde ρ - \frac{\partial p_{th}}{\partial θ} \nabla \tilde θ\)·\tilde{¦v} dv + \int_V \Div\(\tilde λ(\Div \tilde{¦v})®I + 2μ \tilde{®D}_δ\)·\tilde{¦v} dv = $$
	$$ = … + \int_V \Div\(\(\tilde λ(\Div \tilde{¦v})®I + 2μ \tilde{®D}_δ\)\tilde{¦v}\) - \int_V \(\tilde λ(\Div \tilde{¦v})^2 + 2μ\tilde{®D}_δ:\tilde{®D}_δ \Div ¦v\) = … + 0 - …. $$
	$$ \frac{d}{dt} \frac{1}{2} \int_V \hat{ρ}(\tilde{¦v})^2 dv = \int_V\(-\frac{\partial p_{th}}{\partial ρ} \nabla \tilde ρ - \frac{\partial p_{th}}{\partial θ} \nabla \tilde θ\)·\tilde{¦v} dv - \int_V \(\tilde λ (\Div \tilde{¦v})^2 + 2μ \tilde{®D}_δ:\tilde{®D}_δ\)dv. $$
	See that viscosity kills kinetic energy. Now we use $\nabla φ · \tilde{¦v} = \Div(φ \tilde{¦v}) - φ \Div \tilde{¦v}$ and on $\Div(…)$ we use Stokes theorem.
	$$ \frac{d}{dt} \frac{1}{2} \int_V \hat{ρ}(\tilde{¦v})^2 dv = \int_V \frac{\partial p_{th}}{\partial ρ}\tilde ρ \Div \tilde{¦v} dv + \int_V \frac{\partial p_{th}}{\partial θ} \tilde θ \Div \tilde{¦v} dv - \int_V \(\tilde λ (\Div \tilde{¦v})^2 + 2μ \tilde{®D}_δ:\tilde{®D}_δ\)dv. (1) $$

	$$ \frac{\partial \tilde ρ}{\partial t} = - \hat{ρ} \Div \tilde{¦v} \qquad /· \frac{\partial p_{th}}{\partial ρ} \frac{\tilde ρ}{\hat{ρ}}, \int_V $$
	$$ \frac{d}{dt} \int_V \frac{1}{2} (\tilde ρ)^2 TODO!!! dv = -\int_V \frac{\partial p_{th}}{\partial ρ} \tilde ρ \Div \tilde{¦v} dv. (2) $$

	$$ \hat{ρ} c_V|_{ρ=\hat{ρ}, θ = \hat{θ}} \frac{\partial \tilde θ}{\partial t} = \Div(k \nabla \tilde θ) - \hat{θ} \frac{\partial p_{th}}{\partial ρ} |_{ρ = \hat{ρ}, θ = \hat{θ}} \Div \tilde{¦v} \qquad /· \frac{\tilde θ}{\hat{θ}}, \int_V $$
	$$ \frac{d}{dt} \int_V \frac{\hat{ρ} c_V}{\hat{θ}}(\tilde θ)^2 dv = - \frac{1}{\hat{θ}} \int_V \nabla \tilde θ · \nabla \tilde θ dv - \int_V \frac{\partial p_{th}}{\partial ρ}|_{ρ = \hat{ρ}, θ = \hat{θ}} (\Div \tilde{¦v})\tilde θ dv. (3) $$

	$(1) + (2) + (3)$:
	$$ \frac{d}{dt} \int_V \frac{1}{2} \hat{ρ}(\tilde{¦v})^2 + \frac{1}{2} \frac{1}{\hat{ρ}}\frac{\partial p_{th}}{\partial ρ}|_{ρ = \hat{ρ}, θ = \hat{θ}}(\tilde ρ)^2 + \frac{\hat{ρ} c_V}{\hat{θ}}|_{ρ = \hat{ρ}, θ = \hat{θ}} (\tilde θ)^2 = $$
	$$ = - \frac{1}{\hat{θ}} \int_V \nabla \tilde θ \nabla \tilde θ dv - \int_V \tilde λ (\Div \tilde{¦v})^2 + 2μ\tilde{®D}_δ : \tilde{®D}_δ dv. $$o

	When we rightly choose Helmholtz free energy, $\frac{\partial p_{th}}{ρ, θ} > 0$ and $c_V$ will be positive and we win (perturbations go to zero).

	Z $c_V > 0$ máme $\frac{\partial^2 ψ}{\partial θ^2} < 0$.

	TODO!!!
\end{poznamka}

% 08. 12. 2023

\begin{poznamka}[For Homework]
	$\frac{d®H}{dt} ≠ \frac{1}{2} ®B^{-1} \frac{d®B}{dt}$, because there is no commutativity. We must use $®T:…$.
\end{poznamka}

\begin{poznamka}[*]
	$®T = \underbrace{-p_{th}(ρ, τ) I} + \underbrace{\tilde λ(\Div ¦v) + 2μ ®D_δ}_{\text{Why this?}}$.

	$$ ®T = -p_{th}(ρ, θ)®I + ®S(¦v). $$
	Now we use Galilei principle of relativity „$¦x = ¦x + ¦w t$“. We see that this don't work. So we use $\nabla ¦v$ ($\nabla (¦v + ¦w) = \nabla ¦v + \nabla ¦w = \nabla ¦v$):
	$$ ®T = -p_{th}(ρ, θ)®I + ®S(\nabla ¦v). $$
	®T is symmetric matrix, so:
	$$ ®T = -p_{th}(ρ, θ)®I + ®S(®D). $$
\end{poznamka}

\subsection{Representation theorems for isotropic functions}
\begin{definice}
	We say that $φ: ®R^{3 \times 3} \rightarrow ®R$ is isotropic $≡$ $φ(®Q®A®Q^T) = φ(®A)$ holds for any $®Q \in Orth^+$ (= proper orthogonal matrices, i.e. $®Q®Q^T = ®I$, $\det ®Q > 0$).

	\begin{prikladyin}
		$\tr$ is isotropic, $®A \mapsto a_{11}$ is not isotropic.
	\end{prikladyin}

	We say that $®F: ®R^{3 \times 3} \rightarrow ®R^{3 \times 3}$ is isotropic $≡$ $©F(®Q®A®Q^T) = ®Q®F(®A)®Q^T$ holds for any $®Q \in Orth^+$.

	\begin{prikladyin}
		$\id$, $^{-1}$, $\exp$, $\ln$, … is isotropic. $®A \mapsto a_{ii}®I$ is not isotropic.
	\end{prikladyin}
\end{definice}

\begin{veta}
	$φ: Sym(®R^{3 \times 3}) \rightarrow ®R$, $φ$ isotropic $\implies$ $φ(®A) = φ(I_1(®A), I_2(®A), I_3(®A))$.
	
	$f: Sym(®R^{3 \times 3}) \rightarrow Sym(®R^{3 \times 3})$, $f$ isotropic $\implies$ $f(®A) = α_0 ®I + α_1 ®A + α_2 ®A$, where $α_i = α_i(I_1(®A), I_2(®A), I_3(®A))$.

	\begin{poznamkain}
		The second one goes from: Assume $f(®A) = \sum_{i=0}^{+∞} f_i ®A^i$. Then by Cayley–Hamilton $f(®A) = \sum_{i=0}^{2} f_i ®A^i$. Furthermore, $f_i(®A)$ must be isotropic…
	\end{poznamkain}
\end{veta}

\begin{poznamka}[Continuation of *]
	Isotropic fluid: $®S(®D)$ is isotropic function, $®S(®D) = α_0®I + α_1®D + α_2®D^2$. What if we want Linear relation (i.e. $®S(®D)$ is linear function of ®D)? Then $®S(®D) = c_0(\tr ®D)®I + c_1®D$, where $c_0, c_1 = \const$.
\end{poznamka}

\begin{upozorneni}
	This remarks (not theorem) works only for fluids, where 0 velocity means 0 stress.
\end{upozorneni}

\begin{poznamka}[Governing equations for incompressible isotropic fluids]
	TODO?

	\begin{align*}
		\Div ¦v &= 0\\
		ρ \frac{d¦v}{dt} &= -\nabla p + \Div(2μ®D) + ρ¦b\\
		ρ c_V \frac{dθ}{dt} = …
	\end{align*}

	So
	$$ ®T = - p®I + 2μ®D, \qquad \Div ®T = -\nabla p + \Div(2μ®D). $$

	The kind of this $p$ is other than $p_{th}$. This is (artificial) pressure maintaining the incompressibility, and we solve! for it. ($p_{th}$ is function of $ρ$ and $θ$). ®T is not obtained by simply substitution, but from equations above!
\end{poznamka}

\begin{priklad}[Archimedes law]
	Any object, wholly or partially immersed in a fluid, is buoyed up by a force equal to the weight of the fluid displaced by the object.

	\begin{poznamkain}
		Fluid = incompressible fluid. So $®T = -p ®I + 2μ®D$.
	\end{poznamkain}

	\begin{poznamkain}
		Weight = body force = gravitation force
	\end{poznamkain}

	\begin{poznamkain}
		It talks about floating, so nothing moves!
	\end{poznamkain}

	\begin{poznamkain}[Governing equations]
		\begin{align*}
			\Div ¦v &= 0,\\
			ρ \frac{d¦v}{dt} &= \Div ®T + ρ ¦b, \qquad ¦b = -g¦e_{\hat{z}},\\
			®T &= -p®I + 2μ®D,\\
			¦F &= \int_{\partial ©B} ®T ¦n ds.
		\end{align*}
	\end{poznamkain}

	\begin{reseni}
		1. Solve for ¦v and $p$.

		2. Evaluate ¦F.

		„1.“ is easy, we have static problem, so $¦v = 0$.
		$$ ρ \frac{d¦v}{dt} = \Div ®T + ρ ¦b \land \Div  ®T = -\nabla p + \Div(2μ®D) \implies \implies 0 = -\nabla p - ρ_{fluid}g ¦e_{\hat{z}} \implies $$
		$$ \implies \(\frac{\partial p}{\partial x}, \frac{\partial p}{\partial y}, \frac{\partial p}{\partial z}\) = (0, 0, -ρ g) \implies p = -ρ g z + p_0. $$

		„2.“:
		$$ \int_{\partial ©B} ®T ¦n ds = \int_{\partial ©B}-p®I¦n ds = \int_{\partial ©B} (ρ_{fluid} g z - p_0) = \int_{\partial ©B} (ρ_{fluid} g z¦n)ds - p_0 \int_{\partial ©B}¦n ds =: *. $$
		By stokes (for constant $¦w$)
		$$ ¦w·\int_{\partial ©B}¦n ds = \int_{\partial ©B}¦w·¦n = \int_{©B} \Div ¦w dv = 0. $$
		So $\int_{\partial ©B}¦n ds = 0$. Continue:
		$$ * = \int_{\partial ©B} (ρ_{fluid} g z ¦n) ds, $$
		for constant ¦w:
		$$ ¦F·¦w = \int_{\partial ©B} (ρ_{fluid} g z ¦w)·¦n ds = \int_{©B} ρ_{fluid} g \Div (z ¦w) dv + \int_{©B} ρ_{fluid} g ((\nabla z)·¦w + z\Div ¦w dv = ρ_{fluid} g \int_{©B} 1dv ¦e_{\hat{z}}·¦w. $$
	
		So $¦F = ρ_{fluid} g V ¦e_{\hat{z}}$.
	\end{reseni}
\end{priklad}

\begin{priklad}[Stability of flow in a container]
	$¦v|_{\partial Ω} = ¦o$, $®T = -p®I + 2μ®D$.

	$t \rightarrow +∞ \implies ¦v \rightarrow ¦o$.

	\begin{prikladin}
		$$ \Div ®T = -\nabla p + μΔ¦v. $$
	\end{prikladin}

	\begin{poznamkain}[Governing equations]
		\begin{align*}
			\Div ¦v &= 0,\\
			ρ \frac{d¦v}{dt} &= \Div ®T = -\nabla p + μ Δ ¦v,\\
			¦v|_{\partial Ω} &= ¦o,\\
			¦v|_{t_0} &= ¦v_0.
		\end{align*}
	\end{poznamkain}

	\begin{reseni}
		Test equation by solution:
		$$ \int_Ω ρ \frac{d¦v}{dt}·¦v = \int_Ω (-(\nabla p)·¦v + μ(Δ¦v)·¦v) dv. $$
		$$ ρ \int_Ω \((¦v·\nabla)¦v\)·¦v dv = ρ \int_Ω v_i \frac{\partial v_j}{\partial x_i} v_j dv = \frac{1}{2}ρ \int_Ω v_i \frac{\partial}{\partial x_i}(v_j)^2 dv = $$
		$$ = ρ \in_Ω ¦v·\nabla \(\frac{|¦v|^2}{2}\)dv = ρ \int_Ω \Div(¦v \frac{|¦v|^2}{2}) dv = ρ \int_{\partial Ω} \frac{|¦v|^2}{2}(¦v·¦n)ds = 0. $$
		$$ \frac{d}{dt} \int_{©B} \frac{1}{2} ρ(¦v)^2 dv = … - μ\int_{©B} (\nabla ¦v):\nabla ¦v dv. $$
		Poincaré inequality (we live in Dirichlet zero $\impliedby$ boundary conditions):
		$$ \int |\nabla ¦v|^2 ≤ C_p^2 \int |\nabla ¦v|^2. $$
		$$ \frac{1}{2} ρ \frac{d}{dt} \int_Ω ≤ - \frac{μ}{C_p} \int_Ω |¦v|^2 dv $$
		$$ \frac{d}{dt}(\|¦v\|^2_{L^2(Ω)}) ≤ \frac{-2μ}{ρC_p} \|¦v\|_{L^2(Ω)}^2. $$
	\end{reseni}
\end{priklad}

\begin{priklad}
	$$ \Div ¦v = 0 \qquad ρ \frac{d¦v}{dt} = -\nabla p + μ Δ¦v. $$

	\begin{poznamkain}[Dimensionless form]
		$l_{char}$ = Characteristic length = arbitrary chosen length. $v_{char}$ = characteristic velocity. $t_{char} = l_{char} / v_{char}$.

		Dimensionless variables: $¦x^* = ¦x / l_{char}$, $t^* = t / t_{char}$, $¦v^* = ¦v / v_{char}$.

		$$ \Div^* ¦v^* = 0. $$
		$$ ρ \frac{v_{char}}{t_{char}} \frac{d¦v^*}{dt^*} = - \frac{1}{l_{char}}(\nabla^* p^*)p_{char} + \frac{μ}{l_{char}^2}(Δ^*¦v^*)v_{char}, $$
		$$ \frac{d¦v^*}{dt^*} = \frac{t_{char}}{ρ v_{char} l_{char}}(\nabla^* p^*)p_{char} + \frac{μv_{char}}{l_{char}^2} \frac{t_{char}}{ρ v_{char}} Δ^* ¦v^*, $$
		$$ \frac{dv^*}{dt^*} = - \frac{p_{char}}{ρv_{char}^2}(\nabla^* p^*) + \frac{μ}{ρ l_{char} v_{char}} Δ^* ¦v^*. $$

		$p_{char} := ρv_{char}^2$
		$$ \implies \frac{d¦v^*}{dt^*} = -\nabla^* p^* + \frac{1}{Re}Δ^* ¦v^*, $$
		where $\frac{1}{Re} := \frac{μ}{ρ l_{char} v_{char}}$ is Reinold's number.
	\end{poznamkain}
\end{priklad}

% 15. 12. 2023

\section{Solids}

TODO!!!


\begin{priklad}
	TODO!!!

	\begin{reseni}
		TODO!!!



		$$ 0 = \Div \btau = \begin{pmatrix} \frac{\partial τ_{\hat{r}\hat{r}}}{\partial r} + \frac{1}{r} \(\frac{\partial τ_{\hat{r}\hat{φ}}}{\partial φ} - τ_{\hat{φ}\hat{φ}} + τ_{\hat{r}\hat{r}}\) + \frac{\partial τ_{\hat{r}\hat{z}}}{\partial z} \\ \frac{\partial τ_{\hat{φ}\hat{r}}}{\partial r} + \frac{1}{r} \(\frac{\partial τ_{\hat{φ}\hat{φ}}}{\partial φ} + τ_{\hat{r}\hat{φ}} + τ_{\hat{φ}\hat{r}}\) + \frac{\partial τ_{\hat{φ}\hat{z}}}{\partial z} \\ \frac{\partial τ_{\hat{r}\hat{r}}}{\partial r} + \frac{1}{r} \(\frac{\partial τ_{\hat{z}\hat{φ}}}{\partial φ} τ_{\hat{z}\hat{r}}\) + \frac{\partial τ_{\hat{z}\hat{z}}}{\partial z} \end{pmatrix}, $$
		where $\btau = \begin{pmatrix} τ_{\hat{r}\hat{r}} & τ_{\hat{r}\hat{φ}} & τ_{\hat{r}\hat{z}} \\ & τ_{\hat{φ}\hat{φ}} & τ_{\hat{φ}\hat{z}} \\ & & τ_{\hat{z}\hat{z}} \end{pmatrix}$. So $\btau = \begin{pmatrix} 0 & 0 & 0 \\ & 0 & 0\\ & & τ_{\hat{z}\hat{z}} \end{pmatrix} =: \begin{pmatrix} 0 & 0 & 0 \\ 0 & 0 & 0\\ 0 & 0 & T \end{pmatrix}$.

		Thus $\left.\begin{pmatrix} 0 & 0 & 0 \\ 0 & 0 & 0\\ 0 & 0 & T \end{pmatrix}\middle|_{z = L}\begin{pmatrix} 0 \\ 0 \\ 1 \end{pmatrix} = \frac{1}{S} \begin{pmatrix} 0 \\ 0 \\ F \end{pmatrix}\right.$, so $T = \frac{F}{S}$.
		$$ \btau = λ(\tr \bepsilon) ®I + 2μ\bepsilon. \tr \btau = (3λ + 2μ)\tr \bepsilon \implies \tr \bepsilon = \frac{\tr \btau}{3λ + 2μ}. $$
		$$ \bepsilon = f(\btau). \qquad \bepsilon = \frac{1}{2μ} \(\btau - \frac{λ}{3λ + 2μ}(\tr \btau) ®I\). $$

		$$ RHS = \frac{1}{2μ} (\bepsilon - \frac{λ}{3λ + 2μ}(\tr \btau)®I) = \begin{pmatrix} -\frac{λ}{2μ(3λ + 2μ)}\frac{F}{S} & 0 & 0 \\ 0 & -\frac{λ}{2μ(3λ + 2μ)}\frac{F}{S} & 0 \\ 0 & 0 & \frac{λ + μ}{μ(3λ + 2μ)}\frac{F}{S} \end{pmatrix}. $$
		$$ LHS = \frac{1}{2}(\nabla  ¦U + (\nabla ¦U)^T). $$

		Symmetric gradient in cylindrical coordinate system is another long formula, so expect: $¦U = (U_{\hat{r}}(r), 0, U_{\hat{z}}(z))^T$.
		$$ \frac{1}{2} \nabla ¦U + \frac{1}{2}(\nabla ¦U)^T = \begin{pmatrix} \frac{dU_{\hat{r}}}{dr} & 0 & 0 \\ 0 & \frac{U_{\hat{r}}}{r} & 0 \\ 0 & 0 & \frac{dU_{\hat{z}}}{dz} \end{pmatrix}. $$

		Together: $U_{\hat{r}} = -\frac{λ}{2μ(3λ + 2μ)}·\frac{F}{S}·r$, $U_{\hat{z}} = \frac{λ + μ}{μ(3λ + 2μ)}·\frac{F}{S}·z$.

		Length of cylinder: $ε := \frac{ΔL}{L} = \frac{¦U|_{z = L}}{L} = \frac{\frac{λ + μ}{μ(3λ + 2μ)}·\frac{F}{S}·L}{L} = \frac{λ + μ}{μ(3λ + 2μ)}·\frac{F}{S} =: \frac{1}{E}ς$, where $ς = Eε$ is Hooke law, where $E$ is Young modulus, $ς$ is „applied force“ and $ε$ is change of length.

		Change of radius:
		$$ - \frac{\frac{ΔR}{R}}{\frac{ΔL}{L}} = - \frac{\frac{-\frac{λ}{2μ(3λ + 2μ)}·\frac{F}{S}·R}{R}}{\frac{ΔL}{L}} = \frac{\frac{λ}{2μ(3λ + 2μ)}}{\frac{λ + μ}{μ(3λ + 2μ)}} = \frac{λ}{2(λ + μ)} =: ν. $$
	\end{reseni}

	\begin{poznamkain}
		This is easy to measure -> we can measure Young modulus $E := \frac{μ(3λ + 2μ)}{λ + μ}$ and Poisson ration $ν := \frac{λ}{2(λ + μ)}$.
	\end{poznamkain}
\end{priklad}

\begin{poznamka}
	Young modulus is certainly positive. However, Poisson ration can be negative. (Move $V$s horizontally in $\mathrm{V}\!Λ\!\mathrm{V}\!Λ\!\mathrm{V}\!Λ\!\mathrm{V}\!Λ\!\mathrm{V}\!Λ\!\mathrm{V}\!Λ$.)
\end{poznamka}

\begin{poznamka}[Fixing negativity of constants (Poisson ration)]
	$$ \btau = \frac{1}{2μ}\(\btau - \frac{λ}{3λ + 2μ} (\tr \btau)®I\), \qquad \btau = λ(\tr <BS>\bepsilon)®I + 2μ\bepsilon. $$
	$$ \bepsilon = \frac{1}{2μ} \(\(\btau - \frac{1}{3} (\tr \btau)®I\) + \(\frac{1}{3} - \frac{λ}{3λ + 2μ}\)(\tr \btau)®I\). $$
	$$ \bepsilon = \frac{1}{2μ} \btau_δ + \frac{1}{9(λ + \frac{2}{3}μ)} (\tr \btau) ®I. $$

	Experiment: compress material. Then we see, that $K := λ + \frac{2}{3}μ$ must be positive. It is called bulk modulus (related to change of volume). $G := μ > 0$ is shear modulus.
\end{poznamka}

\begin{dusledek}
	$$ λ = \frac{E ν}{(1 + ν)(1 - 2ν)}, \qquad μ = \frac{E}{2(1 + ν)}, \qquad K = \frac{E}{3(1 - 2ν)}, $$
	$$ μ = \frac{E}{2(1 + ν)}, \qquad E = \frac{9Kμ}{μ + 3K}, \qquad ν = \frac{3K - 2μ}{2(3K + μ)}. $$

	$$ \implies -1 < ν < \frac{1}{2}. $$
	$ν = \frac{1}{2}$ means incompressible (solid) material (= no $\bepsilon$).
\end{dusledek}

\begin{definice}
	Spherical stress: $\btau = \begin{pmatrix} τ & 0 & 0 \\ 0 & τ & 0 \\ 0 & 0 & τ \end{pmatrix}$. Sheer stress: $\btau = \begin{pmatrix} 0 & τ & 0 \\ τ & 0 & 0 \\ 0 & 0 & 0 \end{pmatrix}$.
\end{definice}

\subsection{Elastic materials}
\begin{definice}
	Elastic material is a material that does not produce entropy. No energy loss/gain in cyclic mechanical processes.

	We have $®T = ®T(®B) = α_0 ®I + α_1 ®B + α_2 ®B^2$ (for solids, isotropic solids) (hence $®T®B = ®B®T$), $e(η, ρ)$, $ψ(θ, ρ)$, and from $ρ \det ®F = ρ_R$ ($ρ(\det ®B)^{1 / 2} = ρ_R$), we get $ψ(θ, \det ®B)$. So $ψ = ψ(θ, ®B) = ψ(θ, I_1(®B), I_2(®B), I_3(®B))$.

	From $ρ \frac{de}{dt} = ®T:®D - \Div ¦j_q$ we get $ψ(θ, ®B) = (e(η, ®B) - θη)|_{η = η(θ, ®B)}$.
	$$ \frac{\partial ψ}{\partial ®B}:\frac{d®B}{dt} + \frac{\partial ψ}{\partial θ}\frac{dθ}{dt} = \frac{de}{dt} - \frac{dθ}{dt}η - θ·\frac{dη}{dt}. $$
\end{definice}

\begin{dusledek}[$\dall{®B} = ®O$]
	% $$ ®B^{-1}\frac{d®B}{dt} - ®L - ®B®L^T®B^{-1} = ®O. $$
	$$ \frac{de}{dt} = \frac{\partial ψ}{\partial ®B}(θ, ®B):\frac{d®B}{dt} + θ \frac{dη}{dt}, $$
	$$ ρθ\frac{dη}{dt} - ρ \frac{\partial ψ}{\partial ®B}:\frac{d®B}{dt}. $$
	$$ ρ \frac{de}{dt} = ®T:®D - \Div ¦j_q \rightarrow ρθ \frac{dη}{dt} = ®T:®D - \Div ¦j_q - ρ \frac{\partial ψ}{\partial ®B}:\frac{d®B}{dt} = $$
	$$ = \(®T:®D - ρ \frac{\partial ψ}{\partial ®B}:(®L®B + ®B®L^T)\) - \Div ¦j_q = ®T:®D - 2ρ ®B \frac{\partial ψ}{\partial ®B} : ®D - \Div ¦j_q. $$
	$$ \frac{\partial ψ}{\partial ®B} ®B = ®B \frac{\partial ψ}{\partial ®B}, $$
	because $ψ$ is isotropic function of ®B. Thus we get evolution equation for entropy.
	$$ ρθ \frac{dη}{dt} = \(®T - 2ρ ®B \frac{\partial ψ}{\partial ®B}\) : ®D - \Div ¦j_q. $$
\end{dusledek}

\begin{dusledek}
	No entropy production can be done (for all processes at once) "only" by setting $®T = 2ρ ®B \frac{\partial ψ}{\partial ®B}$.
\end{dusledek}

\begin{dusledek}
	$$ 0 = \(\int_V ρ ψ(θ, ®B)\)|_{t_{end}} - \(\int_V ρ ψ(θ, ®B)\)|_{t_{start}} = $$
	$$ = \int_{t_{start}}^{t_{end}} \(\int_V ®T:®D dv\)dt = \int_{t_{start}}^{t_{end}} \int_V 2ρ ®B \frac{\partial ψ}{\partial ®B}: ®D dv dt = $$
	$$ = \int_{t_{start}}^{t_{end}} \int_V \(ρ \frac{\partial ψ}{\partial ®B}\frac{d®B}{dt} dv\) dt = \int_{t_{start}}^{t_{end}} \frac{d}{dt} \int_V ρ ψ dv dt. $$
\end{dusledek}

% 05. 01. 2024

TODO!!!

\end{document}
