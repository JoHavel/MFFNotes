\documentclass[12pt]{article}					% Začátek dokumentu
\usepackage{../../MFFStyle}					    % Import stylu



\begin{document}

% 06. 10. 2023

TODO!!!

% 13. 10. 2023

\begin{definice}[Dot product on the space of matrices]
	$$ ®A : ®B = \tr(®A ®B^T). $$
\end{definice}

\begin{definice}[Norm of matrix]
	$$ |®A| = (®A : ®A)^{\frac{1}{2}}. $$
\end{definice}

\begin{priklad}
	$$ (¦a \otimes ¦b)^T = ¦b \otimes ¦a. $$

	\begin{dukazin}
		$$ ¦u·(¦a \otimes ¦b)^T ¦v = (¦a\otimes¦b)¦u·¦v = (¦a(¦b·¦u))¦v = (¦b·¦u) (¦a·¦v) = ¦u·(¦b(¦a·¦v)) = ¦u·(¦b \otimes ¦a)¦v. $$
	\end{dukazin}
\end{priklad}

\begin{priklad}
	$$ \det(e^{®A}) = e^{\tr ®A}. $$

	\begin{dukazin}
		$$ e^{®A} = \lim\(®I + \frac{®A}{n}\)^n. $$
		$$ \det e^{®A} = \lim_{n \rightarrow ∞} \(\det \(®I + \frac{®A}{n}\)^n\) = \lim_{n \rightarrow ∞} \(\det\(®I + \frac{®A}{n}\)\)^n = ? $$
		Subtask: Is there an approximation for $\det(®I + ®S)$, where $®S$ is a „small“ matrix. Yes, we did it (KontinuumDU1.pdf) for $®S \in ®R^{3 \times 3}$:
		$$ \det(®I + ®S) = \det ®I + \tr(®I \cof ®S) + \tr(®S^T \cof ®I) + \det ®S \approx 1 + \tr(®S^T \cof ®I) + o(®S^2) = 1 + \tr(S) + o(®S^2). $$
		And for $®S \in ®R^{n \times n}$, one can see that:
		$$ \det(®I + ®S) = \begin{pmatrix} 1 + s_{11} & s_{12} & … & s_{1n} \\ s_{21} & 1 + s_{22} & … & s_{2n} \\ \vdots & \vdots & \ddots & \vdots \\ s_{n1} & s_{n2} & … & 1 + s_{nn} \end{pmatrix} = (1 + s_{11})(1 + s_{22})·…·(1 + s_{nn}) + o(®S^2) = $$
		$$ = 1 + s_{11} + s_{22} + … + s_{nn} + o(®S^2) = 1 + \tr ®S + o(®S^2). $$
		$$ ? = \lim_{n \rightarrow ∞} \(1 + \frac{\tr ®A}{n} + …\)^n = e^{\tr ®A}. $$
	\end{dukazin}
\end{priklad}

\begin{tvrzeni}
	$$ \det(®I + ®S) = 1 + \tr ®S + … $$
\end{tvrzeni}

\begin{definice}[Gateaux derivative]
	$$ D¦f(¦x)[¦y] := \frac{d}{dτ} ¦f(¦x + τ¦y)|_{τ=0}. $$
\end{definice}

\begin{definice}[Fréchet derivative]
	$¦f: U \rightarrow V$:
	$$ \lim_{\|¦y\|_U \rightarrow 0} \frac{\|¦f(¦x + ¦y) - ¦f(¦x) - D¦f(¦x)[¦y]\|_V}{\|¦y\|_V} = 0. $$

	\begin{poznamkain}
		Sometimes we write $\nabla f(¦x)·¦y$ instead of $Df(¦x)[¦y]$ (from Riesz representation theorem).
	\end{poznamkain}

	For matrices ($φ: ®A \in ®R^{3\times 3} \rightarrow ®R$):
	$$ \frac{\|φ(®A + ®B) - φ(®A) - Dφ(®A)[®B]\|_{®R}}{\|®B\|_{®R^{3 \times 3}}}. $$

	\begin{poznamkain}
		We write $\frac{\partial φ}{\partial ®A}(®A):®B$ instead of $Dφ(®A)[®B]$, where $\frac{\partial φ}{\partial ®A}(®A)$ is right matrix. Warning $\frac{\partial φ}{\partial ®A}(®A) ≠ Dφ(®A)$, because of transposition ($®A:®B = \tr(®A®B^T) = \tr(®A^T®B)$).
	\end{poznamkain}
\end{definice}

\begin{priklad}
	$$ \frac{\partial \tr ®A}{\partial ®A}(®A)[®B] = \frac{d}{dτ}(\tr(®A + τ®B))|_{τ = 0} = \frac{d}{dτ}\(\tr ®A + τ\tr ®B\)|_{τ = 0} = \tr ®B = ®I:®B. $$
	So $\frac{\partial \tr ®A}{\partial ®A} = ®I$.
\end{priklad}

\begin{priklad}
	$$ \frac{\partial \det ®A}{\partial ®A}(®A)[®B] = \frac{d}{dτ}(\det(®A + τ®B))|_{τ = 0} = \frac{d}{dτ}\(\det(®A)·\det\(®I + τ®A^{-1}®B\)\)|_{τ = 0} = $$
	$$ = \frac{d}{dτ}\((\det ®A)·\(1 + τ\tr(®A^{-1}®B) + o(τ^2)\)\)|_{τ = 0} = (\det ®A) \tr\(®A^{-1}®B\) = $$
	$$ = (\det ®A)\tr\(\(®A^{-T}\)^T®B\) = \((\det ®A)®A^{-T}\):®B. $$
	So $\frac{\partial \det ®A}{\partial ®A} = (\det ®A)®A^{-T} = \cof(®A)$.
\end{priklad}

\begin{priklad}
	$®A: ®R \rightarrow ®R^{3 \times 3}$.
	$$ \frac{d}{dt}(\det ®A(t)) = (\det ®A(t)) \tr\(®A(t)^{-1} \frac{d®A(t)}{dt}\). $$
\end{priklad}

\begin{priklad}
	$®F: ®A \in ®R^{3 \times 3} \rightarrow ®F(®A) \in ®R^{3 \times 3}$. $®F(®A) = ®A^{-1}$. (We know $\frac{1}{1 + x} = 1 - x + …$.)
	$$ \frac{\partial ®F(®A)}{\partial ®A}(®A)[®B] = \frac{d}{dτ}\((®A + τ®B)^{-1}\)|_{τ = 0} = \frac{d}{dτ}\(\(®A\(®I + τ®A^{-1}®B\)\)^{-1}\)|_{τ = 0} = $$
	$$ = \frac{d}{dτ} \(\(®I + τ®A^{-1}®B\)^{-1}®A^{-1}\)|_{τ = 0} = \frac{d}{dτ} \(\(®I - τ®A^{-1}®B + …\)®A^{-1}\)|_{τ = 0} = -®A^{-1}®B®A^{-1}. $$
	So we have $\frac{\partial (®A^{-1})_{ij}}{\partial(®A)_{kl}} (®B)_{kl}$.

	From chain rule (but this is easily solvable by differentiating $®A^{-1}(t)®A(t) = ®I$):
	$$ \frac{d}{dt}\(®A^{-1}\) = -®A^{-1} \frac{d®A}{dt} ®A^{-1}. $$
\end{priklad}

\begin{priklad}
	$®F(®A) = e^{®A}$
	$$ \frac{\partial e^{®A}}{\partial ®A}[®B] = \frac{d}{dτ}(e^{®A + τ®B})|_{τ = 0} = \frac{d}{dτ} \(®I + \frac{®A + τ®B}{1!} + \frac{(®A + τ®B)^2}{2!}\)|_{τ = 0}. $$
\end{priklad}

\begin{veta}[Daleckii–Krein]
	®A real symmetric matrix, $®A \in ®R^{k \times k}$, $®A = \sum_{i=1}^k λ_i ¦v_i \otimes ¦v_i$, where $λ_i$ are eigenvalues and $¦v_i$ are normalised orthogonal ($¦v_i·¦v_j = δ_{ij}$) eigenvectors.

	$f$ continuously differentiable real function defined on open set containing the spectrum of ®A
	$$ ®F(®A) := \sum_{i=1}^k f(λ_i) ¦v_i \otimes ¦v_i =: \sum_{i=1}^k f(λ_i) ®P_i. $$

	Then the formula for the Gateaux derivative of $f$ at point ®A in direction ®X reads
	$$ D®F(®A)[®X] = \frac{\partial ®F}{\partial ®A}[®X] = \sum_{i=1}^k \frac{df}{dλ}|_{λ = λ_i} ®P_i ®X ®P_i + \sum_{i=1}^k\sum_{j=1, j≠i}^k \frac{f(λ_i) - f(λ_j)}{λ_i - λ_j} ®P_i ®X ®P_j. $$

	Sometimes we write $D®F(®A)[®X] = f^{[1]}(®A) \ocircle ®X$ (Schur product of matrices, it is point-wise multiplication). Then
	$$ [f^{[1]}(®A)]_{ij} = \begin{cases}\frac{df}{dλ}|_{λ = λ_i}, & i = j,\\ \frac{f(λ_i) - f(λ_j)}{λ_i - λ_j}, & i ≠ j.\end{cases} $$

	\begin{dukazin}
		No summation conventions, all sums are stated explicitly!
		$$ ®F(®A) = \sum_{i=1}^k f(λ_i)¦v_i \otimes ¦v_i = $$
		$$ = \sum_{i=1}^k f(λ_i(a_{11}, a_{12}, …, a_{21}, …))¦v_i(a_{11}, a_{12}, …, a_{21}, …) \otimes ¦v_i(a_{11}, a_{12}, …, a_{21}, …). $$
		$$ \frac{\partial ®F(®A)}{\partial ®A} = \sum_{i=1}^k \(\frac{\partial f}{\partial λ_i} \frac{\partial λ_i}{\partial ®A} ¦v_i \otimes ¦v_i + f(λ_i) \frac{\partial ¦v_i}{\partial ®A} \otimes ¦v_i + f(λ_i)¦v_i \otimes \frac{\partial ¦v_i}{\partial ®A}\) = ?. $$

		We derivate $®A¦v_i = λ_i¦v_i$:
		$$ \frac{\partial ®A}{\partial ®A}¦v_i + ®A \frac{\partial ¦v_i}{\partial ®A} = \frac{\partial λ_i}{\partial ®A}¦v_i + λ_i \frac{\partial ¦v_i}{\partial ®A}. $$
		We multiply (with dot product) it by $¦v_i$:
		$$ ®P_i + \frac{\partial ¦v_i}{\partial ®A}·®A^T¦v_i = \frac{\partial λ_i}{\partial ®A}·1 + \frac{\partial ¦v_i}{\partial ®A}®A·¦v_i. $$
		$$ \frac{\partial λ_i}{\partial ®A} = ®P_i = ¦v_i \otimes ¦v_i. $$
		
		We again multiply derivative of $®A|¦v_i = λ¦v_i$, but this time by $¦v_j$:
		$$ ¦v_j \otimes ¦v_i + \frac{\partial ¦v_i}{\partial ®A}·λ_j¦v_j = 0 + λ_i \frac{\partial ¦v_i}{\partial ®A}·¦v_j. $$
		$$ (λ_j - λ_i) \frac{\partial ¦v_i}{\partial ®A}·¦v_j = -¦v_j \otimes ¦v_i. $$

		We also need $(¦v_j \otimes ¦v_i)®X_{ij} = … = ®P_i ®X ®P_j$:
		$$ … = (¦v_j \otimes ¦v_i)(¦v_i·®X¦v_j) = (¦v_j \otimes ¦v_i) ®X(¦v_j \otimes ¦v_j). $$
	\end{dukazin}
\end{veta}

% 20. 10. 2023

TODO!!!

% 27. 10. 2023

\section{Kinematics}
\begin{definice}
	We have some abstract body with point $P$. We can look at it in reference configuration (some point in past), where $K_0(P) = ¦X$ ($K_0$ = placer), $t = t_0$. Or in current configuration (how it is situated now), where $K_t(P) = ¦x$.

	The change of configuration, $¦χ$ in $¦x = ¦χ (¦X, t)$ is called deformation (but it contains translation and rotation too!).
\end{definice}

\begin{definice}
	Let us consider quantity $θ$ that describes the given material point. We can describe it by:
	\begin{itemize}
		\item $θ(P, t)$;
		\item $\hat{θ}(¦X, t)$ (referential/Lagrangian description, commonly used for solids because deformation is with respect to reference configuration);
		\item $\tilde θ(¦x, t)$ (spatial/Eulerian description, commonly used for fluids because velocity is time-local property).
	\end{itemize}
	But people write those functions without $\hat{\ }$ or $\tilde{\ }$

	\begin{poznamka}
		$$ \tilde θ(¦x, t)|_{¦x = ¦χ(¦X, t)} = \hat{θ}(¦X, t). $$
	\end{poznamka}
\end{definice}

\begin{definice}[Deformation gradient]
	$$ d¦x = ¦x_2 - ¦x_1 = ¦χ(¦X_2, t) - ¦χ(¦X_1, t) = $$
	$$ = ¦χ(¦X_1 + d¦X, t) - ¦χ(¦X_1, t) = ¦χ(¦X_1, t) + \frac{\partial ¦χ}{¦X}(¦X_1, t) d¦X + … - ¦χ(¦X_1, t) = \frac{\partial ¦χ}{¦X}(¦X_1, t) d¦X. $$

	$$ ®F(¦X, t) := \frac{\partial ¦χ}{¦X}(¦X_1, t) d¦X. \qquad d¦x = ®F d¦X $$

	\begin{poznamkain}
		It can be derived by derivatives on curves (see lecture).
	\end{poznamkain}
\end{definice}

\begin{dusledek}
	Transformation of infinitesimal line segment: $d¦x = ®F d¦X$.

	Transformation of infinitesimal surface elements: $d¦s = (\det ®F)®F^{-T} d¦S = \cof ®F d¦S$.

	Transformation of infinitesimal volume: $dv = (\det ®F) dV$.
\end{dusledek}

\begin{dusledek}[In tangent spaces]
	$$ F(¦X, t_0) = f(¦χ(¦X, t), t). $$

	Representation theorem:
	$$ (Grad F) ¦W = ¦U_{Grad F}·¦W $$
	$$ (Grad f) ¦w = ¦u_{Grad f} ·¦w $$
	$f(¦χ(¦X, t), t) = F(¦X, t_0)$
	$$ Grad f(¦x, t) |_{¦x = ¦χ(¦X, t)} = Grad F(¦X, t_0) $$
	$$ ¦U_{Grad F}·¦W = (Grad F)¦W = (Grad f)®F ¦W = (grad f)(®F ¦W) = ¦u_{grad f}·®F ¦W = ®F^T ¦u_{Grad f}·¦W. $$

	$¦u_{grad f} = ®F^{-T} ¦U_{Grad F}$.
\end{dusledek}

\begin{priklad}[Hollow cylinder]
	$r = f(R)$, $φ = Φ$, $z = Z$.

	\begin{reseni}
		$$ ®F = \frac{\partial χ_i}{\partial x_j} ¦e_i \otimes ¦E_j $$

		$$ X_1 = R \cos Φ, \qquad X_2 = R \sin Φ, x_1 = r \cos Φ, x_2 = r \sin Φ. $$
		$$ x_1 = χ_1(X_1, X_2, t), \qquad x_2 = ¦χ(x_1, x_2, t), x_i = χ_i(X_j, t). $$

		By chain rule:
		$$ \frac{\partial x_1}{\partial X_2} = \frac{\partial r \cos Φ}{\partial \partial X_2} = \frac{\partial}{\partial X_2} f(R) \cos Φ. $$

		$$ ®F = F_{rR}¦e_r \otimes ¦E_R + F_{rΦ} ¦e_r \otimes ¦E_Φ + … $$
	\end{reseni}

	\begin{reseni}
		From image:
		$$ ¦E_R \overset{®F}\rightarrow F_{rR} ¦e_r. $$
		$$ ¦E_{Φ} \overset{®F}\rightarrow F_{φΦ} ¦e_φ $$

		So $®F = \begin{pmatrix}F_{rR} & 0 \\ 0 & F_{φΦ} \end{pmatrix} $
	\end{reseni}

	TODO? (Solution by curve)
\end{priklad}

\begin{poznamka}
	How to differentiate in time tensorial quantities related to the current configuration?
	
	Upper convected derivative:
	$$ \dall{®A} (¦x, t) |_{¦x = ¦χ(¦X, t)} = \det ®F(¦X, t)\[\frac{d}{dt}\(®F^{-1}(¦X, t) ®A(¦χ(¦X, t), t) ®F^{-T}(¦X, t)\)\]®F^T(¦X, t). $$
\end{poznamka}

\subsection{Derivatives}
\begin{definice}[Lagragian velocity]
	$$ ¦V(¦X, t) = \frac{d¦χ(¦X, t)}{dt}. $$
	$$ ¦v(¦x, t)|_{¦x = ¦χ(¦X, t)} $$
\end{definice}

\begin{definice}[Eulerian velocity]
	$$ ¦v(¦x, t) = ¦V(¦X, t)|_{¦X = ¦χ^{-1}(¦x, t)}. $$
\end{definice}

\begin{definice}[Material time derivative]
	$\frac{d}{dt}=$ keep ¦X fixed, and differentiate with respect to time.
	$$ ψ(¦X, t) \rightarrow \frac{d}{dt} ψ(¦X, t) = \frac{\partial ψ}{\partial t}(¦X, t) $$
	$$ ψ(¦x, t) \rightarrow \frac{d}{dt} ψ(¦χ(¦X, t), t) = \frac{\partial ψ}{\partial t}|_{¦x = ¦χ(¦X, t)} + \frac{\partial ψ}{\partial x_i}(¦x, t)|_{¦x = ¦χ(¦X, t)} \frac{dχ_i}{dt}(¦X, t) = $$
	$$ = \(\frac{\partial ψ}{\partial t}(¦x, t)|_{¦x = ¦χ(¦X, t)} + V_i(¦X, t) \frac{\partial ψ}{\partial x_i}(¦x, t)|_{¦x = ¦χ(¦X, t)}\) = $$
	$$ = \(\frac{\partial ψ}{\partial t}(¦x, t) + v_i(¦x, t) \frac{\partial ψ}{\partial x_i}(¦x, t)\)|_{¦x = ¦χ(¦x, t)} $$

	$$ \frac{d}{dt}ψ(¦x, t) = \frac{\partial ψ}{\partial t}(¦x, t) + (¦v(¦x, t)·\nabla)ψ(¦x, t). $$
\end{definice}

\begin{definice}[Time derivative of deformation gradient ®F]
	$$ \frac{d}{dt} ®F (¦X, t) = \frac{d}{dt} \(\frac{\partial ¦χ(¦X, t)}{\partial ¦X}\) = \frac{\partial}{\partial ¦X} \frac{d ¦χ(¦X, t)}{dt} = \frac{\partial}{\partial ¦X}¦V(¦X, t) = $$
	$$ = \frac{\partial}{\partial ¦X} ¦v(¦χ(¦X, t), t) = \frac{\partial ¦v}{\partial ¦x}(¦x, t)|_{¦x = ¦χ(¦X, t)} \frac{\partial ¦χ}{\partial ¦X}(¦X, t) = \frac{\partial ¦v}{\partial x}|_{¦x = ¦χ(¦X, t)}®F(¦X, t). $$

	$$ ®L(¦x, t) := \nabla ¦V(¦x, t) = \frac{\partial ¦v}{\partial ¦x}(¦x, t). $$
\end{definice}

\begin{dusledek}
	$$ \frac{d®F}{dt} = ®L®F $$
\end{dusledek}

\begin{dusledek}
	$$ \dall{®A} = \frac{d®A}{dt} - ®L®A - ®A®L^T $$
\end{dusledek}

% 10. 11. 2023

TODO!!!

% 24. 11. 2023

\begin{poznamka}[Balance laws in Eulerian description (revision, the last lecture)]
	$$ \frac{dρ}{dt} + ρ \Div ¦v = 0; $$
	$$ ρ \frac{d¦v}{dt} = \Div ®T + ρ¦b, \qquad ®T = ®T^T; $$
	$$ ρ \frac{de}{dt} = ®T:®L - \Div ¦j_q; $$
	or
	$$ ρ \frac{d}{dt}(e + \frac{1}{2}¦v·¦v) = \Div(®T^T ¦v) + ρ¦b·¦v - \Div ¦j_q. $$
\end{poznamka}

\begin{poznamka}[Balance laws in Lagrangian description]
	Starting with $\frac{d}{dt}_{V(t)} ρ(¦x, t) dv = \frac{d}{dt} m_{V(t)} = 0$ i.e. mass remains same: $m_{V(t)} = m_{V(t_0)}$). We integrate over volume:
	$$ \int_{V(t_0)} ρ_R(¦X) dV = \int_{V(t)} ρ(¦x, t) dv = \int_{V(t_0)} ρ(¦x, t)|_{¦x = ¦χ(¦X, t)} \det ®F dV. $$
	Localization principle:
	$$ ρ(¦x, t)|_{¦x = ¦χ(¦X, t)} \det ®F = ρ_R(¦X). $$

	$$ \int_{V(t)} ρ \frac{d¦v}{dt} dv = \int_{V(t)} \Div ®T d¦v + \int_{V(t)} ρ¦b dv. $$
	$$ \int_{V(t)} \Div ®T d¦v = \int_{\partial V(t)} ®T ¦n\,ds = \int_{\partial V(t_0)} ®T(\det ®F)®F^{-T} ¦N\, dS \overset{\text{Stokes}} \int_{V(t_0)} \Div_{¦X}((det ®F) ®T ®F^{-T}) dV =: \int_{V(t_0)}(®T_R) dV. $$
	$$ ®T_R(¦X, t) := (\det ®F(¦X, t))®T(¦x, t)|_{¦x = ¦χ(¦X, t)} ®F^{-T}(¦X, t) $$
	is first Piola–Kirchhoff stress tensor. Cauchy ($®T$) is current $\rightarrow$ current. P–K ($®T_R$) is reference $\rightarrow$ current.
	$$ \int_{\partial V(t)} dv \rightarrow \int_{\partial V(t_0)} (\Div_{¦X} ®T_R) dV. $$
	$$ \int_{V(t)} ρ¦b dv = \int_{V(t_0)} \rho(¦x, t)|_{¦x = ¦χ(¦X, t)} ¦b(¦x, t)|_{¦x = ¦χ(¦X, t)} \det ®F dV = \int_{V(t_0)} ρ_R(¦X) ¦b dV. $$
	$$ \int_{V(t)} ρ \frac{d¦v}{dt} dv = \int_{V(t_0)} ρ \frac{\partial^2 ¦χ}{\partial t^2}(¦X, t) \det ®F dV = \int_{V(t_0)} ρ_R \frac{\partial^2 ¦χ}{\partial t^2} dV. $$
	Altogether:
	$$ ρ_R \frac{\partial^2 ¦χ}{\partial t^2} = \Div_{¦X} ®T_R + ρ_R¦b \qquad (\text{Solve for $¦χ$}). $$
	
	$®T = ®T^T \rightarrow ®T_R®F^T = ®F®T_R^T$ (P–K is not symmetric!).

	$$ ρ \frac{de}{dt} = ®T:®L - \Div ¦j_q \rightarrow \int_{V(t)} ρ \frac{de}{dt} dv = \int_{V(t)}®T:®L dv - \int_{V(t)}\Div ¦j_q dv. $$
	$$ \int_{V(t)} \Div ¦j_q dv = \int_{\partial V(t)} ¦j_q·¦n ds = \int_{\partial V(t_0)} ¦j_q(¦x, t)|_{¦x = ¦χ(¦X, t)} · \det ®F(¦X, t)®F^{-T}(¦X, t)¦N dS = $$
	$$ = \int_{\partial V(t_0)}(\det ®F(¦X, t)®F^{-1}(¦X, t)¦j_q(¦x, t))·¦N dS = $$
	$$ = \int_{V(t_0)} \Div((\det ®F)®F^{-1}¦j_q) dV. $$
	$¦J_q = (\det ®F)®F^{-1}¦j_q$ is called referential heat flux. (It cannot be given by Fouriers law ($¦j_q = k \nabla_{¦x} θ$, $\Div ¦j_q = \Div(k \nabla θ)$).)
	$$ \int_{V(t)} \underbrace{®T:\overbrace{®L}^{\nabla_{¦x} ¦v}}_{\tr(®T®L^T) = \tr(®L®T^T)} dv = \int_{V(t_0)} (\det ®F) ®T:®L dV = $$
	$$ = \int_{V(t_0)} \tr((\det ®F)®T®L^T) dV = \int_{V(t_0)} \tr\((\det ®F)®T®F^{-T}\(\frac{d®F}{dt}\)^T\)dV = \int_{V(t_0)} ®T_R : \dot{®F} dV. $$
	Altogether
	$$ ρ_R \frac{\partial e}{\partial t} = ®T_R:\dot{®F} - \Div_{¦X} ¦J_q. $$
\end{poznamka}

\section{Entropy}
\begin{poznamka}[Objective]
	Find quantity that is increasing/decreasing in time.
\end{poznamka}

\begin{poznamka}[With no interior]
	$$ ρ \frac{d}{dt}\(e + \frac{1}{2}¦v·¦v\) = \Div ®T + ρ¦b - \Div ¦j_q = \Div ®T + 0 + \Div(k \nabla θ). $$
	Let us work with\footnote{$®D_δ := ®D - \frac{1}{3} (\tr ®D)®I$. (Traceless part of ®D.)} $\Div ®T = -p_{th}®I + \tilde λ(\Div ¦v) + 2μ®D_δ$, and assume that $®T = -p_{th}(ρ, θ)®I + \tilde λ(\Div ¦v)®I + 2μ®D_δ$ (from $\frac{pV}{T} = \const$).

	$$ ρ \frac{d}{dt}(e + \frac{1}{2}¦v·¦v) = \Div(®T^T¦v) - \Div ¦j_q. $$
	$$ \frac{d}{dt} \int_{V(t)}ρ(e + \frac{1}{2}¦v·¦v) dv = \int_{\partial V(t)} ®T^T ¦v·¦n ds - \int_{\partial V(t)} ¦j_q·¦n ds. $$
	The first part is work and it is zero if we have boundary condition $¦v|_{\partial V} = 0$. The second part is heat exchange which is zero if we have boundary condition $¦j_q·¦n|_{\partial V} = 0$. Both boundary conditions together are math way to say system with \emph{no interactions}.

	$ρ, θ, ¦v \rightarrow ρ, e, θ$.

	Assume $η = η(ρ, e) \rightarrow e=e(η, ρ)$. (We will write $e = e(η, ρ) = e(η(¦x, t), ρ(¦x, t)) = e(¦x, t)$.)

	We have 1. Balance of internal energy
	$$ ρ\frac{de}{dt} = ®T:®D - \Div ¦j_q; $$
	2. Chain rule
	$$ ρ\frac{de}{dt} = \frac{\partial e}{\partial η}(η, ρ) \frac{dη}{dt} + \frac{\partial e}{\partial ρ}(η, ρ) \frac{dρ}{dt}. $$
	$$ ρ\frac{\partial e}{\partial η}(η, ρ) \frac{dη}{dt} = ®T:®D - \Div ¦j_q - \frac{\partial e}{\partial ρ}(η, ρ) \frac{dρ}{dt}, $$
	$$ ρ\frac{\partial e}{\partial η}(η, ρ) \frac{dη}{dt} = ®T:®D - \Div ¦j_q + \frac{\partial e}{\partial ρ}(η, ρ)ρ \Div ¦v, $$
	$$ ρ\frac{\partial e}{\partial η}(η, ρ) \frac{dη}{dt} = (-p_{th}®I + \tilde(\Div ¦v)®I + 2μ®D_δ):®D - \Div ¦j_q + \frac{\partial e}{\partial ρ}(η, ρ)ρ \Div ¦v, $$
	$$ ρ\frac{\partial e}{\partial η}(η, ρ) \frac{dη}{dt} = \(-p_{th} + ρ\frac{\partial e}{\partial ρ}(η, ρ)\) \Div ¦v + \tilde λ (\Div ¦v)^2 + 2μ®D_δ:®D_δ - \Div ¦j_q, $$
	$$ ρ \frac{dη}{dt} = \frac{\(-p_{th} + ρ\frac{\partial e}{\partial ρ}(η, ρ)\)}{\frac{\partial e}{\partial η}}\Div ¦v - \frac{\Div ¦j_q}{\frac{\partial e}{\partial η}} + \frac{\tilde λ (\Div ¦v)^2 + 2μ|®D_δ|^2}{\frac{\partial e}{\partial η}}. $$
	There is no chance that this could be positive. (Its obvious, because the value can flow, so point-wise $≥0$ is lost case.) But we can integrate over volume. Thus instead of $\frac{dη}{dt} ≥ 0$ we want just $\frac{d}{dt} \int_{V(t)} ρη dv ≥ 0$.
	$$ \frac{d}{dt} \int_{V(t)} ρη dv = \int_{V(t)}\frac{\(-p_{th} + ρ\frac{\partial e}{\partial ρ}(η, ρ)\)}{\frac{\partial e}{\partial η}}\Div ¦v dv - \int_{V(t)}\frac{\Div ¦j_q}{\frac{\partial e}{\partial η}}dv + \int_{V(t)} \frac{\tilde λ (\Div ¦v)^2 + 2μ|®D_δ|^2}{\frac{\partial e}{\partial η}} dv. $$
	The third integral OK, if $\frac{\partial e}{\partial η} > 0$.
	$$ \Div \(\frac{¦j_q}{\frac{\partial e}{\partial η}}\) = \frac{\Div ¦j_q}{\frac{\partial e}{\partial η}} + \nabla \(\frac{1}{\frac{\partial e}{\partial η}}\)·¦j_q. $$
	$$ \frac{d}{dt} \int_{V(t)} ρη dv = \int_{V(t)}\frac{\(-p_{th} + ρ\frac{\partial e}{\partial ρ}(η, ρ)\)}{\frac{\partial e}{\partial η}}\Div ¦v dv - \int_{V(t)}\Div \(\frac{¦j_q}{\frac{\partial e}{\partial η}}\)dv + \int_{V(t)} \nabla \(\frac{1}{\frac{\partial e}{\partial η}}\)·¦j_q dv + REST. $$
	The second integral is zero from Stokes and boundary condition $¦j_q·¦n|_{\partial V} = 0$. On the third integral, we can use derivative of inverse value:
	$$ \frac{d}{dt} \int_{V(t)} ρη dv = \int_{V(t)}\frac{\(-p_{th} + ρ\frac{\partial e}{\partial ρ}(η, ρ)\)}{\frac{\partial e}{\partial η}}\Div ¦v dv - \int_{V(t)} \frac{\nabla \(\frac{\partial e}{\partial η}\)·¦j_q}{\(\frac{\partial e}{\partial η}\)^2} dv + REST =  $$
	$$ = \int_{V(t)}\frac{\(-p_{th} + ρ\frac{\partial e}{\partial ρ}(η, ρ)\)}{\frac{\partial e}{\partial η}}\Div ¦v dv + k\int_{V(t)} \frac{\nabla \(\frac{\partial e}{\partial η}\)·\nabla θ}{\(\frac{\partial e}{\partial η}\)^2} dv + REST. $$
	If we set $\frac{\partial e}{\partial η}(η, ρ) = θ$, the second integral is non-negative. Moreover, for $θ ≥ 0$ we satisfy the assumption for the "first third integral". Moreover if we enforce $ρ^2 \frac{\partial e}{\partial ρ}(ρ, η) = p_{th}(θ, ρ)$, the first integral is zero, so we win.
\end{poznamka}

\begin{poznamka}
	Volíme si tedy $\tilde λ, μ > 0$.
\end{poznamka}

\begin{dusledek}
	$$ \frac{d}{dt} \int_{V(t)} ρ η dv ≥ 0 $$
	is granted for quantity that solves equations
	$$ e = e(η, ρ), \qquad \frac{\partial e}{\partial η} = θ, \qquad ρ^2 \frac{de}{dρ} = p_{th}(θ, ρ). $$
\end{dusledek}

\begin{priklad}
	$$ p_{th}(θ, ρ) = c_V(γ - 1)ρθ, \qquad e(θ, ρ) = c_V θ. $$
\end{priklad}

\begin{poznamka}[?]
	\ 
	\begin{enumerate}
		\item Energy is constant.
		\item Energy is function of entropy and volume.
		\item Entropy increases.
	\end{enumerate}
\end{poznamka}

\begin{poznamka}
	$e = e(η, ρ)$ is given $\rightarrow$ we know everything $θ = \frac{\partial e}{\partial η}(η, ρ)$, $p_{th} = ρ^2 \frac{\partial e}{\partial ρ}(η, ρ)$. (Warning: $e = e(ρ, θ)$ is not enough!)
\end{poznamka}

\begin{poznamka}
	Is there a better function that will allow us to do something like this?
\end{poznamka}

\begin{definice}[Helmholtz free energy density]
	$$ ψ(θ, ρ) := e(η, ρ)|_{η = η(θ, ρ)} - θη|_{η = η(θ, ρ)}. $$

	\begin{poznamka}
		This is le Legrende transformation of internal energy.
	\end{poznamka}
\end{definice}

\begin{dusledek}
	$$ \frac{\partial ψ}{\partial θ}(θ, ρ) = -η, \qquad \frac{\partial ψ}{\partial ρ}(θ, ρ) = \frac{\partial e}{\partial ρ}(η, ρ)|_{η = η(θ, ρ)} \quad \(= \frac{p_{th}}{ρ^2}\). $$
	
	\begin{dukazin}
		$$ \frac{\partial ψ}{\partial θ}(θ, ρ) = \frac{\partial e(η, ρ)}{\partial η}|_{η = η(θ, ρ)} \frac{\partial η}{\partial θ}(θ, ρ) - η|_{η = η(θ, ρ)} - θ \frac{\partial η}{\partial θ}(θ, ρ) = -η|_{η = η(θ, ρ)}. $$
		$$ \frac{\partial ψ}{\partial ρ}(θ, ρ) = \frac{\partial e}{\partial η}(η, θ) \frac{\partial η}{\partial ρ}(θ, ρ) + \frac{\partial e}{\partial ρ}(η, ρ)|_{η = η(θ, ρ)} - θ \frac{\partial η}{\partial ρ}(θ, ρ) = \frac{\partial e}{\partial ρ}(η, ρ)|_{η = η(θ, ρ)}. $$
	\end{dukazin}
\end{dusledek}

\end{document}
