\documentclass[12pt]{article}					% Začátek dokumentu
\usepackage{../../MFFStyle}					    % Import stylu



\begin{document}

% 03. 10. 2022

\section{Introduction}
\begin{poznamka}[Literature]
	„Riemann surfaces and algebraic curves“, Renzo Cavalieri and Eric Miles
\end{poznamka}

\subsection{Differentiability}
\begin{definice}[Differentiable]
	A function $f: ®C \rightarrow ®C$ is differentiable (also holomorphic) at a point $z_0 \in ®C$ if the following limit exists
	$$ \lim_{|h| \rightarrow 0} \frac{f(z_0 + h) - f(z_0)}{h} =: f'(z_0) \in ®C. $$

	We call $f'(z_0)$ the derivative of $f$ at $z_0$. A function $f$ is differentiable on a domain (open connected subset of ®C) if its differentiable for all points of this domain.
\end{definice}

\begin{poznamka}[Writing complex numbers in cartesian cooridnates]
	$z = x + iy$, for $x, y \in ®R$, we can write a function $f: ®C \rightarrow ®C$ in terms of two functions $u, v: ®R^2 \rightarrow ®R$ such that
	$$ f(x, y) = u(x, y) + i·v(x, y). $$
\end{poznamka}

\begin{veta}[Cauchy–Riemann equations]
	Let $f: ®C \rightarrow ®C$ be a holomorphic function on an open subset of ®C. Considering $f = u + iv$, then
	$$ \frac{\partial u}{\partial x} = \frac{\partial v}{\partial y}, \qquad \frac{\partial v}{\partial x} = -\frac{\partial u}{\partial y}. $$
\end{veta}

\begin{definice}[Orientability, orientation-preserving function]
	Define and equivalence relation on the set of all bases of $®R^2$ by saying that $B_1 \sim B_2$ iff the determinant of the change of basis matrix is positive.

	A function $f: ®R^2 \supset U \rightarrow ®R^2$ is said to be orientation-preserving if on an open dense subset of $U$, the determinant of the Jacobi matrix is positive. Jacobi matrix:
	$$ J(f) = \begin{pmatrix} \frac{\partial u}{\partial x} & \frac{\partial u}{\partial y}\\ \frac{\partial v}{\partial x} & \frac{\partial v}{\partial y} \end{pmatrix}. $$
\end{definice}

\begin{dusledek}
	Let $f$ be a non-constant holomorphic function, then $f$ is orientation-preserving.

	\begin{dusledek}
		Since $f$ is holomorphic, the Cauchy-Riemann equations implies that
		$$ \det(J(f)) = \frac{\partial u}{\partial x} \frac{\partial v}{y} - \frac{\partial v}{\partial x} \frac{\partial u}{\partial y} \overset{\text{C–R}}= \(\frac{\partial v}{\partial x}\)^2 + \(\frac{\partial v}{\partial y}\)^2 ≥ 0. $$
		Since $f$ is non-constant, the inequality is strict on a dense open subset of the domain of definition.
	\end{dusledek}
\end{dusledek}

\begin{veta}[Open mapping theorem]
	A non-constant holomorphic function $f$ is open (that is if $U$ is an open subset of ®C, then $f(U)$ is also open).
\end{veta}

\subsection{Integration}
\begin{definice}
	For a path $γ$ (smooth function, $γ: ®R \supset [a, b] \rightarrow ®C$) we define
	$$ \int_γ f(x) dx := \int_a^b f(γ(t))·γ'(t) dt $$
\end{definice}

\end{document}
