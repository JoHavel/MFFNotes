\documentclass[12pt]{article}					% Začátek dokumentu
\usepackage{../../MFFStyle}					    % Import stylu



\begin{document}

% 03. 10. 2022

\section{Introduction}
\begin{poznamka}[Literature]
	„Riemann surfaces and algebraic curves“, Renzo Cavalieri and Eric Miles
\end{poznamka}

\subsection{Differentiability}
\begin{definice}[Differentiable]
	A function $f: ®C \rightarrow ®C$ is differentiable (also holomorphic) at a point $z_0 \in ®C$ if the following limit exists
	$$ \lim_{|h| \rightarrow 0} \frac{f(z_0 + h) - f(z_0)}{h} =: f'(z_0) \in ®C. $$

	We call $f'(z_0)$ the derivative of $f$ at $z_0$. A function $f$ is differentiable on a domain (open connected subset of ®C) if its differentiable for all points of this domain.
\end{definice}

\begin{poznamka}[Writing complex numbers in cartesian cooridnates]
	$z = x + iy$, for $x, y \in ®R$, we can write a function $f: ®C \rightarrow ®C$ in terms of two functions $u, v: ®R^2 \rightarrow ®R$ such that
	$$ f(x, y) = u(x, y) + i·v(x, y). $$
\end{poznamka}

\begin{veta}[Cauchy–Riemann equations]
	Let $f: ®C \rightarrow ®C$ be a holomorphic function on an open subset of ®C. Considering $f = u + iv$, then
	$$ \frac{\partial u}{\partial x} = \frac{\partial v}{\partial y}, \qquad \frac{\partial v}{\partial x} = -\frac{\partial u}{\partial y}. $$
\end{veta}

\begin{definice}[Orientability, orientation-preserving function]
	Define and equivalence relation on the set of all bases of $®R^2$ by saying that $B_1 \sim B_2$ iff the determinant of the change of basis matrix is positive.

	A function $f: ®R^2 \supset U \rightarrow ®R^2$ is said to be orientation-preserving if on an open dense subset of $U$, the determinant of the Jacobi matrix is positive. Jacobi matrix:
	$$ J(f) = \begin{pmatrix} \frac{\partial u}{\partial x} & \frac{\partial u}{\partial y}\\ \frac{\partial v}{\partial x} & \frac{\partial v}{\partial y} \end{pmatrix}. $$
\end{definice}

\begin{dusledek}
	Let $f$ be a non-constant holomorphic function, then $f$ is orientation-preserving.

	\begin{dusledek}
		Since $f$ is holomorphic, the Cauchy-Riemann equations implies that
		$$ \det(J(f)) = \frac{\partial u}{\partial x} \frac{\partial v}{y} - \frac{\partial v}{\partial x} \frac{\partial u}{\partial y} \overset{\text{C–R}}= \(\frac{\partial v}{\partial x}\)^2 + \(\frac{\partial v}{\partial y}\)^2 ≥ 0. $$
		Since $f$ is non-constant, the inequality is strict on a dense open subset of the domain of definition.
	\end{dusledek}
\end{dusledek}

\begin{veta}[Open mapping theorem]
	A non-constant holomorphic function $f$ is open (that is if $U$ is an open subset of ®C, then $f(U)$ is also open).
\end{veta}

\subsection{Integration}
\begin{definice}
	For a path $γ$ (smooth function, $γ: ®R \supset [a, b] \rightarrow ®C$) we define
	$$ \int_γ f(x) dx := \int_a^b f(γ(t))·γ'(t) dt $$
\end{definice}

% 10. 10. 2023

\begin{definice}[Continuous deformation]
	For $γ, μ: [a, b] \rightarrow U$ ($U$ simply connected), paths with the same endpoints ($γ(a) = μ(a)$ and $γ(b) = μ(b)$). Then a continuous deformation $γ$ into $μ$ is a continuous function $H: [a, b] \times [0, 1] \rightarrow U \subseteq ®C$ such that $H(s, 0) = γ(s)$, $H(s, 1) = μ(s)$, $H(a, t) = z_a := γ(a) = μ(a)$ and $H(b, t) = z_b := γ(b) = μ(b)$.
\end{definice}

\begin{veta}
	Suppose that $γ, μ: [a, b] \rightarrow U$ ($U$ simply connected) are related by a continuous deformation of paths $H$. Then for any holomorphic function $f$ on $U$ we have
	$$ \int_γ f(z)dz = \int_μ f(z) dz. $$

	\begin{dukazin}[Partial proof assuming $H$ admits partial derivatives]
		For any $t \in [0, 1]$ we integrate the function $INT(t) = \int_{H(·, t)} f(z) dz$. Consider the derivative of $INT(t)$ with respect to $t$:
		$$ \frac{d}{dt} (INT(t)) = \frac{d}{dt} \(\int_a^b f(H(s, t)) \frac{\partial H}{\partial s}(s, t) ds\) \overset{\text{Leibniz + chain rule}}=  $$
		$$ = \int_a^b f'(H(s, t)) \frac{\partial H}{\partial t}(s, t)·\frac{\partial H}{\partial s}(s, t) + f(H(s, t)) \frac{\partial^2 H}{\partial s \partial t}(s, t) ds = $$
		$$ = \int_a^b \frac{d}{ds}\[f(H(s, t)) \frac{\partial H}{\partial t}\]ds = $$
		$$ = f(H(s, t)) \frac{\partial H}{\partial t} |_{s=a}^{s = b} \overset{\text{constant endpoints}}= 0. $$

		Having derivative identically equal to 0, means that $INT(t)$ is a constant function and $\int_γ f(z)dz = INT(0) = INT(1) = \int_μ f(z) dz$.
	\end{dukazin}
\end{veta}

\begin{dusledek}
	Let $U$ be a simply connected subset of ®C and $f: U \rightarrow ®C$ a holomorphic function. For any closed path whose image is inside $U$, $\int_γ f(z) dz = 0$.

	\begin{dukazin}[Sketch]
		The definition of simply connected is (essentially) the same as saying that any closed path can be continuously deformed to a constant path $c$.
		$$ \int_γ f(z) dz = \int_c f(z) dz = \int_a^b f(c(z))·c'(z) dz = \int_a^b f(c(z)) · 0 dz = 0 $$
	\end{dukazin}
\end{dusledek}

\begin{priklad}
	Let $U$ be a simple connected domain and $f: U \rightarrow ®C$ a holomorphic function on $U \setminus \{z_0\}$. For $j = 1, 2$, let $γ_j$ be a path parametrizing a circle centered at $z_0$ of radius $r_j$, oriented counterclockwise and completely contained in $U$. Show that $\oint_{γ_1} f(z) dz = \oint_{γ_2} f(z) dz$.
\end{priklad}

\subsection{Cauchy's integral formula}
\begin{veta}[Cauchy's integral formula]
	Let $γ$ be a loop around $z \in ®C$, and $f: U \rightarrow ®C$ a holomorphic function. For $U$ a neighbourhood of $γ$,
	$$ f(z) = \frac{1}{2πi} \oint_γ \frac{f(w)}{w - z} dw. $$

	\begin{dukazin}
		Conway 1978, Chapter IV.
	\end{dukazin}
\end{veta}

\begin{dusledek}
	$$ f(z) = \frac{1}{2πi} \oint_γ \frac{f(w)}{w - z_0 + z_0 - z} dw = \frac{1}{2πi} \oint \frac{f(w)}{w - z_0}·\(\frac{1}{1 - \frac{z - z_0}{w - z_0}}\)dw = $$
	$$ = \frac{1}{2πi} \oint_γ \frac{f(w)}{w - z_0} \(\sum_{n=0}^∞ \frac{(z - z_0)^n}{(w - z_0)^n}\) dw = $$
	$$ = \sum_{n=0}^∞ \(\frac{1}{2πi} \oint_γ \frac{f(w)}{(w - z_0)^n}\) (z - z_0)^n. $$
	For sufficiently "small" (shrunken) $γ$. So $f$ is smooth (infinitely differentiable). Moreover, it is analytic (that is, its Taylor expansion around $z_0$ converges to $f$ in a neighbourhood of $z_0$).
\end{dusledek}

\begin{definice}[Pole]
	Given a positive integer $n$, a complex function $f$ has pole of order $n$ at the point $z_0 \in ®C$ if $(z - z_0)^n f(z)$ is holomorphic at $z_0$ but $(z - z_0)^{n - 1} f(z)$ is not.
\end{definice}

\begin{priklad}
	Show that if $f$ has a pole of order $n$ at $z_0 \in ®C$. Then it admits a Laurient expansion $f(z) = \sum_{k = -n}^∞ a_k (z - z_0)^k$ with $a_{-n} ≠ 0$.
\end{priklad}

\begin{definice}[Residue]
	Let $f$ have a pole of order $n$ at the point $z_0 \in ®C$. Then the residue of $f$ at $z_0$ is the $k=-1$ coefficient of the Laurent expansion of $f$ at $z_0$.
\end{definice}

\begin{priklad}
	Show that if $f$ has a pole of order $1$ at $z_0$, then the residue of $f$ at $z_0$ can be computed as the following limit:
	$$ \res_{z=z_0} f(z) = \lim_{z \rightarrow z_0} (z - z_0)f(z). $$
\end{priklad}

\begin{priklad}[Residue theorem]
	Let $γ: [a, b] \rightarrow U \subset ®C$ be a simple closed path, bounding a domain $W$ containing the points $z_1, …, z_m$. Assume that $f$ is holomorphic on $U \setminus \{z_1, …, z_m\}$ and has poles at $\{z, …, z_m\}$. Show that
	$$ \oint_γ f(z) dz = 2πi \sum_{j=1}^m \res_{z = z_j} f(z). $$
\end{priklad}

% 17. 10. 2023

TODO!!!

% 24. 10. 2023

\subsection{(Real) Projective space}
\begin{poznamka}[Building structures]
	Set $\rightarrow$ Topology $\rightarrow$ Differential structure (atlas) $\rightarrow$ Riemann metric $\rightarrow$ Connection…
\end{poznamka}

\begin{definice}[Real projective space]
	The set $®P^n(®R)$ is defined to be either of the following bijective sets: Lines through the origin in $®R^{n+1}$; Equivalence classes of $(n+1)$-tuples of real numbers $(x_0, …, x_n) ≠ (0, …, 0)$, such that for any real number $λ \in ®R^* = ®R \setminus \{0\}$: $(x_0, …, x_n) \sim (λ x_0, …, λ x_n)$.
\end{definice}

\begin{priklad}
	Confirm that the sets above are in bijection with each other.
\end{priklad}

\begin{poznamka}[Notation]
	We will often denote a point in $®P^n(®R)$ as the equivalence class $[x_0, …, x_n]$.
\end{poznamka}

\begin{definice}[Topology of $®P^n(®R)$]
	We give a topology to $®P^n(®R)$ by endowing it with following quotient topology: consider the surjection $π: ®R^{n+1} \setminus \{¦o\} \rightarrow ®P^n(®R)$, $(x_0, …, x_n) \mapsto [x_0, …, x_n]$. A set $U \subset ®P^n(®R)$ is defined to be open if $π^{-1}(U) := \{x \in ®R^{n+1} \setminus \{¦o\} \middle| π(x) \in U\}$ is open in $®R^{n+1} \setminus \{¦o\}$.

	That is we give $®P^n(®R)$ the finest topology that makes $π$ continuous.
\end{definice}

\begin{priklad}
	Check that for ®C we can define $®P^n(®C)$ or $®C ®P^n$ the same way.
\end{priklad}

\begin{priklad}[Projective space]
	$®R^* = ®R \setminus \{0\}$ is an abelian group. Let $®R^*$ act on $®R^{n+1}$ by component wise multiplication. When a general group $G$ acts on a set $X$ we have equivalence relation $x \sim y$ if $y = g ∘ x$. We call the equivalence classes the orbits of $G$. So $®P^n(®R) = \(®R^{n+1} \setminus \{¦o\}\) / ®R^*$.

	Sphere quotient: Let $S^n \subseteq ®R^{n+1}$. Denote the unit sphere. Then the group $®Z_2 = \{+1, -1\}$ act on the sphere by $±1(x_0, …, x_n) = (±x_0, …, ±x_n)$. Then $S^n / ®Z_2 = ®P^n(®R)$.

	Disk model: Consider the $n$-dimensional closed unit disk $\overline{®D^n} \subseteq ®R^n$, and the equivalence relation on the points of the boundary: $x \sim -x$ if $\|x\| = 1$. Then $®P^n(®R)$ is the quotient (collection of equivalence classes), i.e. $\overline{D^n} \setminus \sim \simeq ®P^n(®R)$.
\end{priklad}

\begin{priklad}
	Conclude from either of these models of $®P^n(®R)$ that as a topological space, $®R^n(®P)$ is compact and Hausdorff.
\end{priklad}

\begin{poznamka}
	Now we come to the smooth manifold structures. Let's start with $®P^1(®R)$. Define
	$$ U_x := ®P^1(®R) \setminus \{[x, y] \in ®P^1(®R) | x = 0\}, \qquad φ_x: U_x \rightarrow ®R, \quad φ_x([x, y]) = \frac{x}{y}. $$
	Similarly, we define a second chart:
	$$ U_y := ®P^1(®R) \setminus \{[x, y] \in ®P^1(®R) | y = 0\}, \qquad φ_y: U_y \rightarrow ®R, \quad φ_y([x, y]) = \frac{y}{x}. $$

	\begin{prikladin}
		Check that $U_x, U_y$ are open and that $φ_x, φ_y$ are homeomorphisms.
	\end{prikladin}

	\begin{dukazin}
		Consider the transition functions:
		$$ U = U_x \cap U_y = \{[x, y] \in ®P^1(®R) | x, y ≠ 0\}, \qquad φ_x(U) = φ_y(U) = ®R \setminus \{0\}. $$
		The translation function $T_{x, y} := φ_y ∘ (φ_x)^{-1}$ sends, for $y ≠ 0$:
		$$ T_{x, y}: y \overset{(φ_x)^{-1}}\mapsto [1, y] = \[\frac{1}{y}, 1\] \overset{φ_y}\mapsto \frac{1}{y}. $$
		Which is smooth on the domain $®R \setminus \{0\}$.

		TODO smooth. Thus $®P^1(®R)$ is a smooth manifold.
	\end{dukazin}
\end{poznamka}

\begin{priklad}
	Show that $®P^1(®R)$ is homomorphic to the circle $S^1$. We call $®P^1(®R)$ the real projective line.
\end{priklad}

\begin{priklad}
	Try to show that $®C®P^1 = ®P^1(®C)$ is a smooth manifold.
\end{priklad}

\begin{priklad}
	For $®P^2(®R)$ the followings charts form atlas:
	$$ U_x := \{[x, y, z] | x ≠ 0\}, \qquad φ_x: U_x \rightarrow ®R, \quad φ_x([x, y, z]) = \(\frac{y}{x}, \frac{z}{x}\), $$
	$$ U_y := \{[x, y, z] | y ≠ 0\}, \qquad φ_y: U_y \rightarrow ®R, \quad φ_y([x, y, z]) = \(\frac{x}{y}, \frac{z}{y}\), $$
	$$ U_z := \{[x, y, z] | z ≠ 0\}, \qquad φ_z: U_z \rightarrow ®R, \quad φ_z([x, y, z]) = \(\frac{x}{z}, \frac{y}{z}\). $$
	Check these are open subsets and homeomorphisms, with smooth transformation functions. And extend this to $®P^n(®R)$.
\end{priklad}

\subsection{Compact surfaces}
\begin{definice}[Surface]
	A surface is a manifold of real dimension 2.
\end{definice}

\begin{priklady}
	$®R^2$, ®C, and any of their open subsets are surfaces. $S^2$ is a compact surface, as is $®P^2(®R)$.
\end{priklady}

\begin{definice}[Connected surface]
	Given two connected surfaces $S_1$ and $S_2$, the connected surface $S_1 \# S_2$ is the surface obtained by removing an open disc from each of the surfaces and identifying the resulting boundaries via a homeomorphism.
\end{definice}

\begin{priklad}
	At the level of topological spaces, show that the operation $\#$ is well defined up to homeomorphism, that is, show that the choice of disks in $S_1$ and $S_2$ does not change the definition of $S_1 \# S_2$ / homeomorphism.
\end{priklad}

\begin{priklad}
	Show that $\#$ gives the set of homeomorphism classes of connected compact surfaces the structure of a monoid. (Which surface is the identity of the monoid?)
\end{priklad}

\begin{veta}[Classification of compact surfaces]
	Any connected, compact surfaces is homeomorphic to exactly one surface in the following list:
	\begin{itemize}
		\item $S^2$;
		\item $T^{\# g} := T \# … \# T$, $g \in ®N_0$;
		\item $®P^2(®R)^{\#n} := ®P^2(®R) \# … \# ®P^2(®R)$, $n \in ®N_0$.
	\end{itemize}
\end{veta}

% 31. 10. 2023

\begin{poznamka}[Deep fact]
	For $d ≤ 3$, if two $d$-dimensional manifolds are homeomorphic, then they are diffeomorphic.
\end{poznamka}

\section{Riemann surfaces}
\begin{definice}[Riemann surface]
	A Riemann surface is a complex analytic manifold of dimension 1:
	\begin{itemize}
		\item $X$ is a Hausdorff, connected topological space;
		\item for all $x \in X$, there is a homeomorphism $φ_x: U_x \rightarrow V_x$, where $U_x$ is an open neighbourhood of $x \in X$, $V_x$ is an open set in ®C;
		\item for any $U_x$, $U_y$ such that $U_x \cap U_y ≠ \O$, the transition function $T_{x, y} := φ_y ∘ φ_x^{-1}: φ_x(U_x \cap U_y) \rightarrow φ_y(U_x \cap U_y)$ is holomorphic.
	\end{itemize}

	\begin{poznamkain}
		We saw in the first lecture that a holomorphic preserves orientation when thought of as a function from the real plane to itself. Since our transition functions are holomorphic, any Riemann surface is orientable.
	\end{poznamkain}
\end{definice}

\begin{priklad}[The complex projective line]
	Just as for $®P^1(®R)$, we define $®P^1(®C)$ to be the set whose elements are complex $1$-dimensional subspaces of $®C^2$.

	Let $U_1 = U_2 := ®C$ and define $g: U_1 \setminus \{0\} \rightarrow U_2 \setminus \{0\}$, $z \mapsto \frac{1}{z}$. We define $®P^1(®C)$ to be the quotient $®P^1(®C) := U_1 \coprod U_2 / (z \sim g(z))$.
\end{priklad}

\begin{priklad}[Show that]
	As a set $®P^1(®C)$ is ®C plus a point.

	As a topological space $®P^1(®C)$ is the one point compactification of ®C.

	Conclude from the previous sentence that $®P^1(®C)$ is homeomorphic to the two sphere.
\end{priklad}

\begin{poznamka}
	In complex analysis $®P^1(®C)$ is known as the Riemann sphere.
\end{poznamka}

\begin{poznamka}
	For $i = 1, 2$, we denote the image of $U_i$ in the quotient $U_1 \coprod U_2 / (z \sim g(z))$ by $[U_i]$. Note that $U_i$ define the local coordinate functions: $φ_i: [U_i] \rightarrow U_i$, $p \mapsto z_i$, where $z_i$ is the complex numbers in $U_i$ such that $[z_i] = p_i$. Both $φ_1$, $φ_2$ are homeomorphisms.

	We now consider the transition functions: the intersection
	$$ [U_1] \cap [U_2] = [U_i \setminus \{0\}] = [U_2 \setminus \{0\}]. $$
	The image of the intersection under $φ_1$ is $φ_1([U_i] \cap [U_2]) = ®C \setminus \{0\}$ (*). Thus (*) is the domain of our single transition function $T = φ_2 ∘ φ_1^{-1}$. For $z_1 ≠ 0$, we have $T: z_1 \overset{φ_1^{-1}}\mapsto [z_1] = [z_2 := g(z_1) = 1 / z_1] \overset{φ_1}\mapsto z_2 = 1 / z_1$. Thus
	$$ T: ®C \setminus \{0\} \rightarrow ®C \qquad z \mapsto \frac{1}{z}. $$
	Since $T$ has a pole only at $z_1 = 0$, we see that it is holomorphic on $®C \setminus \{0\}$. A symmetric (exchange 1 for 2) calculation shows that $T^{-1}$ is also holomorphic. So $®P^1(®C)$ is a Riemann surface.
\end{poznamka}

\begin{priklad}[Hopf fibration]
	Consider the 3-dimensional real sphere $S^3 \subseteq ®C^2 = ®R^4$. Given a point $p \in S^3$, $\exists$ a unique line $l_p$ through the origin and $p$. Thus we got a function $H: S^3 \rightarrow ®P^1(®C)$, $p \mapsto l_p$.

	Check that $H$ is continuous and surjective. Since $S^3$ is closed and bounded, it is compact. Moreover, since the image of a compact set under a continuous function is compact, $®P^1(®C)$ is compact.

	What is the fiber of the surjective map $H: S^3 \rightarrow ®P^1(®C)$, i.e. what is $H^{-1}(p)$, for any point $p \in ®P^1(®C)$. (Hint: It is $S^1$.)

	This gives us $S^2 \times S^1 = S^3$ as set. (Not as topological space!)
\end{priklad}

\begin{priklad}[Complex tori]
	Definition: Let $τ_1$ and $τ_2$ be two complex numbers, that are linearly independent. The set of all integral linear combinations of $τ_1$ and $τ_2$:
	$$ Λ := \{nτ_1, mτ_2 | n,m \in ®Z\} \subseteq ®C $$
	is called the lattice of $τ_1$ and $τ_2$.

	Observe that we can assume that $τ_1 = 1$, and $\Im(τ_2) > 0$, allowing to make simplifying assumption: that our lattice has the form $Λ = \{n + mτ | n, m \in ®Z, τ \in ®H\}$, where $®H$ is the upper half plane.

	Consider the quotient space $T = ®C / Λ$. THat is the quotient space with respect to the equivalence relation $z_2 \sim z_1 \Leftrightarrow z_2 = z_1 + w$ for $w \in Λ$.

	The canonical projection map $π: ®C \rightarrow T$ (i.e. $π(z) = [z]$) induces a quotient topology on $T$ (i.e. $V \subseteq T$ is open iff $p^{-1}(V)$ is open in ®C).
\end{priklad}

\begin{priklad}
	For $P$ the closed parallelogram with vertices $0, 1, τ, 1+τ$, show that for any $z \in ®C\ \exists z' \in P$ that $π|_p \rightarrow T$ is surjective. Hence we can restrict our attention to $p$.
\end{priklad}

%%% 07. 11. 2023

% 14. 11. 2023

\begin{poznamka}
	By considering the identification points in $p$, we see that $T$ is topologically a torus.
\end{poznamka}

\begin{priklad}
	Prove that $π$ (from previous exercise) is an open map, i.e. that $V$ an open subset of ®C implies that $π(V)$ is open in $T$.
\end{priklad}

\begin{poznamka}
	Now to the complex structure: from the previous exercise, we see that if $π$ restricted to a subset $V \subseteq ®C$ is bijective, then it is a homeomorphism onto its image in $T$. In this case, $(π|_{V})^{-1}$ is also a homeomorphism from the image of $π|_V$ to $V$. Hence we can use $(π|_V)^{-1}$ as a chart for $T$.
\end{poznamka}

\begin{priklad}
	Find a real number $r$ (depends on $t$) such that for any $z \in ®C$: $π$ restricted to $B_r(z)$ is a bijective map.

	Given this $r$, define $U_z := π(B_r(z)) \subseteq T$ and $φ_z := (π|_{B_r(z)})^{-1}$. We claim that the collection $©A = \{U_z, φ_z | z \in ®C\}$ forms an atlas for $T$. It is clear that ©A gives a cover for $T$. Moreover, by definition the maps $φ_z$ are homeomorphic to their images. Assume that $U_{z_1} \cap U_{z_2} ≠ \O$. For $j \in [2]$ denote by $(α_j, β_j)$ the unique pair of real numbers such that $z_j = α_j + t β_j$. We have that $T_{21}(z) = \(φ_{z_1} ∘ φ_{z_2}^{-1}\) (z) = z + k$, where $k = ([α_2] - [α_1]) + ([β_1] - [β_2]) t$ is just a constant depending on $z_1$ and $z_2$. Therefore the transition function $T_{21}$ is holomorphic $\implies$ $T$ is a Riemann surface.
\end{priklad}

\section{Graph of complex functions}
\begin{definice}[Graph]
	Let $f: ®C \rightarrow ®C$ be a continuous function. The graph of $f$ is the set
	$$ Γ_f := \{(z, f(z)) | z \in ®C\} \subseteq ®C \times ®C, $$
	given the subset topology.
\end{definice}

\begin{poznamka}
	Note that $Γ_f$ is Hausdorff since $®C \times ®C$ is Hausdorff. The graph of $f$ is naturally given the structure of a Riemann surface by an atlas, with one chart, namely $Γ_f$: the local coordinate function is the projection map $φ := π_1|_{Γ_f}$, i.e. $(z, f(z)) \mapsto z$. TODO!!!(Jedna celá tabule)
\end{poznamka}

\begin{definice}[Affine plane curve]
	For any polynomial $p(x, y) \in ®C[x, y]$, the set $V(p) := \{(x, y) | p(x, y) = 0\} \subseteq ®C^2$, is called an affine plane curve. We say that $V(p)$ is smooth if $\nexists (x_0, y_0) \in V(p)$ such that $\frac{\partial p}{\partial x}(x_0, y_0) = 0 = \frac{\partial p}{\partial y}(x_0, y_0)$.
\end{definice}

\begin{veta}
	A smooth affine plane curve is a Riemann surface.

	\begin{dukazin}
		Let $(x_0, y_0) \in V(p)$. Since $V(p)$ is smooth, then for at least one of $\frac{\partial p}{\partial x}$, $\frac{\partial p}{\partial y}$ is non-zero at $(x_0, y_0)$. Assume (WLOG) that $\frac{\partial p}{\partial y}(x_0, y_0) ≠ 0$. Then by the implicit function theorem, exists a neighbourhood $U_{(x_0, y_0)} \subseteq ®C^2$, and a neighbourhood $V_{x_0} \subseteq ®C$ and a holomorphic function $f: V_{x_0} \rightarrow ®C$ such that $V(p) \cap U_{(x_0, y_0)} = \{(x, f(x)) | x \in V_{x_0}\}$. We call this the graph of $f$.

		We get a local chart on $V(p)$ around $(x_0, y_0)$ (as in the previous example) by setting $φ_{(x_0, y_0)}: V(p) \cap U_{(x_0, y_0)} \rightarrow V_{x_0}$, $(x, f(x)) \mapsto x$. Finally, we show that the transition functions are holomorphic: for $U_{(x_0, y_0)} \cap U_{(x, y)} \cap V(p) ≠ 0$, if $φ_{(x_0, y_0)}$ and $φ_{(x, y)}$ are both projections to the same axis, the transition function $φ_{(x_0, y_0)} ∘ φ_{(x_0, y_0)}^{-1}$ is the identity function restricted to the appropriate domain in ®C. Assume now that $φ_{(x_0, y_0)}$ is projection onto the $x$-axis and that $φ_{(x, y)}$ is projection to the axis $y$. Then set $U_{(x_0, y_0)} \cap U_{(x, y)} \cap V(p)$ is simultaneously on the graph of a holomorphic function $f_0$ and of a holomorphic function $f_1$. Then functions all $φ_{(x, y)} ∘ φ_{(x_0, y_0)}^{-1} = f_0(x)$ and $φ_{(x_0, y_0)} ∘ φ_{(x, y)}^{-1} = f_1(x)$ restricted to the appropriate domains, which are holomorphic.
	\end{dukazin}
\end{veta}

\section{Projective curves}
\begin{priklad}
	Consider the polynomial $p(x, y, z) = x^2 + y + z + 1$. Note that $p(1, 1, 1) = 4 ≠ 7 = p(2, 2, 2)$ since $[1, 1, 1] = [2, 2, 2]$ in $®P^2(®C)$ $p$ does not restrict to $®P^2(®C)$.
\end{priklad}

\begin{definice}[Homogeneous polynomial]
	A polynomial $p \in ®C[x, y, z]$ is said to be homogeneous of degree $l$, if the following equivalent conditions hold
	\begin{itemize}
		\item every monomial of $p$ has degree $l$;
		\item for each $t \in ®C$: $p(tx, ty, tz) = t^l p(x, y, z)$;
		\item $x \frac{\partial p}{\partial x} + y \frac{\partial p}{\partial y} + z \frac{\partial p}{\partial z} = lp$.
	\end{itemize}

	\begin{dusledekin}
		If $p$ is homogeneous, $V(p) \subset ®P^2(®C)$ is well-defined.
	\end{dusledekin}
\end{definice}

% 15. 11. 2023

\begin{priklad}
	Confirm that these three conditions are equivalent.
\end{priklad}

\begin{priklad}
	Show that if $p \in ®C[x, y, z]$ is a homogeneous polynomial, then the set of points $[x, y, z] \in ®P^2(®C)$ satisfying $p(x, y, z) = 0$ is well-defined.
\end{priklad}

\begin{definice}
	We call
	$$ V(p) := \{[x, y, z] \in ®P^2(®C) | p(x, y, z) = 0\} $$
	the vanishing locus of $p$. Moreover, we call $V(p)$ a (plane) projective curve of degree $l$.

	If
	$$ \{(x, y, z) \in ®C^3 | \frac{\partial p}{\partial x} = \frac{\partial p}{\partial y} = \frac{\partial p}{\partial z} = 0\} $$
	the $V(p)$ is said to be smooth.
\end{definice}

\begin{tvrzeni}
	A smooth projective plane curve $V(p)$ is a compact Riemann surface.

	\begin{dukazin}
		We first show that $V(p)$ is compact by showing that $V(p)$ is closed set in $®P^2(®C)$ which is a compact space.

		Consider the diagram $®P^2(®C) \overset{π}\leftarrow ®C^3 \setminus \{(0, 0, 0)\} \overset{p}\rightarrow ®C$, where $π$ is the natural projection function and $p$ is the continuous function defined by the homogeneous polynomial $p: ®C^3 \setminus \{(0, 0, 0)\} \rightarrow ®C$, $(x, y, z) \mapsto p(x, y, z)$ by definition $V(p)$ is a closed subset of $®P^2(®C)$ if $π^{-1}(V(p))$ is closed in $®C^3 \setminus \{(0, 0, 0)\}$. But $π^{-1}(V(p)) = p^{-1}(0)$ is the inverse image of the closed set $\{0\} \subseteq ®C$. Thus since $p$ is continuous, $p^{-1}(0)$ is closed, in other words, $π^{-1}(V(p))$ is closed. Thus $V(p)$ is compact.

		So, to show that $V(p)$ is Riemann surface, we need to show that its intersection with any of the coordinate open sets of $®P^2(®C)$ is a Riemann surface. So let us consider (WLOG) the chart $U_z = \{[x, y, z] | z ≠ 0\} \subseteq ®P^2(®C)$ with affine coordinates $φ_z(x, y, z) = \(\frac{x}{z}, \frac{y}{z}\)$. The set $φ_z(V(p) \cap U_z)$ is equal to $V(\tilde p)$ where $\tilde p(x, y) = p(x, y, 1)$.

		Now for any $(x, y) \in ®C^2$:
		$$ (*): \frac{\partial \tilde p}{\partial x}(x, y) = \frac{\partial p}{\partial x}(x, y, 1), \qquad (**): \frac{\partial \tilde p}{\partial y}(x, y) = \frac{\partial p}{\partial y}(x, y, 1). $$
		We claim $\nexists$ an $(\hat{x}, \hat{y}) \in ®C^2$ such that
		$$ \tilde p(\hat{x}, \hat{y}) = \frac{\partial \tilde p}{\partial x}(\hat{x}, \hat{y}) = \frac{\partial \tilde p}{\partial y}(\hat{x}, \hat{y}) = 0. $$
		This claim implies that $V(\tilde p)$ is a smooth affine plane curve and hence a Riemann surface. Since the restriction of $V(p)$ with any affine chart is a Riemann surface, then so is $V(p)$.

		So it remains to prove the claim: Assume $\exists (\hat{x}, \hat{y}) \in ®C$ satisfying condition above. By $(*)$ and $(**)$, together smoothness of $V(p)$, which would imply that $\frac{\partial p}{\partial z}(\hat{x}, \hat{y}, 1) ≠ 0$. But now Euler's identity implies $0 ≠ \frac{\partial p}{\partial x}(\hat{x}, \hat{y}, 1) + \frac{\partial p}{\partial y}(\hat{x}, \hat{y}, 0) + \frac{\partial p}{\partial z}(\hat{x}, \hat{y}, 1) = l p(\hat{x}, \hat{y}, 1) = 0$ $\implies$ contradiction $\implies$ we are done.
	\end{dukazin}
\end{tvrzeni}

\begin{priklad}
	Confirm that $V(p)$ is Hausdorff.
\end{priklad}

\begin{priklady}[Elliptic curves]
	Consider a polynomial $p$ of the form $p(x, y, z) = y^2z - (x - α_1z)(x - α_2z)(x - α_3z)$ where $α_1, α_2, α_3 \in ®C$ are distinct complex numbers. Note that the partial derivative with respect to $y$ satisfies $\frac{\partial p}{\partial y} = 2yz$, which is zero only if $y = 0$ or $z = 0$. We show that $V(p)$ is a smooth projective curve by considering the case $z = 0$, $y = 0$, and finding in each chase a non-vanishing partial derivative.

	„Case $z = 0$“: Then the only part in $®P^2(®C)$ belonging to $V(p)$ is $[0, 1, 0]$. But we have $\frac{\partial p}{\partial z} = y^2 + Q(x, z) = 1 + 0 ≠ 0$.

	„Case $y = 0$“: Then the parts belonging to $V(p)$ are $[α_1, 0, 1]$, $[α_2, 0, 1]$, $[α_3, 0, 1]$. For $j \in [3]$: $\frac{\partial p}{\partial x}(α_j, 0, 1) ≠ 0$, follows from the fact that the $α_1$, $α_2$, $α_3$ are distinct.

	So $V(p)$ is a smooth projective curve of degree 3.
\end{priklady}

\begin{priklady}
	Consider the function $φ: ®P^1(®C) \rightarrow ®P^3(®C)$ defined in homogeneous coordinates by $φ[s, t] = [s^3, s^2t, st^2, t^3]$. We call the image of $φ$ the twisted cubic in $®P^3(®C)$.
\end{priklady}

% 21. 11. 2023

\section{Holomorphic maps of Riemann surfaces}
\begin{definice}
	Let $X$, $Y$ be two Riemann surfaces, and $f: X \rightarrow Y$ a function (of sets).

	\begin{itemize}
		\item We say that $f$ is holomorphic at $x \in X$ if for every choice of charts $φ_x$, $φ_{f(x)}$, the function $φ_{f(x)} ∘ f ∘ φ_x^{-1}$ is holomorphic.
		\item If $U \subseteq X$ is open, we say that $f$ is holomorphic on $U$ if $f$ is holomorphic at each $x \in U$.
		\item If $f$ is holomorphic on $U = X$, then we say that $f$ is a holomorphic map.
	\end{itemize}

	The function $F := φ_{f(x)} ∘ f ∘ φ_x^{-1}$ is called a local expression for $f$.
\end{definice}

\begin{priklad}
	Show that a map of Riemann surfaces $f: X \rightarrow Y$ is holomorphic at $x \in X$ $\Leftrightarrow$ $\exists$ a choice of charts $φ_x$, $φ_{f(x)}$ such that $φ_{f(x)} ∘ f ∘ φ_x^{-1}$ is holomorphic at $x$.
\end{priklad}

\begin{priklad}[Eg]
	Let $X$, $Y$ be two Riemann surfaces and choose a point $y \in Y$. Denote associated constraint map $c: X \rightarrow Y$, $x \mapsto y$. We see that $c$ is a holomorphic map.
\end{priklad}

\begin{priklad}[Eg]
	Let $X$ be a Riemann surface. The identity map on $X$ is the function $I_X: X \rightarrow X$, $x \mapsto x$. We see that $I_x$ is a holomorphic map.
\end{priklad}

\begin{priklad}[Eg]
	Consider an arbitrary rational function $f(z) = \frac{p(z)}{q(z)}$, where $p(z), q(z) \in ®C[z]$ are two polynomials with distinct roots. We can extend $f$ to a function from $®P^1(®C)$ to $®P^1(®C)$ as follows:
	\begin{itemize}
		\item $f(α) := ∞$ (point at infinity of the Riemann sphere) where $α$ is a root of $q$;
		\item $f(∞) := \lim_{z \rightarrow ∞} \frac{p(z)}{q(z)}$.
	\end{itemize}

	Show that $f$ is a holomorphic function from $®P^1(®C)$ to $®P^1(®C)$.
%\end{priklad}

%\begin{priklad}[Eg]
	Recall that $®P^1(®C)$ is $®C \cup ∞$ with a topology. Identifying ®C with image of the first affine chart $φ_1([U_1])$, the additional chart point corresponds to the image of zero in the second affine chart. We denote this point by $∞$.

	This we have that $®P^1(®C) = ®C \cup \{∞\}$. Using this identification, define the function $f: ®P^1(®C) \rightarrow ®P^1(®C)$, $z \mapsto z^2 =: ω, ∞ \mapsto ∞$. We will now show that $f$ is a holomorphic map.

	We denote by $z$ the local coordinate on the source and by $ω$ the local coordinate on the range. Then since $ω = f(z) = z^2$ is a holomorphic function on all of ®C, $f$ is holomorphic on the image of $U_1$.

	It remains to show that $f$ is holomorphic at $∞$. We consider the local expression for $f$ using the chart $U_2$ whose image contains $∞$. We denote by $\tilde z := \frac{1}{z}$ the corresponding local coordinate for the source and $\tilde ω := \frac{1}{ω}$ for the range.

	The local expression $\tilde F$ for $f$ in these coordinates is then obtained by composing $F(z)$ with the transition functions for the local coordinate:
	$$ \tilde F(\tilde z) = \tilde ω = \frac{1}{ω} = \frac{1}{z^2} = (\tilde z)^2. $$
	Since the point at $ω$ corresponding to $\tilde z = \tilde ω = 0$, and we have that $f(∞) = ∞$ and $\tilde F(0) = 0$, the local expression extends to the whole chart, and in particular, it is a holomorphic function at $0$. So $f$ is holomorphic at every point of $®P^1(®C)$.
\end{priklad}

\begin{definice}
	Two Riemann surfaces $X, Y$ are called isomorphic (or bi-holomorphic) if $\exists$ holomorphic maps $f: X \rightarrow Y$ and $g: Y \rightarrow X$ such that $g ∘ f = I_X$ and $f ∘ g = I_Y$. In this case we write $X \simeq Y$. We call $f$ and $g$ isomorphisms (or bi-holomorphisms). An isomorphism $h: X \rightarrow X$ from a Riemann surface to itself is called an automorphism.
\end{definice}

\begin{priklad}
	Let $f: ®C \rightarrow ®C$ be a holomorphic function and $Γ_f \subseteq ®C^2$ its graph. Show that $Γ_f \simeq ®C$.
\end{priklad}

\begin{definice}
	We say that a chart $(U_x, φ_x)$ for a Riemann surface is centered at $x$ if $φ_x(x) = 0$.
\end{definice}

\begin{veta}
	Let $f: X \rightarrow Y$ be a non-constant holomorphic map of Riemann surfaces. For any $x \in X$ $\exists$ charts centered at $x$ such that the local expression of $f$ in terms of these charts is $z \mapsto z^h$, for some $h ≥ 1$.

	\begin{dukazin}
		Let $φ$ and $ψ$ be two charts centered at $x$, and $f(x)$, respectively. This gives the local expression $F = ψ ∘ f ∘ φ^{-1}$. Consider the Taylor expansion of $F$ at 0 and let $k$ be the smallest positive integer such that the coefficient of $z^h$ does not vanish. Since $F(0) = 0$, $k > 1$, and $F(z) = z^k \(\sum_{n=0}^∞ a_{k + n} z^n\)$. Denote by $G(z) = a_k + a_{k+1} z + …$ the second factor in that equation. The function $G(z)$ is holomorphic at $0$ and $G(0) = a_k ≠ 0$. Thus we can make a choice of root $\sqrt[n]{G(z)}$ such that the associated map is well-defined and holomorphic around $0 = φ(z)$.

		Defining $h(z) = z·\sqrt[k]{G(z)}$ we have that $F = h^k$. The function $h$ is holomorphic in a neighbourhood $U$ of $0 = φ(x)$, $h(0) = 0$, and $h'(0) = 0$. The inverse function theorem now implies that $h$ is bi-holomorphic on a neighbourhood $U' \subseteq U$ of $φ(x)$, and therefore the composition $\tilde φ = h ∘ φ$ gives local chart for $x$ centered at $x$. The local coordinate $\tilde z$ coming from $\tilde φ$ is related to $z$ with $\tilde z = h(z)$. The local expression for $f$ allows $x$ is now obtained by changing coordinates from $z$ to $\tilde z$ in $F$:
		$$ \tilde F(\tilde z) = F(z(\tilde z)) = h(z)^k = (\tilde z)^k. $$
	\end{dukazin}
\end{veta}

\begin{definice}
	Let $f: X \rightarrow Y$ be a non-konstant holomorphic map of Riemann surfaces
	\begin{itemize}
		\item Given a point $x \in X$, the integer $k_x$ such that $\exists$ a local expression centered at $x$ of the form $F(z) = z^{k_x}$ is called the Ramification index of $f$ at $x$.
		\item The quantity $V_x = k_x - 1$ the differential length of $f$ at $x$.
		\item A point $x$ with Ramification index $k_x = 1$ is called unramified.
		\item A point for which $k_x ≥ 2$ is ramified (ramification point).
		\item If $x$ is a ramification point, then $f(x) \in Y$ is called a branch point.
%
% 28. 11. 2023
%
		\item The ramification locus is the subset of $X$ consisting of all ramification points of $f$.
		\item The branch locus is $f(R)$.
	\end{itemize}
\end{definice}

\begin{priklad}
	Let $f: X \rightarrow Y$ be a non-constant holomorphic map of two Riemann surfaces, and let $x \in X$. Suppose that $f$ has two local expressions around $x$ of the form $F(z) = z^k$, $\tilde F(\tilde z) = \tilde z^{\tilde k}$, then $k = \tilde k$.

	\begin{reseni}[Hint]
		Let $φ$, $\tilde φ$, $ψ$, $\tilde ψ$ be charts. Observe that the change of coordinates near $x$ must be of the form $\tilde z = z α(z)$, where $α(z)$ is holomorphic in a neighbourhood of 0, and $α(0) ≠ 0$. (Try to draw this in picture.)

		Similarly (but going in the opposite direction) $ω = \tilde ω β(\tilde ω)$ for some holomorphic function $β$ with $β(0) ≠ 0$. Write down the Taylor expansions of $α$ and $β$ and use the fact that $F$ is obtained from $\tilde F$ via the above changes of coordinates.
	\end{reseni}
\end{priklad}

\begin{lemma}
	The ramification locus $R$ is a discrete subset of open subsets (i.e. $\exists U_i \subset X$, such that $U_i$ contains exactly one element of $R$).

	\begin{poznamkain}
		A discrete subset of a compact set is finite (an infinite set has a limit point, thus a discrete subset, which has no limit points, must be finite).
	\end{poznamkain}
\end{lemma}

\begin{lemma}
	If $X$ is a compact Riemann surface and $f: X \rightarrow Y$ is a non-constant holomorphic map of two Riemann surfaces, the ramification locus $R$ is a finite set. Moreover, the branch locus is also finite.
\end{lemma}

\section{Holomorphic maps between compact Riemann surfaces}
\begin{veta}
	Let $f: X \rightarrow Y$ be a holomorphic of Riemann surfaces. If $X$ is compact and $f$ is non-constant, then $f$ is onto / surjective.

	\begin{dukazin}
		Consider the image $f(X) \subset Y$. By Liouville's theorem (from complex analysis) $f(x)$ is open. On the other hand, since $X$ is compact and $f$ is continuous, $f(X)$ is compact. Since $Y$ is Hausdorff, we must have that $f(x)$ is closed.

		Finally, since $f(X)$ is an open, closed, and non-empty subset of a connected topological space, we must have that $f(X) = Y$.
	\end{dukazin}
\end{veta}

\begin{dusledek}
	Hence $Y$ is also compact.
\end{dusledek}

\begin{dusledek}
	If $X$ is a compact Riemann surface, the only holomorphic functions $f: X \rightarrow ®C$ are the constant functions.
\end{dusledek}

\begin{poznamka}[Discusion]
	Let $f: X \rightarrow Y$ be a non-constant map of compact Riemann surfaces. Consider a point $x \in f^{-1}(y)$. We know that $\exists$ a neighbourhood $U_x$ and coordinate charts centered at $x$ such that we have local expression $z \rightarrow z^{k_x}$.

	Since $F^{-1}(0) = 0$, there are no other pre-images of $y$ in $U_x$. Thus the pre-image set $f^{-1}(y)$ is discrete, and since $X$ is compact, the pre-image is a finite set.
\end{poznamka}

\begin{veta}
	Let $f: X \rightarrow Y$ be a non-constant holomorphic map of compact Riemann surfaces. If $y_0, y_1$ are not in the branch locus of $f$ then $|f^{-1}(y_0)| = |f^{-1}(y_1)|$.

	\begin{dukazin}
		Next week.
	\end{dukazin}
\end{veta}

\begin{definice}
	For $f: X \rightarrow Y$ a non-constant map of compact Riemann surfaces, the degree of $f$ is the cardinality of the fiber $f^{-1}(y)$,for any $y \notin$ branch locus of $f$. If $f$ is constant, then we say that its degree is 0.
\end{definice}

% 05. 12. 2023

\begin{dukaz}[next week proof]
	Let $B$ be the branch locus of $f$. Since $B$ is finite, $Y \setminus B$ is a connected topological space ($Y$ is a surface). Hence it cannot have a proper subset which is both closed and open. Let $y_0 \in Y \setminus B$ and set $d := |f^{-1}(y_0)|$.

	The set $A := \{y \in Y \setminus B \middle| |f^{-1}(y)| = d\}$ is non-empt since it contains $y_0$. We claim that „$A$ is open in $Y \setminus B$“: For any $y \in A$ we show that $y$ is an interior point of $A$. Denote by $\{x_1, …, x_d\}$ the inverse images of $y$ and by $U_{x_1}, …, U_{x_d}$ pairwise disjoint charts around each inverse such that $f$ admits the local expression $F(z) = z$ on each chart. We can choose (see exercise below) $V_y \subseteq F(U_{x_1}) \cap … \cap f(U_{x_d})$ a connected neighbourhood of $y$ homeomorphic to a disk, such that the inverse image $f^{-1}(V_y)$ consists of $d$ connected components $\tilde U_{x_i}$ (each contains one of the $x_i$ and is contained in $U_{x_i}$). Then $f$ restricted to each $\tilde U_{x_i}$ is bijective onto $V_y$. This implies that every point of $V_y$ has $d$ inverse images, and hence that $y$ is an interior point for $A$.

	Now consider $A^c = \{y \in Y \setminus B \middle| |f^{-1}(y)| ≠ d\}$. With essentially the same argument, one shows that $A^c$ is also open in $Y \setminus B$. Thus we can conclude that $A$ is closed. Since $A$ is a non-empty, open and closed subset of a connected topological space, $A = Y \setminus B$.
\end{dukaz}

\begin{priklad}
	Show that a $V_y$ (as in the previous proof) exists by showing that otherwise one may construct a sequence $\{ξ_n\}$ of points of $X$ such that $ξ_n \rightarrow ξ$, $f(ξ_n) \rightarrow y$ but $f(ξ) ≠ y$ violating continuity.
\end{priklad}

\begin{definice}[Degree]
	For $f: X \rightarrow Y$ a non-constant holomorphic map of compact Riemann surfaces, the degree of $f$ is the cardinality of the fiber $|f^{-1}(y)|$, for any $y \in Y$ such that $y \notin $ branch locus. If $f$ is constant then we say that it has degree 0.
\end{definice}

\begin{priklad}
	Let $f: X \rightarrow Y$ be a holomorphic map of compact Riemann surfaces, of degree $d > 0$, $y \in Y$ and $f^{-1}(y) = \{x_1, …, x_n\}$. Prove that $\sum_{i=1}^n k_{x_i} = d$.
\end{priklad}

\begin{poznamka}[Recall the classification of compact Riemann surfaces]
	Any compact, connected surface is homeomorphic to one surface in the following list: $S^2$ (two sphere, also ®C-projective plane $®P®C$), $T^{\# g}$, $®P^2(®R)^{\# n}$.

	The first two are Riemann surfaces (Ex: the connected sum of Riemann surfaces is again a Riemann surface). The third are not orientable, so not Riemann surfaces.
\end{poznamka}

\begin{definice}[Genus]
	Define the genus of a compact connected Riemann surface as 0 for 2 sphere and $g$ for connected sum of $g$ tori.

	\begin{poznamkain}[Alternative definition of genus]
		A vector field $X$ on a differential manifolds is a derivation on $C^∞(m)$ i.e. a ®R linear map $X: C^∞(m) \rightarrow C^∞(m)$ such that$X(f·g) = X(f)·g + f·X(y)$ $\forall f, g \in C^∞(m)$. The space of vector fields is denoted by $©X(m)$. Note that $©X(m)$ is a $C^∞(m)$-module. $(fX)(g) = f(X(g))$. ($(fX)(g·h) = f(X(g·h)) = f(X(g)·h + g·X(h)) = f(X(g))·h + g·f(X(h))$.)

		Denote $Ω^1(m) = ©X(m)^* = Morf_{C^∞(m)}(©X(m), C^∞(m))$. This is again $C^∞(m)$-module (just as for any dual module)
		$$ (fω)(X) := f(ω(X)) \qquad \((fω)(gX) = f(ω(gX)) = g(fω(X)) = g((fω)(X))\). $$
		Consider the map $d: C^∞(m) \rightarrow Ω^1(m), \qquad f \mapsto df$, where $df: ©X(m) \rightarrow C^∞(m), X \mapsto X(f)$. (Note that $d(fg)(X) = X(fg) = X(f)·g + f·X(g) = df(x)·g + fdg(X) = (g·df)(X) + (f·dg)(x) \implies d(f·g) = g·df + f·dg$.)

		Now let's restrict to surfaces:
		$$ Ω^2(m) := \frac{Ω^1(m) \otimes_{C^∞(m)} Ω^1(m)}{\{ω \otimes ν + ν \otimes ω\}}. $$
		We denote the product in $Ω^2(m)$ by $\wedge$. $\implies$ $ω \wedge ν = - ν \wedge ω$.i

		TODO example?
	\end{poznamkain}
\end{definice}

\begin{veta}[Riemann–Hurwitz]
	For $f: X \rightarrow Y$ be non-constant, degree $d$, holomorphic map of compact Riemann surfaces. Denote by $g_x$, resp. $g_y$, the genus of $X$, and $Y$. Then $2 g_X - 2 = d(2g_Y - 2) + \sum_{x \in X} v_x$, where $v_x = h_x - 1$ is the differential length of $f$ at $x$.
\end{veta}

\begin{priklad}
	Observe that $\sum_{x \in X} v_x = \sum_{x \in \text{Ram. locus}} v_x$ and hence conclude that this sum is well-defined.
\end{priklad}





\end{document}
