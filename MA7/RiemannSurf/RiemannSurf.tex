\documentclass[12pt]{article}					% Začátek dokumentu
\usepackage{../../MFFStyle}					    % Import stylu



\begin{document}

% 03. 10. 2022

\section{Introduction}
\begin{poznamka}[Literature]
	„Riemann surfaces and algebraic curves“, Renzo Cavalieri and Eric Miles
\end{poznamka}

\subsection{Differentiability}
\begin{definice}[Differentiable]
	A function $f: ®C \rightarrow ®C$ is differentiable (also holomorphic) at a point $z_0 \in ®C$ if the following limit exists
	$$ \lim_{|h| \rightarrow 0} \frac{f(z_0 + h) - f(z_0)}{h} =: f'(z_0) \in ®C. $$

	We call $f'(z_0)$ the derivative of $f$ at $z_0$. A function $f$ is differentiable on a domain (open connected subset of ®C) if its differentiable for all points of this domain.
\end{definice}

\begin{poznamka}[Writing complex numbers in cartesian cooridnates]
	$z = x + iy$, for $x, y \in ®R$, we can write a function $f: ®C \rightarrow ®C$ in terms of two functions $u, v: ®R^2 \rightarrow ®R$ such that
	$$ f(x, y) = u(x, y) + i·v(x, y). $$
\end{poznamka}

\begin{veta}[Cauchy–Riemann equations]
	Let $f: ®C \rightarrow ®C$ be a holomorphic function on an open subset of ®C. Considering $f = u + iv$, then
	$$ \frac{\partial u}{\partial x} = \frac{\partial v}{\partial y}, \qquad \frac{\partial v}{\partial x} = -\frac{\partial u}{\partial y}. $$
\end{veta}

\begin{definice}[Orientability, orientation-preserving function]
	Define and equivalence relation on the set of all bases of $®R^2$ by saying that $B_1 \sim B_2$ iff the determinant of the change of basis matrix is positive.

	A function $f: ®R^2 \supset U \rightarrow ®R^2$ is said to be orientation-preserving if on an open dense subset of $U$, the determinant of the Jacobi matrix is positive. Jacobi matrix:
	$$ J(f) = \begin{pmatrix} \frac{\partial u}{\partial x} & \frac{\partial u}{\partial y}\\ \frac{\partial v}{\partial x} & \frac{\partial v}{\partial y} \end{pmatrix}. $$
\end{definice}

\begin{dusledek}
	Let $f$ be a non-constant holomorphic function, then $f$ is orientation-preserving.

	\begin{dusledek}
		Since $f$ is holomorphic, the Cauchy-Riemann equations implies that
		$$ \det(J(f)) = \frac{\partial u}{\partial x} \frac{\partial v}{y} - \frac{\partial v}{\partial x} \frac{\partial u}{\partial y} \overset{\text{C–R}}= \(\frac{\partial v}{\partial x}\)^2 + \(\frac{\partial v}{\partial y}\)^2 ≥ 0. $$
		Since $f$ is non-constant, the inequality is strict on a dense open subset of the domain of definition.
	\end{dusledek}
\end{dusledek}

\begin{veta}[Open mapping theorem]
	A non-constant holomorphic function $f$ is open (that is if $U$ is an open subset of ®C, then $f(U)$ is also open).
\end{veta}

\subsection{Integration}
\begin{definice}
	For a path $γ$ (smooth function, $γ: ®R \supset [a, b] \rightarrow ®C$) we define
	$$ \int_γ f(x) dx := \int_a^b f(γ(t))·γ'(t) dt $$
\end{definice}

% 10. 10. 2023

\begin{definice}[Continuous deformation]
	For $γ, μ: [a, b] \rightarrow U$ ($U$ simply connected), paths with the same endpoints ($γ(a) = μ(a)$ and $γ(b) = μ(b)$). Then a continuous deformation $γ$ into $μ$ is a continuous function $H: [a, b] \times [0, 1] \rightarrow U \subseteq ®C$ such that $H(s, 0) = γ(s)$, $H(s, 1) = μ(s)$, $H(a, t) = z_a := γ(a) = μ(a)$ and $H(b, t) = z_b := γ(b) = μ(b)$.
\end{definice}

\begin{veta}
	Suppose that $γ, μ: [a, b] \rightarrow U$ ($U$ simply connected) are related by a continuous deformation of paths $H$. Then for any holomorphic function $f$ on $U$ we have
	$$ \int_γ f(z)dz = \int_μ f(z) dz. $$

	\begin{dukazin}[Partial proof assuming $H$ admits partial derivatives]
		For any $t \in [0, 1]$ we integrate the function $INT(t) = \int_{H(·, t)} f(z) dz$. Consider the derivative of $INT(t)$ with respect to $t$:
		$$ \frac{d}{dt} (INT(t)) = \frac{d}{dt} \(\int_a^b f(H(s, t)) \frac{\partial H}{\partial s}(s, t) ds\) \overset{\text{Leibniz + chain rule}}=  $$
		$$ = \int_a^b f'(H(s, t)) \frac{\partial H}{\partial t}(s, t)·\frac{\partial H}{\partial s}(s, t) + f(H(s, t)) \frac{\partial^2 H}{\partial s \partial t}(s, t) ds = $$
		$$ = \int_a^b \frac{d}{ds}\[f(H(s, t)) \frac{\partial H}{\partial t}\]ds = $$
		$$ = f(H(s, t)) \frac{\partial H}{\partial t} |_{s=a}^{s = b} \overset{\text{constant endpoints}}= 0. $$

		Having derivative identically equal to 0, means that $INT(t)$ is a constant function and $\int_γ f(z)dz = INT(0) = INT(1) = \int_μ f(z) dz$.
	\end{dukazin}
\end{veta}

\begin{dusledek}
	Let $U$ be a simply connected subset of ®C and $f: U \rightarrow ®C$ a holomorphic function. For any closed path whose image is inside $U$, $\int_γ f(z) dz = 0$.

	\begin{dukazin}[Sketch]
		The definition of simply connected is (essentially) the same as saying that any closed path can be continuously deformed to a constant path $c$.
		$$ \int_γ f(z) dz = \int_c f(z) dz = \int_a^b f(c(z))·c'(z) dz = \int_a^b f(c(z)) · 0 dz = 0 $$
	\end{dukazin}
\end{dusledek}

\begin{priklad}
	Let $U$ be a simple connected domain and $f: U \rightarrow ®C$ a holomorphic function on $U \setminus \{z_0\}$. For $j = 1, 2$, let $γ_j$ be a path parametrizing a circle centered at $z_0$ of radius $r_j$, oriented counterclockwise and completely contained in $U$. Show that $\oint_{γ_1} f(z) dz = \oint_{γ_2} f(z) dz$.
\end{priklad}

\subsection{Cauchy's integral formula}
\begin{veta}[Cauchy's integral formula]
	Let $γ$ be a loop around $z \in ®C$, and $f: U \rightarrow ®C$ a holomorphic function. For $U$ a neighbourhood of $γ$,
	$$ f(z) = \frac{1}{2πi} \oint_γ \frac{f(w)}{w - z} dw. $$

	\begin{dukazin}
		Conway 1978, Chapter IV.
	\end{dukazin}
\end{veta}

\begin{dusledek}
	$$ f(z) = \frac{1}{2πi} \oint_γ \frac{f(w)}{w - z_0 + z_0 - z} dw = \frac{1}{2πi} \oint \frac{f(w)}{w - z_0}·\(\frac{1}{1 - \frac{z - z_0}{w - z_0}}\)dw = $$
	$$ = \frac{1}{2πi} \oint_γ \frac{f(w)}{w - z_0} \(\sum_{n=0}^∞ \frac{(z - z_0)^n}{(w - z_0)^n}\) dw = $$
	$$ = \sum_{n=0}^∞ \(\frac{1}{2πi} \oint_γ \frac{f(w)}{(w - z_0)^n}\) (z - z_0)^n. $$
	For sufficiently "small" (shrunken) $γ$. So $f$ is smooth (infinitely differentiable). Moreover, it is analytic (that is, its Taylor expansion around $z_0$ converges to $f$ in a neighbourhood of $z_0$).
\end{dusledek}

\begin{definice}[Pole]
	Given a positive integer $n$, a complex function $f$ has pole of order $n$ at the point $z_0 \in ®C$ if $(z - z_0)^n f(z)$ is holomorphic at $z_0$ but $(z - z_0)^{n - 1} f(z)$ is not.
\end{definice}

\begin{priklad}
	Show that if $f$ has a pole of order $n$ at $z_0 \in ®C$. Then it admits a Laurient expansion $f(z) = \sum_{k = -n}^∞ a_k (z - z_0)^k$ with $a_{-n} ≠ 0$.
\end{priklad}

\begin{definice}[Residue]
	Let $f$ have a pole of order $n$ at the point $z_0 \in ®C$. Then the residue of $f$ at $z_0$ is the $k=-1$ coefficient of the Laurent expansion of $f$ at $z_0$.
\end{definice}

\begin{priklad}
	Show that if $f$ has a pole of order $1$ at $z_0$, then the residue of $f$ at $z_0$ can be computed as the following limit:
	$$ \res_{z=z_0} f(z) = \lim_{z \rightarrow z_0} (z - z_0)f(z). $$
\end{priklad}

\begin{priklad}[Residue theorem]
	Let $γ: [a, b] \rightarrow U \subset ®C$ be a simple closed path, bounding a domain $W$ containing the points $z_1, …, z_m$. Assume that $f$ is holomorphic on $U \setminus \{z_1, …, z_m\}$ and has poles at $\{z, …, z_m\}$. Show that
	$$ \oint_γ f(z) dz = 2πi \sum_{j=1}^m \res_{z = z_j} f(z). $$
\end{priklad}

\end{document}
