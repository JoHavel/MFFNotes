\documentclass[12pt]{article}					% Začátek dokumentu
\usepackage{../../MFFStyle}					    % Import stylu



\begin{document}

% 05. 10. 2022
\section{Úvod aneb Projektivní přímka a rovina}
\begin{poznamka}[O čem to bude]
	Nevlastní body, homogenní souřadnice. Projektivní geometrie = „geometrie polohy“, tj. neměří se vzdálenosti ani úhly. Máme pojmy (v rovině) bod, přímka, incidence ($X \in p$).

	Inspirováno perspektivou v malířství (realismus, 17. století).

	Klíčové pojmy: nevlastní body („body v nekonečnu“), princip duality.
\end{poznamka}

\begin{poznamka}[Možné přístupy ke geometrii]
	Axiomatický (jen axiomy, bez obrázků) (dnes), syntetický (důraz kladen na obrázky, bez souřadnic) (tento semestr), analytický (souřadnice, bez obrázků) (příští semestr).
\end{poznamka}

\subsection{Axiomatika projektivní geometrie (v rovině)}
\begin{poznamka}[Primitivní pojmy]
	Bod, přímka, incidence.
\end{poznamka}

\begin{definice}[Axiom A1]
	Ke každým dvěma (různým) bodům $\exists!$ přímka s oběma body incidentní. (Přímce říkáme \emph{spojnice} daných bodů.)
\end{definice}

\begin{definice}[Axiom A2]
	Ke každým dvěma (různým) přímkám $\exists!$ bod s oběma přímkami incidentní. (Bodu říkáme \emph{průsečík} daných přímek.)

	\begin{poznamkain}
		A2 vzniklo z A1 záměnnou pojmů bod a přímka. V EG neplatí, ale v PG chceme mít Princip duality.
	\end{poznamkain}
\end{definice}

\begin{definice}[Princip duality]
	Veškerá tvrzení zůstávají v platnosti, pokud v nich zaměníme pojmy bod a přímka, incidence (prochází bodem a leží na přímce, průsečík a spojnice), a pojmy z nich odvozené.
\end{definice}

\begin{definice}[Nevlastní bod, vlastní bod]
	Máme-li dvě rovnoběžky v EG, pak za jejich průsečík v PG označíme společný směr (bez orientace), neboli nevlastní bod (značíme $X_∞$, atd.).

	Původní body v rovině budeme nazývat vlastní.
\end{definice}

\begin{definice}[Nevlastní přímka, vlastní přímka]
	Nevlastní přímka ($n_∞$) = množina všech nevlastních bodů.
\end{definice}

\begin{poznamka}
	S nevlastními body a přímkou splňuje rovina A1 i A2.
\end{poznamka}

\begin{definice}[Axiom A3]
	Existují alespoň 4 body, z nichž každé 3 jsou nekolineární.
\end{definice}

\begin{poznamka}[„A4“]
	Duální tvrzení k A3 už je dokazatelné z A1 až A3.
\end{poznamka}

\begin{definice}[Projektivní rovina]
	Rovina s nevlastními body a nevlastní přímkou splňuje i A3. Takové rovině ($®R^2 \cup n_∞$) budeme říkat projektivní rovina a značit ji $®RP^2$ nebo $P^2$.
\end{definice}

\begin{poznamka}[Idea: existující různé geometrie]
	Euklidovská geometrie (EG) (body, přímky, incidence, vzdálenosti, úhly), Afinní geometrie (AG) (body, přímky, incidence, rozlišení rovnoběžek a různoběžek, případně vlastních a nevlastních bodů), Projektivní geometrie (PG) (body, přímky, incidence).

	(Hyperbolická geometrie = Lobačevského geometrie (body, přímky, incidence, jiné vzdálenosti, jiné úhly))
\end{poznamka}

\subsection{Afinní geometrie}
\begin{poznamka}
	Body $A, B, …$ a vektory $u, v, …$.

	$\rightarrow$ přímky, vzájemné polohy přímek (ale ne kolmost).
\end{poznamka}

\begin{poznamka}[Lze zavést střed úsečky:]
	$\vec{AS} = \frac{1}{2} \vec{AB} \Leftrightarrow \vec{SA} = -\vec{SB}$.
\end{poznamka}

\begin{definice}[Dělící poměr]
	Dělící poměr 3 bodů $A, B, C$ na (jedné) přímce je číslo $λ = (ABC)$ splňující $C - A = λ(C - B)$.

	\begin{poznamkain}
		Odsud lze odvodit Euklidovskou definici dělícího poměru: $|λ| = \frac{\|C - A\|}{\|C - B\|}$.

		$A, B, C$ různé, pak $λ$ nenabývá hodnot $0$ ($A = C$), $1$ ($A = B$) a $∞$ ($B = C$).

		$C$ je středem úsečky $AB$, právě když $(ABC) = -1$.

		Dělící poměr jako graf funkce ($A, B$ pevné, $C$ proměnné) je hyperbola.

		Pro každé dva body $A ≠ B$ a $\forall λ \in ®R \setminus \{0, 1\}$, existuje právě jedno $C$, že $(ABC) = λ$.

		Konstrukce: dány úsečky délek $1$ a $λ$, a body $A, B$.

		Pokud $λ = (ABC)$, tak $(BAC) = \frac{1}{λ}$, $(ACB) = 1 - λ$, $(BCA) = \frac{λ - 1}{λ}$, $(CAB) = \frac{1}{1 - λ}$, $(CBA) = \frac{λ}{λ - 1}$. Tyto permutace se některé rovnají pro $λ$ z trojice $(0, 1, ∞)$ (každé tam bude dvakrát), z trojice $(-1, 2, 1 / 2)$ (také každé dvakrát) a z dvojice $(1 / 2 + i\sqrt{3}/2, 1 / 2 - i\sqrt{3}/2)$ (každé třikrát).
	\end{poznamkain}
\end{definice}

\begin{poznamka}[Role zobrazení v jednotlivých geometriích]
	V EG: posunutí, otáčení a osová souměrnost (tj. shodnosti) zachovávají délky a úhly (tj. (pro zajímavost) jsou to invarianty euklidovské grupy).

	V AG: isomorfismy (lineární zobrazení na) zachovávají dělící poměr.
\end{poznamka}

\subsection{Projektivní přímka}
\begin{definice}[Označení]
	Je-li $v = (x_0, x_1) \in ®R^2 \setminus \{(0, 0)\}$, označíme $\<v\> =$ lineární obal $v =$ přímka generovaná $v$ (procházející počátkem). Tedy $\<(x_0, x_1)\> = \<v\> = \<av\> = \<ax_0, ax_1\>$ pro $\forall a ≠ 0, a \in ®R$.
\end{definice}

\begin{definice}[Projektivní přímka $®RP^1$, geometrický bod, aritmetický zástupce, homogenní souřadnice]
	Projektivní přímka je množina $®RP^1 = \{\<v\> | v \in ®R^2 \setminus \{(0, 0)\}\} =$ množina všech přímek v $®R^2$ (procházejících počátkem). Prvek $\<v\> \in ®RP^2$ nazýváme geometrický bod, vektor $v \in ®R^2 \setminus \{(0, 0)\}$ nazýváme jeho aritmetickým zástupcem.

	\begin{poznamkain}
		Tedy každý geometrický bod má nekonečně mnoho aritmetických zástupců (a ti se všichni liší jen nenulovým násobkem).
	\end{poznamkain}

	Je-li $v = (x_0, x_1)$, píšeme $\<v\> = [x_0:x_1]$. Tomuto se říká homogenní souřadnice geometrického bodu.

	\begin{poznamkain}
		Jsou určeny až na nenulový násobek.
	\end{poznamkain}
\end{definice}

\begin{definice}[Kanonické vnoření afinní přímky ®R do projektivní přímky $®RP^1$]
	Kanonické vnoření afinní přímky ®R do projektivní přímky $®RP^1$ je zobrazení $®R \rightarrow ®RP^1$, bod $x \mapsto [1:x]$ (body vlastní) a vektor $1 \mapsto [0:1]$ (bod nevlastní).

	První souřadnice je tzv. rozlišovací souřadnice (1 znamená vlastní, 0 nevlastní).
\end{definice}

% 19. 10. 2023

TODO!!!

\begin{priklad}[Konstrukce]
	Zkonstruovat 4. harmonický bod. (Mějme body $M, N, A$ a chceme najít $C$ tak, aby $(MNAC) = -1$)

	\begin{reseni}
		Zvolíme nekolineární bod $H$, do kterého vedeme přímky $m, n, a$. Vedeme libovolnou přímku z $A$, průsečíky s $m$ a $n$ označme $X$ a $Y$. Do nich vedeme přímky z $N$ a $M$ a průsečík těchto přímek spojíme s $H$ a máme přímku $c$.
	\end{reseni}
\end{priklad}

TODO!!! (Konstrukce 4. harmonické přímky).

% 26. 10. 2023

\begin{definice}[Projektivní škála]
	Máme body $0$, $1$ a $∞$ na jedné přímce. Následně provedeme několik kroků: 1. najdeme bod -1 tak, aby $(0\, ∞\, 1\, -1) = -1$, najdeme bod -2 tak, aby $(-1\, ∞\, 0\, -2) = -1$
\end{definice}

TODO!!!

\begin{poznamka}
	Této konstrukce se dá použít k nakreslení pražců na sbíhající se koleje (průsečík = $∞$, první pražec 0, druhý 1).
\end{poznamka}

\section{Projektivita a perspektivita lineárních soustav}
\begin{definice}[Soustava]
	Bodová = označené body na přímce. Píšeme $p(A, B, C, …)$.

	Přímková = označené přímky ve svazku. Píšeme $P(a, b, c, …)$.

	Dvě soustavy jsou sourodé, pokud jsou stejného typu a nesourodé, pokud jsou různých typů. Pokud jsou sourodé, pak mohou být soumístné, tedy na stejné přímce / ve stejném svazku, nebo nesoumístné (různé přímky / různé svazky).
\end{definice}

\begin{definice}[Perspektiva]
	Perspektiva nesoumístných sourodých soustav je zobrazení: pro bodové soustavy jde o středové promítání z bodu $O \notin p, p'$ (píšeme $p(A, B< C)::p'(A', B', C')$), pro přímkové soustavy duálně (přímka $o$ protne soustavu procházející $P$ v bodech, které spojíme s bodem $P'$ a dostaneme druhou soustavu).

	Bod $O$ se nazývá bod perspektivity (střed promítání). Přímka $o$ je přímkou perspektivity.

	\begin{poznamkain}
		Bod $O$ nemusí být vlastni.
	\end{poznamkain}

	\begin{poznamkain}[Značení $::$]
		Perspektivita je určená dvěma páry bodů/přímek (potřebujeme najít bod $O$ nebo přímku $o$), proto „dvakrát dvě tečky“.
	\end{poznamkain}
\end{definice}

\begin{dusledek}
	V každé perspektivitě existuje samodružný element: průsečík $p \cap p'$ respektive spojnice $P P'$.
\end{dusledek}

\begin{upozorneni}
	Složení perspektivit obecně není perspektivita! (Nemusí být zachován samodružný element.)
\end{upozorneni}

\begin{definice}[Projektivita]
	Projektivita je složení konečného počtu perspektivit.

	\begin{poznamkain}
		Dá se dokázat, že každá projektivita je složením $≤ 2$ perspektivit.
	\end{poznamkain}
\end{definice}

\begin{dusledek}
	Projektivita obecně nemá samodružný element, ale pokud už ho obsahuje, již je perspektivitou.
\end{dusledek}

\begin{upozorneni}
	Perspektivita nezachovává dělící poměr 3 bodů.
\end{upozorneni}

\begin{tvrzeni}
	Perspektivita zachovává dvojpoměry 4 bodů.
\end{tvrzeni}

\begin{dusledek}
	Projektivita zachovává dvojpoměry 4 bodů.
\end{dusledek}

\begin{tvrzeni}[Lze dokázat i opak]
	Pokud zobrazení zachovává kolinearitu a dvojpoměr, je to nutně projektivita.
\end{tvrzeni}

\begin{poznamka}[Druhý způsob (analytický) zavedení projektivity a perspektivity]
	Nejprve se zavede projektivní souřadný systém (PSS) na projektivní přímce. Je to trojice bodů $0, 1, ∞$. Souřadnice bodu $X$ vůči tomuto PSS je homogenní dvojice $[1:x]$, kde $x = (X 1 0 ∞)$. Pak projektivní zobrazení je $®RP^1 \rightarrow ®RP^1$, $[x_0:x_1] \mapsto [x_0':x_1']$, kde $(x_0', x_1') = A·(x_0, x_1)^T$, kde $A$ je regulární matice $2 \times 2$ určená až na násobek $≠0$.

	\begin{dusledekin}
		Projektivita zachovává dvojpoměr.
	\end{dusledekin}

	Pak perspektivita = projektivita mající samodružný bod.
\end{poznamka}

\begin{poznamka}[Značení projektivity]
	Projektivitu značíme $p(A, B, C) ::: p'(A', B', C')$.

	\begin{poznamkain}
		Projektivita je určena třemi páry bodů.
	\end{poznamkain}
\end{poznamka}

\begin{definice}[Perspektivita nesourodých soustav]
	Dvě nesourodé soustavy jsou v perspektivitě, je-li jedna soustava průmětem/průsekem té druhé.
\end{definice}

\begin{veta}
	Dvě sourodé nesoumístné soustavy jsou v perspektivitě $\Leftrightarrow$ obě jsou v perspektivitě s touž nesourodou soustavou.

	\begin{dukazin}
		Obrázkem. (Dává nám to přesně ty body a přímky, které potřebujeme.)
	\end{dukazin}
\end{veta}

\begin{poznamka}[Doplňování soustav]
	Doplňování perspektivit ($p(A, B, C)::p'(A', B', C')$ dáno, k bodu $X$ na $p$ doplňte $X'$) je jednoduché.

	Doplňování projektivit ($p(A, B, C):::p'(A', B', C')$ dáno, m bodu $X$ na $p$ doplňte $X'$) je těžší, budeme potřebovat následující větu.
\end{poznamka}

\begin{veta}[O direkční přímce]
	Nechť $p(A, B, C) ::: p'(A', B', C')$ je projektivita nesoumístných bodových soustav. Pak průsečíky spojnic $AB'$ a $A'B$, $AC'$ a $CA'$, $BC'$ a $CB'$ leží na jedné přímce $d$.

	\begin{dukazin}
		Zvolme si význačné body $A$, $A'$ a uvažujme přímky: $a = AA'$, $b = AB'$ a $c = AC'$, stejně tak $a' = A'A$, $b' = A'B$, $c' = A'C$. Hned je jasné, že $a = a'$.

		Pak máme $A(a, b, c)::p'(A', B', C'):::p(A, B, C)::A'(a', b', c')$. Tedy $A(a, b, c):::A'(a', b', c')$. Ale ta má samodružnou přímku $a = a'$, tedy je to perspektivita 2 přímkových soustav, tedy podle předchozí věty jsou obě v perspektivitě s touž nesourodou soustavou (body na přímce). Označme ji $d$. A víme, že se s ní protínají odpovídající si páry přímek, tj. páry $B = AB'$, $b' = A'B$, $c = AC'$, $c' = A'C$.

		Potřebujeme ukázat, že přímka $d$ nezávisí na volbě páru $AA'$. A tím také to, že také přímky $BC'$, $B'C$ se protínají na $d$. Označme $M = N'$ průsečík $p \cap p'$. Kde je $M'$ a $N$? Platí $M' = p' \cap d$ a $N = p \cap d$. $N \in p$ zřejmé (leží v soustavě $p(A, B, C)$). $N \in d$? máme přímky $n = AN = p$, $n' = A'N$, víme, že průsečík $n \cap n' = N \in d$. Stejně tak $M' = p' \cap d$.

		Důsledek: $d = M'N$, ale $M$, $N'$ nezáleží na volbě páru $AA'$, tedy máme hotovo.
	\end{dukazin}
\end{veta}

\begin{definice}[Direkční přímka projektivity]
	Přímku z předchozí věty nazveme direkční přímka projektivity.
\end{definice}

\begin{poznamka}
	Projektivita je perspektivita $\Leftrightarrow$ $p \cap p' \in d$.
\end{poznamka}

\begin{priklad}
	Doplňování projektivity (nesoumístných soustav) je teď jednoduché. Doplňování projektivity soumístných soustav uděláme přes další soustavu.
\end{priklad}

\begin{priklad}
	Spojení bodu $V$ s nepřístupným průsečíkem přímek $p, p'$.

	\begin{reseni}
		Na $p$ a $p'$ doplníme body $A$, $A'$, $B$, $B'$ tak, aby $V \in AB', BA'$. $AA'$ a $BB'$ se protínají v bodě, ze kterého vedeme přímku, na která protne $p$ a $p'$ v bodech $C$ a $C'$. Nyní najdeme direkční přímku.
	\end{reseni}
\end{priklad}

\begin{veta}[Papova o šestiúhelníku]
	Stejné jako věta o direkční přímce. (Jinak formulovaná.)
\end{veta}

% 09. 11. 2023

TODO!!! (charakteristika projektivity, involuce, …)

\begin{definice}[Involuce]
	Involuce je projektivita (soumístných soustav) splňující jednu z následujících ekvivalentních podmínek:
	\begin{itemize}
		\item $w = (XX'ST) = -1$; (tzv. charakteristika projektivity, $X$ a $X'$ je libovolný pár, $S$ a $T$ jsou různé samodružné elementy)
		\item $\exists X ≠ S, T: X'' = X$;
		\item $\forall X: X'' = X$.
	\end{itemize}
\end{definice}

\begin{definice}[Hyperbolická a eliptická involuce]
	$S, T$ reálné různé $\implies$ involuce je hyperbolická (nesouhlasné soustavy, neboli směr je proti).

	$S, T$ komplexně sdružené $\implies$ eliptická (souhlasné soustavy, neboli směr se zachovává).

	\begin{poznamka}
		$S = T$ je parabolická involuce (ale není to projektivita, neboť to není prosté zobrazení).
	\end{poznamka}
\end{definice}

% 16. 11. 2023

\begin{poznamka}[Pár involuce]
	$X \rightarrow X'$ a $X' \rightarrow X$, pak $X, X'$ je pár involuce.
\end{poznamka}

\begin{poznamka}
	Projektivita je určena 3 páry, involuce je určena 2 páry.
\end{poznamka}

\begin{dusledek}
	$A A' B' B$ a $B A A' B'$ (ve skutečnosti vzhledem k zahrnutí $∞$ jsou to stejné případy) jsou hyperbolické (říkáme páry se nerozdělují), $A B A' B'$ jsou hyperbolické (páry se rozdělují).
\end{dusledek}

\begin{priklad}[Konstrukce]
	Určit druhý samodružný bod ($T$) involuce určené jedním samodružným bodem ($S$) a jedním párem ($A$, $A'$).

	\begin{reseni}
		Využijeme $(AA'ST) = -1$ a najdeme čtvrtý harmonický bod, jak jsme to již dělali.
	\end{reseni}
\end{priklad}

\begin{veta}[O bodu na direkční přímce]
	Mějme projektivitu $p(A, B, C) ::: p'(A', B', C')$ (nesourodých soustav), $d$ nechť je direkční přímka, $H \in d$ libovolný bod na direkční přímce. Pak páry přímek $a = HA$, $A' = HA'$, $b = HB$, $b' = HB'$, atd. jsou páry téže involuce přímek ve svazku se středem $H$. Neboli $H(a, b, c) ::: H(a', b', c')$ je involuce.

	\begin{dukazin}
		„1. Projektivita“:
		$$ H(a, b, c) :: p(A, B, C) ::: p'(A', B', C') :: H(a', b', c'). $$
		(Složení tří projektivit je projektivita, tedy existuje $H(a, b, c) ::: H(a', b', c')$)

		„2. Involuce“: Tato projektivita je involuce, protože pokud označíme $X = a' \cap p \implies x = a'$, pak z věty o direkční přímce platí $X' = a \cap p' \implies a = x'$, tedy $(a, a')$ je pár involuce a to už stačí.

		(Pokud $H \in AA'$, zvolíme místo $A$ jiný bod.)
	\end{dukazin}
\end{veta}

\begin{veta}[O přímce procházející direkčním bodem]
	Mějme projektivitu $P(a, b, c) ::: P'(a', b', c')$, $D$ nechť je direkční bod a $D \in h$. Pak páry bodů $A = h \cap a$, $A' = h \cap a'$, $B = h \cap b$, $B' = h \cap b'$ jsou páry téže involuce bodů na přímce $h$. ($h(A, B, C) ::: h(A', B', C')$ je involuce.)

	\begin{dukazin}
		Dualita.
	\end{dukazin}
\end{veta}

\begin{priklad}
	Doplňování bodové involuce dané dvěma páry.

	\begin{reseni}
		Použijeme předchozí větu a doplníme projektivitu. (Další způsoby jsou klasicky jako doplňování projektivity, nebo použití původní, neduální, verze předchozí věty.)
	\end{reseni}
\end{priklad}

\subsection{Úplný čtyřroh a úplný čtyřstran}
\begin{definice}[Úplný čtyřroh]
	Čtveřice bodů v rovině ($M$, $N$, $P$, $Q$), přičemž žádné tři nejsou kolineární, se nazývá čtyřroh.

	Tyto body ($M$, $N$, $P$, $Q$) jsou vrcholy čtyřrohu. Jejich 6 spojnic jsou strany čtyřrohu.

	Máme 3 páry protějších stran ($MN$ a $PQ$, $MP$ a $NQ$, $MQ$ a $NP$), 3 diagonální vrcholy = průsečíků protějších stran ($X$, $Y$, $Z$) a 3 diagonální strany ($XY$, $XZ$, $YZ$).

	Všemu tomuto dohromady se říká úplný čtyřroh.
\end{definice}

\begin{definice}[Úplný čtyřstran]
	Čtveřice přímek v rovině ($m$, $n$, $p$, $q$), přičemž žádné tři neprochází jedním bodem, se nazývá čtyřstran.

	Tyto přímky ($m$, $n$, $p$, $q$) jsou strany čtyřstranu. Jejich 6 průsečíků jsou vrcholy čtyřstrnu.

	Máme 3 páry protějších vrcholů ($m \cap n$ a $p \cap q$, $m \cap p$ a $n \cap q$, $m \cap q$ a $n \cap p$), 3 diagonální strany = spojnice protějších vrcholů ($x$, $y$, $z$) a 3 diagonální vrcholy ($x \cap y$, $x \cap z$, $y \cap z$).

	Všemu tomuto dohromady se říká úplný čtyřstran.
\end{definice}

\begin{veta}
	Každá (i diagonální) strana úplného čtyřrohu je proťata ostatními stranami jen ve 4 bodech, které tvoří harmonickou čtveřici.

	\begin{dukazin}
		„První část“ se spočítá z obrázku. „Druhá část“ je vidět z konstrukce čtvrtého bodu harmonické čtveřice.
	\end{dukazin}
\end{veta}

\begin{veta}[Duální k přechozí]
	Každý (i diagonální) vrchol úplného čtyřstranu je spojen s ostatními vrcholy pouze 4 přímkami, ty tvoří harmonickou čtveřici.
\end{veta}

\begin{veta}[O přímce a čtyřrohu]
	Je dán úplný čtyřroh a libovolná přímka $h$ různá od jeho 9 stran. Pak protější strany 4 rohu vytínají na $h$ páry téže involuce.

	\begin{dukazin}
		Nechť $P$ a $Q$ jsou středy svazků $a = PM$, $b = PN$ a $a' = QN$, $b' = QM$ ($a$ a $a'$ protější, $b$ a $b'$ taktéž). Páry $AA'$, $BB'$ zadávají involuci na $h$. Je pár $CC'$ také párem této involuce?

		Zároveň máme projektivitu $P(a, b, …) ::: Q(a', b', …)$. Dle věty o přímce procházející direkčním bodem je průsečík $D = C' = MN \cap h$ direkčním bodem této projektivity a proto $D' = C = PQ \cap h$. Tedy $DD' = C'C$ je pár téže involuce.
	\end{dukazin}
\end{veta}

\begin{veta}[O bodu a čtyřstranu]
	Je dán úplný čtyřstran a bod $H$ různý od jeho vrcholů. Pak spojnice protějších vrcholů čtyřstranu s $H$ tvoří páry téže involuce.
\end{veta}

\begin{poznamka}
	$A$, $A'$, $B$, $B'$, $C$, $C'$ z předchozí věty (původní verze) se nazývá čtyřstranná množina.
\end{poznamka}

\section{Kuželosečky}
\begin{definice}[Bodová kuželosečka]
	Mějme projektivitu nesoumístných přímkových soustav $H(a, b, c) ::: H'(a', b', c')$. Bodová kuželosečka $©B = $ množina průsečíků odpovídajících si přímek (tj. $a \cap a'$, $b \cap b'$, atd.).
\end{definice}

\begin{veta}
	Zadaná projektivita je perspektivitou $\Leftrightarrow$ kuželosečka $B$ se skládá ze dvou přímek, a sice přímky $HH'$ a z přímky perspektivity.

	\begin{dukazin}
		Z obrázku a rozpravy nad ním.
	\end{dukazin}
\end{veta}

\begin{definice}[Singulární a regulární]
	Když $H::H'$ ©B je singulární. V opačném případě je regulární.
\end{definice}

\begin{tvrzeni}[Platí]
	$H, H' \in ©B$. (Pro singulární kuželosečku celá $H H' \in ©B$. Pro regulární křivku $H = n \cap n'$ a $H' = m \cap m'$, tedy $H, H' \in ©B$.)
\end{tvrzeni}

\begin{poznamka}
	Dále budeme uvažovat jen regulární křivky.
\end{poznamka}

\begin{definice}[Vzájemná poloha přímky a kuželosečky]
	Přímka v rovině je (ve vztahu ke kuželosečce)
	\begin{itemize}
		\item vnější přímka, pokud nemají žádný společný bod;
		\item tečna, má-li jeden průsečík;
		\item sečna, má-li dva průsečíky.
	\end{itemize}
\end{definice}

\begin{veta}
	Bodem $H$ (resp. $H'$) prochází jediná tečna, a sice $n$ (resp. $m'$), kde $m = n' = HH'$. Průsečík $D = n \cap m'$ je direkčním bodem zadané projektivity.

	\begin{dukazin}
		Přímka $x \in H(a, b, c)$ protíná kuželosečku ©B ve 2 bodech, pouze pro $x = n$ tyto 2 body splývají do 1 bodu ($H$). Podobně pro $H'$, $m'$. $D = n \cap m'$ už víme.
	\end{dukazin}
\end{veta}

\begin{veta}
	Je-li dána ©B pomocí projektivity $H(a, b, c) ::: H'(a', b', c')$ a zvolíme-li 5 bodů na ©B: $K$, $K'$, $A$, $B$, $C$ a označíme-li $α = KA$, $β = KB$, …, pak projektivita $K(α, β, γ) ::: K(α', β', γ')$ zadává tutéž kuželosečku.

	\begin{dukazin}
		Vynechán.
	\end{dukazin}
\end{veta}

\begin{dusledek}
	V definici kuželosečky můžeme vzít za středy svazků libovolné dva body na kuželosečce.
\end{dusledek}

\begin{dusledek}
	Každým bodem (regulární) kuželosečky prochází jediná tečna.
\end{dusledek}

\begin{dusledek}
	Kuželosečka je zadána 5 body (nebo šesti přímkami, z nichž 3 a 3 prochází stejným a stejným bodem).
\end{dusledek}

% 23. 11. 2023

\begin{priklad}[Konstrukce (!!!)]
	Sestrojit kuželosečku z 5 bodů. (Tj. dány body $H, H', A, B, C$, najít alespoň 1 další bod kuželosečky procházející těmito body. Pak umíme najít libovolný konečný počet bodů.)

	\begin{reseni}
		Nalezneme direkční bod a následně provedeme konstrukci druhé přímky v perspektivitě k nějaké zvolené přímce (ta nám určuje, který bod dostaneme).
	\end{reseni}
\end{priklad}

\begin{priklad}[DÚ]
	K zadaným 5 bodům najít 10–15 dalších bodů kuželosečky. (Ručně nebo v geogebře.)
\end{priklad}

\begin{priklad}[Konstrukce]
	Kuželosečka zadána 5 body, v jednom z nich najít tečnu.

	\begin{reseni}
		Tento bod vezmeme jako bod soustavy, k němu zvolíme druhý bod a najdeme direkční bod perspektivity přímkových soustav procházejících zbylými třemi body. Ten spojíme s naším bodem a máme tečnu.
	\end{reseni}
\end{priklad}

\begin{poznamka}
	Tečna s bodem dotyku = 2 podmínky pro kuželosečku.

	Kuželosečka je tedy zadána 5 podmínkami:
	\begin{itemize}
		\item 5 bodů;
		\item 4 body + tečna v jednom z nich;
		\item 3 body + tečny ve dvou z nich.
	\end{itemize}
\end{poznamka}

\begin{priklad}[Konstrukce]
	Sestrojit kuželosečku z jedné tečny a 4 bodů.

	\begin{reseni}
		Zvolíme ze 3 zbývajících bodů bod druhé přímkové soustavy. Poté průsečík tečny a spojnice (správných) průsečíků vzniklých 4 přímek je direkční bod.
	\end{reseni}
\end{priklad}

\begin{priklad}[Konstrukce]
	Sestrojit kuželosečku ze dvou tečen a 3 bodů.

	\begin{reseni}
		Zde máme direkční bod rovnou.
	\end{reseni}
\end{priklad}

\subsection{Soustavy na bodové kuželosečce}
\begin{definice}
	$©B(A, B, C)$ je bodová soustava na ©B. Zase mohou bít soumístné/nesoumístné. Perspektivita soustavy na kuželosečce a soustavy na přímce je „promítnutí“ bodů soustavy $©B(A, B, C)$ z libovolného bodu $\in ©B$ na danou přímku. Složení perspektivit je zase projektivita. Dvojpoměr 4 bodů na ©B je definován přenesením na bodovou soustavu na přímce (zachovává se v každé projektivitě).

	\begin{poznamka}
		Projektivita je dána 3 páry bodů.

		Projektivita soumístných soustav na ©B má 2/1/0 samodružných bodů.
	\end{poznamka}
\end{definice}

\begin{veta}
	Je-li dána projektivita $©B(A, B, C) ::: ©B(A', B', C')$, pak průsečíky $AB' \cap A'B$, $AC' \cap A'C$, $BC', B'C$ leží na jedné, tzv. direkční přímce $d$. Navíc $d \cap ©B$ jsou samodružné body dané projektivity.

	\begin{dukazin}
		Označíme si $b = A'B$, $b' = AB'$, $c = A'C$, $c = AC'$ a $a = A'A = a'$.
		$$ A(a', b', c') ::: ©B(A', B', C') ::: ©B(A, B, C) ::: A'(a, b, c). $$
		Tato projektivita ($A(a', b', c') ::: A'(a, b, c)$) je perspektivita (neboť $a = a'$ je samodružná). Tedy existuje přímka perspektivity $d$, na níž se kříží $b', b$ a $c, c'$.

		Dále chceme ukázat i $BC' \cap B'C \in d$ a $d \cap ©B = $ samodružné body projektivity. Nejprve ukážeme druhou část (a z ní už plyne první, protože samodružné body nezávisí na volbě $A$): $S = S'$, $T = T'$ samodružné body ($\in ©B$ z definice), potom $S = S' \in d$, neboť $s = A'S$ a $s' = AS'$ se protínají na $d$, ale jediný jejich průsečík je $S = S'$ ($T = T'$ obdobně).
	\end{dukazin}

	\begin{dukazin}
		Důkaz pro 0 samodružných bodů (případně pro 1) by se prováděl algebraicky.
	\end{dukazin}
\end{veta}

\begin{poznamka}
	Direkční přímka je sečnou / tečnou / vnější přímkou k ©B $\Leftrightarrow$ daná projektivita má 2/1/0 reálné samodružné body.
\end{poznamka}

\begin{priklad}[Konstrukce]
	Doplňování projektivity na bodové kuželosečce.

	\begin{reseni}
		Nemůžeme to udělat tak, jak bychom chtěli, protože nemáme „nakreslenou“ kuželosečku (neumíme s ní dělat průsečík).

		Co ale můžeme, můžeme obvyklým způsobem najít dvě přímky procházející doplňovaným bodem a najít jejich průsečík.
	\end{reseni}
\end{priklad}

\begin{definice}[Involuce (totéž, co výše)]
	Involuce je projektivita (soumístných soustav) splňující jednu z následujících ekvivalentních podmínek:
	\begin{itemize}
		\item $w = (XX'ST) = -1$; (tzv. charakteristika projektivity, $X$ a $X'$ je libovolný pár, $S$ a $T$ jsou různé samodružné elementy)
		\item $\exists X ≠ S, T: X'' = X$;
		\item $\forall X: X'' = X$.
	\end{itemize}
\end{definice}

\begin{poznamka}
	Involuce je dána 2 páry bodů.

	Rozlišujeme involuci hyperbolickou a eliptickou podle toho, zda má 2 nebo 0 samodružných bodů.
\end{poznamka}

\begin{veta}[O involuci na bodové kuželosečce]
	Nechť je dána involuce na ©B dvěma páry bodů $A, A'$; $B, B'$. Pak platí:
	\begin{enumerate}
		\item Na direkční přímce $d$ leží nejen průsečíky $AB'$, $A'B$, … ale i průsečíky $AB$, $A'B'$, …
		\item Spojnice $AA'$, $BB'$, … se protínají v jediném bodě $P$.
		\item Průsečíky $α \cap ©B$ jsou samodružné body involuce ($S$, $T$), přímky $PS$ a $PT$ jsou tečny z bodu $P$ k ©B.
		\item Tečny v bodech $A$, $A'$ se také protínají na $d$.
	\end{enumerate}

	\begin{definicein}
		$d$ se pak nazývá osa involuce, $P$ se nazývá střed involuce.
	\end{definicein}

	\begin{dukazin}[1.]
		Průsečík $AB'$ a $A'B$ $\in d$ z definice $d$. Průsečík $AB$ a $A'B'$ $\in d$ záměnou $A \leftrightarrow A'$.
	\end{dukazin}

	\begin{dukazin}[2.]
		Body $A, A', B, B'$ zadávají úplný čtyřroh. $d$ je jednou z jeho diagonálních stran (plyne z 1. bodu) a to každého takového. Tedy bod $R = AA' \cap d$ zůstává pevný pro každý čtyřroh obsahující body $A$ a $A'$. Dále bod $P$ je protější diagonální vrchol k diagonálni straně $d$. Navíc víme, že $A, R, A', P$ je harmonická čtveřice takže (protože $A$, $R$ a $A'$ jsou pevné) $P$ je pevný pro každou volbu $B, B'$.
	\end{dukazin}

	\begin{dukazin}[3.]
		Víme už, že $S, T = d \cap ©B$ a díky $S = S'$ a $T = T'$ se jedná o tečny.
	\end{dukazin}

	\begin{dukazin}[4.]
		„Limitním přechodem“ z bodu 1.
	\end{dukazin}
\end{veta}

\begin{definice}
	Říkáme též, že involuce je indukována svým středem $P$.

	Taktéž definujeme vnější a vnitřní bod kuželosečky v následující tabulce:

	\begin{tabular}{c|ccc}
		Involuce      & Reálné samodružné body & Osa involuce  & Střed involuce \\ \hline
		hyperbolická  & 2                      & sečna         & vnější bod     \\
		eliptická     & 0                      & vnější přímka & vnitřní bod    \\
		„parabolická“ & 1                      & tečna         & $\in ©B$
	\end{tabular}

	(U parabolické se všechny body zobrazí do $P$.)
\end{definice}

\begin{definice}[Další názvy]
	$P$ = pól přímky $d$, $d$ = polára bodu $P$.
\end{definice}

\begin{poznamka}
	$p =$ vnitřní bod ©B $\implies$ každá přímka z bodu $P$ je sečna ©B.

	$p =$ vnější bod ©B $\implies$ existují právě dvě tečny a ty oddělují sečny od vnějších přímek.
\end{poznamka}

\begin{dusledek}
	Je-li $R = A'A \cap T_1T_2$ ($T_i$ tečné body z bodu $P$), pak $(AA'RP) = -1$.
\end{dusledek}

\subsection{Čtyři malé věty}
\begin{veta}[A]
	Mějme na ©B dány dvě involuce se středy $P ≠ Q$. Tyto dvě involuce mají jediný společný pár, jsou to právě průsečíky $PQ \cap ©B$. Navíc je-li alespoň jeden z bodů $P$, $Q$ vnitřní, je tento pár reálný.

	\begin{dukazin}
		Jednoduchý.
	\end{dukazin}
\end{veta}

\begin{veta}[B]
	Nechť $A, A' =$ pár involuce indukovaný na ©B středem $P$, $Q :=$ průsečík tečen k ©B v bodech $A$ a $A'$. $X, X' = PQ \cap ©B$. Pak $X, X'$ je jediný pár involuce se středem $P$, který splňuje $(XX'AA' = -1)$.

	\begin{dukazin}
		Pro involuci ze středem $Q$ platí $A, A'$ jsou samodružné body. $X, X'$ je pár této involuce, tedy $(XX'AA') = -1$. A dle Věty A je to jediný takový pár.
	\end{dukazin}
\end{veta}

\begin{veta}[C]
	$P =$ libovolný vnější bod ©B. $M, N =$ body dotyku tečen z $P$ k ©B. $A, C = $ 2 (libovolné) body na ©B kolineární s bodem $P$. $B =$ libovolný další bod na ©B. $a:=BA$, $c:=BC$, $m:=BM$, $n:=BN$. Pak $(mnac) = -1$.

	\begin{dukazin}
		Okamžitě z Věty B.
	\end{dukazin}
\end{veta}

\begin{veta}[D]
	$P, M, N, A, C$ jako ve větě $C$. $m:=CM$, $n:=CN$, $a:=CA$ a $c$ je tečna v bodě $C$. Pak $(mnac) = -1$. (Věta C s $C = B$.)

	\begin{dukazin}
		$C = B$ ve Větě C.
	\end{dukazin}
\end{veta}

\begin{priklad}[Konstrukce]
	Kuželosečka je dána 3 body a tečnami ve 2 z nich. Sestrojit tečnu ve 3 bodě.

	\begin{reseni}
		Věta D.
	\end{reseni}
\end{priklad}

\end{document}
