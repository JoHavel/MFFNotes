\documentclass[12pt]{article}					% Začátek dokumentu
\usepackage{../../MFFStyle}					    % Import stylu



\begin{document}

% 05. 10. 2022
\section{Úvod aneb Projektivní přímka a rovina}
\begin{poznamka}[O čem to bude]
	Nevlastní body, homogenní souřadnice. Projektivní geometrie = „geometrie polohy“, tj. neměří se vzdálenosti ani úhly. Máme pojmy (v rovině) bod, přímka, incidence ($X \in p$).

	Inspirováno perspektivou v malířství (realismus, 17. století).

	Klíčové pojmy: nevlastní body („body v nekonečnu“), princip duality.
\end{poznamka}

\begin{poznamka}[Možné přístupy ke geometrii]
	Axiomatický (jen axiomy, bez obrázků) (dnes), syntetický (důraz kladen na obrázky, bez souřadnic) (tento semestr), analytický (souřadnice, bez obrázků) (příští semestr).
\end{poznamka}

\subsection{Axiomatika projektivní geometrie (v rovině)}
\begin{poznamka}[Primitivní pojmy]
	Bod, přímka, incidence.
\end{poznamka}

\begin{definice}[Axiom A1]
	Ke každým dvěma (různým) bodům $\exists!$ přímka s oběma body incidentní. (Přímce říkáme \emph{spojnice} daných bodů.)
\end{definice}

\begin{definice}[Axiom A2]
	Ke každým dvěma (různým) přímkám $\exists!$ bod s oběma přímkami incidentní. (Bodu říkáme \emph{průsečík} daných přímek.)

	\begin{poznamkain}
		A2 vzniklo z A1 záměnnou pojmů bod a přímka. V EG neplatí, ale v PG chceme mít Princip duality.
	\end{poznamkain}
\end{definice}

\begin{definice}[Princip duality]
	Veškerá tvrzení zůstávají v platnosti, pokud v nich zaměníme pojmy bod a přímka, incidence (prochází bodem a leží na přímce, průsečík a spojnice), a pojmy z nich odvozené.
\end{definice}

\begin{definice}[Nevlastní bod, vlastní bod]
	Máme-li dvě rovnoběžky v EG, pak za jejich průsečík v PG označíme společný směr (bez orientace), neboli nevlastní bod (značíme $X_∞$, atd.).

	Původní body v rovině budeme nazývat vlastní.
\end{definice}

\begin{definice}[Nevlastní přímka, vlastní přímka]
	Nevlastní přímka ($n_∞$) = množina všech nevlastních bodů.
\end{definice}

\begin{poznamka}
	S nevlastními body a přímkou splňuje rovina A1 i A2.
\end{poznamka}

\begin{definice}[Axiom A3]
	Existují alespoň 4 body, z nichž každé 3 jsou nekolineární.
\end{definice}

\begin{poznamka}[„A4“]
	Duální tvrzení k A3 už je dokazatelné z A1 až A3.
\end{poznamka}

\begin{definice}[Projektivní rovina]
	Rovina s nevlastními body a nevlastní přímkou splňuje i A3. Takové rovině ($®R^2 \cup n_∞$) budeme říkat projektivní rovina a značit ji $®RP^2$ nebo $P^2$.
\end{definice}

\begin{poznamka}[Idea: existující různé geometrie]
	Euklidovská geometrie (EG) (body, přímky, incidence, vzdálenosti, úhly), Afinní geometrie (AG) (body, přímky, incidence, rozlišení rovnoběžek a různoběžek, případně vlastních a nevlastních bodů), Projektivní geometrie (PG) (body, přímky, incidence).

	(Hyperbolická geometrie = Lobačevského geometrie (body, přímky, incidence, jiné vzdálenosti, jiné úhly))
\end{poznamka}

\subsection{Afinní geometrie}
\begin{poznamka}
	Body $A, B, …$ a vektory $u, v, …$.

	$\rightarrow$ přímky, vzájemné polohy přímek (ale ne kolmost).
\end{poznamka}

\begin{poznamka}[Lze zavést střed úsečky:]
	$\vec{AS} = \frac{1}{2} \vec{AB} \Leftrightarrow \vec{SA} = -\vec{SB}$.
\end{poznamka}

\begin{definice}[Dělicí poměr]
	Dělicí poměr 3 bodů $A, B, C$ na (jedné) přímce je číslo $λ = (ABC)$ splňující $C - A = λ(C - B)$.

	\begin{poznamkain}
		Odsud lze odvodit Euklidovskou definici dělicího poměru: $|λ| = \frac{\|C - A\|}{\|C - B\|}$.

		$A, B, C$ různé, pak $λ$ nenabývá hodnot $0$ ($A = C$), $1$ ($A = B$) a $∞$ ($B = C$).

		$C$ je středem úsečky $AB$, právě když $(ABC) = -1$.

		Dělicí poměr jako graf funkce ($A, B$ pevné, $C$ proměnné) je hyperbola.

		Pro každé dva body $A ≠ B$ a $\forall λ \in ®R \setminus \{0, 1\}$, existuje právě jedno $C$, že $(ABC) = λ$.

		Konstrukce: dány úsečky délek $1$ a $λ$, a body $A, B$.

		Pokud $λ = (ABC)$, tak $(BAC) = \frac{1}{λ}$, $(ACB) = 1 - λ$, $(BCA) = \frac{λ - 1}{λ}$, $(CAB) = \frac{1}{1 - λ}$, $(CBA) = \frac{λ}{λ - 1}$. Tyto permutace se některé rovnají pro $λ$ z trojice $(0, 1, ∞)$ (každé tam bude dvakrát), z trojice $(-1, 2, 1 / 2)$ (také každé dvakrát) a z dvojice $(1 / 2 + i\sqrt{3}/2, 1 / 2 - i\sqrt{3}/2)$ (každé třikrát).
	\end{poznamkain}
\end{definice}

\begin{poznamka}[Role zobrazení v jednotlivých geometriích]
	V EG: posunutí, otáčení a osová souměrnost (tj. shodnosti) zachovávají délky a úhly (tj. (pro zajímavost) jsou to invarianty euklidovské grupy).

	V AG: isomorfismy (lineární zobrazení na) zachovávají dělicí poměr.
\end{poznamka}

\subsection{Projektivní přímka}
\begin{definice}[Označení]
	Je-li $v = (x_0, x_1) \in ®R^2 \setminus \{(0, 0)\}$, označíme $\<v\> =$ lineární obal $v =$ přímka generovaná $v$ (procházející počátkem). Tedy $\<(x_0, x_1)\> = \<v\> = \<av\> = \<ax_0, ax_1\>$ pro $\forall a ≠ 0, a \in ®R$.
\end{definice}

\begin{definice}[Projektivní přímka $®RP^1$, geometrický bod, aritmetický zástupce, homogenní souřadnice]
	Projektivní přímka je množina $®RP^1 = \{\<v\> | v \in ®R^2 \setminus \{(0, 0)\}\} =$ množina všech přímek v $®R^2$ (procházejících počátkem). Prvek $\<v\> \in ®RP^2$ nazýváme geometrický bod, vektor $v \in ®R^2 \setminus \{(0, 0)\}$ nazýváme jeho aritmetickým zástupcem.

	\begin{poznamkain}
		Tedy každý geometrický bod má nekonečně mnoho aritmetických zástupců (a ti se všichni liší jen nenulovým násobkem).
	\end{poznamkain}

	Je-li $v = (x_0, x_1)$, píšeme $\<v\> = [x_0:x_1]$. Tomuto se říká homogenní souřadnice geometrického bodu.

	\begin{poznamkain}
		Jsou určeny až na nenulový násobek.
	\end{poznamkain}
\end{definice}

\begin{definice}[Kanonické vnoření afinní přímky ®R do projektivní přímky $®RP^1$]
	Kanonické vnoření afinní přímky ®R do projektivní přímky $®RP^1$ je zobrazení $®R \rightarrow ®RP^1$, bod $x \mapsto [1:x]$ (body vlastní) a vektor $1 \mapsto [0:1]$ (bod nevlastní).

	První souřadnice je tzv. rozlišovací souřadnice (1 znamená vlastní, 0 nevlastní).
\end{definice}

% 19. 10. 2023

TODO!!!

\begin{priklad}[Konstrukce]
	Zkonstruovat 4. harmonický bod. (Mějme body $M, N, A$ a chceme najít $C$ tak, aby $(MNAC) = -1$)

	\begin{reseni}
		Zvolíme nekolineární bod $H$, do kterého vedeme přímky $m, n, a$. Vedeme libovolnou přímku z $A$, průsečíky s $m$ a $n$ označme $X$ a $Y$. Do nich vedeme přímky z $N$ a $M$ a průsečík těchto přímek spojíme s $H$ a máme přímku $c$.
	\end{reseni}
\end{priklad}

TODO!!! (Konstrukce 4. harmonické přímky).

% 26. 10. 2023

\begin{definice}[Projektivní škála]
	Máme body $0$, $1$ a $∞$ na jedné přímce. Následně provedeme několik kroků: 1. najdeme bod -1 tak, aby $(0\, ∞\, 1\, -1) = -1$, najdeme bod -2 tak, aby $(-1\, ∞\, 0\, -2) = -1$
\end{definice}

TODO!!!

\begin{poznamka}
	Této konstrukce se dá použít k nakreslení pražců na sbíhající se koleje (průsečík = $∞$, první pražec 0, druhý 1).
\end{poznamka}

\section{Projektivita a perspektivita lineárních soustav}
\begin{definice}[Soustava]
	Bodová = označené body na přímce. Píšeme $p(A, B, C, …)$.

	Přímková = označené přímky ve svazku. Píšeme $P(a, b, c, …)$.

	Dvě soustavy jsou sourodé, pokud jsou stejného typu a nesourodé, pokud jsou různých typů. Pokud jsou sourodé, pak mohou být soumístné, tedy na stejné přímce / ve stejném svazku, nebo nesoumístné (různé přímky / různé svazky).
\end{definice}

\begin{definice}[Perspektiva]
	Perspektiva nesoumístných sourodých soustav je zobrazení: pro bodové soustavy jde o středové promítání z bodu $O \notin p, p'$ (píšeme $p(A, B, C)::p'(A', B', C')$), pro přímkové soustavy duálně (přímka $o$ protne soustavu procházející $P$ v bodech, které spojíme s bodem $P'$ a dostaneme druhou soustavu).

	Bod $O$ se nazývá bod perspektivity (střed promítání). Přímka $o$ je přímkou perspektivity.

	\begin{poznamkain}
		Bod $O$ nemusí být vlastní.
	\end{poznamkain}

	\begin{poznamkain}[Značení $::$]
		Perspektivita je určená dvěma páry bodů/přímek (potřebujeme najít bod $O$ nebo přímku $o$), proto „dvakrát dvě tečky“.
	\end{poznamkain}
\end{definice}

\begin{dusledek}
	V každé perspektivitě existuje samodružný element: průsečík $p \cap p'$ respektive spojnice $P P'$.
\end{dusledek}

\begin{upozorneni}
	Složení perspektivit obecně není perspektivita! (Nemusí být zachován samodružný element.)
\end{upozorneni}

\begin{definice}[Projektivita]
	Projektivita je složení konečného počtu perspektivit.

	\begin{poznamkain}
		Dá se dokázat, že každá projektivita je složením $≤ 2$ perspektivit.
	\end{poznamkain}
\end{definice}

\begin{dusledek}
	Projektivita obecně nemá samodružný element, ale pokud už ho obsahuje, již je perspektivitou.
\end{dusledek}

\begin{upozorneni}
	Perspektivita nezachovává dělicí poměr 3 bodů.
\end{upozorneni}

\begin{tvrzeni}
	Perspektivita zachovává dvojpoměry 4 bodů.
\end{tvrzeni}

\begin{dusledek}
	Projektivita zachovává dvojpoměry 4 bodů.
\end{dusledek}

\begin{tvrzeni}[Lze dokázat i opak]
	Pokud zobrazení zachovává kolinearitu a dvojpoměr, je to nutně projektivita.
\end{tvrzeni}

\begin{poznamka}[Druhý způsob (analytický) zavedení projektivity a perspektivity]
	Nejprve se zavede projektivní souřadný systém (PSS) na projektivní přímce. Je to trojice bodů $0, 1, ∞$. Souřadnice bodu $X$ vůči tomuto PSS je homogenní dvojice $[1:x]$, kde $x = (X 1 0 ∞)$. Pak projektivní zobrazení je $®RP^1 \rightarrow ®RP^1$, $[x_0:x_1] \mapsto [x_0':x_1']$, kde $(x_0', x_1') = A·(x_0, x_1)^T$, kde $A$ je regulární matice $2 \times 2$ určená až na násobek $≠0$.

	\begin{dusledekin}
		Projektivita zachovává dvojpoměr.
	\end{dusledekin}

	Pak perspektivita = projektivita mající samodružný bod.
\end{poznamka}

\begin{poznamka}[Značení projektivity]
	Projektivitu značíme $p(A, B, C) ::: p'(A', B', C')$.

	\begin{poznamkain}
		Projektivita je určena třemi páry bodů.
	\end{poznamkain}
\end{poznamka}

\begin{definice}[Perspektivita nesourodých soustav]
	Dvě nesourodé soustavy jsou v perspektivitě, je-li jedna soustava průmětem/průsekem té druhé.
\end{definice}

\begin{veta}
	Dvě sourodé nesoumístné soustavy jsou v perspektivitě $\Leftrightarrow$ obě jsou v perspektivitě s touž nesourodou soustavou.

	\begin{dukazin}
		Obrázkem. (Dává nám to přesně ty body a přímky, které potřebujeme.)
	\end{dukazin}
\end{veta}

\begin{poznamka}[Doplňování soustav]
	Doplňování perspektivit ($p(A, B, C)::p'(A', B', C')$ dáno, k bodu $X$ na $p$ doplňte $X'$) je jednoduché.

	Doplňování projektivit ($p(A, B, C):::p'(A', B', C')$ dáno, m bodu $X$ na $p$ doplňte $X'$) je těžší, budeme potřebovat následující větu.
\end{poznamka}

\begin{veta}[O direkční přímce]
	Nechť $p(A, B, C) ::: p'(A', B', C')$ je projektivita nesoumístných bodových soustav. Pak průsečíky spojnic $AB'$ a $A'B$, $AC'$ a $CA'$, $BC'$ a $CB'$ leží na jedné přímce $d$.

	\begin{dukazin}
		Zvolme si význačné body $A$, $A'$ a uvažujme přímky: $a = AA'$, $b = AB'$ a $c = AC'$, stejně tak $a' = A'A$, $b' = A'B$, $c' = A'C$. Hned je jasné, že $a = a'$.

		Pak máme $A(a, b, c)::p'(A', B', C'):::p(A, B, C)::A'(a', b', c')$. Tedy $A(a, b, c):::A'(a', b', c')$. Ale ta má samodružnou přímku $a = a'$, tedy je to perspektivita 2 přímkových soustav, tedy podle předchozí věty jsou obě v perspektivitě s touž nesourodou soustavou (body na přímce). Označme ji $d$. A víme, že se s ní protínají odpovídající si páry přímek, tj. páry $B = AB'$, $b' = A'B$, $c = AC'$, $c' = A'C$.

		Potřebujeme ukázat, že přímka $d$ nezávisí na volbě páru $AA'$. A tím také to, že také přímky $BC'$, $B'C$ se protínají na $d$. Označme $M = N'$ průsečík $p \cap p'$. Kde je $M'$ a $N$? Platí $M' = p' \cap d$ a $N = p \cap d$. $N \in p$ zřejmé (leží v soustavě $p(A, B, C)$). $N \in d$? máme přímky $n = AN = p$, $n' = A'N$, víme, že průsečík $n \cap n' = N \in d$. Stejně tak $M' = p' \cap d$.

		Důsledek: $d = M'N$, ale $M$, $N'$ nezáleží na volbě páru $AA'$, tedy máme hotovo.
	\end{dukazin}
\end{veta}

\begin{definice}[Direkční přímka projektivity]
	Přímku z předchozí věty nazveme direkční přímka projektivity.
\end{definice}

\begin{poznamka}
	Projektivita je perspektivita $\Leftrightarrow$ $p \cap p' \in d$.
\end{poznamka}

\begin{priklad}
	Doplňování projektivity (nesoumístných soustav) je teď jednoduché. Doplňování projektivity soumístných soustav uděláme přes další soustavu.
\end{priklad}

\begin{priklad}
	Spojení bodu $V$ s nepřístupným průsečíkem přímek $p, p'$.

	\begin{reseni}
		Na $p$ a $p'$ doplníme body $A$, $A'$, $B$, $B'$ tak, aby $V \in AB', BA'$. $AA'$ a $BB'$ se protínají v bodě, ze kterého vedeme přímku, na která protne $p$ a $p'$ v bodech $C$ a $C'$. Nyní najdeme direkční přímku.
	\end{reseni}
\end{priklad}

\begin{veta}[Pappova o šestiúhelníku]
	Stejné jako věta o direkční přímce. (Jinak formulovaná.)
\end{veta}

% 09. 11. 2023

TODO!!! (charakteristika projektivity, involuce, …)

\begin{definice}[Involuce]
	Involuce je projektivita (soumístných soustav) splňující jednu z následujících ekvivalentních podmínek:
	\begin{itemize}
		\item $ω = (XX'ST) = -1$; (tzv. charakteristika projektivity, $X$ a $X'$ je libovolný pár, $S$ a $T$ jsou různé samodružné elementy)
		\item $\exists X ≠ S, T: X'' = X$;
		\item $\forall X: X'' = X$.
	\end{itemize}
\end{definice}

\begin{definice}[Hyperbolická a eliptická involuce]
	$S, T$ reálné různé $\implies$ involuce je hyperbolická (nesouhlasné soustavy, neboli směr je proti).

	$S, T$ komplexně sdružené $\implies$ eliptická (souhlasné soustavy, neboli směr se zachovává).

	\begin{poznamka}
		$S = T$ je parabolická involuce (ale není to projektivita, neboť to není prosté zobrazení).
	\end{poznamka}
\end{definice}

% 16. 11. 2023

\begin{poznamka}[Pár involuce]
	$X \rightarrow X'$ a $X' \rightarrow X$, pak $X, X'$ je pár involuce.
\end{poznamka}

\begin{poznamka}
	Projektivita je určena 3 páry, involuce je určena 2 páry.
\end{poznamka}

\begin{dusledek}
	$A A' B' B$ a $B A A' B'$ (ve skutečnosti vzhledem k zahrnutí $∞$ jsou to stejné případy) jsou hyperbolické (říkáme páry se nerozdělují), $A B A' B'$ jsou hyperbolické (páry se rozdělují).
\end{dusledek}

\begin{priklad}[Konstrukce]
	Určit druhý samodružný bod ($T$) involuce určené jedním samodružným bodem ($S$) a jedním párem ($A$, $A'$).

	\begin{reseni}
		Využijeme $(AA'ST) = -1$ a najdeme čtvrtý harmonický bod, jak jsme to již dělali.
	\end{reseni}
\end{priklad}

\begin{veta}[O bodu na direkční přímce]
	Mějme projektivitu $p(A, B, C) ::: p'(A', B', C')$ (nesoumístných soustav), $d$ nechť je direkční přímka, $H \in d$ libovolný bod na direkční přímce. Pak páry přímek $a = HA$, $a' = HA'$, $b = HB$, $b' = HB'$, atd. jsou páry téže involuce přímek ve svazku se středem $H$. Neboli $H(a, b, c) ::: H(a', b', c')$ je involuce.

	\begin{dukazin}
		„1. Projektivita“:
		$$ H(a, b, c) :: p(A, B, C) ::: p'(A', B', C') :: H(a', b', c'). $$
		(Složení tří projektivit je projektivita, tedy existuje $H(a, b, c) ::: H(a', b', c')$)

		„2. Involuce“: Tato projektivita je involuce, protože pokud označíme $X = a' \cap p \implies x = a'$, pak z věty o direkční přímce platí $X' = a \cap p' \implies a = x'$, tedy $(a, a')$ je pár involuce a to už stačí.

		(Pokud $H \in AA'$, zvolíme místo $A$ jiný bod.)
	\end{dukazin}
\end{veta}

\begin{veta}[O přímce procházející direkčním bodem]
	Mějme projektivitu $P(a, b, c) ::: P'(a', b', c')$, $D$ nechť je direkční bod a $D \in h$. Pak páry bodů $A = h \cap a$, $A' = h \cap a'$, $B = h \cap b$, $B' = h \cap b'$ jsou páry téže involuce bodů na přímce $h$. ($h(A, B, C) ::: h(A', B', C')$ je involuce.)

	\begin{dukazin}
		Dualita.
	\end{dukazin}
\end{veta}

\begin{priklad}
	Doplňování bodové involuce dané dvěma páry.

	\begin{reseni}
		Použijeme předchozí větu a doplníme projektivitu. (Další způsoby jsou klasicky jako doplňování projektivity, nebo použití původní, neduální, verze předchozí věty.)
	\end{reseni}
\end{priklad}

\subsection{Úplný čtyřroh a úplný čtyřstran}
\begin{definice}[Úplný čtyřroh]
	Čtveřice bodů v rovině ($M$, $N$, $P$, $Q$), přičemž žádné tři nejsou kolineární, se nazývá čtyřroh.

	Tyto body ($M$, $N$, $P$, $Q$) jsou vrcholy čtyřrohu. Jejich 6 spojnic jsou strany čtyřrohu.

	Máme 3 páry protějších stran ($MN$ a $PQ$, $MP$ a $NQ$, $MQ$ a $NP$), 3 diagonální vrcholy = průsečíků protějších stran ($X$, $Y$, $Z$) a 3 diagonální strany ($XY$, $XZ$, $YZ$).

	Všemu tomuto dohromady se říká úplný čtyřroh.
\end{definice}

\begin{definice}[Úplný čtyřstran]
	Čtveřice přímek v rovině ($m$, $n$, $p$, $q$), přičemž žádné tři neprochází jedním bodem, se nazývá čtyřstran.

	Tyto přímky ($m$, $n$, $p$, $q$) jsou strany čtyřstranu. Jejich 6 průsečíků jsou vrcholy čtyřstrnu.

	Máme 3 páry protějších vrcholů ($m \cap n$ a $p \cap q$, $m \cap p$ a $n \cap q$, $m \cap q$ a $n \cap p$), 3 diagonální strany = spojnice protějších vrcholů ($x$, $y$, $z$) a 3 diagonální vrcholy ($x \cap y$, $x \cap z$, $y \cap z$).

	Všemu tomuto dohromady se říká úplný čtyřstran.
\end{definice}

\begin{veta}
	Každá (i diagonální) strana úplného čtyřrohu je proťata ostatními stranami jen ve 4 bodech, které tvoří harmonickou čtveřici.

	\begin{dukazin}
		„První část“ se spočítá z obrázku. „Druhá část“ je vidět z konstrukce čtvrtého bodu harmonické čtveřice.
	\end{dukazin}
\end{veta}

\begin{veta}[Duální k přechozí]
	Každý (i diagonální) vrchol úplného čtyřstranu je spojen s ostatními vrcholy pouze 4 přímkami, ty tvoří harmonickou čtveřici.
\end{veta}

\begin{veta}[O přímce a čtyřrohu]
	Je dán úplný čtyřroh a libovolná přímka $h$ různá od jeho 9 stran. Pak protější strany čtyřrohu vytínají na $h$ páry téže involuce.

	\begin{dukazin}
		Nechť $P$ a $Q$ jsou středy svazků $a = PM$, $b = PN$ a $a' = QN$, $b' = QM$ ($a$ a $a'$ protější, $b$ a $b'$ taktéž). Páry $AA'$, $BB'$ zadávají involuci na $h$. Je pár $CC'$ také párem této involuce?

		Zároveň máme projektivitu $P(a, b, …) ::: Q(a', b', …)$. Dle věty o přímce procházející direkčním bodem je průsečík $D = C' = MN \cap h$ direkčním bodem této projektivity a proto $D' = C = PQ \cap h$. Tedy $DD' = C'C$ je pár téže involuce.
	\end{dukazin}
\end{veta}

\begin{veta}[O bodu a čtyřstranu]
	Je dán úplný čtyřstran a bod $H$ různý od jeho vrcholů. Pak spojnice protějších vrcholů čtyřstranu s $H$ tvoří páry téže involuce.
\end{veta}

\begin{poznamka}
	$A$, $A'$, $B$, $B'$, $C$, $C'$ z předchozí věty (původní verze) se nazývá čtyřstranná množina.
\end{poznamka}

\section{Kuželosečky}
\begin{definice}[Bodová kuželosečka]
	Mějme projektivitu nesoumístných přímkových soustav $H(a, b, c) ::: H'(a', b', c')$. Bodová kuželosečka $©B = $ množina průsečíků odpovídajících si přímek (tj. $a \cap a'$, $b \cap b'$, atd.).
\end{definice}

\begin{veta}
	Zadaná projektivita je perspektivitou $\Leftrightarrow$ kuželosečka $B$ se skládá ze dvou přímek, a sice přímky $HH'$ a z přímky perspektivity.

	\begin{dukazin}
		Z obrázku a rozpravy nad ním.
	\end{dukazin}
\end{veta}

\begin{definice}[Singulární a regulární]
	Když $H::H'$ ©B je singulární. V opačném případě je regulární.
\end{definice}

\begin{tvrzeni}[Platí]
	$H, H' \in ©B$. (Pro singulární kuželosečku celá $H H' \in ©B$. Pro regulární křivku $H = n \cap n'$ a $H' = m \cap m'$, tedy $H, H' \in ©B$.)
\end{tvrzeni}

\begin{poznamka}
	Dále budeme uvažovat jen regulární křivky.
\end{poznamka}

\begin{definice}[Vzájemná poloha přímky a kuželosečky]
	Přímka v rovině je (ve vztahu ke kuželosečce)
	\begin{itemize}
		\item vnější přímka, pokud nemají žádný společný bod;
		\item tečna, má-li jeden průsečík;
		\item sečna, má-li dva průsečíky.
	\end{itemize}
\end{definice}

\begin{veta}
	Bodem $H$ (resp. $H'$) prochází jediná tečna, a sice $n$ (resp. $m'$), kde $m = n' = HH'$. Průsečík $D = n \cap m'$ je direkčním bodem zadané projektivity.

	\begin{dukazin}
		Přímka $x \in H(a, b, c)$ protíná kuželosečku ©B ve 2 bodech, pouze pro $x = n$ tyto 2 body splývají do 1 bodu ($H$). Podobně pro $H'$, $m'$. $D = n \cap m'$ už víme.
	\end{dukazin}
\end{veta}

\begin{veta}
	Je-li dána ©B pomocí projektivity $H(a, b, c) ::: H'(a', b', c')$ a zvolíme-li 5 bodů na ©B: $K$, $K'$, $A$, $B$, $C$ a označíme-li $α = KA$, $β = KB$, …, pak projektivita $K(α, β, γ) ::: K(α', β', γ')$ zadává tutéž kuželosečku.

	\begin{dukazin}
		Vynechán.
	\end{dukazin}
\end{veta}

\begin{dusledek}
	V definici kuželosečky můžeme vzít za středy svazků libovolné dva body na kuželosečce.
\end{dusledek}

\begin{dusledek}
	Každým bodem (regulární) kuželosečky prochází jediná tečna.
\end{dusledek}

\begin{dusledek}
	Kuželosečka je zadána 5 body (nebo šesti přímkami, z nichž 3 a 3 prochází stejným a stejným bodem).
\end{dusledek}

% 23. 11. 2023

\begin{priklad}[Konstrukce (!!!)]
	Sestrojit kuželosečku z 5 bodů. (Tj. dány body $H, H', A, B, C$, najít alespoň 1 další bod kuželosečky procházející těmito body. Pak umíme najít libovolný konečný počet bodů.)

	\begin{reseni}
		Nalezneme direkční bod a následně provedeme konstrukci druhé přímky v perspektivitě k nějaké zvolené přímce (ta nám určuje, který bod dostaneme).
	\end{reseni}
\end{priklad}

\begin{priklad}[DÚ]
	K zadaným 5 bodům najít 10–15 dalších bodů kuželosečky. (Ručně nebo v geogebře.)
\end{priklad}

\begin{priklad}[Konstrukce]
	Kuželosečka zadána 5 body, v jednom z nich najít tečnu.

	\begin{reseni}
		Tento bod vezmeme jako bod soustavy, k němu zvolíme druhý bod a najdeme direkční bod perspektivity přímkových soustav procházejících zbylými třemi body. Ten spojíme s naším bodem a máme tečnu.
	\end{reseni}
\end{priklad}

\begin{poznamka}
	Tečna s bodem dotyku = 2 podmínky pro kuželosečku.

	Kuželosečka je tedy zadána 5 podmínkami:
	\begin{itemize}
		\item 5 bodů;
		\item 4 body + tečna v jednom z nich;
		\item 3 body + tečny ve dvou z nich.
	\end{itemize}
\end{poznamka}

\begin{priklad}[Konstrukce]
	Sestrojit kuželosečku z jedné tečny a 4 bodů.

	\begin{reseni}
		Zvolíme ze 3 zbývajících bodů bod druhé přímkové soustavy. Poté průsečík tečny a spojnice (správných) průsečíků vzniklých 4 přímek je direkční bod.
	\end{reseni}
\end{priklad}

\begin{priklad}[Konstrukce]
	Sestrojit kuželosečku ze dvou tečen a 3 bodů.

	\begin{reseni}
		Zde máme direkční bod rovnou.
	\end{reseni}
\end{priklad}

\subsection{Soustavy na bodové kuželosečce}
\begin{definice}
	$©B(A, B, C)$ je bodová soustava na ©B. Zase mohou bít soumístné/nesoumístné. Perspektivita soustavy na kuželosečce a soustavy na přímce je „promítnutí“ bodů soustavy $©B(A, B, C)$ z libovolného bodu $\in ©B$ na danou přímku. Složení perspektivit je zase projektivita. Dvojpoměr 4 bodů na ©B je definován přenesením na bodovou soustavu na přímce (zachovává se v každé projektivitě).

	\begin{poznamka}
		Projektivita je dána 3 páry bodů.

		Projektivita soumístných soustav na ©B má 2/1/0 samodružných bodů.
	\end{poznamka}
\end{definice}

\begin{veta}
	Je-li dána projektivita $©B(A, B, C) ::: ©B(A', B', C')$, pak průsečíky $AB' \cap A'B$, $AC' \cap A'C$, $BC', B'C$ leží na jedné, tzv. direkční přímce $d$. Navíc $d \cap ©B$ jsou samodružné body dané projektivity.

	\begin{dukazin}
		Označíme si $b = A'B$, $b' = AB'$, $c = A'C$, $c = AC'$ a $a = A'A = a'$.
		$$ A(a', b', c') ::: ©B(A', B', C') ::: ©B(A, B, C) ::: A'(a, b, c). $$
		Tato projektivita ($A(a', b', c') ::: A'(a, b, c)$) je perspektivita (neboť $a = a'$ je samodružná). Tedy existuje přímka perspektivity $d$, na níž se kříží $b', b$ a $c, c'$.

		Dále chceme ukázat i $BC' \cap B'C \in d$ a $d \cap ©B = $ samodružné body projektivity. Nejprve ukážeme druhou část (a z ní už plyne první, protože samodružné body nezávisí na volbě $A$): $S = S'$, $T = T'$ samodružné body ($\in ©B$ z definice), potom $S = S' \in d$, neboť $s = A'S$ a $s' = AS'$ se protínají na $d$, ale jediný jejich průsečík je $S = S'$ ($T = T'$ obdobně).
	\end{dukazin}

	\begin{dukazin}
		Důkaz pro 0 samodružných bodů (případně pro 1) by se prováděl algebraicky.
	\end{dukazin}
\end{veta}

\begin{poznamka}
	Direkční přímka je sečnou / tečnou / vnější přímkou k ©B $\Leftrightarrow$ daná projektivita má 2/1/0 reálné samodružné body.
\end{poznamka}

\begin{priklad}[Konstrukce]
	Doplňování projektivity na bodové kuželosečce.

	\begin{reseni}
		Nemůžeme to udělat tak, jak bychom chtěli, protože nemáme „nakreslenou“ kuželosečku (neumíme s ní dělat průsečík).

		Co ale můžeme, můžeme obvyklým způsobem najít dvě přímky procházející doplňovaným bodem a najít jejich průsečík.
	\end{reseni}
\end{priklad}

\begin{definice}[Involuce (totéž, co výše)]
	Involuce je projektivita (soumístných soustav) splňující jednu z následujících ekvivalentních podmínek:
	\begin{itemize}
		\item $ω = (XX'ST) = -1$; (tzv. charakteristika projektivity, $X$ a $X'$ je libovolný pár, $S$ a $T$ jsou různé samodružné elementy)
		\item $\exists X ≠ S, T: X'' = X$;
		\item $\forall X: X'' = X$.
	\end{itemize}
\end{definice}

\begin{poznamka}
	Involuce je dána 2 páry bodů.

	Rozlišujeme involuci hyperbolickou a eliptickou podle toho, zda má 2 nebo 0 samodružných bodů.
\end{poznamka}

\begin{veta}[O involuci na bodové kuželosečce]
	Nechť je dána involuce na ©B dvěma páry bodů $A, A'$; $B, B'$. Pak platí:
	\begin{enumerate}
		\item Na direkční přímce $d$ leží nejen průsečíky $AB'$, $A'B$, … ale i průsečíky $AB$, $A'B'$, …
		\item Spojnice $AA'$, $BB'$, … se protínají v jediném bodě $P$.
		\item Průsečíky $α \cap ©B$ jsou samodružné body involuce ($S$, $T$), přímky $PS$ a $PT$ jsou tečny z bodu $P$ k ©B.
		\item Tečny v bodech $A$, $A'$ se také protínají na $d$.
	\end{enumerate}

	\begin{definicein}
		$d$ se pak nazývá osa involuce, $P$ se nazývá střed involuce.
	\end{definicein}

	\begin{dukazin}[1.]
		Průsečík $AB'$ a $A'B$ $\in d$ z definice $d$. Průsečík $AB$ a $A'B'$ $\in d$ záměnou $A \leftrightarrow A'$.
	\end{dukazin}

	\begin{dukazin}[2.]
		Body $A, A', B, B'$ zadávají úplný čtyřroh. $d$ je jednou z jeho diagonálních stran (plyne z 1. bodu) a to každého takového. Tedy bod $R = AA' \cap d$ zůstává pevný pro každý čtyřroh obsahující body $A$ a $A'$. Dále bod $P$ je protější diagonální vrchol k diagonálni straně $d$. Navíc víme, že $A, R, A', P$ je harmonická čtveřice takže (protože $A$, $R$ a $A'$ jsou pevné) $P$ je pevný pro každou volbu $B, B'$.
	\end{dukazin}

	\begin{dukazin}[3.]
		Víme už, že $S, T = d \cap ©B$ a díky $S = S'$ a $T = T'$ se jedná o tečny.
	\end{dukazin}

	\begin{dukazin}[4.]
		„Limitním přechodem“ z bodu 1.
	\end{dukazin}
\end{veta}

\begin{definice}
	Říkáme též, že involuce je indukována svým středem $P$.

	Taktéž definujeme vnější a vnitřní bod kuželosečky v následující tabulce:

	\begin{tabular}{c|ccc}
		Involuce      & Reálné samodružné body & Osa involuce  & Střed involuce \\ \hline
		hyperbolická  & 2                      & sečna         & vnější bod     \\
		eliptická     & 0                      & vnější přímka & vnitřní bod    \\
		„parabolická“ & 1                      & tečna         & $\in ©B$
	\end{tabular}

	(U parabolické se všechny body zobrazí do $P$.)
\end{definice}

\begin{definice}[Další názvy]
	$P$ = pól přímky $d$, $d$ = polára bodu $P$.
\end{definice}

\begin{poznamka}
	$p =$ vnitřní bod ©B $\implies$ každá přímka z bodu $P$ je sečna ©B.

	$p =$ vnější bod ©B $\implies$ existují právě dvě tečny a ty oddělují sečny od vnějších přímek.
\end{poznamka}

\begin{dusledek}
	Je-li $R = A'A \cap T_1T_2$ ($T_i$ tečné body z bodu $P$), pak $(AA'RP) = -1$.
\end{dusledek}

\subsection{Čtyři malé věty}
\begin{veta}[A]
	Mějme na ©B dány dvě involuce se středy $P ≠ Q$. Tyto dvě involuce mají jediný společný pár, jsou to právě průsečíky $PQ \cap ©B$. Navíc je-li alespoň jeden z bodů $P$, $Q$ vnitřní, je tento pár reálný.

	\begin{dukazin}
		Jednoduchý.
	\end{dukazin}
\end{veta}

\begin{veta}[B]
	Nechť $A, A' =$ pár involuce indukovaný na ©B středem $P$, $Q :=$ průsečík tečen k ©B v bodech $A$ a $A'$. $X, X' = PQ \cap ©B$. Pak $X, X'$ je jediný pár involuce se středem $P$, který splňuje $(XX'AA' = -1)$.

	\begin{dukazin}
		Pro involuci ze středem $Q$ platí $A, A'$ jsou samodružné body. $X, X'$ je pár této involuce, tedy $(XX'AA') = -1$. A dle Věty A je to jediný takový pár.
	\end{dukazin}
\end{veta}

\begin{veta}[C]
	$P =$ libovolný vnější bod ©B. $M, N =$ body dotyku tečen z $P$ k ©B. $A, C = $ 2 (libovolné) body na ©B kolineární s bodem $P$. $B =$ libovolný další bod na ©B. $a:=BA$, $c:=BC$, $m:=BM$, $n:=BN$. Pak $(mnac) = -1$.

	\begin{dukazin}
		Okamžitě z Věty B.
	\end{dukazin}
\end{veta}

\begin{veta}[D]
	$P, M, N, A, C$ jako ve větě $C$. $m:=CM$, $n:=CN$, $a:=CA$ a $c$ je tečna v bodě $C$. Pak $(mnac) = -1$. (Věta C s $C = B$.)

	\begin{dukazin}
		$C = B$ ve Větě C.
	\end{dukazin}
\end{veta}

\begin{priklad}[Konstrukce]
	Kuželosečka je dána 3 body a tečnami ve 2 z nich. Sestrojit tečnu ve 3 bodě.

	\begin{reseni}
		Věta D.
	\end{reseni}
\end{priklad}

% 30. 11. 2023

\section{Tečnové kuželosečky}
\begin{definice}[Tečnová kuželosečka]
	Mějme projektivitu nesourodých bodových soustav $h(A, B, C):::h'(A', B', C')$. Tečnová kuželosečka ©T je množina spojnic odpovídajících si bodů.

	\begin{poznamkain}
		Tedy prvky tečnové kuželosečky nejsou body, ale tečny klasické kuželosečky. (Tj. budeme je nazývat tečny kuželosečky ©T.)
	\end{poznamkain}
\end{definice}

\begin{veta}
	$h(A, B, C)::h'(A', B', C')$ $\Leftrightarrow$ ©T se skládá ze dvou bodů (formálněji dvou svazků s těmito středy) a sice středu perspektivity a průsečíku $h \cap h'$.
\end{veta}

\begin{poznamkain}
	Takovýmto kuželosečkám budeme říkat singulární a dále budeme mluvit jen o těch, co to nesplňují, tedy regulárních kuželosečkách.
\end{poznamkain}

\begin{definice}[Vnější bod, vnitřní bod, bod dotyku]
	Bod v rovině se nazývá vnější bod / bod dotyku / vnitřní bod ©T, pokud jím procházejí 2 / 1 / 0 tečny z ©T
\end{definice}

\begin{dusledek}[Pozorování, z definice]
	$h, h' \in ©T$.

	Na $h, h'$ leží jediný bod dotyku, a to odpovídající body průsečíku přímek. Navíc spojnice těchto bodů je direkční přímka zadané projektivity.
\end{dusledek}

\begin{veta}
	Buď ©T tečnová kuželosečka.

	\begin{itemize}
		\item Na volbě tečen $h, h'$ nezáleží, lze je nahradit jinou dvojicí.
		\item Na každé tečně leží jeden bod dotyku.
		\item ©T je zadána 5 tečnami, nebo 4 tečnami s 1 bodem dotyku, nebo 3 tečnami s 2 body dotyku.
	\end{itemize}
\end{veta}

\begin{priklad}[Konstrukce]
	Tečnová kuželosečka z 5 tečen / 4 tečen a 1 bodu dotyku / 3 tečen a 2 bodů dotyku.

	\begin{reseni}
		V případě bodu dotyku zvolíme body dotyku na $h$ nebo $h'$. A poté najdeme direkční přímku projektivity bodových soustav (procházející body dotyku). Pak vždy najdeme další bod této projektivity a spojíme.
	\end{reseni}
\end{priklad}

\begin{priklad}[Konstrukce]
	Na tečně z tečnové kuželosečky najděte bod dotyku.

	\begin{reseni}
		Zvolíme si danou tečnu jako $h$ a najdeme průsečík direkční přímky s $h$.
	\end{reseni}
\end{priklad}

\begin{dusledek}
	Umíme přecházet mezi tečnovými a bodovými kuželosečkami. Tj. nemusíme rozlišovat ©B a ©T.
\end{dusledek}

\subsection{Další duální tvrzení}
\begin{veta}[O direkčním bodu na ©T]
	$©T(a, b, c):::©T(a', b', c')$ $\implies$ spojnice „křížem“, tj. $a \cap b'$ a $a' \cap b$, … prochází jedním bodem, tzv. direkčním bodem projektivity, $D$. Tečny z bodu $D$ jsou samodružné přímky této pojektivity.
\end{veta}

\begin{veta}[O involuci na ©T]
	Je dána involuce tečen na ©T (dvěma páry $a$ a $a'$, $b$ a $b'$). Pak
	\begin{itemize}
		\item direkčním bodem $D$ procházejí nejen spojnice průsečíků $a \cap b'$, $a' \cap b$, ale též $a \cap b$, $a' \cap b'$;
		\item průsečíky $a \cap a'$, $b \cap b'$ leží na jedné přímce $p$;
		\item samodružné přímky involuce jsou tečny z bodu $D$ $=: m, n$, jejichž body dotyku jsou průsečíky s $p$;
		\item spojnice bodů dotyku tečen $a, a'$ prochází $D$.
	\end{itemize}

	\begin{definicein}[Střed involuce, osa involuce]
		Bodu $D$ říkáme střed involuce. Přímce $p$ říkáme osa involuce.
	\end{definicein}
\end{veta}

\begin{definice}
	Říkáme též, že involuce je indukována svojí osou $p$.

	Taktéž definujeme sečnu a vnější přímku kuželosečky v následující tabulce:

	\begin{tabular}{c|ccc}
		Involuce      & Reálné samodružné body & Osa involuce  & Střed involuce \\ \hline
		hyperbolická  & 2                      & sečna         & vnější bod     \\
		eliptická     & 0                      & vnější přímka & vnitřní bod    \\
		„parabolická“ & 1                      & tečna         & $\in ©B$
	\end{tabular}

	(U parabolické se všechny body zobrazí do $P$.)
\end{definice}

\begin{dusledek}
	$(aa'rp) = -1$ ($a, a'$ odpovídající si tečny, $p$ osa involuce, $r$ spojnice průsečíku všech tří předchozích a $D$).
\end{dusledek}

\begin{veta}[D$^*$]
	$p$ libovolná sečna ©T, $m, n$ tečny v průsečících $p \cap ©T$. $a, c \in ©T$ libovolné, konkurentní (duál kolineární, česky sbíhavé) s přímkou $p$. $C$ bod dotyku na $c$. $M = m \cap c$, $N = n \cap c$, $A = a \cap c \cap p$. Potom $(MNAC) = -1$.
\end{veta}

\begin{priklad}[Konstrukce]
	©T je dána 3 tečnami a 2 body dotyku. Najděte bod dotyku na třetí tečně.

	\begin{reseni}
		Z minulé věty a hledání čtvrtého harmonického bodu.
	\end{reseni}
\end{priklad}

\section{Elipsa, parabola, hyperbola, aneb afinní klasifikace (regulárních) kuželoseček}
\begin{poznamka}
	V $®R P^2$ jsou nerozlišitelné.
\end{poznamka}

\begin{poznamka}
	Nyní tedy rozlišujeme vlastní a nevlastní. Tj. máme
	\begin{enumerate}
		\item rozlišení vlastní/nevlastní;
		\item rozlišení rovnoběžek a různoběžek;
		\item dělicí poměr 3 bodů (připomenutí: $(ABC) = (ABCD_∞)$, kde $D_∞$ je směr dané přímky);
		\item úsečky (neobsahuje nevlastní bod / množina bodů, že $ABX < 0$);
		\item střed úsečky ($(ABSD_∞) = (ABS) = -1$);
		\item rovnoběžný přenos délek (2 úsečky na rovnoběžkách jsou stejně dlouhé, když jsou to protější strany rovnoběžníku).
	\end{enumerate}
	Ale ztratili jsme dualitu. (Nevlastní bod není duální k nevlastní přímce.)
\end{poznamka}

\begin{priklad}[Konstrukce]
	Najděte střed úsečky.

	\begin{reseni}
		Přes konstrukci čtvrtého harmonického bodu. Nebo jako průsečík úhlopříček rovnoběžníku (dokážeme přes úplný čtyřroh).
	\end{reseni}
\end{priklad}

\begin{definice}[Elipsa, parabola, hyperbola, asymptoty, střed kuželosečky, průměr kuželosečky, omezený průměr kuželosečky]
	Kuželosečka je elipsa/parabola/hyperbola pokud má 0/1/2 reálné nevlastní body. (Elipsa má 2 komplexní nevlastní body.)

	Asymptoty kuželosečky jsou tečny v nevlastních bodech. (Tj. pro E/P/H máme 0/1/2 reálné asymptoty.)

	Střed kuželosečky je průsečík jejích asymptot. (Pro parabolu za střed považujeme její nevlastní bod. Pro elipsu se ukáže, že ty dvě imaginární asymptoty mají reálný průsečík.) Podle toho rozlišujeme středové kuželosečky (elipsa, hyperbola) a nestředové/osové kuželosečky (parabola).

	Průměr kuželosečky je libovolná přímka procházející středem dané kuželosečky. Omezený průměr je úsečka na průměru vymezená vlastními průsečíky.
\end{definice}

\begin{poznamka}
	Polára ($p$) je spojnice tečných bodů tečen z pólu ($P$). Pól je průsečík tečen z průsečíků kuželosečky s polárou.
\end{poznamka}

\begin{dusledek}
	Střed kuželosečky je pól nevlastní přímky vzhledem k této kuželosečce.
\end{dusledek}

\begin{veta}
	Střed E/H půlí (je středem) každý její omezený průměr.

	\begin{dukazin}
		Víme, že střed je pól nevlastní přímky. A my víme, že $(ARA'P)$, kde $P = S$.
	\end{dukazin}
\end{veta}

\begin{veta}
	Nechť jsou spojnice bodů $X$ a $X'$, $Y$ a $Y'$ (na kuželosečce) rovnoběžné. Pak spojnice středů úseček $XX'$ a $YY'$ prochází středem kuželosečky ($S$).

	\begin{dukazin}
		„Pro elipsu a hyperbolu“: $X$ a $X'$, $Y$ a $Y'$ jsou páry involuce indukované nevlastním bodem (bodem $P = XX' \cap YY'$). A $(AA'RP) = -1$, kde $A = X$ a $A' = X'$, nám dává, že $R = S_X$ leží na ose této involuce a z předchozí věty máme, že $S$ tam leží také.

		„Pro parabolu“: Zase uvažujme involuci danou $XX'$, $YY'$. Její střed je nevlastní bod, tedy jedna z tečen z tohoto bodu je nevlastní přímka, tedy bod dotyku je střed paraboly a zároveň jím prochází osa involuce.
	\end{dukazin}
\end{veta}

\begin{priklad}[Konstrukce]
	Najít střed kuželosečky dané pěti body.

	\begin{reseni}
		Provedeme dvakrát předchozí větu.
	\end{reseni}
\end{priklad}

% 07. 12. 2023

\begin{poznamka}
	Středu v parabole se také říká směr průměrů (nebo směr osy).
\end{poznamka}

\begin{veta}
	Spojnice bodů dotyku rovnoběžných tečen elipsy/hyperboly je průměrem této kuželosečky.

	\begin{dukazin}
		Uvažujme involuci indukovanou společným směrem tečen (označme si ho $P_∞$). Spojnice bodů dotyku je její osa a díky harmonické čtveřici $(AA'SP_∞) = -1$ tam leží i $S$, kde $A, A'$ je  průměr procházející $P_∞$.
	\end{dukazin}

	\begin{poznamkain}
		Tato věta je vlastně limitním přechodem předchozí. (Ale limitní přechody nemáme.)
	\end{poznamkain}
\end{veta}

\begin{upozorneni}
	Parabola nemá rovnoběžné tečny.

	\begin{dukazin}
		Pro spor předpokládejme, že má 2 rovnoběžné tečny $\implies$ jejich průsečík je nevlastní a z něj tedy vedou 3 tečny (ještě nevlastní přímka).
	\end{dukazin}
\end{upozorneni}

\begin{veta}
	Nechť $Y, Y'$ jsou vlastní body dotyku tečen ke kuželosečce z bodu $Z$ a $C = $ střed $YY'$. Pak střed kuželosečky $S \in ZC$.

	\begin{dukazin}
		Uvažujme involuci indukovanou směrem přímky $YY'$. Její osa prochází body $Z$ (z věty o involuci, bod 4), $C$ (harmonická čtveřice) a $S$ (harmonická čtveřice).
	\end{dukazin}
\end{veta}

\begin{priklad}[Konstrukce středu kuželosečky]
	Mějme kuželosečku danou 5 tečnami, zkonstruujte její střed.

	\begin{reseni}
		Nalezneme tečné body (stačí 3) a použijeme předchozí větu.
	\end{reseni}
\end{priklad}

\begin{priklad}[Konstrukce hyperboly]
	Sestrojit hyperbolu včetně středu a asymptot, jsou-li dány 3 body a 2 směry asymptot (5 bodů).

	\begin{reseni}
		Zvolíme směry asymptot jako $H$ a $H'$. Pak direkční bod je střed a jeho spojnice s $H$ a $H'$ jsou asymptoty.
	\end{reseni}
\end{priklad}

\begin{priklad}[Konstrukce hyperboly 2]
	Sestrojte hyperboly ze dvou asymptot a jednoho bodu.

	\begin{reseni}
		Direkční bod projektivity soustav svazků ve směrech asymptot je průsečík asymptot (střed). A máme v podstatě hotovo.
	\end{reseni}

	\begin{poznamkain}
		U hyperboly lze dohledávat body pomocí středové symetrie.
	\end{poznamkain}
\end{priklad}

\begin{veta}[Hyperbola a její asymptoty]
	Hyperbola a její asymptoty vytínají na libovolné sečně stejně dlouhé úsečky.

	\begin{dukazin}
		Z předchozí konstrukce + definice stejně dlouhých úseček.
	\end{dukazin}
\end{veta}

\begin{veta}[Limitní verze předchozí]
	Na tečně také. (Bod dotyku je středem úsečky spojující průsečíky tečny s asymptotami.)
\end{veta}

\begin{poznamka}
	Zkouška bude $±$ na hodinku: dvě konstrukce (provést a zdůvodnit). Domluvíme se na nějakých dvou termínech.
\end{poznamka}

\begin{priklad}[Konstrukce hyperboly 3]
	Sestrojte hyperbolu včetně asymptot a středu, jsou-li dány čtyři body a jeden směr asymptoty.
\end{priklad}

\begin{priklad}[Konstrukce hyperboly 4 a 5]
	DÚ: Sestrojte hyperbolu včetně asymptot a středy, jsou-li dány (3 body + 1 asymptota) nebo (3 tečny a 1 asymptota).
\end{priklad}

\begin{poznamka}
	Úlohu „najít obě asymptoty při zadání 5 vlastních bodů“ zatím neumíme řešit.
\end{poznamka}

\subsection{Speciální konstrukce pro parabolu}
\begin{poznamka}
	Pokud je jeden z pěti bodů v konvexním obalu zbylých 4, pak je to hyperbola. Jak ale zadat parabolu? No tím, že má nevlastní přímku.

	Tedy „parabola je dána 4 tečnami“ (nebo 3 tečny + směr průměrů, nebo 3 tečny a bod na jedné z nich, nebo 2 tečny a směr průměru a bod na jedné z nich, atd.)
\end{poznamka}

\begin{poznamka}[Připomenutí]
	Průměry paraboly jsou rovnoběžné a jejich společný směr je střed, tedy bod dotyku s nevlastní přímkou.

	Parabola nemá rovnoběžné tečny.
\end{poznamka}

\begin{priklad}[Konstrukce paraboly]
	Parabola dána 4 tečnami, najděte 1 další tečnu.

	\begin{reseni}
		Nevlastní přímku zvolíme za $h$. TODO?
	\end{reseni}

	A najít směr průměrů.

	\begin{reseni}
		Je to průsečík direkční přímky v předchozí konstrukci s nevlastní přímkou.
	\end{reseni}
\end{priklad}

% 14. 12. 2023

\begin{priklad}[Konstrukce paraboly 2]
	Sestrojte parabolu, je-li dána 3 tečnami a 1 bodem dotyku (na některé vlastní tečně).

	\begin{reseni}
		Zvolíme si přímku s bodem a nevlastní přímku jako bodové soustavy, najdeme direkční přímku (prochází tečným bodem) a postupujeme jako obvykle.
	\end{reseni}
\end{priklad}

\begin{priklad}[Konstrukce tečny s daným směrem]
	K parabole dané 4 tečnami sestrojte tečnu s daným směrem.

	\begin{reseni}
		Najdeme direkční přímku a pak daným směrem vedeme přímku procházející jedním z bodů. Její průsečík s direkční přímkou spojíme s dalším bodem a vedeme rovnoběžku.
	\end{reseni}
\end{priklad}

\begin{priklad}[Konstrukce paraboly 3]
	Sestrojte parabolu ze tří bodů a směru průměru.

	\begin{reseni}
		Úloha 4 body a tečna -> kuželosečka.
	\end{reseni}
\end{priklad}

\subsection{Elipsa/Kružnice}
\begin{poznamka}
	Pro elipsu nejsou žádné speciální konstrukce (nemá speciální tečny / body). Budeme tedy studovat kružnici.

	Jak ale kružnici definovat? a) Skoro euklidovsky: potřebujeme pojem stejně dlouhé úsečky i na různoběžkách. b) Pomocí kolmosti: tj. k afinní geometrii přidáme pojem kolmosti.
\end{poznamka}

\begin{poznamka}
	Kružnice $\implies$ přenos délek mezi různoběžkami $\implies$ půlení úhlu $\implies$ kolmost.

	Z kolmosti pak přes Thaletovu větu můžeme konstruovat bodově kružnici.
\end{poznamka}

\begin{definice}[Absolutní involuce]
	Absolutní involuce ve svazku přímek je involuce daná dvěma páry kolmic.

	\begin{poznamkain}
		Je to otočení o 90 stupňů.
	\end{poznamkain}

	\begin{poznamkain}[Pozorování]
		Jde o eliptickou involuci (páry se rozdělují, nemá reálné samodružné přímky).

		Má 2 komplexně sdružené samodružné přímky, tzv. izotropické přímky. Jejich směry se nazývají izotropické body.
	\end{poznamkain}
\end{definice}

\begin{definice}
	Kružnice je elipsa, jejíž asymptoty jsou izotropické přímky procházející jejím středem.
\end{definice}

\begin{dusledek}
	Proto je každá kružnice dána třemi body (resp. třemi podmínkami). (Zbývající dva body jsou body na asymptotách.)
\end{dusledek}

\begin{priklad}[Konstrukce kružnice]
	Sestrojte kružnici, je-li dáno
	\begin{itemize}
		\item 3 body (průsečík os je střed, podle toho uděláme středovou souměrnost a máme 6 bodů);
		\item 2 body a 1 tečna (průsečík osy a kolmice na tečnu v bodě je střed);
		\item 1 bod a 2 tečny (bez kružítka asi nejde, potřebujeme osu úhlu).
	\end{itemize}
\end{priklad}

\begin{poznamka}
	V praxi budeme používat kružítko.

	Platí všechny obvyklé vlastnosti kružnice. Zejména umíme vést ke kružnici tečny z vnějšího bodu.
\end{poznamka}

\begin{priklad}[Konstrukce samodružné body projektivity]
	Sestrojte samodružné body projektivity soumístné bodové soustavy dané 3 páry bodů.

	\begin{reseni}
		Sestrojíme tečny z daných bodů k libovolně zvolené kružnici dotýkající se přímky. ($p(A, B, C):::p(A', B', C') \implies k(a, b, c):::k(A', B', C')$). Najdeme direkční bod a z něho vedeme tečny ke kružnici. Ty jsou tedy samodružné body projektivity na tečnové kuželosečce, tedy jejich průsečíky s bodovou soustavou (její přímkou) jsou samodružné body.
	\end{reseni}

	\begin{poznamkain}
		Kdyby direkční bod vyšel na kružnici, pak je jeden dvojitý samodružný bode, a kdyby byl vnitřní, tak jsou samodružné body imaginární.
	\end{poznamkain}
\end{priklad}

\begin{priklad}[Konstrukce duální k předchozí]
	Sestrojte samodružné přímky projektivity soumístných přímkových soustav.

	\begin{reseni}
		Duálně.
	\end{reseni}
\end{priklad}

\begin{priklad}[Konstrukce průsečík kuželosečky s přímkou]
	Najít průsečíky kuželosečky dané 5 body s danou přímkou.

	\begin{reseni}
		Jako obvykle vyrobíme soustavy $H(a, b, c):::H'(a', b', c')$. Ty nám na přímce vytvoří další soustavy v projektivitě a nalezneme samodružné body.
	\end{reseni}
\end{priklad}

\begin{priklad}[Konstrukce duální k předchozí]
	Kuželosečka dána 5 tečnami a dán bod. Nalezněte tečny procházející $R$.
\end{priklad}

\begin{priklad}[Konstrukce asymptot v hyperbole]
	Najděte asymptoty hyperboly zadané 5 body.

	\begin{poznamkain}
		Tj. najít zejména směry asymptot, jelikož střed už najít umíme.
	\end{poznamkain}

	\begin{reseni}
		Směry asymptot jsou průsečíky s nevlastní přímkou, tedy minulá konstrukce. Ale máme nevlastní přímku. Takže to uděláme tak, že pomocnou kružnici položíme procházející bodem $H$, přeneseme si na ní body $A$, $B$, $C$ a pomocí přímek rovnoběžných s čárkovanými přímkami i body $A'$, $B'$, $C'$.
	\end{reseni}
\end{priklad}

% 21. 12. 2023

\section{Pascalova a Brianchonova věta}
\begin{poznamka}[Idea]
	Kuželosečka je dána 5 body. Tedy 6 bodů na kuželosečce musí být nějak vázáno.
\end{poznamka}

\begin{veta}[Pascalova]
	Šest bodů 1, 2, 3, 4, 5, 6 leží na kuželosečce $\Leftrightarrow$ průsečíky spojnic 12 a 45, 23 a 56, 34 a 61 leží na jedné, tzv. Pascalově přímce ($p$).

	\begin{dukazin}
		„$\implies$“: Označíme $1 = A$, $2 = B'$, $3 = C$, $4 = A'$, $5 = B$, $6 = C'$. Pak $p$ = direkční přímka projektivity $©B(A, B, C) ::: ©B(A', B', C')$.

		„$\impliedby$“: Mějme kuželosečku danou body $1, …, 5$. Ukážeme, že i $6$ na ní leží. Označme $6' ≠ 5$ průsečík ©B a přímky $56$ (a chceme ukázat, že $6' = 6$). Protože $6' \in 56$, tak $56' = 56$. Spojnice $12$ a $45$, $23$ a $56'$ se protínají na $p$, na které leží i průsečík $34$ a $6'1$ (dle první části, $p = $ Pascalova přímka pro $1, …, 5, 6'$). Zároveň dle předpokladu věty se na téže přímce $p$ protínají i $34$ a $61$. Proto nutně $6' = 6$.
	\end{dukazin}
\end{veta}

\begin{dusledek}[Speciální případ pro singulární kuželosečku 2 přímky]
	Pappova věta.
\end{dusledek}

\begin{poznamka}[Historická]
	Původní důkaz Pascalovy věty byl euklidovský.
\end{poznamka}

\begin{poznamka}
	Pro $6$ bodů existuje celkem $60 = \frac{6!}{6·2}$ Pascalových přímek.
\end{poznamka}

\begin{priklad}[Konstrukce]
	Kuželosečka je dána $5$ body, přímka $x$ prochází jedním z nich. Najděte druhý průsečík $x \cap ©B$.

	\begin{reseni}
		Použijeme Pascalovu větu. (Bod s přímkou musí být číselně sousední s hledaným.)
	\end{reseni}
\end{priklad}

\begin{priklad}[Konstrukce]
	Kuželosečka dána 5 body, sestrojte v jednom z nich tečnu.

	\begin{reseni}
		Označíme si jeden bod dvěma sousedními čísly a použijeme Pascalovu větu. (Tečna bude rovnoběžka)
	\end{reseni}
\end{priklad}

\begin{priklad}[Konstrukce]
	Kuželosečka je dána 3 body a tečnami ve dvou z nich. Sestrojte tečnu ve třetím bodě.

	\begin{reseni}
		Označíme si každý bod dvěma sousedními čísly a použijeme Pascalovu větu.
	\end{reseni}

	\begin{poznamkain}
		Odpovídá řešení za pomoci nalezení harmonické čtveřice (po Malé větě D).
	\end{poznamkain}
\end{priklad}

\begin{veta}[Brianchonova (duální k Pascalově)]
	6 přímek ($1, …, 6$) je tečnami k téže kuželosečce $\Leftrightarrow$ spojnice průsečíků $1 \cap 2$ a $4 \cap 5$, $2 \cap 3$ a $5 \cap 6$, $3 \cap 4$ a $6 \cap 1$, prochází jedním, tzv. Brianchonovým bodem.
\end{veta}

\begin{poznamka}[Historická]
	Také byla dokázána ještě eukleidovsky.
\end{poznamka}

\begin{priklad}[Konstrukce]
	Kuželosečka dána 5 tečnami, bod $X$ leží na jedné z nich. Najděte druhou tečnu z bodu $X$.

	\begin{reseni}
		Duální ke konstrukci výše.
	\end{reseni}
\end{priklad}

\begin{priklad}[Konstrukce]
	Kuželosečka dána 5 tečnami. Sestrojte na jedné z nich bod dotyku.

	\begin{reseni}
		Duální ke konstrukci výše.
	\end{reseni}
\end{priklad}

\begin{priklad}
	Kuželosečka dána 3 tečnami a 2 body dotyku. Najděte bod dotyku na třetí tečně.

	\begin{reseni}
		Duální ke konstrukci výše.
	\end{reseni}
\end{priklad}

\section{Pól a polára}
\begin{poznamka}[Už víme]
	Involuce na kuželosečce: $P$ = střed involuce = pól přímky $p$. $p$ = osa involuce = polára bodu $P$.

	Involuce je hyperbolická $\Leftrightarrow$ $P$ je vnější bod a $p$ sečna, je eliptická $\Leftrightarrow$ $P$ je vnitřní bod a $p$ vnější přímka, „je“ parabolická $\Leftrightarrow$ $P$ je tečný bod tečny $p$.

	Polára $p$ = spojnice bodů dotyku tečen z pólu $P$. Pól $P$ = průsečík tečen v průsečících poláry s kuželosečkou.

	$(AA'RP) = -1 = (aa'rp)$.
\end{poznamka}

\begin{priklad}[Konstrukce]
	Kuželosečka je dána 5 body, sestrojte poláru k danému bodu $P$.

	\begin{reseni}
		TODO!!! Přes Pascalovu větu najdeme další dva body a použili harmonickou čtveřici?
	\end{reseni}
\end{priklad}

\begin{priklad}
	Kuželosečka dána 5 tečnami. Najděte k přímce $p$ pól.

	\begin{reseni}
		Samostudium (duálně k předchozí konstrukci).
	\end{reseni}
\end{priklad}

\begin{veta}
	Dvojice pól+polára jsou 2 podmínky pro kuželosečku.

	\begin{dukazin}
		Nechť je to $x$ podmínek. Předpokládejme, že známe navíc tečny $s, t$ z bodu $p$, tedy máme $x + 2$ podmínek. $S = s \cap p$, $T = t \cap p$ jsou body dotyku, tedy $s, S, t, T$ jsou 4 podmínky, tedy $x + 2 ≥ 4$. Přitom není $x + 2 > 4$, protože podmínkami $P, p, s, t$ není kuželosečka určena jednoznačně. Tedy $x + 2 = 4$, tj. $x = 2$.
	\end{dukazin}
\end{veta}

\begin{priklad}[Konstrukce, kuželosečka z poláry]
	Sestrojte kuželosečku (tj. najděte další 2 body), je-li dán pól, polára a 3 body $A, B, C$.

	\begin{reseni}
		Přes harmonickou čtveřici. (A druhý se dá získat různými triky.)
	\end{reseni}
\end{priklad}

\begin{priklad}
	Duální k předchozímu.

	\begin{reseni}
		Na doma.
	\end{reseni}
\end{priklad}

\begin{poznamka}[Platí]
	Vepíšeme-li kuželosečce úplný čtyřroh, pak dvojice pól+polára = diagonální vrchol + protější diagonální strana čtyřrohu.
\end{poznamka}

\begin{definice}[Názvosloví]
	$PQR$ (se stranami $p, q, r$) je polární trojúhelník.
\end{definice}

\begin{poznamka}[Platí]
	$P \in q \Leftrightarrow Q \in p$.
\end{poznamka}

\begin{definice}
	V situaci z předchozí poznámky nazýváme body $P, Q$ (resp. přímky $p, q$) polárně sdružené (vůči kuželosečce ©B) (sdružené póly, resp. sdružené poláry).

	\begin{poznamkain}
		V polárním trojúhelníku máme 3 dvojice polárně sdružených bodů a 3 dvojice polárně sdružených přímek.
	\end{poznamkain}
\end{definice}

\begin{veta}
	Dvojice sdružených pólů tvoří 1 podmínku pro kuželosečku.

	\begin{dukazin}
		Ať zadávají $x$ podmínek. Předpokládejme, že navíc známe $s, t$ = dvě tečny z bodu $P$. $T$ bod dotyku na $t$. $(P, Q), s, t, T$ je $x + 3$ podmínek. Známe i $S = s \cap p$, tedy $s, S, t, T$ máme 4 podmínky. Tedy $x + 3 ≥ 4$. Ale opět potřebujeme 1 další bod pro určení kuželosečky, tj. $x + 3 = 4$ a $x = 1$.
	\end{dukazin}
\end{veta}

\begin{veta}
	Polární trojúhelník tvoří 3 podmínky pro kuželosečku.

	\begin{dukazin}
		$(P, Q)$, $(Q, R)$, $(P, R)$ jsou tři podmínky a tím už jsou určeny i strany $p, q, r$.
	\end{dukazin}
\end{veta}

\begin{priklad}[Konstrukce]
	Zkonstruujte kuželosečku z polárního trojúhelníku a 2 bodů ($A$, $B$).

	\begin{reseni}
		Přes harmonickou čtveřici dohledáme body.
	\end{reseni}
\end{priklad}

\begin{poznamka}[Bonusová informace]
	V polárním trojúhelníku je vždy jeden vrchol vnitřní a dva vnější.

% 04. 01. 2024

	\begin{dukazin}
		Kuželosečka + 4 body $\implies$ 3 možná párování. Dvě dávají vnější (hyperbolická involuce), jedno vnitřní (eliptická involuce)
	\end{dukazin}
\end{poznamka}

\subsection{Pól a polára v afinní rovině}
\begin{poznamka}[Platí]
	Střed kuželosečky (tedy průsečík asymptot, tedy průsečík tečen v nevlastním bodě) je pólem nevlastní přímky.

	Polára nevlastního bodu prochází středem kuželosečky (je jejím průměrem).
\end{poznamka}

\begin{poznamka}
	Mějme středovou kuželosečku, střed $O$, nějaký její průměr a uvažujme sdružené póly na tomto průměru.

	Sdruženost pólů (na průměru) je symetrická $\implies$ tvoří to involuci sdružených pólů na průměru. Její samodružné body jsou průsečíky ($S$ a $T$) průměru s kuželosečkou.

	\begin{tabular}{c|ccc}
		S,T        & involuce  & průměr        & definice        \\ \hline
		reálné     & hyperbola & sečna         & reálný průměr   \\
		imaginární & eliptická & vnější přímka & pomyslný průměr
	\end{tabular}
\end{poznamka}

\begin{priklad}
	Jak najít k bodu $P$ sdružený pól na průměru?

	\begin{reseni}[1.]
		Body $S, T, P, P'$ tvoří harmonickou čtveřici.
	\end{reseni}

	\begin{reseni}[2.]
		Zavedeme tzv. afinní vzdálenost (na afinní přímce) viz další definice. A pak z druhého důsledku a Euklidovy věty.
	\end{reseni}
\end{priklad}

\begin{definice}[Afinní vzdálenost (na afinní přímce)]
	Zvolí se 2 body $P, J$ (počátek a jednotka) a definujeme afinní vzdálenost bodů $X, Y$ (na téže přímce jako $P, J$) jako číslo $(XY) := (XJP∞) - (YJP∞)$.	
\end{definice}

\begin{dusledek}
	Afinní vzdálenost splňuje:

	\begin{itemize}
		\item Je to orientovaná vzdálenost, tj. $(YX) = -(XY)$.
		\item Jde-li o eukleidovskou přímku, je $(XY) = \konst · d(X, Y)$.
		\item Je-li $(X, X')$ pár involuce, a $(O, ∞)$ pár téže involuce, pak $(XO)·(X'O) = \konst$. Navíc tato konstanta je kladná pro hyperbolickou involuci a záporná pro eliptickou involuci.
	\end{itemize}
\end{dusledek}

\begin{poznamka}[Použití]
	Pro středové kuželosečky máme pro reálný průměr: $(OP)·(OP') = (OS)^2 = (OT)^2 > 0$, tedy $P, P'$ leží na stejné straně od $O$.

	Pro pomyslný průměr zavádíme tzv. náhradní body $A, B$, jsou to body na pomyslném průměru splňující $(OA) = - (OB)$, $(OS)^2 = (OT)^2 = -(OA)^2 = -(OB)^2 < 0$ ($S, T$ jsou ryze imaginární). Potom $(OP)·(OP') = (OS)^2 = -(OA)^2 < 0$, tedy $P, P'$ leží na opačných stranách od $O$.
\end{poznamka}

\begin{dusledek}
	$OS$ je geometrický průměr $(OP)$ a $(O'P)$. (Geometrický průměr se konstruuje přes Euklidovu větu. Tj. nad $a+b$ narýsujeme půlkružnici a v bodě mezi $a,b$ vztyčíme „výšku“ $c$, pak $c^2 = a·b$.)
\end{dusledek}

\subsection{Sdružené průměry}
\begin{definice}[Sdružené průměry kuželosečky, sdružené směry]
	Sdružené průměry kuželosečky jsou sdružené jakožto poláry. Jejich směry se nazývají sdružené směry.
\end{definice}

\begin{poznamka}
	Toto má smysl jen pro středové kuželosečky, pro parabolu je každý průměr sdružen s nevlastní přímkou.

	Páry sdružených průměrů tvoří involuci, jejíž samodružné přímky jsou asymptoty. U elipsy jde o eliptickou involuci, u hyperboly o hyperbolickou.

	Asymptoty oddělují sdružené průměry harmonicky.

	U hyperboly: pár tvoří vždy reálný a pomyslný průměr.

	$R \in q \Leftrightarrow Q_∞ \in r \implies r \parallel p$.
\end{poznamka}

\begin{veta}[Zobecněná afinní verze Thaletovy věty]
	Spojnice libovolného bodu $X$ na kuželosečce s koncovými body $A, B$ jejího libovolného průměru jsou rovnoběžné s nějakou dvojicí sdružených průměrů.

	Speciálně pro kružnici: sdružené průměry jsou kolmé $\Leftrightarrow$ Thaletova věta.

	\begin{dukazin}
		Bez důkazu.
	\end{dukazin}
\end{veta}

\begin{veta}
	Kuželosečka je dána párem omezených průměrů.

	\begin{dukazin}
		Pár sdružených průměrů + nevlastní přímka tvoří polární trojúhelník (3 podmínky) + máme 2 body na průměrech = 5 podmínek.
	\end{dukazin}
\end{veta}

\begin{priklad}[Konstrukce]
	Určete kuželosečku z páru omezených sdružených průměrů $AB$, $CD$, kde $AB$ je reálný.

	\begin{reseni}
		Na $CD$ sestrojíme 1 pár sdružených pólů $P, P'$ a doplníme 5. (6.) bod? TODO?
	\end{reseni}
\end{priklad}

\begin{priklad}[Konstrukce]
	Sestrojte kuželosečku je-li dán omezený průměr $AB$, neomezený průměr $CD$ a 1 další bod $E$.

	\begin{reseni}
		Podobně jako předchozí konstrukce?
	\end{reseni}
\end{priklad}

\begin{priklad}[Konstrukce]
	Středová kuželosečka je dána omezenými sdruženými průměry $AB$, $CD$. Najděte k dané poláře $p$ pól $P$ nebo naopak.

	\begin{reseni}
		Pomocí $X \in p \Leftrightarrow P \in x$ a $Y \in p \Leftrightarrow P \in y$.

		$x \ni$ pól $CD$ = směr $AB$, $y \ni$ pól $AB$ = směr $CD$. Najdeme sdružené póly na průměrech $X', Y'$ (přes geometrický průměr).
	\end{reseni}
\end{priklad}

\begin{poznamka}
	U hyperboly $AB$ = reálný, $CD$ = pomyslný $\implies$ úhlopříčky = asymptoty $\implies$ hyperbol je určená.

	Záměna rolí $AB \leftrightarrow CD$ $\implies$ sdružená hyperbola.
\end{poznamka}

% 11. 01. 2024

\section{Osy středových kuželoseček}
\begin{veta}
	V involuci sdružených průměrů existuje alespoň 1 pár sdružených průměrů, které jsou na sebe kolmé.

	\begin{dukazin}[Konstrukcí]
		Sestrojte osy kuželosečky, jsou-li dány dva páry neomezených sdružených průměrů $(a, a'; b, b')$. ($O$ střed kuželosečky.)

		Zvolíme libovolně pomocnou kružnici procházející bodem $O$ se středem $S$. Její druhé průsečíky s průměry si pojmenujeme $A, A', B, B'$. Víme, že tyto body jsou v involuci na této kružnici. Její střed je $P = AA' \cap BB'$. Nyní když vezmeme průsečík libovolné přímky z $P$ s kružnicí, tak dostaneme sdružené body. Tedy vezmeme průsečíky $PS$ s kružnicí a jejich spojnice s $O$ budou sdružené (z předchozího) kolmé (z Thaletovy věty) průměry.

		Tímto způsobem tyto průměry vždy zkonstuujeme, což dokazuje jejich existenci.
	\end{dukazin}

	\begin{poznamkain}
		Pokud $P = S$, jedná se o kružnici a všechny sdružené průměry jsou kolmé. Jinak vždy jen 1 pár os.

		Asymptoty této kuželosečky jsou spojnice $O$ s tečnými body z $P$.
	\end{poznamkain}
\end{veta}

\begin{definice}[Osy kuželosečky, vrcholy, vrcholové tečny]
	Kolmé sdružené průměry se nazývají osy (středové) kuželosečky. Jejich průsečíky s kuželosečkou se nazývají vrcholy. Tečny v těchto bodech se nazývají vrcholové tečny.

	\begin{poznamkain}
		Kružnice má obě stejně dlouhé, elipsa má osy reálné (hlavní = větší, vedlejší = menší, k tomu už ale potřebujeme umět porovnávat délky), hyperbola má jednu reálnou a jednu pomyslnou (osy hyperboly = osy úhlu asymptot).

		Přidáme-li ke dvěma párům sdružených průměrů 1 bod, máme 5 podmínek.
	\end{poznamkain}
\end{definice}

\begin{priklad}[Konstrukce]
	Sestrojte kuželosečku včetně os, jsou-li dány 3 body $A, B, C$ a střed $O$ (který neleží na žádné z přímek $AB$, $AC$, $BC$).

	\begin{reseni}
		Ze středové symetrie máme i body $A', B', C'$, tedy jich máme $≥$ 5. Tudíž budeme hledat jen osy.

		Buď přes najití tečen. Nebo vezmeme průměr rovnoběžný s $AB$, pak víme, že jeho sdružený průměr prochází středem úsečky $AB$. Zopakujeme stejný postup a použijeme předchozí důkaz-konstrukci.
	\end{reseni}
\end{priklad}

\begin{dusledek}
	Středové příčky středově souměrného čtyřúhelníku vepsaného kuželosečce jsou sdružené průměry.
\end{dusledek}

\begin{poznamka}
	Naopak úhlopříčky (středově souměrného) opsaného čtyřúhelníku jsou sdružené průměry.
\end{poznamka}

\begin{priklad}
	Kuželosečka dána třemi tečnami a středem, nalezněte kuželosečku včetně os.

	\begin{reseni}
		Využijeme předchozí poznámky.
	\end{reseni}
\end{priklad}

\begin{priklad}[Konstrukce]
	Nalezněte vrcholy kuželosečky, jsou-li dány omezené sdružené průměry.

	\begin{reseni}[Nebude ke zkoušce?]
		Rytzova konstrukce.
	\end{reseni}
\end{priklad}

\section{Sdružené elementy a osa na parabole}
\begin{definice}[Sdružená tečna s průměrem (pro parabolu)]
	Tečna v koncovém bodě průměru paraboly a tento průměr se nazývají sdružené.
\end{definice}

\begin{poznamka}
	Spojnice středů rovnoběžných tětiv je průměrem paraboly (už víme). Tj. směr těchto tětiv je rovněž směrem tečny sdružené s tímto průměrem.
\end{poznamka}

\begin{veta}
	Mějme body dotyku $U, V$ tečen z $P$ a střed $N$ úsečky $UV$ ($\implies$ $PN$ je průměr paraboly podle věty výše). Označme $M = PN \cap $parabola. Pak $M$ je střed $PN$.

	\begin{dukazin}
		Involuce se středem $P$ má osu $UV$, tedy $(PNM) = (PNMS_∞) = (PNS_∞M) = -1$ $\implies$ $M$ je střed $PN$.
	\end{dukazin}
\end{veta}

\begin{priklad}[Konstrukce]
	Parabola dána čtyřmi tečnami, sestrojte směr průměrů.

	\begin{reseni}
		Buď postupem ze stejné konstrukce někdy dávno. (Hledáme bod dotyku na páté, nevlastní, tečně.)

		Nebo (pomaleji) můžeme nalézt dva body dotyku, a spojit průsečík tečen se středem úsečky dané těmito body.

		Tečné body hledáme přes Brianchonova.
	\end{reseni}
\end{priklad}

\begin{priklad}[Konstrukce]
	Sestrojte parabolu, jsou-li dány dvě tečny, pól a polára.

	\begin{reseni}
		Jedna kuželosečka je z předchozí věty rovnoběžka s $p$ v polovině mezi $P$ a $p$.

		Jednu tečnu protneme s $p$, z průsečíku vezmeme přímku procházející $P$ a k těmto 3 přímkám nalezneme harmonickou čtveřici = další tečna.
	\end{reseni}
\end{priklad}

\begin{veta}
	Existuje jediná tečna, jejíž směr je kolmý k průměru paraboly.

	\begin{dukazin}[Konstrukcí]
		Najděte osu paraboly danou čtyřmi tečnami.

		Nalezneme směr průměru paraboly, k němu najdeme kolmý směr. K tomuto směru nalezneme tečnu.
	\end{dukazin}
\end{veta}

\begin{definice}[Vrcholová tečna, vrchol paraboly, osa paraboly]
	Tečna z předchozí věty se nazývá vrcholová tečna, její bod dotyku vrchol paraboly a její sdružený průměr osa paraboly (a směr osy se nazývá směr osy).
\end{definice}

\end{document}
