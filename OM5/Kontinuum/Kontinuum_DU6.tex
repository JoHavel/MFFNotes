\documentclass[12pt]{article}					% Začátek dokumentu
\usepackage{../../MFFStyle}					    % Import stylu



\begin{document}

\begin{priklad}[1.]
	Let $®T_R$ and ®T denote the first Piola–Kirchhoff tensor and the Cauchy stress tensor respectively. Show that
	$$ \Div_{¦X} ®T_R = (\det ®F) \Div ®T, $$
	or, in detail, that
	$$ \Div_{¦X} ®T_R(¦X, t) = (\det ®F(¦X, t)) (\Div_{¦x} ®T(¦x, t))|_{¦x=\chi(¦X,t)}, $$
	where ®F denotes the deformation gradient and $\chi(¦X, t)$ is the deformation.

	\begin{dukazin}
		Přidáme integrál a použijeme Stokesovu větu:
		$$ \int LHS = \int_{V(t_0)} \Div_{¦X} ®T_R(¦X, t) dV = \int_{\partial V(t_0)} ®T_R(¦X, t) · d¦S. $$
		Z přednášky víme, že $®T_R = \det ®F(¦X, t) ®T(¦x, t)|_{¦x = \chi(¦X, t)} F^{-T}(¦X, t)$:
		$$ \int LHS = \int_{\partial V(t_0)} \(\det ®F(¦X, t) ®T(¦x, t)|_{¦x = \chi(¦X, t)} F^{-T}(¦X, t)\) · d¦S. $$
		Také víme, že $(\det ®F(¦X, t)) F^{-T}(¦X, t) d¦S = d¦s$ a že $\det ®F(¦X, t)$ je skalár, tj. je komutativní, tedy můžeme psát:
		$$ \int LHS = \int_{\partial V(t_0)} ®T(¦x, t)|_{¦x = \chi(¦X, t)} · (\det ®F(¦X, t) ®F^{-T}(¦X, t) d¦S) = \int_{\partial V(t)} ®T(¦x, t) · d¦s. $$
		Nyní použijeme znovu Stokesovu větu a znalost z přednášky, že $dv = (\det ®F(¦X, t)) dV$:
		$$ \int LHS = \int_{V(t)} \Div_{¦x} ®T(¦x, t) dv = \int_{V(t_0)} (\Div_{¦x} ®T(¦x, t))|_{¦x = \chi(¦X, t)} (\det ®F(¦X, t)) dV = \int RHS, $$
		neboť $\det ®F(¦X, t)$ je zase skalár. Použitím lokalizačního lemmatu dostáváme chtěnou rovnost:
		$$ \Div_{¦X} ®T_R(¦X, t) = (\det ®F(¦X, t)) (\Div_{¦x} ®T(¦x, t))|_{¦x=\chi(¦X,t)}. $$
	\end{dukazin}
\end{priklad}

\end{document}
