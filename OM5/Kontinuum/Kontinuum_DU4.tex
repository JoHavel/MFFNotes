\documentclass[12pt]{article}					% Začátek dokumentu
\usepackage{../../MFFStyle}					    % Import stylu



\begin{document}

\begin{priklad}[1.]
	Show that the Leibniz integral rule (LIR)
	$$ \frac{d}{dt} \int_{\xi = a(t)}^{b(t)} f(\xi, t) d\xi = \int_{\xi = a(t)}^{b(t)} \frac{\partial f(\xi, t)}{\partial t} d\xi + f(b(t), t) \frac{db}{dt} - f(a(t), t) \frac{da}{dt} $$
	where $f$, $a$ and $b$ are some smooth scalar valued functions, is a special case of Reynolds transport theorem (RTT).
\end{priklad}

	\begin{dukaz}
		V RTT zvolíme $\forall ¦x, t: \phi(¦x, t) = 1$, $\chi: [0, 1]^3\times ®R \rightarrow ®R^3$ $\chi(X_1, X_2, X_3, t) = (a(t) + X_1 (b(t) - a(t)), X_2 f(a(t) + X_1 (b(t) - a(t)), t), X_3)$, tedy nebudeme integrovat „žádnou funkci“, jen nás zajímá změna objemu, který právě v první souřadnici odpovídá proměnné v LIR, v druhé funkční hodnotě v LIR a ve třetí souřadnici se nemění.

		Nejdříve dosadíme a pomocí Gaussovy věty a linearity integrálu převedeme RTT na
		$$ \frac{d}{dt} \int_{V(t)} 1 dv = \int_{V(t)} \frac{d 1}{dt} + \int_{\partial V(t)} 1·(¦v·¦n) ds = 0 + \int_{\partial V(t)} 1·(¦v·¦n) ds. $$
		Dále můžeme použít Lagrangeovo kritérium pro vyjádření $¦v·¦n$:
		$$ \frac{d}{dt} \int_{V(t)} dv = \int_{\partial V(t)} \frac{\frac{\partial g}{\partial t}}{|\nabla_x g|} ds, $$
		pro diferencovatelnou funkci $g$, která je na $\Int V(t)$ kladná a na $\partial V(t)$ nulová (oproti přednášce je tedy gradient opačný vůči normále, tedy jsme dostali výraz bez mínus).

		Teď bychom si chtěli zvolit správnou funkci $g$. Můžeme využít toho, že nulový činitel nám zaručuje nulový součin, tedy podmínky $x_i ≤ h$ zapíšeme jako $(h - x_i)$ a vynásobíme:
		$$ g(x_1, x_2, x_3, t) = (x_1 - a(t))·(b(t) - x_1)·x_2·(f(x_1, t) - x_2)·x_3·(1 - x_3). $$

		Teď můžeme spočítat (podle vzorců pro derivování) vyžadované $\frac{\partial g}{\partial t}$, $\nabla_{¦x}$, $g / …$ označuji $g$ bez tohoto členu (tedy v $… = 0$, kde nás tento výraz reálně zajímá, je to dodefinováno intuitivně):
		$$ \frac{\partial g}{\partial t} = -\frac{da}{dt}·\frac{g}{x_1 - a(t)} + \frac{db}{dt}·\frac{g}{b(t) - x_1} + \frac{\partial f}{\partial t}·\frac{g}{f(x_1, t) - x_2}, $$
		$$ \nabla_{¦x} g = \(\frac{g}{x_1 - a(t)}) - \frac{g}{b(t) - x_1} + \frac{\partial f(x_1, t)}{\partial x_1} · \frac{g}{f(x_1, t) - x_2}, \frac{g}{x_2} - \frac{g}{f(x_1, t) - x_2}, \frac{g}{x_3} - \frac{g}{1 - x_3}\). $$

		Teď se zase vrátíme k RTT. Vždy když $g(¦x, t) = 0$, tak musí být nulový jeden z činitelů, tedy integrál přes povrch můžeme rozložit na jednotlivé případy:
		\begin{itemize}
			\item $x_3 = 0$, pak $\frac{\partial g}{\partial t} = 0$, neboť ve všech členech je $g$ nevydělené $x_3$. Tedy
				$$ \int_{\partial V(t), x_3 = 0} \frac{\frac{\partial g}{\partial t}}{|\nabla_x g|} ds = \int 0 = 0. $$
			\item $x_3 = 1$, pak ze stejného důvodu $\int_{\partial V(t), x_3 = 1} \frac{\frac{\partial g}{\partial t}}{|\nabla_x g|} ds = 0$.
			\item $x_2 = 0$ taktéž dává $\int_{\partial V(t), x_2 = 0} \frac{\frac{\partial g}{\partial t}}{|\nabla_x g|} ds = 0$.
			\item $x_2 = f(x_1, t)$ je složitější, neboť $\frac{\partial g}{\partial t} = \frac{\partial f(x_1, t)}{\partial t} \frac{g}{f(x_1, t) - x_2}$, jelikož je to zase jediný nenulový člen. Stejně tak $\nabla¦x g = (\frac{\partial f(x_1, t)}{\partial x_1} \frac{g}{f(x_1, t) - x_2}, -\frac{g}{f(x_1, t) - x_2}, 0)$. Takže v $\frac{\frac{\partial g}{\partial t}}{|\nabla_x g|}$ můžeme zkrátit $\frac{g}{f(x_1, t) - x_2}$ a zbude nám:
				$$ \int_{\partial V(t), x_2 = f(x_1, t)} \frac{\frac{\partial f(x_1, t)}{\partial t}}{\left|\(\frac{\partial f(x_1, t)}{\partial x_1}, -1, 0\)\right|} dv = \int_{\partial V(t), x_2 = f(x_1, t)} \frac{\frac{\partial f(x_1, t)}{\partial t}}{\sqrt{\(\frac{\partial f(x_1, t)}{\partial x_1}\)^2 + 1}} dv. $$
				Což můžeme z Fubiniovy věty rozložit na nezajímavý integrál přes $z$ a křivkový integrál přes křivku $f(x_1, t)$ tedy
				$$ … = \int_0^1 \int_{a(t)}^{b(t)} \frac{\frac{\partial f(x_1, t)}{\partial t}}{\sqrt{\(\frac{\partial f(x_1, t)}{\partial x_1}\)^2 + 1}} · \sqrt{\(\frac{\partial f(x_1, t)}{\partial x_1}\)^2 + 1} dx_1 dx_3 = \int_{a(t)}^{b(t)} \frac{\partial f(x_1, t)}{\partial t} dx_1. $$
			\item $x_1 = a(t)$, potom (jediné nenulové, Fubini, …)
				$$ \frac{\frac{\partial g}{\partial t}}{\nabla_{¦x} g} = \frac{-\frac{da}{dt}·\frac{g}{x_1 - a(t)}}{\left|\(\frac{g}{x - a(t)}, 0, 0\)\right|} = -\frac{da}{dt} \implies $$
				$$ \implies \int_{\partial V(t), x_1 = a(t)} \frac{\frac{\partial g}{\partial t}}{|\nabla_x g|} ds = \int_0^1 \int_0^{f(x_1, t)} - \frac{da}{dt} dx_2 dx_3 = - \frac{da}{dt} f(a(t), t). $$
			\item $x_1 = b(t)$, potom úplně stejně jako v předchozím
				$$ \frac{\frac{\partial g}{\partial t}}{\nabla_{¦x} g} = \frac{\frac{db}{dt}·\frac{g}{b(t) - x_1}}{\left|\(-\frac{g}{b(t) - x_1}, 0, 0\)\right|} = -\frac{da}{dt} \implies $$
				$$ \implies \int_{\partial V(t), x_1 = b(t)} \frac{\frac{\partial g}{\partial t}}{|\nabla_x g|} ds = \int_0^1 \int_0^{f(x_1, t)} \frac{db}{dt} dx_2 dx_3 = \frac{db}{dt} f(b(t), t). $$
		\end{itemize}

		Tedy máme
		$$ \frac{d}{dt}\int_{V(t)} dv = \int_{a(t)}^{b(t)} \frac{\partial f(x_1, t)}{\partial t} dx_1 + \frac{db}{dt} f(b(t), t) - \frac{da}{dt} f(a(t), t), $$
		což už je skoro to, co chceme, stačí jen rozložit integrál na levé straně pomocí Fubiniovy věty:
		$$ \frac{d}{dt}\int_{V(t)} dv = \frac{d}{dt}\int_0^1 \int_{a(t)}^{b(t)} \int_0^{f(x_1, t)} dx_2 dx_1 dx_3 = \frac{d}{dt}\int_{a(t)}^{b(t)} f(x_1, t) dx_1. $$
	\end{dukaz}

\begin{priklad}
	Consider the deformation $\chi$ given by the following formulae
		\begin{align*}
			r &= f(R),\\
			\phi &= \Phi,\\
			z &= Z.
		\end{align*}
	Show that the deformation gradient ®F is given by the formula
	$$ ®F = \frac{df}{dR} ¦e_{\hat{r}} \otimes ¦E_{\hat{R}} + \frac{f}{R} ¦e_{\hat{\phi}} \otimes ¦E_{\hat{\phi}} + ¦e_{\hat{z}} \otimes ¦E_{\hat{Z}}. $$
\end{priklad}

	\begin{dukaz}

		Z celého znění zadání máme vzorec pro ®F v kartézských souřadnicích. Následně budeme počítat jednotlivé členy:
		{\tiny
			$$ ®F =
			\frac{\partial \chi^{\hat{x}}}{\partial X} ¦e_{\hat{x}}\otimes ¦E^{\hat{X}} +
			\frac{\partial \chi^{\hat{x}}}{\partial Y} ¦e_{\hat{x}}\otimes ¦E^{\hat{Y}} +
			\frac{\partial \chi^{\hat{x}}}{\partial Z} ¦e_{\hat{x}}\otimes ¦E^{\hat{Z}} +
			\frac{\partial \chi^{\hat{y}}}{\partial X} ¦e_{\hat{y}}\otimes ¦E^{\hat{X}} +
			\frac{\partial \chi^{\hat{y}}}{\partial Y} ¦e_{\hat{y}}\otimes ¦E^{\hat{Y}} +
			\frac{\partial \chi^{\hat{y}}}{\partial Z} ¦e_{\hat{y}}\otimes ¦E^{\hat{Z}} +
			\frac{\partial \chi^{\hat{z}}}{\partial X} ¦e_{\hat{z}}\otimes ¦E^{\hat{X}} +
			\frac{\partial \chi^{\hat{z}}}{\partial Y} ¦e_{\hat{z}}\otimes ¦E^{\hat{Y}} +
			\frac{\partial \chi^{\hat{z}}}{\partial Z} ¦e_{\hat{z}}\otimes ¦E^{\hat{Z}}
			$$

			$$ \frac{\partial \chi^{\hat{x}}}{\partial X} = \frac{\partial}{\partial X} (r \cos \phi) = \frac{\partial r}{\partial X} \cos \phi - r \sin\phi \frac{\partial \phi}{\partial X} = \(\frac{\partial r}{\partial R}\frac{\partial R}{\partial X} + \frac{\partial r}{\partial \Phi}\frac{\partial \Phi}{\partial X} + \frac{\partial r}{\partial Z}\frac{\partial Z}{\partial X}\)\cos \phi - r \sin \phi \(\frac{\partial \phi}{\partial R}\frac{\partial R}{\partial X} + \frac{\partial \phi}{\partial \Phi} \frac{\partial \Phi}{\partial X} + \frac{\partial \phi}{\partial Z}\frac{\partial Z}{\partial X}\) = $$
			$$ = \(\frac{df}{dR} \frac{\partial R}{\partial X} + 0 + 0\)\cos \phi - r \sin \phi\(0 + \frac{\partial \Phi}{\partial X} + 0\) $$
			$$ \frac{\partial \chi^{\hat{x}}}{\partial Y} = \frac{\partial}{\partial Y} (r \cos \phi) = \frac{\partial r}{\partial Y} \cos \phi - r \sin\phi \frac{\partial \phi}{\partial Y} = \(\frac{\partial r}{\partial R}\frac{\partial R}{\partial Y} + \frac{\partial r}{\partial \Phi}\frac{\partial \Phi}{\partial Y} + \frac{\partial r}{\partial Z}\frac{\partial Z}{\partial Y}\)\cos \phi - r \sin \phi \(\frac{\partial \phi}{\partial R}\frac{\partial R}{\partial Y} + \frac{\partial \phi}{\partial \Phi} \frac{\partial \Phi}{\partial Y} + \frac{\partial \phi}{\partial Z}\frac{\partial Z}{\partial Y}\) = $$
			$$ = \(\frac{df}{dR} \frac{\partial R}{\partial Y} + 0 + 0\)\cos \phi - r \sin \phi\(0 + \frac{\partial \Phi}{\partial Y} + 0\) $$
			$$ \frac{\partial \chi^{\hat{x}}}{\partial Z} = \frac{\partial}{\partial Z} (r \cos \phi) = \frac{\partial r}{\partial Z} \cos \phi - r \sin\phi \frac{\partial \phi}{\partial Z} = \(\frac{\partial r}{\partial R}\frac{\partial R}{\partial Z} + \frac{\partial r}{\partial \Phi}\frac{\partial \Phi}{\partial Z} + \frac{\partial r}{\partial Z}\frac{\partial Z}{\partial Z}\)\cos \phi - r \sin \phi \(\frac{\partial \phi}{\partial R}\frac{\partial R}{\partial Z} + \frac{\partial \phi}{\partial \Phi} \frac{\partial \Phi}{\partial Z} + \frac{\partial \phi}{\partial Z}\frac{\partial Z}{\partial Z}\) = $$
			$$ = \(\frac{df}{dR} \frac{\partial R}{\partial Z} + 0 + 0\)\cos \phi - r \sin \phi\(0 + \frac{\partial \Phi}{\partial Z} + 0\) = 0 $$
			$$ \frac{\partial \chi^{\hat{y}}}{\partial X} = \frac{\partial}{\partial X} (r \sin \phi) = \frac{\partial r}{\partial X} \sin \phi + r \cos\phi \frac{\partial \phi}{\partial X} = \(\frac{\partial r}{\partial R}\frac{\partial R}{\partial X} + \frac{\partial r}{\partial \Phi}\frac{\partial \Phi}{\partial X} + \frac{\partial r}{\partial Z}\frac{\partial Z}{\partial X}\)\sin \phi + r \cos \phi \(\frac{\partial \phi}{\partial R}\frac{\partial R}{\partial X} + \frac{\partial \phi}{\partial \Phi} \frac{\partial \Phi}{\partial X} + \frac{\partial \phi}{\partial Z}\frac{\partial Z}{\partial X}\) = $$
			$$ = \(\frac{df}{dR} \frac{\partial R}{\partial X} + 0 + 0\)\sin \phi + r \cos \phi\(0 + \frac{\partial \Phi}{\partial X} + 0\) $$
			$$ \frac{\partial \chi^{\hat{y}}}{\partial Y} = \frac{\partial}{\partial Y} (r \sin \phi) = \frac{\partial r}{\partial Y} \sin \phi + r \cos\phi \frac{\partial \phi}{\partial Y} = \(\frac{\partial r}{\partial R}\frac{\partial R}{\partial Y} + \frac{\partial r}{\partial \Phi}\frac{\partial \Phi}{\partial Y} + \frac{\partial r}{\partial Z}\frac{\partial Z}{\partial Y}\)\sin \phi + r \cos \phi \(\frac{\partial \phi}{\partial R}\frac{\partial R}{\partial Y} + \frac{\partial \phi}{\partial \Phi} \frac{\partial \Phi}{\partial Y} + \frac{\partial \phi}{\partial Z}\frac{\partial Z}{\partial Y}\) = $$
			$$ = \(\frac{df}{dR} \frac{\partial R}{\partial Y} + 0 + 0\)\sin \phi + r \cos \phi\(0 + \frac{\partial \Phi}{\partial Y} + 0\) $$
			$$ \frac{\partial \chi^{\hat{y}}}{\partial Z} = \frac{\partial}{\partial Z} (r \sin \phi) = \frac{\partial r}{\partial Z} \sin \phi + r \cos\phi \frac{\partial \phi}{\partial Z} = \(\frac{\partial r}{\partial R}\frac{\partial R}{\partial Z} + \frac{\partial r}{\partial \Phi}\frac{\partial \Phi}{\partial Z} + \frac{\partial r}{\partial Z}\frac{\partial Z}{\partial Z}\)\sin \phi + r \cos \phi \(\frac{\partial \phi}{\partial R}\frac{\partial R}{\partial Z} + \frac{\partial \phi}{\partial \Phi} \frac{\partial \Phi}{\partial Z} + \frac{\partial \phi}{\partial Z}\frac{\partial Z}{\partial Z}\) = $$
			$$ = \(\frac{df}{dR} \frac{\partial R}{\partial Z} + 0 + 0\)\sin \phi + r \cos \phi\(0 + \frac{\partial \Phi}{\partial Z} + 0\) = 0 $$
			$$ \frac{\partial \chi^{\hat{z}}}{\partial X} = \frac{\partial z}{\partial X} = \frac{\partial z}{\partial Z}\frac{\partial Z}{\partial X} = 0, \quad \frac{\partial \chi^{\hat{z}}}{\partial Y} = \frac{\partial z}{\partial Y} = \frac{\partial z}{\partial Z}\frac{\partial Z}{\partial Y} = 0, \quad \frac{\partial \chi^{\hat{z}}}{\partial Z} = \frac{\partial z}{\partial Z} = 1 $$

			\hrule

			$$ \implies ®F =
			\frac{\partial \chi^{\hat{x}}}{\partial X} ¦e_{\hat{x}}\otimes ¦E^{\hat{X}} +
			\frac{\partial \chi^{\hat{x}}}{\partial Y} ¦e_{\hat{x}}\otimes ¦E^{\hat{Y}} +
			\frac{\partial \chi^{\hat{y}}}{\partial X} ¦e_{\hat{y}}\otimes ¦E^{\hat{X}} +
			\frac{\partial \chi^{\hat{y}}}{\partial Y} ¦e_{\hat{y}}\otimes ¦E^{\hat{Y}} +
			¦e_{\hat{z}}\otimes ¦E^{\hat{Z}}
			=
			\(\frac{\partial \chi^{\hat{x}}}{\partial X} ¦e_{\hat{x}} +
			\frac{\partial \chi^{\hat{y}}}{\partial X} ¦e_{\hat{y}}\) \otimes ¦E^{\hat{X}} +
			\(\frac{\partial \chi^{\hat{x}}}{\partial Y} ¦e_{\hat{x}} +
			\frac{\partial \chi^{\hat{y}}}{\partial Y} ¦e_{\hat{y}}\) \otimes ¦E^{\hat{Y}} +
			¦e_{\hat{z}}\otimes ¦E^{\hat{Z}}
			$$

			$$ ¦e_{\hat{x}} = (\cos \phi) ¦e_{\hat{r}} - (\sin \phi) ¦e_{\hat{\phi}}, \quad ¦e_{\hat{y}} = (\sin \phi) ¦e_{\hat{r}} + (\cos \phi) ¦e_{\hat{\phi}} $$

			$$ \frac{\partial \chi^{\hat{x}}}{\partial X} ¦e_{\hat{x}} + \frac{\partial \chi^{\hat{y}}}{\partial X} ¦e_{\hat{y}} =
			\(\frac{df}{dR} \frac{\partial R}{\partial X} \cos \phi - \frac{\partial \Phi}{\partial X} r \sin \phi\)\((\cos \phi) ¦e_{\hat{r}} - (\sin \phi) ¦e_{\hat{\phi}}\) +
			\(\frac{df}{dR} \frac{\partial R}{\partial X} \sin \phi + \frac{\partial \Phi}{\partial X} r \cos \phi\)\((\sin \phi) ¦e_{\hat{r}} + (\cos \phi) ¦e_{\hat{\phi}}\) = $$
			$$ = \frac{df}{dR} \frac{\partial R}{\partial X} ¦e_{\hat{r}} + 0 + 0 + \frac{\partial \Phi}{\partial X} r ¦e_{\hat{\phi}} $$
			$$ \frac{\partial \chi^{\hat{x}}}{\partial Y} ¦e_{\hat{x}} + \frac{\partial \chi^{\hat{y}}}{\partial Y} ¦e_{\hat{y}} =
			\(\frac{df}{dR} \frac{\partial R}{\partial Y} \cos \phi - \frac{\partial \Phi}{\partial Y} r \sin \phi\)\((\cos \phi) ¦e_{\hat{r}} - (\sin \phi) ¦e_{\hat{\phi}}\) +
			\(\frac{df}{dR} \frac{\partial R}{\partial Y} \sin \phi + \frac{\partial \Phi}{\partial Y} r \cos \phi\)\((\sin \phi) ¦e_{\hat{r}} + (\cos \phi) ¦e_{\hat{\phi}}\) = $$
			$$ = \frac{df}{dR} \frac{\partial R}{\partial Y} ¦e_{\hat{r}} + 0 + 0 + \frac{\partial \Phi}{\partial Y} r ¦e_{\hat{\phi}} $$

			\hrule

			$$ \implies ®F =
			\(\frac{df}{dR} \frac{\partial R}{\partial X} ¦e_{\hat{r}} + \frac{\partial \Phi}{\partial X} r ¦e_{\hat{\phi}}\)\otimes ¦E^{\hat{X}} + 
			\(\frac{df}{dR} \frac{\partial R}{\partial Y} ¦e_{\hat{r}} + \frac{\partial \Phi}{\partial Y} r ¦e_{\hat{\phi}}\)\otimes ¦E^{\hat{Y}} + 
			¦e_{\hat{z}}\otimes ¦E^{\hat{Z}} =
			\frac{df}{dR} ¦e_{\hat{r}} \otimes \(\frac{\partial R}{\partial X}¦E^{\hat{X}} + \frac{\partial R}{\partial Y}¦E^{\hat{Y}}\) +
			r ¦e_{\hat{\phi}} \otimes \(\frac{\partial \Phi}{\partial X}¦E^{\hat{X}} + \frac{\partial \Phi}{\partial Y}¦E^{\hat{Y}}\) +
			¦e_{\hat{z}}\otimes ¦E^{\hat{Z}} $$
			
			$$ ¦E^{\hat{X}} = (\cos \Phi) ¦E^{\hat{R}} - (\sin \Phi) ¦E^{\hat{\Phi}}, \quad ¦E^{\hat{Y}} = (\sin \Phi) ¦E^{\hat{R}} + (\cos \Phi) ¦E^{\hat{\Phi}}, \quad R^2 = X^2 + Y^2, \quad R = \sqrt{X^2 + Y^2}, \quad \sin \Phi = \frac{Y}{R}, \quad \cos \Phi = \frac{X}{R} $$

			$$ \frac{\partial R}{\partial X}¦E^{\hat{X}} + \frac{\partial R}{\partial Y}¦E^{\hat{Y}} = \frac{2X¦E^{\hat{X}} + 2Y¦E^{\hat{Y}}}{2\sqrt{X^2 + Y^2}} = \frac{X¦E^{\hat{X}} + Y¦E^{\hat{Y}}}{R} =
			\frac{X(\cos\Phi)¦E^{\hat{R}} - X(\sin\Phi)¦E^{\hat{\Phi}} + Y(\sin\Phi)¦E^{\hat{R}} + Y(\cos\Phi)¦E^{\hat{\Phi}}}{R} = $$
			$$ = \frac{\frac{X^2}{R}¦E^{\hat{R}} - \frac{XY}{R}¦E^{\hat{\Phi}} + \frac{Y^2}{R}¦E^{\hat{R}} + \frac{XY}{R}¦E^{\hat{\Phi}}}{R} =
			\frac{X^2 + Y^2}{R^2}¦E^{\hat{R}} = ¦E^{\hat{R}} $$
			$$ \frac{\partial \Phi}{\partial X}¦E^{\hat{X}} + \frac{\partial \Phi}{\partial Y}¦E^{\hat{Y}} = \frac{-Y¦E^{\hat{X}} + X¦E^{\hat{Y}}}{X^2 + Y^2} = \frac{-Y¦E^{\hat{X}} + X¦E^{\hat{Y}}}{R^2} =
			\frac{-Y(\cos\Phi)¦E^{\hat{R}} + Y(\sin\Phi)¦E^{\hat{\Phi}} + X(\sin\Phi)¦E^{\hat{R}} + X(\cos\Phi)¦E^{\hat{\Phi}}}{R^2} = $$
			$$ = \frac{-\frac{XY}{R}¦E^{\hat{R}} + \frac{Y^2}{R}¦E^{\hat{\Phi}} + \frac{XY}{R}¦E^{\hat{R}} + \frac{X^2}{R}¦E^{\hat{\Phi}}}{R^2} =
			\frac{Y^2 + X^2}{R^3}¦E^{\hat{\Phi}} = \frac{1}{R}¦E^{\hat{\phi}} $$

			\hrule
	
			$$ \implies ®F = \frac{df}{dR} ¦e_{\hat{r}} \otimes ¦E_{\hat{R}} + \frac{r}{R} ¦e_{\hat{\phi}} \otimes ¦E_{\hat{\phi}} + ¦e_{\hat{z}} \otimes ¦E_{\hat{Z}}. $$
		}
	\end{dukaz}

\end{document}
