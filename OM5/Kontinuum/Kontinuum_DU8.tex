\documentclass[12pt]{article}					% Začátek dokumentu
\usepackage{../../MFFStyle}					    % Import stylu



\begin{document}

Consider a linearised homogeneous isotropic elastic solid, that is a continuous medium where the stress tensor is given by the formula $\btau = \lambda (\tr \bepsilon) ®I + 2\mu\bepsilon$, where $\bepsilon := \frac{1}{2} \(\nabla U + (\nabla U)^T\)$ denotes the linearised strain tensor.

\begin{priklad}[1.]
	Show that dynamic governing equation $\rho_R \frac{\partial^2 ¦U}{\partial t^2} = \Div \btau$ in this case reduces to
	$$ \rho_R \frac{\partial^2 ¦U}{\partial t^2} = (\lambda + \mu) \nabla (\Div ¦U) + \mu \Delta ¦U. $$

	\begin{dukazin}
		$$ (\Div \btau)_j = (\Div (\lambda (\tr \bepsilon) ®I + 2 \mu \bepsilon))_j = $$
		$$ = \lambda \partial_k\(®I_{jk} \sum_i \(\frac{1}{2} \partial_i U_i + \frac{1}{2} \partial_i U_i\)\) + 2\mu\(\frac{1}{2} \sum_i \partial_i (\partial_i U_j) + \frac{1}{2} \sum_i \partial_i(\partial_j U_i)\) = $$
		$$ = \lambda \partial_j\(\sum_i (\partial_i U_i)\) + \mu\(\sum_i \partial_i \partial_i U_j + \partial_j\(\sum_i \partial_i U_i\)\) = $$
		$$ = \lambda (\nabla (\Div ¦U))_j + \mu \Delta U_j + \mu (\nabla (\Div ¦U))_j = ((\lambda + \mu)\nabla (\Div ¦U) + \mu \Delta ¦U)_j $$
	\end{dukazin}
\end{priklad}

\newpage

\begin{priklad}[2.]
	Assume that solution to previous equation has the form of a travelling plane wave, that is $¦U = ¦A \sin(¦K·¦X - \omega t)$. Substitute and show that the result can be rewritten in the form
	$$ \rho_R c^2 ¦A = \[\mu \(®I - \frac{¦K}{K} \otimes \frac{¦K}{K}\) + (\lambda + 2\mu) \frac{¦K}{K} \otimes \frac{¦K}{K}\]¦A. $$

	\begin{dukazin}
		Levá strana:
		$$ \rho_R \frac{\partial^2 ¦U}{\partial t^2} = \rho_R ¦A \frac{\partial \sin(¦K·¦X - \omega t)}{\partial t^2} = \rho_R ¦A \frac{\partial (-\omega \cos(K·X - \omega t))}{\partial t} = $$
		$$ = \rho_R ¦A - \omega^2 \sin(¦K·¦X - \omega t) = \rho_R c^2 ¦A (-K^2\sin(…)). $$
		
		První člen pravé strany:
		$$ \((\lambda + \mu) \nabla (\Div ¦U)\)_j = (\lambda + \mu) \partial_j\(\sum_i \partial_i A_i \sin(¦K·¦X - \omega t)\) = $$
		$$ = (\lambda + \mu) \sum_i \partial_j A_i K_i \cos(¦K·¦X - \omega t) = -(\lambda + \mu) \sum_i A_i K_i K_j \sin(¦K·¦X - \omega t) = $$
		$$ = -(\lambda + \mu) (¦K \otimes ¦K)¦A \sin(¦K·¦X - \omega t) = (\lambda + \mu) \(\frac{¦K}{K} \otimes \frac{¦K}{K}\)¦A (-K^2 \sin(…)). $$

		Druhý člen pravé strany:
		$$ \mu \Delta ¦U = \mu \Delta \(¦A \sin(¦K·¦X - \omega t)\) = \mu ¦A \Delta \(\sin(¦K·¦X - \omega t)\) = \mu ¦A \sum_i K_i^2 (-\sin(¦K·¦X - \omega t)) = $$
		$$ = \mu ¦A (- K^2·\sin(…)). $$

		Předpokládáme, že $K ≠ 0$, tedy můžeme dělit rovnici $K^2$ a navíc $\sin(…)$ bude nulový pouze na množině s prázdným vnitřkem, takže rovnici můžeme dodefinovat ze spojitosti a můžeme dělit i $-\sin(…)$:
		$$ \rho_R c^2 ¦A = (\lambda + \mu) \(\frac{¦K}{K} \otimes \frac{¦K}{K}\)¦A + \mu ¦A = \[\mu \(®I - \frac{¦K}{K} \otimes \frac{¦K}{K}\) + (\lambda + 2\mu) \frac{¦K}{K} \otimes \frac{¦K}{K}\]¦A $$
	\end{dukazin}

	This is an eigenvalue problem of the form $c^2 ¦A = ®A ¦A$, where
	$$ ®A := \frac{\mu}{\rho_R}\(®I - \frac{¦K}{K} \otimes \frac{¦K}{K}\) + \frac{\lambda + 2\mu}{\rho_R} \frac{¦K}{K} \otimes \frac{¦K}{K}. $$
\end{priklad}

\newpage

\begin{priklad}[3.]
	Show that the displacement in the form $¦U := ¦A \sin (¦K · ¦X - \omega t)$ is a solution in $®R^3$ provided that either ¦A is parallel to ¦K and the speed of propagation is $c_{||} = \sqrt{\frac{\lambda + 2\mu}{\rho_R}}$ or ¦A is perpendicular to ¦K and the speed of propagation is $c_\perp = \sqrt{\frac{\mu}{\rho_R}}$.

	\begin{dukazin}
		$\(\frac{¦K}{K} \otimes \frac{¦K}{K}\)¦A = \frac{¦K}{K} \(\frac{¦K}{K}·¦A\)$ je kolmá projekce vektoru $¦A$ na $¦K$ podle toho, co víme o skalárním součinu (skalární součin s jednotkovým vektorem dává velikost projekce, vynásobením tím samým jednotkovým vektorem získáme pak celou projekci).

		Tím pádem $\(\frac{¦K}{K} \otimes \frac{¦K}{K}\)¦A$ je tedy projekce $¦A$ na $¦K$ a $(®I - …)¦A$ je tedy projekce ¦A na ortogonální doplněk ¦K. ¦A tedy násobení maticí ®A rozloží na vektor ve směru ¦K, ten vynásobí $\frac{\lambda + 2\mu}{\rho_R}$, a na vektor v kolmém směru, a ten vynásobíme $\frac{\mu}{\rho_R}$, a pak je zase sečteme.

		Na levé straně násobíme skalárem $c^2$, na což se můžeme podívat jako na rozložení ¦A na složku ve směru ¦K a na složku v ortogonálním směru a následně vynásobíme obě části $c^2$ a sečteme. Tedy

		$$ c^2\(®I - \frac{¦K}{K} \otimes \frac{¦K}{K}\)¦A + c^2\(\frac{¦K}{K} \otimes \frac{¦K}{K}\)¦A = \frac{\mu}{\rho_R} \(®I - \frac{¦K}{K} \otimes \frac{¦K}{K}\)¦A + \frac{\lambda + 2\mu}{\rho_R}\(\frac{¦K}{K} \otimes \frac{¦K}{K}\)¦A. $$

		Jelikož jsou podprostory ortogonální, musí rovnost platit v každém z nich, tedy:
		$$ c^2\(®I - \frac{¦K}{K} \otimes \frac{¦K}{K}\)¦A = \frac{\mu}{\rho_R} \(®I - \frac{¦K}{K} \otimes \frac{¦K}{K}\)¦A, $$
		$$ c^2\(\frac{¦K}{K} \otimes \frac{¦K}{K}\)¦A = \frac{\lambda + 2\mu}{\rho_R}\(\frac{¦K}{K} \otimes \frac{¦K}{K}\)¦A. $$

		Tudíž se vždy buď rovnají skaláry, nebo jsou nulové příslušné projekce. Obě projekce být nulové nemohou (součet dává původní vektor a předpokládáme, že $¦A ≠ ¦o$). Kdyby obě byly nenulové, tak $\mu = c^2 = \lambda + 2\mu$, tedy $\mu + \lambda = 0 \implies \gamma = +∞ \not≤ \frac{1}{2}$. Tedy zbývá možnost, že je právě jedna projekce nenulová.

		Když ¦A je paralelní na ¦K (projekce na ¦K je nenulová a druhá je nulová), pak $c^2 = \frac{\lambda + 2\mu}{\rho_R}$, tj. $c = \sqrt{\frac{\lambda + 2\mu}{\rho_R}}$, když ¦A je na ¦K kolmé, pak $c^2 = \frac{\mu}{\rho_R}$, tj. $c = \sqrt{\frac{\mu}{\rho_R}}$.
	\end{dukazin}
\end{priklad}

\end{document}
