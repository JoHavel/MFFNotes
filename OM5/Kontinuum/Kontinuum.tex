\documentclass[12pt]{article}					% Začátek dokumentu
\usepackage{../../MFFStyle}					    % Import stylu



\begin{document}

% 05. 10. 2022
\begin{poznamka}[Note from me -- autor of notes]
	Bad English in this text is my fault, not lecturer's one.
\end{poznamka}

\section*{Úvod}
\begin{poznamka}
	3 part exam: theorem -> proof; scientific paper -> understand + explain; therms + concepts -> explain

	credits: homework (time demanding)

	Microsoft teams
\end{poznamka}

\subsection{Matrix analysis / linear algebra}
\begin{poznamka}
	Scalar product: $¦u, ¦v \in ®R^3$: $¦u·¦v$, cross product: $¦u \times ¦v$, and more: $¦u = u^i ¦e_i$ $¦u·¦v = \delta_{ij}(u^i v^j)$, $(¦u \times ¦v)_i = \epsilon_{ijk}u_jv_k$ (where $\epsilon_{ijk}$, Levi-Civita symbol, does expecting thing).
\end{poznamka}

\begin{definice}[Tensor product]
	$$ ¦u \otimes ¦v \qquad (¦u \otimes ¦v)¦w := ¦u(¦v·¦w) $$
\end{definice}

\begin{tvrzeni}[Identities for Levi-Civita symbol]
	$$ \epsilon_{ijk}\epsilon_{lmn} = \det \begin{pmatrix} \delta_{il} & \delta_{im} & \delta_{in} \\ \delta_{jl} & \delta_{jm} & \delta_{jn} \\ \delta_{kl} & \delta_{km} & \delta_{kn} \end{pmatrix} $$
	$$ \epsilon_{ijk}·\delta_{lm} = \epsilon_{jkm}·\delta_{il} + \epsilon_{klm}·\delta_{jl} + \epsilon_{ijm}·\delta_{kl} $$
	$$ \epsilon_{ijk}\epsilon_{imn} = \delta_{jm} \delta_{kn} - \delta_{jn} \delta_{km} $$
	$$ \epsilon_{ijm}\epsilon_{ijn} = 2\delta_{mn} $$
\end{tvrzeni}

\begin{definice}[Transpose matrix]
	$®A \in ®R^{3 \times 3}$, $®A^T$ is defined as $\forall ¦u, ¦v \in ®R^3: ®A^T ¦u · ¦v := ¦u·®A ¦v$.
\end{definice}

\begin{definice}[Trace of matrix]
	$  ®A \in ®R^{3 \times 3}$, $\tr ®A$ is defined as $\tr (¦u \times ¦v) = ¦u·¦v$.
\end{definice}

\begin{poznamka}
	Matrix, tensor and linear operator is the same.

	$$ ®A = A_{ij} ¦e_i \otimes ¦e_j, \qquad ®A ¦v = (A_{ij} ¦e_i \otimes ¦e_j)(v_m¦e_m) = A_{ij} v_m ¦e_i(¦e_j · ¦e_m) = (A_{ij}v_j) ¦e_i. $$
\end{poznamka}

\begin{definice}[Axial vector]
	$®A \in ®R^{3 \times 3}$, ®A is stew-symetric ($-®A = ®A^T$). Then we can prove that $\forall ¦w \in ®R^3: ®A ¦w = ¦v_{®A} \times ¦w$. We call $¦v$ the axial vector.

	\begin{poznamkain}
		$¦v_{®A} = (A_{23}, A_{13}, A_{12})^T$.
	\end{poznamkain}
\end{definice}

\begin{tvrzeni}
	$®A ¦v_{®A} = ¦o$ and $(¦u \otimes ¦v)^T = (¦v \otimes ¦u)$.
\end{tvrzeni}

\begin{definice}[Determinant in 3D]
	$\det ®A := \frac{®A ¦u · (®A ¦v \times ®A ¦w)}{¦u·(¦v \times ¦w)} $ for three arbitrary vectors $¦u, ¦v, ¦w \in ®R^3$.
\end{definice}

\begin{poznamka}[Nanson formula]
	$$ ¦w · (¦u \times ¦v) = (\det ®A)^{-1} ®A¦w · (®A¦u \times ®A ¦v) = ¦w · (\det ®A)^{-1} ®A^T (®A ¦u \times ®A ¦v) \implies $$
	$$ \implies ¦u \times ¦v = (\det A)^{-1} ®A^T (®A¦u \times ®A ¦v) $$
	$$ ®A ¦u \times ®A ¦v = (\det ®A) ®A^{-T} (¦u \times ¦v) $$
\end{poznamka}

\begin{definice}[Cofactor]
	$$ \cof ®A := (\det ®A)®A^{-T}. $$

	(Change of surface area under linear mapping ®A.)
\end{definice}

\begin{definice}[Eigenvalues, eigenvectors]
	$®A ¦v = \lambda ¦v$.

	Characteristic polynomial: $\det(®A - \mu ®I) = - \mu^3 + c_1\mu^2 - c_2 \mu + c_3$.
\end{definice}

\begin{veta}[Cayley-Hamilton]
	$$ -®A^3 + c_1®A^2 - c_2 ®A + c_3®I = ®O $$
\end{veta}

\begin{tvrzeni}
	$$ c_3 = \lambda_1·\lambda_2·\lambda_3 = \det ®A $$
	$$ c_2 = \lambda_1 \lambda_2 + \lambda_1 \lambda_3 + \lambda_2 \lambda_3 = \tr \cof ®A = \frac{1}{2} ((\tr ®A)^2 - \tr(®A^2)) $$
	$$ c_1 = \lambda_1 + \lambda_2 + \lambda_3 = \tr ®A $$

	\begin{dukazin}
		With definition of characteristic polynomial, Cayley-Hamilton and Schur decomposition. Schur decomposition: $®A \in ®R^{3 \times 3}$. There exists an invertible matrix ®U and upper triangular matrix ®T such that
		$$ ®A = ®U^{-1}®T®U, \qquad ®T = \begin{pmatrix} \lambda_1 & T_{12} & T_{13} \\ 0 & \lambda_2 & T_{23} \\ 0 & 0 & \lambda_3 \end{pmatrix}. $$
	\end{dukazin}
\end{tvrzeni}

\begin{tvrzeni}[Useful identity from CH]
	$$ ®A^{-1} = \frac{1}{c_3}®A^2 - \frac{c_1}{c_3} ®A + \frac{c_2}{c_3}®I = \frac{1}{\det ®A}®A^2 - \frac{\tr ®A}{\det ®A} ®A + \frac{\tr \cof ®A}{\det ®A} ®I $$
\end{tvrzeni}

\begin{poznamka}[Functions of matrices]
	$\exp ®A$, $\ln ®A$, $\sin ®A$, …

	There are several ways of define it: Analytics calculus = Taylor series, Borel calculus: $®A = \sum \lambda_i ¦v_i \otimes ¦v_i$ () $\implies$ $f(®A) := \sum f(\lambda_i) ¦v_i \otimes ¦v_i$, Holomorphic calculus ($f(z) = \frac{1}{2 \pi i} \int_{\gamma} \frac{f(\zeta)}{(\zeta - z)} d\zeta$) $f(®A) = \frac{1}{2 \pi i} \int_{\gamma} f(\zeta)(\zeta ®I - ®A)^{-1} d\zeta$ (where curve $\zeta$ envelops eigenvalues of $®A$)
\end{poznamka}

\begin{tvrzeni}[Useful identities for functions]
	$$ \det(\exp ®A) = \exp(\tr ®A) $$
	$$ \exp ®A = \lim_{n \rightarrow ∞}\(®I + \frac{®A}{n}\)^n $$
\end{tvrzeni}

% 12. 10. 2022

\begin{definice}[Invariants of matrix]
	$$ \lambda_1 + \lambda_2 + \lambda_3 = \tr ®A = I_1; $$
	$$ \lambda_1 \lambda_2 + \lambda_1 \lambda_3 + \lambda_2 \lambda_3 = \tr \cof ®A = \frac{1}{2} ((\tr ®A)^2 - \tr(®A^2)) = I_2; $$
	$$ \lambda_1·\lambda_2·\lambda_3 = \det ®A = I_3 $$
\end{definice}

\subsection{Representation theorems for isotropic functions}
\begin{definice}[Isotropic function]
	$\phi: ®R^{3 \times 3} \rightarrow ®R$ is isotropic $≡$ $\phi(®Q ®A ®Q^T) = \phi(®A)$ for all proper orthogonal matrices ($®Q®Q^T = ®I$, $\det ®Q > 0$).

	$f: ®R^{3 \times 3} \rightarrow ®R^{3 \times 3}$ is isotropic $≡$ $f(®Q ®A ®Q^T) = ®Q f(®A) ®Q^T$ for all proper orthogonal matrices.
\end{definice}

\begin{veta}
	A scalar function $\phi: ®A \in ®R^{3 \times 3} \rightarrow ®R$ of symmetric matrices is isotropic if and only if it can be rewritten as a function of invariants of ®A.
\end{veta}

\begin{veta}
	A matrix valued function $f: ®A \in ®R^{3 \times 3} \rightarrow ®R^{3 \times 3}$ (from symmetric matrices to symmetric matrices) is isotropic if and only if it can be rewritten as
	$$ f(®A) = \alpha_0 ®I + \alpha_1 ®A + \alpha_2 ®A^2, $$
	where $\{\alpha_i\}_{i=1}^3$ are scalar function of the invariants.
\end{veta}

\begin{dusledek}
	$®A \mapsto ®A^{-1}$ is isotropic function.
\end{dusledek}

\begin{poznamka}[Notation]
	$$ ®A, ®B \in ®R^{3 \times 3}, \qquad ®A:®B := \tr(®A ®B^T), \qquad ||®A|| := (\tr (®A ®A^T))^{1 / 2} $$
\end{poznamka}

\subsection{Calculus}
\begin{definice}[Gateaux derivative]
	$$ Df(x)[y] = \(\frac{d}{d\tau} f(x + \tau y)\)|_{\tau=0}. $$
\end{definice}

\begin{definice}[Fréchet derivative]
	$$ \lim_{||y|| \rightarrow 0} \frac{||f(x + y) - f(x) - D f(x)[y]||}{||y||} = 0. $$
\end{definice}

\begin{poznamka}
	$Df(®A)[®B] \sim \frac{\partial f}{\partial ®A}(®A)[®B] \sim \frac{\partial f}{\partial ®A}(®A) : ®B$.
\end{poznamka}

\begin{priklad}
	$$ D(I_2(®A))[B] = D(-\frac{1}{2} \tr ®A^2 + \frac{1}{2} (\tr ®A)^2)[B] = \frac{d}{d\tau} (-\frac{1}{2} \tr (®A + \tau ®B)^2 + \frac{1}{2} (\tr (®A + \tau®B))^2)|_{\tau = 0} = $$
	$$ = - \tr(®A®B) + (\tr ®A)(\tr ®B) = (\tr ®A)®I : ®B - ®A^T:®B = ((\tr ®A)®I - ®A^T):®B. $$

	$$ D(\det ®A)[®B] = \frac{d}{d\tau}(\det (®A + \tau ®B))|_{\tau = 0} = (\det ®A) \frac{d}{d\tau}(\det(®I + \tau ®A^{-1}®B))_{\tau=0} = $$
	$$ = \det ®A \frac{d}{d\tau}(1 + \tau \tr (®A^{-1} ®B) + …)|_{\tau=0} = (\det ®A) \tr(®A^{-1}®B) = (\det ®A)®A^{-T}:®B. $$
\end{priklad}

\begin{poznamka}
	Chain rule works as usual.
\end{poznamka}

\begin{priklad}
	$$ \frac{d}{dt}(\det ®A(t)) = (\det ®A) \tr\(®A^{-1} \frac{d®A}{dt}\). $$
\end{priklad}

\begin{priklad}
	$$ \frac{\partial ®A^{-1}}{\partial ®A}[®B] = \frac{d}{d \tau}\((®A + \tau ®B)^{-1}\)|_{\tau = 0} = \frac{d}{d \tau}\((®I + \tau ®A^{-1} ®B)^{-1}®A^{-1}\)|_{\tau = 0} = $$
	$$ = \frac{d}{d\tau}\((®I - \tau ®A^{-1}®B + …)®A^{-1}\)|_{\tau = 0} = - ®A^{-1}®B®A^{-1}. $$
\end{priklad}

\begin{priklad}
	$$ \frac{\partial e^{®A}}{\partial ®A}[®B] = \frac{d}{d \tau}\(e^{®A + \tau ®B}\)|_{\tau = 0} = \frac{d}{d\tau} \(®I + (®A + \tau ®B) + \frac{(®A + \tau ®B)^2}{2!} + …\)|_{\tau = 0} = $$
	$$ = \frac{d}{d\tau}\(®I + (®A \tau ®B) + … + \tau(®A ®B + ®B ®A) + \tau(®A®A®B + ®A®B®A + ®B®A®A) + …\) $$
\end{priklad}

\begin{veta}[Daleckii-Krein theorem]
	$®A \in ®R^{3 \times 3}$ real symmetric matrix. $®A = \sum_{i=1}^3 \lambda_i ®P_i$, $®P_i$-projector to $i$-th eigenvector, $®P_i = ¦v_i \otimes ¦v_i$. $f$ real valued function $f: ®R \rightarrow ®R$ differentiable.
	$$ f(®A) := \sum_{i=1}^3 f(\lambda_i) ®P_i = \sum_{i=1}^3 f(\lambda_i) ¦v_i \otimes ¦v_i. $$
	$$ Df(®A)[®B] = \sum_{i=1}^3 \frac{df}{d\lambda}|_{\lambda = \lambda_i} ®P_i ®B ®P_i + \sum_{i=1}^3 \sum_{j=1, j≠i}^3 \frac{f(\lambda_i) - f(\lambda_j)}{\lambda_i - \lambda_j} ®P_i ®B ®P_j $$
	$$ (Df(®A)[®B])_{ij} = \frac{f(\lambda_i) - f(\lambda_j)}{\lambda_i - \lambda_j} B_{ij}, if i ≠ j, (Df(®A)[®B])_{ij} = \frac{df}{d\lambda}|_{\lambda = \lambda_j} B_{ij}, if i=j. $$

\end{veta}

\begin{dukaz}
	From chain rule:
	$$ \frac{\partial f(®A)}{\partial ®A} = \sum_{i=1}^3 \frac{df(\lambda_i)}{d\lambda} |_{\lambda = \lambda_i} \frac{\partial \lambda_i}{\partial ®A} ¦v_i \otimes ¦v_i + \sum_{i=1}^3 f(\lambda_i) \frac{\partial ¦v_i}{\partial ®A}\otimes ¦v_i + \sum_{i=1}^3 f(\lambda_i) ¦v_i \otimes \frac{\partial ¦v_i}{\partial ®A} $$

	First derivative at right side:
	$$ ®A ¦v_i = \lambda_i¦v $$
	$$ \frac{\partial ®A}{\partial ®A} ¦v_i + ®A \frac{\partial ¦v_i}{\partial ®A} = \frac{\partial \lambda_i}{\partial ®A}¦v_i + \lambda_i \frac{\partial ¦v_i}{\partial ®A} \qquad ·¦v_i $$
	$$ \frac{\partial A_{mn}}{\partial A_{kl}}(¦v_i)_n + (A_{mn}) \frac{\partial (¦v_i)_n}{\partial A_{kl}} = \frac{\partial \lambda_i}{\partial A_{kl}}(¦v_i)_m + \lambda_i \frac{\partial (¦v_i)_n}{\partial A_{kl}} $$
	$$ \delta_{mk}\delta_{nl}(¦v_i)_n + A_{mn}\frac{\partial (¦v_i)_n}{\partial A_{kl}} = \frac{\partial \lambda_i}{\partial A_{kl}}(¦v_i)_m + \lambda_i \frac{\partial (¦v_i)_n}{\partial A_{kl}} \qquad ·(¦v_i)_m \sum_m $$

	$$ \sum_m \frac{\partial \lambda_i}{\partial A_{kl}}(¦v_i)_m (¦v_i)_m = \frac{\partial \lambda_i}{\partial A_{kl}} $$

	From symmetry of ®A and definition of eigenvector:
	$$ \sum_m A_{mn}\frac{\partial (¦v_i)_n}{\partial A_{kl}} (¦v_i)_m = \lambda_i \frac{\partial (¦v_i)_n}{\partial A_{kl}}(¦v_i)_n $$

	$$ \sum_m \delta_{mk}\delta_{nl}(¦v_i)_n (¦v_i)_m = \delta_{nl}(¦v_i)_n(¦v_i)_k $$
		
	So
	$$ \lambda_i \frac{\partial (¦v_i)_n}{\partial A_{kl}}(¦v_i)_n + \delta_{nl}(¦v_i)_n(¦v_i)_k = \frac{\partial \lambda_i}{\partial A_{kl}} + \lambda_i \frac{\partial (¦v_i)_n}{\partial A_{kl}} \qquad \sum_n $$
	$$ \sum_n \lambda_i \frac{\partial (¦v_i)_n}{\partial A_{kl}}(¦v_i)_n + (¦v_i)_l(¦v_i)_k = \frac{\partial \lambda_i}{\partial A_{kl}} + \lambda_i \frac{\partial (¦v_i)_n}{\partial A_{kl}} $$
	$$ (¦v_i)_l(¦v_i)_k = \frac{\partial \lambda_i}{\partial A_{kl}} $$
	$$ \frac{\partial \lambda_i}{\partial ®A} = ¦v_i \otimes ¦v_j $$

	Second derivative at right side:
	$$ \frac{\partial ®A}{\partial ®A} ¦v_i + ®A \frac{\partial ¦v_i}{\partial ®A} = \frac{\partial \lambda_i}{\partial ®A}¦v_i + \lambda_i \frac{\partial ¦v_i}{\partial ®A} \qquad ·¦v_j $$
	$$ ®A \frac{\partial ¦v_i}{\partial ®A}·¦v_j = \lambda_i \frac{\partial ¦v_i}{\partial ®A}·¦v_j $$
	…
	$$ \frac{\partial ¦v_i}{\partial ®A}·¦v_j = \frac{¦v_j \otimes ¦v_i}{\lambda_i - \lambda_j}i = \frac{\delta_{kj}\delta_{il}}{\lambda_i - \lambda_j} $$
	$$ \(\frac{\partial ¦v_i}{\partial ®A}[®X]\)_j = \frac{\delta_{im}\delta_{jn}}{\lambda_i - \lambda_j} = \frac{¦v_i · ®X ¦v_j}{\lambda_i - \lambda_j}. $$

	$$ \frac{\partial ¦v_i}{\partial ®A}[®X] = \sum_{j=1}^3 \frac{¦v_i · ®X¦v_j}{\lambda_i - \lambda_j}¦v_j. $$
\end{dukaz}

\begin{poznamka}[V dokončení důkazu se ještě použije]
	$$ (¦a \otimes ¦b):(¦c \otimes ¦d) = (¦b·¦c)(¦a·¦d). $$
\end{poznamka}

% 19. 10. 2022 (from notes of my schoolmate)

\subsection{Differential operators}
\begin{definice}
	$$ \Div ¦v := ¦h(\nabla ¦v)? $$
	$$ (\rot ¦v)·¦w = \Div(¦v \times ¦w), \qquad \forall ¦w \in ®R^3 \text{ fixed} $$
	$$ (\Div ®A)·¦w = \Div(®A^T ¦w), \qquad \forall ¦w \in ®R^3 \text{ fixed}, \qquad (\Div ®A)_i = \frac{\partial A_{im}}{\partial x_m} $$
	$$ (\rot ®A)^T ¦w := \rot(®A^T ¦w), \qquad \forall ¦w \in ®R^3 \text{ fixed}, \qquad (\rot ®A)_{ij} = \epsilon_{jkl} \frac{\partial A_{il}}{\partial x_k} $$
\end{definice}

\begin{tvrzeni}
	$$ \rot \nabla \phi = ¦o, \quad \rot(\nabla ¦v) = ®O, \quad \Div(\rot ¦v) = 0, \quad \Div(\rot ®A) = ¦o. $$

	\begin{dukazin}
		$$ \int_\Omega \rot(\nabla \phi)· d¦S \overset{\text{Stokes}} \int_{\partial \Omega} \nabla \phi · d¦l = \phi(\text{end point}) - \phi(\text{starting point}) \overset{\text{closed curve}}= 0. $$

		$$ \int_\Omega \Div(\rot ¦v) dV = \int_{\partial \Omega} \rot ¦v · d¦S = \int_{\partial \Omega^+} \rot ¦v · d¦S + \int_{\partial \Omega^-} \rot ¦v · d¦S = 0 $$
	\end{dukazin}
\end{tvrzeni}

\begin{veta}[Stokes theorem]
	$$ \int_\Omega \Div ¦v dV = \int_{\partial \Omega} ¦v·d¦S := \int_{\partial \Omega} ¦v·¦n dS $$

	$$ \int_{S} \rot ¦v · dS = \int_{\partial S} ¦v·d¦l $$

	Using previous identities we can get integral identities using Stokes theorem:
	$$ \int_{\Omega} (\Div ®A) · ¦v dV = \int_{\partial \Omega} ®A^T ¦v·¦n dS - \int_\Omega ®A : \nabla ¦v dV, $$
	$$ \int_\Omega(\rot ¦v) dV = -\int_{\partial \Omega} ¦v \times ¦n dS. $$
\end{veta}

\begin{poznamka}[Kinematics]
	How to describe motion (no motion, no physics, just geometry): We have starting point $P$, set coordinates $\implies$ we have point ¦X. Then we do some deformation in time: $¦x = \chi(¦x, t)$ describes motion and deformation.

	How to describe properties of continuously distributed matter? We have 3 possibilities (e. g. for $\theta$):
	\begin{itemize}
		\item $\theta(P, t)$,
		\item $\theta(¦X, t)$ – Lagrangian description,
		\item $\theta(¦x, t)$ – Eulerian description.
	\end{itemize}

	But we don't know what $\chi$ is and how to work with it $\rightarrow$ we linearize (use derivations), write down PDEs and hopefully form local solutions. So

	Problem: What is the relation between $d¦x$ and $d¦X$?
	$$ d¦x = ¦x_2 - ¦x_1 = \chi(¦X_2, t) - \chi(¦X_1, t) = \chi(¦X_1 + d¦X, t) - \chi(¦X_1, t) = \chi(¦X_1, t) + \frac{\partial \chi}{\partial ¦X}(¦X_1, t)d¦X + … $$
	$$ \implies d¦x = \frac{\partial \chi}{\partial ¦X}(¦X, t) d¦X. $$
\end{poznamka}

\begin{definice}[Deformation gradient]
	$$ ®F(¦X, t) := \frac{\partial \chi}{\partial ¦X}(¦X, t). $$
\end{definice}

\begin{dusledek}
	$$ dv = (\det ®F) dV $$
	$$ d¦s = (\det ®F)®F^{-T} d¦S $$
	$$ ¦n = ®F^{-T} ¦N $$
\end{dusledek}

TODO?

\begin{poznamka}[How to characterise deformation/strain?]
	$$ d¦x·d¦x - d¦X·d¦X = ®F d¦X·®Fd¦X - d¦X·d¦X = (®F^T ®F - ®I)d¦X·d¦X $$
\end{poznamka}

\begin{definice}
	Green-Stieland? strain tensor:
	$$ ®E(¦X, t) := \frac{1}{2} (®F^T ®F - ®I) $$

	Right Cauchy-Green tensor:
	$$ ®C(¦X, t) := ®F^T(¦X, t)·®F(¦X, t) $$

	Euler-Almans strain tensor:
	$$ \bepsilon (¦x, t)|_{¦x = \chi(¦X, t)} = \frac{1}{2}\(®I - F^{-T}(¦X, t)®F^{-1}(¦X, t)\) $$

	Left Cauchy-Green tensor
	$$ ®B(¦x, t) = ®F(¦X, t)®F^T(¦X, t) $$
\end{definice}

TODO?

\begin{poznamka}[Velocity]
	$$ ¦V(¦X, t) := \frac{d \chi(¦X, t)}{dt} $$
	$$ ¦v(¦x, t) = ¦V(\chi^{-1}(¦x, t), t) $$
	or
	$$ ¦v(\chi(¦X, t), t) = ¦V(¦X, t) $$
	(Euler velocity field.)
\end{poznamka}

\begin{definice}
	$$ \frac{d¦v}{dt}(¦x, t) := \frac{\partial ¦v}{\partial t}(¦x, t) + (¦v(¦x, t) · \nabla_{¦x})¦v(¦x, t), $$
	$$ \frac{d \phi(¦x, t)}{dt} := \frac{\partial \phi}{\partial t} + (¦v·\nabla) \phi. $$
\end{definice}

\begin{definice}[Time derivative of the deformation gradient]
	$$ \frac{d®F}{dt}(¦X, t) = \frac{d}{dt} \frac{\partial \chi(¦X, t)}{\partial ¦X} = \frac{\partial}{\partial ¦X} \frac{d\chi(¦X, t)}{dt} = \frac{\partial}{\partial ¦X} ¦V(¦X, t) = \frac{\partial}{\partial ¦X}(¦v(\chi(¦X, t), t)) = $$
	$$ = \frac{\partial ¦V}{\partial ¦x}|_{¦x = \chi(¦X, t)} \frac{\partial \chi}{\partial ¦X}(¦X, t) \implies \frac{d ®F}{dt}(¦X, t) = ®L(¦x, t)|_{¦x = \chi(¦X, t)} ®F(¦X, t), $$
	where $®L(¦x, t) := \frac{\partial ¦v}{\partial ¦x}(¦x, t) = \nabla_{¦x} ¦v(¦x, t)$.
\end{definice}

\begin{tvrzeni}[Time derivateives of $d¦x, d¦s, dv$]
	$$ \frac{d}{dt} d¦x = \frac{d}{dt} ®Fd¦X = \(\frac{d}{dt} ®F(¦X, t)\) d¦X = ®L(¦x, t)®F(¦X, t) d¦X = ®L(¦x, t) d¦x $$
\end{tvrzeni}

% 26. 10. 2022 (from notes of my schoolmate)

\begin{tvrzeni}[Rate(s) of strain]
	$$ \frac{d ®E}{dt}(¦X, t) = \frac{d}{dt} \(\frac{1}{2}(®F^T ®F - ®I)\) = \frac{1}{2}\(\frac{d®F^T}{dt} ®F + ®F^T \frac{d®F}{dt}\) = \frac{1}{2}\(®F^T®L^T®F + ®F^T®L®F\) = $$
	$$ = ®F^T \(\frac{1}{2} (®L^T + ®L)\)®F =: ®F®D®F, $$
	$$ ®D(¦x, t) := \frac{1}{2}(®L + ®L^T). $$

	$$ \frac{d\bepsilon}{dt}(¦x, t) = \frac{1}{2}\(\frac{d}{dt} \(®I - ®F^{-T}®F^{-1}\)\) = \frac{1}{2}\(-\(\frac{d®F^{-1}}{dt}\)^T®F^{-1} - ®F^{-T} \frac{d®F^{-1}}{dt}\) = $$
	$$ = \frac{1}{2}(®L^T®F^{-T}®F^{-1} + ®F^{-T}®F^{-1}®L) = \frac{1}{2}(-®L^T(®I - ®F^{-T}®F^{-1}) + ®L^T - (®I - ®F^{-T}®F^{-1})®L + ®L) = $$
	$$ = -®L^T \bepsilon - \bepsilon ®L + ®D \implies $$
	$$ \frac{d®E}{dt}(¦x, t) + ®L^T(¦x, t)\bepsilon(¦x, t) + \bepsilon(¦x, t)®L(¦x, t) = ®D(¦x, t) $$

	$$ \frac{d®B}{dt}(¦x, t) = ®L®F®F^T + ®F®F^T®L^T = ®L®B + ®B®L^T \implies $$
	$$ \implies \overset{\nabla}{®B} := \frac{d®B}{dt} - ®L®B - ®B®L^T = ®O $$
\end{tvrzeni}

\begin{tvrzeni}
	$$ \overset{\nabla}{®A}(¦x, t) := ®F(¦X, t)\[\frac{d}{dt}(®F^{-1}(¦X, t) ®A(\chi(¦X, t), t)®F^{-T}(¦X, t))\]®F^T(¦X, t)$$

	$$ \overset{\nabla}{®A}(¦x, t) = -®L®A + \frac{d®A}{dt} - ®L®A^T $$
\end{tvrzeni}

\begin{poznamka}
	A reasonable definition of time derivative for a tensor field of type "normal vector $\rightarrow$ tangent vector" is previous.
\end{poznamka}

\begin{poznamka}[Material surface]
	$f(¦x, t) = 0$ \ldots implicit declaration of surface.

	$$ \frac{\partial f}{\partial t}(¦x, t) + ¦v(¦x, t)·\nabla_{¦x} f(¦x, t) = 0 \qquad \forall ¦x, f(¦x, t) = 0 $$



	$\frac{\partial f}{\partial t} + ¦v·\nabla f = 0$ for $z = g(x, y, t)$ and the velocities ¦v are computed through this equation.

	$\nabla_{¦x} f(¦x, t) ds$ a normal to the surface $\implies$ $\frac{1}{|\nabla_{¦x}f|} \frac{\partial f}{\partial t} + ¦v·\frac{\nabla_{¦x} f}{|\nabla_{¦x} f|} = 0$ $\implies$ $¦v·¦n = - \frac{\frac{\partial f}{\partial t}}{|\nabla_{¦x} f|}$.
\end{poznamka}

\begin{veta}[Reynolds transport theorem]
	$$ \frac{d}{dt} \int_{V(t)} \phi(¦x, t) dv = \int_{V(t)}\(\frac{d\phi}{dt}(¦x, t) + \phi(¦x, t) \Div ¦v(¦x, t)\) dv $$

	\begin{dukazin}
		$$ \frac{d}{dt} \int_{V(t_0)} \phi(¦x, t) dv = \frac{d}{dt} \int_{V(t_0)} \phi(¦x, t)|_{¦x = \chi(¦X, t)} \det ®F(¦X, t) dV = $$
		$$ = \int_{V(t_0)} \frac{d}{dt}\(\phi(\chi(¦X, t), t) \det ®F(¦X, t)\)dV = $$
		$$ = \int_{V(t_0)}\[\frac{\partial \phi}{\partial t}(¦x, t) + ¦v(¦x, t)·\nabla_{¦x}\phi(¦x, t)|_{…} \det ®F(¦X, t) + \phi(¦x, t)|_{…}\frac{d}{dt}\det ®F(¦X, t)\]dV = $$
		$$ = \int_{V(t_0)} \(\frac{d \phi}{dt}(¦x, t) + \phi(¦x, t) \Div ¦v(¦x, t)\)|_{¦x = \chi(¦X, t)} (\det ®F) dV = $$
		$$ = \int_{V(t_0)}\(\frac{d\phi}{dt}(¦x, t) + \phi(¦x, t) \Div ¦v(¦x, t)\) dv $$
	\end{dukazin}
\end{veta}

TODO?

\begin{definice}[Displacement]
	$$ ¦U(¦X, t) := \chi(¦X, t) - ¦X $$
\end{definice}

% 02. 11. 2022
\section{Dynamics}
\begin{poznamka}[Outline]
	Newtons 2nd law for single particle: $m \frac{d^2 ¦x}{dt^2} = ¦F$. We want to find a counterpart for continuously distributed matter…
	$$ \frac{d}{dt} (m \frac{d¦x}{dt}) = ¦F $$
	We have $\frac{dm}{dt} = 0$ ($m = \const$). In continuum mechanics we have $\frac{d}{dt} m_{V(t)} = 0$, $m_{V(t)} := \int_{V(t)} \rho(¦x, t) dv$  ($\rho$ = density). Than (from Reynolds transport theorem)
	$$ 0 = \frac{d}{dt} m_{V(t)} = \frac{d}{dt} \int_{V(t)} \rho(¦x, t) dv = \int_{V(t)} \(\frac{d\rho}{dt}(¦x, t) + \rho(¦x, t) \Div_{¦x} ¦v(¦x, t)\)dv $$
	From localization lemma ($V(t)$ is arbitrary material volume) we get Balance of mass:
	$$ \frac{d \rho}{d t}(¦x, t) + \rho(¦x, t) \Div_{¦x} ¦v(¦x, t) = 0. $$

	Spring: $m \frac{d^2 ¦x}{dt^2} = - k¦x - b\frac{d¦x}{dt}$, $k$ is spring stiffness, $-b…$ is damping (proportional to the velocity).
	$$ m \frac{d^2¦x}{dt^2}·\frac{d¦x}{dt} = - k¦x · \frac{d¦x}{dt} - b \frac{d¦x}{dt}·\frac{d¦x}{dt}, $$
	$$ \frac{d}{dt} \(\frac{1}{2} m \(\frac{d¦x}{dt}\)^2 + \frac{k}{2}(¦x)^2\) = - b \(\frac{d¦x}{dt}\)^2, $$
	where $(¦x)^2 = ¦x·¦x$. With $b=0$ we have conservation of energy, with $b > 0$ we have energy loss through $-b \(\frac{d¦x}{dt}\)^2$ = dissipation = conversion of mechanical energy into heat.

	$$ dU = dQ + dW, \qquad dS = \frac{dQ}{T} \qquad \text{(classical equilibrium thermodynamic)} $$
\end{poznamka}

\begin{lemma}
	$$ \frac{d}{dt} \int_{V(t)} \rho(¦x, t) \phi(¦x, t) dv = \int_{V(t)} \rho(¦x, t) \frac{d\phi}{dt}(¦x, t) dv. $$

	\begin{dukazin}
		$$ \frac{d}{dt} \int_{V(t)} \rho \phi dv \overset{\text{Reynolds}}= \int_{V(t)} \(\frac{d}{dt} (\rho \phi) + \rho \phi \Div ¦v\)dv = $$
		$$ = \int_{V(t)} \(\frac{d\rho}{dt}\phi + \rho \frac{d\phi}{dt} + \rho\phi \Div¦v\)dv \overset{\text{Balance of mass}}= \int_{V(t)} \rho(¦x, t) \frac{d\phi}{dt}(¦x, t) dv. $$
	\end{dukazin}
\end{lemma}

\begin{poznamka}[Newton second law]
	$$ ¦p_{V(t)} := \int_{V(t)} \rho(¦x, t)¦v(¦x, t) dv $$
	$$ LHS = \frac{d}{dt}¦p_{V(t)} = \frac{d}{dt} \int_{V(t)} \rho(¦x, t)¦v(¦x, t) dv $$
	$$ RHS = ¦F_{V(t)} = \underbrace{¦F_{body}}_{:= \int_{V(t)} \rho(¦x, t) ¦b(¦x, t) dv} + \underbrace{¦F_{surface}}_{:= \int_{\partial V(t)}¦t(t, ¦x) ds} $$
	$¦F_{surface}$ is called surface faces or contact forces. ¦t is traction vector.
\end{poznamka}

\begin{poznamka}
	We will assume that $¦t = ¦t(¦x, t, ¦n(¦x, ¦t))$.

	From Newton's 3rd law (action + reaction): $¦t(¦x, t, ¦n) = -¦t(¦x, t, -¦n)$. Simplest formula that guarantees this is linear, so we will assume (it can be proved with Cauchy tetrahedron argument, which isn't part of this lecture, but proof can be in exam) that ¦t is linear with respect to ¦n:
	$$ ¦t(¦x, t, ¦n) = ®T(¦x, t)¦n, $$
	®T is tensor matrix $®R^{3 \times 3}$ and is called Cauchy stress tensor.

	\begin{prikladyin}
		$$ ®T(¦x, t) = -p(¦x, t)®I, \qquad \text{where $p$ is pressure}. $$
	\end{prikladyin}

	With this, we can rewrite Newton's second law:
	$$ \frac{d}{dt} \int_{V(t)} \rho ¦v dx = \int_{V(t)} \rho ¦b dv + \int_{\partial V(t)} ®T¦n ds \overset{\text{Stokes}}= \int_{V(t)} \rho ¦b dv + \int_{V(t)}(\Div_{¦x} ®T) dv. $$
	With Reynolds and Localization lemma we get balance of (linear) momentum (and it's counterpart of N. 2nd law):
	$$ \rho \frac{d¦v}{dt} = \rho ¦b dv + \Div_{¦x} ®T. $$
\end{poznamka}

\begin{poznamka}
	$$ \frac{d}{dt} \(\frac{1}{2} m \(\frac{d¦x}{dt}\)^2 + \frac{k}{2}(¦x)^2\) = - b \(\frac{d¦x}{dt}\)^2, $$
	$$ \frac{d}{dt} \underbrace{\int_{V(t)} \(\frac{1}{2} \rho·(¦v)^2 + \underbrace{\rho e}_{\text{internal energy density}}\)dv}_{\text{net total energy}} = $$
	$$ = \int_{V(t)} \rho¦b·¦v dv + \int_{\partial V(t)} ®T ¦n·¦v ds - \int_{\partial V(t)} \underbrace{¦J_q}_{\text{heat flux vector}}·¦n ds $$
	$(¦J_q := -k\nabla\theta)$

	$dU$ is internal energy, $dW$ is work done by internal + surface forces (first two expressions on RHS), $dQ$ is exchange of heat on surface (the third expression)

	Stokes theorem, Reynolds transport theorem, Localization lemma $\implies$ First law of thermodynamics:
	$$ \rho \frac{d}{dt}\(\frac{1}{2}(¦v)^2 + e\) = \rho¦b·¦v - \Div ¦J_q + \Div (®T^T ¦v) $$
	
	With $\rho \frac{d}{dt}(\frac{1}{2}¦v·¦v) = (\Div ®T)·¦v + \rho ¦b·¦v$
	$$ \rho \frac{de}{dt} = \Div(®T^T ¦v) - (\Div ®T)·¦v - \Div ¦J_q $$
	$$ \rho \frac{de}{dt} = ®T:®L - \Div ¦J_q, \qquad ®L := \nabla_{¦x} ¦v. $$
\end{poznamka}

\begin{definice}[Entropy]
	Basic feat: Continuous medium in an isolated vessel: $¦v \rightarrow 0$, $\theta \rightarrow \const$ ($\theta$ = temperature field).

	Isolated = no mechanical energy exchange with outside environment ($¦b = ¦o$, $¦v|_{\partial \Omega} = ¦o$) and no heat flux through boundary $(¦J_q · ¦n|_{\partial \Omega} = 0)$

	We want a quantity that increases in time! Let denote it $\eta$. Assume $\eta = \eta(e, \rho)$ and $e = e(\eta, \rho)$. Chain rule:
	$$ \frac{de}{dt} = \frac{\partial e}{\partial \eta}(\eta, \rho) \frac{d\eta}{dt} + \frac{\partial e}{\partial \rho}\frac{d\rho}{dt} $$
	We know balance of internal energy $\rho \frac{de}{dt} = ®T:®L - \Div ¦J_q$. So
	$$ \rho\frac{\partial e}{\partial \eta}(\eta, \rho) \frac{d\eta}{dt} + \rho\frac{\partial e}{\partial \rho}\frac{d\rho}{dt} = ®T:®L - \Div ¦J_q $$
	$$ \rho\frac{\partial e}{\partial \eta}(\eta, \rho) \frac{d\eta}{dt} = ®T:®L - \Div ¦J_q - \rho\frac{\partial e}{\partial \rho}\frac{d\rho}{dt} $$
	$$ \rho\frac{\partial e}{\partial \eta}(\eta, \rho) \frac{d\eta}{dt} = ®T:®L - \Div ¦J_q - \rho^2\frac{\partial e}{\partial \rho} \Div¦v $$
	$$ \rho\frac{d\eta}{dt} = \frac{®T:®L - \Div ¦J_q - \rho^2\frac{\partial e}{\partial \rho} \Div¦v}{\frac{\partial e}{\partial \eta}(\eta, \rho)} $$

	This is point-wise equation and it is too demanding. So let us work with ret quantities:
	$$ \frac{d}{dt} \int_{V(t)} \rho\eta = \int_{V(t)}\frac{1}{\frac{\partial e}{\partial \eta}(\eta, \rho)}\(-p + \rho^2 \frac{\partial e}{\partial \rho}\) \Div ¦v dv + \int_{V(t)} \frac{\Div(k\nabla \theta)}{\frac{\partial e}{\partial \eta}(\eta, \rho)} dv $$
	Let us try to fix $\rho^2 \frac{\partial e}{\partial \rho}(\eta, \rho) = p$:
	$$ \frac{d}{dt} \int_{V(t)} \rho\eta = 0 + \int_{V(t)} \frac{\Div(k\nabla \theta)}{\frac{\partial e}{\partial \eta}(\eta, \rho)} dv $$
	$$ \frac{d}{dt} \int_{V(t)} \rho\eta = \int_{V(t)} \Div\(\frac{k\nabla \theta}{\frac{\partial e}{\partial \eta}(\eta, \rho)}\) dv - \int_{V(t)}k\nabla \theta · \nabla \(\frac{1}{\frac{\partial e}{\partial \eta}}\) dv $$
	From boundary condition $k \nabla \theta · ¦n|_{\partial \Omega} = 0$ and stokes theorem
	$$ \frac{d}{dt} \int_{V(t)} \rho\eta = 0 - \int_{V(t)}k\nabla \theta · \nabla \(\frac{1}{\frac{\partial e}{\partial \eta}}\) dv $$
	$$ \frac{d}{dt} \int_{V(t)} \rho\eta = - \int_{V(t)}\frac{k\nabla \theta · \nabla \(\frac{\partial e}{\partial \eta}(\eta, \rho)\)}{\(\frac{\partial e}{\partial \eta}(\eta, \rho)\)^2} dv. $$
	Let us try to fix $\frac{\partial e}{\partial \eta}(\eta, \rho) = \theta$ $\implies$ $\frac{d}{dt} \int_{V(t)} \rho\eta ≥ 0$.

	Summary: Solve $\frac{\partial e}{\partial \eta}(\eta, \rho) = \theta$ and $\rho^2 \frac{\partial e}{\partial \rho}(\eta, \rho) = p$ and we will get $\eta = \eta(e, \rho)$ such that $\frac{d}{dt} \int_{V(t)} \rho \eta ≥ 0$. Function $\eta$ is called the entropy density.
\end{definice}

\begin{poznamka}[Experimental facts]
	$$ p(\rho, \theta) = c_V(\gamma - 1)\rho\theta \qquad (\text{something like } \frac{PV}{T} = \const) $$
	$$ e(\rho, \theta) = c)V \theta $$
	$$ c_V = \const, \qquad \gamma = \const $$

	$c_V$ = specific heat at constant volume.

	\begin{poznamkain}[$c_V$]
		$$ \frac{de}{dt} = \frac{\partial e}{\partial \rho} \frac{d\rho}{dt} + \frac{\partial e}{\partial \theta} \frac{d\theta}{dt} $$
		$$ \rho \frac{de}{dt} = ®T:®L - \Div ¦J_q = -p \Div ¦v + \Div(\kappa \nabla \theta) $$

		Constant volume $\implies$ $\frac{d\rho}{dt} = 0$ and $\Div ¦v = 0$
		$$ \text{change of temperature } = \int_{V(t)} \rho \frac{\partial e}{\partial \theta}(\theta, \rho) \frac{d\theta}{dt} = \int_{V(t)} \Div(\kappa \nabla \theta) dv = \text{ heat exchange} $$
	\end{poznamkain}
\end{poznamka}

% 16. 11. 2022

\begin{poznamka}[Balance of angular momentum]
	$$ (m \frac{d¦x^2}{dt^2} = ¦F \qquad \times ¦x \qquad m \frac{d¦x^2}{d¦t} \times ¦x = ¦F \times ¦x.) $$
	Primitive equation:
	$$ \int_{V(t)} \rho \frac{d ¦v}{d t} \times ¦x dv = \int_{V(t)} \rho ¦b \times ¦x + \underline{\int_{\partial V(t)} ®T ¦n \times ¦x ds}. $$

	We already know that
	$$ \rho \frac{d¦v}{dt} = \rho ¦b + \Div ®T $$
	$$ \int_{V(t)} \rho \frac{d¦v}{dt} \times ¦x dv = \int_{V(t)} \rho ¦b \times ¦x dv + \underline{\int_{V(t)} \Div ®T \times ¦x dv} $$

	Is underlined parts equivalent?
	$$ \int_{\partial V(t)} ®T ¦n \times ¦x ds = \int_{V(t)} \Div ®T \times ¦x dv. $$
	$$ LHS = \int_{\partial V(t)} \epsilon_{ijk} T_{jl} n_l x_k ds = \int_{\partial V(t)}(\epsilon_{ijk} T_{jl} x_k)·n_l ds \overset{\text{Stokes}}= \int_{V(t)} \frac{\partial}{\partial x_l}(\epsilon_{ijk} T_{jl} x_k)dv =  $$
	$$ = \int_{V(t)} \epsilon_{ijk} \frac{\partial T_{jl}}{\partial x_l} x_k + \int_{V(t)} \epsilon_{ijk} T_{jl} \underbrace{\frac{\partial x_k}{\partial x_l}}_{\delta_{kl}} dv = \int_{V(t)} (\Div ®T \times ¦x) dv + \int_{V(t)} \epsilon_{ijl} T_{jl} dv. $$

	LHS is equal to RHS (balance of angular momentum holds) provided that
	$$ \int_{V(t)} \epsilon_{ijl} ®T_{jl} dv = 0. $$
	This holds provided that ®T is symmetric matrix.

	Balance of linear momentum + symmetry of ®T $\implies$ balance of angular momentum holds. So from now we work with symmetric ®T. (There are parts of continuum mechanics, where this doesn't hold (especially some electric continuum), but we continue in our part.)
\end{poznamka}

\begin{poznamka}[Eulerian description]
	So far we have ($\rho(¦x, t)$, $\frac{d}{dt} = \frac{\partial}{\partial t} + ¦v · \nabla$)
	$$ \frac{d \rho}{d t} + \rho \Div ¦v = 0, \qquad \rho \frac{d¦v}{dt} = \Div ®T + \rho ¦b, \qquad \rho \frac{d e}{dt} = ®T:®L - \Div ¦j_q, \qquad ®T = ®T^T.  $$
\end{poznamka}

\begin{poznamka}[Lagragian description]
	($\rho(¦X, t)$)
	Balance of mass:
	$$ \frac{d}{dt} \int_{V(t)} \rho(¦x, t) dv = 0 \implies \int_{V(t)} \rho(¦x, t) dv = \int_{V(t_0)} \rho_R(¦X) dV $$
	$$ \int_{V(t)} \rho(\chi(¦X, t), t) \det ®F(¦X, t) dV = \int_{V(t_0)} \rho_R(¦X) dV $$
	$$ \overset{\text{Localization lemma}} \rho(¦x, t)|_{¦x = \chi(¦X, t)} \det ®F(¦X, t) = \rho_R(¦X). $$

	Balance of momentum:
	$$ \int_{V(t)} \rho(¦x, t) \frac{d¦v}{dt} (¦x, t) dv = \int_{V(t)} \Div_{¦x} ®T(¦x, t) dv + \int_{V(t)} \rho(¦x, t)¦b(¦x, t)dv $$
	$$ \int_{V(t_0)} \rho(\chi(¦X, t), t) \frac{d¦v}{dt}(¦x, t)|_{¦x = \chi(¦X, t)} \det ®F(¦X, t) dv = \int_{\partial V(t)} ®T(¦x, t)¦n(¦x, t) ds + \int_{V(t_0)} \rho(\chi(¦x, t), t)¦b(¦x, t)|_{¦x = \chi(¦X, t)} \det ®F(¦X, t) dv $$
	$$ \int_{V(t_0)} \rho_R(¦X) \frac{\partial^2 \chi}{\partial t^2}(¦X, t) dV = \int_{\partial V(t_0)} (\det ®F(¦X, t))®T(¦x, t)|_{¦x = \chi(¦X, t)} ®F^{-T}(¦X, t)¦N(¦X) dS + \int_{V(t_0)} \rho_R(¦X)¦b(\chi(¦X, t), t) dV $$
	First Piola-Kirchhoff stress tensor: $®T_R(¦X, t) := \det ®F(¦X, t) ®T(¦x, t)|_{¦x = \chi(¦X, t)} ®F^{-T}(¦X, t)$.
	$$ \int_{V(t_0)} \rho_R(¦X) \frac{\partial^2 \chi}{\partial t^2}(¦X, t) dV = \int_{\partial V(t_0)} ®T_R(¦X, t) ¦N dS + \int_{V(t_0)} \rho_R(¦X)¦b(\chi(¦X, t), t) dV $$
	$$ \int_{V(t_0)} \rho_R(¦X) \frac{\partial^2 \chi}{\partial t^2}(¦X, t) dV = \int_{V(t_0)} \Div_{¦X}®T_R(¦X, t) dV + \int_{V(t_0)} \rho_R(¦X)¦b(\chi(¦X, t), t) dV $$
	$$ \rho_R(¦X) \frac{\partial^2 \chi}{\partial t^2} = \Div_{¦X} ®T_R dV + \rho_R ¦b. $$

	Internal energy:
	$$ \int_{V(t)} \rho \frac{de}{dt} dV = \int_{V(t)} ®T:®L dv - \int_{\partial V(t)} ¦j_q · ¦n ds $$
	$®T:®L = \tr(®T ®L^T) = \tr((\frac{1}{\det ®F} ®T_R ®F^T)(®F^{-T} \frac{d®F^T}{dt})) = \frac{1}{\det ®F} ®T_R : \frac{d®F}{dt}$.
	$$ \int_{V(t_0)} \rho_R \frac{de}{dt} dv = \int_{V(t)} \frac{1}{\det ®F} ®T_R : \frac{d®F}{dt} dv  - \int_{\partial V(t_0)} ¦j_q·(\det ®F) ®F^{-T} ¦N dS $$
	$$ \int_{V(t_0)} \rho_R \frac{de}{dt} dv = \int_{V(t_0)} ®T_R : \dot{®F} dV - \int_{V(t_0)} \Div_{¦X} ((\det ®F) ®F^{-1} ¦j_q) dV $$
	Refferential heat flux: $¦J_q(¦X, t) := (\det ®F(¦X, t)) ®F^{-1}(¦X, t) ¦j_q(¦x, t)|_{¦x = \chi(¦X, t)}$
	$$ \int_{V(t_0)} \rho_R \frac{de}{dt} dv = \int_{V(t_0)} ®T_R : \dot{®F} dV - \int_{V(t_0)} \Div_{¦X} ¦J_q dV $$
	$$ \rho_R \frac{de}{dt} = ®T_R : \dot{®F} - \Div_{¦X} ¦J_q. $$

	Angular momentum:
	$$ ®T_R ®F^T = ®F ®T_R^T $$
\end{poznamka}

\begin{poznamka}[How to find a formula for the Cauchy stress tensor?]
	Solids: $®T = f(\text{deformation})$, but $\chi$ contains also rotations or translations. $®T = f(\nabla \chi)$, but it still contains rotations. $®T = f(®F ®F^T - ®I) = f(®B - ®I)$. $¦x = ®Q ¦X \implies ®F = ®Q \implies ®F^T ®F = ®I$. So
	$$ ®T = f(®B). $$

	Can we formulate an evolution equation for ®B? Yes, we did it long time ago:
	$$ \overset{\nabla}{®B} = ®O, \qquad \overset{\nabla}{®B} := \frac{d®B}{dt} - ®L®B - ®B®L^T. $$
	We add this to equations we have for Eulerian description and solve them for $\rho, ¦v, ®B, e$ at points $¦x \in V(t) = \chi(V(t_0))$, so we must find $\chi$, but we may find it a posteriori after solving equations.
\end{poznamka}

% 23. 11. 2022

\subsection{Cauchy elastic solid}
\begin{poznamka}
	We consider isotropic solid $\implies$ $®T$ is isotropic function of ®B. Representation theorem for isotropic functions:
	$$ ®T = \alpha_0®I + \alpha_1 ®B + \alpha_2 ®B^t. $$
\end{poznamka}

\begin{definice}[Naive definition of elastic solid]
	In elastic solid we can store energy and get all the energy back (like in spring).

	Other view is that elastic solid recovers its shape.

	Storing energy we add with $e = e(®B, \eta)$.
\end{definice}

\begin{definice}[(Isotropic) elastic solid]
	$$ e = e(®B, \eta), \qquad \frac{d\eta}{dt} = 0 \text{ in mechanical processes (so it can still conduct heat).} $$
\end{definice}

\begin{definice}[Helmholk free energy]
	$$ \psi(\theta, \rho) := e(\eta, \rho)|_{\eta = \eta(\theta, \rho)} - \theta \eta(\theta, \rho) $$
	
	\begin{poznamkain}
		$$ \theta = \frac{\partial e}{\partial \eta}(\eta, \rho), $$
		solve for $\eta$: $\eta = \eta(\theta, \rho)$.
	\end{poznamkain}
\end{definice}

\begin{dusledek}
	Chain rule:
	$$ \frac{\partial \psi}{\partial \theta}(\theta, \rho) = \frac{\partial e}{\partial \eta}(\eta, \rho)|_{\eta=\eta(\theta, \rho)} · \frac{\partial \eta}{\partial \theta}(\theta, \rho) - \eta(\theta, \rho) - \theta \frac{\partial \eta}{\partial \theta}(\theta, \rho) =  $$
	$$ = \theta · \frac{\partial \eta}{\partial \theta}(\theta, \rho) - \eta(\theta, \rho) - \theta \frac{\partial \eta}{\partial \theta}(\theta, \rho) = -\eta(\theta, \rho). $$

	$$ \frac{\partial \psi}{\partial \rho}(\theta, \rho) = \frac{\partial e}{\partial \eta}(\eta, \rho)|_{\eta=\eta(\theta, \rho)} · \frac{\partial \eta}{\partial \rho}(\theta, \rho) + \frac{\partial\eta}{\partial \rho}(\eta, \rho)|_{\eta = \eta(\theta, \rho)} - \theta \frac{\partial \eta}{\partial \rho}(\theta, \rho) = \frac{\partial\eta}{\partial \rho}(\eta, \rho)|_{\eta = \eta(\theta, \rho)}. $$

	\begin{poznamkain}
		$$ \frac{\partial \psi}{\partial \rho}(\theta, \rho) := \frac{P_{th}(\theta, \rho)}{\rho^2}. $$
	\end{poznamkain}

	With $\psi(\theta, ®B) = (e(\eta, ®B) - \theta \eta)|_{\eta = \eta(\theta, ®B)}$:
	$$ \frac{\partial \psi}{\partial \theta}(\theta, ®B) = -\eta(\theta, ®B); $$
	$$ \frac{\partial \psi}{\partial ®B}(\theta, ®B) = \frac{\partial e}{\partial ®B}(\eta, ®B). $$
\end{dusledek}

\begin{poznamka}[Find evolution equation for the entropy]
	$$ \psi=e - \theta\eta, \qquad \frac{d}{dt} · \rho $$
	$$ \underbrace{\rho \frac{d\phi}{dt}}_{\rho \frac{\partial \psi}{\partial \theta}·\frac{d\theta}{dt} + \rho \frac{\partial \psi}{\partial ®B}:\frac{d®B}{dt}} = \underbrace{\rho \frac{de}{dt}}_{®T:®D - \Div ¦j_q} - \rho \frac{d\theta}{dt} \eta - \rho \theta \frac{d\eta}{d t} $$
	$$ -\rho \eta \frac{d\theta}{dt} + \rho \frac{\partial \psi}{\partial ®B}:\frac{d®B}{dt} = ®T:®D - \Div ¦j_q - \rho \frac{d\theta}{dt} \eta - \rho \theta \frac{d\eta}{d t} $$
	$$ \rho \theta \frac{d\eta}{dt} = ®T:®D - \rho \frac{\partial \psi}{\partial ®B}:\frac{d®B}{dt} - \Div¦j_q. $$
	We already know $\overset{\nabla}{®B} = ®O$ $\implies$ $\frac{d®B}{dt} - ®L®B - ®B®L^T = ®O$ $\implies$ $\frac{d®B}{dt} = ®L®B + ®B®L^T$. So
	$$ \rho \theta \frac{d\eta}{dt} = ®T:®D - \rho \tr\(\frac{\partial \psi}{\partial ®B}(®L®B - ®B®L^T)\) - \Div ¦j_q. $$
	In isotropic material:
	$$ TR = \tr\(\frac{\partial \psi}{\partial ®B}(®L®B - ®B®L^T)\) = \tr\(®B \frac{\partial \psi}{\partial ®B}®L + \frac{\partial \psi}{\partial ®B}®B®L^T\). $$
	For isotropic functions of tensor variable: $\psi = \psi(\theta, I_1, I_2, I_3)$,
	$$ \frac{\partial \psi}{\partial ®B} = \frac{\partial \psi}{\partial I_1}·\frac{\partial I_1}{\partial ®B} + \frac{\partial \psi}{\partial I_2}·\frac{\partial I_2}{\partial ®B} + \frac{\partial \psi}{\partial I_3}·\frac{\partial I_3}{\partial ®B} $$
	We know $\frac{\partial I_3}{\partial ®B} = (\det ®B)®B^{-T}$, $\frac{\partial I_1}{\partial ®B} = ®I$, $\frac{\partial I_2}{\partial ®B} = (\tr ®B)®I - ®B^T$. So we get that $\frac{\partial \psi}{\partial ®B}$ commutes with $®B$.

	$$ TR = \tr\(®B \frac{\partial \psi}{\partial ®B}(®L + ®L^T)\) = \tr\(2®B \frac{\partial \psi}{\partial ®B}®D\). $$
	$$ \rho \theta \frac{d\eta}{dt} = ®T:®D - 2\rho®B\frac{\partial \psi}{\partial ®B}:®D - \Div ¦j_q = \(®T - 2\rho®B\frac{\partial \psi}{\partial ®B}\):®D - \Div ¦j_q. $$
	$$ \rho \frac{d\eta}{dt} = \frac{1}{\theta}\(®T - 2\rho®B\frac{\partial \psi}{\partial ®B}\):®D - \Div \frac{¦j_q}{\theta} - \frac{¦j_q · \nabla \theta}{\theta^2}. $$
	$$ \rho \frac{d\eta}{dt} + \Div \frac{¦j_q}{\theta} = \frac{1}{\theta}\(®T - 2\rho®B\frac{\partial \psi}{\partial ®B}\):®D - \frac{¦j_q · \nabla \theta}{\theta^2}. $$

	$$ ®T = 2\rho®B\frac{\partial \psi}{\partial ®B} \implies \text{ no entropy change due to mechanical processes.} $$
	This is called hyperelactic solid or Green elastic solid.
\end{poznamka}

\begin{poznamka}[Green elastic solid evolution equation]
	Solve for $\theta, ¦v, ®B$:
	$$ \rho \frac{d¦v}{dt} = \Div ®T + \rho¦b, \qquad ®T = 2\rho ®B \frac{\partial \psi}{\partial ®B}(\theta, ®B), \qquad \overset{\nabla}{®B} = ®O. $$

	Where is evolution equation for temperature field?
	$$ \rho \theta \frac{d\eta}{dt} = -\Div ¦j_q $$
	$$ -\rho\theta \frac{\partial^2 \psi}{\partial \theta^2}(\theta, ®B) \frac{d\theta}{dt} - \rho\theta\frac{\partial^2 \psi}{\partial \theta\partial ®B}(\theta, ®B) \frac{d®B}{dt} = -\Div¦j_q. $$

	$$ c_V(\theta, ®B) := - \theta \frac{\partial^2 \psi}{\partial \theta^2}(\theta, ®B), $$
	specific heat at constant volume.

	$$ \rho c_V(\theta, ®B) \frac{d\theta}{dt} = - \Div ¦j_q + \rho \theta \frac{\partial^2 \psi}{\partial \theta \partial ®B}:\frac{d®B}{dt}. $$
	We know $\frac{d®B}{dt} = ®L®B + ®B®L^T$, $¦j_q = -k\nabla \theta$.
\end{poznamka}

\begin{poznamka}[Other view]
	$$ \overline{®B} := \frac{®B}{J^{\frac{2}{3}}}, \qquad J = \det ®F, \qquad \implies \det \overline{®B} = 1. $$
	$$ \psi(\theta, ®B) = \psi(\theta, \rho, \overline{®B}) = \psi(\theta, J^2, \overline{®B}). $$
\end{poznamka}

\begin{poznamka}[Some specific formulas]
	(Lagrangian setting: $ ®T_R = (\det ®F) ®T ®F^{-T}$.)

	$\psi(\theta, ®B)$, isotropic elastic solid: $\psi(\theta, ®B) = \frac{\mu}{2\rho}(\tr ®B - 3)$. Incompressible solid ($\det ®F = 1$) $\rho = \rho_R$.

	$$ ®T = 2\rho ®B \frac{\partial \psi}{\partial ®B} $$
	$$ ®T = 2\rho ®B \frac{\mu}{2\rho} ®I = \mu ®B $$
	$$ ®T = -p®I + \mu(®B - ®I). $$
\end{poznamka}

\begin{poznamka}[Boundary value problems]
	$$ (\rho \det ®F = \rho_R) $$
	$$ \rho_R \frac{\partial^2 \chi}{\partial t^2} = \Div ®T_R + \rho_R ¦b $$
	$$ (®T_R = (\det ®F)®T®F^{-T}, \qquad ®T = 2\rho ®B \frac{\partial \psi}{\partial ®B}) $$

	We have this problem, but we must specify initial conditions:
	$$ \chi|_{t = t_0} = \chi_0, \qquad \frac{\partial \chi}{\partial t}|_{t=t_0} = ¦V_0 \text{ on }\Omega_0 $$
	and boundary conditions. There are two options for b. conditions:
	\begin{itemize}
		\item (displacement boundary condition) $\chi(t, ¦X)_{¦X \in \partial \Omega_0} = \chi_B(t, ¦X)$.
		\item (traction boundary condition = we specify forces on the boundary) for example no-traction: $®T¦n |_{¦x \in \partial \Omega} = ¦o$ ($(\det ®F) ®T ®F^{-T}¦N|_{¦X = \chi^{-1}(\partial \Omega)} = ¦o$ $\implies$ $®T_R ¦N|_{¦X \in (\partial \Omega_0)}$).

			$$ ®T(¦x, t)¦n(¦x, t)|_{¦x \in \partial \Omega} = ¦f(¦x, t) $$
			$$ ®T(¦x, t)¦n(¦x, t)|_{¦x \in \partial \Omega} ds = ¦f(¦x, t) ds $$
			$$ ®T_R(¦X, t)|_{¦X \in \partial\Omega_0} ¦NdS = ¦f(\chi(¦X, t), t)\det ®F(¦X, t) |®F^{-T}(¦X, t)¦N| dS $$
			$$ ®T_R(¦X, t)|_{¦X \in \partial\Omega_0} ¦N = ¦g(\chi(¦X, t), t) = ¦g(¦X, t) $$
			this is called dead load (= it doesn't depend on deformation).
	\end{itemize}
\end{poznamka}

% 30. 11. 2022

\begin{poznamka}[Linearized setting]
	We want to simplify the governing equations. (We want to linearise them.)

	We can't talk about small quantity if it has units (it's small only in some units). So we want dimensionless (without unit) quantity. Fortunately we have one – $\nabla_{¦X}\chi$. We introduce displacement: $¦U = ¦x - ¦X = \chi(¦X, t) - ¦X$ (but it has units), so small quantity will be $|\nabla_{¦X}¦U| \ll 1$.

	$$ ®F = \frac{\partial \chi}{\partial ¦X} \implies \nabla_{¦X} ¦U = ®F - ®I \implies ®F = ®I + \nabla_{¦X} ¦U. $$
	$$ \det ®F = \det (®I + \nabla_{¦X}) ≈ 1 + \tr(\nabla_{¦X}¦U) $$

	How to say that the material is incompressible: $\rho_R = \rho \det ®F \implies \det ®F = 1$. Linear setting: $1 + \tr(\nabla_{¦X}¦U) = 1 \implies \tr(\nabla_{¦X}¦U) = 0$.

	$$ \rho_R ≈ \rho(1 + \tr(\nabla_{¦X}¦U)). $$

	$$ ®B = ®F ®F^T = (®I + \nabla_{¦X}¦U)(®I + \nabla_{¦X}¦U)^T ≈ ®I + \nabla_{¦X} ¦U + (\nabla_{¦X} ¦U)^T, $$
	small/infinitesimal strain tensor: $\bepsilon := \frac{1}{2}\(\nabla_{¦X} ¦U + (\nabla_{¦X} ¦U)^T\)$,
	$$ ®B ≈ ®I + 2\bepsilon. $$
	
	$$ ®T = f(®B - ®I) ≈ \lambda (\tr \bepsilon)®I + 2 \mu \bepsilon.z $$

	Warning: $®B ≈ ®I + 2\bepsilon$ may be dangerous for example with $¦x = ®Q¦X$ (rotation: $®T = ®O$ but $®T \not≈ ®O$). But $\bepsilon = ®O$ for $¦x = ¦k \times ¦X$ for constant ¦k, this approximate small rotations.

	For isotropic elastic solids:
	$$ ®F^{-1} = (®I + \nabla_{¦X}¦U)^{-1} ≈ ®I - \nabla_{¦X}¦U. $$

	$$ ®T_R = (\det ®F) ®T ®F^{-1} ≈ (1 + \tr \nabla_{¦X}¦U)®T(®I - \nabla_{¦X}¦U) ≈ ®T (≈ \lambda \tr \bepsilon ®I + 2\mu\bepsilon \text{ for isotropic elastic solid}) $$

	We will write $\btau ≈ ®T_R ≈ ®T$.

	$$ \rho_R(¦X) \frac{\partial^2 ¦U}{\partial t^2} ≈ \Div \btau(¦X, t) + \rho_R(¦X)¦b(¦X, t). $$

	Boundary conditions: $¦U(¦X, t) = ¦U_0(¦X, t)$ for $¦X \in \partial \Omega$ (displacement boundary condition). $\btau ¦N = ¦g(¦X, t)$.
\end{poznamka}

\begin{definice}
	Parameters from linearised elasticity: $\btau = \lambda (\tr \bepsilon)®I + 2\mu \bepsilon$ are called Lamé parameters.

	But Young modulus – $E = \frac{\mu(3\lambda + 2\mu)}{\lambda + \mu}$ – and Poisson ratio – $\gamma = \frac{\lambda}{2(\lambda + \mu)}$ – are more used.
\end{definice}

\begin{poznamka}[Solve $\btau = \lambda (\tr \bepsilon)®I + 2\mu \bepsilon$ for $\bepsilon$]
	We take trace:
	$$ \tr \btau = 3\lambda \tr \bepsilon + 2\mu \tr \bepsilon \implies \tr \bepsilon = \frac{1}{3\lambda + 2\mu} \tr \btau $$

	And replace
	$$ \btau = \lambda\(\frac{1}{3\lambda + 2\mu} \tr \btau\) + 2\mu \bepsilon \implies \bepsilon = \frac{1}{E}\((1 + \gamma)\btau - \nu(\tr \btau)®I\) = $$
	$$ = \frac{1}{\mu}(\btau - \frac{\lambda}{3\lambda + 2\mu}(\tr \btau)®I) = \frac{1}{9\(\lambda + \frac{2}{3}\mu\)}(\tr \btau)®I + \frac{1}{2\mu}\btau_\delta, $$
	where $\btau_\delta := \(\btau - \frac{1}{3}(\tr \btau)®I\)$.

	$$ K := \lambda + \frac{2}{3}\mu \land G := \mu \implies \bepsilon \frac{1}{9K}(\tr \btau)®I + \frac{1}{2G}\btau_\delta. $$
	$K$ is bulk modulus, $G$ is shear modulus.
\end{poznamka}

\begin{poznamka}
	Compressive stress: $\btau = -p®I$. It's compression, so $K>0$.

	Shear stress: $\btau = \begin{bmatrix} 0 & \tau \\ 0 & 0 \end{bmatrix}$. We want sheer in the direction of force ($\tau$), so $G > 0$.

	$G = \mu = \frac{E}{2(\gamma + 1)} > 0$, $E = \frac{9K\mu}{\mu + 3K} > 0$, $K = \frac{E}{3(1 - 2\gamma)}$, $\gamma = \frac{3K - 2\mu}{2(3K + \mu)}$. So $\gamma > -1$ and $\gamma ≤ \frac{1}{2}$ (equality implies $K = +∞$, $\tr \bepsilon = 0$, $\tr \nabla_{¦X}¦U = 0$, so material is incompressible).
\end{poznamka}

\begin{priklad}[Extension of right circular cylinder]
	We take cylinder of radius $R$ and length $L$ and we apply force $F$ to both sides.

	We will ignore difference between current and reference configuration.

	For $(x, y, z) \in \{(x, y, z) \in ®R^3 |x^2 + y^2 < ®R^2, z \in (0, L)\}$. Static problem: $\frac{\partial^2 ¦U}{\partial t^2} = 0$, from $\rho \frac{\partial^2 ¦U}{\partial t^2} = \Div \btau + \rho_R ¦b$: $\Div \btau = ®O$. $\btau¦n|_{x^2 + y^2 = R^2} = ¦o, ¦n = ¦e_{\hat{r}}$. $\btau ¦n|_{z\in \{L, 0\}} = ±\frac{¦F}{S} = ±\frac{¦F}{\pi R^2}$, $¦n = ±¦e_{\hat{z}}$.

	But this isn't well defined, we have only gradients of ¦U. So we must add for example $¦U|_{0, 0, 0} = 0$.

	$$ \Div \btau = \begin{pmatrix} \frac{\partial \tau^{\hat{r}\hat{r}}}{\partial r} + \frac{1}{r}\(\frac{\partial \tau^{\hat{r}\hat{\phi}}}{\partial \phi} - \tau^{\hat{\phi}\hat{\phi}} + \tau^{\hat{r}}\hat{r}\) + \frac{\partial \hat{\phi}\hat{z}}{} \\ \frac{\partial \tau^{\hat{\phi}\hat{r}}}{\partial \phi} + \frac{1}{r} \(\frac{\partial \tau^{\hat{\phi}\hat{\phi}}}{\partial \phi} + \tau^{\hat{r}\hat{\phi}} + \tau^{\hat{\phi}\hat{r}}\) + \frac{\partial \tau^{\hat{\phi}\hat{z}}}{\partial \phi} \\ \frac{\partial \tau^{\hat{r}\hat{z}}}{\partial r} + \frac{1}{z}\(\frac{\partial \tau^{\hat{z} \hat{\phi}}}{\partial \phi} + \tau^{\hat{z}}\hat{r}\) + \frac{\partial \tau^{\hat{z}\hat{z}}}{\partial z} \end{pmatrix} $$

	First boundary condition: $\tau^{\hat{r}\hat{r}} = \tau^{\hat{\phi}\hat{r}} = \tau^{\hat{z}\hat{r}} = 0$. So let's assume, that this is zero everywhere. Moreover we have some symmetry:
	$$ \btau = \begin{pmatrix} 0 & 0 & 0\\ 0 & 0 & 0 \\ 0 & 0 & \tau^{\hat{z}\hat{z}} \end{pmatrix}. $$

	From second boundary condition we have $\tau^{\hat{z}\hat{z}} = \frac{F}{S}$ on both ends. So we can assume that $\tau^{\hat{z}\hat{z}} = \const = \frac{F}{S}$.
	$$ \btau = \begin{pmatrix} 0 & 0 & 0\\ 0 & 0 & 0 \\ 0 & 0 & \frac{F}{S} \end{pmatrix}. $$

	Now we must find the displacement!
	$$ \bepsilon = \frac{1}{2\mu}\(\delta_{3i}\delta_{3j} \frac{F}{S} - \frac{\lambda}{3\lambda + 2\mu} \frac{F}{S} ®I\) $$
	$$ \bepsilon = \begin{pmatrix} \frac{\partial U^{\hat{r}}}{\partial r} & \frac{1}{2}\(\frac{1}{r}\frac{\partial U^{\hat{r}}}{\partial \phi} + \frac{\partial U^{\hat{\phi}}}{\partial r} - \frac{U^{\hat{\phi}}}{r}\) & \frac{1}{2}\(\frac{\partial U^{\hat{r}}}{\partial z} + \frac{\partial U^{\hat{z}}}{\partial r}\) \\ … & \frac{1}{r}\frac{\partial U^{\hat{\phi}}}{\partial \phi} + \frac{U^{\hat{r}}}{r} & \frac{1}{2}\(\frac{\partial U^{\hat{\phi}}}{\partial z} + \frac{\partial U^{\hat{z}}}{\partial z} + \frac{\partial U^{\hat{z}}}{\partial \phi}\) \\ … & … & \frac{\partial U^{\hat{z}}}{\partial z} \end{pmatrix}. $$
	Cylinder is not rotating, so $\frac{\partial}{\partial \phi}$ are zeros. So
	$$ \frac{\partial U^{\hat{r}}}{\partial r} = \frac{1}{2\mu}\(-\frac{\lambda}{3\lambda + 2\mu}\) \frac{F}{S}r, \qquad \frac{U^{\hat{r}}}{r} = \frac{1}{2\mu}\(-\frac{\lambda}{3\lambda + 2\mu}\)\frac{F}{S}, \qquad \frac{\partial U^{\hat{z}}}{\partial z} = \frac{1}{2\mu}\(\frac{2(\lambda + \mu)}{3\lambda + 2\mu}\)\frac{F}{S}. $$


	$U^{\hat{r}}$ we have, $U^{\hat{\phi}} = 0$ and $U^{\hat{z}} = \frac{\lambda + \mu}{\mu(3\lambda + 2\mu)}\frac{F}{S}z$.

	We have all:
	$$ \Delta L = U^{\hat{z}}|_{z = L} = \frac{\lambda + \mu}{\mu(3\lambda + 2\mu)} L \frac{F}{S}, $$
	$$ \frac{\Delta L}{L} = \frac{\lambda + \mu}{\mu(3\lambda + 2\mu)}\frac{F}{S}. $$
	We get $(\frac{F}{S} =) \sigma = E \epsilon (= \frac{\mu(3\lambda + 2\mu)}{\lambda + \mu})$ – Hooke's law.

	$$ \Delta R = U^{\hat{r}}|_{r = R} = -\frac{1}{2\mu}\frac{\lambda}{3\lambda + 2\mu}\frac{F}{S}R, $$
	$$ -\frac{\frac{\Delta R}{R}}{\frac{\Delta L}{L}} = + \frac{\frac{1}{2\mu}\frac{\lambda}{3\lambda + 2\mu} \frac{F}{S}}{\frac{\lambda + \mu}{\mu(3\lambda + 2\mu)}} \frac{F}{S} = \frac{\lambda}{2(\lambda + \mu) = \nu.} $$

	So $\nu < 0$ implies that if we elongate cylinder, it will expand (which is exotic, but possible).
\end{priklad}

\begin{priklad}[Elastic waves]
	We are interested in a wave-like solution $¦U = \underbrace{¦A}_{amplitude} \sin\(\underbrace{¦K}_{\text{wave vector}}·¦X - \underbrace{\omega}_{\text{angular frequency}} t\)$.
	$$ \rho_R \frac{\partial^2 ¦U}{\partial t^2} = (\lambda + \mu)\nabla (\Div ¦U) + \mu \Delta ¦U. $$
	$$ \nabla ¦U = ¦A \cos\(¦K·¦X - \omega t\) \nabla (¦K·¦X) = ¦A \cos\(¦K·¦X - \omega t\)\otimes ¦K. $$

	$$ \rho_R \omega^2 ¦A = (\lambda + \mu)(¦A·¦K)¦K + \mu(¦K·¦K)¦A. $$
	$$ c^2 = \frac{\omega^2}{K^2}\qquad \rho_R c^2 ¦A = (\mu ®I + (\lambda + \mu) \frac{¦K}{K} \otimes \frac{¦K}{K})¦A $$
	Eigenvalues problem:
	$$ \rho_R c^2 ¦A = \[\mu(®I - \frac{¦K \otimes ¦K}{K^2}) + (\lambda + 2\mu) \frac{¦K \otimes ¦K}{K^2}\]¦A. $$
	$$ ¦A = ¦K_\perp \implies c_\perp = \sqrt{\frac{\mu}{\rho_R}}, $$
	$$ ¦A = ¦K_{||} \implies c_{||} = \sqrt{\frac{\lambda + 2\mu}{\rho_R}}. $$
\end{priklad}

% 07. 12. 2022

\begin{poznamka}[When $\bepsilon = \frac{1}{2}\(\nabla ¦u + (\nabla ¦u)^T\)$?]
	Naive approach: 2D settings: Suppose that $\bepsilon = \frac{1}{2}\(\nabla ¦u + (\nabla ¦u)^T\)$. If $\bepsilon$ is symmetric, then
	$$ \frac{\partial^2 \epsilon_{11}}{\partial x_2^2} + \frac{\partial^2 \epsilon_{22}}{\partial x_1^2} - 2 \frac{\partial^2 \epsilon_{12}}{\partial x_1 \partial x_2} = \frac{\partial^2}{\partial x_2^2}\frac{\partial u^1}{\partial x_1} + \frac{\partial^2}{\partial x_1^2} \frac{\partial u^2}{\partial x_2} - \frac{\partial^2}{\partial x_1 \partial x_2} \(\frac{\partial u^1}{\partial x_2} + \frac{\partial u^2}{\partial x_1}\) = 0. $$
	This is necessary condition for it.

	\hrule

	Systematic approach: We know how to check and construct $¦v = -\nabla \phi$.

	Given ®F check whether it is possible to find $\chi$ such that $®F = \nabla \chi$. Choose $¦w \in ®R^3$, arbitrary fixed vector. Can we find $\phi_{¦w}$ such that $®F^T ¦w = \nabla \phi_{¦w}$.

	So condition is $\rot(®F^T ¦w) = ¦o$. We know (it's definition) $(\rot ®A)^T ¦v = \rot(®A^T ¦v)$ $\forall ¦v$ constant vector. So we can rewrite condition as $(\rot ®F)^T = ®O$.

	From linearity + invariance: $\phi_{¦w} = \chi · ¦w$.
	$$ ®F^T ¦w = \nabla \phi_{¦w} = \nabla (\chi·¦w) = (\nabla \chi)^T¦w, \qquad ®F = \nabla \chi. $$

	$$ ¦v := ®F^T ¦w = \nabla \phi_{¦w}, \qquad \phi_{¦w}(¦x) = \int_{¦x_0}^{¦x} ®F^T ¦w · d¦l = ¦w · \int_{¦x_0}^{¦x} ®H d ¦l $$
	$$ \phi_{¦w}(¦x) = \chi(¦x)·¦w \implies \chi = \int_{¦x_0}^{¦x} ®F d¦l, \qquad ®F = \nabla \chi, \rot ®F = ®O. $$

	$\bepsilon$ is generated as the symmetric gradient:
	$$ \nabla ¦u = \underbrace{\bepsilon}_{\text{symmetric part}} + \underbrace{@\omega}_{\text{skew-symmetric part}} \implies @\omega = \frac{1}{2}\(\nabla ¦u - (\nabla ¦u)^T\). $$
	$$ \rot(\nabla ¦u) = \rot \bepsilon + \rot @\omega, \qquad \O = \rot \bepsilon + \rot @\omega, $$
	$$ \rot \bepsilon = -\rot @\omega, \qquad \rot \bepsilon = \frac{1}{2} \(\nabla (\rot ¦u)\)^T. $$
	We see that $(\rot \bepsilon)^T$ is generated as the full gradient of something. Is this equation solvable? Solvability condition: $\rot((\rot \bepsilon)^T) = \O$.

	How to find $@\omega$? $\omega$ is skew symmetric matrix, so it has axial vector: $@\omega = ®A_{¦a}$, $¦a$ denotes axial vector.
	$$ \nabla ¦u = \bepsilon + @\omega = \bepsilon + ®A_{¦a}. $$
	Can we solve this equation for ¦u? Solvability conditions: $\rot (\bepsilon + ®A_{¦a}) = ®O$.
	$$ \rot \bepsilon + \rot ®A_{¦a} \stackrel?= ®O $$
	$$ LHS = (\nabla ¦a)^T + \((\Div ¦a) ®I - (\nabla ¦a)^T\) = (\Div ¦a)®I \Leftrightarrow \Div¦a \stackrel?= 0. $$

	$$ \Div ¦a = \nabla · ¦a = \tr(\nabla ¦u) = \tr (\rot \bepsilon) = \epsilon_{jkl} \frac{\partial \epsilon_{jl}}{\partial x_k} = 0. $$
	So we can solve $\nabla ¦u = \bepsilon ®A_{¦a}$ for ¦u.

	Conclusion: Is $\bepsilon$ generated as the symmetric gradient of ¦u? Compatibility condition: $\rot((\rot \bepsilon)^T) = ®O$.
\end{poznamka}

\begin{poznamka}
	Same question about $®C = ®F^T®F$.
\end{poznamka}

\begin{poznamka}[What is this good for?]
	Assume that we want to solve state problem in linearised elasticity
	$$ ¦o = \Div \btau + \rho ¦f $$
	and the boundary conditions are traction boundary conditions, $\btau ¦n |_{\partial \Omega} = ¦b$.

	We want to find stress field $\btau$ in $\Omega$
	$$ \bepsilon = \frac{1}{E} ((1 + \nu) \btau - \nu(\tr \btau)®I). $$
	$\btau$ is the linearised strain $\implies$ $\btau = \frac{1}{2}(\nabla ¦u + (\nabla ¦u)^T)$, $\rot((\rot \bepsilon)^T) = ®O$.

	$$ \bepsilon = \frac{1}{E}((1 + \nu)\btau - \nu(\tr \btau)®I) \land \rot((\rot \bepsilon)^T) = ®O. $$
	This gives us a system of equations for $\btau$.
\end{poznamka}

\begin{priklad}
	Inflation of a hollow cylinder. Inner radius $R_{in}$ (or $r_{in}$), outer radius $R_{out}$ (or $r_{out}$). And there is some pressure: inner $P_{in}$ and outer $P_{out}$. We want find $c = \frac{r_{in}^2 - R_{in}^2}{R_{in}^2}$.

	Two theories: Linear and Finite elasticity.
\end{priklad}

\begin{priklad}[Linear theory]
	Material:
	$$ ®F = ®I + \nabla ¦u, \qquad ®B = ®F®F^T \sim ®I + (\nabla ¦u) + (\nabla ¦u)^T + … $$
	$$ \btau = -p®I + 2\mu\bepsilon. $$
	Assume incompressible material: $1 = \det ®F = \det (®I + \nabla ¦u) \sim 1 + \tr(\nabla ¦u)$ $\implies$ $\tr \bepsilon = 0$.
	
	Boundary conditions: $\btau ¦E_{\hat{R}} |_{R = R_{out}} = -P_{out}¦E_{\hat{R}}$, $\btau ¦E_{\hat{R}} |_{R = R_{in}} = -P_{in}¦E_{\hat{R}}$.

	Geometry: Assume $¦u = (g(R), 0, 0)^T$.
	$$ 0 = \tr \bepsilon = \tr (\nabla ¦u) \implies \frac{dg}{dR} + \frac{g}{R} = 0 \implies g(R) = \frac{D}{R}. $$
	$$ g(R_{in}) = r_{in} - R_{in} \implies D =  $$
	
	Force blance (balance of linear momentum):
	$$ \Div \btau = ¦o, \qquad R \in (R_{in}, R_{out}), \Phi \in (0, 2\pi], Z \in … $$

	$$ \btau = -p®I + \begin{pmatrix} S^{\hat{R}\hat{R}} & 0 & 0 \\ 0 & S^{\hat{\Phi}\hat{\Phi}} & 0 \\ 0 & 0 & 0 \end{pmatrix}. $$
	$$ \Div ®T = ¦o \implies \frac{d\tau^{\hat{R}\hat{R}}}{dR} + \frac{1}{R}(\tau^{\hat{R}\hat{R}} - \tau^{\hat{\Phi}\hat{\Phi}}) = 0. $$
	$$ … $$
	$$ \int_{R=R_{in}}^{R_{out}} \frac{S^{\hat{R}\hat{R}} - S^{\hat{\Phi}\hat{\Phi}}}{R} dR = -(P_{out} - P_{in}). $$
	
	Use constitutive relation:
	$$ \btau = -p®I + 2\mu \bepsilon, \qquad \bepsilon = \frac{1}{2}(…). $$
	$$ \btau = -p®I + 2\mu \begin{pmatrix} \frac{du^{\hat{R}}}{dR} & 0 & 0 \\ 0 & \frac{u^{\hat{R}}}{R} & 0 \\ 0 & 0 & 0 \end{pmatrix}.  $$
	$$ 2\mu \int_{R = R_{in}}^{R_{out}} \frac{\(\frac{du^{\hat{R}}}{dR}\)^2 - \(\frac{u^{\hat{R}}}{R}\)^2}{R} dR = -(P_{out} - P_{in}). $$
	$$ u^{\hat{R}} = \frac{D}{R} \implies P_{out} - P_{in} = -\mu \int_{R=R_{in}}^{R_{out}} \frac{4D}{R^3} dR $$
	$$ c = \frac{r_{in}^2 - R_{in}^2}{R_{in}^2} = \frac{(R_{in} + u^{\hat{R}}(R_{in}))^2 - R_{in}^2}{R_{in}^2} = \frac{2D + \frac{D^2}{R_{in}^2}}{R_{in}^2}. $$
	Solve for D and substitute.
	$$ D = \frac{-2R_{in}^2 ± \sqrt{4 R_{in}^4 + 4c R_{in}^4}}{2} \sim \frac{R_{in}^2c}{2}. $$
	$$ P_{out} - P_{in} = -\mu \int_{R = R_{in}}^{R_{out}} \frac{2 R_{in}^2}{R^3} c dR. $$
\end{priklad}

\begin{priklad}[Finity elasticity]
	Material:
	$$ ®T = -p ®I + \mu(®B - ®I). $$
	Assume incompressible material: $\det ®F = 1$.

	Boundary conditions: $®T ¦e_{\hat{r}} |_{r = r_{out}} = -P_{out}¦e_{\hat{r}}$, $®T ¦e_{\hat{r}} |_{r = r_{in}} = -P_{in}¦e_{\hat{r}}$.

	Geometry: Assume that $r = f(R)$, $\phi = \Phi$, $z = Z$. So (was homework)
	$$ ®F = \begin{pmatrix} \frac{df}{dR} & 0 & 0 \\ 0 & \frac{f}{R} & 0 \\ 0 & 0 & 1 \end{pmatrix}. $$
	$$ 1 = \det ®F = \frac{df}{dR} \frac{1}{R} \implies f \frac{df}{dR} = R \implies f(R) = \sqrt{R^2 + C_1}. $$
	$$ f(R_{in}) = r_{in} \implies f(R) = \sqrt{R^2 + cR_{in}^2}. $$

	Force blance (balance of linear momentum):
	$$ \Div ®T = ¦o, \qquad r \in (r_{in}, r_{out}), \phi \in (0, 2\pi], z \in … $$
	$$ ®T = -p®I + \begin{pmatrix} S^{\hat{r}\hat{r}} & 0 & 0 \\ 0 & S^{\hat{\phi}\hat{\phi}} & 0\\ 0 & 0 & 0 \end{pmatrix} $$
	Find formula for $\Div ®T$ in cylindrical coordinates.
	$$ \Div ®T = ¦o \implies \frac{d T^{\hat{r}\hat{r}}}{dr} + \frac{1}{R}\(T^{\hat{r}\hat{r}} - T^{\hat{\phi}\hat{\phi}}\) = 0 \implies $$
	$$ \implies -\frac{dp}{dr} + \frac{dS^{\hat{r}\hat{r}}}{dr} + \frac{S^{\hat{r}\hat{r}} - S^{\hat{\phi}\hat{\phi}}}{r} = 0 \implies $$
	$$ \implies p(r) = \int_{r=r_{in}}^r \(\frac{dS^{\hat{r}\hat{r}}}{dr} + \frac{S^{\hat{r}\hat{r}} - S^{\hat{\phi}\hat{\phi}}}{r}\) dr + P|_{r = r_{in}}. $$
	$$ \begin{pmatrix} -p + S^{\hat{r}\hat{r}} & 0 & 0 \\ 0 & -p + S^{\hat{\phi}\hat{\phi}} & 0 \\ 0 & 0 & -p \end{pmatrix} \begin{pmatrix} 1 \\ 0 \\ 0 \end{pmatrix} = \begin{pmatrix} -P_{out} \\ 0 \\ 0 \end{pmatrix} \implies -p|_{r = r_{out}} + S^{\hat{r}\hat{r}}|_{r=r_{out}} = -P_{out}, $$
	$$ … \implies p|_{r = r_{in}} + S^{\hat{r}\hat{r}}|_{r = r_{in}} = -P_{in}. $$
	$$ \int_{r = r_{in}}^{r_{out}} \(\frac{dS^{\hat{r}\hat{r}}}{dr} + \frac{S^{\hat{r}\hat{r}} - S^{\hat{\phi}\hat{\phi}}}{r}\) dr + p|_{r = r_{in}} + S^{\hat{r}\hat{r}}|_{r = r_{out}} = -P_{out}, \qquad p|_{r=r_{in}} + S^{\hat{r}\hat{r}}|_{r = r_{in}} = -P_{in}. $$
	Subtract:
	$$ \int_{r = r_{in}}^{r_{out}} \frac{S^{\hat{r}\hat{r}} - S^{\hat{\phi}\hat{\phi}}}{r} dr = -(P_{out} - P_{in}). $$

	Use constitutive relation:
	$$ ®B = ®F®F^T = \begin{pmatrix} \(\frac{df}{dR}\)^2 & 0 & 0 \\ 0 & \(\frac{f}{R}\)^2 & 0 \\ 0 & 0 & 1 \end{pmatrix} $$
	$$ \mu\int_{r_{in}}^{r_{out}} \frac{\(\frac{df}{dR}\)^2 - \(\frac{f}{R}\)^2}{r} dr = -(P_{out} - P_{in}) $$
	$$ \frac{f}{R} = \frac{r}{\sqrt{r^2 - cR_{in}^2}}, \qquad \frac{df}{dR} \frac{f}{R} = 1 \implies \frac{df}{dR} = \frac{R}{f}, $$
	$$ \mu\int_{r=\sqrt{R_{in}^2 + cR_{in}^2}}^{\sqrt{R_{out}^2 + cR_{in}^2}} \frac{(r^2 - c R_{in}^2)^2 - r^4}{r^3(r^2 - cR_{in}^2)} dr = P_{out} - P_{in}. $$

	(After linearization of this formula we get result from linear theory.)
\end{priklad}
	
\end{document}
