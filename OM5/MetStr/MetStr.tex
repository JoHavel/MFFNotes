\documentclass[12pt]{article}					% Začátek dokumentu
\usepackage{../../MFFStyle}					    % Import stylu



\begin{document}

% 04. 10. 2022
\section*{Úvod}
\begin{poznamka}[Historie]
	MP zavedl Maurice Fréchet na podnět Felixe Hausdorffa.
\end{poznamka}

\begin{poznamka}
	Dále se opakovali metrické prostory.
\end{poznamka}

\begin{definice}[Baireův prostor]
	$®N^{®N}$, $d(\{x_n\}, \{y_n\}) = \frac{1}{k}$, kde $k$ je první index, že $x_k ≠ y_k$.
\end{definice}

\begin{poznamka}
	V Bairově prostoru platí $d(x, y) ≤ \max(d(x, z), d(z, y))$. Metriky s touto vlastností se nazývají ultrametriky (dříve archimédovské metriky).
\end{poznamka}

\begin{definice}[Peadická metrika]
	$(Q, d_p)$, kde $p$ je prvočíslo:
	$$ d_p(a, b) = p^{-n}, \frac{a}{b} = p^n · c. $$
\end{definice}

\begin{definice}[Stejnoměrně ekvivalentní]
	Metriky jsou stejnoměrně ekvivalentní, jestliže identická zobrazení ($(X, d) \mapsto (X, e)$ a opačně) jsou stejnoměrně spojitá.
\end{definice}

\begin{definice}[Hölderovské zobrazení]
	Nechť $\alpha ≥ 0$. Říkáme, že zobrazení $f: (X, d) \rightarrow (Y, e)$ je hölderovské stupně $\alpha$ (nebo $\alpha$-hölderovské), jestliže existuje $k \in ®R$ tak, že pro všechna $x, y \in X$ platí
	$$ e(f(x), f(y)) ≤ k·d^\alpha(x, y) $$
	Hölderovské zobrazení stupně 1 se nazývá lipschitzovské. Lipschitzovské zobrazení s konstantou $k < 1$ se nazývá kontrakce.
\end{definice}



\begin{tvrzeni}
	Je-li $f: (X, d) \rightarrow (Y, e)$ $\alpha$-hölderovské pro.
\end{tvrzeni}

\end{document}
