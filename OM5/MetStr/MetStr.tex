\documentclass[12pt]{article}					% Začátek dokumentu
\usepackage{../../MFFStyle}					    % Import stylu



\begin{document}

% 04. 10. 2022
\section*{Úvod}
\begin{poznamka}[Historie]
	MP zavedl Maurice Fréchet na podnět Felixe Hausdorffa.
\end{poznamka}

\begin{poznamka}
	Dále se opakovali metrické prostory.
\end{poznamka}

\begin{definice}[Baireův prostor]
	$®N^{®N}$, $d(\{x_n\}, \{y_n\}) = \frac{1}{k}$, kde $k$ je první index, že $x_k ≠ y_k$.
\end{definice}

\begin{poznamka}
	V Bairově prostoru platí $d(x, y) ≤ \max(d(x, z), d(z, y))$. Metriky s touto vlastností se nazývají ultrametriky (dříve archimédovské metriky).
\end{poznamka}

\begin{definice}[Peadická metrika]
	$(Q, d_p)$, kde $p$ je prvočíslo:
	$$ d_p(a, b) = p^{-n}, \frac{a}{b} = p^n · c. $$
\end{definice}

\begin{definice}[Stejnoměrně ekvivalentní]
	Metriky jsou stejnoměrně ekvivalentní, jestliže identická zobrazení ($(X, d) \mapsto (X, e)$ a opačně) jsou stejnoměrně spojitá.
\end{definice}

\begin{definice}[Hölderovské zobrazení]
	Nechť $\alpha ≥ 0$. Říkáme, že zobrazení $f: (X, d) \rightarrow (Y, e)$ je hölderovské stupně $\alpha$ (nebo $\alpha$-hölderovské), jestliže existuje $k \in ®R$ tak, že pro všechna $x, y \in X$ platí
	$$ e(f(x), f(y)) ≤ k·d^\alpha(x, y) $$
	Hölderovské zobrazení stupně 1 se nazývá lipschitzovské. Lipschitzovské zobrazení s konstantou $k < 1$ se nazývá kontrakce.
\end{definice}

% 11. 10. 2022

\begin{tvrzeni}
	Je-li $f: (X, d) \rightarrow (Y, e)$ $\alpha$-hölderovské pro $\alpha > 0$, pak je $f$ stejnoměrně spojité.

	\begin{dukazin}
		Máme
		$$ e(f(x), f(y)) ≤ K d^\alpha(x, y). $$
		Chceme
		$$ \forall \epsilon > 0\ \exists \delta > 0: (d(x, y) < \delta \implies e(f(x), f(y)) < \epsilon). $$
		Zvolíme $\delta = \sqrt[\alpha]{\frac{\epsilon}{K}}$.
	\end{dukazin}
\end{tvrzeni}

\begin{tvrzeni}
	Složení $f$ $\alpha$-hölderovské a $g$ $\beta$-hölderovské je $\alpha·\beta$-hölderovská funcke.

	\begin{dukazin}
		$$ \rho(g(f(x)), g(f(y))) ≤ Ke^\beta(f(x), f(y)) ≤ K(L d^\alpha(x, y))^\beta = K·L^\beta d^{\alpha·\beta}(x, y). $$
	\end{dukazin}
\end{tvrzeni}

\begin{dusledek}
	Složení lipschitzovských zobrazení je lipschitzovské. Z důkazu pak i složení kontrakcí je kontrakce.
\end{dusledek}

\begin{definice}[Pseudonorma funkce]
	Nechť $f: (X, d) \rightarrow (Y, e)$. Označíme
	$$ |f|_\alpha = \inf \{K, e(f(x), f(y)) ≤ K d^\alpha(x, y)\} = \sup \{\frac{e(f(x), f(y))}{d^\alpha(x, y)} | x ≠ y\} $$
\end{definice}

\begin{veta}
	Nechť $(X, d)$ je omezený prostor a $0 ≤ \beta ≤ \alpha$. Pak každé $\alpha$-hölderovské zobrazení na $(X, d)$ je $\beta$-hölderovské.

	\begin{dukazin}
		$$ e(x, y) ≤ K d^\alpha(f(x), f(y)) = K d^\beta(x, y)·d^{\alpha - \beta}(x, y) ≤ K·(\diam f(x))^{\alpha - \beta} d^\beta(x, y). $$
	\end{dukazin}
\end{veta}

\begin{tvrzeni}
	$f: J \rightarrow ®R$, $J$ interval v ®R, $J = (a, b)$, $a ≠ -∞$. Potom
	$$ f\text{ je stejnoměrně spojitá } \implies \exists F: [a, b) \rightarrow ®R, \text{ stejnoměrně spojitá, } F|_{(a, b)} = f. $$

	\begin{dukazin}
		Dokázáno na přednášce (jednoduché).
	\end{dukazin}
\end{tvrzeni}

\begin{veta}
	\ 
	\begin{enumerate}
		\item Je-li $f$ $\alpha$-hölderovská pro $\alpha > 1$ je konstantní.
		\item Má-li $f$ derivaci, pak je $f$ lipschitzovská právě když je její derivace omezená.
		\item Lipschitzovská funkce je absolutně spojitá.
	\end{enumerate}

	\begin{dukazin}
		$$ „1.)“ |f(x) - f(y)| ≤ K|x - y|^\alpha \implies |\frac{f(x) - f(y)}{x - y} ≤ K |x - y|^{\alpha - 1} \implies f'(x) = 0 \implies f = \konst. $$

		$$ „2.) \implies“ \exists f' na J \land |f(x) - f(y)| ≤ K |x - y| \implies |\frac{f(x) - f(y)}{x - y}| ≤ K. $$
		$$ „2.) \impliedby“ |f'(x)| ≤ K \forall x \in J \implies |f(x) - f(y)| = |f'(c)|·|x-y| ≤ K(x - y). $$

		$$ „3.)“ f \text{ abs. spoj.}: \forall \epsilon > 0\ \exists \delta > 0: (\forall a ≤ a_1 < b_1 ≤ a_2 < b_2 ≤ … ≤ a_n < b_n ≤ b: \sum_{i=1}^n (b_i - a_i) < \delta \implies \sum_{i=1}^n |f(b_i) - f(a_i)|), $$
		$$ f\text{ je lips.} \implies |f(x) - f(y)| ≤ K·|x - y| \implies \sum_{i=1}^n |f(a_i) - f(b_i)| ≤ \sum_{i=1}^n K(b_i - a_i) = K \sum_{i=1}^n (b_i - a_i) < K\delta < \epsilon. $$
	\end{dukazin}
\end{veta}

\begin{definice}[Lipschitzovsky ekvivalentní metriky]
	Metriky $d$, $e$ na $X$ se nazývají lipschitzovsky ekvivalentní, jestliže obě identická zobrazení jsou lipschitzovské.
\end{definice}

\begin{poznamka}
	To je ekvivalentní
	$$ \exists K, L > 0\ \forall x, y \in X: K d(x, y) < e(x, y) < L d(x, y). $$
\end{poznamka}

\begin{veta}
	Buď $p > 0$. Funkce $x^p$ na $J \subseteq [0, +∞)$ je lipschitzovská, právě když buď $0 < a < b < +∞$ nebo $a = 0$ a $p ≥ 1$ nebo $b = +∞$ a $p ≤ 1$.
\end{veta}

\begin{veta}
	Nechť $\alpha \in (0, 1]$. Pak $x^\alpha$ je $\alpha$-hölderovská na $[0, +∞)$.

	\begin{dukazin}
		$$ |x^\alpha - y^\alpha| ≤ K |x - y|^\alpha, \qquad y_0 ≥ 0, x ≥ y_0, g(x) = (x - y_0)^\alpha - (x^\alpha - y_0^\alpha) na [y_0, +∞), g(y_0) = 0, $$
		$$ g'(x) = \alpha(x - y_0)^{\alpha - 1} - \alpha x^{\alpha-1} = \alpha\((x - y_0)^{\alpha - 1} - x^{\alpha - 1}\) ≥ 0 \implies \forall x ≥ y_0: x^\alpha - y^\alpha ≤ (x - y)^\alpha. $$
	\end{dukazin}
\end{veta}

\begin{veta}
	$$ |x^\alpha|_\alpha = \frac{b^\alpha - a^\alpha}{(b - a)^\alpha}, \qquad \alpha \in [0, 1]. $$

	\begin{dukazin}
		$\alpha = 0 \implies |x^\alpha|_\alpha = 0$. $a = 0 \stackrel?\implies |x^\alpha|_\alpha = 1$. $b = +∞ \stackrel?\implies |x^\alpha|_\alpha = 1$.
	\end{dukazin}
\end{veta}

\end{document}
