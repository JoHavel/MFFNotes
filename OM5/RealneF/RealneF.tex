\documentclass[12pt]{article}					% Začátek dokumentu
\usepackage{../../MFFStyle}					    % Import stylu



\begin{document}

% 03. 10. 2022
\section*{Úvod}
\begin{poznamka}[Organizační úvod]
	K ukončení předmětu je třeba pouze udělat zkoušku: 2 příklady na definice, 2 věta-důkaz.

	Literatura:
	\begin{itemize}
		\item L.C. Evans, R.F. Gariepy, Measure Theory and Fine Properties of Functions, CRC Press, Boca Raton, 1992.
		\item W. Rudin, Analýza v reálném a komplexním oboru, Academia, 2003.
	\end{itemize}
\end{poznamka}

\section{Differentiation of measures}
\subsection{Covering theorems}

\begin{definice}[Vitali cover]
	Let $A \subset ®R^n$ we say that a system ©V consisting of closed balls from $®R^n$ forms Vitali cover of $A$, if
	$$ \forall x \in A\ \forall \epsilon > 0 \exists B \in ©V: x \in B \land \diam B < \epsilon. $$
\end{definice}

\begin{definice}[Notation]
	$\lambda_n$ is Lebesgue measure on $®R^n$. $\lambda_n^*$ is outer Lebesgue measure on $®R^n$. If $B \subset ®R^n$ is a ball and $\alpha > 0$, then $\alpha · B$ stands for the ball, which is concentric with B and with $\alpha$-times greater radius than $B$.
\end{definice}

\begin{veta}[Vitali]
	Let $A \subset ®R^n$ and ©V be a system of closed balls forming a Vitali cover of $A$. Then there exists a countable disjoint subsystem $©A \subseteq ©V$ such that $\lambda_n(A \setminus \bigcup ©A) = 0$.

	\begin{dukazin}
		First assume that A is bounded. Take an open bounded set $G \subset ®R^n$ with $A \subset G$. We set
		$$ ©V^* = \{B \in ©V | V \subset G\}. $$
		Then $©V^*$ is a Vitali cover of $A$. If there exists a finite disjoint subsystem of $©V^*$ covering $A$, we are done. So Assume that there is no such subsystem. Mathematical induction:

		First step: We set $s_1 = \sup\{\diam B | B \in ©V^*\}$. We choose a ball $B_1 \in ©V^*$ such that $B_1 > \frac{1}{2} s_1$.

		$k$-th step: Suppose that we have already constructed balls $B_1, B_2, …, B_{k-1}$. We set
		$$ s_k = \sup\{\diam B | B \in ©V^* \land B \cap \bigcup_{i=1}^{k-1} B_i = \O\}. $$
		We find $B_k \in ©V^*$ such that $\diam B_k > \frac{1}{2} s_k > 0$, $B_k \cap \bigcup_{i=1}^{k-1} B_i = \O$.

		Let $©A = \{B_k | k \in ®N\}$. It is disjoint, it is countable, it holds $\lambda_n(A \setminus \bigcup ©A) = 0$:

		$$ \sum_{i=1}^∞ \lambda_n (B_i) = \lambda_n(\bigcup_{i=1}^∞ B_i) ≤ \lambda_n(G) < ∞ \implies $$
		$$ \implies \lim_{i \rightarrow ∞} = 0 \implies \lim_{i \rightarrow ∞} \diam(B_i) = 0 \implies \lim_{i \rightarrow ∞} s_i = 0. $$
		We show that
		$$ \forall x \in A \setminus \bigcup ©A\ \forall i \in ®N \exists j \in ®N, j > i: x \in 5 · B_j $$
		$$ \Leftrightarrow A \setminus \bigcup ©A \subseteq \bigcup_{j = i + 1}^∞ 5·B_j $$
		Take $x \in A \setminus \bigcup ©A$ and $i \in ®N$. Denote $\delta = \dist(x, \bigcup_{k=1}^i B_k) > 0$. There exists $B \in ©V^*$ such that $x \in B$ and $\diam B < \delta$ $\implies$ $B \cap \bigcup_{k=1}^i B_k = \O$. Then we have $\diam B > s_p$ for some $p \in ®N$.

		Therefore there exists $j > i$ with $B_j \cap B ≠ \O$. Let $j$ be the smallest number with this property. Then we have $s_j ≥ \diam B$ since $B \cap \bigcup_{l=1}^{j-1} B_l = \O$. Further we have $\diam B_j > \frac{1}{2} s_j ≥ \frac{1}{2} \diam B$ $\implies 2 \diam B_j ≥ \diam B$ This implies that $x \in B \subset 5·B_j$.

		$$ \lambda_n^*(A \setminus \bigcup A) ≤ \lambda_n \(\bigcup_{j=i+1}^∞ 5·B_j\) ≤ \sum_{j=i+1}^∞ \lambda_n (5·B_j) = \sum_{j=i+1}^∞ 5^n \lambda_n(B_j) = 5^n · \sum_{j = i+1}^∞ \lambda_n(B_j) \rightarrow 0 \implies \lambda_n(A \setminus \bigcup ©A) = 0. $$

		
		General case ($A$ not bounded): Let $(G_j)_{j=1}^∞$ be a sequence of disjoint open sets such that $\lambda_n(®R^n \setminus \bigcup_{j=1}^∞ G_j) = 0$. We define $©V_j = \{B \in ©V_i, B \subseteq G_j\}$. $©V_j$ is a Vitali cover of $A \cap G_j$ $\implies \exists ©A_j \subseteq ©V_j$ countable disjoint and $\lambda_n(A \cap G_j \setminus \bigcup A_j) = 0$. We set $©A = \bigcup_{j=1}^∞ ©A_j$. $©A$ is countable, disjoint and $\lambda_n(A \setminus \bigcup ©A) = 0$.
	\end{dukazin}
\end{veta}

% 10. 10. 2022

\begin{definice}
	We say that a measure $\mu$ on $®R^n$ satisfies Vitali theorem, if for every Vitaly cover ©V of $M \subseteq ®R^n$ there exists a disjoint countable $©A \subset ©V$ with $\mu(M \setminus \bigcup ©A) = 0$.
\end{definice}

\begin{poznamka}
	If $\mu$ satisfies Vitali theorem and $\nu \ll \mu$, then $\nu$ satisfies Vitali theroem.
\end{poznamka}

\begin{veta}
	Set $E \subset ®R^n$ be Lebesgue measurable and ©S be a finite system of closed balls covering $E$. Then there exists a disjoint system $©L \subset ®S$ such that $\lambda_n(E) ≤ 3^n · \sum_{B \in ©L} \lambda_n(B)$.

	\begin{dukazin}
		WLOG $©S ≠ \O$. SUppose $B_1 \in ©S$ with maximal radius among balls from ©S.

		Suppose that we have already constructed $B_1, …, B_{k-1} \in ©S$. If possible, choose $B_k \in ©S$ disjoint with $\bigcup_{j < k} B_j$ and with maximal radius among balls satisfying this property.

		We set $©L = \{B_1, …, B_N\}$. We show $E \subseteq \bigcup_{B \in ©L} 3*B = \bigcup_{i=1}^N 3*B_i$. $x \in E$. Find $B \in ©S$ with $x \in B$. Find smallest $k$ with $B \cap B_k ≠ \O$. This means $\rad (B) ≤ \rad (B_k)$ $\implies$ $x \in B \subseteq 3*B_k$.

		Now $\lambda_n(E) ≤ \lambda_n\(\bigcup_{i=1}^N 3*B_i\) ≤ \sum_{i=1}^N \lambda_n(3*B_i) = 3^n \sum_{i=1}^N \lambda_n(B_i)$.
	\end{dukazin}
\end{veta}

\begin{veta}[Besicovitch theorem]
	For each $n \in ®N$ there exists $N \in ®N$ with the following property:

	If $A \subset ®R^n$ and $\Delta: A \rightarrow (0, ∞)$ is a bounded function, then there exist sets $A_1$, …, $A_N$ $\subseteq A$ such that

	\begin{itemize}
		\item $\{\overline{B}(x, \Delta x) | x \in A_j\}$ is disjoint for every $j \in [N]$;
		\item $A \subset \bigcup\{\overline{B}(x, \Delta x) | \in \bigcup_{i=1}^N A_i\}$.
	\end{itemize}

	\begin{dukazin}[Case $A$ is bounded]
		Let $R := \sup_A \Delta$. Choose $B_1 := \overline{B}(a_1, \Delta(a_1))$ such that $a_1 \in A$ and $r_1 := \Delta(a_1) > \frac{3}{4}R$.

		Assume that we already constructed $B_1, …, B_{j-1}$, $j ≥ 2$. $B_{j-1} = \overline{B}(a_{j-1}, \Delta(a_{j-1})) = \overline{B}(a_{j-1}, r_{j-1})$. Let $F_j := A \setminus \bigcup_{i=1}^{j-1} B_i$. If $F_j = \O$ we set $J := j$. If not $B_j := \overline{B}(a_j, \Delta(a_j)) = \overline{B}(a_j, r_j)$, $a_j \in F_j$ and $r_j > \frac{3}{4} \sup_{F_j} \Delta$.

		If $F_j ≠ \O$ for every $j \in ®N$, then we set $J := ∞$. So we have $(B_j)_{j < J}$. If $J < ∞$, then we covered $A$. „If $J = ∞$, then $A \subset \bigcup_{j < J} B_j$“:

		„$\lim_{j \rightarrow ∞} r_j = 0$“: because $A$ is bounded
		$$ ||a_i - a_j|| ≥ r_i = \frac{1}{3} r_i + \frac{2}{3} r_i > \frac{1}{3} r_i + \frac{1}{2} r_j > \frac{1}{3} r_i + \frac{1}{3} r_j = \frac{1}{3}(r_i + r_j) \implies \frac{1}{3} * B_i \cap \frac{1}{3} * B_j = \O. $$
		$\{\frac{1}{3}B_j | j < J\}$ is a disjoint family $\implies \sum_{j=1}^∞ \lambda_n(\frac{1}{3} * B_j) < ∞$.

		If $A \in A \setminus \bigcup_{j=1}^∞ B_j$, then $a \in \bigcap_{j=1}^∞ F_j$. We find $j_0 \in ®N$ with $r_{j_0} ≤ \frac{3}{4} \Delta(a)$. \lightning.

		Fix $k < J$. We set $I = \{i < k | B_i \cap B_k ≠ \O\}$, $I_1 = \{i < k_i | B_i \cap B_k ≠ \O \land r_i < 10 r_k\}$, $I_2 = \{i < k_i | B_i \cap B_k \land r_i ≥ 10 r_k\}$. The estimate of $I_1$: „We have $\frac{1}{3} B_i \subseteq 15 * B_k$ for every $i \in I_1$“: Take $x \in \frac{1}{3} * B_i$. Then
		$$ ||x - a_k|| ≤ ||x - a_j|| + ||a_i - a_k|| ≤ \frac{1}{3} r_i + r_i + r_k ≤ \frac{10}{3} r_k + 10 r_k + r_k ≤ 15 r_k $$

		$$ \lambda_n(\frac{1}{3} * B_i) = \lambda(\overline{B}(0, 1))·(\frac{1}{3}r_i)^n ≥ \lambda_n(\overline{B}(0, 1)) · (\frac{1}{3}·\frac{3}{4} r_k)^n = \lambda_n(\overline{B}(0, 1))·\frac{1}{4^n} r_k^n = \frac{1}{6O^n} \lambda_n(15 * B_k) \implies |I_1| ≤ 60^n. $$
	\end{dukazin}
\end{veta}

\end{document}
