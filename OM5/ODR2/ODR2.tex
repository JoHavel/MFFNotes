\documentclass[12pt]{article}					% Začátek dokumentu
\usepackage{../../MFFStyle}					    % Import stylu



\begin{document}

% 04. 10. 2022

\section{Dynamické systémy}
\begin{definice}[Dynamický systém]
	$(\phi, \Omega)$, $\Omega \subset ®R^n$ otevřená, $\phi: ®R \times \Omega \rightarrow \Omega$ $\phi(t, x)$.
	
	\begin{itemize}
		\item $\phi(0, x) = x$;
		\item $\phi(t, \phi(s, x)) = \phi(t + s, x)$
		\item $\phi$ je spojité.
	\end{itemize}
\end{definice}

\begin{definice}[Orbit]
	$\gamma^+(x_0) = \{\phi(t, x_0) | t ≥ 0\}$ je pozitivní orbit.

	$\gamma^-(x_0) = \{\phi(t, x_0) | t ≤ 0\}$ je negativní orbit.

	$\gamma(x_0) = \{\phi(t, x_0) | t \in ®R\}$ je plný orbit.
\end{definice}

\begin{definice}[Pozitivně, negativně a úplně invariantní]
	$(\phi, \Omega)$ dynamický systém, $M \subset \Omega$.

	$M$ je pozitivně invariantní $≡$ $\forall x \in M: \gamma^+(x) \subset M$.

	$M$ je negativně invariantní $≡$ $\forall x \in M: \gamma^-(x) \subset M$.

	$M$ je úplně invariantní $≡$ $\forall x \in M: \gamma(x) \subset M$.
\end{definice}

\begin{poznamka}
	$\gamma^+(x_0)$ je pozitivně invariantní, $\gamma^-(x_0)$ je negativně invariantní a $\gamma(x_0)$ je úplně invariantní.
\end{poznamka}

\begin{definice}
	$$ \omega(x_0) = \{y \in \Omega | \exists \{t_k\}_{k=1}^∞, t_k \rightarrow ∞: \phi(t_k, x_0) \rightarrow y\}, $$
	$$ \alpha(x_0) = \{y \in \Omega | \exists \{t_k\}_{k=1}^∞, t_k \rightarrow -∞: \phi(t_k, x_0) \rightarrow y\}. $$
\end{definice}

\begin{poznamka}[To je ekvivalentní]
	$\omega(x_0) = \{y \in \Omega | \forall \epsilon > 0\ \forall T > 0\ \exists t ≥ T: |\phi(t_r, x_0) - y| < \epsilon\}$.
\end{poznamka}

\begin{lemma}
	$\omega(x_0) = \bigcap_{\tau ≥ 0} \overline{\gamma^+(\tau, x_0)}$.

	\begin{dukazin}
		„$\subseteq$“: $y \in \omega(x_0)$: $\forall \epsilon > 0\ \forall T\ \exists t ≥ T: |\phi(t, x_0) - y| < \epsilon$. Chceme:
		$$ \forall \tau ≥ 0\ \forall \epsilon > 0\ \exists z \in \gamma^+(\tau, x_0): |y - z| < \epsilon \Leftrightarrow $$
		$$ \Leftrightarrow \forall \tau ≥ 0\ \forall \epsilon > 0\ \exists s ≥ \tau, z = \phi(s, x_0): |y - \phi(s, x_0)| < \epsilon. $$

		„$\supseteq$“: $\forall \tau ≥ 0\ y \in \overline{\gamma^+(\tau, x_0)}$ $\implies$
		$$ \implies \forall \epsilon\ \exists s ≥ \tau: |\phi(s, x_0) - y| < \epsilon. $$
	\end{dukazin}
\end{lemma}

\begin{veta}[Vlastnosti $\omega$-limitní množiny]
	Nechť $(\phi, \Omega)$ je dynamický systém, $x_0 \in \Omega$. Potom
	
	\begin{enumerate}
		\item $\omega(x_0)$ je uzavřená, úplně invariantní.
		\item Pokud $\gamma^+(x_0)$ je relativně kompaktní v $®R^n$, pak $\omega(x_0) ≠ \O$, $\omega(x_0)$ je kompaktní, souvislá.
	\end{enumerate}

	\begin{dukazin}
		1. $\omega(x_0)$ je průnik uzavřených množin, tedy uzavřená. $y \in \omega(x_0)$ $\exists t_k \nearrow ∞$ $\phi(t_k, x_0) \rightarrow y$.

		$$ s_k = t_k + t \qquad \phi(s_k, x_0) = \phi(t_k + t, x_0) = \phi(t, \phi(t_k, x_0)) $$
		$$ t_k \rightarrow ∞, \phi\ \text{spojitá} \qquad \phi(s_k, x_0) = \phi(t, \phi(t_k, x_0)) \rightarrow \phi(t, y) $$

		2. Víme $\exists K \subset ®R^n$ kompaktní $\gamma^+(x_0) \subset K$. a) pokud $t_n ≥ 0, t_n \rightarrow ∞ \{\phi(t_n, x_0)\}_{n=1}^∞$ omezená posloupnost $\implies \exists \{t_{n_k}\}_{k=1}^∞ \subset \{t_n\}_{n=1}^∞$, podposloupnost, $\exists y \in \Omega \phi(t_{n_k}, x_0) \rightarrow y$. Pak $y \in \omega(x_0)$.

		b) $\omega(x_0)$ je tedy úplná a omezená, takže kompaktní. c) ať $\omega(x_0)$ je nesouvislá, tedy $\omega(x_0) \subseteq U \cup V$, $U, V$ otevřené disjunktní neprázdné, $U, V \subseteq K$. Vezměme $y \in \omega(x_0) \cap U$, $z \in \omega(x_0) \cap V$. Nechť $t_n$ je posloupnost taková, že $\phi(t_{2n} x_0) \rightarrow y$, $\phi(t_{2n + 1}, x_0) \rightarrow z$, $t_{2n} < t_{2n+1}$, $\phi(t_{2n}, x_0) \in U$, $\phi(t_{2n + 1}, x_0) \in V$. $F = K \setminus (U \cup V)$ uzavřená, tedy $\exists s_n \in (t_{2n}, t_{2n + 1}): \phi(s_n, x_0) \in F$. Tedy $\{\phi(s_n, x_0)\}$ je omezená posloupnost $\implies$ $\exists$ podposloupnost konvergující k $w \in F$.
	\end{dukazin}
\end{veta}

\begin{definice}[Topologická konjugovanost]
	$(\phi, \Omega)$, $\psi, \Theta$ dynamické systémy. $\exists: \Omega \rightarrow \Theta$ homeomorfismus (bijekce, spojité, spojitá inverze) $h$:
	$$ \forall x \in \Omega\ \forall t \in ®R \qquad h(\phi(t, x)) = \psi(t, h(x)). $$
\end{definice}

\begin{poznamka}
	Dá se zobecnit ještě zobrazováním časů.
\end{poznamka}

\begin{veta}[O rektifikaci]
	$\dot{x} = f(x), f(x_0) ≠ 0$, $(\phi, \Omega)$ příslušný dynamický systém. $\dot{y} = \begin{pmatrix} 1 \\ 0 \\ 0 \\ \vdots \\ 0 \end{pmatrix}$, $y(0) = 0$ a $(\psi, \Theta)$ je příslušný dynamický systém. Potom $(\phi, \Omega)$, $(\psi, \Theta)$ jsou lokálně topologicky konjugované ($\exists U$ okolí $x_0 \in \Omega$ a $V$ okolí $¦o \in ®R^n$ taková, že $\exists g: U \rightarrow V$ homeomorfismus $g(\phi(t, x)) = \psi(t, g(x))$ $\forall x \in U$, $\forall t: \phi(t, x) \in U$).

	\begin{dukazin}
		BÚNO $f_1(x_0) = \alpha ≠ 0$ (první souřadnice funkce $f$) a $x_0 = ¦o$. Buď $\tilde{V}$ okolí $¦o \in ®R^n$ $G: \tilde{V} \rightarrow ®R^n$, $G(y_1, …, y_n) = \phi(y_1, (0, y_2, …, y_n))$. Chceme ukázat, že $G$ je invertibilní na nějakém okolí.
		$$ \frac{\partial G(y_1, …, y_n)}{\partial y_1} |_{(0, …, 0)} = \frac{\partial \phi}{\partial t}(t = y_1, (0, y_2, …, y_n)) |_{y_1=0, …, y_n=0} = $$
		$$ = f(\phi(y_1 (0, y_2, …, y_n)))|_{y_1 = 0, …, y_n = 0} = f(\phi(0, (0, …, 0))) = f(x_0) = \alpha. $$
		$$ \frac{\partial G(y_1, …, y_n)}{\partial y_j} |_{(0, …, 0)} = \lim_{h \rightarrow 0} \frac{G(0, …, h, …, 0) - G(0, …, 0)}{h} = $$
		$$ = \lim_{h\rightarrow 0} \frac{(0, …, h, …, 0)^T - (0, …, 0)^T}{h} = (0, …, 1, …, 0)^T = e_j. $$
		Tedy $\nabla G(0, …, 0)$ je „jednotková matice, až na to, že $a_{11}$ je $\alpha$“, tudíž podle věty o inverzi funkce $\exists V \subseteq \tilde{V}$ okolí 0, $\exists U$ okolí bodu $x_0$ tak, že $G: V \rightarrow U$ je homeomorfismus. Položme $g = G^{-1}$.

		Nyní stačí $g(\phi(t, x_0)) = \psi(t, g(x_0))$ $\forall x_0 \in U$ $\forall t: \phi(t, x_0) \in U$. $\phi(t, x_0) = G(\psi(t, g(x_0)))$

		3. $x \in U = G(V)$ $\exists y \in V$ $x = G(y)$
		$$ x = \phi(y, (x_{01}, x_{02} + y_2, …, x_{0n} + y_n)) $$
		$$ \phi(t, x) = \phi(t, \phi(y, (x_{01}, x_{02} + y_2, …, x_{0n} + y_n))) = \phi(t + y, (x_{01}, x_{02} + y_2, …, x_{0n} + y_n)) $$
	\end{dukazin}
\end{veta}

\begin{veta}[La Salle invariance principle]
	$$ x' = f(x), (\phi, \Omega) \quad \phi: ®R \rightarrow \Omega, f \text{ loc. Lip.} $$
	$$ \exists V: \Omega \rightarrow ®R, \text{ bounded from below}. $$
	$$ \exists l \in ®R: \Omega_l = \{x \in \Omega | V(x) ≤ l\}\ – \text{ bounded} $$
	$$ \dot{V}_f(x) := \nabla V(x) · f(x) = \sum_{j=1}^n  \frac{\partial V(x)}{\partial x_j}·f_j(x) ≤ 0 \qquad \forall x \in \Omega_l. $$
	$$ R = \{x \in \Omega_l | \dot{V}_f(x) = 0\}, \quad M = \{y \in R | \gamma^+(y) \subset R\}. $$

	Then $\forall x \in \Omega_l: \omega(x) \subset M$.

	\begin{dukazin}
		Let $x \in \Omega_l$. $\forall y \in \omega(x)\ \exists t_k \nearrow ∞: x(t_k) \rightarrow y$. $\phi(t, x_0) = x(t)$.
		$$ \frac{d}{dt} V(x(t)) = \nabla V(x(t))·x'(t) = \dot{V}_f(x(t)) ≤ 0. $$
		$V(x(t)) \searrow$ and $\exists C: \forall x \in \Omega: V(x) > -C$ so $\exists \lim_{t \rightarrow ∞} V(x(t)) = c$.

		So $\exists c\ \forall y \in \omega(x_0) V(y) = c$. $V(x(t_k)) \rightarrow V(y) = c$.
		$$ \gamma^+(y) \subset \omega(x_0)\ V(\phi(t, y)) = c\ \forall t ≥ 0 \implies $$
		$$ \implies \frac{d}{dt}V(\phi(t, y)) = 0. $$
		$\gamma^+(y) \subset R$ in particular, $y \in R$. Hence $y \in M$.
	\end{dukazin}
\end{veta}

\section{Poincaré-Bendixson theory}
\begin{veta}[Poincaré-Bendixson]
	Let $p \in \Omega$, $\Omega$ open connected. $\omega(p)$ doesn't contain stationary points and $\gamma^+(p)$ is relatively compact ($\overline{\gamma^+(p)}$ is compact). Then $\omega(p) = \Gamma$-periodic orbit.
\end{veta}

\begin{veta}[Bendixson-Dulas]
	$\Omega$-simply connected ($\forall$ closed Jordan curve $\gamma$ in $\Omega$, $\Int(\gamma) \subset \Omega$). $\exists B: \Omega \rightarrow ®R: (\Div Bf)(x) = \frac{\partial Bf_1}{\partial x_1}(x_1, x_2) + \frac{\partial Bf_2}{\partial x_2}(x_1, x_2) > 0$ for almost every $x \in \Omega$. Then $x' = f(x)$ doesn't have nontrivial periodic solutions.
\end{veta}

\begin{definice}[Transverzála]
	$\Sigma$ segment on a line such that $\forall p \in \Sigma: \Sigma \not\parallel f(p)$.
\end{definice}

\begin{lemma}
	$\Sigma$ transverzála, $p \in \Sigma \subset \Omega$. Then $\exists \tilde U \subset U$ neighborhood of $p$, $\exists \Delta > 0$ such that
	$$ \forall y \in \tilde U: \phi(t, y) \subset U\ \forall t: |t| < \Delta \land \exists \tau: |\tau| < \frac{\Delta}{2}: \phi(\tau, y) \in \Sigma \cap \tilde U. $$

	\begin{dukazin}
		Use Th. of rect.
	\end{dukazin}
\end{lemma}

\begin{lemma}
	Let $p \in \Omega$ and assume that $|\gamma^+(p) \cap \Sigma| ≥ 3$, i. e. $\exists t_1 < t_2 < t_3$ $\phi(t_j, p) \in \Sigma$, $j=1, 2, 3$. Then $\phi(t_2, p)$ lies between $\phi(t_1, p)$ and $\phi(t_3, p)$.

% 18. 10. 2022 Z poznámek spolužáka

	\begin{dukazin}
		Simple closed curve:
		$$ \psi := \{\phi(t, x), t \in [t_1, t_2]\} \cup \underbrace{\conv\{z_1, z_2\}}_{\subset \Sigma}. $$

		By uniqueness of $\phi$ and by the Jordan Lemma.
	\end{dukazin}
\end{lemma}

\begin{lemma}
	$\Sigma \subseteq \Omega \subseteq ®R^2$ transversal, $p \in \Omega \implies |\omega(p) \cap \Sigma| ≤ 1$.

	\begin{dukazin}
		$$ y ≠ z \in \omega(p) \cap \Sigma \implies \exists t_k \nearrow ∞: x(t_{2k}) \rightarrow y \land x(t_{2k + 1}) \rightarrow z. $$

		From lemma above: $\exists \tilde U \subset U$ – neighbourhoods of $y$ and $\exists \Delta$:
		$$ \exists k_0: \forall k > k_0, (x(t_{2k}) \in \tilde U) \implies (\exists \tilde t_{2k}: |\tilde t_{2k} - t_{2k}| < \frac{\Delta}{2} \land x(\tilde t_{2k}) \in \Sigma \cap \tilde U. $$

		Similarly $\exists \tilde V$ – neighbourhood of $z$, $\exists \tilde t_{2k+1}$: $|\tilde t_{2k+1} - t_{2k+1}| < \frac{\Delta}{2}$ and $x(\tilde t_{2k+1}) \in \Sigma \cap \tilde V$.

		WLOG $\tilde V \cap \tilde U = O$. Now continue with Lemma 2 (not monotonic).
	\end{dukazin}
\end{lemma}

\begin{dukaz}[Poincaré-Bendixson theorem]
	Step 1: For $q \in \omega(p)$ we want to show that $q$ belongs to $q \in \Gamma$, where $\Gamma$ is non-trivial periodic orbit.

	$\exists x_0 \in \omega(q)$, $\exists t_k \nearrow ∞$: $\phi(t_k, q) \rightarrow x_0$. $x_0$ is not a stationary point ($q \in \omega(p) \implies \omega(q) \subseteq \omega(p)$). So there exists a transversal $\Sigma \subseteq \Omega, x_0 \in \Sigma$.

	By lemma above $\exists \tilde t_k$, $\exists \Delta > 0$: $|\tilde t_k - t_k| < \frac{\Delta}{2}$. $q \in \omega(p)$ $\implies$ $\phi(\tilde t_k, q) \in \omega(p)$ $\implies$ $\phi(\tilde t_k, q) \in \Sigma \cap \omega(p)$ at most 1-point set by theorem…
	$$ \phi(\tilde t_k, q) \rightarrow x_0 \implies \phi(\tilde t_k, q) = x_0. $$

	Periodic orbit $\implies$ $x_0 \in \Gamma = \{\phi(t, q) | \tilde t_k < t < \tilde t_{k+1}, k \in ®N\}$ $\implies$ $q \in \Gamma$ (uniqueness).

	Now we want to show that $\omega(p) \subseteq \Gamma$. Let $M ≠ \O$, $M = \omega(p) \setminus \Gamma$: $\gamma^+(p)$ is bounded $\implies$ $\omega(p)$ is connected
	$$ \exists x_0 \in \Gamma: \exists \{p_n\}_{n \in ®N} \subseteq M, p_n \rightarrow x_0. $$
	$\exists \Sigma$ transversal: $x_0 \in \Sigma$ (because not stationary). By lemma above we have
	$$ \exists \tilde p_n \in \gamma^+(p_n): \tilde p_n \in \Sigma \cap \gamma^+(p_n) \cap \tilde U(x_0). $$
	Since $p_n \in \omega(p)$, $n \in ®N$, then $\gamma^+(p_n) \subseteq \omega(p) \implies \tilde p_n \in \omega(p)$.

	By previous lemma $\tilde p_n = x_0$ and $p_n \in \gamma^-(\tilde p_n) = \gamma^-(x_0) \subseteq \Gamma$. \lightning.
\end{dukaz}

\begin{dukaz}[Bendixson-Dulas theorem]
	Let $\Gamma$ be a non-trivial periodic orbit, $\Gamma \subset \Omega$, $\Gamma = \partial M$
	$$ 0 < \int_M \Div[B(x)·f(x)] d\lambda^2 = \int_{\partial M} \<B(x)·f(x), \nu(x)\> dS = 0. $$
\end{dukaz}

\section{Caratheodory theory}
\begin{definice}[Caratheodory theory]
	$f$ measurable, $x(t)$ absolutely continuous, Lebesgue integral.
\end{definice}

\begin{definice}
	$\Omega \subseteq ®R^{n+1}$, $f \in Car(\Omega) ≡$ $\forall I \times B \subset \Omega$, $I \subseteq ®R$ bounded interval, $B \subseteq ®R^n$ bounded closed ball:
	\begin{itemize}
		\item $\forall x \in B$: $t \mapsto f(t, x(t))$ is measurable;
		\item for almost every $t \in I$: $x \mapsto f(t, x)$ is continuous;
		\item $\exists h \in L^1(I)$: $|f(t, x)| ≤ |h(t)|$ for almost every $t \in I$ and $\forall x \in B$.
	\end{itemize}
\end{definice}

% 25. 10. 2022 Z poznámek spolužáka

\begin{definice}[*]
	$x' = f(t, x)$, $x(t_0) = x_0$, $\Omega \subseteq ®R^{n+1}$ open, $f: \Omega \rightarrow ®R^n$, $f \in Car(\Omega)$.
\end{definice}

\begin{definice}
	$x: I \rightarrow ®R^n$ ($I$ interval) is a solution of * in the sense of Caratheodory, if $x \in AC_{loc}(I)$ and $graph(x) \subset \Omega$ and for almost every $t \in I: x'(t) = f(t, x(t))$ and $x(t_0) = x_0$.

	\begin{poznamka}
		$$ \Leftrightarrow x(t) = x_0 + \int_{t_0}^t f(s, x(s)) ds, \qquad \text{for almost every } t. $$
	\end{poznamka}
\end{definice}

\begin{lemma}
	$f \in Car(\Omega)$, $x: I \rightarrow ®R$ is continuous, $graph(f) \subseteq \Omega$, then $f(t, x(t)) \in L_{loc}^1(I)$.

	\begin{dukazin}
		Step 1: „$f$ is measurable“: We approximate $x(t)$ by step function on $I = I_1 \cup … \cup I_n \cup …$, $\{x_n(x)\}_{n=1}^∞$, piecewise constant functions, $x_n(t) \rightrightarrows x(t)$ on $I_k$, $I = \bigcup_j I_{j, n}$ disjoint union, $x_n(x) = \xi_{j, n}$ for $t \in I_{j, n}$. $f(t, x_n(t)) = f(t, \xi_{j, n})$ for $t \in I_{j, n}$, $f(t, \xi_{j, n})$ is measurable.

		$f(t, x_n(t)) \rightarrow f(t, x(t))$ for almost every $t \in I$ $\implies f(t, x(t))$ is measurable.

		Step 2: $|f(t, x(t))| ≤ l(t)$ for almost every $t$ $\implies$ $f \in L_{loc}^1(I)$.
	\end{dukazin}
\end{lemma}

\begin{lemma}
	$x: I \rightarrow ®R^n$ continuous, $graph(x) \subseteq \Omega$, $f \in Car(\Omega)$ then $x$ is solution of * $\Leftrightarrow$ $\forall t_1, t_2: x(t_2) - x(t_1) = \int_{t_1}^{t_2} f(s, x(s)) ds$.

	\begin{dukazin}
		„$\implies$“ $x \in AC_{loc}(I)$, $x'(t) = f(t, x)$ for almost every $t \in I$, add $\int$:
		$$ \int_{t_1}^{t_2} x'(t) dt = \int_{t_1}^{t_2} f(s, x(s)) ds. $$

		„$\impliedby$“ $t_1 = t_0$, $t_2 = t$:
		$$ x(t) - x_0 = \int_{t_0}^t f(s, x(s)) ds, \qquad x(0) = x_0. $$
		($f \in L_{loc}^1$, so it make sense).

		$\implies x$ is AC, $graph(x) \subseteq \Omega$, $x'(t) = f(t, x)$ for almost every $t \in I$.
	\end{dukazin}
\end{lemma}

\begin{veta}[Uniform contraction theorem (generalized Banach theorem)]
	$\Lambda$, $X$ metric spaces, $X ≠ \O$ complete, $\Phi: \Lambda \times X \rightarrow X$. $\forall x \in X: \Phi(·, x)$ is continuous, $\exists \kappa \in (0, 1)$: $\rho(\Phi(\lambda, x), \Phi(\lambda, y)) ≤ \kappa · \rho(x, y)$ $\forall \lambda \in \Lambda$, $\forall x, y \in X$.

	Then $\forall \lambda \in \Lambda\ \exists! x(\lambda) \in X$ such that $\Phi(\lambda, x(\lambda)) = x(\lambda)$, $\lambda \mapsto x(\lambda)$ is continuous and
	$$ \rho(y, x(\lambda)) ≤ \frac{\rho(y, \Phi(\lambda, y))}{1 - \kappa}\ \forall y \in X\ \forall \lambda \in \Lambda. $$

	\begin{dukazin}
		Let $x_0 \in X$, $x_1 = x_1(\lambda, x_0) := \Phi(\lambda, x_0)$, $x_{n+1} = x_{n+1}(\lambda, x_0) := \Phi(\lambda x_n)$. $\lambda \in \Lambda$ fixed:
		$$ \rho(x_n(\lambda, x_0), x_m(\lambda, x_0)) ≤ \sum_{k=n}^{m - 1} \rho(x_k(\lambda, x_0), x_{k+1}(\lambda, x_0)) = $$
		$$ = \sum_{k=n}^{m - 1} \rho(\Phi(\lambda, x_{k-1}(x, x_0)), \Phi(\lambda, x_k(\lambda, x_0))) ≤ $$
		$$ ≤ \sum_{k=n}^{m - 1} \kappa \rho(x_k, x_{k-1}) ≤ $$
		$$ \(\rho(x_k, x_{k-1}) ≤ \kappa \rho(x_k, x_{k-1}) ≤ … ≤ \kappa^k \rho(x_0, x_1).\) $$
		$$ ≤ \sum_{k=n}^{m-1} \kappa^k \rho(x_0, x_1(\lambda, x_0)) ≤ \rho(x_0, x_1(\lambda, x_0)) \underbrace{\sum_{k=n}^∞ \kappa^k}_{\frac{\kappa^m}{1 - \kappa}}. $$

		$\implies$ sequence $\{x_n(\lambda, x_0)\}_{k=1}^∞$ is Cauchy $\implies$ it has a limit:
		$$ \exists x(\lambda, x_0): \lim_{n \rightarrow ∞} x_n(\lambda, x_0) = x(\lambda, x_0). $$

		We want to show that $x(\lambda, x_0)$ does not depend on $x_0$. $\tilde x_0: x_n(\lambda, \tilde x_0)=: \tilde x_n$.
		$$ \rho(x_n, \tilde x_n) = \rho(\Phi(\lambda x_{n-1}), \Phi(x, \tilde x_{n-1})) ≤ \kappa^n \rho(x_0, \tilde x_0) \rightarrow 0 \implies x = \tilde x. $$

		„$\Phi(\lambda, x(\lambda)) = x(\lambda)$“:
		$$ \rho(\Phi(\lambda, x(\lambda)), x(\lambda)) = \lim_{n \rightarrow ∞} \rho(\Phi(\lambda, x(\lambda)), x_n(\lambda)) = $$
		$$ = \lim_{n \rightarrow ∞} \rho(\Phi(\lambda, x(\lambda)), \Phi(\lambda, x_{n-1}(\lambda))) ≤ \kappa \rho(x(\lambda, x_{n-1}) = 0. $$
	\end{dukazin}

	\begin{dukazin}[„$\lambda \mapsto x(\lambda)$ is continuous“]
		$$ \rho(x(\lambda), x(\mu)) = \rho(\Phi(\lambda, x(\lambda)), \Phi(\mu, x(\mu))) ≤ $$
		$$ ≤ \rho(\Phi(\lambda, x(\lambda)), \Phi(\lambda, x(\mu))) + \rho(\Phi(\mu, x(\mu)), \Phi(\lambda, x(\mu))) ≤ $$
		$$ ≤ \kappa \rho(x(\lambda), x(\mu)) + \rho(\Phi(\mu, x(\mu)), \Phi(\lambda, x(\mu))) \implies $$
		$$ \implies \rho(x(\lambda), x(\mu)) ≤ (1 - \kappa)^{-1} \rho (\Phi(\lambda, x(\mu)), \Phi(\mu, x(\mu))) \rightarrow 0. $$
		So $\Phi$ is continuous in the first variable.

		$\mu$ fixed, $\lambda = \mu_n \rightarrow \mu$. $y$, $\lambda$ fixed, $x_1 = \Phi(\lambda, y)$, $x_n = \Phi(\lambda, x_{n-1})$, $n ≥ 2$.

		$n = 0$ ($y = x_0$): $m \rightarrow ∞$
		$$ \rho(y, \lambda(x)) ≤ \frac{\kappa^0}{1 - \kappa} \rho(y, \Phi(\lambda, y)) = \frac{\kappa^0}{1 - \kappa}\rho(y, x_1). $$
	\end{dukazin}
\end{veta}

\begin{veta}[Generalized Picard theorem]
	$$ *: x' = f(t, x), \qquad x(t_0) = x_0, \qquad f: \Omega \rightarrow ®R^n, \qquad \Omega \subseteq ®R^{n+1}. $$

	$I = [0, T]$, $\Pi$ metric space, $f: I \times ®R^n\times \Pi \rightarrow ®R^n$, $f(t, x, p)$
	\begin{itemize}
		\item $\forall p \in \Pi$ fixed $f(·, ·, p) \in Car(I \times ®R^n)$;
		\item $\|f(t, x, p) - f(t, y, p)\| ≤ l(t)\|x - y\|$ for some $l(t) \in L^1(I)$, $\forall x, y \in ®R^n$, $\forall p \in \Pi$ and almost all $t \in I$;
		\item for every $x(·) \in ©L(I)$ the map
			$$ p \mapsto \int_0^t f(s, x(s), p) ds, \qquad t \in I $$
			is continuous.
	\end{itemize}

	Then for all $x_0 \in ®R^n$ a $\forall p \in \Pi$ $\exists! x( ·) = x(x_0, p) \in AC(I)$ satisfying * in Caratheodory sense with initial condition $x(t_0) = x_0$ and $x(·)$ depends continuously on $x_0$, $p$, tj.
	$$ (x_0)_n \rightarrow x_0 \land p_n \rightarrow p \implies x_n(·) ≡ x((x_0)_n, p_n) \rightrightarrows x(x_0, p). $$

	\begin{dukazin}
		$X := \phi(I)$ is complete, $\|f\|_x = \sup_{t \in I} \{f(x)·e^{-L t}\}$, $L$ will be chosen $L > \epsilon$. $\Lambda := ®R^n \times \Pi \ni (\lambda_0, p)$, $\int_0^t e^{L(t - s)} ds ≤ \int_0^t e^{-L y} dy ≤ \int_0^∞ e^{-L y} dy = \frac{1}{L}$.

		$$ \Phi(x_0, p, x(·)) (t) := x_0 + \int_0^t f(s, x(s)) ds, \qquad t \in [0, T]. $$

		$\Phi$ is continuous in $x_0$ and $p$. $\Phi$ is contraction:
		$$ \|(\Phi(x_0, \rho, x(·)) - \Phi(x_0, p, y(·)))\| = \int_0^t f(s, x(x)) - f(s, y(s)) ≤ $$
		$$ ≤ \int_0^T \|f(s, x(s), p) - f(s, y(s), p)\| ds ≤ \int_0^T l(s) \|x - y\| ds, $$
		for almost every $t$.

		$$ \|x(s) - y(s)\| e^{-L·s} ≤ \|x - y\|_X ≤ \int_0^t l(s) e^{+L(s)} ds·\|x - y\|_X ? $$
		$$ \|\Phi(x_0, p, x(·)) - \Phi(x_0, p, y(·))\|_X ≤ \|x(·) - y(·)\|_X \sup_t \int_0^t l(s) e^{-L(t - s)} ds ≤ $$
		$$ ≤ \int_0^T l(s) e^{-L s} ds, \qquad l \in L^1([0, T]). $$

		$$ \exists l_1, l_2 ≥ 0: \int_0^T l_1(x) dt ≤ \frac{1}{3} $$
		$$ \exists c > 0: \|l_2\| ≤ c \text{ for almost every } t \in I $$
		$$ ? ≤ \frac{1}{3} + c·\frac{1}{L} ≤ \frac{2}{3}. $$

		Then $x \in ©L(I)$ fixed point of $\Phi$.
		$$ \implies x(t) = x_0 + \int_0^t f(s, x(s), p) ds \implies x \in AC(I). $$

		Continuously depends on $p, x_0$:
		$$ \sup_{t \in I} \{(x(t, x_0, p) -y(t))e^{-L t}\} ≤ (1 - \kappa)^{-1} ((y(x) - x_0 + \int_0^t f(s, y, TODO))) TODO $$
	\end{dukazin}
\end{veta}

% 01. 11. 2022

\section{Controllability}
\begin{definice}[Control theory]
	$$ x' = f(x, u), f: \Omega \times U, \Omega \subset ®R^n, U \subset ®R^n, $$
	$$ u \in ©U := \{u: [0, T] \rightarrow ®R^n | \text{measurable}, ||u||_∞ < ∞\} = L^∞(0, T, ®R^n). $$
	(©U is admissible functions).
\end{definice}

\begin{definice}[Linear task]
	$x' = Ax + Bu$, $A, B \in ®R^{n \times m}$, $m < n$.
\end{definice}

\begin{definice}
	$x_0 \underset{u(0)}{\overset{t}\rightarrow} 0$ iff $x(0) = x_0$, $x(t) = 0$.
\end{definice}

\newcommand{\converg}{\underset{u(0)}{\overset{t}\rightarrow}}
\begin{definice}[Area of controllability]
	$$ ©R(t) = \{x_0 \in ®R^n \middle| \exists u \in L^∞(0, t, ®R^n): x_0 \underset{u(0)}{\overset{t}\rightarrow} 0 \} $$
\end{definice}

\begin{definice}[Kalman matrix]
	$$ ©K(A, B) := (B | AB | A^2B | … | A^{n-1}B) $$
\end{definice}

\begin{veta}
	For linear problem $©R(t) = \LO (g_1, g_2, …, g_{n·m})$, where $©K(A, B) = (g_1 | g_2 | … | g_{n·m})$

	\begin{tvrzeniin}[Observation]
		$x(t) = e^{At} x_0 + \int_0^t e^{A(t - s) B u(s) ds}$.
		$$ x_0 \converg 0 \Leftrightarrow x(t) = 0 \Leftrightarrow x_0 = - \int_0^t e^{-A s} B u(s) ds. $$
	\end{tvrzeniin}

	\begin{lemmain}[1]
		$$ A^k \in \LO(I, A, A^2, …, A^{n - 1}), k \in ®N_0 $$

		\begin{dukazin}
			Cayley-Hamilton.
		\end{dukazin}
	\end{lemmain}

	\begin{dukazin}
		1) $©R(t)$ is vector subspace of $®R^n$ from definition $x_0 + x_1 \underset{(u_1 + u_2)(0)}{\overset{t}\rightarrow} 0$, $\alpha x \underset{\alpha u(0)}{\overset{t}\rightarrow} 0$.

		2) We want $©R(t)^\perp = (\LO(g_1, …, g_n))^\perp$. „$\supseteq$“: $p \in (\LO(g_1, …, g_n))^\perp$. $x_0 \in ©R(t)$ arbitrary. From observation.:
		$$ 0 \overset?= p^T x_0 = -\int_0^t p^Te^{-As} B u(s) ds = - \int_0^t \sum_{k=0}^∞ \frac{(-s)^k}{k!} p^T A^k B u(s) ds$$
		We know $(p, g_j) = 0$, $p^Tg_j = 0$, $p^T ©K(A, B) = 0$, $p^T A^k B = 0$, $k \in [n-1]$. And from lemma 1 $k \in ®N$.
		„$\subseteq$“: $p \in ®R^n$, $p \in ©R(t)^\perp$. We want to prove $p \perp B, AB, A^2B, …, A^{n-1}B$. $B = (b_1 | … | b_m)$. $\forall j \in [n]: p \perp b_j, Ab_j, …, A^{n-1}b_j$. $\phi \in L^∞(0, T, ®R)$, $u(t) = \phi(t)·¦e_j$, where $x_0 = - \int_0^t e^{-A s} B u(s) ds$. We have $x_0 \converg 0$, hence $x_0 \in ©R(t)$.
		$$ 0 = p^T x_0 = - p^T\int_0^t e^{-As} B u(s) ds = - \int_0^t p^T e^{-As}b_j \phi(s) ds \implies y(s) := p^T e^{-As}b_j ≡ 0 $$
		So we have $p^T e^{-As} b_j ≡ 0$, we derivate it, $p^T A^n e^{-As} b_j ≡ 0$, and set $s = 0$.
	\end{dukazin}
\end{veta}

\begin{dusledek}
	$©R(t)$ doesn't depend on time.
\end{dusledek}

\begin{definice}[Locally and globally controllable]
	Linear problem is called locally controllable, iff $\exists \delta > 0: \{x_0 \in ®R^2 |\ |x_0| < \delta\} \subset ©R(t)$. And globally if $®R^n = ©R(t)$.
\end{definice}

\begin{dusledek}
	Linear problem is controllable $\Leftrightarrow$ $\rank K(A, B) = n$.
\end{dusledek}

\subsection{Observability}
\begin{definice}[System for observability]
	$$ x' = f(x), x(0) = x_0, f: \Omega \subset ®R^n \rightarrow ®R^n y = g(x), g: \Omega \subset ®R^n \rightarrow ®R^m, m < n. $$
\end{definice}

\begin{definice}
	We say that system $x' = f(x)$ is observable through $g(·)$ on $[0, t]$, iff $\forall x_1(·), x_2(·): [0, T] \rightarrow ®R^n: g(x_1(t)) = g(x_2(t))$ $\forall t \in [0, T]$ $\implies x_1(0) = x_2(0)$.
\end{definice}

\begin{definice}[Linear observability]
	$x' = Ax$, $y = Bx$, $A \in ®R^{n \times n}$, $B \in ®R^{m \times n}$.
\end{definice}

\begin{veta}
	$x' = Ax$ is observable on $[0, T]$ through $y = B x$ $\Leftrightarrow$ $x' = A^T x + B^T u$ is controllable.

	\begin{dukazin}
		(We will prove equivalence with $\rank ©K(A^T, B^T) = n$.) „$\impliedby$“: For contradiction
		$$ \exists x_1(t), x_2(t), t \in [0, T], B x_1(t) ≡ Bx_2(t): x(t) = x_1(t) - x_2(t), x(0) = x_0 ≠ 0, B x(t) ≡ 0. $$
		$$ x(t) = e^{At} x_0, Bx(t) = Be^{At} x_0 ≡ 0 \qquad \forall t \in [0, T]. $$
		We differentiate it, set $t = 0$ and get $B x_0 = 0$, $B A x_0 = 0$, …, $BA^{n-1}x_0 = 0$. So $x_0^T B^T = 0$, …, $x_0^T \(A^T\)^{n-1} B^T = 0$. $x_0^T ©K(A^T, B^T) = 0$, $x_0 \perp ©K(A^T, B^T)$, \lightning.

		„$\implies$“: For contradiction $\rank(A^T, B^T) < n$ $\implies$ $\exists x_0 ≠ 0: x_0^T ©K(A^T, B^T) = 0$. $x_0^T\(A^T\)^k B^T = 0$ $\forall k \in [n - 1]$ and from lemma 1 $\forall k \in ®N$. $B A^T x_0 = 0$. $B e^{At} x_0 = 0$ $\forall t \in [0, T]$. \lightning.
	\end{dukazin}
\end{veta}

\begin{veta}[Local controllability]
	Let $V \subset ®R^n$ neighbourhood of $0$, $U \subset ®R^n$ neighbourhood of $0$, $f: V \times U \rightarrow ®R^n$ $C^1$ smooth, $f(0, 0) = 0$, $©U = \{u: [0, T] \rightarrow U \text{ measurable}\}$, $A = \nabla_x f(0, 0)$, $B = \nabla_u f(0, 0)$, $\rank ©K(A, B) = n$. Then
	$$ x' = f(x, u), x(0) = x_0 \text{ is locally controllable }\forall t \in (0, T]. $$
	
	\begin{dukazin}
		Fix $t > 0$, consider $x' = Ax + Bu$. Since $\rank(A, B) = n$, the linear problem is globally controllable. Take initial condition $y_1, …, y_n$ linearly independent.
		$$ \exists u_i \in L^∞(0, t, ®R^n): y_j \rightarrow_{u_i(0)}^t 0 $$
		$\forall \lambda = (\lambda_1, …, \lambda_n) \in ®R^n$ denote by $u_{\lambda(t)} = \sum_{j=1}^n \lambda_j u_j(t)$. We know $\sum_{j=1}^n \lambda_j y_j \rightarrow_{u_\lambda(0)}^t 0$.

		Step 2:
		$$ x_\lambda' = f(x_\lambda, u_\lambda), \qquad x_\lambda(t) = 0 $$
		If $\lambda = 0$, then $u_\lambda ≠ 0$, then $x_\lambda ≡ 0$.
		$$ \psi(\lambda) := x_\lambda(0), \psi: U_\lambda(0) \subset ®R^n \rightarrow ®R^n. $$
		We want to prove $\psi(U_\lambda(0)) \supseteq \tilde V$, for some $\tilde V \subset ®R^n$ open, $0 \in \tilde V$. We will prove that $\psi$ is $C^1$ smooth, and that $\nabla \phi(0)$ is regular (if this is proved, than $\psi$ is local diffeomorphism).

		Step 3:
		$$ x_\lambda(s) = x_\lambda(t) + \int_t^s f(x_\lambda(s), u_\lambda(s)) ds. $$
		Formally differentiate:
		$$ \frac{\partial x_\lambda(s)}{\partial \lambda_j} = \int_t^s(\nabla_x f(x_\lambda(s), u_\lambda(s))·\frac{\partial x_\lambda(s)}{\partial \lambda_j} + \nabla_u f(x_\lambda(s), u_j(s))) ds. $$
		Denote $y_{\lambda, j}(s) = \frac{\partial x_\lambda(s)}{\partial \lambda_j}$.
		$$ y_{\lambda, j}'(s) = \nabla_x f(x_\lambda(s), u_\lambda(s))·y_{\lambda, j}(s) + \nabla_u f(x_\lambda(s), u_\lambda(s))·u_j(s). $$
		$$ y_{\lambda, j}(t) = 0. $$

		Consider $(LP y) \rightarrow y_{\lambda, j}(·)$.
		$$ x_{\lambda + \Delta \lambda}(s) - x_{\lambda}(s) - \Delta \lambda · y_{\lambda, j}(s) = 0 $$
		(as in Thn? of differentiability w. r. t. initial condition)
		$$ \frac{\partial \psi}{\partial \lambda_j}(\lambda = 0) = \frac{\partial x_\lambda(s = 0)}{\partial \lambda_j}|_{\lambda = 0} = y_{\lambda, j}(s = 0) |_{\lambda = 0} = y_{\lambda, j}(s = 0)|_{\lambda =0} = y_j. $$
		If $\lambda = 0$, then $(LPy)$: $y_{0, j}' (s) = A y_{0, t}(s) + B u_j(s)$, $y_{0, j}(t) = 0$. From uniq.: $y_{0, j}(0) = y_{j, 0}$.
		$$ \nabla \psi(0) = \(\frac{\partial \psi}{\partial \lambda_1}(0) … \frac{\partial \psi}{\partial \lambda_n}(0)\) = (y_1, …, y_n) $$
		regular matrix.
	\end{dukazin}
\end{veta}

\begin{poznamka}
	$$ x' = Ax + Bu, u \in ©U = \{u: [0, T] \rightarrow [-1, 1] \text{ measurable}\}, x(0) = x_0. $$
\end{poznamka}

\begin{definice}
	$$ ©R(t) = \{x_0 \in ®R^n | \exists u \in ©U \land x_0 \rightarrow_{u(0)}^t 0\}. $$
\end{definice}

\begin{definice}
	$u_n \in ©U_0$: $u_n \rightharpoonup^* u \in ©U$ $≡$ $\forall f \in L(0, T, ®R^n): \int_0^T f(s) u_n(s) ds \rightarrow \int_0^T f(s) u^*(s) ds$.
\end{definice}

\begin{veta}[Alaoglu]
	$©U$ is weak-* sequentially compact (i. e. $\forall \{u_n\}_{n=1}^∞ \in ©U\ \exists \{u_{n_k}\}$ weekly-* convergent).
\end{veta}

\begin{veta}
	$©R(t)$ convex, symmetric, closed $0 < t_1 < t_2 \implies ©R(t_1) \subset ©R(t_2)$.

	\begin{dukazin}
		Convex: $x_{01}, x_{02} \in ©R(t)$, $\alpha \in [0, 1] \implies \alpha x_{01} + (1 - \alpha)x_{02} \in ©R(t)$.
		$$ x(t) = e^{At} x_0 + \int_0^t e^{As} B u(s) ds. x_{01} \rightarrow_{u_{01}}^t 0 \land x_{02} \rightarrow_{u_{02}}^t 0 \Leftrightarrow x_{0i} = -\int_0^t e^{(s - t)A} Bu_{0i}(s) ds. $$
		
		Symmetry: $x_0 \in ©R(t) \implies -x_0 \in ©R(t)$, $x_0 \rightarrow_u^t 0 \implies -x_0 \rightarrow_{-u}^t 0$.

		Closedness: $x_{0n} \in ®R(t), x_{0n} \rightarrow x_0$. $x_0 \in ©R(t)$? $\exists u_n(0) \in ©U$, $x_{0n} = - \int_0^t e^{(s - t)A} B u_n(s) ds \rightarrow - \int_0^t e^{(s - t)A} Bu(s) ds$.
		WLOG $u_n \rightharpoonup^* u \in ©U$. Then $x_0 \rightarrow_u^t 0$.

		$$ ©R(t_1) \subset ©R(t_2), \qquad 0 < t_1 < t_2 < T $$
		$$ \exists u_1 \in ©U\qquad x_0 = -\int_0^t e^{(s - t)A} Bu_1(s) ds. $$
		Define $u_2(s) = u_1(s)$ if $0 ≤ s ≤ t$, else 0.
	\end{dukazin}
\end{veta}

\begin{definice}[Area of controllability]
	$$ ©R := \bigcup_{t > 0} ©R(t). $$
\end{definice}

\begin{veta}
	$$ \rank©K(A, B) = n \Leftrightarrow \forall t > 0: ©R(t) \supseteq U(0), $$
	where $U(0) \subset ®R^n$ is some neighbourhood of 0.

	\begin{dukazin}
		„$\impliedby$“: If $\exists t > 0\ ©R(t) \supset U(0)$ open, $0 \in U(0)$. $\tilde{©R}: u \in L^∞, ©R: ||u||_∞ ≤ 1$, then $\tilde{©R}(t) \supset ©R(t) \supset U(0) \implies \tilde{©R}(t) = ®R^n$. $\implies \rank ©K(A, B) = n$.

		„$\implies$“: $\rank©K(A, B) = n \implies \tilde{©R}(t) = ®R^n$. From theorem of local controllability.
	\end{dukazin}
\end{veta}

\begin{veta}[Minimal time]
	$$ x' = Ax + Bu $$
	$$ \forall x_0 \in ©R = \bigcup_{s > 0} ©R(s) $$
	$$ \exists t > 0\ \exists u(0) \in ©U: x_0 \rightarrow_u^t 0 $$
	$$ t = \inf\{s > 0|x_0 \in ©R(s)\}. $$

	\begin{dukazin}
		$$ t > 0, \exists t_n \searrow t, t_n \in (0, T], \exists u_n \in U, x_0 = - \int_0^{t_n}e^{(t_n-s)A}V u_n (s). $$
		Alaoglu: WLOG $u_n \overset*\rightharpoonup u \in U$.
		$$ x_0 = - \int_0^{t_n} e^{(t - s)A}B u_n(s) ds - \int_0^{t_n}\[e^{(t - s)A} - e^{(t_n - s)A}\]Bu_n (s) ds $$
		$$ x_0 = - \underbrace{\int_0^t e^{(t - s)A}B u_n(s) ds}_{\overset*\rightharpoonup \int_0^t e^{(t - s)A}B u(s)ds} - \underbrace{\int_t^{t_n} e^{(t - s)A}B u_n(s) ds}_{\rightarrow 0} - \underbrace{\int_0^t\[e^{(t - s)A} - e^{(t_n - s)A}\]Bu_n (s) ds}_{\rightarrow 0 (DLCT)}. $$
	\end{dukazin}
\end{veta}

\begin{definice}[Bang-bang]
	We say that a regulation $u \in U(0)$ is of type bang-bang, if for almost every $t \in [0, T]$: $u(t) = ±1$.
\end{definice}

\begin{veta}[Bang-bang]
	If $x_0 \in ©R(t) \implies \exists \tilde u(0)$ of type bang-bang $x_0 \rightarrow_{\tilde u}^t 0$.
\end{veta}

% 15. 11. 2022

\begin{definice}[Extremal point]
	$X$ vector space, $K \subset X$. $x \in K$ is called an extremal point, if it cannot be written as $x = \frac{y + z}{2}$, $y, z \in K$, $y ≠ z$. We denote $ex(K)$ the set of extremal points.
\end{definice}

\begin{tvrzeni}[Krein-Milman theorem]
	$X$ locally convex vector space, $K \subset X$: $K ≠ \O$, $K$ convex and compact. Then $ex(K) \cap K ≠ \O$.
\end{tvrzeni}

\begin{dukaz}[Bang-bang]
	$$ K = \{u \in ©U | x_0 \rightarrow_{u(0)}^t 0\}, \qquad X = L^∞(0, T, ®R^n). $$
	$K ≠ \O$ ($u \in ©R(t)$), $K$ convex, $K$ is compact (sequential compactness: Alaoglu theorem? $L'(0, T, ®R^n)$ separable $\implies$ $L^∞(0, T, ®R^n)$ with locale $*$ topology is metrizable $\implies$ sequential compactness $\implies$ compactness.

	Choose $\tilde u_j \in ex(K)$ (from Krein-Milman). It remains to check that $\tilde{u}_j(s) = ± 1$, $\forall j \in [n]$ for almost every $s \in (0, t)$. For contradiction: for some $j \in [n]$ $\exists E \subset (0, t)$, $\lambda(E) > 0$ $\forall s \in E$ $|\tilde{u}_j(s)|<1$. WLOG
	$$ \exists \epsilon > 0\ \forall s \in E |\tilde{u}_j(s)| < 1 - \epsilon \qquad \[E = \bigcup_{n \in ®N}\{s \in (0, t) \middle| |\tilde{u}_j(s)| ≤ 1-\frac{1}{n}\}\]. $$

	$$ x_0 = - \int_0^t e^{-sA} B \tilde{u}(s) ds $$
	We find (from ortogonality to $B_i e^{-s A}$) $\phi \in L^∞(0, T, ®R)$ such that:
	\begin{enumerate}
		\item $\supp \phi \subset E$;
		\item $\int_E e^{-sA} B (0, …, 0, \phi(s), 0, …, 0)^T ds = 0$;
		\item $\forall s \in E |\phi(s)| < \epsilon$.
	\end{enumerate}
	Define $u_1(s) = \tilde{u}(s) + (0, …, 0, \phi(s), 0, …, 0)^T$ and $u_2(s) = \tilde{u}(s) - (0, …, 0, \phi(s), 0, …, 0)^T$. Then $x_0 \rightarrow_{u_{1, 2}(0)}^t 0$, and $u_1, u_2 \in K$.
\end{dukaz}

\begin{veta}[Global controllability]
	We have (LTP) $x' = Ax + Bu$, $x(0) = x_0$, $u \in ©U$.

	1. $\rank ©K(A, B) = n$ $\implies$ (LTP) is locally controllable.

	2. $\rank ©K(A, B) = n$ and $\Re \lambda ≤ 0$ $\forall \lambda$-eigenvalues of $A$. Then (LTP) is globally controllable $©R = \bigcup_{t > 0} ©R(t) = ®R^n$.

	\begin{dukazin}
		1) folows from „In theorem of local controllability for the problem $x' = f(x, u)$ we could take $u \in ©U$.“

		2a) If $\forall \lambda$ eigenvalue of $A$ we have $\Re \lambda < 0$ $\implies$ theorem follows from text above: first, set $u = 0$. Then we arrive at a neighbourhood of zero.

		2b) For contradiction $x_0 \in ®R^n \setminus ©R$. ©R convex $\exists z_0 \in \partial ©R$, $n$ normal vector. $\forall x_1 \in ©R: n^T(x_1 - x_0) ≤ 0$, $n^T x_1 ≤ n^T x_0 =: M$.
		$$ x_1 = - \int_0^t e^{-s A} B u(s) ds $$
		$$ n^T x_1 = - \int_0^t \underbrace{n^T e^{-s A} B}_{v(s)} u(s) ds $$
		$$ \tilde{u}(s) := \begin{cases}0,& v(s) = 0,\\\frac{-v(s)}{||v(s)||_2},& v(s) ≠ 0.\end{cases} $$
		If $v(s) ≡ 0$, then apply $\frac{d^p}{(ds)^p}$, $n^T A^p e^{-sA}B ≡ 0$, then $n^T ©K(A, B) = 0$. \lightning

		$$ \int_0^∞ ||v(s)||_2ds = ∞. $$
		If this is true, then $t_k \nearrow ∞$, $u_k = \tilde{u} |_{[0, t_k]}$, $x_{1,k} = -\int_0^{t_k} e^{-s A} B u_k(s) ds$.
		$$ n^T x_{1, k} = - \int_0^{t_k} v^t(s) · \tilde{u}(s) ds = \int_0^{t_k} ||v(s)||_2 ds \rightarrow ∞. \text{\lightning} $$

		$v(s)$ is linear combination of $s^j e^{-s \lambda_p}, \Re \lambda_p ≤ 0$. Then $\int_0^∞ |v(s)| ds = ∞$.
	\end{dukazin}
\end{veta}

\begin{veta}[Pontrjagin maximum]
	$$ x' = Ax + Bu, ||u||_∞ ≤ 1, x(0) = x_0. $$
	Let $x_0 \rightarrow^{t^*}_{u^*(0)} 0$, $t^*$ is the minimal. Then $\exists h \in ®R^n \setminus \{¦o\}$:
	$$ h^T·e^{-s A} B u^*(s) = \max_{\eta \in [-1, 1]^m} h^t e^{-sA} B \eta $$
	for almost every $s \in (0, t^*)$.

	\begin{dukazin}
		$x_0 \in \partial ©R(t^*)$.

		Step 2 – contradiction: $\exists E \subset (0, t^*), \lambda(E) > 0$, $\forall s \in E$ $\exists \eta_s \in [-1, 1]^m$ $h^T e^{-sA}B u^*(s) < h^T e^{-sA} B \eta_s$. But $x_j(\delta) \in ©R(t^* - \delta)$, hence $x_0 \in ©R(t^* - \delta)$ and $t^*$ is not minimal.

		Step 1: $x_0 \in \partial ©R(t^*)$. For contradiction $x_0 \in \Int ©R(t^*)$.
		$$ \exists x_1, …, x_{n+1} \in ©R(t^*), x_0 \in CO(x_1, …, x_{n+1}). $$
		$$ \exists u_1, …, u_{n+1} \in U, x_j \rightarrow_{u_j(·)}^{t^*} 0\ \forall j \in [n+1]. $$

		Let $\tilde u_j(t)$ are the corresponding solutions

		TODO!!!
	\end{dukazin}
\end{veta}

% 22. 11. 2022 bylo pouze 2·cvičení

% 29. 11. 2022

\begin{veta}[Pontrjagin]
	$x'(f, u)$, $x(0) = x_0$, $u \in ©U = \{u: (0, T) \rightarrow U \subset ®R^n\}$, $T$ fixed, 
	$$ P[u(·)] = g(x(T)) + \int_0^T r(x(s), u(s))ds \rightarrow maximum. $$
	$f, g, r, \nabla_x f, \nabla_x g, \nabla_x r$ are continuous.

	Let $u$ is a local maximum of this problem (it maximizes $P$), then for $p$ solving:
	$$ H(x, p, u) := p^T f(x, u) + r(x, u), $$
	$$ p' = -\nabla_x H(x, p, u), $$
	$$ p(T) = \nabla_x g(x(T)), $$
	we have
	$$ H(x, p, u) = \max_{\eta \in U} H(x, p, \eta) \text{ for almost every } t \in (0, T). $$

	\begin{dukazin}
		Step one „WLOG $r = 0$“: We set
		$$ x' = f(x, u), \qquad x_{n+1}' = r(x, u), x_{n+1}(0) = 0, P[u(·)] = \hat{g}(\hat{x}(T)) = g(x(T)) + x_{n+1}(T). $$

		Step 2: Fix $\tau \in (0, T)$, $\eta \in U$, $u_\epsilon(T) = \begin{cases}\eta,& t \in (\tau - \epsilon, \tau),\\ u(t), & \text{ elsewhere},\end{cases}$ and corresponding $x_\epsilon(t)$.
		$$ u \text{ "best" } \implies P[u_\epsilon(0)] ≤ P[u(0)] \implies g(x_\epsilon(T)) ≤ g(x(t)). $$
		$$ D := \frac{d}{d\epsilon}|_{\epsilon = 0^+} \qquad D g(x_\epsilon(T))|_{\epsilon = 0^+} ≤ 0 $$
		$$ \nabla_x g(x(T)) · D x_\epsilon(T)|_{\epsilon = 0^+} ≤ 0. $$

		Step 2.2: $x_\epsilon(t) = x_0 + \int_0^t f(x_\epsilon(s), u_\epsilon(s)) ds$. If $t < \tau$, then $u_\epsilon ≡ u$, $x_\epsilon ≡ x$, $D x_\epsilon(t) ≡ 0$ on $[0, t]$. If $t > \tau$, then $x_\epsilon(t) =: y(t)$, $y'(t) = f(y(t), u(t))$, $u(\tau) = x_\epsilon(\tau)$,
		$$ D x_\epsilon(t) ≡ z(t): z' = \nabla_x f(y(t), u(t)) z, z(\tau) = D x_\epsilon(\tau), \qquad \text{variational equation}. $$

		Statement: $z' = A(t)z$, $p' = -A^T(t)p$ $\implies$ $p^T z = const$. Proof: $(p^T z)' = (p^T)'z + p^T z' = -p^T A z + p^T A z = 0$.

		Step 2.3: $p' = -(\nabla_x f(y(t), u(t)))^T p$, $p(T) = (\nabla_x g(x(T)))^T$. Then $p^T(t)z(t)$ constant on $(\tau, T)$, $p^T(\tau) z(\tau) ≤ 0$.

		Step 2.4: $D x_\epsilon(\tau)|_{\epsilon = 0^+} \stackrel?= f(x(\tau), \eta) - f(x(\tau), u(\tau))$. Then
		$$ p^T(\tau)\(f(x(\tau), \eta) - f(x(\tau), u(\tau))\) ≤ 0 $$
		$$ \frac{1}{\epsilon}(x_\epsilon(\tau) - x(\tau)) = \frac{1}{\epsilon} \int_{\tau - \epsilon}^\tau \[f(x_\epsilon(s), \eta) - f(x(s), u(s))\] ds = $$
		$$ \frac{1}{\epsilon} \int_{\tau - \epsilon}^\tau \[f(x_\epsilon(s), \eta) - f(x(s), \eta)\] ds + \int_{\tau - \epsilon}^\tau \[f(x(s), \eta) - f(x(s), u(s))\] ds. $$
		Fist converge to zero from Lebesgue theorem about average value. Second to $f(x(\tau), \eta) - f(x(\tau), u(\tau)) \rightarrow 0$.
	\end{dukazin}
\end{veta}

\begin{veta}[Potrjagin for fixed point („fixed finish“)]
	Same as previous, but $T$ is not fixed, $x(T)$ is fixed $\implies$ $g ≡ 0$ (we don't "rate" final point, because it's the same for all $u$).
\end{veta}

\section{Bifurcation}
\begin{definice}
	$x' = \mu - x^2$ is saddle-node bifurcation, $x' = \mu x - x^2 = x(\mu - x)$ is transcritical bifurcation, $x' = \mu x - x^3 = x(\mu - x^2)$ is fork bifurcation, in $x' = x - \sin \mu$ there is no bifurcation.
\end{definice}

\begin{pozorovani}
	$f(x_0, \mu_0) ≠ 0 \implies $ no bifurcation. (From lemma of rect.) (Bifurcation $\implies f = 0$.)
\end{pozorovani}

\begin{pozorovani}
	$$ f(x_0, \mu_0) = 0, \sigma(\nabla_x f(x_0, \mu_0)) = \{\lambda_j | \Re \lambda_j ≠ 0\}. $$
\end{pozorovani}

\begin{definice}
	Point from previous observation is called hyperbolic stationary point.
\end{definice}

\begin{veta}
	$f: ®R^{n+1} \rightarrow ®R^n$ be $C^1$, $(x_0, \mu_0)$ is a hyperbolic stationary point. Then $\exists \Delta > 0$ $\exists \delta > 0$ $\forall \mu \in U_\delta(\mu_0)$ $\exists x = x(\mu) \in U_\Delta(x_0)$, stationary point $x(\mu)$ is a hyperbolic stationary point of $\mu \mapsto x(\mu)$, which is $C^1$.

	\begin{dukazin}
		IFT (Implicit function theorem):
		$$ f(x_0, \mu_0) = 0 \land \nabla_x f(x_0, \mu_0) \text{regular?} \land f \in C^1 \implies x = x(\mu), f(x(\mu), \mu) = 0. $$
		Hyperbolic? Eigenvalues of $A = \nabla_x f(x(\mu), \mu)$, $\det(\lambda I - A(\mu))$ – polynomial in $\lambda$, $\deg = n$.
	\end{dukazin}
\end{veta}

\begin{veta}
	$f: ®R^2 \rightarrow ®R$ be $C^2$ on neighborhood $(0, 0) \in ®R^2$.
	$$ f(0, 0) = 0, \qquad f_\mu(0, 0) ≠ 0, \qquad f_x(0, 0) = 0, \qquad f_{xx}(0, 0) ≠ 0. $$
	Then $f$ has bifurcation at $(0, 0)$ of the type saddle-node.

	\begin{dukazin}
		Without proof.
	\end{dukazin}
\end{veta}

\begin{veta}
	$f: ®R^2 \rightarrow ®R$ be $C^2$ on neighborhood $(0, 0) \in ®R^2$.
	$$ f(0, 0) = 0, \quad f_x(0, 0) = 0, \quad f(0, \mu) = 0\ \forall \mu \in U_\delta(0), \quad f_{xx}(0, 0) ≠ 0, \quad f_{x\mu}(0, 0) ≠ 0. $$
	Then $f$ has bifurcation at $(0, 0)$ of the type transcritical. ($f(0, \mu) = 0 \implies f_\mu(0, 0) = 0$.)

	\begin{dukazin}
		Without proof.
	\end{dukazin}
\end{veta}

\begin{lemma}[About division]
	$h: U(0, 0) \rightarrow ®R$ be $C^k$ for some $k \in ®N$. $h(0, \lambda) = 0$ $\forall \lambda \in U_\delta(0)$. Then
	$$ h(x, \lambda) = x H(x, \lambda), H \in C^{k - 1}(U(0, 0), ®R). $$
	$$ H(0, 0) = h_x(0, 0), \qquad H_x(0, 0) = \frac{1}{2} h_{xx}(0, 0), \qquad H_\lambda(0, 0) = h_{x\lambda}(0, 0), $$
	$$ H_{xx}(0, 0) = \frac{1}{3} h_{xxx}(0, 0), $$
	if $k$ is sufficiently large.

	\begin{dukazin}
		$$ H(x, \lambda) := \int_0^1 \partial_x h(\sigma x, \lambda) d\sigma. $$
	\end{dukazin}
\end{lemma}

% 06. 12. 2022

\begin{dukaz}[Theorem of transcritical bifurcation]
	$f(x, \mu) = x F(x, \mu)$. $F_\mu(0, 0) ≠ 0$?

	$F(x, \mu(x)) = 0$? $\rightarrow \frac{d}{dt}: \mu'(x) = \frac{-\partial_x F(x, \mu(x))}{\partial_\mu(F(x, \mu(x)))}$.

	$$ f_{x\mu}(x, \mu) = F_\mu(x, \mu) + x F_{x\mu}(x, \mu) \implies F_\mu(0, 0) = f_{x\mu}(0, 0) ≠ 0. $$
%
%?	$$ f_x(x, \mu(x)) = F(x, \mu(x)) + x F_x(x, \mu(x)) + x F_\mu(x, \mu(x))·\mu'(x). $$
\end{dukaz}

\begin{veta}[Fork]
	$$ f \in C^3(U), \qquad f(0, 0) = f_x(0, 0) = f_{xx}(0, 0) = 0, \qquad f_{xxx}(0, 0) ≠ 0, $$
	$$ f(0, \mu) = 0\ \forall (0, \mu) \in U, \qquad (f_\mu(0, 0) = 0), \qquad f_{x\mu}(0, 0) ≠ 0. $$
	Then $f$ has bifurcation at $(0, 0)$ of type fork.

	\begin{dukazin}
		$\mu'(0) = 0$, $\mu''(0) = \frac{…}{-\partial_\mu F(x, \mu(x))}$, $\partial_{x, x} F(0, 0) = \frac{1}{3} f_{xxx}(0, 0) ≠ 0$.
		$$ \mu''(0) ≠ 0 \implies \mu''(x) \text{ doesn't change sign } \implies \mu(x) \text{ has a local extreme at } (0, 0). $$
	\end{dukazin}
\end{veta}

\begin{veta}[?]
	$$ \binom{x'}{y'} = f(x, y, \mu), f \in C^2, f(0, 0, \mu) = \binom{0}{0}, \sigma\(\nabla f(0, 0, \mu)\) = \{\alpha(\mu) ± i \omega(\mu)\} .$$
	$$ \alpha(0) = 0, \alpha'(0) ≠ 0, \omega(0) ≠ 0, \quad \alpha, \omega \in C^1. $$

	Then $\exists \delta > 0, \epsilon > 0: \mu = \mu(a), a \in (0, \epsilon) \mapsto \mu(a) \in(-\delta, \delta)$.

	$\forall a \in (0, \epsilon)$ $\exists$ nontrivial periodic solution passing through $(a, 0)$.

	\begin{dukazin}
		Rotation: $x' = \alpha(\mu) x - \omega(\mu) y + f_1(x, y, \mu)$, $y' = \omega(\mu) x + \alpha(\mu) y + f_2(x, y, \mu)$, $f_2(x, y, \mu) = O(x^2 + y^2)$.

		Polar coords:
		$$ x = r \cos \theta, y = r \sin \theta, x' = g_1(x, y), y' = g_2(x, y), $$
		$$ r'\cos \theta - r \sin\theta ·\theta' = g_1, \qquad r' \sin \theta + r \cos \theta · \theta' = g_2. $$
		$$ r' = g_i · \cos \theta + g_2 \sin \theta, \qquad r·\theta' = -g_1 \sin \theta + g_2 \cos \theta. $$
		$$ r; = \alpha·r + \underbrace{f_1·\cos \theta + f_2·\sin\theta}_{=R}, \qquad r \theta' = \omega(\mu)·r + \underbrace{(- f_1·\sin\theta+f_2·\cos \theta)}_{=r·Q}. $$

		$$ r' = \alpha(\mu) r + R(r, \theta, \mu), R = O(r^2), \quad \theta' = \Omega(\mu) + Q(r, \theta, \mu), Q = O(r). $$

		WLOG $\omega(0) > 0$. $\exists \epsilon, \delta > 0$ $\forall r ≤ \epsilon$ $\forall \mu \in [-\delta, \delta]$, $\theta'(t) > 0$. $r(t) \mapsto \hat{r}(\theta) := r(t(\theta))$. $t \mapsto \theta(t)$ is simple $\implies$ $\exists t = t(\theta)$.
		$$ \frac{dr}{d\theta} = \frac{\frac{dr}{dt}}{\frac{d\theta}{dt}} = \frac{\alpha(\mu) r + R}{\omega(\mu) + Q} = \frac{\alpha(\mu)}{\omega(\mu)}r + T(r, \theta, \mu). $$

		$$ \lambda(\mu) := \frac{\alpha(\mu)}{\omega(\mu)}: r'(\theta) = \lambda(\mu) r(\theta) + T(r, \theta, \mu), $$
		$$ (r(\theta) e^{-\lambda(\mu)\theta})' = T(r, \theta, \mu) e^{-\lambda(\mu)\theta} $$
		$$ r(\theta)·e^{-\lambda(\mu) \theta} = r'(\theta_0)·e^{-\lambda(\mu)\theta_0} + \int_{\theta_0}^\theta T(r(\psi), \psi, \mu)·e^{-\lambda(\mu) \psi} d\psi. $$
		We get $r(\theta_0) = r(\theta_0 + 2\pi)$ – periodicity. If we denote $r(\theta_0) = a$, we get
		%$$ a = a·e^{2\pi \lambda(\mu)} + \int_{\theta_0}^{\theta_0 + 2\pi} T(r(\psi), \psi, \mu)·e^{-\lambda(\mu)(\psi - \theta_0)} d\psi. $$
		$$ \(e^{2\pi \lambda(\mu)} - 1\) a + \int_{\theta_0}^{\theta_0 + 2\pi} T(r(\psi), \psi, \mu)·e^{-\lambda(\mu)(\psi - \theta_0)} d\psi. $$

		$h_\mu(0, 0) ≠ 0$? $h''(a, \mu)$, $a = 0 \implies r = 0$, $T = 0$, $h(0, \mu) = 0$.
		$$ h(a, \mu) =: a·H(a, \mu). $$
		$$ H(0, 0) = 0? \quad H_\mu(0, 0) ≠ 0? \quad H \in C^1 \quad H(0, 0) = \partial_? h(0, 0). $$
	\end{dukazin}
\end{veta}

% 13. 12. 2022

\newpage

\section{Central manifolds}
\begin{poznamka}
	$$ x' = A x + f(x, y), \qquad y' = B y + g(x, y), $$
	$$ \sigma(B) \subset \{z \in ®C | \Re z < -\beta\}, \quad \beta > 0, \quad \sigma(A) \subset \{z \in ®C | \Re z ≥ 0\}, $$
	$$ x \in ®R^n, \quad y \in ®R^n, \quad f, g \in C^1(®R^{n + m}), $$
	$$ f(0) = 0, \quad g(0) = 0, \quad \nabla f(0) = 0, \quad \nabla g(0) = 0, $$
	$$ |f| ≤ \rho, \quad |g| ≤ \rho, \quad |\nabla f| ≤ \sigma, \quad |\nabla g| ≤ \sigma. $$

	Goal: $\exists \phi: ®R^n \rightarrow ®R^m$ Lipschitz, $\phi(0) = 0$, $\nabla \phi(0) = 0$:

	INV: if $(x(t), y(t))$ solution to previous and $y(0) = \phi(x(0))$, then $\forall t: y(t) = \phi(x(t))$.
\end{poznamka}

\begin{definice}[Reduced equation]
	$$ p'(t) = A·p(t) + f(p(t), \phi(p(t))), p(t) \in ®R^n. $$
\end{definice}

\begin{lemma}
	$\phi$ satisfies INV iff $\phi$ satisfies RED: (if $p(t)$ satisfies reduced equation then $(x(t), y(t)) := (p(t), \phi(p(t)))$ satisfies INV equation)

	\begin{dukazin}
		Straightforward.
	\end{dukazin}
\end{lemma}

\begin{definice}[RED']
	If $p(t)$ satisfies reduced equation, then $y(t):=\phi(p(t))$ satisfies
	$$ y'(t) = B y(t) + g(p(t), \phi(p(t))). $$
\end{definice}

\begin{lemma}
	$\gamma(t)$ is bounded on $(-∞, 0]$. Then $\exists! y(t): y'(t) = B y(t) + \gamma(t)$, such that $y(t)$ is bounded on $(-∞, 0]$. For this $y$, $y(0) = \int_{-∞}^0 e^{-B s} \gamma(s) ds$.

	\begin{dukazin}
		$$ e^{-Bt} y'(t) - B e^{-Bt} y(t) = e^{-Bt} \gamma(t). (e^{-B t} y(t))' = e^{-Bt} \gamma(t). $$
		$$ e^{-Bt} y(t) = y(0) + \int_0^t e^{-Bs} \gamma(s) ds. $$
		If $y$ is bounded on $(-∞, 0]$, then $y(0) + \int_0^{-∞} e^{-B s} \gamma(s) ds = 0$, $y(0) = \int_{-∞}^0 e^{-B s} \gamma(s) ds$.

		Take $y$ with (i. c.). Then
		$$ y(t) = e^{Bt} \(\int_{-∞}^t e^{-B s} \gamma(s) ds\) = \int_{-∞}^t e^{B(t - s)} \gamma(s) ds. $$
		$$ |e^{s·B}| ≤ c_0 e^{-\beta s}, \quad c_0 > 0, \forall s $$
		$$ |y(t)| ≤ \int_{-∞}^t |e^{B(t - s)}|·|\gamma(s)| ds ≤ \|\gamma\|_∞ \int_{-∞}^t c_0 e^{-\beta(t - s)} ds = \frac{\|\gamma\|_∞ c_0}{\beta}. $$
	\end{dukazin}
\end{lemma}

\begin{lemma}
	$\phi$ satisfies INV $\Leftrightarrow$ $\phi$ satisfies (RED) $\Leftrightarrow$ $\phi$ satisfies (RED') $\Leftrightarrow$ $\phi$ satisfies FP:
	$$ \forall p_0 \in ®R^n: \phi(p_0) = \int_{-∞}^0 e^{-B s} g(p(s), \phi(p(s))) ds, $$
	where $p$ satisfies reduced equation with $p(0) = p_0$.

	\begin{dukazin}
		„$\implies$“: $\phi$ RED' $\implies$ $y$ satisfies $y'(t) = B y(t) + g(p(t), \phi(p(t)))$ and $y$ is bounded. Then previous lemma:
		$$ \phi(p_0) = \phi(p(0)) = y(0) = \int_{-∞}^0 e^{-B·s} g(p(s), \phi(p(s))) ds. $$

		%„$\impliedby$“: $p(t)$ satisfies reduced equation. Why $(\phi(p(t)))' = B \phi(p(t)) + g(p(t), \phi(p(t)))$?
		%$$ e^{-B·t} \phi(p(t)) = e^{-B·t_0}\phi(p(t_0)) + \int_{t_0}^t e^{-B·s} g(p(s), \phi(p(s))) ds. $$
		„$\impliedby$“: Let $y$ satisfy $y'(t) = B y(t) + g(p(t), \phi(p(t)))$, $y(0) = \phi(p(0))$. $\forall t_1: y(t_1) = \phi(p(t_1))$?
		$$ t_1 \text{ arbitrary } y_1 := y(t + t_1), p_1(t) := p(t + t_1). p_1'(t) = A p_1(t) + f(p_1(t), \phi(p, t)), p_1(0) = p(t_1), $$
		$$ y_1'(t) = B·y_1(t) + g(p_1(t), \phi(p, t)), y_1(0) = y(t_1). $$

		$y(0) = \phi(p(0)) = \int_{-∞}^0 e^{-B·s} g(p(s), \phi(p(s))) ds$ $\implies$ $y$ is bounded on $(-∞, 0]$ $\implies$ $y_1$ is bounded. From lemma:
		$$ y(t_1) = y_1(0) = \int_{-∞}^0 e^{-B·s} g(p(s), \phi(p, s)) ds = \phi(p(0)) = \phi(p(t_1)). $$
		\vspace{-1em}
	\end{dukazin}
\end{lemma}

\begin{veta}[Existence of central manifold]
	$$ \forall \beta\ \exists \rho > 0, \sigma > 0, b > 0, l > 0: \exists! \phi \in ©X \text{ satisfying INV}. $$

	\begin{dukazin}
		$$ ©X \subset C(®R^n, ®R^n), \|\phi\|_{©X} = \sup_{x \in ®R^n} |\phi(x)|, $$
		$$ T: ©X \rightarrow ©X, \phi \mapsto T\phi, \quad (T\phi)(p_0) = \int_{-∞}^0 e^{-B·s} g(p(s), \phi(p(s))) ds. $$

		Step 1: $T$ is well-defined $\forall \phi \in ©X$: $T \phi \in ©X$.

		Step 2: $T$ is contraction.

		Step 1 + 2 $\implies$ (Banach) $\exists!\phi \in ©X$: $T\phi = \phi$.

		\begin{itemize}
			\item $(T \phi)(0) = 0$? Take $p(0)$ satisfying reduced equation, $p(0) = 0 \implies p(t) ≡ 0$.
			\item $|(T\phi)(p_0)| ≤ \int_{-∞}^0 |e^{-B·s}|·|g(p(s), \phi(p(s)))| ds ≤ \frac{\rho c_0}{\beta} \overset?≤ f$ (true for sufficiently small $\rho$).
			\item $(T \phi)(p_0) - (T \phi)(q_0) = \int_{-∞}^0 e^{-B·s} \[g(p(s), \phi(p(s))) - g(q(s), \phi(q(s)))\] ds$.
		\end{itemize}
	\end{dukazin}
\end{veta}

\begin{definice}[Central manifold]
	$\phi: ®R^n \rightarrow ®R^m$ is called a central manifold if $\phi(0) = 0$, $\nabla \phi(0) = 0$, $\phi \in C^1$, it satisfies INV.
\end{definice}

\begin{definice}
	$$ M[\phi](x) = \nabla \phi(x) (A x + f(x, \phi(x))) - B \phi(x) - g(x, \phi(x)). $$
\end{definice}

\begin{dusledek}
	$\phi$ is a central manifold $\Leftrightarrow$ $M[\phi] = 0$.
\end{dusledek}

% 20. 12. 2022

\begin{poznamka}
	Dělal se podrobně důkaz Existence centrální variety.
\end{poznamka}


\end{document}
