\documentclass[12pt]{article}					% Začátek dokumentu
\usepackage{../../MFFStyle}					    % Import stylu



\begin{document}

% 04. 10. 2022

\section{Dynamické systémy}
\begin{definice}[Dynamický systém]
	$(\phi, \Omega)$, $\Omega \subset ®R^n$ otevřená, $\phi: ®R \times \Omega \rightarrow \Omega$ $\phi(t, x)$.
	
	\begin{itemize}
		\item $\phi(0, x) = x$;
		\item $\phi(t, \phi(s, x)) = \phi(t + s, x)$
		\item $\phi$ je spojité.
	\end{itemize}
\end{definice}

\begin{definice}[Orbit]
	$\gamma^+(x_0) = \{\phi(t, x_0) | t ≥ 0\}$ je pozitivní orbit.

	$\gamma^-(x_0) = \{\phi(t, x_0) | t ≤ 0\}$ je negativní orbit.

	$\gamma(x_0) = \{\phi(t, x_0) | t \in ®R\}$ je plný orbit.
\end{definice}

\begin{definice}[Pozitivně, negativně a úplně invariantní]
	$(\phi, \Omega)$ dynamický systém, $M \subset \Omega$.

	$M$ je pozitivně invariantní $≡$ $\forall x \in M: \gamma^+(x) \subset M$.

	$M$ je negativně invariantní $≡$ $\forall x \in M: \gamma^-(x) \subset M$.

	$M$ je úplně invariantní $≡$ $\forall x \in M: \gamma(x) \subset M$.
\end{definice}

\begin{poznamka}
	$\gamma^+(x_0)$ je pozitivně invariantní, $\gamma^-(x_0)$ je negativně invariantní a $\gamma(x_0)$ je úplně invariantní.
\end{poznamka}

\begin{definice}
	$$ \omega(x_0) = \{y \in \Omega | \exists \{t_k\}_{k=1}^∞, t_k \rightarrow ∞: \phi(t_k, x_0) \rightarrow y\}, $$
	$$ \alpha(x_0) = \{y \in \Omega | \exists \{t_k\}_{k=1}^∞, t_k \rightarrow -∞: \phi(t_k, x_0) \rightarrow y\}. $$
\end{definice}

\begin{poznamka}[To je ekvivalentní]
	$\omega(x_0) = \{y \in \Omega | \forall \epsilon > 0\ \forall T > 0\ \exists t ≥ T: |\phi(t_r, x_0) - y| < \epsilon\}$.
\end{poznamka}

\begin{lemma}
	$\omega(x_0) = \bigcap_{\tau ≥ 0} \overline{\gamma^+(\tau, x_0)}$.

	\begin{dukazin}
		„$\subseteq$“: $y \in \omega(x_0)$: $\forall \epsilon > 0\ \forall T\ \exists t ≥ T: |\phi(t, x_0) - y| < \epsilon$. Chceme:
		$$ \forall \tau ≥ 0\ \forall \epsilon > 0\ \exists z \in \gamma^+(\tau, x_0): |y - z| < \epsilon \Leftrightarrow $$
		$$ \Leftrightarrow \forall \tau ≥ 0\ \forall \epsilon > 0\ \exists s ≥ \tau, z = \phi(s, x_0): |y - \phi(s, x_0)| < \epsilon. $$

		„$\supseteq$“: $\forall \tau ≥ 0\ y \in \overline{\gamma^+(\tau, x_0)}$ $\implies$
		$$ \implies \forall \epsilon\ \exists s ≥ \tau: |\phi(s, x_0) - y| < \epsilon. $$
	\end{dukazin}
\end{lemma}

\begin{veta}[Vlastnosti $\omega$-limitní množiny]
	Nechť $(\phi, \Omega)$ je dynamický systém, $x_0 \in \Omega$. Potom
	
	\begin{enumerate}
		\item $\omega(x_0)$ je uzavřená, úplně invariantní.
		\item Pokud $\gamma^+(x_0)$ je relativně kompaktní v $®R^n$, pak $\omega(x_0) ≠ \O$, $\omega(x_0)$ je kompaktní, souvislá.
	\end{enumerate}

	\begin{dukazin}
		1. $\omega(x_0)$ je průnik uzavřených množin, tedy uzavřená. $y \in \omega(x_0)$ $\exists t_k \nearrow ∞$ $\phi(t_k, x_0) \rightarrow y$.

		$$ s_k = t_k + t \qquad \phi(s_k, x_0) = \phi(t_k + t, x_0) = \phi(t, \phi(t_k, x_0)) $$
		$$ t_k \rightarrow ∞, \phi \text{spojitá} \qquad \phi(s_k, x_0) = \phi(t, \phi(t_k, x_0)) \rightarrow \phi(t, y) $$

		2. $\exists K \subset ®R^n$ kompaktní $\gamma^+(x_0) \subset K$. a) pokud $t_n ≥ 0, t_n \rightarrow ∞ \{\phi(t_n, x_0)\}_{n=1}^∞$ omezená posloupnost $\implies \exists \{t_{n_k}\}_{k=1}^∞ \subset \{t_n\}_{n=1}^∞$, podposloupnost, $\exists y \in \Omega \phi(t_{n_k}, x_0) \rightarrow y$. Pak $y \in \omega(x_0)$.

		b) $\omega(x_0)$ je tedy úplná a omezená, takže kompaktní. c) ať $\omega(x_0)$ je nesouvislá, tedy $\omega(x_0) \subseteq U \cup V$, $U, V$ otevřené disjunktní neprázdné, $U, V \subseteq K$. Vezměme $y \in \omega(x_0) \cap U$, $z \in \omega(x_0) \cap V$. Nechť $t_n$ je posloupnost taková, že $\phi(t_{2n} x_0) \rightarrow y$, $\phi(t_{2n + 1}, x_0) \rightarrow z$, $t_{2n} < t_{2n+1}$, $\phi(t_{2n}, x_0) \in U$, $\phi(t_{2n + 1}, x_0) \in V$. $F = K \setminus (U \cup V)$ uzavřená, tedy $\exists s_n \in (t_{2n}, t_{2n + 1}): \phi(s_n, x_0) \in F$. Tedy $\{\phi(s_n, x_0)\}$ je omezená posloupnost $\implies$ $\exists$ podposloupnost konvergující k $w \in F$.
	\end{dukazin}
\end{veta}

\begin{definice}[Topologická konjugovanost]
	$(\phi, \Omega)$, $\psi, \Theta$ dynamické systémy. $\exists: \Omega \rightarrow \Theta$ homeomorfismus (bijekce, spojité, spojitá inverze):
	$$ \forall x \in \Omega\ \forall t \in ®R \qquad h(\phi(t, x)) = \psi(t, h(x)). $$
\end{definice}

\begin{poznamka}
	Dá se zobecnit ještě zobrazováním časů.
\end{poznamka}

\begin{veta}[O rektifikaci]
	$\dot{x} = f(x), f(x_0) ≠ 0$, $(\phi, \Omega)$ příslušný dynamický systém. $\dot{y} = \begin{pmatrix} 1 \\ 0 \\ 0 \\ \vdots \\ 0 \end{pmatrix}$, $y(0) = 0$ a $(\psi, \Theta)$ je příslušný dynamický systém. Potom $(\phi, \Omega)$, $(\psi, \Theta)$ jsou lokálně topologicky konjugované ($\exists U$ okolí $x_0 \in \Omega$ a $V$ okolí $¦o \in ®R^n$ taková, že $\exists g: U \rightarrow V$ homeomorfismus $g(\phi(t, x)) = \psi(t, g(x))$ $\forall x \in U$, $\forall t: \phi(t, x) \in U$).

	\begin{dukazin}
		BÚNO $f_1(x_0) = \alpha ≠ 0$ (první souřadnice funkce $f$) a $x_0 = ¦o$. Buď $\tilde{V}$ okolí $¦o \in ®R^n$ $G: \tilde{V} \rightarrow ®R^n$, $G(y_1, …, y_n) = \phi(y, (0, y_2, …, y_n))$. Chceme ukázat, že $G$ je invertibilní na nějakém okolí.
		$$ \frac{\partial G(y_1, …, y_n)}{\partial y_1} |_{(0, …, 0)} = \frac{\partial \phi}{\partial t}(t = y_1, (0, y_2, …, y_n)) |_{y_1=0, …, y_n=0} = f(\phi(y_1 (0, y_2, …, y_n)))|_{y_1 = 0, …, y_n = 0} = f(\phi(0, (0, …, 0))) = f(x_0) = \alpha. $$
		$$ \frac{\partial G(y_1, …, y_n)}{\partial y_j} |_{(0, …, 0)} = \lim_{h \rightarrow 0} \frac{G(0, …, h, …, 0) - G(0, …, 0)}{h} = \lim_{h\rightarrow 0} \frac{(0, …, h, …, 0)^T - (0, …, 0)^T}{h} = (0, …, 1, …, 0)^T = e_j. $$
		Tedy $\nabla G(0, …, 0)$ je „jednotková matice, až na to, že $a_{11}$ je $\alpha$“, tudíž podle věty o inverzi funkce $\exists V \subseteq \tilde{V}$ okolí 0, $\exists U$ okolí bodu $x_0$ tak, že $G: V \rightarrow U$ je homeomorfismus. Položme $g = G^{-1}$.

		Nyní už stačí $g(\phi(t, x_0)) = \psi(t, g(x_0))$ $\forall x_0 \in U$ $\forall t: \phi(t, x_0) \in U$. $\phi(t, x_0) = G(\psi(t, g(x_0)))$

		3. $x \in U = G(V)$ $\exists y \in V$ $x = G(y)$
		$$ x = \phi(y, (x_{01}, x_{02} + y_2, …, x_{0n} + y_n)) $$
		$$ \phi(t, x) = \phi(t, \phi(y, (x_{01}, x_{02} + y_2, …, x_{0n} + y_n))) = \phi(t + y, (x_{01}, x_{02} + y_2, …, x_{0n} + y_n)) $$
	\end{dukazin}
\end{veta}

\begin{veta}[La Salle invariance principle]
	$$ x' = f(x), (\phi, \Omega) \quad \phi: ®R \rightarrow \Omega, f loc. Lip. $$
	$$ \exists V: \Omega \rightarrow ®R, \text{ bounded from below}. $$
	$$ \exists l \in ®R: \Omega_l = \{x \in \Omega | V(x) ≤ l\} -- bounded $$
	$$ \dot{V}_f(x) := \nabla V(x) · f(x) = \sum_{j=1}^n  \frac{\partial V(x)}{\partial x_j}·f_j(x) ≤ 0 \qquad \forall x \in \Omega_l. $$
	$$ R = \{x \in \Omega_l | \dot{V}_f(x) = 0\}, \quad M = \{y \in R | \gamma^+(y) \subset R\}. $$

	Then $\forall x \in \Omega_l: \omega(x) \subset M$.

	\begin{dukazin}
		Let $x \in \Omega_l$. $\forall y \in \omega(x)\ \exists t_k \nearrow ∞: x(t_k) \rightarrow y$. $\phi(t, x_0) = x(t)$.
		$$ \frac{d}{dt} V(x(t)) = \nabla V(x(t))·x'(t) = \dot{V}_f(x(t)) ≤ 0. $$
		$V(x(t)) \searrow$ and $\exists C: \forall x \in \Omega: V(x) > -C$ so $\exists \lim_{t \rightarrow ∞} V(x(t)) = c$.

		So $\exists c\ \forall y \in \omega(x_0) V(y) = c$. $V(x(t_k)) \rightarrow V(y) = c$.
		$$ \gamma^+(y) \subset \omega(x_0)\ V(\phi(t, y)) = c\ \forall t ≥ 0 \implies $$
		$$ \implies \frac{d}{dt}V(\phi(t, y)) = 0. $$
		$\gamma^+(y) \subset R$ in particular, $y \in R$. Hence $y \in M$.
	\end{dukazin}
\end{veta}

\section{Poincaré-Bendixson theory}
\begin{veta}[Poincaré-Bendixson]
	Let $p \in \Omega$, $\Omega$ open connected. $\omega(p)$ doesn't contain stat points and $\gamma^+(p)$ is relatively compact ($\gamma^+(p)$ is compact). Then $\omega(p) = \Gamma$-periodic orbit.
\end{veta}

\begin{veta}[Bendixon-Dulas]
	$\Omega$-simply connected ($\forall$ closed Jordan curve $\gamma$ in $\Omega$, $\Int(\gamma) \subset \Omega$). $\exists B: \Omega \rightarrow ®R: (\Div B)(x) = \frac{\partial B}{\partial x_1}(x_1, x_2) + \frac{\partial B}{\partial x_2}(x_1, x_2) > 0$ for almost every $x \in \Omega$. Then $x' = f(x)$ doesn't have nontrivial periodic solutions.
\end{veta}

\begin{definice}[Transverzála]
	$\Sigma$ segment on a line such that $\forall p \in \Sigma: \Sigma \not\parallel f(p)$.
\end{definice}

\begin{lemma}
	$\Sigma$ transverzála, $p \in \Sigma \subset \Omega$. Then $\exists \tilde \subset U$ neighborhood of $p$. $\exists \Delta > 0$ such that
	$$ \forall y \in \tilde U: \phi(t, y) \subset U\ \forall t: |t| < \Delta \land \exists \tau: |\tau| < \frac{\Delta}{2}: \phi(\tau, y) \in \Sigma \cap \tilde U. $$

	\begin{dukazin}
		Use Th. of rect.
	\end{dukazin}
\end{lemma}

\begin{lemma}
	Let $p \in \Omega$ and assume that $|\gamma^+(p) \cap \Sigma| ≥ 3$, i. e. $\exists t_1 < t_2 < t_3$ $\phi(t_j, p) \in \Sigma$, $j=1, 2, 3$. Then $\phi(t_2, p)$ lie between $\phi(t, p)$ and $\phi(t_3, p)$.
\end{lemma}

% 18. 10. 2022

TODO!!!

% 25. 10. 2022

TODO!!!

% 01. 11. 2022

\subsection{Controllability}
\begin{definice}[Control theory]
	$$ x' = f(x, u), f: \Omega \times U, \Omega \subset ®R^n, U \subset ®R^n, $$
	$$ u \in ©U := \{u: [0, T] \rightarrow ®R^n | \text{measurable}, ||u||_∞ < ∞\} = L^∞(0, T, ®R^n). $$
	(©U is admissible functions).
\end{definice}

\begin{definice}[Linear task]
	$x' = Ax + Bu$, $A, B \in ®R^{n \times m}$, $m < n$.
\end{definice}

\begin{definice}
	$x_0 \underset{u(0)}{\overset{t}\rightarrow} 0$ iff $x(0) = x_0$, $x(t) = 0$.
\end{definice}

\newcommand{\converg}{\underset{u(0)}{\overset{t}\rightarrow}}
\begin{definice}[Area of controlability]
	$$ ©R(t) = \{x_0 \in ®R^n | \exists u \in L^∞(0, t, ®R^n): x_0 \underset{u(0)}{\overset{t}\rightarrow} 0 \} $$
\end{definice}

\begin{definice}[Kalman matrix]
	$$ ©K(A, B) := (B | AB | A^2B | … | A^{n-1}B) $$
\end{definice}

\begin{veta}
	For linear problem $©R(t) = \LO (g_1, g_2, …, g_{n·m})$, where $©K(A, B) = (g_1 | g_2 | … | g_{n·m})$

	\begin{tvrzeniin}[Observation]
		$x(t) = e^{At} x_0 + \int_0^t e^{A(t - s) B u(s) ds}$.
		$$ x_0 \converg 0 \Leftrightarrow x(t) = 0 \Leftrightarrow x_0 = - \int_0^t e^{-A s} B u(s) ds \qquad (KO) $$
	\end{tvrzeniin}

	\begin{lemmain}[1]
		$$ A^k \in \LO(I, A, A^2, …, A^{n - 1}), k \in ®N_0 $$

		\begin{dukazin}
			Cayley-Hamilton.
		\end{dukazin}
	\end{lemmain}

	\begin{dukazin}
		1) $©R(t)$ is vector subspace of $®R^n$ from definition $x_0 + x_1 \underset{(u_1 + u_2)(0)}{\overset{t}\rightarrow} 0$, $\alpha x \underset{\alpha u(0)}{\overset{t}\rightarrow} 0$.

		2) We want $©R(t)^\perp = (\LO(g_1, …, g_n))^\perp$. „$\supseteq$“: $p \in (\LO(g_1, …, g_n))^\perp$. $x_0 \in ©R(t)$ arbitrary. From KO:
		$$ 0 \overset?= p^T x_0 = -\int_0^t p^Te^{-As} B u(s) ds = - \int_0^t \sum_{k=0}^∞ \frac{(-s)^k}{k!} p^T A^k B u(s) ds$$
		We know $(p, g_j) = 0$, $p^Tg_j = 0$, $p^T ©K(A, B) = 0$, $p^T A^k B = 0$, $k \in [n-1]$. And from lemma 1 $k \in ®N$.
		„$\subseteq$“: $p \in ®R^n$, $p \in ©R(t)^\perp$. We want to prove $p \perp B, AB, A^2B, …, A^{n-1}B$. $B = (b_1 | … | b_m)$. $\forall j \in [n]: p \perp b_j, Ab_j, …, A^{n-1}b_j$. $\phi \in L^∞(0, T, ®R)$, $u(t) = \phi(t)·¦e_j$, where $x_0 = - \int_0^t e^{-A s} B u(s) ds$. We have $x_0 \converg 0$, hence $x_0 \in ©R(t)$.
		$$ 0 = p^T x_0 = - p^T\int_0^t e^{-As} B u(s) ds = - \int_0^t p^T e^{-As}b_j \phi(s) ds \implies y(s) := p^T e^{-As}b_j ≡ 0 $$
		So we have $p^T e^{-As} b_j ≡ 0$, we derivate it, $p^T A^n e^{-As} b_j ≡ 0$, and set $s = 0$.
	\end{dukazin}
\end{veta}

\begin{dusledek}
	$©R(t)$ doesn't depend on time.
\end{dusledek}

\begin{definice}[Locally and globally controllable]
	Linear problem is called locally controllable, iff $\exists \delta > 0: \{x_0 \in ®R^2 |\ |x_0| < \delta\} \subset ©R(t)$. And globally if $®R^n = ©R(t)$.
\end{definice}

\begin{dusledek}
	Linear problem is controllable $\Leftrightarrow$ $\rank K(A, B) = n$.
\end{dusledek}

\subsection{Observability}
\begin{definice}[System for observability]
	$$ x' = f(x), x(0) = x_0, f: \Omega \subset ®R^n \rightarrow ®R^n y = g(x), g: \Omega \subset ®R^n \rightarrow ®R^m, m < n. $$
\end{definice}

\begin{definice}
	We say that system $x' = f(x)$ is observable through $g(·)$ on $[0, t]$, iff $\forall x_1(·), x_2(·): [0, T] \rightarrow ®R^n: g(x_1(t)) = g(x_2(t))$ $\forall t \in [0, T]$ $\implies x_1(0) = x_2(0)$.
\end{definice}

\begin{definice}[Linear observability]
	$x' = Ax$, $y = Bx$, $A \in ®R^{n \times n}$, $B \in ®R^{m \times n}$.
\end{definice}

\begin{veta}
	$x' = Ax$ is observable on $[0, T]$ through $y = B x$ $\Leftrightarrow$ $x' = A^T x + B^T u$ is controllable.

	\begin{dukazin}
		(We will prove equivalence with $\rank ©K(A^T, B^T) = n$.) „$\impliedby$“: For contradiction
		$$ \exists x_1(t), x_2(t), [0, T], B x_1(t) ≡ Bx_2(t): x(t) = x_1(t) - x_2(t), x(0) = x_0 ≠ 0, B x(t) ≡ 0. $$
		$$ x(t) = e^{At} x_0, Bx(t) = B^{At} x_0 ≡ 0 \qquad \forall t \in [0, T]. $$
		We differentiate it, set $t = 0$ and get $B x_0 = 0$, $B A x_0 = 0$, …, $BA^{n-1}x_0 = 0$. So $x_0^T B^T = 0$, …, $x_0^T \(A^T\)^{n-1} B^T = 0$. $x_0^T ©K(A^T, B^T) = 0$, $x_0 \perp ©K(A^T, B^T)$, \lightning.

		„$\implies$“: For contradiction $\rank(A^T, B^T) < n$ $\implies$ $\exists x_0 ≠ 0: x_0^T ©K(A^T, B^T) = 0$. $x_0^T\(A^T\)^k B^T = 0$ $\forall k \in [n - 1]$ and from lemma 1 $\forall k \in ®N$. $B A^T x_0 = 0$. $B e^{At} x_0 = 0$ $\forall t \in [0, T]$. \lightning.
	\end{dukazin}
\end{veta}

\end{document}
