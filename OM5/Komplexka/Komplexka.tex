\documentclass[12pt]{article}					% Začátek dokumentu
\usepackage{../../MFFStyle}					    % Import stylu



\begin{document}

% 29. 09. 2022
\section*{Úvod}

\begin{poznamka}
	Mluvilo se o historii ®C.
\end{poznamka}

\begin{definice}[Prostor ®C]
	Prostor ®C komplexních čísel je prostor $®R^2$, v němž navíc definujeme násobení:
	$$ (x_1, y_1) \cdot (x_2, y_2) = (x_1·x_2 - y_1·y_2, x_1·y_2 + x_2·y_1). $$

	Ztotožníme $(x, 0) = x$, neboli $®R \subset ®C$. Značíme $i := (0, 1)$ (imaginární jednotka).
\end{definice}

\begin{definice}[Značení (komplexně sdružené číslo, reálná a imaginární složka)]
	$$ z = x + i·y \in ®C \implies \overline{z}:= x - i·y \land \Re z := x, \Im z := y. $$
\end{definice}

\begin{definice}[Modul / absolutní hodnota]
	$$ |z| := \sqrt{x^2 + y^2} $$
\end{definice}

\begin{tvrzeni}[Vlastnosti]
	\ 
	\begin{itemize}
		\item $z = (x, y) \in ®C$, potom $z = x + i·y$ a $(±i)^2 = -1$.
		\item Násobení $·: ®C^2 \rightarrow ®C$ je asociativní, komutativní a distributivní (vzhledem k $+$). Navíc $·$ zahrnuje i násobení v ®R a násobení skalárem.
		\item $|z|^2 = z \overline{z}$, $\overline{z_1 · z_2} = \overline{z_1}·\overline{z_2}$, $|z_1·z_2| = |z_1|·|z_2|$, $z + \overline{z} = 2 \Re z$, $z - \overline{z} = 2i\Im z$, $z, z_1, z_2 \in ®C$.
		\item $\forall z \in ®C, z ≠ 0: \exists z^{-1} \in ®C: z z^{-1} = 1$, konkrétně $z^-1 = \frac{\overline{z}}{|z|^2}$.
		\item $(®C, +, ·)$ je komutativní těleso.
	\end{itemize}
\end{tvrzeni}

\begin{upozorneni}
	®C nelze „rozumně“ lineárně uspořádat.
\end{upozorneni}

\begin{poznamka}[Lineární zobrazení]
	Lineární zobrazení $®R^2 \rightarrow ®R^2$ (®R-lineární zobrazení) jsou reálné matice řádu 2. Lineární zobrazení $®C \rightarrow ®C$ (®C-lineární) jsou komplexní čísla.

	Lineární zobrazení $L = \begin{bmatrix} a & c \\ b & d \end{bmatrix}$ je tedy ®C-lineární právě tehdy, když $a = d$ a $b = -c$.
\end{poznamka}

\begin{poznamka}[Úmluva]
	„Funkce“ znamená funkci z ®C do ®C, není li řečeno jinak.
\end{poznamka}

\begin{definice}[Značení (okolí, prstencové okolí)]
	$$ z_0 \in ®C, \epsilon > 0: ©U(z_0, \epsilon):=\{z \in ®C: |z - z_0| < \epsilon\}, ©P(z_0, \epsilon) := \{z \in ®C: 0 < |z - z_0| < \epsilon\}. $$
\end{definice}

\begin{definice}[Limita, spojitost]
	$$ \lim_{z \rightarrow z_0} f(z) = w \in ®C ≡ \forall \epsilon \exists \delta > 0 \forall z \in ®C: z \in ©P(z_0, \delta) \implies f(z) \in ©U(w, \epsilon). $$
	$f$ je spojitá v $z_0$, jestliže $\lim_{z \rightarrow z_0} f(z) = f(z_0)$.
\end{definice}

\begin{definice}[Derivace]
	$f: ®R^2 \rightarrow ®R^2$ je v bodě $z_0 \in ®R^2$ ®R-diferencovatelná, jestliže existuje ®R-lineární zobrazení $L: ®R^2 \rightarrow ®R^2$ takové, že
	$$ \lim_{h \rightarrow 0} \frac{f(z_0 + h) - f(z) - Lh}{|h|} = 0 $$
	Značíme $L =: df(z_0)$.
	$$ df(z_0) = \begin{bmatrix} \frac{\partial f_1}{\partial x}(z_0) & \frac{\partial f_1}{\partial y}(z_0) \\ \frac{\partial f_2}{\partial x}(z_0) & \frac{\partial f_2}{\partial y}(z_0) \end{bmatrix}  $$

	$f: ®C \rightarrow ®C$ je v bodě $z_0 \in ®C$ ®C-diferencovatelná, jestliže existuje
	$$ f'(z_0) := \lim_{h \rightarrow 0} \frac{f(z_0 + h) - f(z_0)}{h} \in ®C. $$
	$f'$ nazýváme komplexní derivace funkce.
\end{definice}

\begin{poznamka}
	Pro $(f±g)'$, $(f·g)'$, $(f / g)'$, $(f \circ g)'$ platí stejné vzorce jako pro funkce $®R \rightarrow ®R$.
\end{poznamka}

\begin{veta}[Cauchy-Riemann]
	Nechť $f$ je komplexní funkce definovaná na nějakém okolí bodu $z_0 \in ®C$. Pak následující podmínky jsou ekvivalentní:
	
	\begin{itemize}
		\item Existuje $f'(z_0)$.
		\item Existuje $d f(z_0)$ a $d f(x_0)$ je ®C-lineární.
		\item Existuje $d f(z_0)$ a platí
			$$ \frac{\partial f_1}{\partial x}(z_0) = \frac{\partial f_2}{\partial y}(z_0), \frac{\partial f_1}{\partial y}(z_0) = - \frac{\partial f_2}{\partial x}(z_0). $$
			(Tzv. Cauchy-Riemannova podmínka.)
	\end{itemize}

	\begin{dukazin}
		Druhý a třetí bod je ekvivalentní z poznámky o lineárních zobrazeních.

		$w = f'(z_0) \Leftrightarrow 0 = \lim_{h \rightarrow 0} \frac{f(z_0 + h) - f(z_0) - w h}{h}$. Vynásobíme $\frac{h}{|h|}$:
		$$ \Leftrightarrow 0 = \lim_{h \rightarrow 0} \frac{f(z_0 + h) - f(z_0) - wh}{|h|} \Leftrightarrow df(z_0) h = wh. $$
	\end{dukazin}
\end{veta}

\begin{poznamka}
	Existuje-li $f'(z_0)$, pak $df(z_0)h = f'(z_0)h, h \in ®C$ a $f'(z_0) = \frac{\partial f}{\partial x}(z_0)$.

	Cauchy-Riemannova podmínka je ekvivalentní $\frac{\partial f}{\partial x}(z_0) = -i \frac{\partial f}{\partial y}(z_0)$.
\end{poznamka}

% 06. 10. 2022

\begin{definice}[Holomorfní funkce]
	Nechť $G \subseteq ®C$ je otevřená a $f: G \rightarrow ®C$. Potom $f$ je holomorfní na $G$, pokud je $f$ ®C-diferencovatelná v každém bodě $G$.
\end{definice}

\begin{definice}[Exponenciála]
	$$ \exp(z) := e^x·(\cos y + i·\sin y), z = x + y·i \in ®C. $$
\end{definice}

\begin{tvrzeni}[Vlastnosti exponenciály]
	$\exp|_{®R}$ je reálná exponenciála, $\exp(z + w) = \exp(z)·\exp(w)$, $\exp'(z) = \exp(z)$ ($z \in ®C$), $\exp(z) = \sum_{n=0}^∞ \frac{z^n}{n!}$, $\exp(®C) = ®C \setminus \{ 0 \}$, $\exp$ není prostá na ®C a je $2\pi$ periodická, dokonce $\exp(z) = \exp(w) \Leftrightarrow \exists k \in ®Z: w = z + 2k\pi·i$, nechť $P := \{z \in ®C | \Im z \in (-\pi, \pi]\}$, potom $\exp|_P$ je prostá a $\exp(P) = ®C \setminus \{0\}$.
\end{tvrzeni}

\begin{definice}[Logaritmus a hlavní hodnota logaritmu]
	Nechť $z \in ®C \setminus \{0\}$. Položme
	$$ Log z := \{w \in ®C | \exp w = z\}, $$
	$$ \log z := \log|z| + i·\arg z. \qquad (\text{Hlavní hodnota logaritmu.}) $$
\end{definice}

\begin{tvrzeni}[Vladstnosti logaritmu]
	Nechť $z \in ®C \setminus \{0\}$. Potom

	\begin{itemize}
		\item $Log z = \{\log z + 2k\pi i | k \in ®Z\}$, $\log=(\exp|_P)^{-1}$
		\item $\log$ není spojitá na žádném $z \in (-∞, 0]$, ale $\log \in ©H(®C \setminus (-∞, 0])$. Navíc $\log' z = \frac{1}{z}$, $z \in ®C \setminus (-∞, 0]$
		\item $\log(1 - z) = - \sum_{n=1}^∞ \frac{z^n}{n}$, $|z| < 1$.
	\end{itemize}
\end{tvrzeni}

\begin{upozorneni}
	Neplatí $\log \exp z = z$ a $\log(z·w) = \log z + \log w$!
\end{upozorneni}

\begin{definice}
	Nechť $z \in ®C \setminus \{0\}$ a $\alpha \in ®C$. Potom hlavní hodnotou $\alpha$-té mocniny $z$ definujeme
	$$ z^{\alpha} := \exp(\alpha \log z). $$
	Položme
	$$ M_\alpha(z) := \{\exp(\alpha · w) | w \in Log z\}. $$
\end{definice}

\begin{tvrzeni}[Vlastnosti mocniny]
	\ 

	\begin{itemize}
		\item $e^z = \exp(z·\log e) = \exp(z)$.
		\item Je-li $z > 0$ a $\alpha \in ®R$, potom $z^\alpha$ je definována stejně jako v MA.
		\item $M_{\alpha}(z) = \{z^\alpha · e^{2k\pi · i · \alpha} | k \in ®Z\}$, $z ≠ 0$.
		\item $\(z^\alpha\) = \alpha·z^{\alpha - 1}$, $z \in ®C \setminus (-∞, 0]$, $\alpha \in ®C$.
		\item $(1 + z)^\alpha = \sum_{n=0}^∞ \binom{\alpha}{n} z^n$, $|z| < 1$, kde
			$$ \binom{\alpha}{n} := \frac{\alpha·(\alpha - 1)·…·(\alpha - n + 1)}{n!}, \qquad \alpha \in ®C. $$
	\end{itemize}
\end{tvrzeni}

\begin{poznamka}[Zápočet]
	Zápočet dostaneme za aktivní účast na cvičení
\end{poznamka}

% 13. 10. 2022

\begin{poznamka}
	Je-li $f: ®C \rightarrow ®C$, potom
	$$ f(x) = \frac{f(x) + f(-x)}{2} + \frac{f(x) - f(-x)}{2}, $$
	tedy $f$ lze rozložit na sudou a lichou část.

	Sudá část exponenciely je $\cosh$ a lichá $\sinh$.
\end{poznamka}

\begin{definice}[Goniometrické funkce]
	$$ e^{iz} = \cos z + i·\sin z, $$
	kde
	$$ \cos z := \frac{e^{iz} + e^{-iz}}{2}, \qquad \sin z := \frac{e^{iz} - e^{-iz}}{2i}, \qquad z \in ®C. $$
\end{definice}

\begin{tvrzeni}[Vlastnosti]
	\ 
	\begin{itemize}
		\item $\cos$ i $\sin$ jsou rozšířením funkcí z ®R do ®C.
		\item $\sin' z = \cos z$, $\cos' z = \sin z$.
		\item $\sin$ i $\cos$ jsou $2\pi$ periodické funkce, ale nejsou omezené, platí, že $\sin ®C = ®C = \cos ®C$.
		\item Platí $\sin^2 z + \cos^2 z = 1$.
		\item $\sin z = \sum_{n=0}^{∞} \frac{(-1)^n z^{2n + 1}}{(2n + 1)!}$, $\cos z = \sum_{n=0}^∞ (-1)^n \frac{z^{2n}}{(2n)!}$.
	\end{itemize}
\end{tvrzeni}

\section{Křivkový integrál}
\begin{definice}[Značení]
	Nechť $\phi:[\alpha, \beta] \rightarrow ®C$. Potom $\phi$ je křivka, pokud je $\phi$ spojité, $\phi$ je regulární křivka, pokud je $\phi$ po částech spojitě diferencovatelné tzn. $\phi$ je spojitá na $[\alpha, \beta]$ a existuje dělení $\alpha = t_0 < t_1 < … < t_n = \beta$ takové, že $\phi|_{[t_i, t_n]}$ je diferencovatelné.

	Úsečka: Nechť $a, b \in ®C$, potom $\phi(t) = a + t·(b - a)$, $t \in [0, 1]$ je úsečka z $a$ do $b$. Značíme $[a, b]$.

	Řekneme, že křivka $\phi$ je lomená čára v ®C, existují-li $z_1, …, z_k \in ®C$ taková, že
	$$ \phi = [z_1, z_2] + [z_2, z_3] + … + [z_{k-1}, z_k]. $$
\end{definice}

\begin{poznamka}[Úmluva]
	Pokud neřekneme něco jiného, křivkou budeme rozumět regulární křivku v ®C.
\end{poznamka}

\begin{definice}[Délka křivky]
	$$ V(\phi) = \int_\alpha^\beta |\phi'(t)| dt, $$
	je-li $\phi$ regulární.
\end{definice}

\begin{definice}
	Nechť $\phi: [\alpha, \beta] \rightarrow ®C$ je regulární křivka a $f: <\phi> \rightarrow ®C$ je spojitá. Potom definujeme
	$$ \int_\phi f := \int_\alpha^\beta f(\phi(t))·\phi'(t) dt. $$
\end{definice}

\begin{poznamka}
	Křivkový integrál konverguje jako Riemannův.

	$$ \int_\phi f(z) dz, $$
	Nechť $z_0 \in ®C$, $r \in (0, +∞)$ a $\phi(t) = z_0 + r e^{it}$, $t \in [0, 2\pi]$. Potom
	$$ \int_\phi (z - z_0)^n dz = \int_0^{2\pi} r^n e^{i n t}·2·r·e^{it} dt = i·r^{n+1} \int_0^{2\pi} e^{i(n + 1) t} dt = $$
	$2\pi i$, pokud $n = -1$, 0, pokud $n \in ®Z$ a $n ≠ -1$.
\end{poznamka}

\begin{tvrzeni}[Vlastnosti křivkového integrálu]
	Je-li $\phi$ křivka, $f$ a $g$ jsou spojité funkce na $<\phi>$ a $A \in ®C$, ptotom
	$$ \int_\phi (Af + g) = A\int_\phi f + \int_\phi g. $$

	Je-li $\phi$ křivka a $f$ je spojitá funkce na $<\phi>$, potom
	$$ |\int_\phi f| ≤ \max_{<\phi>} |f|·V(\phi). $$

	Nechť $\phi: [\alpha, \beta] \rightarrow ®C$, $\psi: [\gamma, \delta] \rightarrow ®C$ a $\phi(\beta) = \psi(\gamma)$. Potom
	$$ \int_{\phi + \psi} f = \int_\phi f + \int_\psi f \land \int_{-\phi} f = - \int_\phi f, $$
	kde $(-\phi)(t) := \phi(-t)$, $t \in [-\beta, -\alpha]$ je opačná křivka k $\phi$.

	Křivkový integrál nezávisí na parametrizaci křivky: Nechť $\phi: [\alpha, \beta] \rightarrow ®C$ je křivka, $\omega: [\gamma, \delta] \rightarrow [\alpha, \beta]$ je spojitě diferencovatelné s $\omega' > 0$ a $\psi = \phi\circ\omega$. Potom $\int_\psi f = \int_\phi f$.

	\begin{dukazin}
		Jednoduchý, ukázán na přednášce pro některé body.
	\end{dukazin}
\end{tvrzeni}

\begin{definice}[Primitivní funkce]
	Řekneme, že funkce $f$ má na otevřené $G \subset ®C$ primitivní funkci $F$, pokud $F' = f$ na $G$.
\end{definice}

\begin{veta}[O výpočtu křivkového integrálu pomocí primitivní funkce]
	Nechť $G \subset ®C$ je otevřená a $f$ má na $G$ primitivní funkci $F$. Nechť $\phi:[\alpha, \beta] \rightarrow G$ je regulární křivka a $f$ je spojitá\footnote{Tohle je zbytečný předpoklad, ale to ještě neumíme dokázat.} na $<\phi>$. Potom
	$$ \int_\phi f = F(\phi(\beta)) - F(\phi(\alpha)), $$
	je-li navíc $\phi$ uzavřená, tzn. $\phi(\alpha) = \phi(\beta)$, pak
	$$ \int_\phi f = 0. $$

% 20. 10. 2022 Z poznámek
% 27. 10. 2022 Z poznámek

	\begin{dukazin}
        Z Cauchy-Riemannovy věty plyne, že
        $$ \frac{d}{dt}(F(\phi(t))) = \frac{\partial F}{\partial x} \phi_1' + \frac{\partial F}{\partial y} \phi_2' = F'\phi_1' + i F'\phi_2' = F'(\phi(t))\phi'(t). $$

        Tato rovnost platí až na konečně mnoho $t \in [\alpha, \beta]$, neboli $F \circ \phi$ je zobecněná primitivní funkce k integrandu. Máme tedy
        $$ \int_\phi f = \int_\alpha^\beta f(\phi(t))\phi'(t)dt = \int_\alpha^\beta \frac{d}{dt} (F(\phi(t))) dt = F(\phi(\beta)) - F(\phi(\alpha)). $$
	\end{dukazin}
\end{veta}

\begin{veta}
	Funkce $f$ je konstantní na oblasti $G \subset ®C$, právě když $f' = 0$ na $G$.

	\begin{dukazin}
		„$\implies$“: Jasné. „$\impliedby$“: Nechť $z, w \in G$ a $\phi$ je lomená čára v $G$ spojující $z$ a $w$. Potom $f(w) - f(z) = \int_\phi f' = 0$, protože $f$ je primitivní funkcí k $f'$ na $G$.
	\end{dukazin}
\end{veta}

\begin{dusledek}
	Jsou-li $F_1$, $F_2$ primitivní funkce k $f$ na oblasti $G \subset ®C$, potom existuje $c \in ®C$ tak, že $F_2 = F_1 + c$

	\begin{dukazin}
		$$ (F_2 - F_1)' = F_2' - F_1' = f - f = 0 $$
	\end{dukazin}
\end{dusledek}

\begin{veta}[O existenci primitivní funkce]
	Nechť $G \subset ®C$ je oblast a $f$ je spojitá na $G$, tak následující je ekvivalentní
	\begin{enumerate}
		\item $f$ má na $G$ primitivní funkci;
		\item $\int_\phi f = 0$ pro každou uzavřenou křivku $\phi$ v $G$;
		\item $\int_\phi f$ nezávisí v $G$ na křivce $\phi$, tzn. pro každé dvě křivky $\phi: [\alpha, \beta] \rightarrow G$, $\psi: [\gamma, \delta] \rightarrow G$ takové, že $\phi(\alpha) = \psi(\gamma)$ a $\phi(\beta) = \psi(\delta)$ platí $\int_\phi f = \int_\psi f$.
	\end{enumerate}

	\begin{dukazin}
		„$1. \implies 2.$“: Víme z věty o výpočtu integrálu pomocí primitivní funkce.

		„$2. \implies 3.$“: Položme $\tau := \phi + (- \psi)$. Potom je $\tau$ uzavřená a z 2. dostaneme
		$$ 0 = \int_\tau f = \int_\phi f - \int_\psi f. $$

		„$3. \implies 1.$“: Volme $z_0 \in G$ pevně. Pro každé $z \in G$ najděme lomenou čáru $\phi_z$ v $G$, která začíná v $z_0$ a končí v $z$. Definujeme $F(z) := \int_{\phi_z} f$, $z \in G$. Definice $F$ je korektní z 3. Ukážeme, že $F$ je hledaná primitivní funkce k $f$ na $G$. Nechť $z_1 \in G$. Dokážeme, že $F'(z_1) = f(z_1)$. Volme $r > 0$, aby $U(z_1, r) \subset G$. Je-li $|h| < r$, potom
		$$ F(z_1 + h) - F(z_1) = \int_{\phi_{z_1} + u} f - \int_{\phi_{z_1}} f = \int_u f, $$
		kde $u = [z_1; z_1 + h]$ je úsečka. Tedy
		$$ F(z_1 + h) - F(z_1) = \int_u f = \int_0^1 f(z_1 + th)h dt, $$
		$$ \frac{F(z_1 + h) - F(z_1)}{h} - f(z_1) = \int_0^1 (f(z_1 + th) - f(z_1))dt \rightarrow 0, $$
		neboť $|\int_0^1 (f(z_1 + th) - f(z_1))dt| ≤ \max_{z \in [z_1; z_1+h]} |f(z) - f(z_1)| \rightarrow 0$ ze spojitosti $f$ v $z_1$.
	\end{dukazin}
\end{veta}

\begin{poznamka}[Značení]
	Řekněme, že $M \subset ®C$ je hvězdovitá, pokud existuje $z_0 \in M$ (tzv. střed hvězdovitosti), pro který $[z_0; z] \subset M$ pro každé $z \in M$.

	\begin{poznamkain}
		Konvexní $\subsetneq$ hvězdovitá.
	\end{poznamkain}

	Řekneme, že $\triangle \subset ®C$ je trojúhelník s vrcholy $a, b, c \in ®C$, pokud
	$$ \triangle := \{\alpha·a + \beta·b + \gamma·c | \alpha, \beta, \gamma ≥ 0, \alpha + \beta + \gamma = 1\}, $$
	a značíme $\partial \triangle := [a; b] + [b; c] + [c; a]$.
\end{poznamka}

\begin{tvrzeni}[Dodatek]
	Nechť $f$ je spojitá funkce na hvězdovité oblasti $G \subset ®C$. Je-li
	$$ \int_{\partial \triangle} f = 0, $$
	pro každý trojúhelník $\triangle \subset G$, potom $f$ má na $G$ primitivní funkci.

	\begin{dukazin}
		Nechť $z_0$ je střed hvězdovitosti $G$. Pro každé $z \in G$ položme $\phi_z := [z_0; z]$ a $F(z) := \int_{\phi_z} f$.
	\end{dukazin}
\end{tvrzeni}

\begin{poznamka}[Cauchyho věty]
	Nechť $G \subset ®C$ je otevřená, $f \in ©H(G)$ a $\phi$ je uzavřená křivka v $G$. Potom Cauchyho věty nám říkají, za jakých podmínek na $G$ a $\phi$ je $\int_\phi f = 0$.
\end{poznamka}

\begin{veta}[Goursatovo lemma (Cauchyho věta pro $\triangle$)]
	Nechť $G \subset ®C$ je otevřená, $f \in ©H(G)$ a $\triangle$ je trojúhelník v $G$. Potom
	$$ \int_{\partial \triangle} f = 0. $$

	\begin{dukazin}
		Označme $\phi_0 := \partial \triangle$. Sporem: Předpokládejme, že $|\int_{\phi_0} f| =: K > 0$. Zřejmě $\triangle$ je nedegenerovaný. V $\triangle$ veďme střední příčky a označme $\psi_1$, $\psi_2$, $\psi_3$, $\psi_4$ obvody čtyř vzniklých trojúhelníků. Obvody vnitřních trojúhelníků $\psi_1$ (vlevo dole), $\psi_2$ (vpravo dole), $\psi_3$ (nahoře) a $\psi_4$ (uprostřed) probíháme proti směru hodinových ručiček. Potom
		$$ \int_{\phi_0} f = \int_{\psi_1} f + \int_{\psi_2} f + \int_{\psi_3} f + \int_{\psi_4} f. $$
		Ex. $j_1 = 1, …, 4$ tak, že $|\int_{\psi_{j_1}} f| ≥ \frac{K}{4}$ a $V(\psi_{j_1}) = \frac{V(\phi)}{2}$. Označme $\phi_1 = \psi_{j_1}$. Indukcí sestrojíme posloupnost trojúhelníků tak, že
		$$ |\int_{\phi_j} f| ≥ \frac{K}{4^j} \land V(\phi_j) = \frac{V(\phi)}{2^j}. $$

		Máme, že $\bigcup_{j=0}^∞ \triangle_j = \{z_0\} \subset G$, protože $\diam(\triangle_j) \rightarrow 0$. Položme
		$$ \epsilon(z) := \begin{cases}\frac{f(z) - f(z_0)}{z - z_0} - f'(z_0), & z \in G\setminus \{z_0\},\\ 0, & z = z_0.\end{cases} $$

		Potom $\epsilon$ je spojitá na $G$ a máme pro $j \in ®N_0$:
		$$ \int_{\phi_j} f(z) dz = \int_{\phi_j} (f(z_0) + f'(z_0)(z - z_0))dz + \int_{\phi_j}\epsilon(z)(z - z_0) dz, $$
		kde první integrand vpravo má primitivní funkci na ®C a první integrál je roven 0. Pro každé $j \in ®N_0$ dostaneme
		$$ 0 < \frac{K}{4^j} ≤ |\int_{\phi_j} f| = |\int_{\phi_j} \epsilon(z)(z - z_0)| ≤ V^2(\phi_j)·\max_{<\phi_j>} |\epsilon| = \frac{V^2(\phi)}{4^j}·\max_{<\phi_j>}|\epsilon|, $$
		kde třetí nerovnost platí díky tomu, že $|z - z_0| ≤ V(\phi_j)$. Z předchozího tedy máme (po vynásobení $4^j$):
		$$ 0 < K ≤ V^2(\phi)·\max_{<\phi_j>}|\epsilon| \rightarrow 0, $$
		protože $\epsilon$ je spojitá v $z_0$ a $\epsilon(z_0) = 0$. \lightning.
	\end{dukazin}
\end{veta}

\begin{veta}[Cauchyho věta pro hvězdovité oblasti]
	Nechť $G \subset ®C$ je hvězdovitá oblast a $f \in ©H(G)$. Potom $f$ má na $G$ primitivní funkci. (Ekvivalentně: $\int_{\phi} f = 0$ pro každou uzavřenou křivku $\phi$ v $G$.)

	\begin{dukazin}
		Z Goursatova lemmatu a dodatku.
	\end{dukazin}
\end{veta}

\begin{poznamka}
	Goursatovo lemma a tedy i předchozí věta platí i pro funkci $f$, která je spojitá na $G$ a holomorfní na $G \setminus \{z_0\}$ pro nějaké $z_0 \in G$.

	\begin{dukazin}
		Nechť $\triangle$ je nedegenerovaný trojúhelník v $G$. Pak rozebereme případy kde leží $z_0$.
	\end{dukazin}
\end{poznamka}

\begin{veta}[O derivování podle komplexního parametru]
	Nechť $\phi$ je křivka v ®C a $\Omega \subset ®C$ je otevřená. Nechť $F(z, s)$ a komplexní derivace $\frac{\partial F}{\partial s}(z, s)$ jsou spojité komplexní funkce na $<\phi> \times \Omega$. Pro každé $s \in \Omega$ položme $\Phi(s) := \int_\phi F(z, s) dz$. Potom $\Phi \in ©H(\Omega)$ a $\Phi'(s) = \int_\phi \frac{\partial F}{\partial s}(z, s) dz$, $s \in \Omega$.

	\begin{dukazin}
		Pro $s = s_1 + i s_2 = (s_1, s_2) \in \Omega$ máme $\Phi(s) = \int_\alpha^\beta F(\phi(t), (s_1, s_2)) \phi'(t)dt$. Podle vět o spojitosti a derivování integrálu závislého na parametru máme
		$$ \frac{\partial \Phi}{\partial s_j}(s) = \int_\phi \frac{\partial F}{\partial s_j}(z, s) dz, $$
		pro $s \in \Omega$ a $j \in [2]$. Navíc jsou tyto parciální derivace spojité a splňují podmínky Cauchy-Riemannovy věty, tedy $\Phi$ je komplexně diferencovatelná a komplexní derivace se rovná derivaci vzhledem k té první proměnné.
	\end{dukazin}
\end{veta}

\begin{definice}[Index bodu křivky]
	Nechť $\phi$ je uzavřená křivka v ®C a $s \in ®C \setminus <\phi>$. Potom číslo
	$$ \ind_\phi s := \frac{1}{2\pi i} \int_\phi \frac{dz}{z - s} $$
	nazveme indexem bodu s vzhledem ke křivce $\phi$.
\end{definice}

\begin{poznamka}
	Ukážeme si, že $\ind_\phi$ se rovná počtu oběhů $\phi$ kolem $s$ v kladném směru (tzn. proti směru hodinových ručiček).
\end{poznamka}

\begin{veta}[O základních vlastnostech indexu]
	Nechť $\phi$ je uzavřená křivka v ®C a $G := ®C \setminus <\phi>$. Potom $G$ je otevřena, funkce $s \mapsto \ind_\phi s$ je konstantní na každé komponentě $G$ a na jediné její neomezené komponentě je nulová.

	\begin{dukazin}
		Podle předchozí věty je $\Phi(s) := \frac{1}{2 \pi i} \int_\phi \frac{dz}{z - s}$, $s \in G$ holomorfní a pro každé $s \in G$ je $\Phi'(s) = \frac{1}{2\pi i} \int_\phi \frac{dz}{(z - s)^2} = 0$, protože $f(z) := \frac{1}{(z - s)^2}$ má primitivní funkci na $®C \setminus \{s\}$. Tedy $\Phi$ je konstantní na každé komponentě $G$.

		Volíme $R > 0$, aby $<\phi> \subset U(0, R)$. Potom $®C \setminus U(0, R)$ je obsaženo v jediné neomezené komponentě $G_0$ množiny $G$. Navíc pro $s \in ®C \setminus U(0, R)$ je funkce $g(z) := \frac{1}{z - s}$, $z \in U(0, R)$ holomorfní a dle Cauchyho věty pro hvězdovitou oblast je $\Phi(s) = 0$.
	\end{dukazin}
\end{veta}

% 03. 11. 2022

\begin{veta}[Cauchyův vzorec na kruhu]
	Nechť $G \subset ®C$ je otevřené a $f \in ©H(G)$. Nechť $\overline{U(z_0), r} \subset G$ a $p(t) = z_0 + r·e^{it}$, $t \in [0, 2\pi]$. Potom platí
	$$ \frac{1}{2\pi} \int_\phi \frac{f(z)}{z - s} dz = f(s), |s - z_0| < r; $$
	$$ \frac{1}{2\pi} \int_\phi \frac{f(z)}{z - s} dz = 0, |s - z_0| > r; $$

	\begin{dukazin}
		1. Nechť $|s - z_0|<r$. Volme $R > r$, aby $U(z, R) \subset G$. Položme
		$$ h(z) := \frac{f(z) - f(s)}{z - s}, z \in U(z_0, R) \setminus \{s\}; $$
		$$ h(z) := f'(s), z = s. $$
		
		Zřejmě $h \in ©H(U(z_0, R) \setminus \{s\})$ he spojitá hvězdovitá oblast $U(z_0, R)$. Z Cauchyho věty je
		$$ 0 = \frac{1}{2\pi} \int_\phi h = \frac{1}{2\pi} \int_\phi \frac{f(z) dz}{z - s} - \frac{1}{2\pi} \int_\phi \frac{dz}{z - d}·f_s. $$

		2. Nechť $|s - z_0| > r$. Volme $R \in (r, |z_0 - s|)$, aby $U(z_0, R) \subset G$. Potom
		$$ g(t) := \frac{f(z)}{z - s} \in ©H(U(z_0, R)) $$
		a z Cauchyho věty je
		$$ \frac{1}{2\pi} \int_\phi g = 0. $$
	\end{dukazin}
\end{veta}

\begin{dusledek}
	Nechť $G \subset ®C$ je otevřená a $f \in ©H(G)$. Potom $f$ má komplexní derivace libovolného řádu všude na $G$.

	Tedy nechť $U(z_0, r) \subset G$ a $\phi$ je jako v předchozím. Potom
	$$ \frac{k!}{2\pi} \int_\phi \frac{f(z) dz}{(z - s)^{k+1}} = f^{(k)}(s), |s - z_0| < r, k \in ®N. $$
	Zde $f^{(0)} := f$ a $k$-tá komplexní derivace $f^{(k)}$ je definovaná jako $f^{(k)} := \(f^{(k-1)}\)'$, má-li pravá strana smysl.

	\begin{dukazin}
		Z věty o derivaci integrálu podle komplexního parametru a předchozí věty, protože
		$$ \frac{d^k}{ds^k} \(\frac{1}{z - s}\) = \frac{k!}{(z - s)^{k+1}}, \qquad z ≠ s. $$
	\end{dukazin}
\end{dusledek}

\begin{veta}[Morera]
	Nechť $f$ je spojitá funkce na otevřené $G \subset ®C$. Potom $f \in ©H(G)$, právě když
	$$ \int_{\partial \triangle} f = 0, \qquad \forall \triangle \subset G \text{ trojúhelník}. $$

	\begin{dukazin}
		„$\implies$“: Goursat. „$\impliedby$“: Nechť $U := U(z_0, R) \subset G$. Protože $f$ je spojitá na hvězdovité oblasti $U$ a platí pro ni rovnost výše, má $f$ na $U$ primitivní funkci $F$, tzn. $f = F'$ na $U$. Protože $F \in ©H(U)$, máme $f' = F''$ na $U$, tudíž $f \in ©H(U)$. Tedy i $f \in ©H(G)$.
	\end{dukazin}
\end{veta}

\begin{veta}[Cauchyho odhady]
	Nechť $z_0 \in ®C$, $r \in (0, +∞)$ a $f$ je holomorfní funkce na otevřené množině obsahující $\overline{U(z_0, r)}$. Potom pro každé $k \in ®N_0$ je
	\begin{itemize}
		\item[CO1] $\forall s \in U := U(z_0, r)$:
			$$ |f^{(k)}(s)| ≤ \frac{(k!) r}{(d(s))^{k+1}}, $$
			kde $d(s) := \dist(s, \partial U)$;
		\item[CO2] $\forall s \in U(z_0, \frac{r}{2})$:
			$$ |f^{(k)}(s)| ≤ \frac{k! 2^{k+1}}{r^2}·\max_{\partial U} |f|; $$
		\item[CO3] $|f^{(k)}(z_0)| ≤ \frac{k!}{r^k}·\max_{\partial U} |f|$.
	\end{itemize}

	\begin{dukazin}[CO1]
		Z věty výše (pro $\phi$ stejné jako tam)
		$$ |f^{(k)}(z)| = |\frac{k!}{2\pi} \int_\phi \frac{f(z) ds}{(z - s)^{k+1}}| ≤ \frac{k!}{2\pi}·2\pi r \max_{<\phi>} |f| \frac{1}{(d(s))^{k+1}}, $$
		protože $|z - s| ≥ d(s)$ $\forall z \in <\phi>$.
	\end{dukazin}

	\begin{dukazin}[CO2 a CO3]
		Plyne z CO1, neboť $d(s) ≥ \frac{r}{2} \forall s \in U(z_0, \frac{r}{2})$ a $d(z_0) = r$.
	\end{dukazin}
\end{veta}

\begin{veta}[Liouville]
	Je-li $f$ holomorfní a omezená funkce na ®C, potom je $f$ konstantní.

	\begin{dukazin}
		Ukážeme, že $f' = 0$ na ®C: Označme $M := \sup_{®C} |f| < +∞$. Nechť $z_0 \in ®C$. Z CO3 pro každé $r > 0$ platí
		$$ |f'(z_0)| ≤ \frac{M}{r} \rightarrow 0, $$
		tudíž $f'(z_0) = 0$.
	\end{dukazin}
\end{veta}

\begin{dusledek}[Zakladní věta algebry]
	V ®C má každý polynom stupně alespoň 1 vždy alespoň jeden kořen.

	\begin{dukazin}
		Nechť $p(z) := a_n z^n + … + a_0 z^0$, kde $a_j \in ®C$, $n ≥ 1$ a $a_n ≠ 0$. Sporem: Předpokládejme, že $p ≠ 0$ na ®C. Potom $f:=\frac{1}{p}$ je holomorfní a omezená na ®C. Z Liouvilleovy věty je konstantní, tedy i $p = \frac{1}{f}$ je konstantní a $p' = 0 = p^{(n)} = a_n · n!$ $\implies$ $a_n = 0$. \lightning.
	\end{dukazin}
\end{dusledek}

\begin{lemma}
	Nechť $\phi$ je křivka v ®C, $f_i$ jsou spojité funkce na $<\phi>$ pro $n \in ®N$ a $f_i \rightrightarrows f$ na $<\phi>$. Potom $f$ je spojitá na $<\phi>$ a
	$$ \int_\phi f_n \rightarrow \int_\phi f. $$

	\begin{dukazin}
		Platí
		$$ |\int_\phi f_i - \int_\phi f| = |\int_\phi (f_n - f)| ≤ V(\phi)·\max_{<\phi>} |f_n - f| \rightarrow 0. $$
	\end{dukazin}
\end{lemma}

\begin{veta}[Weierstrass]
	Nechť $G \subset ®C$ je otevřená, $f_n \in ©H(F)$ pro $n \in ®N$ a $f_n \overset{\text{Loc.}}\rightrightarrows f$ na $G$. Potom $f \in ©H(G)$ a $f_n^{(k)} \overset{\text{Loc.}}\rightrightarrows f^{(k)}$ na $G$ pro každé $k \in ®N$.

	\begin{dukazin}
		1. Zřejmě $f$ je spojitá na $G$. Nechť $\triangle$ je trojúhelník v $G$. Potom
		$$ 0 \overset{G ?}= \int_{\partial \triangle} f_n \rightarrow \int_{\partial \triangle} f = 0. $$
		Z Morera je $f \in ©H(G)$.

		2. Nechť $k \in ®N$ a $z_0 \in G$. Volme $r > 0$, aby $\overline{U(z_0, r)} \subset G$. Z CO2 máme, že $\forall s \in U(z_0, \frac{r}{2}):$
		$$ |f_n^{(k)}(s) - f^{(k)}(s)| = |(f_n - f)^{(k)}(s)| ≤ \frac{k! 2^{k+1}}{r^k}·\max_{\partial U(z_0, r)} |f_n - f| \rightarrow 0. $$
	\end{dukazin}
\end{veta}

% 10. 11. 2022

\section{Mocninné řady}
\begin{definice}[Mocninná řada]
	Nechť $\{a_n\}_{n=0}^∞ \subset ®C$ a $z_0 \in ®C$. Potom
	$$ \sum_{n=0}^∞ a_n (z - z_0)^n $$
	je mocninná řada s koeficienty $\{a_n\}$ a středem $z_0$.
\end{definice}

\begin{poznamka}[Vlastnosti]
	\begin{itemize}
		\item[Konvergence] Existuje $R \in [0, +∞]$ takové, že řada konverguje absolutně a lokálně stejnoměrně na $U(z_0, R) := \{z \in ®C |\ |z - z_0| < R\}$ a řada diverguje pro $|z - z_0| > R$. Číslo $R$ se nazývá poloměr konvergence a platí, že
			$$ R = \frac{1}{\limsup_{n\rightarrow ∞} \sqrt[n]{|a_n|}}, $$
			kde $\frac{1}{0} = +∞$ a $\frac{1}{+∞} = 0$.
		\item Označíme-li součet řady na $U(z_0, R)$ jako $f$, potom $f \in ©H(U(z_0, R))$ a pro každé $k \in ®N$ je
			$$ f^{(k)}(z) - \sum_{n=k}^∞a_n · n · (n-1) · … ·(n-k+1)·(z - z_0)^{n-k}, \qquad z \in U(z_0, R), $$
			speciálně $a_k = f^{(k)}(z) / k!$.
		
			Plyne z Weierstrassovy věty pro
			$$ S_N(z) = \sum_{n=0}^∞ a_n(z - z_0)^n. $$
			Zřejmě $S_N \rightrightarrows^{loc} f$ na $U(z_0, R)$, tudíž $S_N^{(k)} \rightrightarrows^{loc} f^{(k)}$ na $U(z_0, R)$, tudíž rovnost výše platí. Dosadíme-li do ní $z = z_0$, dostaneme
			$$ f^{(k)}(z_0) = a_k·k!. $$
	\end{itemize}
\end{poznamka}

\begin{veta}[O rozvoji holomorfní funkce do mocninné řady na kruhu]
	Nechť $R \in (0, +∞]$ a $f \in ©H(U(z_0, R))$. Potom existuje jediná mocninná řada $\sum_{n=0}^∞ a_n(z - z_0)^n$, která má na $U(z_0, R)$ součet $f$. Navíc platí, že $a_n = f^{(n)}(z_0) / n!$ pro každé $n \in ®N_0$.

	\begin{dukazin}
		Jednoznačnost plyne ze vzorce. Existence: Nechť $z \in U(z_0, R)$. Volme $r > 0$, aby $|z - z_0| < r < R$. Potom z Cauchyho věty je
		$$ f(z) = \frac{1}{2\pi i} \int_\phi \frac{f(w) dw}{w - z}, $$
		kde $\phi(t) := z_0 + r e^{i t}$, $t \in [0, 2\pi]$. Pro každé $w \in <\phi>$ máme
		$$ \frac{1}{w - z} = \frac{1}{(w - z_0) - (z - z_0)} = \frac{1}{w - z}·\frac{1}{1 - \frac{z - z_0}{w - z_0}} = \sum_{n=0}^∞ \frac{(z - z_0)^n}{(w - z_0)^{n+1}}, $$
		což konverguje stejnoměrně pro $w \in <\phi>$. Dosadíme:
		$$ f(z) = \frac{1}{2\pi i} \int_\phi \sum_{n=0}^∞ \frac{(z - z_0)^n}{(w - z_0)^{n+1}} f(w) dw = $$
		$$ \sum_{n=0}^∞ (z - z_0)^n · \frac{1}{2\pi i} \int_\phi \frac{f(w) dw}{(w - z_0)^{n+1}} = \sum_{n=0}^∞ (z - z_0)^n · \frac{f^{(n)}}{n!}. $$
	\end{dukazin}
\end{veta}

\begin{veta}[O nulovém bodě]
	Nechť $f$ je holomorfní funkce na okolí $x_0 \in ®C$ a $f(z_0) = 0$. Potom buď $\exists r > 0: f = 0$ na $U(z_0, r)$; nebo $\exists r > 0: f ≠ 0$ na $P(z_0, r) := U(z_0, r) \setminus \{z_0\}$.

	V druhém případě existuje jediné $p \in ®N$ tak, že
	$$ f(x_0) = 0 = f'(z_0) = … = f^{(p-1)}(z_0), f^{(p)}(z_0) ≠ 0. $$
	Číslo $p$ je tzv. násobnost nulového bodu $z_0$ funkce $f$.

	Navíc $z_0$ je nulový bod $f$ násobnosti $p \in ®N$, právě když existuje $r > 0$ a $g \in ©H(U(z_0, r))$ tak, že $\forall z \in U(z_0, r)$:
	$$ g(z) ≠ 0 \land f(z) = (z - z_0)^p g(z). $$

	\begin{dukazin}
		Máme $f(z) = \sum_{n=0}^∞ a_n(z - z_0)^n$, $z \in U(z_0, R)$. Pokud nenastane první případ, potom existuje $n \in ®N$ tak, že $a_n ≠ 0$. Zvolme nejmenší $p \in ®N$, aby $0 ≠ a_p = f^{(p)}(z_0) / p!$. Potom platí rovnost pro druhý případ a $\forall z \in U(z_0, R)$:
		$$ f(z) = a_p(z - z_0)^p = … = (z - z_0)^p \underbrace{\sum_{n=p}^∞ a_n(z - z_0)^{n - p}}_{=:g(z)}. $$
		Zřejmě $g \in ©H(U(z_0, R))$. Protože $g(z_0) = a_p ≠ 0$, $\exists r>0$ tak, že $g ≠ 0$ na $U(z_0, r)$. Tudíž $f(z) = (z - z_0)^p·g(z) ≠ 0$ na $P(z_0, r)$. Obrácené tvrzení plyne stejně snadno.
	\end{dukazin}
\end{veta}

\begin{veta}[O jednoznačnosti pro holomorfní funkce]
	Nechť $\O = G \subset ®C$ je oblast a $f, g \in ©H(G)$. Následující je ekvivalentní

	\begin{enumerate}
		\item $f = g$ na $G$;
		\item $M := \{z \in G | f(z) = g(z)\}$ má v $G$ hromadný bod, tzn. existuje $z_0$ tak, že $P(z_0, r) \cap M ≠ \O\ \forall r > 0$;
		\item Existuje $z_0 \in G$ tak, že
			$$ f^{(k)}(z_0) = g^{(k)}(z_0) \qquad \forall k \in ®N_0. $$
	\end{enumerate}

	\begin{dukazin}
		BÚNO předpokládejme, že $G = 0$, jinak uvažujme $f - g$. „$1 \implies 2$ a $2 \implies 3$“: Nechť $z_0 \in G$ je hromadný bod $M := \{z \in G | f(z) = 0\}$. Zřejmě $f(z_0) = 0$ a z předchozí věty je $f = 0$ na nějakém okolí $z_0$.

		„$3 \implies 1$“: Uvažme $N := \{z \in G | \forall k \in ®N_0: f^{(k)}(z) = 0\}$. Potom $\O ≠ N$ a $N$ je uzavřená v $G$, protože všechny $f^{(k)}$ jsou spojité. Navíc $N$ je otevřená. Nechť $z_1 \in N$. Podle věty o nulovém bodě existuje $r > 0$ tak, že $f = 0$ na $U(z_1, r)$. Tedy $U(z_0, r) \subset N$. Protože $G$ je oblast (tj. je souvislá, tedy neexistuje vlastní obojetná podmnožina), je $N = G$.
	\end{dukazin}
\end{veta}

% 24. 11. 2022

\begin{veta}[Princip maxima modulu]
	Nechť $G \subset ®C$ je oblast a $f \in ©H(G)$. Potom $f$ je na $G$ konstantní, pokud $|f|$ nabývá v $G$ lokální maximum, tzn. existuje $z_0 \in G$ a $r > 0$, že $\forall z \in U(z_0, r) \subset G: |f(z)| ≤ |f(z_0)|$.

	\begin{dukazin}
		Nechť to platí. Potom
		$$ f(z) = \sum_{n=0}^∞ a_n(z - z_0)^n, z \in U(z_0, r). $$
		Nechť $0 < \rho < r$. Potom
		$$ |a^2| = |f(z_0)|^2 ≥ \frac{1}{2\pi} \int_0^{2\pi} |f(z_0 + \rho e^{it})|^2 dt = $$
		$$ = \frac{1}{2\pi} \int_0^{2\pi} \(\sum_{n=1}^∞ a_n\rho^n e^{i n t}\)·\(\sum_{m=1}^∞ \overline{a_m}\rho^m e^{-i n t}\) dt = |f(z)|^2 = f(z)\overline{f(z)}. $$
	\end{dukazin}
\end{veta}

\section{Riemannova sféra}
\begin{definice}[Riemannova sféra (®S)]
	Označme $®S = ®C \cup \{∞\}$ a definujeme okolí v $∞$ následovně: Pro každou $\epsilon > 0$ položme
	$$ P(∞, \epsilon) := \{z \in ®C |\ |z| > \frac{1}{\epsilon}\}, \qquad U(∞, \epsilon) := P(∞, \epsilon) \cup \{∞\}. $$
\end{definice}

\begin{definice}[Limita na ®S]
	Je-li $z_0, L \in ®S$, potom $L = \lim_{z \rightarrow z_0} f(z)$, pokud $\forall \epsilon > 0\ \exists \delta > 0$:
	$$ z \in P(z_0, \delta) \implies f(z) \in U(L, \epsilon). $$
\end{definice}

\begin{tvrzeni}[Vlastnosti limity na ®S]
	\ 
	\begin{itemize}
		\item $\lim_{z \rightarrow ∞}f(z) = \lim_{z \rightarrow 0} f\(\frac{1}{z}\)$, má-li alespoň jedna strana smysl.
		\item Následující je ekvivalentní: $\lim_{z \rightarrow z_0} f(z) = ∞$, $\lim_{z \rightarrow z_0} |f(z)| = +∞$, $\lim_{z \rightarrow z_0} \frac{1}{f(z)} = 0$.
		\item Počítání s $∞$: $\frac{a}{∞} = 0\ \forall a \in ®C$, $\frac{a}{0} = ∞\ \forall a \in ®S \setminus \{0\}$, $a±∞ = ∞ \forall a \in ®C$, $a·∞ = ∞\ \forall a \in ®S \setminus \{0\}$. Nedefinujeme $\frac{0}{0}$, $\frac{∞}{∞}$, $∞±∞$, $0·∞$. Potom platí i v $®S$ aritmetika limit.
		\item ®S je jednobodovou kompaktifikací ®C.
		\item ®S je homeomorfní s jednotkovou sférou $S^2 := \{[\alpha, \beta, \gamma] \in ®R^3 | \alpha^2 + \beta^2 + \gamma^2 = 1\}$, speciálně ®S je kompaktní.
	\end{itemize}
\end{tvrzeni}

\section{Izolované singularity}
\begin{definice}
	Nechť $f$ je holomorfní funkce na $P(z_0) = P(z_0, r)$. Potom $f$ má v $z_0$:
	
	\begin{itemize}
		\item odstranitelnou singularitu, existuje-li $\lim_{z \rightarrow z_0} f(z) \in ®C$;
		\item pól, je-li $\lim_{z \rightarrow z_0} f(z) = ∞$;
		\item podstatnou singularitu, pokud $\lim_{z \rightarrow z_0} f(z)$ neexistuje v ®S.
	\end{itemize}

	\begin{veta}[O odstranitelné singularitě]
		Nechť $f$ je holomorfní funkce na $P(z_0)$. Následující je ekvivalentní:

		\begin{enumerate}
			\item $z_0$ je odstranitelná singularita $f$.
			\item Existuje $r > 0$ tak, že $f$ je omezená na $P(z_0, r)$.
			\item Existuje $F \in ©H(U(z_0))$ tak, že $F = f$ na $P(z_0)$.
		\end{enumerate}

		\begin{poznamkain}[Úmluva]
			Každá odstranitelná singularita se považuje za odstraněnou (tj. bereme $F$ místo $f$).
		\end{poznamkain}

		\begin{dukazin}
			„$1. \implies 2.$, $2. \implies 3.$“: Položme
			$$ g(z) := \begin{cases} (z - z_0)^2 f(z),& \quad z \in P(z_0),\\ 0,& \quad z = z_0. \end{cases} $$
			Potom $g \in ©H(U(z_0))$, protože
			$$ g'(z_0) = \lim_{z \rightarrow z_0} \frac{g(z) - g(z_0)}{z - z_0} = \lim_{z \rightarrow z_0} \underbrace{(z - z_0)}_{\rightarrow 0}·\underbrace{f(z)}_{\text{omezená}} = 0. $$

			Dále (pro „$3. \implies 1.$“) $\forall z \in U(z_0)$:
			$$ g(z) = \sum_{n=2}^∞ a_n(z - z_0)^n = (z-z_0)^2·F(z), $$
			kde $F(z) := \sum_{n=2}^∞ a_n(z - z_0)^{n-2}$. Potom $F \in ©H(U(z_0))$ a $\forall z \in P(z_0)$:
			$$ (z - z_0)^2·f(z) = (z - z_0)^2·F(z). $$
		\end{dukazin}
	\end{veta}

	\begin{poznamka}
		Píšeme $f(z)\sim g(z)$ pro $z \rightarrow z_0$, pokud $\lim_{z \rightarrow z_0} \frac{f(z)}{g(z)} \in ®C \setminus \{0\}$.
	\end{poznamka}

	\begin{veta}[O póle]
		Nechť $f$ je holomorfní funkce na $P(z_0)$. Následující je ekvivalentní:
		
		\begin{enumerate}
			\item $z_0$ je pól $f$.
			\item $h := \frac{1}{f}$ (po dodefinování $h(z_0) = 0$) má v $z_0$ nulový bod násobnosti $p \in ®N$.
			\item Existuje $p \in ®N$ tak, že $\lim_{z \rightarrow z_0}(z - z_0)^p f(z) \in ®C \setminus \{0\}$. (Tedy $f(z) \sim \frac{1}{(z - z_0)^p}$ pro $z \rightarrow z_0$.)
			\item Existuje $p \in ®N$ tak, že $\forall k \in ®Z$:
				$$ \lim_{z \rightarrow z_0} (z - z_0)^k f(z) \begin{cases} = ∞, & \text{ je-li } k < p,\\ \in ®C \setminus \{0\}, & k=p, \\ 0, & k > p.\end{cases} $$
		\end{enumerate}

		\begin{dukazin}
			„$1. \implies 2.$“: Protože $\lim_{z \rightarrow z_0} f(z) = ∞$, je $\lim_{z \rightarrow z_0} \frac{1}{f(z)} = 0$. Po odstranění singularity, tzn. po dodefinování $h(z_0) = 0$, má $h = \frac{1}{f}$ nulový bod v $z_0$ konečné násobnosti $p \in ®N$.

			„$2. \implies 3.$“: Existuje $r > 0$ a $g \in ©H(U(z_0, r))$ tak, že $g ≠ 0$ na $U(z_0, r)$ a
			$$ h(z) = (z - z_0)^p·g(z), \qquad z \in U(z_0, r). $$
			Potom $\lim_{z \rightarrow z_0} (z - z_0)^p f(z) = \lim_{z \rightarrow z_0} \frac{1}{g(z)} = \frac{1}{g(z_0)} \in ®C \setminus \{0\}$.

			„$3. \implies 4.$“: Máme
			$$ \lim_{z \rightarrow z_0} (z - z_0)^k f(z) = \lim_{z \rightarrow z_0} \underbrace{(z - z_0)^{k - p}}_{\rightarrow 0, k > p / 1, k=p / ∞, k < p}·\underbrace{(z - z_0)^p f(z)}_{\rightarrow … \in ®C \setminus \{0\}}. $$

			„$4. \implies 1.$“: $\lim_{z \rightarrow z_0} f(z) = \lim_{z \rightarrow z_0} (z - z_0)^k f(z) |_{k=0} = ∞$.
		\end{dukazin}
	\end{veta}

	\begin{definice}[Násobnost pólu]
		Číslo $p$ je určeno jednoznačně a nazývá se násobnost pólu $z_0$ funkce $f$.	
	\end{definice}
\end{definice}

\begin{veta}[Casorati-Weierstrass]
	Nechť $f$ je holomorfní na $P(z_0)$. Následující je ekvivalentní:

	\begin{enumerate}
		\item $z_0$ je podstatná singularita $f$.
		\item $\forall r > 0: \overline{f(P(z_0, r))} = ®C$.
	\end{enumerate}

	\begin{poznamka}[Velká Picardova věta]
		Platí dokonce (i když je to těžké dokázat)
		$$ \forall r > 0: ®C \setminus f(P(z_0, r)) $$
		je nejvýše jednobodová.
	\end{poznamka}

% 01. 12. 2022

	\begin{dukazin}
		„$2. \implies 1.$“ jasné, použije se definice limity.

		„$1. \implies 2.$“ (konkrétně ukážeme $\neg 2. \implies \neg 1.$): Předpokládejme, že existuje $r > 0$, že $®C \setminus \overline{f(P(z_0, r))} ≠ \O$ a $f \in ©H(P(z_0, r))$. Potom existuje $U(u_0, \beta) \subset ®C \setminus \overline{f(P(z_0, r))}$, speciálně $0 < |z - z_0| < r$ $\implies$ $|f(z) - u_0| ≥ \beta$.

		Definujeme
		$$ g(z) := \frac{1}{f(z) - u_0}, \qquad z \in P(z_0, r). $$
		Zřejmě je $g$ holomorfní a $|g| ≤ \frac{1}{\beta}$ na $P(z_0, r)$. Tedy $z_0$ je odstranitelná singularita $g$ a $\exists L:=\lim_{z \rightarrow z_0} g(z) \in ®C$. Potom
		$$ \lim_{z \rightarrow z_0} f(z) = \lim_{z \rightarrow z_0} \(\frac{1}{g(z)} + u_0\) \begin{cases}\in ®C, & L≠0,\\∞, & L=∞.\end{cases} $$
		Tedy $f$ má v $z_0$ buď pól nebo odstranitelnou singularitu. Nikdy podstatnou.
	\end{dukazin}
\end{veta}

\section{Laurentovy řady}
\begin{definice}[Laurentova řada (LŘ), regulární část, hlavní část, konvergence LŘ]
	Nechť $\{a_n\}_{n=-∞}^∞ \subset ®C$ a $z_0 \in ®C$. Potom
	$$ \sum_{n=-∞}^∞ a_n(z - z_0)^n = \sum_{n=0}^∞ a_n(z - z_0)^n + \sum_{n=1}^∞ a_{-n} (z - z_0)^{-n} $$
	je Laurentova řada s koeficienty $\{a_n\}$ a středem $z_0$. První řada na pravé straně je tzv. regulární část, druhá je pak hlavní část. Řekneme, že řada konverguje, pokud obě řady na pravé straně konvergují.
\end{definice}

\begin{tvrzeni}[Vlastnosti]
	\ 
	
	\begin{itemize}
		\item Konvergence: Existuje jediné $r, R \in [0, +∞]$ tak, že regulární část konverguje absolutně a lokálně stejnoměrně na $|z - z_0| < R$ a diverguje na $|z - z_0| > R$; hlavní část konverguje absolutně a lokálně stejnoměrně na $|z - z_0| > r$ a diverguje pro $|z - z_0|<r$.

			$R$ je zřejmé, pro získání $r$ dosadíme $w := (z - z_0)^{-1}$ a vezmeme 1 / poloměr konvergence vyšlé mocninné řady.
		\item Součet: Nechť $0 ≤ r < R ≤ +∞$. Pak Laurentova řada konverguje absolutně a lokálně stejnoměrně na vnitřku mezikruží dané $r$ a $R$ (značíme $P(z_0, r, R)$) a bude divergovat mimo něj (na hranici nevíme).

			Označíme-li součet jako $f$, potom $f \in ©H(P(z_0, r, R))$ a řadu derivujeme člen po členu.
	\end{itemize}
\end{tvrzeni}

\begin{lemma}
	Nechť $f$ je holomorfní funkce na $P(z_0, r, R) =: P$, kde $0 ≤ r < R ≤ +∞$. Pro každé $\rho \in (r, R)$ označíme $\phi_\rho(t) := z_0 + \rho e^{it}$, $t \in [0, 2\pi]$ a $J(\rho) := \int_{\phi_\rho} f$. Potom $J$ je konstantní na $(r, R)$.

	\begin{dukazin}
		BÚNO: Nechť $z_0 = 0$. Nechť $\rho \in (r, R)$. Potom
		$$ J(\rho) = i \int_0^{2\pi} f(\rho e^{it}) \rho e^{it} = i \int_0^{2\pi} g(\rho e^{it}) dt, $$
		kde $g(z) := f(z)·z, z \in P$.

		Dále $J'(\rho) \overset?= \frac{i}{\rho} \int_0^{2\pi} g'(\rho e^{it}) \rho e^"{it} dt = \frac{1}{\rho} \int_{\phi_\rho} g' = 0$, protože $g'$ má primitivní funkci $g$ na $P$.

		? platí, protože
		$$ \frac{d}{d\rho}g(\rho e^{it}) = \frac{\partial g}{\partial x} \cos t + \frac{\partial g}{\partial y} \sin t = g' \cos t + i g' \sin t = \rho e^{it} = (\rho \cos t, \rho \sin t). $$
	\end{dukazin}
\end{lemma}

\begin{veta}[Cauchyho vzorec na mezikruží]
	Nechť $f \in ©H(P)$, $P := P(z_0, r, R)$. Nechť $r < r_0 < R_0 < R$ a $s \in P(z_0, r_0, R_0)$. Potom platí
	$$ f(s) = \frac{1}{2\pi} \int_{\phi_{R_0}} \frac{f(z) dz}{z - s} - \frac{1}{2\pi} \int_{\phi_{r_0}} \frac{f(z) dz}{z - s}, $$
	kde $\phi_\rho$ je jako v předchozím lemmatu.

	\begin{dukazin}
		Pro $z_0 \in P$ položme $h(z) := \frac{f(z) - f(s)}{z - s}$, $z ≠ s$, a $h(z) := f'(s)$, $z = s$. Potom $h \in ©H(P)$, protože $h$ má v $s$ „odstraněnou“ singularitu.

		Podle předchozího lemmatu máme
		$$ \underbrace{\int_{\rho_{R_0}} h}_{=} = \int_{\phi_{R_0}} \frac{f(z) dz}{z - s} - f(s) · \int_{\phi_{R_0}} \frac{dz}{z - s} $$
		$$ \overbrace{\int_{\rho_{r_0}} h} = \int_{\phi_{r_0}} \frac{f(z) dz}{z - s} - f(s) · \int_{\phi_{r_0}} \frac{dz}{z - s} = \int_{\phi_{r_0}} \frac{f(z) dz}{z - s} - 0. $$
	\end{dukazin}
\end{veta}

\begin{veta}[O Laurentově rozvoji holomorfní funkce na mezikruží]
	Nechť $P := P(z_0, r, R)$, kde $0 ≤ r < R ≤ ∞$. Nechť $f \in ©H(P)$. Pak existuje jediná Laurentova řada
	$$ \sum_{n=-∞}^∞ a_n(z - z_0)^n, $$
	která má na $P$ součet $f$.

	Navíc platí $a_m = \frac{1}{2\pi} \int \frac{f(z) dz}{(z - z_0)^{m+1}}$, kde $m \in ®Z$ a $\phi_\rho$ je jako výše.

	\begin{dukazin}[Jednoznačnost]
		Nechť $f(z) = \sum_{n=-∞}^{+∞} a_n(z - z_0)^n$, $z \in P$. Nechť $\rho \in (r, R)$ a $m \in ®Z$. Potom
		$$ \int_{\phi_\rho} f(z)(z - z_0)^{-(m+1)} dz = \int_{\phi_\rho} \sum_{-∞}^∞ a_n(z - z_0)^{n - m - 1} dz = $$
		$$ = \sum_{n=-∞}^∞ a_n \int_{\phi_\rho}(z - z_0)^{n - m - 1} dz = \begin{cases}0, & m≠n,\\ 2\pi \ind_{\phi_\rho} z_0, & m=n.\end{cases} $$
	\end{dukazin}
\end{veta}

% 08. 12. 2022 Z poznámek

\begin{veta}[O Laurentově rozvoji kolem izolované singularity]
	Nechť $f \in ©H(P(z_0, r))$ a $f(z) = \sum_{-∞}^{+∞} a_n(z - z_0)^n$, $z \in P(z_0, r)$. Potom

	\begin{itemize}
		\item $f$ má v $z_0$ odstranitelnou singularitu $\Leftrightarrow$ $\forall n < 0: a_n = 0$;
		\item $f$ má v $z_0$ pól násobnosti $p \in ®N$ $\Leftrightarrow$ $a_{-p} ≠ 0$ a $\forall n < -p: a_n = 0$;
		\item $f$ má v $z_0$ podstatnou singularitu $\Leftrightarrow$ $a_n ≠ 0$ pro nekonečně mnoho $n < 0$.
	\end{itemize}

	\begin{dukazin}
		Odstranitelná je jasná. $f$ má v $z_0$ pól násobnosti $p$ právě když $g(z) := (z - z_0)^p f(z)$ má v $z_0$ odstranitelnou singularitu a po jejím odstranění je $g(z_0) ≠ 0$. Neboli $(z - z_0)^p f(z) = \sum_{n=0}^{+∞} b_n(z - z_0)^n$, $z \in P(z_0, r)$ a $b_0 = g(z_0) ≠ 0$, tzn.
		$$ f(z) = \frac{b_0}{(z - z_0)^p} + \frac{b_1}{(z - z_0)^{p-1}} + … = \sum_{n=0}^∞ b_n(z - z_0)^{n - p}, z \in P(z_0, r). $$

		Z prvních dvou vidíme, že $f$ nemá v $z_0$ podstatnou singularitu, právě když $a_n ≠ 0$ pro konečně mnoho $n < 0$.
	\end{dukazin}
\end{veta}

\begin{veta}[Rozklad holomorfní funkce s konečně mnoha izolovanými singularitami]
	Nechť $G \subset ®C$ je otevřena, $M \subset F$ je konečná a $f \in ©H(G \setminus M)$. Pro každé $s \in M$ označme $H_s$ součet hlavní části Laurentova rozvoje funkce $f$ kolem $s$. Potom existuje jediná $h \in ©H(G)$ tak, že $f = \sum_{s \in M} H_s + h$ na $G \setminus M$.

	\begin{dukazin}
		Zřejmě $\forall s \in M: H_s \in ©H(®C \setminus \{s\})$. Funkce $h := f - \sum_{s \in M} H_s$ je holomorfní na $G \setminus M$ a v bodech $s \in M$ má odstranitelné singularity. Skutečně, nechť $s_0 \in M$, potom existuje $r_0 > 0$ tak, že $P(s_0, r_0) \subseteq G \setminus M$ a $f = R_{s_0} + H_{s_0}$ na $P(s_0, r_0)$, kde $R_{s_0}$ je součet regulární části Laurentova rozvoje $f$ kolem $s_0$ a $R_{s_0} \in ©H(U(s_0, r_0))$. Tedy na $P(s_0, r_0)$ máme
		$$ h = R_{s_0} + H_{s_0} - \sum_{s \in M} H_s = R_{s_0} - \sum_{s ≠ s_0, s \in M} H_s \in ©H(U(s_0, r_0)). $$
	\end{dukazin}
\end{veta}

\subsection{Reziduum}

\begin{definice}[Reziduum]
	Nechť $f \in ©H(P(z_0))$ a nechť $f(z) = \sum_{n=-∞}^{+∞} a_n(z - z_0)^n$, $z \in P(z_0)$. Potom reziduem $f$ v $z_0$ nazveme číslo $\res_{z_0} f := a_{-1}$.
\end{definice}

\begin{veta}[Reziduová na hvězdovitých oblastech]
	Nechť $G \subset ®C$ je hvězdovitá oblast, $M \subset G$ je konečná a $f \in ©H(G \setminus M)$. Nechť $\phi$ je uzavřená křivka v $G \setminus M$. Potom máme
	$$ \int_\phi f = 2\phi i \sum_{s \in M} \res_s f·\ind_\phi s. $$

	\begin{poznamkain}
		Pro $M = \O$ dostaneme Cauchyho větu.
	\end{poznamkain}

	\begin{dukazin}
		Podle předchozí věty existuje $h \in ©H(G)$ tak, že $f = \sum_{s \in M} H_s + h$ na $G \setminus M$. Potom máme $\int_\phi f = \sum_{s \in M} \int_\phi H_s$, protože $\int_\phi h = 0$ z Cauchyho věty pro hvězdovité oblasti. Pro každé $s \in M$:
		$$ \int_\phi H_s(z) dz = \int_\phi \sum_{n=1}^{+∞} a_{-n}^s \frac{1}{(z - s)^n} dz = \sum_{n=1}^{+∞} a_{-n}^s \int_\phi \frac{dz}{(z - s)^n} = 2\pi i·\res_s f · \ind_\phi s, $$
		jelikož suma konverguje stejnoměrně na $<\phi>$ a poslední integrál je roven $0$ pro $n ≠ 1$ (jinak má integrand primitivní funkci, a tudíž je $\oint$ nulový) a $2\pi i· \ind_\phi s$, je-li $n = 1$.
	\end{dukazin}
\end{veta}

\subsection{Speciální typy integrálů}
\begin{veta}
	Nechť $R = P / Q$, kde $P$, $Q$ jsou polynomy, které nemají společné kořeny a platí $Q ≠ 0$ na ®R a $\st(Q) ≥ \st(P) + 2$, kde $\st(Q)$ je stupeň polynomu $Q$. Potom
	$$ \int_{-∞}^{+∞} R(x) dx = 2 i \pi · \sum_{Q(s) = 0, \Im(s) > 0} \res_s R. $$

	\begin{dukazin}
		Integrál konverguje, právě když platí podmínky (cvičení). Nechť $r > 0$ a $\phi_r := \phi_r^1 + \phi_r^2$, kde $\phi_r^1(t) := t$, $t \in [-r, r]$ a $\phi_r^2(t) := r e^{i t}$, $t \in [0, \pi]$. Je-li $r > 0$ tak velké, aby uvnitř $\phi_r$ ležely všechny póly $R$ z horní poloroviny, potom 
		$$ 2 i \pi \sum_{Q(s) = 0, \Im(s) > 0} \res_s R = \int_{\phi_r} R = \int_{\phi_r^1} R + \int_{\phi_r^2} R. $$
		Máme
		$$ \int_{\phi_r^1} R = \int_{-r}^r R \rightarrow \int_{-∞}^{+∞} R. $$
		Protože $\int_{\phi_r^2} \rightarrow 0$, dostaneme, že věta platí, neboť existuje $C > 0$, $r_0 > 0$ tak, že $|R(z)| ≤ \frac{C}{r^2}$, je-li $|z| = r ≥ r_0$. Máme totiž
		$$ |R(z)| = \left|\frac{a_0 z^n + … + a_n z^0}{b_0 z^m + … + b_m z^0}\right| = \frac{1}{|z|^2} |z|^{n - m + 2}·\left|\frac{a_0 + \frac{a_1}{z} + … + \frac{a_n}{z^n}}{b_0 + \frac{b_1}{z} + … + \frac{b_m}{z}}\right|. $$
		Tedy
		$$ \left|\int_{\phi_r^2} R\right| ≤ V(\phi_r^2)·\max_{<\phi_r^2>}|R| ≤ r \pi \frac{C}{r^2} \rightarrow 0. $$
	\end{dukazin}
\end{veta}

\begin{veta}
	Nechť $R = P / Q$, kde $P$, $Q$ jsou polynomy, které nemají společné kořeny a platí $Q ≠ 0$ na ®R a $\st(Q) ≥ \st(P) + 1$. Nechť $a > 0$. Potom
	$$ \int_{-∞}^{+∞} R(x) e^{i a x} dx = 2 i \pi · \sum_{Q(s) = 0, \Im(s) > 0} \res_s\(R(z) e^{i a z}\). $$

	\begin{dukazin}[Cvičení]
		Newtonův integrál konverguje právě za těchto podmínek, jak spočteme tento integrál pro $a < 0$?

		Jako v předešlé větě integrujeme podél $\phi_r$ funkci $R(z) e^{i a z}$ a pošleme $r \rightarrow +∞$. Platí, že
		$$ \int_{\phi_r^2} R(z) e^{i a z} dz \rightarrow 0 $$
		z Jordanova Lemmatu (bylo na cvičení?), z podmínky na stupně totiž máme, že $\lim_{z\rightarrow ∞} R(z) = 0$.
	\end{dukazin}
\end{veta}


TODO!!!

% 15. 12. 2022

\section{Výpočet indexu}
\begin{poznamka}[Úmluva]
	Bod, pro který počítáme index, je 0.
\end{poznamka}

\begin{definice}[Jednoznačná větev argumentu (j. v. a.), jednoznačná větev logaritmu (j. v. l.)]
	Nechť $\phi: [\alpha, \beta] \rightarrow ®C \setminus \{0\}$ je spojitá.

	\begin{poznamkain}
		Víme: $0 ≠ z = |z|e^{i\theta} = e^\Phi$, kde $\theta \in Arg(z)$ a $\Phi := \log |z| + i\theta \in Log z$.

		Tedy $\forall t \in [\alpha, \beta]: 0 ≠ \phi(t) = |\phi(t)|·e^{i\theta(t)} = e^{\Phi(t)}$, kde $\theta(t) \in Arg(\phi(t))$ a $\Phi(t) := \log|\phi(t)| + i \theta(t) \in Log(\phi(t))$
	\end{poznamkain}

	Řekněme, že $\theta: [\alpha, \beta] \rightarrow ®R$ (resp. $\Phi: [\alpha, \beta] \rightarrow ®C$) je jednoznačná větev argumentu (respektive jednoznačná větev logaritmu) křivky $\phi$, pokud je $\theta$ (resp. $\Phi$) spojitá na $[\alpha, \beta]$ a $\forall t \in [\alpha, \beta]: \theta(t) \in Arg(\phi(t))$ (resp. $\Phi(t) \in Log(\phi(t))$).
\end{definice}

\begin{veta}[O jednoznačnosti j. v. a. a j. v. l.]
	Nechť $\phi: [\alpha, \beta] \rightarrow ®C \setminus \{0\}$ je spojitá. Potom $\Phi$ je j. v. l. $\phi$, právě když $\Re \Phi = \log|\phi|$ a $\Im \Phi$ je j. v. a. $\phi$.

	Jsou-li $\theta_1$, $\theta_2$ j. v. a. $\phi$, potom existuje $k in ®Z$ tak, že
	$$ \theta_2 = \theta_1 + 2k\pi \qquad \text{ na } [\alpha, \beta]. $$

	\begin{dukazin}
		j. v. l. z definice. „j. v. a.“: Pro každé $t \in [\alpha, \beta]$ existuje $k(t) \in ®Z$ tak, že
		$$ \theta_2(t) = \theta_1(t) + 2k(t) \pi. $$
		Protože $k: [\alpha, \beta] \rightarrow ®Z$ je spojitá, je $k$ na $[\alpha, \beta]$ konstantní.
	\end{dukazin}
\end{veta}

\begin{veta}[O existenci j. v. logaritmu pro regularní křivky]
	Nechť $\phi: [\alpha, \beta] \rightarrow ®C \setminus \{0\}$ je regulární křivka. Potom existuje j. v. l. $\Phi$ křivky $\phi$ a platí, že
	$$ \int_\phi \frac{dz}{z} = \Phi(\beta) - \Phi(\alpha). $$
	Navíc $\Im \Phi$ je j. v. a. $\phi$.

	\begin{dukazin}
		Hledáme spojitou $\Phi$ takovou, že $\phi = e^\Phi$. Zřejmě
		$$ \int_\phi \frac{dz}{z} = \int_\alpha^\beta \frac{\phi'(t)}{\phi} dt, \qquad \Phi_0(s) := \int_\alpha^s \frac{\phi'(t)}{\phi(t)} dt, \quad s \in [\alpha, \beta]. $$

		Potom $\Phi_0$ je spojitá na $[\alpha, \beta]$ a $\Phi_0' = \frac{\phi'}{\phi}$ na $[\alpha, \beta] \setminus K$, kde $K$ je konečná.

		Potom na $[\alpha, \beta] \setminus K$ platí
		$$ \(\phi·e^{-\Phi_0}\)' = (\phi' - \phi·\Phi_0')·e^{-\Phi_0} = 0, $$
		tudíž existuje $c \in ®C$, že $\phi·e^{-\Phi_0} = e^c$ na $[\alpha, \beta]$. Stačí položit $\Phi := \Phi_0 + c$.
	\end{dukazin}
\end{veta}

\begin{veta}[O výpočtu indexu]
	Nechť $\phi: [\alpha, \beta] \rightarrow ®C$ je regulární uzavřená křivka a $s \in ®C \setminus <\phi>$. Nechť $\tilde \phi := \phi - s$ a $\theta$ je j. v. a. $\tilde \phi$. Potom
	$$ \ind_\phi s = \ind_{\tilde\phi} 0 = \frac{\theta(\beta) - \theta(\alpha)}{2\pi}. $$

	Speciálně $\ind_\phi s \in ®Z$.

	\begin{dukazin}
		Platí $\ind_\phi s = \frac{1}{2\pi i} \int_\alpha^\beta \frac{\phi'(t)}{\phi(t) - s} dt = \ind_{\tilde \phi} 0 = \frac{1}{2\pi i} \int_{\tilde \phi} \frac{dz}{z} =$
		$$ = \frac{1}{2\pi i} (\Phi(\beta) - \Phi(\alpha)) = \frac{1}{2\pi i}(i \Im \Phi(\beta) - \Im \Phi(\alpha)) = \frac{\theta(\beta) - \theta(\alpha)}{2\pi} \in ®Z, $$
		kde $\Theta$ je j. v. l. $\tilde \phi$, $\Re \Phi(\beta) = \log |\tilde\phi(\beta)| = \log |\tilde \phi(\alpha)| = \Re \Phi(\alpha)$ a $\theta := \Im \Phi$ je j. v. a. $\tilde \phi$.
	\end{dukazin}
\end{veta}


\subsection{Obecná Cauchyho a reziduální věta pro cykly}
\begin{definice}[Cyklus]
	Konečnou posloupnost $\Gamma := \{\phi_1, …, \phi_n\}$, kde $n \in ®N$ a $\phi_1, …, \phi_n$ jsou uzavřené (regulární) křivky v ®C budeme nazývat cyklus.
\end{definice}

\begin{definice}
	Označíme:

	\begin{itemize}
		\item $<\Gamma> := \bigcup_{k=1}^n <\phi_k>$ graf $\Gamma$;
		\item $V(\Gamma) := \sum_{k=1}^n V(\phi_k)$ délka $\Gamma$;
		\item $\int_\Gamma f := \sum_{k=1}^n \int_{\phi_k} f$, je-li $f$ spojitá na $<\Gamma>$;
		\item $\ind_\Gamma z_0 := \sum_{k=1}^n \ind_{\phi_k} z_0 = \frac{1}{2 \pi i} \int_\Gamma \frac{dz}{z - z_0}$, $z_0 \in ®C \setminus <\Gamma>$;
		\item $\Int \Gamma := \{z_0 \in ®C \setminus <\Gamma> | \ind_\Gamma z_0 ≠ 0\}$ vnitřek $\Gamma$;
		\item $\Ext \Gamma := \{z_0 \in ®C \setminus <\Gamma> | \ind_\Gamma z_0 = 0\}$ vnějšek $\Gamma$.
	\end{itemize}
\end{definice}

\begin{poznamka}[Úmluva]
	Uzavřenou křivku $\phi$ chápeme jako cyklus.
\end{poznamka}

\begin{veta}[Obecná Cauchyho pro cykly]
	Nechť $G \subset ®C$ je otevřená a $\Gamma$ je cyklus v $G$, tzn. $<\Gamma> \subset G$. Potom platí, že
	$$ \int_\Gamma f = 0 \qquad \forall f \in ©H(G), $$
	právě když $\Int \Gamma \subset G$.
\end{veta}

\end{document}
