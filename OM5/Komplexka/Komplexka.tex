\documentclass[12pt]{article}					% Začátek dokumentu
\usepackage{../../MFFStyle}					    % Import stylu



\begin{document}

% 29. 09. 2022
\section*{Úvod}

\begin{poznamka}
	Mluvilo se o historii ®C.
\end{poznamka}

\begin{definice}[Prostor ®C]
	Prostor ®C komplexních čísel je prostor $®R^2$, v němž navíc definujeme násobení:
	$$ (x_1, y_1) \cdot (x_2, y_2) = (x_1·x_2 - y_1·y_2, x_1·y_2 + x_2·y_1). $$

	Ztotožníme $(x, 0) = x$, neboli $®R \subset ®C$. Značíme $i := (0, 1)$ (imaginární jednotka).
\end{definice}

\begin{definice}[Značení (komplexně sdružené číslo, reálná a imaginární složka)]
	$$ z = x + i·y \in ®C \implies \overline{z}:= x - i·y \land \Re z := x, \Im z := y. $$
\end{definice}

\begin{definice}[Modul / absolutní hodnota]
	$$ |z| := \sqrt{x^2 + y^2} $$
\end{definice}

\begin{tvrzeni}[Vlastnosti]
	\ 
	\begin{itemize}
		\item $z = (x, y) \in ®C$, potom $z = x + i·y$ a $(±i)^2 = -1$.
		\item Násobení $·: ®C^2 \rightarrow ®C$ je asociativní, komutativní a distributivní (vzhledem k $+$). Navíc $·$ zahrnuje i násobení v ®R a násobení skalárem.
		\item $|z|^2 = z \overline{z}$, $\overline{z_1 · z_2} = \overline{z_1}·\overline{z_2}$, $|z_1·z_2| = |z_1|·|z_2|$, $z + \overline{z} = 2 \Re z$, $z - \overline{z} = 2i\Im z$, $z, z_1, z_2 \in ®C$.
		\item $\forall z \in ®C, z ≠ 0: \exists z^{-1} \in ®C: z z^{-1} = 1$, konkrétně $z^-1 = \frac{\overline{z}}{|z|^2}$.
		\item $(®C, +, ·)$ je komutativní těleso.
	\end{itemize}
\end{tvrzeni}

\begin{upozorneni}
	®C nelze „rozumně“ lineárně uspořádat.
\end{upozorneni}

\begin{poznamka}[Lineární zobrazení]
	Lineární zobrazení $®R^2 \rightarrow ®R^2$ (®R-lineární zobrazení) jsou reálné matice řádu 2. Lineární zobrazení $®C \rightarrow ®C$ (®C-lineární) jsou komplexní čísla.

	Lineární zobrazení $L = \begin{bmatrix} a & c \\ b & d \end{bmatrix}$ je tedy ®C-lineární právě tehdy, když $a = d$ a $b = -c$.
\end{poznamka}

\begin{poznamka}[Úmluva]
	„Funkce“ znamená funkci z ®C do ®C, není li řečeno jinak.
\end{poznamka}

\begin{definice}[Značení (okolí, prstencové okolí)]
	$$ z_0 \in ®C, \epsilon > 0: ©U(z_0, \epsilon):=\{z \in ®C: |z - z_0| < \epsilon\}, ©P(z_0, \epsilon) := \{z \in ®C: 0 < |z - z_0| < \epsilon\}. $$
\end{definice}

\begin{definice}[Limita, spojitost]
	$$ \lim_{z \rightarrow z_0} f(z) = w \in ®C ≡ \forall \epsilon \exists \delta > 0 \forall z \in ®C: z \in ©P(z_0, \delta) \implies f(z) \in ©U(w, \epsilon). $$
	$f$ je spojitá v $z_0$, jestliže $\lim_{z \rightarrow z_0} f(z) = f(z_0)$.
\end{definice}

\begin{definice}[Derivace]
	$f: ®R^2 \rightarrow ®R^2$ je v bodě $z_0 \in ®R^2$ ®R-diferencovatelná, jestliže existuje ®R-lineární zobrazení $L: ®R^2 \rightarrow ®R^2$ takové, že
	$$ \lim_{h \rightarrow 0} \frac{f(z_0 + h) - f(z) - Lh}{|h|} = 0 $$
	Značíme $L =: df(z_0)$.
	$$ df(z_0) = \begin{bmatrix} \frac{\partial f_1}{\partial x}(z_0) & \frac{\partial f_1}{\partial y}(z_0) \\ \frac{\partial f_2}{\partial x}(z_0) & \frac{\partial f_2}{\partial y}(z_0) \end{bmatrix}  $$

	$f: ®C \rightarrow ®C$ je v bodě $z_0 \in ®C$ ®C-diferencovatelná, jestliže existuje
	$$ f'(z_0) := \lim_{h \rightarrow 0} \frac{f(z_0 + h) - f(z_0)}{h} \in ®C. $$
	$f'$ nazýváme komplexní derivace funkce.
\end{definice}

\begin{poznamka}
	Pro $(f±g)'$, $(f·g)'$, $(f / g)'$, $(f \circ g)'$ platí stejné vzorce jako pro funkce $®R \rightarrow ®R$.
\end{poznamka}

\begin{veta}[Cauchy-Riemann]
	Nechť $f$ je komplexní funkce definovaná na nějakém okolí bodu $z_0 \in ®C$. Pak následující podmínky jsou ekvivalentní:
	
	\begin{itemize}
		\item Existuje $f'(z_0)$.
		\item Existuje $d f(z_0)$ a $d f(x_0)$ je ®C-lineární.
		\item Existuje $d f(z_0)$ a platí
			$$ \frac{\partial f_1}{\partial x}(z_0) = \frac{\partial f_2}{\partial y}(z_0), \frac{\partial f_1}{\partial y}(z_0) = - \frac{\partial f_2}{\partial x}(z_0). $$
			(Tzv. Cauchy-Riemannova podmínka.)
	\end{itemize}

	\begin{dukazin}
		Druhý a třetí bod je ekvivalentní z poznámky o lineárních zobrazeních.

		$w = f'(z_0) \Leftrightarrow 0 = \lim_{h \rightarrow 0} \frac{f(z_0 + h) - f(z_0) - w h}{h}$. Vynásobíme $\frac{h}{|h|}$:
		$$ \Leftrightarrow 0 = \lim_{h \rightarrow 0} \frac{f(z_0 + h) - f(z_0) - wh}{|h|} \Leftrightarrow df(z_0) h = wh. $$
	\end{dukazin}
\end{veta}

\begin{poznamka}
	Existuje-li $f'(z_0)$, pak $df(z_0)h = f'(z_0)h, h \in ®C$ a $f'(z_0) = \frac{\partial f}{\partial x}(z_0)$.

	Cauchy-Riemannova podmínka je ekvivalentní $\frac{\partial f}{\partial x}(z_0) = -i \frac{\partial f}{\partial y}(z_0)$.
\end{poznamka}

% 06. 10. 2022

\begin{definice}[Holomorfní funkce]
	Nechť $G \subseteq ®C$ je otevřená a $f: G \rightarrow ®C$. Potom $f$ je holomorfní na $G$, pokud je $f$ ®C-diferencovatelná v každém bodě $G$.
\end{definice}

\begin{definice}[Exponenciála]
	$$ \exp(z) := e^x·(\cos y + i·\sin y), z = x + y·i \in ®C. $$
\end{definice}

\begin{tvrzeni}[Vlastnosti exponenciály]
	$\exp|_{®R}$ je reálná exponenciála, $\exp(z + w) = \exp(z)·\exp(w)$, $\exp'(z) = \exp(z)$ ($z \in ®C$), $\exp(z) = \sum_{n=0}^∞ \frac{z^n}{n!}$, $\exp(®C) = ®C \setminus \{ 0 \}$, $\exp$ není prostá na ®C a je $2\pi$ periodická, dokonce $\exp(z) = \exp(w) \Leftrightarrow \exists k \in ®Z: w = z + 2k\pi·i$, nechť $P := \{z \in ®C | \Im z \in (-\pi, \pi]\}$, potom $\exp|_P$ je prostá a $\exp(P) = ®C \setminus \{0\}$.
\end{tvrzeni}

\begin{definice}[Logaritmus a hlavní hodnota logaritmu]
	Nechť $z \in ®C \setminus \{0\}$. Položme
	$$ Log z := \{w \in ®C | \exp w = z\}, $$
	$$ \log z := \log|z| + i·\arg z. \qquad (\text{Hlavní hodnota logaritmu.}) $$
\end{definice}

\begin{tvrzeni}[Vladstnosti logaritmu]
	Nechť $z \in ®C \setminus \{0\}$. Potom

	\begin{itemize}
		\item $Log z = \{\log z + 2k\pi i | k \in ®Z\}$, $\log=(\exp|_P)^{-1}$
		\item $\log$ není spojitá na žádném $z \in (-∞, 0]$, ale $\log \in ©H(®C \setminus (-∞, 0])$. Navíc $\log' z = \frac{1}{z}$, $z \in ®C \setminus (-∞, 0]$
		\item $\log(1 - z) = - \sum_{n=1}^∞ \frac{z^n}{n}$, $|z| < 1$.
	\end{itemize}
\end{tvrzeni}

\begin{upozorneni}
	Neplatí $\log \exp z = z$ a $\log(z·w) = \log z + \log w$!
\end{upozorneni}

\begin{definice}
	Nechť $z \in ®C \setminus \{0\}$ a $\alpha \in ®C$. Potom hlavní hodnotou $\alpha$-té mocniny $z$ definujeme
	$$ z^{\alpha} := \exp(\alpha \log z). $$
	Položme
	$$ M_\alpha(z) := \{\exp(\alpha · w) | w \in Log z\}. $$
\end{definice}

\begin{tvrzeni}[Vlastnosti mocniny]
	\ 

	\begin{itemize}
		\item $e^z = \exp(z·\log e) = \exp(z)$.
		\item Je-li $z > 0$ a $\alpha \in ®R$, potom $z^\alpha$ je definována stejně jako v MA.
		\item $M_{\alpha}(z) = \{z^\alpha · e^{2k\pi · i · \alpha} | k \in ®Z\}$, $z ≠ 0$.
		\item $\(z^\alpha\) = \alpha·z^{\alpha - 1}$, $z \in ®C \setminus (-∞, 0]$, $\alpha \in ®C$.
		\item $(1 + z)^\alpha = \sum_{n=0}^∞ \binom{\alpha}{n} z^n$, $|z| < 1$, kde
			$$ \binom{\alpha}{n} := \frac{\alpha·(\alpha - 1)·…·(\alpha - n + 1)}{n!}, \qquad \alpha \in ®C. $$
	\end{itemize}
\end{tvrzeni}

\begin{poznamka}[Zápočet]
	Zápočet dostaneme za aktivní účast na cvičení
\end{poznamka}

% 13. 10. 2022

\begin{poznamka}
	Je-li $f: ®C \rightarrow ®C$, potom
	$$ f(x) = \frac{f(x) + f(-x)}{2} + \frac{f(x) - f(-x)}{2}, $$
	tedy $f$ lze rozložit na sudou a lichou část.

	Sudá část exponenciely je $\cosh$ a lichá $\sinh$.
\end{poznamka}

\begin{definice}[Goniometrické funkce]
	$$ e^{iz} = \cos z + i·\sin z, $$
	kde
	$$ \cos z := \frac{e^{iz} + e^{-iz}}{2}, \qquad \sin z := \frac{e^{iz} - e^{-iz}}{2i}, \qquad z \in ®C. $$
\end{definice}

\begin{tvrzeni}[Vlastnosti]
	\ 
	\begin{itemize}
		\item $\cos$ i $\sin$ jsou rozšířením funkcí z ®R do ®C.
		\item $\sin' z = \cos z$, $\cos' z = \sin z$.
		\item $\sin$ i $\cos$ jsou $2\pi$ periodické funkce, ale nejsou omezené, platí, že $\sin ®C = ®C = \cos ®C$.
		\item Platí $\sin^2 z + \cos^2 z = 1$.
		\item $\sin z = \sum_{n=0}^{∞} \frac{(-1)^n z^{2n + 1}}{(2n + 1)!}$, $\cos z = \sum_{n=0}^∞ (-1)^n \frac{z^{2n}}{(2n)!}$.
	\end{itemize}
\end{tvrzeni}

\section{Křivkový integrál}
\begin{definice}[Značení]
	Nechť $\phi:[\alpha, \beta] \rightarrow ®C$. Potom $\phi$ je křivka, pokud je $\phi$ spojité, $\phi$ je regulární křivka, pokud je $\phi$ po částech spojitě diferencovatelné tzn. $\phi$ je spojitá na $[\alpha, \beta]$ a existuje dělení $\alpha = t_0 < t_1 < … < t_n = \beta$ takové, že $\phi|_{[t_i, t_n]}$ je diferencovatelné.

	Úsečka: Nechť $a, b \in ®C$, potom $\phi(t) = a + t·(b - a)$, $t \in [0, 1]$ je úsečka z $a$ do $b$. Značíme $[a, b]$.

	Řekneme, že křivka $\phi$ je lomená čára v ®C, existují-li $z_1, …, z_k \in ®C$ taková, že
	$$ \phi = [z_1, z_2] + [z_2, z_3] + … + [z_{k-1}, z_k]. $$
\end{definice}

\begin{poznamka}[Úmluva]
	Pokud neřekneme něco jiného, křivkou budeme rozumět regulární křivku v ®C.
\end{poznamka}

\begin{definice}[Délka křivky]
	$$ V(\phi) = \int_\alpha^\beta |\phi'(t)| dt, $$
	je-li $\phi$ regulární.
\end{definice}

\begin{definice}
	Nechť $\phi: [\alpha, \beta] \rightarrow ®C$ je regulární křivka a $f: <\phi> \rightarrow ®C$ je spojitá. Potom definujeme
	$$ \int_\phi f := \int_\alpha^\beta f(\phi(t))·\phi'(t) dt. $$
\end{definice}

\begin{poznamka}
	Křivkový integrál konverguje jako Riemannův.

	$$ \int_\phi f(z) dz, $$
	Nechť $z_0 \in ®C$, $r \in (0, +∞)$ a $\phi(t) = z_0 + r e^{it}$, $t \in [0, 2\pi]$. Potom
	$$ \int_\phi (z - z_0)^n dz = \int_0^{2\pi} r^n e^{i n t}·2·r·e^{it} dt = i·r^{n+1} \int_0^{2\pi} e^{i(n + 1) t} dt = $$
	$2\pi i$, pokud $n = -1$, 0, pokud $n \in ®Z$ a $n ≠ -1$.
\end{poznamka}

\begin{tvrzeni}[Vlastnosti křivkového integrálu]
	Je-li $\phi$ křivka, $f$ a $g$ jsou spojité funkce na $<\phi>$ a $A \in ®C$, ptotom
	$$ \int_\phi (Af + g) = A\int_\phi f + \int_\phi g. $$

	Je-li $\phi$ křivka a $f$ je spojitá funkce na $<\phi>$, potom
	$$ |\int_\phi f| ≤ \max_{<\phi>} |f|·V(\phi). $$

	Nechť $\phi: [\alpha, \beta] \rightarrow ®C$, $\psi: [\gamma, \delta] \rightarrow ®C$ a $\phi(\beta) = \psi(\gamma)$. Potom
	$$ \int_{\phi + \psi} f = \int_\phi f + \int_\psi f \land \int_{-\phi} f = - \int_\phi f, $$
	kde $(-\phi)(t) := \phi(-t)$, $t \in [-\beta, -\alpha]$ je opačná křivka k $\phi$.

	Křivkový integrál nezávisí na parametrizaci křivky: Nechť $\phi: [\alpha, \beta] \rightarrow ®C$ je křivka, $\omega: [\gamma, \delta] \rightarrow [\alpha, \beta]$ je spojitě diferencovatelné s $\omega' > 0$ a $\psi = \phi\circ\omega$. Potom $\int_\psi f = \int_\phi f$.

	\begin{dukazin}
		Jednoduchý, ukázán na přednášce pro některé body.
	\end{dukazin}
\end{tvrzeni}

\begin{definice}[Primitivní funkce]
	Řekneme, že funkce $f$ má na otevřené $G \subset ®C$ primitivní funkci $F$, pokud $F' = f$ na $G$.
\end{definice}

\begin{veta}[O výpočtu křivkového integrálu pomocí primitivní funkce]
	Nechť $G \subset ®C$ je otevřená a $f$ má na $G$ primitivní funkci $F$. Nechť $\phi:[\alpha, \beta] \rightarrow G$ je regulární křivka a $f$ je spojitá\footnote{Tohle je zbytečný předpoklad, ale to ještě neumíme dokázat.} na $<\phi>$. Potom
	$$ \int_\phi f = F(\phi(\beta)) - F(\phi(\alpha)), $$
	je-li navíc $\phi$ uzavřená, tzn. $\phi(\alpha) = \phi(\beta)$, pak
	$$ \int_\phi f = 0. $$

% 20. 10. 2022 Z poznámek
% 27. 10. 2022 Z poznámek

	\begin{dukazin}
        Z Cauchy-Riemannovy věty plyne, že
        $$ \frac{d}{dt}(F(\phi(t))) = \frac{\partial F}{\partial x} \phi_1' + \frac{\partial F}{\partial y} \phi_2' = F'\phi_1' + i F'\phi_2' = F'(\phi(t))\phi'(t). $$

        Tato rovnost platí až na konečně mnoho $t \in [\alpha, \beta]$, neboli $F \circ \phi$ je zobecněná primitivní funkce k integrandu. Máme tedy
        $$ \int_\phi f = \int_\alpha^\beta f(\phi(t))\phi'(t)dt = \int_\alpha^\beta \frac{d}{dt} (F(\phi(t))) dt = F(\phi(\beta)) - F(\phi(\alpha)). $$
	\end{dukazin}
\end{veta}

\begin{veta}
	Funkce $f$ je konstantní na oblasti $G \subset ®C$, právě když $f' = 0$ na $G$.

	\begin{dukazin}
		„$\implies$“: Jasné. „$\impliedby$“: Nechť $z, w \in G$ a $\phi$ je lomená čára v $G$ spojující $z$ a $w$. Potom $f(w) - f(z) = \int_\phi f' = 0$, protože $f$ je primitivní funkcí k $f'$ na $G$.
	\end{dukazin}
\end{veta}

\begin{dusledek}
	Jsou-li $F_1$, $F_2$ primitivní funkce k $f$ na oblasti $G \subset ®C$, potom existuje $c \in ®C$ tak, že $F_2 = F_1 + c$

	\begin{dukazin}
		$$ (F_2 - F_1)' = F_2' - F_1' = f - f = 0 $$
	\end{dukazin}
\end{dusledek}

\begin{veta}[O existenci primitivní funkce]
	Nechť $G \subset ®C$ je oblast a $f$ je spojitá na $G$, tak následující je ekvivalentní
	\begin{enumerate}
		\item $f$ má na $G$ primitivní funkci;
		\item $\int_\phi f = 0$ pro každou uzavřenou křivku $\phi$ v $G$;
		\item $\int_\phi f$ nezávisí v $G$ na křivce $\phi$, tzn. pro každé dvě křivky $\phi: [\alpha, \beta] \rightarrow G$, $\psi: [\gamma, \delta] \rightarrow G$ takové, že $\phi(\alpha) = \psi(\gamma)$ a $\phi(\beta) = \psi(\delta)$ platí $\int_\phi f = \int_\psi f$.
	\end{enumerate}

	\begin{dukazin}
		„$1. \implies 2.$“: Víme z věty o výpočtu integrálu pomocí primitivní funkce.

		„$2. \implies 3.$“: Položme $\tau := \phi + (- \psi)$. Potom je $\tau$ uzavřená a z 2. dostaneme
		$$ 0 = \int_\tau f = \int_\phi f - \int_\psi f. $$

		„$3. \implies 1.$“: Volme $z_0 \in G$ pevně. Pro každé $z \in G$ najděme lomenou čáru $\phi_z$ v $G$, která začíná v $z_0$ a končí v $z$. Definujeme $F(z) := \int_{\phi_z} f$, $z \in G$. Definice $F$ je korektní z 3. Ukážeme, že $F$ je hledaná primitivní funkce k $f$ na $G$. Nechť $z_1 \in G$. Dokážeme, že $F'(z_1) = f(z_1)$. Volme $r > 0$, aby $U(z_1, r) \subset G$. Je-li $|h| < r$, potom
		$$ F(z_1 + h) - F(z_1) = \int_{\phi_{z_1} + u} f - \int_{\phi_{z_1}} f = \int_u f, $$
		kde $u = [z_1; z_1 + h]$ je úsečka. Tedy
		$$ F(z_1 + h) - F(z_1) = \int_u f = \int_0^1 f(z_1 + th)h dt, $$
		$$ \frac{F(z_1 + h) - F(z_1)}{h} - f(z_1) = \int_0^1 (f(z_1 + th) - f(z_1))dt \rightarrow 0, $$
		neboť $|\int_0^1 (f(z_1 + th) - f(z_1))dt| ≤ \max_{z \in [z_1; z_1+h]} |f(z) - f(z_1)| \rightarrow 0$ ze spojitosti $f$ v $z_1$.
	\end{dukazin}
\end{veta}

\begin{poznamka}[Značení]
	Řekněme, že $M \subset ®C$ je hvězdovitá, pokud existuje $z_0 \in M$ (tzv. střed hvězdovitosti), pro který $[z_0; z] \subset M$ pro každé $z \in M$.

	\begin{poznamkain}
		Konvexní $\subsetneq$ hvězdovitá.
	\end{poznamkain}

	Řekneme, že $\triangle \subset ®C$ je trojúhelník s vrcholy $a, b, c \in ®C$, pokud
	$$ \triangle := \{\alpha·a + \beta·b + \gamma·c | \alpha, \beta, \gamma ≥ 0, \alpha + \beta + \gamma = 1\}, $$
	a značíme $\partial \triangle := [a; b] + [b; c] + [c; a]$.
\end{poznamka}

\begin{tvrzeni}[Dodatek]
	Nechť $f$ je spojitá funkce na hvězdovité oblasti $G \subset ®C$. Je-li
	$$ \int_{\partial \triangle} f = 0, $$
	pro každý trojúhelník $\triangle \subset G$, potom $f$ má na $G$ primitivní funkci.

	\begin{dukazin}
		Nechť $z_0$ je střed hvězdovitosti $G$. Pro každé $z \in G$ položme $\phi_z := [z_0; z]$ a $F(z) := \int_{\phi_z} f$.
	\end{dukazin}
\end{tvrzeni}

\begin{poznamka}[Cauchyho věty]
	Nechť $G \subset ®C$ je otevřená, $f \in ©H(G)$ a $\phi$ je uzavřená křivka v $G$. Potom Cauchyho věty nám říkají, za jakých podmínek na $G$ a $\phi$ je $\int_\phi f = 0$.
\end{poznamka}

\begin{veta}[Goursatovo lemma (Cauchyho věta pro $\triangle$)]
	Nechť $G \subset ®C$ je otevřená, $f \in ©H(G)$ a $\triangle$ je trojúhelník v $G$. Potom
	$$ \int_{\partial \triangle} f = 0. $$

	\begin{dukazin}
		Označme $\phi_0 := \partial \triangle$. Sporem: Předpokládejme, že $|\int_{\phi_0} f| =: K > 0$. Zřejmě $\triangle$ je nedegenerovaný. V $\triangle$ veďme střední příčky a označme $\psi_1$, $\psi_2$, $\psi_3$, $\psi_4$ obvody čtyř vzniklých trojúhelníků. Obvody vnitřních trojúhelníků $\psi_1$ (vlevo dole), $\psi_2$ (vpravo dole), $\psi_3$ (nahoře) a $\psi_4$ (uprostřed) probíháme proti směru hodinových ručiček. Potom
		$$ \int_{\phi_0} f = \int_{\psi_1} f + \int_{\psi_2} f + \int_{\psi_3} f + \int_{\psi_4} f. $$
		Ex. $j_1 = 1, …, 4$ tak, že $|\int_{\psi_{j_1}} f| ≥ \frac{K}{4}$ a $V(\psi_{j_1}) = \frac{V(\phi)}{2}$. Označme $\phi_1 = \psi_{j_1}$. Indukcí sestrojíme posloupnost trojúhelníků tak, že
		$$ |\int_{\phi_j} f| ≥ \frac{K}{4^j} \land V(\phi_j) = \frac{V(\phi)}{2^j}. $$

		Máme, že $\bigcup_{j=0}^∞ \triangle_j = \{z_0\} \subset G$, protože $\diam(\triangle_j) \rightarrow 0$. Položme
		$$ \epsilon(z) := \begin{cases}\frac{f(z) - f(z_0)}{z - z_0} - f'(z_0), & z \in G\setminus \{z_0\},\\ 0, & z = z_0.\end{cases} $$

		Potom $\epsilon$ je spojitá na $G$ a máme pro $j \in ®N_0$:
		$$ \int_{\phi_j} f(z) dz = \int_{\phi_j} (f(z_0) + f'(z_0)(z - z_0))dz + \int_{\phi_j}\epsilon(z)(z - z_0) dz, $$
		kde první integrand vpravo má primitivní funkci na ®C a první integrál je roven 0. Pro každé $j \in ®N_0$ dostaneme
		$$ 0 < \frac{K}{4^j} ≤ |\int_{\phi_j} f| = |\int_{\phi_j} \epsilon(z)(z - z_0)| ≤ V^2(\phi_j)·\max_{<\phi_j>} |\epsilon| = \frac{V^2(\phi)}{4^j}·\max_{<\phi_j>}|\epsilon|, $$
		kde třetí nerovnost platí díky tomu, že $|z - z_0| ≤ V(\phi_j)$. Z předchozího tedy máme (po vynásobení $4^j$):
		$$ 0 < K ≤ V^2(\phi)·\max_{<\phi_j>}|\epsilon| \rightarrow 0, $$
		protože $\epsilon$ je spojitá v $z_0$ a $\epsilon(z_0) = 0$. \lightning.
	\end{dukazin}
\end{veta}

\begin{veta}[Cauchyho věta pro hvězdovité oblasti]
	Nechť $G \subset ®C$ je hvězdovitá oblast a $f \in ©H(G)$. Potom $f$ má na $G$ primitivní funkci. (Ekvivalentně: $\int_{\phi} f = 0$ pro každou uzavřenou křivku $\phi$ v $G$.)

	\begin{dukazin}
		Z Goursatova lemmatu a dodatku.
	\end{dukazin}
\end{veta}

\begin{poznamka}
	Goursatovo lemma a tedy i předchozí věta platí i pro funkci $f$, která je spojitá na $G$ a holomorfní na $G \setminus \{z_0\}$ pro nějaké $z_0 \in G$.

	\begin{dukazin}
		Nechť $\triangle$ je nedegenerovaný trojúhelník v $G$. Pak rozebereme případy kde leží $z_0$.
	\end{dukazin}
\end{poznamka}

\begin{veta}[O derivování podle komplexního parametru]
	Nechť $\phi$ je křivka v ®C a $\Omega \subset ®C$ je otevřená. Nechť $F(z, s)$ a komplexní derivace $\frac{\partial F}{\partial s}(z, s)$ jsou spojité komplexní funkce na $<\phi> \times \Omega$. Pro každé $s \in \Omega$ položme $\Phi(s) := \int_\phi F(z, s) dz$. Potom $\Phi \in ©H(\Omega)$ a $\Phi'(s) = \int_\phi \frac{\partial F}{\partial s}(z, s) dz$, $s \in \Omega$.

	\begin{dukazin}
		Pro $s = s_1 + i s_2 = (s_1, s_2) \in \Omega$ máme $\Phi(s) = \int_\alpha^\beta F(\phi(t), (s_1, s_2)) \phi'(t)dt$. Podle vět o spojitosti a derivování integrálu závislého na parametru máme
		$$ \frac{\partial \Phi}{\partial s_j}(s) = \int_\phi \frac{\partial F}{\partial s_j}(z, s) dz, $$
		pro $s \in \Omega$ a $j \in [2]$. Navíc jsou tyto parciální derivace spojité a splňují podmínky Cauchy-Riemannovy věty, tedy $\Phi$ je komplexně diferencovatelná a komplexní derivace se rovná derivaci vzhledem k té první proměnné.
	\end{dukazin}
\end{veta}

\begin{definice}[Index bodu křivky]
	Nechť $\phi$ je uzavřená křivka v ®C a $s \in ®C \setminus <\phi>$. Potom číslo
	$$ \ind_\phi s := \frac{1}{2\pi i} \int_\phi \frac{dz}{z - s} $$
	nazveme indexem bodu s vzhledem ke křivce $\phi$.
\end{definice}

\begin{poznamka}
	Ukážeme si, že $\ind_\phi$ se rovná počtu oběhů $\phi$ kolem $s$ v kladném směru (tzn. proti směru hodinových ručiček).
\end{poznamka}

\begin{veta}[O základních vlastnostech indexu]
	Nechť $\phi$ je uzavřená křivka v ®C a $G := ®C \setminus <\phi>$. Potom $G$ je otevřena, funkce $s \mapsto \ind_\phi s$ je konstantní na každé komponentě $G$ a na jediné její neomezené komponentě je nulová.

	\begin{dukazin}
		Podle předchozí věty je $\Phi(s) := \frac{1}{2 \pi i} \int_\phi \frac{dz}{z - s}$, $s \in G$ holomorfní a pro každé $s \in G$ je $\Phi'(s) = \frac{1}{2\pi i} \int_\phi \frac{dz}{(z - s)^2} = 0$, protože $f(z) := \frac{1}{(z - s)^2}$ má primitivní funkci na $®C \setminus \{s\}$. Tedy $\Phi$ je konstantní na každé komponentě $G$.

		Volíme $R > 0$, aby $<\phi> \subset U(0, R)$. Potom $®C \setminus U(0, R)$ je obsaženo v jediné neomezené komponentě $G_0$ množiny $G$. Navíc pro $s \in ®C \setminus U(0, R)$ je funkce $g(z) := \frac{1}{z - s}$, $z \in U(0, R)$ holomorfní a dle Cauchyho věty pro hvězdovitou oblast je $\Phi(s) = 0$.
	\end{dukazin}
\end{veta}

% 03. 11. 2022

\begin{veta}[Cauchyův vzorec na kruhu]
	Nechť $G \subset ®C$ je otevřené a $f \in ©H(G)$. Nechť $\overline{U(z_0), r} \subset G$ a $p(t) = z_0 + r·e^{it}$, $t \in [0, 2\pi]$. Potom platí
	$$ \frac{1}{2\pi} \int_\phi \frac{f(z)}{z - s} dz = f(s), |s - z_0| < r; $$
	$$ \frac{1}{2\pi} \int_\phi \frac{f(z)}{z - s} dz = 0, |s - z_0| > r; $$

	\begin{dukazin}
		1. Nechť $|s - z_0|<r$. Volme $R > r$, aby $U(z, R) \subset G$. Položme
		$$ h(z) := \frac{f(z) - f(s)}{z - s}, z \in U(z_0, R) \setminus \{s\}; $$
		$$ h(z) := f'(s), z = s. $$
		
		Zřejmě $h \in ©H(U(z_0, R) \setminus \{s\})$ he spojitá hvězdovitá oblast $U(z_0, R)$. Z Cauchyho věty je
		$$ 0 = \frac{1}{2\pi} \int_\phi h = \frac{1}{2\pi} \int_\phi \frac{f(z) dz}{z - s} - \frac{1}{2\pi} \int_\phi \frac{dz}{z - d}·f_s. $$

		2. Nechť $|s - z_0| > r$. Volme $R \in (r, |z_0 - s|)$, aby $U(z_0, R) \subset G$. Potom
		$$ g(t) := \frac{f(z)}{z - s} \in ©H(U(z_0, R)) $$
		a z Cauchyho věty je
		$$ \frac{1}{2\pi} \int_\phi g = 0. $$
	\end{dukazin}
\end{veta}

\begin{dusledek}
	Nechť $G \subset ®C$ je otevřená a $f \in ©H(G)$. Potom $f$ má komplexní derivace libovolného řádu všude na $G$.

	Tedy nechť $U(z_0, r) \subset G$ a $\phi$ je jako v předchozím. Potom
	$$ \frac{k!}{2\pi} \int_\phi \frac{f(z) dz}{(z - s)^{k+1}} = f^{(k)}(s), |s - z_0| < r, k \in ®N. $$
	Zde $f^{(0)} := f$ a $k$-tá komplexní derivace $f^{(k)}$ je definovaná jako $f^{(k)} := \(f^{(k-1)}\)'$, má-li pravá strana smysl.

	\begin{dukazin}
		Z věty o derivaci integrálu podle komplexního parametru a předchozí věty, protože
		$$ \frac{d^k}{ds^k} \(\frac{1}{z - s}\) = \frac{k!}{(z - s)^{k+1}}, \qquad z ≠ s. $$
	\end{dukazin}
\end{dusledek}

\begin{veta}[Morera]
	Nechť $f$ je spojitá funkce na otevřené $G \subset ®C$. Potom $f \in ©H(G)$, právě když
	$$ \int_{\partial \triangle} f = 0, \qquad \forall \triangle \subset G \text{ trojúhelník}. $$

	\begin{dukazin}
		„$\implies$“: Goursat. „$\impliedby$“: Nechť $U := U(z_0, R) \subset G$. Protože $f$ je spojitá na hvězdovité oblasti $U$ a platí pro ni rovnost výše, má $f$ na $U$ primitivní funkci $F$, tzn. $f = F'$ na $U$. Protože $F \in ©H(U)$, máme $f' = F''$ na $U$, tudíž $f \in ©H(U)$. Tedy i $f \in ©H(G)$.
	\end{dukazin}
\end{veta}

\begin{veta}[Cauchyho odhady]
	Nechť $z_0 \in ®C$, $r \in (0, +∞)$ a $f$ je holomorfní funkce na otevřené množině obsahující $\overline{U(z_0, r)}$. Potom pro každé $k \in ®N_0$ je
	\begin{itemize}
		\item[CO1] $\forall s \in U := U(z_0, r)$:
			$$ |f^{(k)}(s)| ≤ \frac{(k!) r}{(d(s))^{k+1}}, $$
			kde $d(s) := \dist(s, \partial U)$;
		\item[CO2] $\forall s \in U(z_0, \frac{r}{2})$:
			$$ |f^{(k)}(s)| ≤ \frac{k! 2^{k+1}}{r^2}·\max_{\partial U} |f|; $$
		\item[CO3] $|f^{(k)}(z_0)| ≤ \frac{k!}{r^k}·\max_{\partial U} |f|$.
	\end{itemize}

	\begin{dukazin}[CO1]
		Z věty výše (pro $\phi$ stejné jako tam)
		$$ |f^{(k)}(z)| = |\frac{k!}{2\pi} \int_\phi \frac{f(z) ds}{(z - s)^{k+1}}| ≤ \frac{k!}{2\pi}·2\pi r \max_{<\phi>} |f| \frac{1}{(d(s))^{k+1}}, $$
		protože $|z - s| ≥ d(s)$ $\forall z \in <\phi>$.
	\end{dukazin}

	\begin{dukazin}[CO2 a CO3]
		Plyne z CO1, neboť $d(s) ≥ \frac{r}{2} \forall s \in U(z_0, \frac{r}{2})$ a $d(z_0) = r$.
	\end{dukazin}
\end{veta}

\begin{veta}[Liouville]
	Je-li $f$ holomorfní a omezená funkce na ®C, potom je $f$ konstantní.

	\begin{dukazin}
		Ukážeme, že $f' = 0$ na ®C: Označme $M := \sup_{®C} |f| < +∞$. Nechť $z_0 \in ®C$. Z CO3 pro každé $r > 0$ platí
		$$ |f'(z_0)| ≤ \frac{M}{r} \rightarrow 0, $$
		tudíž $f'(z_0) = 0$.
	\end{dukazin}
\end{veta}

\begin{dusledek}[Zakladní věta algebry]
	V ®C má každý polynom stupně alespoň 1 vždy alespoň jeden kořen.

	\begin{dukazin}
		Nechť $p(z) := a_n z^n + … + a_0 z^0$, kde $a_j \in ®C$, $n ≥ 1$ a $a_n ≠ 0$. Sporem: Předpokládejme, že $p ≠ 0$ na ®C. Potom $f:=\frac{1}{p}$ je holomorfní a omezená na ®C. Z Liouvilleovy věty je konstantní, tedy i $p = \frac{1}{f}$ je konstantní a $p' = 0 = p^{(n)} = a_n · n!$ $\implies$ $a_n = 0$. \lightning.
	\end{dukazin}
\end{dusledek}

\begin{lemma}
	Nechť $\phi$ je křivka v ®C, $f_i$ jsou spojité funkce na $<\phi>$ pro $n \in ®N$ a $f_i \rightrightarrows f$ na $<\phi>$. Potom $f$ je spojitá na $<\phi>$ a
	$$ \int_\phi f_n \rightarrow \int_\phi f. $$

	\begin{dukazin}
		Platí
		$$ |\int_\phi f_i - \int_\phi f| = |\int_\phi (f_n - f)| ≤ V(\phi)·\max_{<\phi>} |f_n - f| \rightarrow 0. $$
	\end{dukazin}
\end{lemma}

\begin{veta}[Weierstrass]
	Nechť $G \subset ®C$ je otevřená, $f_n \in ©H(F)$ pro $n \in ®N$ a $f_n \overset{\text{Loc.}}\rightrightarrows f$ na $G$. Potom $f \in ©H(G)$ a $f_n^{(k)} \overset{\text{Loc.}}\rightrightarrows f^{(k)}$ na $G$ pro každé $k \in ®N$.

	\begin{dukazin}
		1. Zřejmě $f$ je spojitá na $G$. Nechť $\triangle$ je trojúhelník v $G$. Potom
		$$ 0 \overset{G ?}= \int_{\partial \triangle} f_n \rightarrow \int_{\partial \triangle} f = 0. $$
		Z Morera je $f \in ©H(G)$.

		2. Nechť $k \in ®N$ a $z_0 \in G$. Volme $r > 0$, aby $\overline{U(z_0, r)} \subset G$. Z CO2 máme, že $\forall s \in U(z_0, \frac{r}{2}):$
		$$ |f_n^{(k)}(s) - f^{(k)}(s)| = |(f_n - f)^{(k)}(s)| ≤ \frac{k! 2^{k+1}}{r^k}·\max_{\partial U(z_0, r)} |f_n - f| \rightarrow 0. $$
	\end{dukazin}
\end{veta}

% 10. 11. 2022

\section{Mocninné řady}
\begin{definice}[Mocninná řada]
	Nechť $\{a_n\}_{n=0}^∞ \subset ®C$ a $z_0 \in ®C$. Potom
	$$ \sum_{n=0}^∞ a_n (z - z_0)^n $$
	je mocninná řada s koeficienty $\{a_n\}$ a středem $z_0$.
\end{definice}

\begin{poznamka}[Vlastnosti]
	\begin{itemize}
		\item[Konvergence] Existuje $R \in [0, +∞]$ takové, že řada konverguje absolutně a lokálně stejnoměrně na $U(z_0, R) := \{z \in ®C |\ |z - z_0| < R\}$ a řada diverguje pro $|z - z_0| > R$. Číslo $R$ se nazývá poloměr konvergence a platí, že
			$$ R = \frac{1}{\limsup_{n\rightarrow ∞} \sqrt[n]{|a_n|}}, $$
			kde $\frac{1}{0} = +∞$ a $\frac{1}{+∞} = 0$.
		\item Označíme-li součet řady na $U(z_0, R)$ jako $f$, potom $f \in ©H(U(z_0, R))$ a pro každé $k \in ®N$ je
			$$ f^{(k)}(z) - \sum_{n=k}^∞a_n · n · (n-1) · … ·(n-k+1)·(z - z_0)^{n-k}, \qquad z \in U(z_0, R), $$
			speciálně $a_k = f^{(k)}(z) / k!$.
		
			Plyne z Weierstrassovy věty pro
			$$ S_N(z) = \sum_{n=0}^∞ a_n(z - z_0)^n. $$
			Zřejmě $S_N \rightrightarrows^{loc} f$ na $U(z_0, R)$, tudíž $S_N^{(k)} \rightrightarrows^{loc} f^{(k)}$ na $U(z_0, R)$, tudíž rovnost výše platí. Dosadíme-li do ní $z = z_0$, dostaneme
			$$ f^{(k)}(z_0) = a_k·k!. $$
	\end{itemize}
\end{poznamka}

\begin{veta}[O rozvoji holomorfní funkce do mocninné řady na kruhu]
	Nechť $R \in (0, +∞]$ a $f \in ©H(U(z_0, R))$. Potom existuje jediná mocninná řada $\sum_{n=0}^∞ a_n(z - z_0)^n$, která má na $U(z_0, R)$ součet $f$. Navíc platí, že $a_n = f^{(n)}(z_0) / n!$ pro každé $n \in ®N_0$.

	\begin{dukazin}
		Jednoznačnost plyne ze vzorce. Existence: Nechť $z \in U(z_0, R)$. Volme $r > 0$, aby $|z - z_0| < r < R$. Potom z Cauchyho věty je
		$$ f(z) = \frac{1}{2\pi i} \int_\phi \frac{f(w) dw}{w - z}, $$
		kde $\phi(t) := z_0 + r e^{i t}$, $t \in [0, 2\pi]$. Pro každé $w \in <\phi>$ máme
		$$ \frac{1}{w - z} = \frac{1}{(w - z_0) - (z - z_0)} = \frac{1}{w - z}·\frac{1}{1 - \frac{z - z_0}{w - z_0}} = \sum_{n=0}^∞ \frac{(z - z_0)^n}{(w - z_0)^{n+1}}, $$
		což konverguje stejnoměrně pro $w \in <\phi>$. Dosadíme:
		$$ f(z) = \frac{1}{2\pi i} \int_\phi \sum_{n=0}^∞ \frac{(z - z_0)^n}{(w - z_0)^{n+1}} f(w) dw = $$
		$$ \sum_{n=0}^∞ (z - z_0)^n · \frac{1}{2\pi i} \int_\phi \frac{f(w) dw}{(w - z_0)^{n+1}} = \sum_{n=0}^∞ (z - z_0)^n · \frac{f^{(n)}}{n!}. $$
	\end{dukazin}
\end{veta}

\begin{veta}[O nulovém bodě]
	Nechť $f$ je holomorfní funkce na okolí $x_0 \in ®C$ a $f(z_0) = 0$. Potom buď $\exists r > 0: f = 0$ na $U(z_0, r)$; nebo $\exists r > 0: f ≠ 0$ na $P(z_0, r) := U(z_0, r) \setminus \{z_0\}$.

	V druhém případě existuje jediné $p \in ®N$ tak, že
	$$ f(x_0) = 0 = f'(z_0) = … = f^{(p-1)}(z_0), f^{(p)}(z_0) ≠ 0. $$
	Číslo $p$ je tzv. násobnost nulového bodu $z_0$ funkce $f$.

	Navíc $z_0$ je nulový bod $f$ násobnosti $p \in ®N$, právě když existuje $r > 0$ a $g \in ©H(U(z_0, r))$ tak, že $\forall z \in U(z_0, r)$:
	$$ g(z) ≠ 0 \land f(z) = (z - z_0)^p g(z). $$

	\begin{dukazin}
		Máme $f(z) = \sum_{n=0}^∞ a_n(z - z_0)^n$, $z \in U(z_0, R)$. Pokud nenastane první případ, potom existuje $n \in ®N$ tak, že $a_n ≠ 0$. Zvolme nejmenší $p \in ®N$, aby $0 ≠ a_p = f^{(p)}(z_0) / p!$. Potom platí rovnost pro druhý případ a $\forall z \in U(z_0, R)$:
		$$ f(z) = a_p(z - z_0)^p = … = (z - z_0)^p \underbrace{\sum_{n=p}^∞ a_n(z - z_0)^{n - p}}_{=:g(z)}. $$
		Zřejmě $g \in ©H(U(z_0, R))$. Protože $g(z_0) = a_p ≠ 0$, $\exists r>0$ tak, že $g ≠ 0$ na $U(z_0, r)$. Tudíž $f(z) = (z - z_0)^p·g(z) ≠ 0$ na $P(z_0, r)$. Obrácené tvrzení plyne stejně snadno.
	\end{dukazin}
\end{veta}

\begin{veta}[O jednoznačnosti pro holomorfní funkce]
	Nechť $\O = G \subset ®C$ je oblast a $f, g \in ©H(G)$. Následující je ekvivalentní

	\begin{enumerate}
		\item $f = g$ na $G$;
		\item $M := \{z \in G | f(z) = g(z)\}$ má v $G$ hromadný bod, tzn. existuje $z_0$ tak, že $P(z_0, r) \cap M ≠ \O\ \forall r > 0$;
		\item Existuje $z_0 \in G$ tak, že
			$$ f^{(k)}(z_0) = g^{(k)}(z_0) \qquad \forall k \in ®N_0. $$
	\end{enumerate}

	\begin{dukazin}
		BÚNO předpokládejme, že $G = 0$, jinak uvažujme $f - g$. „$1 \implies 2$ a $2 \implies 3$“: Nechť $z_0 \in G$ je hromadný bod $M := \{z \in G | f(z) = 0\}$. Zřejmě $f(z_0) = 0$ a z předchozí věty je $f = 0$ na nějakém okolí $z_0$.

		„$3 \implies 1$“: Uvažme $N := \{z \in G | \forall k \in ®N_0: f^{(k)}(z) = 0\}$. Potom $\O ≠ N$ a $N$ je uzavřená v $G$, protože všechny $f^{(k)}$ jsou spojité. Navíc $N$ je otevřená. Nechť $z_1 \in N$. Podle věty o nulovém bodě existuje $r > 0$ tak, že $f = 0$ na $U(z_1, r)$. Tedy $U(z_0, r) \subset N$. Protože $G$ je oblast (tj. je souvislá, tedy neexistuje vlastní obojetná podmnožina), je $N = G$.
	\end{dukazin}
\end{veta}

\end{document}
