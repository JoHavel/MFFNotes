\documentclass[12pt]{article}					% Začátek dokumentu
\usepackage{../../MFFStyle}					    % Import stylu



\begin{document}

% 29. 09. 2022
\section*{Úvod}

\begin{poznamka}
	Mluvilo se o historii ®C.
\end{poznamka}

\begin{definice}[Prostor ®C]
	Prostor ®C komplexních čísel je prostor $®R^2$, v němž navíc definujeme násobení:
	$$ (x_1, y_1) \cdot (x_2, y_2) = (x_1·x_2 - y_1·y_2, x_1·y_2 + x_2·y_1). $$

	Ztotožníme $(x, 0) = x$, neboli $®R \subset ®C$. Značíme $i := (0, 1)$ (imaginární jednotka).
\end{definice}

\begin{definice}[Značení (komplexně sdružené číslo, reálná a imaginární složka)]
	$$ z = x + i·y \in ®C \implies \overline{z}:= x - i·y \land \Re z := x, \Im z := y. $$
\end{definice}

\begin{definice}[Modul / absolutní hodnota]
	$$ |z| := \sqrt{x^2 + y^2} $$
\end{definice}

\begin{tvrzeni}[Vlastnosti]
	\ 
	\begin{itemize}
		\item $z = (x, y) \in ®C$, potom $z = x + i·y$ a $(±i)^2 = -1$.
		\item Násobení $·: ®C^2 \rightarrow ®C$ je asociativní, komutativní a distributivní (vzhledem k $+$). Navíc $·$ zahrnuje i násobení v ®R a násobení skalárem.
		\item $|z|^2 = z \overline{z}$, $\overline{z_1 · z_2} = \overline{z_1}·\overline{z_2}$, $|z_1·z_2| = |z_1|·|z_2|$, $z + \overline{z} = 2 \Re z$, $z - \overline{z} = 2i\Im z$, $z, z_1, z_2 \in ®C$.
		\item $\forall z \in ®C, z ≠ 0: \exists z^{-1} \in ®C: z z^{-1} = 1$, konkrétně $z^-1 = \frac{\overline{z}}{|z|^2}$.
		\item $(®C, +, ·)$ je komutativní těleso.
	\end{itemize}
\end{tvrzeni}

\begin{upozorneni}
	®C nelze „rozumně“ lineárně uspořádat.
\end{upozorneni}

\begin{poznamka}[Lineární zobrazení]
	Lineární zobrazení $®R^2 \rightarrow ®R^2$ (®R-lineární zobrazení) jsou reálné matice řádu 2. Lineární zobrazení $®C \rightarrow ®C$ (®C-lineární) jsou komplexní čísla.

	Lineární zobrazení $L = \begin{bmatrix} a & c \\ b & d \end{bmatrix}$ je tedy ®C-lineární právě tehdy, když $a = d$ a $b = -c$.
\end{poznamka}

\begin{poznamka}[Úmluva]
	„Funkce“ znamená funkci z ®C do ®C, není li řečeno jinak.
\end{poznamka}

\begin{definice}[Značení (okolí, prstencové okolí)]
	$$ z_0 \in ®C, \epsilon > 0: ©U(z_0, \epsilon):=\{z \in ®C: |z - z_0| < \epsilon\}, ©P(z_0, \epsilon) := \{z \in ®C: 0 < |z - z_0| < \epsilon\}. $$
\end{definice}

\begin{definice}[Limita, spojitost]
	$$ \lim_{z \rightarrow z_0} f(z) = w \in ®C ≡ \forall \epsilon \exists \delta > 0 \forall z \in ®C: z \in ©P(z_0, \delta) \implies f(z) \in ©U(w, \epsilon). $$
	$f$ je spojitá v $z_0$, jestliže $\lim_{z \rightarrow z_0} f(z) = f(z_0)$.
\end{definice}

\begin{definice}[Derivace]
	$f: ®R^2 \rightarrow ®R^2$ je v bodě $z_0 \in ®R^2$ ®R-diferencovatelná, jestliže existuje ®R-lineární zobrazení $L: ®R^2 \rightarrow ®R^2$ takové, že
	$$ \lim_{h \rightarrow 0} \frac{f(z_0 + h) - f(z) - Lh}{|h|} = 0 $$
	Značíme $L =: df(z_0)$.
	$$ df(z_0) = \begin{bmatrix} \frac{\partial f_1}{\partial x}(z_0) & \frac{\partial f_1}{\partial y}(z_0) \\ \frac{\partial f_2}{\partial x}(z_0) & \frac{\partial f_2}{\partial y}(z_0) \end{bmatrix}  $$

	$f: ®C \rightarrow ®C$ je v bodě $z_0 \in ®C$ ®C-diferencovatelná, jestliže existuje $f'(z_0) := \lim_{h \rightarrow 0} \frac{f(z_0 + h) - f(z_0)}{h} \in ®C$. $f'$ nazýváme komplexní derivace funkce.
\end{definice}

\begin{poznamka}
	Pro $(f±g)'$, $(f·g)'$, $(f / g)'$, $(f \circ g)'$ platí stejné vzorce jako pro funkce $®R \rightarrow ®R$.
\end{poznamka}

\begin{veta}[Cauchy-Riemann]
	Nechť $f$ je komplexní funkce definovaná na nějakém okolí bodu $z_0 \in ®C$. Pak následující podmínky jsou ekvivalentní:
	
	\begin{itemize}
		\item Existuje $f'(z_0)$.
		\item Existuje $d f(z_0)$ a $d f(x_0)$ je ®C-lineární.
		\item Existuje $d f(z_0)$ a platí
			$$ \frac{\partial f_1}{\partial x}(z_0) = \frac{\partial f_2}{\partial y}(z_0), \frac{\partial f_1}{\partial y}(z_0) = - \frac{\partial f_2}{\partial x}(z_0). $$
			(Tzv. Cauchy-Riemannova podmínka.)
	\end{itemize}

	\begin{dukazin}
		Druhý a třetí bod je ekvivalentní z poznámky o lineárních zobrazeních.

		$w = f'(z_0) \Leftrightarrow 0 = \lim_{h \rightarrow 0} \frac{f(z_0 + h) - f(z_0) - w h}{h}$. Vynásobíme $\frac{h}{|h|}$:
		$$ \Leftrightarrow 0 = \lim_{h \rightarrow 0} \frac{f(z_0 + h) - f(z_0) - wh}{|h|} \Leftrightarrow df(z_0) h = wh. $$
	\end{dukazin}
\end{veta}

\begin{poznamka}
	Existuje-li $f'(z_0)$, pak $df(z_0)h = f'(z_0)h, h \in ®C$ a $f'(z_0) = \frac{\partial f}{\partial x}(z_0)$.

	Cauchy-Riemannova podmínka je ekvivalentní $\frac{\partial f}{\partial x}(z_0) = -i \frac{\partial f}{\partial y}(z_0)$.
\end{poznamka}

% 06. 10. 2022

\begin{definice}[Holomorfní funkce]
	Nechť $G \subseteq ®C$ je otevřená a $f: G \rightarrow ®C$. Potom $f$ je holomorfní na $G$, pokud je $f$ ®C-diferencovatelná v každém bodě $G$.
\end{definice}

\begin{definice}[Exponenciála]
	$$ \exp(z) := e^x·(\cos y + i·\sin y), z = x + y·i \in ®C. $$
\end{definice}

\begin{tvrzeni}[Vlastnosti exponenciály]
	$\exp|_{®R}$ je reálná exponenciála, $\exp(z + w) = \exp(z)·\exp(w)$, $\exp'(z) = \exp(z)$ ($z \in ®C$), $\exp(z) = \sum_{n=0}^∞ \frac{z^n}{n!}$, $\exp(®C) = ®C \setminus \{ 0 \}$, $\exp$ není prostá na ®C a je $2\pi$ periodická, dokonce $\exp(z) = \exp(w) \Leftrightarrow \exists k \in ®Z: w = z + 2k\pi·i$, nechť $P := \{z \in ®C | \Im z \in (-\pi, \pi]\}$, potom $\exp|_P$ je prostá a $\exp(P) = ®C \setminus \{0\}$.
\end{tvrzeni}

\begin{definice}[Logaritmus a hlavní hodnota logaritmu]
	Nechť $z \in ®C \setminus \{0\}$. Položme
	$$ Log z := \{w \in ®C | \exp w = z\}, $$
	$$ \log z := \log|z| + i·\arg z. \qquad (\text{Hlavní hodnota logaritmu.}) $$
\end{definice}

\begin{tvrzeni}[Vladstnosti logaritmu]
	Nechť $z \in ®C \setminus \{0\}$. Potom

	\begin{itemize}
		\item $Log z = \{\log z + 2k\pi i | k \in ®Z\}$, $\log=(\exp|_P)^{-1}$
		\item $\log$ není spojitá na žádném $z \in (-∞, 0]$, ale $\log \in ©H(®C \setminus (-∞, 0])$. Navíc $\log' z = \frac{1}{z}$, $z \in ®C \setminus (-∞, 0]$
		\item $\log(1 - z) = - \sum_{n=1}^∞ \frac{z^n}{n}$, $|z| < 1$.
	\end{itemize}
\end{tvrzeni}

\begin{upozorneni}
	Neplatí $\log \exp z = z$ a $\log(z·w) = \log z + \log w$!
\end{upozorneni}

\begin{definice}
	Nechť $z \in ®C \setminus \{0\}$ a $\alpha \in ®C$. Potom hlavní hodnotou $\alpha$-té mocniny $z$ definujeme
	$$ z^{\alpha} := \exp(\alpha \log z). $$
	Položme
	$$ M_\alpha(z) := \{\exp(\alpha · w) | w \in Log z\}. $$
\end{definice}

\begin{tvrzeni}[Vlastnosti mocniny]
	\ 

	\begin{itemize}
		\item $e^z = \exp(z·\log e) = \exp(z)$.
		\item Je-li $z > 0$ a $\alpha \in ®R$, potom $z^\alpha$ je definována stejně jako v MA.
		\item $M_{\alpha}(z) = \{z^\alpha · e^{2k\pi · i · \alpha} | k \in ®Z\}$, $z ≠ 0$.
		\item $\(z^\alpha\) = \alpha·z^{\alpha - 1}$, $z \in ®C \setminus (-∞, 0]$, $\alpha \in ®C$.
		\item $(1 + z)^\alpha = \sum_{n=0}^∞ \binom{\alpha}{n} z^n$, $|z| < 1$, kde
			$$ \binom{\alpha}{n} := \frac{\alpha·(\alpha - 1)·…·(\alpha - n + 1)}{n!}, \qquad \alpha \in ®C. $$
	\end{itemize}
\end{tvrzeni}

\begin{poznamka}[Zápočet]
	Zápočet dostaneme za aktivní účast na cvičení
\end{poznamka}

% 13. 10. 2022

\begin{poznamka}
	Je-li $f: ®C \rightarrow ®C$, potom
	$$ f(x) = \frac{f(x) + f(-x)}{2} + \frac{f(x) - f(-x)}{2}, $$
	tedy $f$ lze rozložit na sudou a lichou část.

	Sudá část exponenciely je $\cosh$ a lichá $\sinh$.
\end{poznamka}

\begin{definice}[Goniometrické funkce]
	$$ e^{iz} = \cos z + i·\sin z, $$
	kde
	$$ \cos z := \frac{e^{iz} + e^{-iz}}{2}, \qquad \sin z := \frac{e^{iz} - e^{-iz}}{2i}, \qquad z \in ®C. $$
\end{definice}

\begin{tvrzeni}[Vlastnosti]
	\ 
	\begin{itemize}
		\item $\cos$ i $\sin$ jsou rozšířením funkcí z ®R do ®C.
		\item $\sin' z = \cos z$, $\cos' z = \sin z$.
		\item $\sin$ i $\cos$ jsou $2\pi$ periodické funkce, ale nejsou omezené, platí, že $\sin ®C = ®C = \cos ®C$.
		\item Platí $\sin^2 z + \cos^2 z = 1$.
		\item $\sin z = \sum_{n=0}^{∞} \frac{(-1)^n z^{2n + 1}}{(2n + 1)!}$, $\cos z = \sum_{n=0}^∞ (-1)^n \frac{z^{2n}}{(2n)!}$.
	\end{itemize}
\end{tvrzeni}

\section{Křivkový integrál}
\begin{definice}[Značení]
	Nechť $\phi:[\alpha, \beta] \rightarrow ®C$. Potom $\phi$ je křivka, pokud je $\phi$ spojité, $\phi$ je regulární křivka, pokud je $\phi$ po částech spojitě diferencovatelné tzn. $\phi$ je spojitá na $[\alpha, \beta]$ a existuje dělení $\alpha = t_0 < t_1 < … < t_n = \beta$ takové, že $\phi|_{[t_i, t_n]}$ je diferencovatelné.

	Úsečka: Nechť $a, b \in ®C$, potom $\phi(t) = a + t·(b - a)$, $t \in [0, 1]$ je úsečka z $a$ do $b$. Značíme $[a, b]$.

	Řekneme, že křivka $\phi$ je lomená čára v ®C, existují-li $z_1, …, z_k \in ®C$ taková, že
	$$ \phi = [z_1, z_2] + [z_2, z_3] + … + [z_{k-1}, z_k]. $$
\end{definice}

\begin{poznamka}[Úmluva]
	Pokud neřekneme něco jiného, křivkou budeme rozumět regulární křivku v ®C.
\end{poznamka}

\begin{definice}[Délka křivky]
	$$ V(\phi) = \int_\alpha^\beta |\phi'(t)| dt, $$
	je-li $\phi$ regulární.
\end{definice}

\begin{definice}
	Nechť $\phi: [\alpha, \beta] \rightarrow ®C$ je regulární křivka a $f: <\phi> \rightarrow ®C$ je spojitá. Potom definujeme
	$$ \int_\phi f := \int_\alpha^\beta f(\phi(t))·\phi'(t) dt. $$
\end{definice}

\begin{poznamka}
	Křivkový integrál konverguje jako Riemannův.

	$$ \int_\phi f(z) dz, $$
	Nechť $z_0 \in ®C$, $r \in (0, +∞)$ a $\phi(t) = z_0 + r e^{it}$, $t \in [0, 2\pi]$. Potom
	$$ \int_\phi (z - z_0)^n dz = \int_0^{2\pi} r^n e^{i n t}·2·r·e^{it} dt = i·r^{n+1} \int_0^{2\pi} e^{i(n + 1) t} dt = $$
	$2\pi i$, pokud $n = -1$, 0, pokud $n \in ®Z$ a $n ≠ -1$.
\end{poznamka}

\begin{tvrzeni}[Vlastnosti křivkového integrálu]
	Je-li $\phi$ křivka, $f$ a $g$ jsou spojité funkce na $<\phi>$ a $A \in ®C$, ptotom
	$$ \int_\phi (Af + g) = A\int_\phi f + \int_\phi g. $$

	Je-li $\phi$ křivka a $f$ je spojitá funkce na $<\phi>$, potom
	$$ |\int_\phi f| ≤ \max_{<\phi>} |f|·V(\phi). $$

	Nechť $\phi: [\alpha, \beta] \rightarrow ®C$, $\psi: [\gamma, \delta] \rightarrow ®C$ a $\phi(\beta) = \psi(\gamma)$. Potom
	$$ \int_{\phi + \psi} f = \int_\phi f + \int_\psi f \land \int_{-\phi} f = - \int_\phi f, $$
	kde $(-\phi)(t) := \phi(-t)$, $t \in [-\beta, -\alpha]$ je opačná křivka k $\phi$.

	Křivkový integrál nezávisí na parametrizaci křivky: Nechť $\phi: [\alpha, \beta] \rightarrow ®C$ je křivka, $\omega: [\gamma, \delta] \rightarrow [\alpha, \beta]$ je spojitě diferencovatelné s $\omega' > 0$ a $\psi = \phi\circ\omega$. Potom $\int_\psi f = \int_\phi f$.

	\begin{dukazin}
		Jednoduchý, ukázán na přednášce pro některé body.
	\end{dukazin}
\end{tvrzeni}

\begin{definice}[Primitivní funkce]
	Řekneme, že funkce $f$ má na otevřené $G \subset ®C$ primitivní funkci $F$, pokud $F' = f$ na $G$.
\end{definice}

\begin{veta}[O výpočtu křivkového integrálu pomocí primitivní funkce]
	Nechť $G \subset ®C$ je otevřená a $f$ má na $G$ primitivní funkci $F$. Nechť $\phi:[\alpha, \beta] \rightarrow G$ je regulární křivka a $f$ je spojitá\footnote{Tohle je zbytečný předpoklad, ale to ještě neumíme dokázat.} na $<\phi>$. Potom
	$$ \int_\phi f = F(\phi(\beta)) - F(\phi(\alpha)), $$
	je-li navíc $\phi$ uzavřená, tzn. $\phi(\alpha) = \phi(\beta)$, pak
	$$ \int_\phi f = 0. $$
\end{veta}

\end{document}
