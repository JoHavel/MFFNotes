\documentclass[12pt]{article}					% Začátek dokumentu
\usepackage{../../MFFStyle}					    % Import stylu



\begin{document}

% 03. 10. 2022
\section*{Úvod}
\begin{poznamka}[Organizační úvod]
	Dnes česky, ale pravděpodobně časem přepneme do angličtiny.

	Na webu přednášejícího jsou zápisky, česko-anglická skripta.

	Taková bible pro lidi studující PDR je Evans (… PDE …).

	Zápočet bude za 2 velké domácí úkoly. Zkouška je písemná (požadavky jsou na stránkách): 3 části: A -- nutné, B -- teorie, C -- praxe?
\end{poznamka}

\begin{poznamka}[Konvence pro PDR]
	$\Omega \subseteq ®R^d$ je otevřená. Měřitelná = lebesgueovsky měřitelná.

	$$ \partial_t u := \frac{\partial u}{\partial t} $$
\end{poznamka}

\begin{poznamka}
	Dále se ukazovali konkrétní parciální rovnice.
\end{poznamka}

\begin{poznamka}[Je potřeba znát]
	\ 
	\begin{itemize}
		\item Prostory funkcí a Lebesgueův integrál: $L^p(\Omega)$, $L^p_{loc}(\Omega)$, $||u||_p$, $C^k(\Omega)$, $C^k(\overline{\Omega})$,
			$$ C^{0, \alpha}(\overline{\Omega}) = \{u \in C(\Omega) | \sup_{x ≠ y} \frac{u(x) - u(y)}{|x - y|^{\alpha}} < ∞\}, ||u||_{C^{0, \alpha}} = \sup_{x ≠ y} \frac{u(x) - u(y)}{|x - y|^{\alpha}}. $$
		\item $\int_\Omega \frac{\partial u}{\partial x_i} dx = \int_{\partial \Omega} u n_i dS$, $\vec{n} = (n_1, …, n_d)$.
		\item Funkcionální analýza 1: Banachův prostor, $u^n \rightarrow u$ silná konvergence, $u^n \rightharpoonup u$ slabá konvergence, Hilbertův prostor, Věta o reprezentaci (duálů), spektrální analýza operátorů, reflexivita (+ existence slabě konvergentní podposloupnosti v omezené podmnožině reflexivního prostoru).
		\item Separabilita ($L^p$ jsou separabilní až na $p = ∞$, $C^k(\overline{\Omega})$ je separabilní, $C^{0, \alpha}$ není separabilní pro $\alpha \in (0, 1]$).
	\end{itemize}
\end{poznamka}

\begin{poznamka}[Motivace k pojmu slabé řešení (weak solution)]
	$$ - \Delta = f, f \notin C(\overline{\Omega}) $$

	A další ukázané na přednášce.
\end{poznamka}

TODO?

% 04. 10. 2022

\section{Sobolevovy prostory}
\begin{definice}[Multiindex]
	$\alpha$ je multiindex $≡$ $d = (\alpha_1, …, \alpha_d)$, $\alpha_i \in ®N_0$. Délka $\alpha$ je $|\alpha| := \alpha_1 + … + \alpha_d$. Pro $u \in C^k(\Omega)$ definujeme $D^\alpha u = \frac{\partial^{|d|} u}{\partial x_1^{\alpha_1} … \partial x_d^{\alpha_d}}$.
\end{definice}

\begin{definice}[Slabá derivace]
	Buď $u, v_\alpha \in L^1_{loc}(\Omega)$. Řekneme, že $v_\alpha$ je $\alpha$-tá slabá derivace $u$ $≡$
	$$ ≡ \int_\Omega u D^\alpha \phi = (-1)^{|\alpha|} \int_\Omega v_\alpha \phi \qquad \forall \phi \in C_0^∞(\Omega). $$
\end{definice}

\begin{priklad}
	$u = \mathrm{sign} x$ nemá slabou derivaci.
\end{priklad}

\begin{lemma}[O smysluplnosti]
	Slabá derivace je nejvýše 1. Pokud existuje klasická derivace, tak obě splývají.

	\begin{dukazin}
		$v_\alpha^1$, $v_\alpha^2$ dvě $\alpha$-té derivace $u$.
		$$ (-1)^{|\alpha|} \int v_\alpha^1 \phi = \int_\Omega u D^\alpha \phi \forall \qquad \phi \in C_0^∞(\Omega) $$
		$$ (-1)^{|\alpha|} \int v_\alpha^2 \phi = \int_\Omega u D^\alpha \phi \forall \qquad \phi \in C_0^∞(\Omega) $$
		$$ \int_\Omega (v_\alpha^1 - v_\alpha^2) \phi = 0 \qquad \forall \phi \in C_0^∞(\Omega) $$
		$\implies v_\alpha^1 = v_\alpha^2$ skoro všude v $\Omega$.

		Klasická derivace je zřejmě zároveň slabá, tedy z první části splývají.
	\end{dukazin}
\end{lemma}

\begin{definice}[Sobolevův prostor]
	$\omega \subseteq ®R^d$ otevřená, $k \in ®N_0$, $p \in [1, ∞]$.
	$$ W^{k, p}(\Omega) := \{u \in L^p(\Omega) | \forall \alpha, |\alpha| ≤ k: D^\alpha u \in L^p(\Omega)\}. $$
	$$ ||u||_{W^{k, p}(\Omega)} ||u||_{k, p} := \begin{cases}\(\sum_{|\alpha| ≤ k} ||D^\alpha u ||_p^p\)^{\frac{1}{p}}, & p < ∞,\\ \max_{|\alpha| ≤ k} ||D^\alpha u||_∞, & p=∞.\end{cases} $$

	\begin{poznamkain}
		Od teď $D^\alpha$ nebo $\frac{\partial}{\partial x_1}$ nebo $\partial_i$ značí slabou derivaci.
	\end{poznamkain}
\end{definice}

\begin{lemma}[Základní vlastnosti slabých derivací a Sobolevových prostorů]
	Nechť $u, v \in W^{k, p}(\Omega), k \in ®N$, a $\alpha$ multiindex s délkou $≤ k$.
	 
	\begin{itemize}
		\item $D^\alpha u \in W^{k - |\alpha|, p}(\Omega)$ a $D^\alpha(D^\beta u) = D^\beta (D^\alpha u) = D^{\alpha + \beta} u$, pro $|\alpha| + |\beta| ≤ k$.
		\item $\lambda, \mu \in ®R$, $\lambda u + \mu v \in W^{k, p}(\Omega)$ a $D^\alpha(\lambda u + \mu v) = \lambda D^\alpha u + \mu D^\alpha v$.
		\item $\forall \tilde\Omega \subseteq \Omega$ otevřená
			$$ u \in W^{k, p}(\Omega) \implies u \in W^{k, p}(\tilde\Omega) $$
		\item $\forall \eta \in C^∞(\Omega)$: $\eta u \in W^{k, p}(\Omega)$ a $D^\alpha(\eta u) = \sum_{\beta_i ≤ \alpha_i} D^\beta \eta D^{\alpha - \beta} u \binom{\alpha}{\beta}$, kde $\binom{\alpha}{\beta} = \prod_{i=1}^d \binom{\alpha_i}{\beta_i}$.
	\end{itemize}
	
	\begin{dukazin}
		Cvičení na doma.
	\end{dukazin}
\end{lemma}

% 10. 10. 2022

\begin{veta}[Basic properties of Sobolev spaces]
	Let $\Omega \subseteq ®R^d$ be open set, $k \in ®N$ and $p \in [1, ∞]$. Then

	\begin{itemize}
		\item $W^{k, p}(\Omega)$ is a Banach space;
		\item if $p < ∞$ it is separable space;
		\item if $p \in (1, ∞)$ it is reflexive space.
	\end{itemize}

	\begin{dukazin}
		BS means linear normed space, which is complete. Linearity and norm? are easy. Completeness: $u^n$ is Cauchy in $L^p(\Omega)$ so $\exists u \in L^p: u^n \rightarrow u$ in $L^p$. $D^\alpha u^n$ is Cauchy in $L^p(\Omega)$ $\forall |\alpha| < k$ so $\exists v_\alpha \in L^p: D^\alpha u^n \rightarrow v_a \in L^p$. It remains prove that $D^\alpha u = v_\alpha$.
		%$$ \forall \phi \in C_0^∞: \int_\Omega v_\alpha \phi = \int_\Omega (v_\alpha - D^\alpha u_n) $$
		%$$ = \int_\Omega(v_\alpha - D^\alpha u) \phi + (-1)^{|\alpha|} \int_\omega(u^n - u) $$
		$$ TODO $$
		$$ |\int_\Omega (v_\alpha - D^\alpha u^n) \phi| ≤ ||v_\alpha - D^\alpha u^n||_p ||\phi||_{p'} ≤ C ||v_\alpha - D^\alpha u^n|| \rightarrow 0. $$
		$$ |\int_\Omega (u^n - u) D^\alpha \phi| ≤ ||u^n - u||_p ||D^\alpha \phi||_{p'} ≤ C ||u^n - u||_p \rightarrow 0. $$

		„2+3“: $W^{1, p}(\Omega) \simeq X \subseteq L^p(\Omega) \times … \times L^P(\Omega)$ ($d+1$ times), $X$ closed subspace from first property. Lemma: if $X \subseteq Y$ is closed subspace then $Y$ separable $\implies$ $X$ separable and $Y$ reflexive $\implies$ $X$ reflexive. (From functional analysis and topology.)
	\end{dukazin}
\end{veta}

\section{Approximation of Sobolev function}
\begin{veta}
	Let $\Omega \subseteq ®R^d$ open, ?. $p \in [1, ∞)$.
	$$ \overline{\{u \in C^∞(\Omega)\}}^{||·||_{k, p}} = W^{k, p}(\Omega). $$

	\begin{upozorneni}
		$$ \overline{\{u \in C^∞(\Omega)\}}^{||·||_{k, p}} \subsetneq W^{k, p}(\Omega). $$
	\end{upozorneni}

	\begin{dukazin}
		Summer semester.
	\end{dukazin}
\end{veta}

% 11. 10. 2022

\begin{veta}[Local density]
	$$ \forall u \in W^{k, p}(\Omega) \exists \{u^n\}_{n=1}^∞ $$
	$$ u^n \in C_0^∞(®R^d) \forall \tilde \Omega open, \overline{\tilde\Omega} \subseteq \Omega $$
	$$ u^n \rightarrow u in W^{k, p}(\tilde\Omega) $$

	\begin{dukazin}
		$u$ is extended by 0 to $®R^d \setminus \Omega$.
		$$ u^\epsilon = u * \eta^\epsilon \qquad \eta^\epsilon(x) = \frac{\eta(\frac{x}{\epsilon})}{\epsilon^d} \qquad \eta \in C_0^∞(B_1), \eta ≥ 0, \eta(x) = \eta(|x|), \int_{®R^d}\eta(x) dx = 1. $$
		$$ u \in L^P(SET) \qquad u^\epsilon \rightarrow u in L^p(SET). $$
		
		We need: $D^\alpha u^\epsilon \rightarrow D^\alpha u$ in $L^p(\tilde\Omega)$ $\forall \alpha, |\alpha| ≤ k$. Essential step: $D^\alpha u^\epsilon = (D^\alpha u)^\epsilon$ in $\tilde\Omega$ for $\epsilon ≤ \epsilon_0$ (so that ball of radius $\epsilon_0$ and center in $\tilde\Omega$ is in $\Omega$):
		$$ (D^\alpha u)^\epsilon(x) = \int_{®R^d} D^\alpha u(y) \eta_\epsilon(x-y) dy = \int_{B_\epsilon(x)} D^\alpha u(y) \eta_\epsilon(x - u) dy = $$
		$$ = (-1)^{|\alpha|} \int_{B_\epsilon(x)} u(y) D^\alpha_y \eta_\epsilon(x - y) dy = \int_{®R^d} u(y) D_x^\alpha \eta(x - y) dy. $$
		$$ D^\alpha u^\epsilon = D_x^\alpha \int_{®R^d} u(y) \eta_\epsilon(x - y) dy = \int_{®R^d} u(y) D_x^\alpha \eta_\epsilon(x - y) dy. $$
	\end{dukazin}
\end{veta}

\begin{tvrzeni}
	$\Omega$ is open connected set, $u \in W^{1, 1}(\Omega)$, then $u = \const. \Leftrightarrow \frac{\partial u}{\partial x_i} = 0\ \forall i \in [d]$.

	$W^{1, 1}(I) \hookrightarrow C(I)$ for $I$ interval.

	$W^{d, 1}(B_1) \hookrightarrow C(B_1)$.

	\begin{dukazin}
		„1. $\implies$“ trivial. „1. $\impliedby$“: $\tilde\Omega \subseteq \Omega$ connected $\epsilon_0$ as before and $\epsilon \in (0, \epsilon_0)$. $u^\epsilon$-modification of $u$ is smooth, so
		$$ \frac{\partial u^\epsilon}{\partial x_i} = \(\frac{\partial u}{\partial x_i}\)^\epsilon = 0 \quad in \tilde\Omega $$
		$$ \implies u^\epsilon = \const(\epsilon) \quad in \tilde\Omega. $$

		$$ c(\epsilon) = \int_{®R} c(\epsilon) \eta_\delta(x - y) dy = \int_{®R} u^\epsilon(y) \eta_\delta(x - y) dy = \int_{®R^d}\int_{®R^d} u(z) \eta_\epsilon(y - z) \eta_\delta(x - y) dz dy = $$
		$$ \int\!\int u(z + y) \eta_\epsilon(z) \eta_\delta(y - x) dz dy = \int\!\int u(z + x + y) \eta_\epsilon(z) \eta_\delta(u) dz dw = $$
		$$ \int\!\int u(z + x + y) \eta_\epsilon(z) \eta_\delta(u) dw dz = \int_{®R^d} u^\delta(z + x) \eta_\epsilon(z) dz = \int c(\delta) \eta_\epsilon(z) dz = c(\delta). $$

		„2.“: WLOG $I = (0, 1)$. Define $v(x) = \int_0^x \frac{\partial u}{\partial y}(y) dy$. We show: $v \in W^{1, 1}(I)$, $\frac{\partial v}{\partial x} = \frac{\partial u}{\partial x}$.

		$$ |v(x)| ≤ \int_0^1 |\frac{\partial u}{\partial x}| ≤ ||u||_{1, 1}. $$
		$$ \phi \in C_0^1(0, 1) \qquad \int_0^1 v(x) \frac{\partial \phi}{\partial x}(x) dx $$
		$$ = \int_0^1 \(\int_0^x \frac{\partial u}{\partial y}(y) dy\) \frac{\partial \phi}{\partial x}(x) dx = \int_0^1 \int_0^1 \frac{\partial u(y)}{\partial y} \frac{\partial \phi(x)}{\partial x} x_{0 < y < x} dy dx = \int_0^1 \int_0^1 \frac{\partial u(y)}{\partial y} \frac{\partial \phi(x)}{\partial x} x_{0 < y < x} dx dy = $$
		$$ = \int_0^1 \(\int_y^1 \frac{\partial \phi(x)}{\partial x} dx\) \frac{\partial u}{\partial y}(y) dy = - \int_0^1 \phi(y) \frac{\partial u}{\partial y}(y) dy \Leftrightarrow \frac{\partial v}{\partial x} = \frac{\partial u}{\partial x}. $$

		TODO.
		$$ x \rightarrow y \implies \int_y^x |\frac{\partial u}{\partial z}|^\alpha \rightarrow 0 \implies |u(x) - u(y)| \rightarrow 0 $$
		$$ ||u||_{C(I)} ≤ ||v + c||_{C(I)} ≤ ||u||_{1, 1} + |c| = ||u||_{1, 1} + |u(x) - v(x)| \forall x \in I $$
		$$ ||u||_{C(I)} ≤ ||u||_{1, 1} + \int_0^1 |u(x) - v(x)| dx ≤ -||- + \int_0^1 |u| + \int_0^1 |v| ≤ ||u||_{1,1}. $$

		„3.“ was shown without proof.
	\end{dukazin}
\end{tvrzeni}

\section{Characterization of Sobolev function}
\begin{veta}
	$\Omega \subseteq ®R^d$, $p \in [1, ∞]$, $\delta > 0$, $\Omega_\delta := \{x \in \Omega | \dist(x, \delta \Omega) > \delta\}$? Then
	$$ \forall u \in W^{1, p}(\Omega): ||\Delta_i^h u||_{L^p(\Omega_delta)} ≤ ||\frac{\partial u}{\partial x_i}||_{L^p(\Omega)} $$
	$$ \Delta_i^h u(x) = \frac{u(x + he_i) - u(x)}{h}. $$
	$$ u \in L^P \implies \forall \delta, h: ||\Delta_i^h u||_{L^p(\Omega_\delta)} ≤ c. $$
	$p > 1 \implies \frac{\partial u}{\ partial x_i}$ exists and $||\frac{\partial u}{\partial x_i}||_{L^p(\Omega)} ≤ c$.
\end{veta}

TODO!

\end{document}
