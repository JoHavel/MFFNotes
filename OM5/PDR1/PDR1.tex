\documentclass[12pt]{article}					% Začátek dokumentu
\usepackage{../../MFFStyle}					    % Import stylu



\begin{document}

% 03. 10. 2022
\section*{Úvod}
\begin{poznamka}[Organizační úvod]
	Dnes česky, ale pravděpodobně časem přepneme do angličtiny.

	Na webu přednášejícího jsou zápisky, česko-anglická skripta.

	Taková bible pro lidi studující PDR je Evans (… PDE …).

	Zápočet bude za 2 velké domácí úkoly. Zkouška je písemná (požadavky jsou na stránkách): 3 části: A -- nutné, B -- teorie, C -- praxe?
\end{poznamka}

\begin{poznamka}[Konvence pro PDR]
	$\Omega \subseteq ®R^d$ je otevřená. Měřitelná = lebesgueovsky měřitelná.

	$$ \partial_t u := \frac{\partial u}{\partial t} $$
\end{poznamka}

\begin{poznamka}
	Dále se ukazovali konkrétní parciální rovnice.
\end{poznamka}

\begin{poznamka}[Je potřeba znát]
	\ 
	\begin{itemize}
		\item Prostory funkcí a Lebesgueův integrál: $L^p(\Omega)$, $L^p_{loc}(\Omega)$, $||u||_p$, $C^k(\Omega)$, $C^k(\overline{\Omega})$,
			$$ C^{0, \alpha}(\overline{\Omega}) = \{u \in C(\Omega) | \sup_{x ≠ y} \frac{u(x) - u(y)}{|x - y|^{\alpha}} < ∞\}, ||u||_{C^{0, \alpha}} = \sup_{x ≠ y} \frac{u(x) - u(y)}{|x - y|^{\alpha}}. $$
		\item $\int_\Omega \frac{\partial u}{\partial x_i} dx = \int_{\partial \Omega} u n_i dS$, $\vec{n} = (n_1, …, n_d)$.
		\item Funkcionální analýza 1: Banachův prostor, $u^n \rightarrow u$ silná konvergence, $u^n \rightharpoonup u$ slabá konvergence, Hilbertův prostor, Věta o reprezentaci (duálů), spektrální analýza operátorů, reflexivita (+ existence slabě konvergentní podposloupnosti v omezené podmnožině reflexivního prostoru).
		\item Separabilita ($L^p$ jsou separabilní až na $p = ∞$, $C^k(\overline{\Omega})$ je separabilní, $C^{0, \alpha}$ není separabilní pro $\alpha \in (0, 1]$).
	\end{itemize}
\end{poznamka}

\begin{poznamka}[Motivace k pojmu slabé řešení (weak solution)]
	$$ - \Delta = f, f \notin C(\overline{\Omega}) $$

	A další ukázané na přednášce.
\end{poznamka}

TODO?

% 04. 10. 2022

\section{Sobolevovy prostory}
\begin{definice}[Multiindex]
	$\alpha$ je multiindex $≡$ $d = (\alpha_1, …, \alpha_d)$, $\alpha_i \in ®N_0$. Délka $\alpha$ je $|\alpha| := \alpha_1 + … + \alpha_d$. Pro $u \in C^k(\Omega)$ definujeme $D^\alpha u = \frac{\partial^{|d|} u}{\partial x_1^{\alpha_1} … \partial x_d^{\alpha_d}}$.
\end{definice}

\begin{definice}[Slabá derivace]
	Buď $u, v_\alpha \in L^1_{loc}(\Omega)$. Řekneme, že $v_\alpha$ je $\alpha$-tá slabá derivace $u$ $≡$
	$$ ≡ \int_\Omega u D^\alpha \phi = (-1)^{|\alpha|} \int_\Omega v_\alpha \phi \qquad \forall \phi \in C_0^∞(\Omega). $$
\end{definice}

\begin{priklad}
	$u = \mathrm{sign} x$ nemá slabou derivaci.
\end{priklad}

\begin{lemma}[O smysluplnosti]
	Slabá derivace je nejvýše 1. Pokud existuje klasická derivace, tak obě splývají.

	\begin{dukazin}
		$v_\alpha^1$, $v_\alpha^2$ dvě $\alpha$-té derivace $u$.
		$$ (-1)^{|\alpha|} \int v_\alpha^1 \phi = \int_\Omega u D^\alpha \phi \forall \qquad \phi \in C_0^∞(\Omega) $$
		$$ (-1)^{|\alpha|} \int v_\alpha^2 \phi = \int_\Omega u D^\alpha \phi \forall \qquad \phi \in C_0^∞(\Omega) $$
		$$ \int_\Omega (v_\alpha^1 - v_\alpha^2) \phi = 0 \qquad \forall \phi \in C_0^∞(\Omega) $$
		$\implies v_\alpha^1 = v_\alpha^2$ skoro všude v $\Omega$.

		Klasická derivace je zřejmě zároveň slabá, tedy z první části splývají.
	\end{dukazin}
\end{lemma}

\begin{definice}[Sobolevův prostor]
	$\omega \subseteq ®R^d$ otevřená, $k \in ®N_0$, $p \in [1, ∞]$.
	$$ W^{k, p}(\Omega) := \{u \in L^p(\Omega) | \forall \alpha, |\alpha| ≤ k: D^\alpha u \in L^p(\Omega)\}. $$
	$$ ||u||_{W^{k, p}(\Omega)} ||u||_{k, p} := \begin{cases}\(\sum_{|\alpha| ≤ k} ||D^\alpha u ||_p^p\)^{\frac{1}{p}}, & p < ∞,\\ \max_{|\alpha| ≤ k} ||D^\alpha u||_∞, & p=∞.\end{cases} $$

	\begin{poznamkain}
		Od teď $D^\alpha$ nebo $\frac{\partial}{\partial x_1}$ nebo $\partial_i$ značí slabou derivaci.
	\end{poznamkain}
\end{definice}

\begin{lemma}[Základní vlastnosti slabých derivací a Sobolevových prostorů]
	Nechť $u, v \in W^{k, p}(\Omega), k \in ®N$, a $\alpha$ multiindex s délkou $≤ k$.
	 
	\begin{itemize}
		\item $D^\alpha u \in W^{k - |\alpha|, p}(\Omega)$ a $D^\alpha(D^\beta u) = D^\beta (D^\alpha u) = D^{\alpha + \beta} u$, pro $|\alpha| + |\beta| ≤ k$.
		\item $\lambda, \mu \in ®R$, $\lambda u + \mu v \in W^{k, p}(\Omega)$ a $D^\alpha(\lambda u + \mu v) = \lambda D^\alpha u + \mu D^\alpha v$.
		\item $\forall \tilde\Omega \subseteq \Omega$ otevřená
			$$ u \in W^{k, p}(\Omega) \implies u \in W^{k, p}(\tilde\Omega) $$
		\item $\forall \eta \in C^∞(\Omega)$: $\eta u \in W^{k, p}(\Omega)$ a $D^\alpha(\eta u) = \sum_{\beta_i ≤ \alpha_i} D^\beta \eta D^{\alpha - \beta} u \binom{\alpha}{\beta}$, kde $\binom{\alpha}{\beta} = \prod_{i=1}^d \binom{\alpha_i}{\beta_i}$.
	\end{itemize}
	
	\begin{dukazin}
		Cvičení na doma.
	\end{dukazin}
\end{lemma}

% 10. 10. 2022

\begin{veta}[Basic properties of Sobolev spaces]
	Let $\Omega \subseteq ®R^d$ be open set, $k \in ®N$ and $p \in [1, ∞]$. Then

	\begin{itemize}
		\item $W^{k, p}(\Omega)$ is a Banach space;
		\item if $p < ∞$ it is separable space;
		\item if $p \in (1, ∞)$ it is reflexive space.
	\end{itemize}

	\begin{dukazin}
		BS means linear normed space, which is complete. Linearity and norm? are easy. Completeness: $u^n$ is Cauchy in $L^p(\Omega)$ so $\exists u \in L^p: u^n \rightarrow u$ in $L^p$. $D^\alpha u^n$ is Cauchy in $L^p(\Omega)$ $\forall |\alpha| < k$ so $\exists v_\alpha \in L^p: D^\alpha u^n \rightarrow v_a \in L^p$. It remains prove that $D^\alpha u = v_\alpha$.
		%$$ \forall \phi \in C_0^∞: \int_\Omega v_\alpha \phi = \int_\Omega (v_\alpha - D^\alpha u_n) $$
		%$$ = \int_\Omega(v_\alpha - D^\alpha u) \phi + (-1)^{|\alpha|} \int_\omega(u^n - u) $$
		$$ TODO $$
		$$ |\int_\Omega (v_\alpha - D^\alpha u^n) \phi| ≤ ||v_\alpha - D^\alpha u^n||_p ||\phi||_{p'} ≤ C ||v_\alpha - D^\alpha u^n|| \rightarrow 0. $$
		$$ |\int_\Omega (u^n - u) D^\alpha \phi| ≤ ||u^n - u||_p ||D^\alpha \phi||_{p'} ≤ C ||u^n - u||_p \rightarrow 0. $$

		„2+3“: $W^{1, p}(\Omega) \simeq X \subseteq L^p(\Omega) \times … \times L^P(\Omega)$ ($d+1$ times), $X$ closed subspace from first property. Lemma: if $X \subseteq Y$ is closed subspace then $Y$ separable $\implies$ $X$ separable and $Y$ reflexive $\implies$ $X$ reflexive. (From functional analysis and topology.)
	\end{dukazin}
\end{veta}

\section{Approximation of Sobolev function}
\begin{veta}
	Let $\Omega \subseteq ®R^d$ open, ?. $p \in [1, ∞)$.
	$$ \overline{\{u \in C^∞(\Omega)\}}^{||·||_{k, p}} = W^{k, p}(\Omega). $$

	\begin{upozorneni}
		$$ \overline{\{u \in C^∞(\Omega)\}}^{||·||_{k, p}} \subsetneq W^{k, p}(\Omega). $$
	\end{upozorneni}

	\begin{dukazin}
		Summer semester.
	\end{dukazin}
\end{veta}

% 11. 10. 2022

\begin{veta}[Local density]
	$$ \forall u \in W^{k, p}(\Omega) \exists \{u^n\}_{n=1}^∞ $$
	$$ u^n \in C_0^∞(®R^d) \forall \tilde \Omega open, \overline{\tilde\Omega} \subseteq \Omega $$
	$$ u^n \rightarrow u in W^{k, p}(\tilde\Omega) $$

	\begin{dukazin}
		$u$ is extended by 0 to $®R^d \setminus \Omega$.
		$$ u^\epsilon = u * \eta^\epsilon \qquad \eta^\epsilon(x) = \frac{\eta(\frac{x}{\epsilon})}{\epsilon^d} \qquad \eta \in C_0^∞(B_1), \eta ≥ 0, \eta(x) = \eta(|x|), \int_{®R^d}\eta(x) dx = 1. $$
		$$ u \in L^P(SET) \qquad u^\epsilon \rightarrow u in L^p(SET). $$
		
		We need: $D^\alpha u^\epsilon \rightarrow D^\alpha u$ in $L^p(\tilde\Omega)$ $\forall \alpha, |\alpha| ≤ k$. Essential step: $D^\alpha u^\epsilon = (D^\alpha u)^\epsilon$ in $\tilde\Omega$ for $\epsilon ≤ \epsilon_0$ (so that ball of radius $\epsilon_0$ and center in $\tilde\Omega$ is in $\Omega$):
		$$ (D^\alpha u)^\epsilon(x) = \int_{®R^d} D^\alpha u(y) \eta_\epsilon(x-y) dy = \int_{B_\epsilon(x)} D^\alpha u(y) \eta_\epsilon(x - u) dy = $$
		$$ = (-1)^{|\alpha|} \int_{B_\epsilon(x)} u(y) D^\alpha_y \eta_\epsilon(x - y) dy = \int_{®R^d} u(y) D_x^\alpha \eta(x - y) dy. $$
		$$ D^\alpha u^\epsilon = D_x^\alpha \int_{®R^d} u(y) \eta_\epsilon(x - y) dy = \int_{®R^d} u(y) D_x^\alpha \eta_\epsilon(x - y) dy. $$
	\end{dukazin}
\end{veta}

\begin{tvrzeni}
	$\Omega$ is open connected set, $u \in W^{1, 1}(\Omega)$, then $u = \const. \Leftrightarrow \frac{\partial u}{\partial x_i} = 0\ \forall i \in [d]$.

	$W^{1, 1}(I) \hookrightarrow C(I)$ for $I$ interval.

	$W^{d, 1}(B_1) \hookrightarrow C(B_1)$.

	\begin{dukazin}
		„1. $\implies$“ trivial. „1. $\impliedby$“: $\tilde\Omega \subseteq \Omega$ connected $\epsilon_0$ as before and $\epsilon \in (0, \epsilon_0)$. $u^\epsilon$-modification of $u$ is smooth, so
		$$ \frac{\partial u^\epsilon}{\partial x_i} = \(\frac{\partial u}{\partial x_i}\)^\epsilon = 0 \quad in \tilde\Omega $$
		$$ \implies u^\epsilon = \const(\epsilon) \quad in \tilde\Omega. $$

		$$ c(\epsilon) = \int_{®R} c(\epsilon) \eta_\delta(x - y) dy = \int_{®R} u^\epsilon(y) \eta_\delta(x - y) dy = \int_{®R^d}\int_{®R^d} u(z) \eta_\epsilon(y - z) \eta_\delta(x - y) dz dy = $$
		$$ \int\!\int u(z + y) \eta_\epsilon(z) \eta_\delta(y - x) dz dy = \int\!\int u(z + x + y) \eta_\epsilon(z) \eta_\delta(u) dz dw = $$
		$$ \int\!\int u(z + x + y) \eta_\epsilon(z) \eta_\delta(u) dw dz = \int_{®R^d} u^\delta(z + x) \eta_\epsilon(z) dz = \int c(\delta) \eta_\epsilon(z) dz = c(\delta). $$

		„2.“: WLOG $I = (0, 1)$. Define $v(x) = \int_0^x \frac{\partial u}{\partial y}(y) dy$. We show: $v \in W^{1, 1}(I)$, $\frac{\partial v}{\partial x} = \frac{\partial u}{\partial x}$.

		$$ |v(x)| ≤ \int_0^1 |\frac{\partial u}{\partial x}| ≤ ||u||_{1, 1}. $$
		$$ \phi \in C_0^1(0, 1) \qquad \int_0^1 v(x) \frac{\partial \phi}{\partial x}(x) dx $$
		$$ = \int_0^1 \(\int_0^x \frac{\partial u}{\partial y}(y) dy\) \frac{\partial \phi}{\partial x}(x) dx = \int_0^1 \int_0^1 \frac{\partial u(y)}{\partial y} \frac{\partial \phi(x)}{\partial x} x_{0 < y < x} dy dx = \int_0^1 \int_0^1 \frac{\partial u(y)}{\partial y} \frac{\partial \phi(x)}{\partial x} x_{0 < y < x} dx dy = $$
		$$ = \int_0^1 \(\int_y^1 \frac{\partial \phi(x)}{\partial x} dx\) \frac{\partial u}{\partial y}(y) dy = - \int_0^1 \phi(y) \frac{\partial u}{\partial y}(y) dy \Leftrightarrow \frac{\partial v}{\partial x} = \frac{\partial u}{\partial x}. $$

		TODO.
		$$ x \rightarrow y \implies \int_y^x |\frac{\partial u}{\partial z}|^\alpha \rightarrow 0 \implies |u(x) - u(y)| \rightarrow 0 $$
		$$ ||u||_{C(I)} ≤ ||v + c||_{C(I)} ≤ ||u||_{1, 1} + |c| = ||u||_{1, 1} + |u(x) - v(x)| \forall x \in I $$
		$$ ||u||_{C(I)} ≤ ||u||_{1, 1} + \int_0^1 |u(x) - v(x)| dx ≤ -||- + \int_0^1 |u| + \int_0^1 |v| ≤ ||u||_{1,1}. $$

		„3.“ was shown without proof.
	\end{dukazin}
\end{tvrzeni}

\section{Characterization of Sobolev function}
\begin{veta}
	$\Omega \subseteq ®R^d$, $p \in [1, ∞]$, $\delta > 0$, $\Omega_\delta := \{x \in \Omega | \dist(x, \delta \Omega) > \delta\}$. Then
	$$ \forall u \in W^{1, p}(\Omega): ||\Delta_i^h u||_{L^p(\Omega_delta)} ≤ ||\frac{\partial u}{\partial x_i}||_{L^p(\Omega)}, \qquad \forall h, i, \delta $$
	$$ \Delta_i^h u(x) = \frac{u(x + he_i) - u(x)}{h}. $$

	$$ u \in L^P \implies \forall \delta, h: ||\Delta_i^h u||_{L^p(\Omega_\delta)} ≤ c. $$
	$p > 1 \implies \frac{\partial u}{\ partial x_i}$ exists and $||\frac{\partial u}{\partial x_i}||_{L^p(\Omega)} ≤ c$.
\end{veta}

% 17. 10. 2022 Z poznámek vyučujícího
% 18. 10. 2022 Z poznámek vyučujícího
% 24. 10. 2022 Z poznámek vyučujícího
% 25. 10. 2022 Z poznámek vyučujícího

\begin{definice}[Class $C^{k, \mu}$]
	Let $\Omega \subseteq ®R^d$ open bounded set. We say that $\Omega \in C^{k, \mu}$ $(\partial\Omega \in C^{k, \mu})$ iff:
	\begin{itemize}
		\item there exist $M$ coordinate systems $¦x = \(x_{r_1}, …, x_{r_d}\) = \(x_r', x_{r_d}\)$ and functions $a_r: \Delta_r \rightarrow ®R$ where $\Delta_r = \{x_r' \in ®R^{d - 1} |\ |x_{r_i}| ≤ \alpha\}$ such that $a_r \in C^{k, \mu}(\Delta_r)$,
		\item denoting $\tr$ the orthogonal transformation from $(x_r', x_{r_d})$ to $(x', x_d)$, then $\forall x \in \partial \Omega$ $\exists r \in \{1, …, M\}$ such that $x = \tr\(x_{r_1}', a(x_{r_d})\)$,
		\item $\exists \beta > 0$, if we define
			$$ V_r^+ := \{(x_r', x_{r_d}) \in ®R^d | x_r' \in \Delta_r, a(x_r') < x_{r_d} < a(x_r') + \beta\} $$
			$$ V_r^- := \{(x_r', x_{r_d}) \in ®R^d | x_r' \in \Delta_r, a(x_r') - \beta < x_{r_d} < a(x_r')\} $$
			$$ \Lambda_r := \{(x_r', x_{r_d}) \in ®R^d | x_r' \in \Delta_r, a(x_r') = x_{r_d}\} $$
			Then $\tr (V_r^+) \subset \Omega, \tr(V_r^-) \subset ®R^d \setminus \overline{\Omega}, \tr(\Lambda_r) \subseteq \partial \Omega$ and $\bigcup_{r=1}^M \tr(\Lambda_r) = \partial \Omega$.
	\end{itemize}
\end{definice}

\begin{veta}[Density of smooth functions]
	Let $\Omega \in C^0$. Then $W^{k, p}(\Omega) = \overline{C^∞(\overline{\Omega})}^{||·||_{k, p}}$, $p \in [1, ∞)$.
\end{veta}

\begin{veta}[Extension of Sobolev functions]
	Let $\Omega \in C^{0, 1}$ ($\Omega$ is Lipschitz) and $k \in ®N$, $p \in [1, ∞]$. Then there exists a continuous linear operator $E: W^{k, p}(\Omega) \rightarrow W^{k, p}(®R^d)$ such that:
	\begin{itemize}
		\item $||E u||_{W^{k, p}(®R^d)} ≤ C ||E u||_{W^{k, p}(\Omega)}$ ($C$ is independent of $u$)
		\item $E u = u$ almost everywhere in $\Omega$.
	\end{itemize}
\end{veta}

\begin{veta}[Trace theorem]
	Let $\Omega \in C^{0, 1}$, $p \in [1, ∞]$. Then there exists a continuous linear operator $\tr: W^{1, p}(\Omega) \rightarrow L^p(\partial \Omega)$ such that:
	\begin{itemize}
		\item $||\tr u||_{L^p(\partial \Omega)} ≤ c ||u||_{1, p}$,
		\item $\forall u \in W^{1, p}(\Omega) \cap C(\overline{\Omega}): \tr u|_{\partial \Omega} = u |_{\partial \Omega}$.
	\end{itemize}
\end{veta}

\begin{definice}
	$$ W_0^{k, p}(\Omega) = \overline{C_0^∞(\Omega)}^{||·||_{k, p}}. $$
\end{definice}

\begin{veta}
	Let $\Omega \in C^{0, 1}$ and let $p \in [1, ∞]$. Then
	\begin{itemize}
		\item if $p < d$, then $W^{1, p}(\Omega) \hookrightarrow L^q(\Omega)$ for all $1 ≤ \frac{dp}{d - p}$,
		\item if $p = d$, then $W^{1, p}(\Omega) \hookrightarrow L^q(\Omega)$ for all $q < ∞$,
		\item if $p > d$, then $W^{1, p}(\Omega) \hookrightarrow C^{0, 1 - \frac{d}{p}}(\overline{\Omega})$.
	\end{itemize}
	Moreover
	\begin{itemize}
		\item if $p < d$, then $W^{1, p}(\Omega) \hookrightarrow\hookrightarrow L^q(\Omega)$ for all $1 ≤ \frac{dp}{d - p}$,
		\item if $p = d$, then $W^{1, p}(\Omega) \hookrightarrow\hookrightarrow L^q(\Omega)$ for all $q < ∞$,
		\item if $p > d$, then $W^{1, p}(\Omega) \hookrightarrow\hookrightarrow C^{0, \alpha}(\overline{\Omega})$ for all $\alpha < 1 - \frac{d}{p}$.
	\end{itemize}

	$$ X \hookrightarrow\hookrightarrow Y \Leftrightarrow X ≤ Y \land \(A \subseteq X \text{ is bounded in } X \implies A \text{ is precompact in } Y\). $$

	$$ X\hookrightarrow\hookrightarrow Y \implies X \subseteq Y \land \(\{u^n\}_{n=1}^∞, \exists c: ||u^n||_{1, p} ≤ c \implies \exists u^{n_j}: u^{n_j} \rightarrow u \text{ in } Y\). $$
\end{veta}

\begin{dusledek}[Trace theorem]
	Let $\Omega \in C^{0, 1}$. Then $\forall u \in W^{1, p}(\Omega)$ and $v \in W^{1, p'}(\Omega)$ we have integration by parts:
	$$ \int_\Omega u \frac{\partial v}{\partial x_i} dx = - \int_\omega v \frac{\partial u}{\partial x_i} dx + \int_{\partial \Omega} u v |_{u = \tr u, v = \tr v} n_i ds. $$
\end{dusledek}

\begin{veta}[Poincaré]
	Let $\Omega \in C^{0, 1}$ and $p \in [1, ∞]$. Let $\Omega_1, \Omega_2 \subseteq \Omega$, $|\Omega_i| > 0$ and $\Gamma_1, \Gamma_2 \subseteq \partial \Omega$, $|\Gamma_i|_{d - 1} > 0$. Let $\alpha_1, \alpha_2 ≥ 0$ and $\beta_1, \beta_2 ≥ 0$ and at least one of $\alpha_1, \alpha_2, \beta_1, \beta_2 > 0$.

	Then there exist $c_1, c_2 > 0$ such that $\forall u \in W^{1, p}(\Omega)$
	$$ c_1 ||u||_{1, p}^p ≤ ||\nabla u||_p^p + \alpha_1 \int_{\Omega_1} |u|^p + \alpha_2 |\int_{\Omega_2} u|^p + \beta_1 \int_{\Gamma_1} |u|^p + \beta_2 |\int_{\Gamma_2} u|^p ≤ c_2 ||u||_{1, p}^p. $$
	$$ (||u||_{1, p}^p = ||u||_p^p + ||\nabla u||_p^p.) $$

	\begin{dukazin}[Of the first (the only difficult) inequality]
		TODO!!!
	\end{dukazin}
\end{veta}

\section{Linear elliptic PDEs}
\begin{definice}[Elliptic]
	Let $a_{ij}, b, c_i, d_i \in L^∞(\Omega)$, where $\Omega ≤ ®R^d$ is bounded. We say that $L$ is elliptic if $\exists c_1 > 0$ such that $\forall \zeta \in ®R^d$ and almost all $x \in \Omega$
	$$ A \zeta · \zeta ≥ c_1 |\zeta|^2. $$
\end{definice}

\begin{lemma}
	If u is classical solution, then $\forall \phi \in C^1(\overline{\Omega}), \phi = 0$ on $\Gamma_1$: $B_{L, \delta}(u, \phi) = \int_\Omega f \phi + \int_{\Gamma_2 \cup \Gamma_3} g \phi$.

	\begin{dukazin}
		TODO!!!
	\end{dukazin}
\end{lemma}

\begin{lemma}
	If $u \in C^2(\overline{\Omega})$ and $A, b, ¦c, ¦d$ are smooth and previous lemma holds $\forall \phi \in C^1$, $\phi|_{\Gamma_1} = 0$ and $u = u_0$ on $\Gamma_1$, then $u$ is a classical solution.

	\begin{dukazin}
		TODO!!!
	\end{dukazin}
\end{lemma}

\begin{definice}[Weak solution]
	Let $\Omega \subseteq ®R^d$ Lipschitz, $L$ be an elliptic operator, $u_0 \in W^{1, 2}(\Omega)$, $f \in (W^{1, 2}(\Omega))^*$, $g \in L^2(\Gamma_2 \cup \Gamma_3)$. We say that $u \in W^{1, 2}(\Omega)$ is a weak solution iff
	\begin{itemize}
		\item $\tr u = \tr u_0$ on $\Gamma_1$ and
		\item $B_{L \sigma}(u, \phi) = \<f, \phi\> + \int_{\Gamma_2 \cup \Gamma_3} g \phi$, $\forall \phi \in V$, where $V := \{\phi \in W^{1, 2}(\Omega) | \tr \phi = 0 \text{ on } \Gamma_1\}$.
	\end{itemize}
	
\end{definice}

\subsection{Existence of solution for coercive operators}
\begin{definice}[Elliptic form]
	Let $B: V \times V \rightarrow ®R$ bilinear nad $V$ be a Hilbert space, $c_1, c_2 > 0$. We say that $B$ is elliptic if it is
	\begin{itemize}
		\item $V$-bounded $\Leftrightarrow$ $|B(u, \phi)| ≤ c_2 ||u||_V ||\phi||_V$ and
		\item $V$-coercive $\Leftrightarrow$ $B(u, u) ≥ c_1 ||u||_V^2$.
	\end{itemize}
\end{definice}

\begin{veta}[Lax-Milgram]
	Let $B$ be a bilinear elliptic form. Then
	$$ \forall F \in V^*\ \exists! u \in V\ \forall \phi \in V: B(u, \phi) = <F, \phi>. $$
\end{veta}

\begin{definice}
	Let $B: V \rightarrow V^*$. We say that $B$ is
	\begin{itemize}
		\item Lipschitz $≡$ $\forall u, v \in V: ||B(u) - B(v)||_{V^*} ≤ c_2 ||u - v||_V$, $c_2 > 0$;
		\item Uniformly monotone $≡$ $\forall u, v \in V: <B(u) - B(v), u - v>_V ≥ c_1 ||u - v||_V^2$, $c_1 > 0$.
	\end{itemize}
\end{definice}

\begin{veta}[Non-linear Lax-Milgram]
	Let $B$ be Lipschitz continuous and uniformly monotone. Then
	$$ \forall F \in V^*\ \exists! u \in V\ \forall \phi \in V: <B(u), \phi> = <F, \phi>. $$

	\begin{dukazin}
		TODO!!!
	\end{dukazin}
\end{veta}

\begin{dukaz}[Lax-Milgram]
	TODO!!!
\end{dukaz}

\begin{veta}
	If $B_{L, \sigma}$ is bilinear, $V$-bounded and $V$-elliptic. Then there exists a unique weak solution $u$.

	\begin{dukazin}
		TODO!!!
	\end{dukazin}
\end{veta}

\subsection{Existence via Fredholm alternative}

TODO!!!

% 31. 10. 2022

\begin{veta}
	Let $\Omega \in C^{0, 1}$, $L$ be an elliptic operator and $\Gamma_1 = \partial \Omega$. Then
	\begin{enumerate}
		\item $\Sigma$ is at most countable and if infinite $\{\lambda_k\}_{k=1}^∞ \implies \lambda_k \rightarrow ∞$;
		\item $(\lambda \notin \Sigma) \Leftrightarrow \forall f \in L^1\ \exists! u: Lu = f + \lambda u$;
		\item $\forall \lambda \notin \Sigma\ \exists C > 0\ \forall f \in L^2\ \exists! u \in W_0^{1,2}(\Omega): Lu = f + \lambda u$ and $||u||_{1,2} ≤ c ||f||_2$;
	\end{enumerate}

	\begin{dukazin}
		3) TODO
		improve convergence of $u^{n_k}$ and show
		$$ u^{n_k} \rightarrow u \text{ in } W_0^{1, 2}(\Omega) \text{ Strongly!}; $$
		show $\{u^{n_k}\}$ is Cauchy in $W_0^{1, 2}(\Omega)$
		$$ v^{n, m} = u^n - u^m $$
		$$ C_1||\nabla (u^n - u^m)||_2^2 ≤ \int_{\Omega} A \nabla v^{n, m} \nabla v^{n, m} = V_l(v^{n, m}, v^{n, m}) - $$
		$$ \int_\Omega ¦c \nabla  v^{n, m}v^{n, m} - b(v^{n, m})^2 + ¦d \nabla v^{n, m}v^{n, m} = $$
		$$ = \int_\Omega (f^n - f^m) v^{n, m} + \lambda(v^{n, m})^2 ± -||- ≤ $$
		$$ ≤ ||v^{n, m}||_2 (||f^n - f^m||_2 + \lambda ||v^{n, m}||_2 + ||¦c||_∞ ||\nabla v^{n, m}||_2 + ||¦d||_∞ ||\nabla v^{n, m}||_2 + ||b||_∞ ||v^{n, m}||_2) ≤$$
		$$ ≤ ||v^{n, m}|| C(\lambda) \overset{u^n \text{ is Cauchy}} ≤ C(\lambda) \epsilon $$
		$\implies \nabla u^n$ is Cauchy sequence $\implies u^n \rightarrow u$ in $W_0^{1, 2}(\Omega)$ $\implies$ $||?||_{n_k} = 1$

		$$ \int_\Omega A \nabla a u^n \nabla a \phi + b u^n \phi + ¦c \nabla u^n \phi - ¦d \nabla ? u^n = \int_\Omega f^n \phi + \lambda u^n \phi. $$
		$$ n \rightarrow ∞ $$
		$$ \int_\Omega A \nabla u \nabla \phi + b u \phi + ¦c \nabla u \phi - ¦d \nabla \phi u = \lambda \int u \phi \Leftrightarrow Lu = \lambda u $$
		But $\lambda \notin \Sigma$.
	\end{dukazin}
\end{veta}

\begin{poznamka}
	Next we discussed homework.
\end{poznamka}

% 01. 11. 2022

\subsection{Variational approach – minimization}
\begin{poznamka}
	$B_{L, \sigma}(u, v)$ must be symmetric! ($B_{L, \sigma}(u, v) = B_{L, \sigma}(v, u)$)

	$$ L = - \div (A \nabla u) + b u + ¦c \nabla u + \div (¦d u) $$
	$$ B_{L, \sigma}(u, v) := \int_{\Omega} A \nabla u·\nabla v + B u v + ¦c·\nabla u v - ¦d \nabla v u + \int_{\Gamma} \sigma u v $$
	$$ B_{L, \sigma}(v, u) := \int_{\Omega} A \nabla v·\nabla u + B v u + ¦c·\nabla v u - ¦d \nabla u v + \int_{\Gamma} \sigma v u $$
	$$ \implies A = A^T, \qquad ¦c = -¦d $$
\end{poznamka}

\begin{veta}
	Let $B_{L, \sigma}$ be linear symmetric $V$-elliptic and $V$-bounded. $f \in V^*$, $g \in L^2(\Gamma_2 \cup \Gamma_3)$, $u \in ?$. Then the following is equivalent:
	\begin{itemize}
		\item $u - u_0 \in V$ and $B_{L, \sigma}(u, v) = <f, \phi> + \int_{\Gamma_2 \cup \Gamma_3} g \phi$;
		\item $u - u_0 \in V$ $\forall v \in W^{1, 2}(\Omega)$, $v, u_0 \in V$
			$$ \frac{1}{2} B_{L, \sigma} (u, u) - <f, u> - \int_{\Gamma_2 \cup \Gamma_3} g u ≤ \frac{1}{2} B_{L, \sigma} (v, v) - <f, v> - \int_{\Gamma_2 \cup \Gamma_3} g v. $$
	\end{itemize}

	\begin{dukazin}[„$1 \implies 2$“]
		$$ 0 \overset{V-\text{elliptic}}≤ \frac{1}{2} B_{L, \sigma}(v - u, v - u) \overset{\text{linearity}}= \frac{1}{2} B_{L, \sigma}(v, v) + \frac{1}{2} B_{L, \sigma}(u, u)  - \frac{1}{2} B_{L, \sigma}(u, v) - \frac{1}{2} B_{L, \sigma}(v, u) = $$
		$$ = \frac{1}{2} \(B_{L, \sigma}(v, v) - B_{L, \sigma}(u, u)\) + B_{L, \sigma} (u, u) - B_{L, \sigma}(u, v) = $$
		$$ = \frac{1}{2} \(B_{L, \sigma}(v, v) - B_{L, \sigma}(u, u)\) + B_{L, \sigma}(u, u - v) \overset{\text{weak formulation}}= $$
		$$ = \frac{1}{2} \(B_{L, \sigma}(v, v) - B_{L, \sigma}(u, u)\) + <f, u - v> + \int_{\Gamma_2 \cup \Gamma_3} g(u - v) $$
	\end{dukazin}

	\begin{dukazin}[„$2 \implies 1$“]
		$u$ is minimizer, so set $v = u + \epsilon \phi$, $\phi \in V$
		$$ \frac{1}{2} B_{L, \sigma}(u, u) - <j, u> - \int g u ≤ \frac{1}{2} B_{L, \sigma}(u + \epsilon \phi, u + \epsilon \phi) - <j, u + \epsilon \phi> - \int g(u + \epsilon \phi) = $$
		$$ = \frac{1}{2} B_{L, \sigma}(u, u) + \frac{1}{2} \epsilon \frac{1}{2} B_{L, \sigma}(\phi, \phi) + \epsilon B_{L, \sigma}(u, \phi) - <f, u> - \epsilon <f, \phi> - \int ga - \epsilon \int g \phi $$
		divide by $\epsilon$ and $\epsilon \rightarrow 0_+$
		$$ 0 ≤ B_{L, \sigma}(u, \phi) - <j, \phi> - \int_{\Gamma_2 \cup \Gamma_3} g \phi, \qquad \forall \phi \in V $$
		(Euler-Lagrange inequality?), which is true also for $- \phi$ $\implies$ $0 = -||-$ $\implies u$ is weak solution.
	\end{dukazin}
\end{veta}

\begin{veta}[Duel formulation]
	Let $L u = -\Div(A \nabla u)$ with $A$ elliptic, bounded and symmetric, $\Gamma_1 ≠ \O$, $\Gamma = \O$, $f \in V^*$, $g \in L^2(\Gamma_2)$, $u_0 \in W^{1, 2}(\Omega)$. Then the $f$ following are equivalent:
	\begin{itemize}
		\item $u$ is a weak solution;
		\item $\nabla u = A^{-1} ¦T$, where $¦T$ minimizes $\int \frac{A^{-1}¦T·¦T}{2} = \nabla u_0 ¦T$ over the set $\tilde V := \{¦T \in L^2(\Omega, ®R^d)\}$, $\forall \phi \in V$.
			$$ \int_\Omega ¦T · \nabla \phi = <f, \phi> + \int_{\Gamma_2} g \phi \Leftrightarrow - \Div ¦T = f \text{ in } \Omega, T ¦u = g \text{ on } \Gamma_2 $$
	\end{itemize}

	\begin{dukazin}[„$1 \implies 2$“]
		Let $¦V \in \tilde V$ and $¦T := A \nabla  u \in \tilde V$.
		$$  0 ≤ \frac{1}{2} \int_{\Omega} A^{-1}(¦V - ¦T)·(¦V - ¦T) = \int \frac{A^{-1} ¦V · ¦V}{2} - \frac{A^{-1}¦T · ¦T}{2} \int_{\Omega} A^{-1} ¦T·¦T - A^{-1}¦T¦V = $$
		$$ = \int \(\frac{A^{-1}¦V·¦V}{2} - \nabla u_0 ¦V\) - \int \(\frac{A^{-1}¦T·¦T}{2} - \nabla u_0 ¦T\) + \int_{\Omega}\(\nabla u_0(¦V - ¦T) + A^{-1} ¦T(¦T - ¦V)\) = $$
		$$ = \int \(\frac{A^{-1}¦V·¦V}{2} - \nabla u_0 ¦V\) - \int \frac{A^{-1}¦T·¦T}{2} - \int_\Omega(A^{-1}¦T - \nabla u_0)·(¦V - ¦T) = $$
		$$ \int \(\frac{A^{-1}¦V·¦V}{2} - \nabla u_0 ¦V\) - \int \frac{A^{-1}¦T·¦T}{2} - \int_\Omega \nabla (u - u_0)·(¦V - ¦T) = $$
		$$ \int \(\frac{A^{-1}¦V·¦V}{2} - \nabla u_0 ¦V\) - \int \frac{A^{-1}¦T·¦T}{2} + 0. $$
		So $¦T$ is minimizer of the formula above.
	\end{dukazin}

	\begin{dukazin}[„$2 \implies 1$“]
		$¦T \in \tilde V$ $\forall V \in \tilde V$: $\int_\Omega \frac{1}{2} A^{-1} ¦T·¦T - \nabla u_0 ¦T ≤ \int_{\Omega}\frac{A^{-1}¦V · ¦V}{2} - \nabla u_0 ¦V$. $¦V = ¦T + \epsilon ¦W$, $¦W \in L^2(\Omega, ®R^d)$ $\forall \phi \in V$: $\int_\Omega ¦W · \nabla \phi = 0$.
		$$ \int_{\Omega} \frac{A^{-1} ¦T·¦T}{2} - \nabla u_0 ¦T ≤ \int_\Omega \frac{A^{-1} ¦T · ¦T + \epsilon^2 A^{-1}¦W · ¦W + 2 \epsilon A^{-1} ¦T · ¦W}{2} - \nabla u_0 ¦T - \epsilon \nabla u_0 ¦W $$
		divide by $\epsilon$ and $\epsilon \rightarrow 0_+$:
		$$ 0 ≤ \int_{\Omega} A^{-1} ¦T · ¦W - \nabla u_0 · ¦W. $$
		This also holds for $- ¦W$, co $0 = -||-$.

		Now we find unique $u \in W^{1, 2}$ $u - u_0 \in V$: $\int_\Omega \nabla u · \nabla \phi = \int_\Omega A^{-1} ¦T · \nabla \phi$ ($<F, \phi>_V$).
		$$ \int_\Omega |A^{-1}¦T - \nabla u|^2 = \int_\Omega (A^{-1} ¦T - \nabla u)(A^{-1}¦T - \nabla u) = $$
		$$ = \int_\Omega (A^{-1}¦T - \nabla u_0)·(A^{-1}¦T - \nabla u) + \int_\Omega \nabla (u_0 - u)(A^{-1}¦T - \nabla u) = 0 + 0 = 0 $$
	\end{dukazin}
\end{veta}

\begin{lemma}
	Let $X$ be a reflexive space and $\{u^n\}_{n=1}^∞$ be a bounded sequence, $||u^n||_X ≤ c < ∞$. Then $\exists u^{n_k}$, $\exists u \in x$: $u^{n_k} \rightharpoonup u$ ($\forall F \in X^*: <F, u^{n_k}> \rightarrow <F, u>$).
\end{lemma}

\end{document}
