\documentclass[12pt]{article}					% Začátek dokumentu
\usepackage{../../MFFStyle}					    % Import stylu



\begin{document}

% 30. 09. 2022

TODO!!!

% 07. 10. 2022

\begin{definice}[Lineární PDR]
	Parciální diferenciální rovnice (PDR) je lineární, jde-li ji zapsat ve tvaru
	$$ \sum_{|\alpha| ≤ m, \alpha \in (®N_0)^n} a_\alpha D^{\alpha} u = f $$
	pro neznámou funkci $u$, $f(x)$ a $a_\alpha(x)$ je dáno ($x \in \Omega \in ®R^n$).

	Je-li $f ≡ 0$, pak říkáme, že PDR je homogenní (bez pravé strany). Pokud $a_\alpha$ jsou konstanty, pak říkáme, že PDR je s konstantními koeficienty.
\end{definice}

\begin{definice}[Semilineární PDR]
	Semilineární rovnice má tvar
	$$ \sum_{|\alpha| = m} a_{\alpha} D^\alpha u + b = 0, $$
	kde $a(x)$ a $b(x, u, \nabla u, …, \nabla^{n-1} u)$ je dáno.
\end{definice}

\begin{definice}[Kvazilineární PDR]
	Kvazilineární rovnice je
	$$ \sum_{|\alpha| = m} a_{\alpha} D^\alpha u + f = 0, $$
	kde $a_\alpha(x, u, \nabla u, …, \nabla^{m-1} u)$ a $f(x, u, \nabla u, …, \nabla^{m-1} u)$ je dáno.
\end{definice}

\begin{definice}[Řád rovnice]
	$m$ v předchozích definicích nazýváme řád rovnice.
\end{definice}

\begin{definice}[Korektně zadaný problém]
	Problém je korektně zadaný podle Hadamarda, pokud má řešení, řešení je jednoznačné a řešení závisí spojitě na datech.
\end{definice}

\begin{definice}[Klasické řešení]
	Rovnice platí bodově, derivace jsou spojité.
\end{definice}

\begin{definice}[Okrajové podmínky]
	Dirichlet: zadaná hodnota na hranici.

	Neumann: zadány normálové tečny na hranici.
\end{definice}

\section{Cauchyova úloha pro kvazilineární rovnici 1. řádu}
\begin{definice}
	Buď $a_1, …, a_n$, $f \in ®C(®R \times ®R^n)$, $n \in ®N \setminus \{1\}$. Rovnici
	$$ \sum_{j=1}^n a_j(u(x), x) \partial_j u(x) = f(u(x), x), \qquad x \in ®R^n $$
	nazveme kvazilineární rovnici prvního řádu.

	Počáteční podmínku předepisujeme ve tvaru $u(0, \overline{x}) = u_0(\overline{x})$, kde $\overline{x} \in ®R^{n-1}$. Funkci $u : \Omega \rightarrow ®R$, $\omega \subseteq ®R^n$ nazveme klasickým řešením Cauchyovy úlohy pro kvazilineární rovnici 1. řádu, pokud $u \in ®C^1(\Omega)$ a podmínky platí bodově v $\Omega$.
\end{definice}

% 04. 11. 2022
\section{Klasifikace lineárních rovnic 2. řádu}
\begin{poznamka}[Lineární rovnice druhého řádu]
	$$ \sum_{i,j = 1}^n a_{ij}(x) \partial_i\partial_j u(x) + \sum_{i=1}^n b_i(x) \partial_i u(x) + c(x)u(x) = f(x), $$
	kde $a_{ij}, b_i, c, f$ jsou dané funkce, $i, j \in [n]$, $u$ neznámá funkce.

	Zafixujeme $x_0 \in ®R^n$, aby rovnice byla definována na nějakém $U(x_0)$. Chceme také rovnici transformovat tak, aby $A = (a_{ij})$ byla diagonální. Budeme pp. $A$ je symetrická (neboť pro $u \in C^2(…)$: $\partial_i\partial_j u = \partial_j\partial_i u$)
\end{poznamka}

\begin{definice}[Transformace diferenciální rovnice]
	Vezmeme nějaké $y_0$ a $U(y_0)$ a hladké? zobrazení $\phi(y_0) = x_0$ a $\phi(U(y_0)) \subset U(x_0)$.

	Definujeme funkci $v$: $u(x) = v(P·x)$, kde $P \in M^{n \times n}$ je regulární matice. $u(\overbrace{P^{-1}y}^{\phi(y)}) = v(y)$.

	Dosadíme do rovnice výše:
	$$ \partial_i u(x) = \sum_{k=1}^n \partial_k v (P x) P_{ki}, \qquad \partial_j\partial_i u(x) = \sum_{k=1}^m P_{ki} \sum_{l=1}^n P_{lj} \partial_k\partial v(P x), $$
	$$ \sum_{i,j,k,l=1}^n \partial_k\partial_l v(P x) P_{ki} a_{ij}(x) (P^T)_{jl} = \sum_{k, l = 1}^n \partial_k\partial_l v(P x) (PA(x)P^T)_{kl} $$

	LA: $A(x_0)$ je symetrická, tedy ze Sylvestrova zákona setrvačnosti existuje $P$ regulární taková, že $P A(x_0) P^T = \diag(d_1, …, d_n)$ pro $d_i \in \{-1, 0, 1\}$. Pozor, $P$ není určena jednoznačně, ale $d_1, …, d_n$ ano až na permutaci.

	Taktéž lze najít $P$ tak, aby $P^T = P^{-1}$ a $PA(x_0)P^{-1} = \diag(d_1, …, d_n)$ pro $d \in ®R$.
\end{definice}

\begin{priklady}
	Vlnová rovnice v 1D: $\partial_t^2 u - \partial_x^2 u = 0$.

	Laplaceova rovnice v 2D: $\partial_x^2 u + \partial_y^2 u = 0$.

	Rovnice vedení tepla: $\partial_t u - \partial_x^2 u = 0$.
\end{priklady}

\begin{definice}[Typy diferenciální rovnice 2. řádu]
	Řekneme, že lineární diferenciální rovnice je
	\begin{itemize}
		\item[eliptická] v $x_0$, pokud $\sgn A(x_0) = (n, 0, 0)$ nebo $(0, 0, n)$; (Laplace)
		\item[hyperbolická] v $x_0$, pokud $\sgn A(x_0) = (n-1, 0, 1)$ nebo $(1, 0, n-1)$; (vlnová)
		\item[parabolická] v $x_0$, pokud $\sgn A(x_0) = (n-1, 1, 0)$ nebo $(0, 1, n-1)$ a v případě $\sgn A(x_0) = (n-1, 1, 0)$ navíc požadujeme, aby koeficient $b_n$ (odpovídající $d_n=0$) po transformaci byl v bodě $x_0$ záporný, a v opačném případě kladný; (vedení tepla)
	\end{itemize}
\end{definice}

\begin{veta}
	Buď $S$ hyperbolická na okolí $x_0 \in ®R^2$, $a_{11}, a_{12}, a_{22} \in C^1(U(x_0))$, $a_{11} ≠ 0$ na $U(x_0)$. Pak lze 
	$$ a_{11} \partial_1^2 u + 2a_{12} \partial_1\partial_2 u + a_{22} \partial_2^2 u = 0 $$
	transformovat do tvaru $\partial_1\partial_2 v = f(\partial_1v, \partial_2v, v)$ na $V(x_0)$ pro vhodnou funkci $f$ a okolí $V$.

	\begin{dukazin}
		Dokázáno na cvičení.
	\end{dukazin}
\end{veta}

\section{Vlnová rovnice}
\begin{tvrzeni}[Obecné řešení vlnové rovnice v 1D]
	Řešení $\partial_t^2 u - c^2 \partial_x^2 u = 0$, kterou lze transformovat na $\partial_1\partial_2 v = 0$, dostaneme skrze $\partial_2 v(\rho \sigma) = \tilde{V_1(\sigma)}$, tedy $\int_0^∞ \tilde{V_1(\tau)} d\tau + V_2(\rho) = V_1(\sigma) + V_2(\rho) = v(\rho, \sigma)$.

	Obecným řešením je tedy
	$$ u(t, x) = V_1(x - ct) + V_2(x + ct), $$
	pro dost hladké funkce $V_1, V_2$.
\end{tvrzeni}

\begin{poznamka}[Úloha pro vlnovou rovnici (Cauchyova úloha)]
	Pro dané $f: (0, T) \times ®R \rightarrow ®R$. Hledáme řešení $u: [0, +∞) \times ®R \rightarrow ®R$ takové, že
	$$ \partial_t^2 u - \partial_x^2 u = f \text{ v } (0, T) \times ®R. $$
	A $u(0, x) = u_0(x)$, $\partial_t u(0, x) = u_1(x)$ ($\partial_t u$ musí jít spojitě rozšířit do $(0, x)$ a $\partial_t u(0, x)$ je hodnota tohoto rozšíření).
\end{poznamka}

\end{document}
