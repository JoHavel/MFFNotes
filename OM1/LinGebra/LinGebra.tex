\documentclass[12pt]{article}					% Začátek dokumentu
\usepackage{../../MFFStyle}					    % Import stylu



\begin{document}

\section*{Organizační úvod}
    Přednáška bude asi touto formou celý semestr.
    \begin{upozorneni}
        Bude důležitá zpětná vazba (přímo přes Zoom, kde bude snad nějaký cvičící, nebo přes diskuzi k předmětu v Moodle, či kvízy (od 5. 10.)).
    \end{upozorneni}
    \begin{poznamka}
        Číst mail a popis předmětu na webu.
    \end{poznamka}
    \begin{poznamka}[Organizace]
        2krát týdně přednáška na Zoomu. K zápočtu jsou třeba 2 věci:

        Průběžná práce (nejlépe přečíst skripta už před přednáškou, na dotazy bude přihlíženo) = kvízy (od 5. 10., 3 / 4 otázky a,b,c, za jeden kvíz až 2 body do bodovacího systému) + domácí úkoly (zadávány se zpožděním = další týden, odevzdává se ve středu, za jeden úkol až 8 bodů), počítá se 10 nejlepších kvíz + úkol (minimum pro zápočet je 60\%).

        Aktivní účast alespoň na 9 cvičeních. Je to záměrně trochu neurčité (optimálně připojit se na cvičení a počítat příklady a interagovat s cvičícím (v menších skupinkách)), dá se i bez toho (např. méně interaktivně) podle domluvy se cvičícími.

        Odevzdávání bude elektronické, většinou v Moodle. Pro skenování přes mobil používat aplikace jako Adobe scan / Genius scan. Nepoužívejme aplikace na automatické výpočty (1. je potřeba i postup, 2. děláme to pro sebe), ale můžeme je používat pro experimenty, ověření nebo představu.

        Sociální stránka, spolupracujme (viz cvičení, klidně si na cvičeních můžete skupinky domluvit), ale úkol zkusme vypracovat každý sám.

        Zkouška je standardně nějakých 5 termínů + jeden záchranný v září?, kdy bude 3hodinový test, požadavky ke zkoušce budou na webu. Midtermy (konec listopadu a půlka prosince) budou takové menší zkoušky, které se možná budou počítat ke zkoušce (minimálně si to můžeme vyzkoušet nanečisto).

    \end{poznamka}

    \begin{poznamka}[Materiály]
        \ 
        \begin{itemize}
            \item Skripta (Barto \& Tůma, na webu), kapitola 1-7
            \item Cvičení, přímočaré úlohy (nejen mechanicky řešit)
            \item Přednášky nejsou směrodatný studijní materiál!
        \end{itemize}
        \begin{upozorneni}
                Přednášky nejsou směrodatný studijní materiál! (Protože jejich cílem je pomoci nám látku pochopit, ne přečíst skripta. Navíc matematiku musí pochopit každý sám (doporučit přečíst text J. Hrnčíře\url{https://msekce.karlin.mff.cuni.cz/~smid/pmwiki/uploads/Main/vzkazHrncir.pdf}).)
        \end{upozorneni}
    \end{poznamka}

\section{Opakování (?)}
    \begin{itemize}
        \item Analytická geometrie
        \item Komplexní čísla
        \item Zobrazení (hlavně terminologie)
    \end{itemize}
    \begin{poznamka}[Co je lineární algebra]
        Abstraktní studium rovných útvarů a hezkých zobrazení mezi nimi.

        Rovný útvar: Přímka, rovina, 3D prostor $\R^3 \equiv \{(x, y, z): x, y, z \in \R\}$.

        Hezkých zobrazení: 
        \begin{align}
            \R^2 & \longrightarrow \R^2 \\
            (x, y) & \longrightarrow \text{otočení }(x, y)\text{ o }\frac{\pi}{6}
        \end{align}

        $$ \R^2 \longrightarrow \R $$

        Abstraktní: Časem dojdeme k tomu, že nás zajímají pouze vlastnosti.
    \end{poznamka}
    \begin{poznamka}[Proč lineární algebra]
        \ 
        \begin{itemize}
            \item Často se vyskytuje na různých místech matematiky
            \item Geometrie
            \item Matematická analýza (na limitním okolí bodu jsou funkce většinou také rovné)
            \item Kvantová teorie (a celkově fyzika)
            \item Trisekce úhlu
            \item Z historie -- nalezení planetky Ceres
            \item Moderní aplikace -- komprese dat (JPEG, …), kódování a šifrování (samoopravné kódy, RSA)
        \end{itemize}
        
    \end{poznamka}

    \subsection{Analytická geometrie}
        \begin{poznamka}[Názvosloví]
            \begin{itemize}
                \item Body vs. vektory: bod hranatou závorkou, vektor (jako rozdíl bodů) kulatou $\leftarrow$ je to skoro to samé, budeme uvažovat jen vektory (body jsou dané polohovými vektory)
                \item Operace s vektory: sčítání (sečteme po složkách), násobení skalárem (vynásobíme po složkách).
                \item Souřadnicové systémy (často nejsou kartézské)
            \end{itemize}
        \end{poznamka}

        \begin{poznamka}[Jak zadat přímku v $\R^2$]
            \begin{itemize}
                \item Bod a směrový vektor (parametricky)
                \item 2 různé body
                \item Rovnicí přímky: $L = \{(x, y)\in \R^2: ax + by = c\}\ \ a, b, c \in \R$ (implicitně)
            \end{itemize}
        \end{poznamka}

        \begin{poznamka}[Jak zadat rovinu v $\R^3$]
            \begin{itemize}
                \item Bod a 2 vektory s jiným směrem (parametricky)
                \item 3 různé body
                \item přímka a bod, který na ní neleží
                \item Rovnicí roviny: $P = {(x, y, z)\in \R^3: ax + by + cz = d}  a, b, c, d \in \R$  (implicitně)
            \end{itemize}
        \end{poznamka}

        \begin{poznamka}[Jak zadat rovinu v $\R^3$]
            \begin{itemize}
                \item Bod a směrový vektor
                \item 2 různé body
                \item 2 roviny (soustava 2 lineárních rovnic)
            \end{itemize}
        \end{poznamka}

    \subsection{Zobrazení}
        $\X, \Y$ množiny\\
        $f: \X \longrightarrow \Y$ $x \in \X \longrightarrow |f| \longrightarrow f(x) \in \Y$

        \begin{priklady}
            $$ \X = \{\text{Různé symboly}\}, \Y = \{A, B, …, Z\}$$
            \begin{itemize}
                \item Zadání pomocí definice: $f(x):= \text{První písmeno tvaru} x$
                \item Zadání pomocí tabulky: tabulka (jeden řádek symboly ($x$), druhý písmena ($f(x)$))
                \item Zadání pomocí grafu: Graf (jedna osa symboly, druhá písmena)
            \end{itemize}
        \end{priklady}

        \begin{definice}[Rovnost zobrazení]  
            $$ f: \X \longrightarrow \Y = g: \M \longrightarrow \Y \Leftrightarrow$$
            $$  $$ 
        \end{definice}

        \begin{definice}[Prosté (= injektivní)]
            $$ \forall x,y \in \X, x \neq y: f(x) \neq f(y) $$ 
        \end{definice}

        \begin{definice}[Na (= surjektivní)]
            $$ \forall y \in \Y \exists x \in \X: f(x) = y $$ 
        \end{definice}

        \begin{definice}[Vzájemně jednoznačné (= bijektivní)]
            Prosté a na.
        \end{definice}

        \begin{definice}[Skládání]
                $$ \X \stackrel{\longrightarrow}{f} \Y \stackrel{\longrightarrow}{g}: \X \stackrel{\longrightarrow}{g\circ f} \Z, g\circ f(x) = g(f(x)) $$ 
        \end{definice}

        Dále obraz zobrazení je množina NA kterou se zobrazuje, obraz množiny v zobrazení je množina, na kterou zobrazí zobrazení prvky dané množiny. Vzor obdobně (z které zobrazuje, ).

        \begin{definice}[Inverze]
            $f: \X \longrightarrow \Y$ má inverzi $g: \Y \longrightarrow \Y$, pokud $ f \circ g = id_Y$ (identita na Y)
        \end{definice}

        \begin{veta}
            $f$ má inverzi, právě když je bijektivní
            \begin{dukazin}
                Nechť má inverzi, když není na, tak v \Y přebývají prvky, které se nikam nezobrazují, když není prosté, tak chybí (TODO zlepšit popis).

                Nechť je bijektivní, pak zřejmě má inverzi.
            \end{dukazin}
        \end{veta}
        



    

\end{document}
