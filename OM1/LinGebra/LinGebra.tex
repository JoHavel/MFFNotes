\documentclass[12pt]{article}					% Začátek dokumentu
\usepackage{../../MFFStyle}					    % Import stylu



\begin{document}

\section*{Organizační úvod}
    Přednáška bude asi touto formou celý semestr.
    \begin{upozorneni}
        Bude důležitá zpětná vazba (přímo přes Zoom, kde bude snad nějaký cvičící, nebo přes diskuzi k předmětu v Moodle, či kvízy (od 5. 10.)).
    \end{upozorneni}
    \begin{poznamka}
        Číst mail a popis předmětu na webu.
    \end{poznamka}
    \begin{poznamka}[Organizace]
        2krát týdně přednáška na Zoomu. K zápočtu jsou třeba 2 věci:

        Průběžná práce (nejlépe přečíst skripta už před přednáškou, na dotazy bude přihlíženo) = kvízy (od 5. 10., 3 / 4 otázky a,b,c, za jeden kvíz až 2 body do bodovacího systému) + domácí úkoly (zadávány se zpožděním = další týden, odevzdává se ve středu, za jeden úkol až 8 bodů), počítá se 10 nejlepších kvíz + úkol (minimum pro zápočet je 60\%).

        Aktivní účast alespoň na 9 cvičeních. Je to záměrně trochu neurčité (optimálně připojit se na cvičení a počítat příklady a interagovat s cvičícím (v menších skupinkách)), dá se i bez toho (např. méně interaktivně) podle domluvy se cvičícími.

        Odevzdávání bude elektronické, většinou v Moodle. Pro skenování přes mobil používat aplikace jako Adobe scan / Genius scan. Nepoužívejme aplikace na automatické výpočty (1. je potřeba i postup, 2. děláme to pro sebe), ale můžeme je používat pro experimenty, ověření nebo představu.

        Sociální stránka, spolupracujme (viz cvičení, klidně si na cvičeních můžete skupinky domluvit), ale úkol zkusme vypracovat každý sám.

        Zkouška je standardně nějakých 5 termínů + jeden záchranný v září?, kdy bude 3hodinový test, požadavky ke zkoušce budou na webu. Midtermy (konec listopadu a půlka prosince) budou takové menší zkoušky, které se možná budou počítat ke zkoušce (minimálně si to můžeme vyzkoušet nanečisto).

    \end{poznamka}

    \begin{poznamka}[Materiály]
        \ 
        \begin{itemize}
            \item Skripta (Barto \& Tůma, na webu), kapitola 1-7
            \item Cvičení, přímočaré úlohy (nejen mechanicky řešit)
            \item Přednášky nejsou směrodatný studijní materiál!
        \end{itemize}
        \begin{upozorneni}
                Přednášky nejsou směrodatný studijní materiál! (Protože jejich cílem je pomoci nám látku pochopit, ne přečíst skripta. Navíc matematiku musí pochopit každý sám (doporučit přečíst text J. Hrnčíře\url{https://msekce.karlin.mff.cuni.cz/~smid/pmwiki/uploads/Main/vzkazHrncir.pdf}).)
        \end{upozorneni}
    \end{poznamka}

\section{Opakování (?)}
    \begin{itemize}
        \item Analytická geometrie
        \item Komplexní čísla
        \item Zobrazení (hlavně terminologie)
    \end{itemize}
    \begin{poznamka}[Co je lineární algebra]
        Abstraktní studium rovných útvarů a hezkých zobrazení mezi nimi.

        Rovný útvar: Přímka, rovina, 3D prostor $®R^3 \equiv \{(x, y, z): x, y, z \in ®R\}$.

        Hezkých zobrazení: 
        \begin{align}
            ®R^2 & \longrightarrow ®R^2 \\
            (x, y) & \longrightarrow \text{otočení }(x, y)\text{ o }\frac{\pi}{6}
        \end{align}

        $$ ®R^2 \longrightarrow ®R $$

        Abstraktní: Časem dojdeme k tomu, že nás zajímají pouze vlastnosti.
    \end{poznamka}
    \begin{poznamka}[Proč lineární algebra]
        \ 
        \begin{itemize}
            \item Často se vyskytuje na různých místech matematiky
            \item Geometrie
            \item Matematická analýza (na limitním okolí bodu jsou funkce většinou také rovné)
            \item Kvantová teorie (a celkově fyzika)
            \item Trisekce úhlu
            \item Z historie -- nalezení planetky Ceres
            \item Moderní aplikace -- komprese dat (JPEG, …), kódování a šifrování (samoopravné kódy, RSA)
        \end{itemize}
        
    \end{poznamka}

    \subsection{Analytická geometrie}
        \begin{poznamka}[Názvosloví]
            \begin{itemize}
                \item Body vs. vektory: bod hranatou závorkou, vektor (jako rozdíl bodů) kulatou $\leftarrow$ je to skoro to samé, budeme uvažovat jen vektory (body jsou dané polohovými vektory)
                \item Operace s vektory: sčítání (sečteme po složkách), násobení skalárem (vynásobíme po složkách).
                \item Souřadnicové systémy (často nejsou kartézské)
            \end{itemize}
        \end{poznamka}

        \begin{poznamka}[Jak zadat přímku v $®R^2$]
            \begin{itemize}
                \item Bod a směrový vektor (parametricky)
                \item 2 různé body
                \item Rovnicí přímky: $L = \{(x, y)\in ®R^2: ax + by = c\}\ \ a, b, c \in ®R$ (implicitně)
            \end{itemize}
        \end{poznamka}

        \begin{poznamka}[Jak zadat rovinu v $®R^3$]
            \begin{itemize}
                \item Bod a 2 vektory s jiným směrem (parametricky)
                \item 3 různé body
                \item přímka a bod, který na ní neleží
                \item Rovnicí roviny: $P = {(x, y, z)\in ®R^3: ax + by + cz = d}  a, b, c, d \in ®R$  (implicitně)
            \end{itemize}
        \end{poznamka}

        \begin{poznamka}[Jak zadat rovinu v $®R^3$]
            \begin{itemize}
                \item Bod a směrový vektor
                \item 2 různé body
                \item 2 roviny (soustava 2 lineárních rovnic)
            \end{itemize}
        \end{poznamka}

    \subsection{Zobrazení}
        $X, Y$ množiny\\
        $f: X \longrightarrow Y$ $x \in X \longrightarrow |f| \longrightarrow f(x) \in Y$

        \begin{priklady}
            $$ X = \{\text{Různé symboly}\}, Y = \{A, B, …, Z\}$$
            \begin{itemize}
                \item Zadání pomocí definice: $f(x):= \text{První písmeno tvaru} x$
                \item Zadání pomocí tabulky: tabulka (jeden řádek symboly ($x$), druhý písmena ($f(x)$))
                \item Zadání pomocí grafu: Graf (jedna osa symboly, druhá písmena)
            \end{itemize}
        \end{priklady}

        \begin{definice}[Rovnost zobrazení]  
            $$ f: X \longrightarrow Y = g: M \longrightarrow Y \Leftrightarrow$$
            $$  $$ 
        \end{definice}

        \begin{definice}[Prosté (= injektivní)]
            $$ \forall x,y \in X, x \neq y: f(x) \neq f(y) $$ 
        \end{definice}

        \begin{definice}[Na (= surjektivní)]
            $$ \forall y \in Y \exists x \in X: f(x) = y $$ 
        \end{definice}

        \begin{definice}[Vzájemně jednoznačné (= bijektivní)]
            Prosté a na.
        \end{definice}

        \begin{definice}[Skládání]
            $$ X \stackrel{\longrightarrow}{f} Y \stackrel{\longrightarrow}{g}: X \stackrel{\longrightarrow}{g\circ f} Z, g\circ f(x) = g(f(x)) $$ 
        \end{definice}

        Dále obraz zobrazení je množina NA kterou se zobrazuje, obraz množiny v zobrazení je množina, na kterou zobrazí zobrazení prvky dané množiny. Vzor obdobně (z které zobrazuje, ).

        \begin{definice}[Inverze]
            $f: X \longrightarrow Y$ má inverzi $g: Y \longrightarrow Y$, pokud $ f \circ g = id_Y$ (identita na $Y$)
        \end{definice}

        \begin{veta}
            $f$ má inverzi, právě když je bijektivní
            \begin{dukazin}
                Nechť má inverzi, když není na, tak v $Y$ přebývají prvky, které se nikam nezobrazují, když není prosté, tak chybí (TODO zlepšit popis).

                Nechť je bijektivní, pak zřejmě má inverzi.
            \end{dukazin}
        \end{veta}

% 6. 10. 2020

        
\section{Soustavy lineárních rovnic}
    6 motivačních příkladů v části 2.1 skript

    \begin{definice}[lineární rovnice]
        Lineární rovnice o $n$ neznámých s reálnými koeficienty je rovnice tvaru $a_1 x_1 + a_2 x_2 + … + a_n x_n = b$, kde $a_1, …, a_n, b \in ®R$ jsou koeficienty a $x_1, …, x_n$ jsou neznámé.
    \end{definice}

    \begin{upozorneni}[Jak vypadá definice]
        \ 
        \begin{itemize}
            \item Psát celé věty!
            \item Vysvětlit použité symboly!
            \item Nezahlcovat definice balastem!
        \end{itemize}
    \end{upozorneni}

    \begin{definice}[Soustava lineárních rovnic (SLR)]
        Soustavou $m$ lineárních rovnic o $n$ neznámých (s reálnými koeficienty) je soustava tvaru:
        \begin{align*}
                a_{11} x_1 + … + a_{1n} \cdot x_n &= b_1\\
                a_{21} x_2 + … + a_{2n} \cdot x_n &= b_2\\
                … … + … + …\cdot … &= …
                a_{m1} x_1 + a_{m2} x_2 + … + a_{mn} \cdot x_n &= b_m, 
        \end{align*}
        kde každý řádek je ve smyslu předchozí definice.
    \end{definice}

    \begin{definice}[Aritmetické vektory (nad ®R)]
        Aritmetickým vektorem rozumíme uspořádanou $n$-tici reálných čísel. Zapisujeme ho jako:
        $$
            \begin{pmatrix}
                x_1\\x_2\\\vdots\\x_n 
            \end{pmatrix} \in ®R^n
        $$
    \end{definice}

    \begin{definice}[Množina řešení SLR]
        Množina řešení SLR ve smyslu definice výše je množina:
        $$
            \{
                \begin{pmatrix}
                    x_1\\x_2\\\vdots\\x_n
                \end{pmatrix}
            \in ®R^n: a_i1 x_1 + … + a_{in} x_n = b_i \forall 1≤i≤m\}
        $$
    \end{definice}

    \begin{definice}[Operace s (aritm.) vektory ($n$ fixní)]
        $v, w \in ®R^n$ $\longrightarrow$ $v + w$ sčítání po složkách\\
        $t \in ®R$ $\longrightarrow$ $t \cdot w$ násobení po složkách\\
        $-v := (-1) \cdot v$\\
        $v - w := v + (-w)$\\
        $ 0 = \begin{pmatrix} 0\\ \vdots \\ 0 \end{pmatrix} \in ®R^n$
    \end{definice}

    \begin{definice}[Elementární úpravy SLR]
        Elementární úpravy dané SLR jsou:
        \begin{enumerate}
            \item Prohození pořadí 2 rovnic
            \item Vynásobíme jednu rovnici číslem $t \in ®R - \{0\}$
            \item Přičteme $t$-násobek jedné rovnice ($t \in ®R$) k jiné rovnici.
        \end{enumerate}
    \end{definice}

    \begin{tvrzeni}
        Elementární úpravy jsou ekvivalentními úpravami.
        \begin{dukazin}
            Skripta
        \end{dukazin}
        
        \begin{poznamka}
            1. lze nahradit kombinací druhých dvou.
        \end{poznamka}
    \end{tvrzeni}

    \begin{upozorneni}
        V přednášce jsou tvrzení (věty, lemmata, …). Typicky mají strukturu: Pokud platí …předpoklad…, pak platí …závěr….
    \end{upozorneni}

% 8. 10 .2020

    \begin{definice}[Gaussova eliminace]
        Algoritmus na úpravu matice do odstupňovaného tvaru za pomoci elementárních úprav.
    \end{definice}

    \begin{definice}[Pivot]
        První nenulový prvek v řádku v odstupňovaném tvaru.

        Pokud je pivot až ve „vektoru“?, SLR nemá řešení. 

        Počet pivotů je počet bázových proměnných, počet neznámých - počet pivotů je počet volných proměnných (neboli parametrů).
    \end{definice}

    \begin{definice}[Odstupňovaný tvar matice]
        Matice $m\times n$ je v odstupňovaném tvaru, jestliže
        $$ \exists r \in ®N, r<m\text{, že řádky }r+1,…,m\text{ jsou nulové a} $$
        $$ k_1 < k_2 < \cdots < k_r\text{, kde }k_i\text{ je index prvního nenulového prvku v }i\text{-tém řádku.} $$
    \end{definice}

    \begin{definice}[Hodnost matice]
        Buď $A = (a_{ij})_{m\times n}$ matice. Potom $\rank(A) = \text{r}(A)$ je tzv. hodnost matice, tedy počet nenulových řádků v odstupňovaném tvaru matice.
    \end{definice}

    \begin{upozorneni}
        Počet nenulových řádků v odstupňovaném tvaru nezáleží na volbě algoritmu, kterým jsme se do OT dostali.
    \end{upozorneni}

    \begin{poznamka}
        $\rank(A)≤m$ a $\rank(A)≤n$
    \end{poznamka}

    \begin{poznamka}[Geometrický význam]
        Řádky popisují jednotlivé rovnice, netriviální rovnice má za množinu řešení nadrovinu v $®R^n$, řádky popisují proměnné, významově jsou lineárními kombinacemi.
    \end{poznamka}

% 13. 10. 2020

\section{Tělesa}
    Otázka: V jakých číselných oborech můžeme řešit soustavy lineárních rovnic?

    \begin{poznamka}[Těleso budeme zavádět jako]
        Uvažujme množinu „čísel“ ®T s binárními operacemi +, $\cdot$. $+, \cdot: ®T \times ®T \rightarrow ®T$.

        Pak $(®T, +, \cdot)$ je těleso, pokud splňuje:…

    \end{poznamka}
    
    \begin{poznamka}[Co musí splňovat?]
        Uvažujme množinu ®T s jednou binární operací $\triangle: ®T\times ®T \rightarrow ®T$.

        Budeme požadovat tyto vlastnosti:\\
        (H1) $\forall a, b, c \in ®T: (a\triangle b)\triangle c = a\triangle (b\triangle c)$ (Asociativita)\\
        (H2) $\exists 0 \in ®T \forall a \in ®T: 0 \triangle a = a = a\triangle 0$ (Existence neutrálního prvku)\\
        (H3) $\forall a \in ®T \exists (-a) \in ®T: a\triangle (-a) = 0 = (-a) \triangle a$ (Existence opačného / inverzního prvku)
    \end{poznamka}

    \begin{tvrzeni}
        Ať $(®T, \triangle)$ splňuje axiomy (H1), (H2), (H3). Pak pro každé $a \in ®T$ je prvek $(-1)$ z (H3) určen jednoznačně.

        \begin{dukazin}
            Předpokládejme, že pro prvek $a \in ®T$ existují prvky $b, c \in ®T$ splňující:
            $$ a\triangle b = 0 = b\triangle a \land a\triangle c = 0 = c = c \triangle a $$
            Pak:
            $$ (b\triangle a)\triangle c \stackrel{\text{(H1)}}{=} b \triangle (a\triangle c) $$
            $$ c \stackrel{\text{(H2)}}{=} 0 \triangle c = b \triangle 0 \stackrel{\text{(H2)}}{=} b $$
            $$ b = c \lightning $$ 
        \end{dukazin}
    \end{tvrzeni}

    \begin{poznamka}[Ještě často doplňujeme]
        (H4) $\forall a, b \in ®T: a\triangle b = b\triangle a$
    \end{poznamka}

    \begin{definice}[Těleso]
        Tělesem (angl. field) rozumíme množinu $®T$ s binárními operacemi $+, \cdot : ®T\times ®T \rightarrow ®T$, které splňují tyto axiomy:\\
        (S1) $\forall a, b, c \in ®T: (a+b)+c = a+(b+c)$\\
        (S2) $\exists 0\in ®T \forall a \in ®T: a+0 = a$\\
        (S3) $\forall a \in ®T \exists (-a) \in ®T: a+(-1) = 0$\\
        (S4) $\forall a, b \in ®TL a+b = b+a$\\
        (N1) $\forall a, b, c \in ®T: (a\cdot b)\cdot c = a\cdot (b\cdot c)$\\
        (N2) $\exists 1 \in ®T: a\cdot 1 = a$\\
        (N3) $\forall a \in ®T, a≠0 \exists a^{-1}\in®T: a\cdot a^{-1} = 1$\\
        (N4) $\forall a,b \in ®T: a\cdot b = b\cdot a$\\
        (D) $\forall a,b,c \in ®T: (a+b)\cdot c = a\cdot c + b\cdot c$\\
        ($\neg$T) $|®T|>1$ (jinými slovy $0 ≠ 1$)

    \end{definice}
    
    \begin{priklady}
        $(®R, +, \cdot), (®Q, +, \cdot), (®C, +, \cdot), (\{0, 1\}, XOR, AND)$
    \end{priklady}

    \begin{tvrzeni}[Další vlastnosti těles]
        Je-li $a, b \in ®T, a≠0$, pak $a\cdot x+b = 0$ má v ®T právě jedno řešení $(x = \frac{-b}{a})$.

        $\forall a\in ®T: a\cdot 0 = 0$ ($a\cdot 0 + a\cdot 0 \stackrel{\text{(D)}}{=} a\cdot(0+0) \stackrel{\text(S2)}{=} a\cdot 0$, odečteme dvakrát pomocí (S2))

        $a\cdot b = 0 \implies a = 0 \lor b=0$

        $0 ≠ 1$ (Vezměme libovolné $a \in ®T$. Kdyby $0=1$, pak $0 = 0\cdot a = 1\cdot a = a \lightning$)
    \end{tvrzeni}

    \begin{priklady}
        Buď $n>2$ a $®Z_n = \{0,1,…,n-1\}$. Uvažujme násobení a dělení modulo. Potom $®Z_n$ splňuje všechny axiomy kromě (4N), které splňuje pokud $n$ je prvočíslo.
    \end{priklady}

% 15. 10. 2020

    \begin{definice}[Charakteristika tělesa]
        Nechť ®T je těleso, čísly $1, 2, 3, 4, …$ budeme značit součty $1, 1+1, 1+1+1, 1+1+1+1, …$. Charakteristikou tělesa budeme značit nejmenší číslo z $1, 2, 3, 4, …$, které je rovno 0. Pokud takové neexistuje je charakteristika rovna 0.

        \begin{prikladyin}
            Char. ®R, ®C, ®Q je 0. Char. $®Z_p$ je $p$.
        \end{prikladyin}
    \end{definice}

    \begin{tvrzeni}
        Je-li ®T těleso, pak charakteristika ®T je buď 0 nebo prvočíslo.
        \begin{dukazin}[Sporem]
                Předpokládejme, že char. ®T je $n≠0$ a $n$ je složené číslo, tj. $n = k\cdot l$, $1<k, l<n$. Všimněme si\footnote{$n$, $k$ a $l$ jedniček.} $(1+…+1) = (1+…+1) \cdot(1+…+1)\ \ (n=k\cdot l\text{ v }®T)$. Z předpokladu $n=0$ v ®T, tak $0=k\cdot l$ v ®T, $k,l ≠ 0$ a z vlastnosti těles $\lightning$.
        \end{dukazin}
    \end{tvrzeni}

    \subsection{Matice a operace s nimi}
        \begin{definice}[Matice nad tělesem]
            Matice typu $m \times n$ nad tělesem ®T je obdélníhové schéma prvků ®T, $m$ řádků, $n$ sloupců.
        \end{definice}

        \begin{poznamka}[Speciální případy]
            $n=1$ -- sloupcový aritmetický vektor.

            $m=1$ -- řádkový vektor.
        \end{poznamka}

        \begin{poznamka}[Operace]
            Podobně jako vektory lze matice sčítat a násobit prvkem z ®T.
        \end{poznamka}

        \begin{veta}[Vlastnosti operace]
            $$(A+B)+C = A+(B+C) \text{Má-li jedna strana smysl ($A, b, c$ stejného typu)}$$
            $$ A+B = B+A, 0+A=A $$ 
            $$ (s+t)\cdot A = s\cdot A + t\cdot A, (s\cdot t)\cdot A = s\cdot (t\cdot A), s\cdot (A+B) = s\cdot A + s\cdot B $$ 
        \end{veta}

        \begin{definice}[Čtvercová matice]
            Matice je čtvercová, pokud je typu $n\times n$, $n≥1$.

            \begin{prikladyin}
                Jednotková matice $©I = \begin{pmatrix}1 & 0 & …\\0 & 1 & …\\ \vdots & \vdots & \ddots\end{pmatrix}$.

                Trojúhelníkové matice (horní tj. $a_{ij} = 0 \forall i>j$ a dolní tj. $a_{ij} = 0 \forall j>i$).

                Diagonální (horní a dolní trojúhelníková zároveň) $a_{ij} = 0 \forall j≠i$.

                Permutační tj. v každém řádku a v každém sloupci je právě jedna 1, jinak 0.
            \end{prikladyin}
        \end{definice}

        \begin{definice}[Transponovaná matice]
            Je-li $A = (a_{ij})_{m\times n}$, pak $A^T = (b_{ij})_{n\times m}$, kde $b_{ij} := a_{ji}$, je matice transponovaná.
        \end{definice}

        \begin{definice}[Symetrická matice]
            Matice $A$ je symetrická, pokud je čtvercová a $A = A^T$.
        \end{definice}

        \begin{definice}[Násobení matic]
            Ať $A, B$ jsou matice nad tělesem ®T, $A=(a_{ij})_{m\times n}, B = (b_{jk})_{n\times p})$. Potom $AB = (\sum_{j=1}^n a_{ij} \cdot b_{jk})_{m\times p}$
        \end{definice}

        \begin{poznamka}[Proč násobení matic]
            Uvažujme SLR. Dá se zapsat jako $Ax = b$
        \end{poznamka}

        \begin{poznamka}[Další pohledy na součin matic]
            Sloupcový: sloupce výsledné matice jsou lineární kombinace sloupců první matice s koeficienty ze sloupců v druhé matici.

            Řádkový pohled: totéž transponovaně.

            Blokové násobení: můžeme násobit matice v maticích!
        \end{poznamka}

        \begin{poznamka}[Vlastnosti násobení matic]
            \begin{upozorneni}
                $$ AB ≠ BA $$
            \end{upozorneni}
            $$ (AB)C = A(BC) \text{Má-li jedna strana smysl.} $$
            $$ (A+B)C = AC + BC, C(E+F) = CE + CF $$ 
            $$ ©IA = A = A©I $$ 
            $$ s\in ®T, s(AB) = (aA)B = A(sB) $$ 
        \end{poznamka}

% 20. 10. 2020

    \subsection{Matice jako zobrazení}
        Budeme se zabývat speciálním případem maticového násobení, kdy druhá matice má šířku 1 (je to sloupcový vektor).

        \begin{definice}[Zobrazení určené maticí]
            Buď $A$ pevně zvolená matice typu $m\times n$ nad $T$. Zobrazením určeným maticí $A$ rozumíme $f_A : T^n \rightarrow T^m$, $x \rightarrow A · x$
        \end{definice}

        \begin{priklad}
            Jak popsat zobrazení rotace $r: ®R^2 \rightarrow ®R^2$.
            \begin{reseni}
                Nejdříve vyzkoušíme na speciálních vektorech, např. vektorech tzv. kanonické báze $®R^2: e_1 = (1 0)^T, e_2 = (0 1)^T$:
                $$ r_\alpha(e_1) = (\cos\alpha\ \sin\alpha)^T $$ 
                $$ r_\alpha(e_2) = (-\sin\alpha\ \cos\alpha)^T $$ 

                Následuje pozorování, že $r_\alpha$ je tzv. lineární zobrazení:
                $$ \forall v \in ®R^2 \forall t \in ®R: r_\alpha(t·v) = t·r_\alpha(v) $$
                $$ \forall u, v \in ®R^2: r_\alpha(u+v) = r_\alpha(u) + r_\alpha(v) $$ 

                Je-li $u = (u_1, u_2)^T \in ®R$ obecný vektor, pak $u = u_1·e_1 + u_2·e_2$. Tedy $r_\alpha(u) = u_1·r_\alpha(e_1) + u_2·r_\alpha(e_2) = u_1·(\cos\alpha\ \sin\alpha)^T + u_2·(-\sin\alpha\ \cos\alpha) =\begin{pmatrix} \cos\alpha & -\sin\alpha\\ \sin\alpha & \cos\alpha \end{pmatrix} (u_1 u_2)^T$
            \end{reseni}
        \end{priklad}

        \begin{tvrzeni}[Vlastnosti zobrazení určeného maticí]
            Buď $A$ matice $n\times m$ nad $T$, $f_A: T^n \rightarrow T^m$ jako výše, $x, y \in T^n, s\in T$. Pak platí
            $$ f_A(s·x) = s·f_A(x), $$ 
            $$ f_A(x+y) = f_A(x) + f_A(y) $$ 
            \begin{dukazin}
                $$f_A(s·x) \overset{\text{DEF}}{=} A·(s·x) = x·(A·x) \overset{\text{DEF}}{=} s·f_A(x) $$

                $$ f_A(x+y) \overset{\text{DEF}}{=} A·(x+y) = A·x + A·y \overset{\text{DEF}}{=} f_A(x) + f_A(y) $$ 
            \end{dukazin}
        \end{tvrzeni}

        \begin{definice}[Prvky kanonické báze $T^n$]
            Prvky kanonické báze $T^n$ jsou vektory $e_i = (0 … 0 1 0 … 0)^T \in T^n, i \in \{1, …, n\}$.
        \end{definice}

        \begin{dusledek}
            $$ f_A((x_1 … x_n)^T) = x_1·f_A(e_1) + … + x_n·f_A(e_n) = (f_A(e_1)|…|f_A(e_n))·(x_1 … x_n)^T $$

            Tedy pro $A, B$ matice $m \times n$ nad $T$ platí $(f_A = f_B) \Leftrightarrow (A = B)$.
        \end{dusledek}

        \begin{poznamka}[Další pozorování]
            Splňuje-li nějaké zobrazení $f: T^n \rightarrow T^m$ vlastnosti zobrazení daného maticí, pak je to zobrazení určené maticí.
        \end{poznamka}

        \begin{tvrzeni}[Skládání zobrazení určených maticí]
            Mějme zobrazení $f_A: T^n \rightarrow T^m$ a $f_B: T^p \rightarrow T^p$, dané maticemi $A$ a $B$. Potom zobrazení $f_A \circ f_B$ je dáno maticí $A\times B$.

            \begin{dukazin}
                Totiž pro libovolné $x \in T^p$ platí
                $$ f_A \circ f_B (x) = f_A(f_B(x)) = f_A(B·x) = A·(B·x) = (A·B)·x = f_{A·B}(x) $$ 
            \end{dukazin}
        \end{tvrzeni}

        \begin{priklad}
            Složení otočení o úhly $\alpha$ a $\beta$.
        \end{priklad}

% 22.10. 2020

    \subsection{Soustavy lineárních rovnic podruhé (oddíl 4.5.1, skripta)}
        Zvolme těleso $T$.

        \begin{poznamka}[Vztah dvou řešení SLR]
            $Ax = b$, $Ay = b$, potom
            $$ A(x-y) = Ax - ay = b - b = 0 $$
        \end{poznamka}

        \begin{definice}[Homogenní SLR]
            Soustava rovnic tvaru $A·x = o$ se nazývá homogenní.
        \end{definice}

        \begin{poznamka}
            SLR budeme řešit tak, že najdeme jedno řešení (nebo zjistíme, že žádné neexistuje) a poté budeme hledat řešení homogenní SLR, čímž můžeme podle poznámky výše už jednoduše získat všechna řešení původní rovnice.
        \end{poznamka}

        \begin{definice}[Jádro / nulový prostor]
            Je-li $A$ matice typu $m \times n$, pak
            $$ \{v \in T^n: A·v = 0\} \(=f^{-1}_A(0) = \{v:v \text{řeší} A·x = 0\}\) $$
            se nazývá jádrem (anglicky Kerner), nebo též nulovým prostorem, a značí se $\Ker A$.
        \end{definice}

        \begin{veta}
            Máme-li SLR $A·x = b$ a známe-li jedno řešení $v \in T^n$, pak množina všech řešení je tvaru $v + \Ker A := \{v + w: w \in \Ker A\}$.
            \begin{dukazin}
                Označme $S = \{u \in T^n: A·u = b\}$. Chceme ukázat $S = \{v+w: w \in \Ker A\}$.

                $S \subseteq v + \Ker A$: Vezměme libovolné $u \in S$, tj. $A·u = b$. Pak $A·(u-v) = 0$ (poznámka výše), $w := u-v \in \Ker A$, tj. $u = vv+w \in v + \Ker A$.

                $v + \Ker A \subseteq S$: Vezměme libovolné $u \in v + \Ker A$, tj. $u = v + w, w \in \Ker A$. Pak $A·u = A·(v+w) = A·v + A·w = b + 0 = b$, tj. $u \in S$.
            \end{dukazin}
        \end{veta}

        \begin{priklad}
            Ukazovalo se jedno řešení SLR pomocí jádra. Jádro má tu výhodu, že má bázi (vektory s 1 v jedné proměnné a nulami jinde).
        \end{priklad}

        \begin{definice}[Elementární matice]
            SKRIPTA oddíl 4.2.3
        \end{definice}

% 27. 10. 2020

    \subsection{Inverzní matice}

        \begin{poznamka}[Motivace]
            $$ A·x = b \rightarrow x = A^{-1}·b $$

            Inverzní zobrazení (tedy $f_A$ musí být prosté / na / bijekce).
        \end{poznamka}

        \begin{definice}[]
            Buď $A$ matice typu $m\times n$ nad $T$. Pak:
            \begin{itemize}
                \item $A$ je zprava invertovatelná, pokud $\exists X \in T^{n \times m}: A·X = I_m$.
                \item $A$ je zleva invertovatelná, pokud $\exists Y \in T^{n\times m}: Y·A = I_n$.
                \item $A$ je invertovatelná, pokud $n = m$ a existuje $X \in T^{n\times n}: A·X = I_n = X·A$.
            \end{itemize}
            
            \begin{poznamkain}
                V posledním případě značíme $X = A^{-1}$ a jednoduše se dokáže, že existuje nejvýše jedna $A^{-1}$.
            \end{poznamkain}
        \end{definice}

        \begin{tvrzeni}[(T4.67)]
            Buď $A$ matice typu $m\times n$ nad $T$. Potom následující je ekvivalentní:
            \begin{itemize}
                \item[a)] $A$ je zprava invertovatelná.
                \item[b)] $f_A: T^n \rightarrow T^m$ je na.
                \item[c)] Matice $C$, kterou dostaneme z $A$ Gaussovou eliminací, má všechny řádky nenulové.
                \item[d)] $\rank(A) = m$
            \end{itemize}

            \begin{dukazin}
                $1 \implies 2$: Předpoklad $\exists X: A·X = I_m$. Vezměme $b \in T^m$, pak $A(X·b) = (AX)b = I_m · b = b$

                $2 \implies 3$: Předpoklad $f_A$ je na ($Ax=b$ má řešení $\forall b$). Kdyby $C$ měla nulový $i$-tý řádek, pak $C·x = e_i$ nemá řešení $\implies Ax=e_i$ nemá řešení, $\lightning$.

                $3 \implies 1$: Předpokládáme (3). Hledáme $X$: $A·X = I_m$, čili $A·(x_1|…|x_m) = (e_1|…|e_m)$, čili řešíme SLR. SLR upravíme $(A|e_j) \sim … \sim (C|d_j) \implies C·x_j = d_j$ má podle (3) řešení.
            \end{dukazin}
        \end{tvrzeni}

        \begin{dusledek}
            Pokud je $A \in T^{m\times n}$ invertovatelná zprava, pak $n ≥ \rank{A} = m$.

            Pokud je navíc čtvercová, pak $(A|I_n) \sim … \sim (I_n|A^{-1})$, tedy $A$ je invertovatelná.
        \end{dusledek}

% 29. 10. 2020

        \begin{tvrzeni}[Čtvercové matice a inverze]
            Buď $A$ čtvercová matice řádu $n$ nad $T$. Pak následující je ekvivalentní: $A$ je invertovatelná zprava, $A$ je invertovatelná zleva, $A$ je invertovatelná.
            \begin{dukazin}
                $(1\implies 3)$: Viz důsledek výše.
                
                $(3 \implies 2)$: Triviální.

                $(2 \implies 1)$: Transpozice. ($1 \implies 3$). A transpozice znovu.
            \end{dukazin}
        \end{tvrzeni}

        \begin{poznamka}[Jak najdeme inveri zleva]
            Matici transponujeme a hledáme inverzi z prava. Nebo použijeme EŘÚ podobně jako u čtvercových.
        \end{poznamka}

        \begin{tvrzeni}[(4.70)]
                Buď $A$ matice typu $m\times n$ nad ¦T. pak NPJE (následující podmínky jsou ekvivalentní):
            \begin{itemize}
                \item[1] $A$ je invertovatelná zleva,
                \item[2] $f_A: ¦T^n \rightarrow ¦T^m$ je prosté,
                \item[3] $Ax = 0$ má právě jedno řešení $x=0$,
                \item[4] Matice $C$ vzniklá z $A$ Gaussovou eliminací má všechny sloupce bázové,
                \item[5] $\rank(A) = n$.
            \end{itemize}

            \begin{dukazin}
                ($1\implies 2$): Předpoklad: $\exists Y: Y·A = I_n$. Vezměme $x, y \in ¦T^n$ takové, že $f_A(x) = f_A(y)$. Vynásobíme $Y$ zleva: $x=y$.

                ($2 \implies 3$): (3) je speciální případ (2).

                ($3 \implies 4$): Předpokládejme, že neplatí (4). Pak dostáváme více řešení $Ax=Cx=0$ (pokud je $i$-tý sloupec nulový, pak za $i$-tou proměnnou lze dosadit cokoliv).

                ($4\implies 1 (5)$) $C$ má čtvercovou „podmatici“ řádu $n$, můžeme doupravovat tak, aby podmaticí $C$ byla $I_n$. Součin elementárních matic měnících $A$ na $I_n$ je inverze zleva (s několika řádky navíc, je to přesně postup na hledání inverze zleva).

                ($5\implies 4$) Necháno na rozmyšlení.
            \end{dukazin}
        \end{tvrzeni}

    \subsection{Regulární matice}
        \begin{definice}[Regulární matice]
            Matice $A$ nad ¦T je regulární, pokud $f_A: ¦T^n \rightarrow ¦T^m$ je bijekce.
        \end{definice}

        \begin{veta}[V4.81 (+ T4.88)]
            10 (+1) ekvivalentních podmínek pro Regulární matice, viz skripta.
        \end{veta}

    \subsection{Inverzní matice k horním a dolním trojúhelníkovým maticím}
        \begin{tvrzeni}[T4.98]
            Buď $A$ dolní trojúhelníková matice (tj. čtvercová) s nenulovými prvky na diagonále. Pak $A$ je regulární a $A^{-1}$ je dolní trojúhelníková. Měla-li na diagonále samé 1, pak $A^{-1}$, pak $A^{-1}$ je má také. Podobně pro horní trojúhelníkové rovnice.
        \end{tvrzeni}

% 3. 11. 2020

        \begin{definice}[Vektorový prostor]
            Buď ®T těleso. Vektorový prostor nad tělesem ®T je množina ¦V (množina vektorů) s operací sčítání na ¦V ($¦V\times ¦V \rightarrow ¦V$) a operací násobení skalárem ($®T\times ¦V \rightarrow ¦V$), které splňují tyto axiomy:
            \begin{itemize}
                \item[vS1] $\forall ¦u, ¦v, ¦w \in ¦V: (¦u + ¦v) + ¦w = ¦u + (¦v + ¦w)$,
                \item[vS2] $\exists ¦o \in ¦V \forall ¦v \in ¦V: ¦o + ¦v = v$,
                \item[vS3] $\forall ¦v \in ¦V \exists (-¦v) \in ¦V: ¦v+(-¦v) = ¦o$,
                \item[vS4] $\forall ¦u, ¦v \in ¦V: ¦u + ¦v = ¦v + ¦u$,
                \item[vN1] $\forall a, b \in ®T \forall ¦v \in ¦V: (a·b)·¦c = a·(b·¦v)$,
                \item[vN2] $\forall ¦v \in ¦V: 1·¦v = ¦v$,
                \item[vD1] $\forall a, b \in ®T \forall ¦v \in ¦V: (a+b)·¦v = a·¦v + b·¦v$,
                \item[vD2] $\forall a \in ®T \forall ¦u, ¦v \in ¦V: a·(¦u + ¦v) = a·¦u + a·¦v$.
            \end{itemize}
        \end{definice}

        \begin{priklady}
            ®T těleso, $n≥0$, $¦V = ®T^n$ s běžnými operacemi na aritmetických vektorech tvoří vektorový prostor.

            ®T těleso, $¦V = \{(x, y, z)^T \in ®T^3: x+y+z = 0\} \subseteq ®T^3$. Operace z $®T^3$ zúžené na ¦V.
        \end{priklady}

        \begin{tvrzeni}[Vlastnosti vektorových prostorů]
            \ 
            \begin{itemize}
                \item Nulový vektor je určen jednoznačně.
                \item Opačný vektor je určen jednoznačně.
                \item Vektorový prostor nikdy není prázdný, ale existuje tzv. nulový vektorový prostor.
                \item Pro $¦v \in ¦V: 0·¦v = ¦o$.
            \end{itemize}
        \end{tvrzeni}

        \begin{definice}[Vektorový podprostor]
            Buď ®T těleso a ¦V vektorový prostor nad ®T. Podprostorem rozumíme množinu $®U \subseteq ®V$ takovou, že operace lze zúžit na ¦U a ¦U s těmito operacemi tvoří samo V. P. nad ®T.
        \end{definice}

        \begin{tvrzeni}[T5.12]
            Vezměme ®T a ¦V jako výše. Pak $¦U \subseteq ¦V$ tvoří podprostor právě tehdy, když je uzavřený na sčítání a násobení skalárem.
        \end{tvrzeni}

        \begin{poznamka}
            Prostor ¦U je podprostorem ¦V značíme $¦U ≤ ¦V$.

            Nulový vektorový prostor a prostor sám sobě je podprostorem (tzv. triviální nebo nevlastní podprostory).
        \end{poznamka}

% 5. 11. 2020

        \begin{priklady}
            Prostor všech polynomů stupně $n$ je podprostor všech polynomů stupně $n+1$ je podprostor všech polynomů.

            Prostor všech spojitých funkcí (spolu s bodovým sčítáním a násobením konstantou, $®R^2 \rightarrow ®R$) je podprostorem všech zobrazení nad ®R. Prostor polynomiálních funkcí je podprostorem spojitých funkcí…

            $®R ≤ ®C$ se sčítáním na ®C a násobením zúženým na $®R\times ®C \rightarrow ®C$.
        \end{priklady}

        \begin{definice}
            Buď ¦V vektorový prostor nad ®T. Vezměme $¦v_1, …, ¦v_k \in ¦V$, $t_1, …, t_k \in ®T$. Pak vektor
            $$ t_1·¦v + … + t_k·¦v_k $$ 
            se nazývá lineární kombinace vektorů $¦v_1, …, ¦v_k$ (s koeficienty $t_1, …, t_k$).
        \end{definice}

        \begin{definice}
            Buď ¦V V.P. nad ®T, vezměme $X \subseteq ¦V$. Pak
            $$ \LO X = \{t_1·¦v_1 + … + t_k · ¦v_k : k \in ®N_0, ¦v_1, …, ¦v_k \in X, t_1, …, t_k \in T \} \subseteq ¦V $$
            se nazývá lineární obal $X$ (v prostoru ¦V).
        \end{definice}

        \begin{tvrzeni}
            Buď $x \subseteq ¦V$. Pak $\LO X ≤ ¦V$.
        \end{tvrzeni}

        \begin{definice}
            Buď ¦V vektorový prostor nad ®T a $X \in ¦V$. pak řekneme, že $X$ generuje ¦V ($X$ je množina generátorů ¦V), pokud $\LO X = ¦V$
        \end{definice}

% 10. 11. 2020

        \begin{upozorneni}[MIDTERMY]
            Budou! První bude 24. 11. 2020. Druhý 15. 12. 2020.
        \end{upozorneni}

        \begin{definice}[S maticí $A$ jsou spojeny 4 vektorové prostory]
            Jádro: $\ker A$.

            Jádro transponované: $\ker A^T$.

            Obraz (= LO sloupců = sloupcový prostor matice): $\Im A$.

            Obraz transponované (= LO řádků = řádkový prostor matice): $\Im A^T$.
        \end{definice}

        \begin{poznamka}
            Sloupcové úpravy nemění obraz, řádkové nemění jádro (u transponovaných opačně).
        \end{poznamka}

        \begin{tvrzeni}[Prostory spojené s maticí a elementární úpravy]
            $A$ typu $m \times n$ nad ¦T. Posloupnost EŘÚ $A \sim … \sim RA$. Posloupnost sloupcových úprav $A \sim … \sim AQ$.

            $$ \ker A = \ker RA $$
            \begin{dukazin}
                $f_R$ je lineární a bijekce, tedy zobrazuje nulu na nulu. Tedy $Ax=0 \implies f_R(Ax) = 0$. Opačně $RAx = 0 \implies f_R^{-1}(RAx) = Ax = 0$.
            \end{dukazin}

            $$ \Im AQ = \Im A $$ 
            \begin{dukazin}
                $f_Q$ je bijekce, tedy $\exists ¦x : A¦x = ¦y \implies AQ(Q^-1¦x) = ¦y \implies ¦y \in \Im(AQ)$ a $\exists ¦x: AQ¦x = ¦y \implies A(Q¦x) = ¦y \implies ¦y \in \Im(A)$
            \end{dukazin}
        \end{tvrzeni}

        $$ \(A = \(¦a_1|…|¦a_n\)\) $$ 
        $$ \Im RA = \LO {R¦a_1, …, R¦a_n} $$
        \begin{dukazin}
                Přímo z maticového násobení. $RA¦x = R¦a_1x_1+…+R¦a_nx_n$
        \end{dukazin}

        \begin{definice}[Lineární (ne)závislost]
            Buď ¦V vektorový prostor nad ®T a $(¦v_1, …, ¦v_k)$ posloupnost vektorů z ¦V. Řekněme, že posloupnost vektorů je lineárně nezávislá, pokud žádný vektor $¦v_i$ z posloupnosti nelze vyjádřit jako lineární kombinaci zbývajících. Tj. neplatí:
            $$ ¦v_i = t_1 ¦v_1 + … +t_{i-1}¦v_{i-1} +t_{i+1}¦v_{i+1} +… + t_k¦v_k $$
            pro žádné $1≤i≤k$ a žádná $t_i \in T$. Ekvivalentně lze psát
            $$ \forall 1≤i≤k: v_i \notin \LO\{¦v_1 + … + ¦v_{i-1} + ¦v_{i+1} + … + ¦v_k\} $$ 

            V opačném případě je posloupnost lineárně závislá.
        \end{definice}

        \begin{poznamka}
            Lineární (ne)závislost se nezmění permutací posloupnosti.
        \end{poznamka}

        \begin{definice}[Zkratky]
            LN = lineárně nezávislá, LZ = lineárně závislá.
        \end{definice}

        \begin{priklady}
            $k=0$ je posloupnost LN. $k = 1$ je posloupnost vždy LN, pokud není nulovým vektorem. $k=2$ je posloupnost LZ právě tehdy, když je jeden vektor násobkem druhého.

            Podposloupnost LN posloupnosti je LN.

            Posloupnost je vždy LZ, pokud je nějaký vektor nulový, nebo pokud se dva vektory rovnají, ale LZ je v mnoha dalších případech.
        \end{priklady}

        \begin{tvrzeni}
            ¦V vektorový prostor nad ®T, $()$ posloupnost vektorů z ¦V. Pak NPJE:
            \begin{itemize}
                \item $()$ je LN.
                \item $\forall 1 ≤ i ≤ k: ¦v_i \notin \LO \{¦v_1, …, ¦v_{i-1}\}$
                \item $¦v_1·t_1 + … +¦v_k·t_k = 0$ pouze triviálně (pro všechna $t_i$ nulová).
                \item Máme-li $¦b \in ¦V$, pak existuje nejvýše 1 $k$-tice $(t_1, …, t_k)$ prvků ®T taková, že
                        $$ ¦b = t_1·¦v_1 + … + t_k · ¦v_k. $$ 
            \end{itemize}

% 12. 11. 2020

            \begin{dukazin}
                $(1) \implies (2):$ Triviální.

                $(2) \implies (3):$ Předpokládejme negaci (3), tj. máme vyjádření $¦o = t_1 · ¦v_1 + … + t_k · ¦v_k$ takové, že $t_i ≠ 0$ pro nějaké $1≤i≤k$. Zvolíme největší takové $i$, tj. $¦o = t_1·¦v_1 + … + t_i · ¦v_i \implies v_i = (-t_i^{-1}t_1)·¦v_1 + … + (-t_i^{-1}t_{i-1})·¦v_{i-1} \implies$ negace (2).

                $(3) \implies (4):$ Předpokládejme negaci (4) 2 vyjádření $b$ následně od sebe odečteme a dostaneme negaci (3).

                $(4) \implies (3):$ Speciální případ.

                $(3) \implies (1):$ Předpokládejme $\neg (1)$, tj. posloupnost je LZ, tj. existuje $1≤i≤k$, tak že $¦v_i$ lze vyjádřit za pomoci lineární kombinace ostatních vektorů. Odečteme $¦v_i$ a hned dostáváme netriviální LK rovnou 0, jelikož $t_i = -1$.
            \end{dukazin}
        \end{tvrzeni}

        \begin{tvrzeni}
            Vezměme matici $A = (a_1|…|a_n)$ typu $m \times n$ nad $T$. Buď $R$ regulární matice řádu $m$ a $Q$ regulární matice řádu $n$. Pak NTJE: a) sloupce $A$ jsou LN, b) sloupce $R·A$ jsou LN, c) sloupce $A·Q$ jsou LN.
        \end{tvrzeni}

        \begin{tvrzeni}
            Buď $A$ matice typu $m \times n$. Pak řádky $A$ jsou LN, právě když matice $C$, která vznikne Gaussovou eliminací z $A$, má všechny řádky nenulové.
        \end{tvrzeni}

% 19. 11. 2020

        \begin{definice}[Báze]
            ¦V vektorový prostor nad ®T. Pak posloupnost vektorů je báze ¦V, pokud je lineárně nezávislá a její lineární obal je ¦V.
        \end{definice}
        \begin{dusledek}
            Posloupnost vektorů je báze ¦V právě tehdy, když každý vektor z ¦V lze vyjádřit jako právě jednu lineární kombinaci této posloupnosti.
        \end{dusledek}

        \begin{definice}[Kanonická báze]
            ®T těleso, $¦V = ®T^n (n≥1)$. Pak posloupnost vektorů $(e_1, e_2, …, e_n)$ je báze ¦V.
        \end{definice}

        \begin{tvrzeni}
            ¦V VP nad ®T, pak minimální posloupnost generátorů je báze ¦V.
            \begin{dukazin}
                Přímo z definice.
            \end{dukazin}
        \end{tvrzeni}
\end{document}
