\documentclass[12pt]{article}					% Začátek dokumentu
\usepackage{../../MFFStyle}					    % Import stylu



\begin{document}

    \begin{priklad}[2.1]
        Najděte všechna řešení soustavy rovnic v závislosti na parametrech $a, b \in ®C$.
    $$\(\begin{array}{ccc|c}
            −i & a  & 2 − 4i & 0\\
            1  & 2i & b      & 0\\
            i  & −2 & i      & −1
        \end{array}\)$$

        \begin{reseni}
            Upravíme Gaussovou eliminací:
            $$ \(\begin{array}{ccc|c}−i & a & 2 − 4i & 0\\1 & 2i & b & 0\\i & −2 & i & −1\end{array}\) \sim \(\begin{array}{ccc|c}1 & ai & 2i + 4 & 0\\1 & 2i & b & 0\\1 & 2i & 1 & i\end{array}\) \sim $$
            $$ \sim \(\begin{array}{ccc|c}1 & ai & 2i + 4 & 0\\0 & i(2-a) & b-(2i+4) & 0\\0 & 0 & 1-b & i\end{array}\)  \sim \(\begin{array}{ccc|c}1 & 2i & 1 & i\\0 & i(2-a) & -2i-3 & i\\0 & 0 & 1-b & i\end{array}\) $$

            Nyní vidíme, že pokud $b = 1$, vychází $0 = i$ (soustava nemá řešení), dále tedy $b ≠ 1$. Zároveň pokud $a = 2$, pak z rovnosti pravých stran vychází $1 - b = -3-2i$, tj. $b = 4+2i$ (tedy v případě $a=2$ a $b ≠ 4+2i$ soustava nemá řešení).

            Pokud $a = 2$ a $b = 4+2i$, pak $x_2 \in ®R$, $x_3·(1-4-2i) = i \Leftrightarrow x_3 = \frac{i(-3+2i)}{9+4} = -\frac{2+3i}{13}$ a $x_1 = i - x_3 - 2i·x_2 = \frac{2+16i}{13} - 2i·x_2$. Tedy řešením je
            $$ \{\frac{1}{13} \begin{pmatrix} 2+16i \\ 0 \\ -2-3i \end{pmatrix} + t\begin{pmatrix} -2i \\ 1 \\ 0 \end{pmatrix}: t \in ®R\} $$ 

            Nakonec, pokud $a≠2$, dostáváme právě jedno řešení $x_3 = \frac{i}{1-b}$, $x_2 = \frac{i + (2i+3)x_3}{(2-a)i} = \frac{(1-b) + (2i + 3)}{(1-b)(2-a)} =  \frac{4-b+ 2i}{(1-b)(2-a)}$ a $x_1 = i - x_3 - 2i·x_2 = \frac{(1-b)(2-a)i - i(2-a) -2i·(4-b+2i)}{(1-b)(2-a)} =$\penalty -10000$= \frac{2i - 2bi + abi -ai -2i + ai -8i+2bi+4}{(1-b)(2-a)} = \frac{abi -8i + 4}{(1-b)(2-a)}$.
            


        \end{reseni}
    \end{priklad}

    \pagebreak

    \begin{priklad}[2.2]
        Uvažujme přímku $p$ v $®R^3$ zadanou parametricky jako
        $$ p = \{\begin{pmatrix} 1\\2\\3 \end{pmatrix} + t\begin{pmatrix} -1\\0\\1  \end{pmatrix}: t \in ®R \}. $$

        \begin{itemize}
            \item[a)] Pro která $a, b, c, d \in ®R$ obsahuje rovina s obecnou rovnicí $ax + by + cz = d$ přímku $p$.

            \item[b)] Najděte nějakou soustavu 2 lineárních rovnic o 3 neznámých, jejichž množina všech řešení je rovna $p$.
        \end{itemize}

        \begin{reseni}[a]
            Chceme, aby po dosazení přímky do rovnice roviny tato rovnice „vycházela“. Tedy máme:
            $$ (1-t)a + 2b + (3+t)c = d $$
            $t$ je parametr, kdežto vše ostatní konstanty, tedy jelikož rovnice musí „vycházet“ pro všechna $t$, musí se $t$ na levé straně „odečíst“, tudíž $a = c$. Konstanty následně „vychází“ $a + 2b +3a = d$, tj. $d = 4a + 2b$. Zároveň chceme vyloučit případ $\{a,b,c,d\} = \{0\}$, jelikož tomu odpovídá celý prostor a ne jen rovina. Řešením potom je
            $$ \begin{pmatrix} a\\ b\\ c\\ d \end{pmatrix} \in \{\begin{pmatrix} \alpha\\ \beta\\ \alpha\\ 4\alpha + 2\beta \end{pmatrix}: \alpha, \beta \in ®R \land (\alpha ≠ 0 \lor \beta ≠ 0)\} $$ 
        \end{reseni}

        \begin{reseni}[b]
            Rozepíšeme si přímku soustavou rovnic:
            $$ x = 1-t, y = 2, z = 3+t, t \in ®R $$
            A následně sečteme 1. a 3. rovnici, tím odstraníme $t$ a dostaneme tak soustavu 2 lineárních rovnic o 3 neznámých.
            $$ x + z = 4, y = 2 $$ 
            $$ x + 0y + z = 4, 0x + y + 0z = 2 $$ 
        \end{reseni}
    \end{priklad}

\end{document}
