\documentclass[12pt]{article}					% Začátek dokumentu
\usepackage{../../MFFStyle}					    % Import stylu

% \renewcommand{\baselinestretch}{0.7}
\renewcommand{\arraystretch}{1.2}

\let\oldbinom\binom
\def\binom#1#2{\mathchoice{\begin{pmatrix} #1 \\ #2 \end{pmatrix}}{\oldbinom{#1}{#2}}{}{}}

\begin{document}

    \begin{priklad}[10.1]
        O lineárním zobrazení $f: ®R^2 \rightarrow ®R^2$ máme následující informace:

        $$ f \circ f = f \text{ a } f\binom{1}{2} = \binom{-1}{1}. $$

        Určete obraz vektoru $(x, y)^T$ při zobrazení $f$ (v závislosti na $x$ a $y$).

        \begin{reseni}
            Jelikož $f\circ f = f$, tak $f\(f\binom{1}{2}\) = f\binom{1}{2}$, tj. $f\binom{-1}{1} = f\binom{1}{2}$. Protože $f$ je lineární zobrazení mezi prostory dimenze 2, můžeme $f$ zadefinovat pomocí matice typu $2\times 2$:
            $$ f\binom{1}{2} = \begin{pmatrix} a & b \\ c & d \end{pmatrix}\binom{1}{2} = \binom{1a + 2b}{1c + 2d} = \binom{-1}{1}, $$ 
            $$ f\binom{-1}{1} \begin{pmatrix} a & b \\ c & d \end{pmatrix}\binom{-1}{1} = \binom{-1a + 1b}{-1c + 1d} = \binom{-1}{1}. $$

            Vyřešíme jednoduchou soustavu 4 lineárních rovnic o 4 neznámých (nebo v podstatě 2 soustavy rovnic o 2 rovnicích a 2 neznámých, jež jsou až na znaménka totožné) a získáme $a = \frac{1}{3}$, $b = -\frac{2}{3}$, $c = -\frac{1}{3}$, $d = \frac{2}{3}$. Tedy hledané zobrazení je dané maticí $$\begin{pmatrix} \phantom{-}\frac{1}{3} & -\frac{2}{3} \\ -\frac{1}{3} & \phantom{-}\frac{2}{3} \end{pmatrix}, $$ tj. obraz vektoru $\binom{x}{y}$ je:
            $$ \begin{pmatrix} \phantom{-}\frac{1}{3} & -\frac{2}{3} \\ -\frac{1}{3} & \phantom{-}\frac{2}{3} \end{pmatrix}\binom{x}{y} = \binom{\phantom{+}\frac{x}{3}-\frac{2y}{3}}{-\frac{x}{3}+\frac{2y}{3}} = \frac{1}{3}\binom{x-2y}{2y-x}. $$ 
        \end{reseni}
    \end{priklad}

\pagebreak

    \begin{priklad}[10.2]
        Matice lineárního zobrazení $f: ®Z^2_5 \rightarrow ®Z^2_5$ vzhledem ke kanonickým bázím je
        $$ A = \begin{pmatrix} 4 & 4 \\ 4 & 0 \end{pmatrix}. $$

        Najděte nějakou bázi $B$ prostoru $®Z^2_5$ takovou, že matice $f$ vzhledem k $B$ a $B$ je
        $$ [f]^B_B = \begin{pmatrix} 2 & 1 \\ 0 & 2 \end{pmatrix}. $$ 
        \begin{reseni}
            Ze skládání lineárních zobrazení víme, že ($K$ značí kanonickou bázi):
            $$ [id]^B_K·[f]^B_B = [f]^B_K = [f]^K_K·[id]^B_K. $$

            Tj. hledáme $e, f, g, h$ tak, aby:
            $$ \begin{pmatrix} e & f \\ g & h \end{pmatrix}·\begin{pmatrix} 2 & 1 \\ 0 & 2 \end{pmatrix} = \begin{pmatrix} 4 & 4 \\ 4 & 0 \end{pmatrix}·\begin{pmatrix} e & f \\ g & h \end{pmatrix}. $$

            Z definice maticového násobení a rovnosti matic:
            $$ 2e + 0f = 4e + 4g, \hskip 1cm 1e + 2f = 4f + 4h, $$
            $$ 2g + 0h = 4e + 0g, \hskip 1cm 1g + 2h = 4f + 0h. $$

            Aby výsledek byl maticí přechodu od báze k bázi, musí být matice $[id]^B_K$ regulární. To splňuje například řešení $e = f = h = 1$, $g = 2$. Tudíž matice přechodu může být
            $$ [id]^B_K = \begin{pmatrix} 1 & 1 \\ 2 & 1 \end{pmatrix} = \(\[¦b_1\]_K\middle|\[¦b_2\]_K\), $$ 
            a báze $B$ tedy:
            $$ B = \(\binom{1}{2}, \binom{1}{1}\). $$ 
        \end{reseni}
    \end{priklad}

\pagebreak

    \begin{priklad}[10.*]
        Existuje zobrazení $f: ®R \rightarrow ®R$, které pro libovolné $x, y \in ®R$ splňuje $f(x+y) = f(x) + f(y)$, jiné než zobrazení tvaru $f(x) = kx$ (pro $k \in ®R$)? Existuje takové spojité zobrazení?
        
        \begin{reseni}
            Dosazením $x = y = 0$ dostaneme $f(0) = 0$. Následně dosazením $y=-x$ dostaneme $f(-x) = -f(x)$. Použitím tohoto a dosazováním $y = xn, n \in ®N (\implies f(x(n+1)) = (n+1)f(x)$ získáme $f(zx) = zf(x)$ pro všechna $z \in ®Z$. Dosazením $x' = xz$ do předchozího vztahu nám dá $\frac{f(x')}{z} = f(\frac{x'}{z})$ a zkombinováním těchto vztahů získáme $f(qx) = qf(x)$ pro všechna $q \in ®Q$.

            Začněme spojitým zobrazením. Pokud zvolíme $f(1) = k$, pak dostáváme $f(q) = qk$ pro všechna $q \in ®Q$. Ale jelikož je ®R (jako obraz $f$) Hausdorffův prostor a ®Q (jako obor hodnot $f$) husté v ®R (jako množině, na kterou chceme rozšířit obor hodnot $f$), tak zobrazení $f$ má jediné rozšíření na reálná čísla, jež je zřejmě $f(x) = kx$. Tj. jiné zobrazení neexistuje.

            Pokud hledáme ne nutně spojité zobrazení, pak nechť $B$ je báze ®R nad ®Q. Zřejmě je báze víceprvková\footnote{dokonce $B$ je nespočetná, jelikož ®Q je spočetná, tedy pokud by $B$ byla spočetná, pak $B \times ®Q$ by byla spočetná, ale $®R\cong ®Q \times B$ je nespočetná.}, jelikož $®Q ≠ ®R$. Pokud každé „souřadnici“ $¦b$ (prvku) báze $B$ přiřadíme $f(¦b) = k_{¦b} \in ®R$, potom můžeme zobrazení $f$ definovat jako $f(\sum_{¦b \in B} x_{¦b}·¦b) = \sum_{¦b \in B} k_{¦b}·x_{¦b}·¦b$ a z jednoznačnosti vyjádření v bázi ověříme, že $f(x + y) = f(x) + f(y)$, tedy opravdu $f$ je hledané zobrazení, které rozhodně není (pokud jsme všechna $k_{¦b}$ nevolili totožná) shodné s žádným $f'(x) = kx$. Tj. zobrazení ze zadání existuje.
        \end{reseni}
    \end{priklad}

\end{document}
