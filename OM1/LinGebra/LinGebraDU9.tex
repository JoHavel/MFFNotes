\documentclass[12pt]{article}					% Začátek dokumentu
\usepackage{../../MFFStyle}					    % Import stylu

\renewcommand{\baselinestretch}{0.7}

\begin{document}

    \begin{priklad}[9.1]
        V závislosti na parametrech $a, b \in ®Z_5$ určete dimenze prostorů prostorů $\Im A_{a, b}$, $\Im A^T_{a, b}$, $\Ker A_{a, b}$ a $\Ker A^T_{a, b}$ pro matici
        $$ A_{a, b} = \begin{pmatrix} a+2 & b & 1 & 2 \\ 2a+1 & 2b & a+3 & 1 \\ 3a+b+3 & 3b & a+4 & b+3 \end{pmatrix} \in ®Z^{3 \times 4}_5. $$ 

        \begin{reseni}
            Elementární řádkové a sloupcové úpravy nemění hodnost matice, tedy provedeme nějaké řádkové a nějaké sloupcové úpravy:
            $$ \begin{pmatrix} a+2 & b & 1 & 2 \\ 2a+1 & 2b & a+3 & 1 \\ 3a+b+3 & 3b & a+4 & b+3 \end{pmatrix} \overset{\text{EŘÚ}}{\sim} \begin{pmatrix} a+2 & b & 1 & 2 \\ -3 & 0 & a+1 & -3 \\ b-3 & 0 & a+1 & b-3 \end{pmatrix} $$
            $$ \overset{\text{ESÚ}}{\sim} \begin{pmatrix} a & b & 1 & 2 \\ 0 & 0 & a+1 & -3 \\ 0 & 0 & a+1 & b-3 \end{pmatrix} \overset{\text{EŘÚ}}{\sim} \begin{pmatrix} a & b & -a & 0 \\ 0 & 0 & a+1 & -3 \\ 0 & 0 & 0 & b \end{pmatrix} $$

            Když řešíme hodnost matice, tak nás zajímá, které prvky budou v odstupňovaném tvaru nulové a které ne. Tedy musíme prozkoumat $a = 0$, $b = 0$ a $a = -1$.
            \begin{itemize}
                \item Pokud $-1≠a≠0$ a $b≠0$, pak je matice v odstupňovaném tvaru a má všechny 3~řádky nenulové, tedy má hodnost 3.
                \item Při $a=0$ a $b ≠ 0$ nic nemění, jelikož matice bude stále v odstupňovaném tvaru a všechny řádky budou nenulové.
                \item Pokud bude $a=0$ i $b = 0$, pak je první a poslední řádek nulový, tedy matice má zřejmě hodnost 1.
                \item Naopak jestliže $a≠0$ a $b=0$, pak první a druhý řádek jsou nenulové na rozdíl od třetího, matice je v odstupňovaném tvaru, tedy její hodnost je 2.
                \item Nyní schází už jen případ $b≠0 \land a = -1$, v tomto případě není matice v odstupňovaném tvaru, musíme odečíst příslušný násobek 2. řádku od 3. Potom to již bude matice v odstupňovaném tvaru s 2 nenulovými řádky, tj. hodností 2.
            \end{itemize}

            Víme $\dim(\Ker A) + \dim(\Im A) =$ \#počet sloupců a $\dim(\Ker A^T) + \dim(\Im A^T) =$ \#počet řádků. Zároveň hodnost je podle druhé definice $\dim(\Im A) = \dim(\Im A^T)$, tedy můžeme vše dopočítat z hodnosti:
            \begin{itemize}
                \item $b≠0 \land a≠-1 \implies \rank A_{a, b} = 3 \implies \dim(\Im A) = \dim(\Im A^T) = 3$, $\dim(\Ker A) = 4-3 = 1$ a $\dim(\Ker A^T) = 3-3 = 0$,
                \item $a≠0 \land b=0 \lor a=-1 \implies \rank A_{a, b} = 2 \implies \dim(\Im A) = \dim(\Im A^T) = 2$, $\dim(\Ker A) = 4-2 = 2$ a $\dim(\Ker A^T) = 3-2 = 1$,
                \item $a=b=0 \implies \rank A_{a, b} = 1 \implies \dim(\Im A) = \dim(\Im A^T) = 1$, $\dim(\Ker A) = 4-1 = 3$ a $\dim(\Ker A^T) = 3-1 = 2$.
            \end{itemize}
        \end{reseni}
    \end{priklad}

\pagebreak

    \begin{priklad}[9.2]
        Najděte nějakou bázi průniku podprostorů $¦X$ a $¦Y$ prostoru $®R^4$.
        \begin{align*}
            ¦X &= \{\(x_1, x_2, x_3, x_4\)^T \in ®R^4: x_1 + 2x_2 - x_3 + 3x_4 = 0\}\\
            ¦Y &= \LO\{(2, 1, 1, 0)^T, (1, -1, 1, 1)^T\}
        \end{align*} 

        \begin{reseni}
            Z definice $\LO$ si libovolný prvek $¦Y$ (tím pádem i libovolný $¦Y \cap ¦X$) můžeme zapsat jako $¦y = (y_1, y_2, y_3, y_4)^T = a(2, 1, 1, 0)^T + b(1, -1, 1, 1)^T$. Pokud je tento prvek z $¦Y \cap ¦X$, tak navíc musí splňovat $y_1 + 2y_2 - y_3 + 3y_4 = 0$, tj.:
            $$ (2a + b) + 2·(a-b) - (a + b) + 3(b) = 0 $$
            $$ 3a + b = 0 $$

            Vidíme, že řešení má jednu dimenzi (např. $a$ je volná proměnná, $b = -3a$), tj. báze má jeden prvek a to např. (libovolná jiná $a$ a $b$ splňující $b = -3a$ budou zjevně jen násobky tohoto) $a = 1, b = -3$:
            $$ ¦X \cap ¦Y = \LO\(\begin{pmatrix} 1·2 - 3·1 \\ 1·1 - 3·(-1) \\ 1·1 - 3·1 \\ 1·0 - 3·1 \end{pmatrix}\) = \LO\(\begin{pmatrix} -1 \\ 4 \\ -2 \\ -3 \end{pmatrix}\) $$ 

        \end{reseni}
    \end{priklad}

\end{document}
