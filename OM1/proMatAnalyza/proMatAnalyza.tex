\documentclass[12pt]{article}					% Začátek dokumentu
\usepackage{../../MFFStyle}					    % Import stylu



\begin{document}

\section*{Organizační úvod}
   Přednášky budou nahrávány, referáty ne.

   Kontaktovat přes e-mail slavikova@karlin.mff.cuni.cz

   Teoretické příklady odevzdávat přes Moodle.

\section{Prvočísla}
    \begin{definice}[Dělitel]
        Číslo $d \in ®Z$ nazýváme dělitelem čísla $n \in ®Z$, značeno $d\div n$, pokud existuje $k \in ®Z$ splňující $n = kd$.
    \end{definice}
    
    \begin{definice}[Prvočíslo]
        Řekněme, že $n \in ®N$ je prvočíslo, pokud $n>1$ a jeho jediní kladní dělitelné jsou $1≥n$.
        \begin{prikladyin}[Několik prvních prvočísel]
            2, 3, 5, 7, 11, 13, 17, …
        \end{prikladyin}
    \end{definice}

    \begin{veta}[Základní věta aritmetiky]
        Každé přirozené číslo $n≥2$ lze zapsat právě jedním způsobem jako součin prvočísel ve tvaru:
        $$ n = p_1^{\alpha_1}p_2^{\alpha_2}\cdots p_k^{\alpha_k} $$ 
        $$ k \in N, p_1<p_2<\cdots<p_k \text{jsou prvočísla}, \alpha_1,…,\alpha_k \in ®N$$
        \begin{prikladyin}
                $$ 2020 = 2^2 \cdot 5 \cdot 101 (k = 3, p_1 = 2, p_2 = 5, p_3 = 101, \alpha_1 = 2, \alpha_2 = 1, \alpha_3 = 1) $$ 
        \end{prikladyin}
        \begin{dukazin}
            1. krok = existence rozkladu (indukcí):
            
            Pro $n = 2$ zjevně platí $2 = 2^1$ ($k = 1, p_1 = 2, \alpha_1 = 1$).

            Předpokládejme, že tvrzení platí pro všechna $2≤x≤n$. Pokud je $n+1$ prvočíslo, pak $n+1 = (n+1)^1$ ($ k = 1, p_1 = n+1, \alpha_1 = 1$). Pokud není, pak $n+1 = a\cdot b$, kde $1 < a ≤ b < n+1$. Podle indukčního předpokladu lze $a$ i $b$ rozložit na prvočísla. Zápis rozkladu $n+1$ pak bude sjednocením všech prvočísel a součtem příslušných $\alpha$, pokud se prvočísla vyskytují v $a$ i $b$. (V přednášce byl zaveden zápis bez mocnin, kde prvočísla nemusí být různá, a pak proveden součin.)

            2. krok = jednoznačnost rozkladu:

            \begin{lemmain}[Euklidovo lemma (bez důkazu)]
                Nechť $a, b \in ®Z$ a nechť $p$ je prvočíslo takové, že $p\mid ab$. Pak $p\mid a$ nebo $p\mid b$.
            \end{lemmain}

            Použijeme důkaz sporem. Předpokládejme, že tvrzení neplatí. Vybereme nejmenší z přirozených čísel, pro které rozklad není jednoznačný. Označme ho $n$.
            $$ n = q_1\cdots q_l = r_1 \cdot r_m\ (q_1,…,q_l,r_1,…,r_m \text{prvočísla}) $$
            A není pravda, že $(r_1,…,r_m)$ je permutací $(q_1,…,q_l)$.

            Protože $q_1\mid n$, pak $q_1 \mid r_1 \cdots r_m$ a podle Euklidova lemmatu $q_1$ dělí alespoň jedno z čísel $r_1,…r_m$. BÚNO $q_1 | r_1$, tedy $q_1 = r_1$. Vydělením $n$ číslem $q_1$ dostaneme menší přirozené číslo, které nemá jednoznačný rozklad. ($\frac{n}{q_1} = q_2\cdots q_l = r_2 \cdots r_m$). $\lightning$
        \end{dukazin}
    \end{veta}

    \begin{veta}
        Prvočísel je nekonečně mnoho.
        \begin{dukazin}
            Důkaz sporem. Předpokládejme, že prvočísel je konečně mnoho, a  označme $p$ největší prvočíslo. Definujeme:
            $$ n_p = 2\cdot 3\cdot 5 \cdot\cdots\cdot p + 1 $$
            Pak $n_p > p$ a $n_p$ dává zbytek 1 po dělení všemi prvočísly, tedy není ani jedním dělitelné. Tedy $n_p$ nemá prvočíselný rozklad. $\lightning$ se základní větou aritmetiky.
        \end{dukazin}

        \begin{poznamkain}
            Důkaz nedává konstrukci vyššího prvočísla, pouze dokazuje jeho existenci.
        \end{poznamkain}
    \end{veta}

    \begin{veta}[Hustota prvočísel (bez důkazu), Legendre 1896]
        \begin{prikladyin}
            Mezi 1 a 100 je 25 prvočísel.

            Mezi $10^7$ a $10^7 + 100$ jsou pouze 2 prvočísla.
        \end{prikladyin}

        Označme $\Pi(N)$ počet prvočísel $≤N$.

        Existují konstanty $c_1, c_2>0$ takové, že
        $$ \frac{c_1}{\log N} ≤ \frac{\Pi(N)}{N} ≤ \frac{c_2}{\log N} $$ 
        \begin{poznamkain}
            Prvočísel je nekonečně mnoho, ale „řídnou“. Musí tedy existovat dlouhé úseky bez prvočísel.
            \begin{prikladyin}
                Interval $\[n! + 2, …, n! + n\]$ neobsahuje žádné prvočíslo. (Jelikož $k$-té číslo je dělitelné $k+1$.)
            \end{prikladyin}
        \end{poznamkain}
    \end{veta}

\section{Čísla racionální a iracionální}
    \begin{definice}[Racionální a iracionální číslo]
        Číslo $x \in ®R$ je racionální, pokud ho lze zapsat ve tvaru $x = \frac{p}{q},\ q \in ®N,\ p \in ®Z$.

        Číslo $y \in ®R$ je iracionální, pokud není racionální.
    \end{definice}

    \begin{priklady}[Z přednášky]
        $\sqrt{2}$ je iracionální.
    \end{priklady}

    \begin{veta}
        Nechť $n \in ®N$ je taková, že $\sqrt{n} \notin ®N$ (tedy $n$ není druhou mocninou přirozeného čísla). Pak $\sqrt{n}$ je iracionální.
        \begin{lemmain}
            Jsou-li $p, q$ nesoudělná, pak $p^2,\ q^2$ jsou také nesoudělná.
            \begin{dukazin}
                Dle základní věty aritmetiky každé přirozené číslo lze rozložit na součin prvočísel. Rozložíme a dokážeme.
            \end{dukazin}
        \end{lemmain}

        \begin{dukazin}[Sporem]
            Předpokládejme, že $\sqrt{n}$ je racionální, ale není to celé číslo. Pak $\sqrt{n} = \frac{p}{q}$, kde $p, q$ jsou nesoudělná přirozená čísla ($q≥2$). Umocníme: $n = \frac{p^2}{q^2}$. $q|p$ $lightning$.
        \end{dukazin}
    \end{veta}

    \begin{veta}[Referát 1]
        Existují iracionální čísla a,b taková, že $a^b$ je racionální. (Text: skripta z MA, str. 14-15.)
        \begin{dukazin}
            Buď $\sqrt{2}^{\sqrt{2}}$ nebo $(\sqrt{2}^{\sqrt{2}})^{\sqrt{2}} = {\sqrt{2}}^2 = 2$
        \end{dukazin}
    \end{veta}

    \begin{priklad}[Teoretický příklad 1]
        Nechť $n \in ®N$ a nechť $a_1, …, a_n$ jsou kladná reálná čísla, taková, že $a_1 \cdot\cdots\cdot a_n = 1$.

        Dokažte, že
        $$ (1+a_1) \cdot\cdots\cdot (1+a_n) ≥ 2^n. $$ 
    \end{priklad}

% 12. 10. 2020

    \begin{priklad}[Teoretický příklad 2]
        Nalezněte supremum a infimum množiny
        $$ \{\sqrt{n} - \lfloor\sqrt{n}\rfloor : n \in ®N\} $$ 
    \end{priklad}

\section{Mohutnost množin}
    \begin{definice}
        Množiny ®X, ®Y mají stejnou mohutnost, pokud existuje bijekce ®X na ®Y. Značíme $®X ≈ ®Y$.

        Množina ®X má mohutnost menši nebo rovnu mohutnosti ®Y, pokud existuje prosté zobrazení ®X do ®Y. Značíme $®X \preceq ®Y$.

        Množina ®X má menší mohutnost než ®Y, pokud $®X \preceq ®Y$, ale neplatí $®Y \preceq ®X$. Značíme $®X \prec ®Y$.
    \end{definice}

    \begin{veta}(Cantor-Bernstein)
        Nechť ®X a ®Y jsou množiny splňující $®X \preceq ®Y$ a $®Y \preceq ®X$, pak $®X ≈ ®Y$.

% 19. 10. 2020

        \begin{lemmain}
            Nechť $®X$ je množina a $H: ©P(®X) \rightarrow ©P(®X)$ je zobrazení splňující podmínku $\forall ®A, ®B \in ©P(®X): ®A \subset ®B \implies H(®A) \subset H(®B)$. Pak existuje $®C \subset ®X$ takové, že $H(C) = C$.
            \begin{dukazin}
                Položme $©C = \{®A \in ©P(®X): ®A \subset H(®A)\}$. Ukážeme, že $®C = \bigcap ©C$ je hledanou množinou. $C \subset ®X$ je zřejmé, $C \subset H(C):$ Pokud $®A \in ©C$, pak $®A \subset C$, pak z vlastnosti zobrazení plyne $H(®A) \subset H(®C)$. Tedy $®A \subset H(®A) \subset H(®C)$. Z definice ®C dostáváme $C \subset H(®C)$. Nakonec musíme ještě dokázat $H(®C) \subset ®C$. Z $®C \subset H(®C)$ a z vlastnosti zobrazení $H(®C)\subset H(H(®C))$ TODO! $H(®C) \subset ®C$.
            \end{dukazin}
        \end{lemmain}
        \begin{dukaz}
            Předpokládáme $®X \preceq ®Y \implies$ existuje prosté zobrazení $f: ®X \rightarrow ®Y$ a  $®Y \preceq X \implies$ existuje prosté zobrazení $f: ®Y \rightarrow ®X$.

            Definujeme $H: ©P(®X) \rightarrow ©P(®X)$ předpisem $H(®A) = ®X \setminus g(®Y \setminus f(®A))$. (Pozorování, jestliže $f = g^{-1}$ je prosté a na, tak $H$ je identita.) Nyní ověříme předpoklady Lemmatu.

            Nechť $®U \subset ®V \subset ®X$. Pak $f(®U) \subset f(®V) \implies ®Y \setminus f(®V) \subset ®Y \setminus f(®U) \implies g(®Y \setminus f(®V)) \subset g(®Y \setminus f(®U)) \implies ®X \setminus g(®Y \setminus f(®U)) \subset ®X \setminus g(®Y \setminus f(®V)) \implies H(®U) \subset H(®V)$.

            Dle lemmatu existuje $®C \subset ®X$ takové, že $H(®C) = ®C$. Pak $®C = H(®C) = ®X \setminus g(®Y\setminus f(®C)), g(®Y \setminus f(®C)) = ®X \setminus ®C$. Tedy $g|_{®Y \setminus f(®C)}$ je prosté zobrazení $®Y\setminus f(®C)$ ne $®X|®C$, a tedy $g^{-1}|_{®X \setminus ®C}$ je prosté zobrazení $®X\setminus ®C$ na $®Y\setminus f(®C)$. Navíc jistě $f|_{®C}$ je prosté zobrazení $®C$ na $f(®C)$

            Definujeme $h(a) = f(a), a\in®C|h(a) = g^{-1}(a), a\in ®X\setminus ®C$. Potom $h$ je prosté zobrazení ®X na ®Y.
        \end{dukaz}
    \end{veta}

\section{Aritmetický, geometrický a harmonický průměr}
    \begin{definice}
        Nechť $x_1, …, x_n > 0$. Definujeme jejich
        \begin{itemize}
            \item aritmetický průměr jako $A_n = \frac{x_1 + … + x_n}{n}$.
            \item geometrický průměr jako $G_n = \sqrt[n]{x_1\cdot \cdots \cdot x_n}$
            \item harmonický průměr jako $H_n = \frac{n}{\frac{1}{x_1}+…+\frac{1}{x_n}}$
        \end{itemize}
    \end{definice}

    \begin{veta}[AGH nerovnost]
        $$ A_n ≥ G_n ≥ H_n $$
        \begin{poznamkain}[Pozorování]
            Nerovnost $G_n≥H_n$ snadno plyne z $A_n ≥ G_N$, stačí dosadit $x_i = \frac{1}{y_i}$. Stačí ukázat nerovnost $A_n ≥ G_n$.
        \end{poznamkain}

        \begin{dukazin}[1. Zpětnou indukcí]
            Dokážeme pro mocniny 2. Následně dosadíme za jedno $x$ geometrický průměr těch ostatních a budeme „indukovat“ zpět.
        \end{dukazin}

        \begin{dukazin}[2. Indukcí]
            Dokážeme pro 1.

            Chceme odhadnout aritmetický průměr $n+1$ čísel. Použijeme indukční předpoklad pro $n$ čísel, jejichž aritmetický průměr je stejný. Za tato čísla zvolíme $x_1, …, x_{n-1}, x_n'$, kde $x_n'$ je doplnění, aby byly shodné aritmetické průměry. $x_n'$ vyjádříme. Upravíme, řekneme, že $x_n$ a $x_{n+1}$ jsme zvolili, že budou nejmenší a největší číslo.
        \end{dukazin}
    \end{veta}

    \begin{poznamka}[2. referát]
        Existuje i aritmeticko-geometrický průměr a aritmeticko-harmonický průměr (je roven geometrickému).
    \end{poznamka}

    \begin{priklad}[3. teoretický]
        Najděte všechna celá čísla $m$ splňující
        $$ (1+m)^n ≥ 1+mn, \forall n \in ®N $$ 
    \end{priklad}
    
    TODO? (Posloupnost $(1+\frac{1}{n})^n$ je rostoucí a shora omezená číslem 3, posloupnost $(1+\frac{1}{n})^{n+1}$ je naopak klesající).

% 2. 11. 2020

    \begin{definice}[Aritmeticko-geometrický průměr]
        Nechť $0 < b_1 < a_1$. $a_{n+1} = \frac{a_n + b_n}{2}$ a $b_{n+1} = \sqrt{a_n·b_n}$. Limita těchto posloupností se nazývá aritmeticko-geometrický průměr.

        \begin{dukazin}[Shodnost a existence limit]
            $a_n ≥ b_n$ z AG nerovnosti. Dokáže se monotónnost, z toho plyne existence limit. Pak z AL $A = \frac{A+B}{2}$, tedy $A = B$.
        \end{dukazin}
    \end{definice}

    \begin{definice}[Aritmeticko-harmonický průměr]
        Definujeme a dokazujeme obdobně jako výše.
    \end{definice}

    \begin{veta}
        Aritmeticko-harmonický průměr je roven geometrickému.
        \begin{dukazin}
            Součin členů se stejným indexem je roven součinu prvních členů, z toho limita součinu je součin prvních členů, z toho vyplývá, že limita činitelů (jelikož je shodná) je odmocninou ze součinu prvních členů (geometrický průměr).
        \end{dukazin}
    \end{veta}

    \begin{veta}[Referáty]
        Z množiny hromadných bodů nelze „vykonvergovat“.

        O množině hromadných bodů posloupnosti $\{a_n\}$ splňující $\lim_{n \rightarrow ∞} (a_{n+1} - a_n)$.
    \end{veta}

    \begin{priklad}[Teoretický příklad 5]
        Nechť $\{a_n\}_{n = 1}^∞$ je omezená posloupnost splňující
        $$ a_{n+1} ≥ a_n - \frac{1}{2^n}. $$
        Dokažte, že posloupnost $\{a_n\}$ je konvergentní.
    \end{priklad}

\section{Odhady faktoriálu}
    \begin{tvrzeni}
        $$ \(\frac{n}{3}\)^n < n! < \(\frac{n+1}{2}\) $$

        \begin{dukazin}[První nerovnost]
            Označme $\beta_n = \(\frac{n}{3}\)^n, n \in ®N$. Pak $\(\frac{n}{3}\)^n = \beta_n = \beta_1 · \frac{\beta_2}{\beta_1} · … ·\frac{\beta_n}{\beta_{n-1}}$. Odhadneme $\frac{\beta_k}{\beta_{k-1}}<k$. Tedy $\beta_n < 1·2·3·…·n = n!$.
        \end{dukazin}

        \begin{dukazin}[Druhá nerovnost]
            AG.
        \end{dukazin}
    \end{tvrzeni}

    Horní odhad lze zlepšit:
    \begin{tvrzeni}
        $$ n! < \(\frac{n}{2}\)^n,\ n≥6. $$
        \begin{dukazin}[Indukcí]
            $n = 6: 6! = 720 = 9·80 < 9·81 = 3^6$.

            $(n+1)! = (n+1)n! < (n+1) \(\frac{n}{2}\)^n \overset{?}{<} \(\frac{n+1}{2}\)^{n+1}$. Stačí ukázat $\frac{n^n}{2^n} < \frac{(n+1)^n}{2^{n+1}}$, tedy $2<\(\frac{n+1}{n}\)^n = \(1+\frac{1}{n}\)^n$.
        \end{dukazin}
    \end{tvrzeni}

    \begin{definice}[e]
        $$ \lim_{n \rightarrow ∞} (1+\frac{1}{n})^n = e. $$
    \end{definice}

    \begin{tvrzeni}
        $$ \lim_{n \rightarrow ∞} (1+\frac{1}{n})^{n+1} = e. $$
        
        \begin{dukazin}
            Z AL: $ \lim_{n \rightarrow ∞} (1+\frac{1}{n})^{n+1} = \lim_{n \rightarrow ∞} (1+\frac{1}{n})^{n} · \lim_{n \rightarrow ∞} \(1+\frac{1}{n}\) = e·1 = e $.
        \end{dukazin}
    \end{tvrzeni}

    \begin{tvrzeni}
        $$ \(\frac{n}{e}\)^n < n! < \(\frac{n}{e}\)^{n+1},\ n≥2. $$ 

        \begin{dukazin}
            1. nerovnost stejně jako výše s 3 místo e. Druhá nerovnost indukcí.
        \end{dukazin}
    \end{tvrzeni}

    \begin{poznamka}[Stirlingova formule (odhad fakotriálu)]
        $$ n!≈\sqrt{2\pi} \(\frac{n}{e}\)^n \sqrt{n}. $$

        (Ve smyslu $\lim_{n \rightarrow ∞} \frac{n!}{\sqrt{2\pi} \(\frac{n}{e}\)^n \sqrt{n}} = 1$.)
    \end{poznamka}

\end{document}
