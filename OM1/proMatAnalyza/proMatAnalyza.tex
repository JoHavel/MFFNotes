\documentclass[12pt]{article}					% Začátek dokumentu
\usepackage{../../MFFStyle}					    % Import stylu



\begin{document}

\section*{Organizační úvod}
   Přednášky budou nahrávány, referáty ne.

   Kontaktovat přes e-mail slavikova@karlin.mff.cuni.cz

   Teoretické příklady odevzdávat přes Moodle.

\section{Prvočísla}
    \begin{definice}[Dělitel]
        Číslo $d \in \Z$ nazýváme dělitelem čísla $n \in \Z$, značeno $d\div n$, pokud existuje $k \in \Z$ splňující $n = kd$.
    \end{definice}
    
    \begin{definice}[Prvočíslo]
        Řekněme, že $n \in \N$ je prvočíslo, pokud $n>1$ a jeho jediní kladní dělitelné jsou $1≥n$.
        \begin{prikladyin}[Několik prvních prvočísel]
            2, 3, 5, 7, 11, 13, 17, …
        \end{prikladyin}
    \end{definice}

    \begin{veta}[Základní věta aritmetiky]
        Každé přirozené číslo $n≥2$ lze zapsat právě jedním způsobem jako součin prvočísel ve tvaru:
        $$ n = p_1^{\alpha_1}p_2^{\alpha_2}\cdots p_k^{\alpha_k} $$ 
        $$ k \in N, p_1<p_2<\cdots<p_k \text{jsou prvočísla}, \alpha_1,…,\alpha_k \in \N$$
        \begin{prikladyin}
                $$ 2020 = 2^2 \cdot 5 \cdot 101 (k = 3, p_1 = 2, p_2 = 5, p_3 = 101, \alpha_1 = 2, \alpha_2 = 1, \alpha_3 = 1) $$ 
        \end{prikladyin}
        \begin{dukazin}
            1. krok = existence rozkladu (indukcí):
            
            Pro $n = 2$ zjevně platí $2 = 2^1$ ($k = 1, p_1 = 2, \alpha_1 = 1$).

            Předpokládejme, že tvrzení platí pro všechna $2≤x≤n$. Pokud je $n+1$ prvočíslo, pak $n+1 = (n+1)^1$ ($ k = 1, p_1 = n+1, \alpha_1 = 1$). Pokud není, pak $n+1 = a\cdot b$, kde $1 < a ≤ b < n+1$. Podle indukčního předpokladu lze $a$ i $b$ rozložit na prvočísla. Zápis rozkladu $n+1$ pak bude sjednocením všech prvočísel a součtem příslušných $\alpha$, pokud se prvočísla vyskytují v $a$ i $b$. (V přednášce byl zaveden zápis bez mocnin, kde prvočísla nemusí být různá, a pak proveden součin.)

            2. krok = jednoznačnost rozkladu:

            \begin{lemmain}[Euklidovo lemma (bez důkazu)]
                Nechť $a, b \in \Z$ a nechť $p$ je prvočíslo takové, že $p\mid ab$. Pak $p\mid a$ nebo $p\mid b$.
            \end{lemmain}

            Použijeme důkaz sporem. Předpokládejme, že tvrzení neplatí. Vybereme nejmenší z přirozených čísel, pro které rozklad není jednoznačný. Označme ho $n$.
            $$ n = q_1\cdots q_l = r_1 \cdot r_m\ (q_1,…,q_l,r_1,…,r_m \text{prvočísla}) $$
            A není pravda, že $(r_1,…,r_m)$ je permutací $(q_1,…,q_l)$.

            Protože $q_1\mid n$, pak $q_1 \mid r_1 \cdots r_m$ a podle Euklidova lemmatu $q_1$ dělí alespoň jedno z čísel $r_1,…r_m$. BÚNO $q_1 | r_1$, tedy $q_1 = r_1$. Vydělením $n$ číslem $q_1$ dostaneme menší přirozené číslo, které nemá jednoznačný rozklad. ($\frac{n}{q_1} = q_2\cdots q_l = r_2 \cdots r_m$). $\lightning$
        \end{dukazin}
    \end{veta}

    \begin{veta}
        Prvočísel je nekonečně mnoho.
        \begin{dukazin}
            Důkaz sporem. Předpokládejme, že prvočísel je konečně mnoho, a  označme $p$ největší prvočíslo. Definujeme:
            $$ n_p = 2\cdot 3\cdot 5 \cdot\cdots\cdot p + 1 $$
            Pak $n_p > p$ a $n_p$ dává zbytek 1 po dělení všemi prvočísly, tedy není ani jedním dělitelné. Tedy $n_p$ nemá prvočíselný rozklad. $\lightning$ se základní větou aritmetiky.
        \end{dukazin}

        \begin{poznamkain}
            Důkaz nedává konstrukci vyššího prvočísla, pouze dokazuje jeho existenci.
        \end{poznamkain}
    \end{veta}

    \begin{veta}[Hustota prvočísel (bez důkazu), Legendre 1896]
        \begin{prikladyin}
            Mezi 1 a 100 je 25 prvočísel.

            Mezi $10^7$ a $10^7 + 100$ jsou pouze 2 prvočísla.
        \end{prikladyin}

        Označme $\Pi(N)$ počet prvočísel $≤N$.

        Existují konstanty $c_1, c_2>0$ takové, že
        $$ \frac{c_1}{\log N} ≤ \frac{\Pi(N)}{N} ≤ \frac{c_2}{\log N} $$ 
        \begin{poznamkain}
            Prvočísel je nekonečně mnoho, ale „řídnou“. Musí tedy existovat dlouhé úseky bez prvočísel.
            \begin{prikladyin}
                Interval $\[n! + 2, …, n! + n\]$ neobsahuje žádné prvočíslo. (Jelikož $k$-té číslo je dělitelné $k+1$.)
            \end{prikladyin}
        \end{poznamkain}
    \end{veta}

\section{Čísla racionální a iracionální}
    \begin{definice}[Racionální a iracionální číslo]
        Číslo $x \in \R$ je racionální, pokud ho lze zapsat ve tvaru $x = \frac{p}{q},\ q \in \N,\ p \in \Z$.

        Číslo $y \in \R$ je iracionální, pokud není racionální.
    \end{definice}

    \begin{priklady}[Z přednášky]
        $\sqrt{2}$ je iracionální.
    \end{priklady}

    \begin{veta}
            Nechť $n \in \N$ je taková, že $\sqrt{n} \notin \N$ (tedy $n$ není druhou mocninou přirozeného čísla). Pak $\sqrt{n}$ je iracionální.
    \end{veta}

    \begin{veta}[Referát 1]
            Existují iracionální čísla a,b taková, že $a^b$ je racionální. (Text: skripta z MA, str. 14-15.)
    \end{veta}

    \begin{priklad}[Teoretický příklad 1]
        Nechť $n \in \N$ a nechť $a_1, …, a_n$ jsou kladná reálná čísla, taková, že $a_1 \cdot\cdots\cdot a_n = 1$.

        Dokažte, že
        $$ (1+a_1) \cdot\cdots\cdot (1+a_n) ≥ 2^n. $$ 
    \end{priklad}

\end{document}
