\documentclass[12pt]{article}					% Začátek dokumentu
\usepackage{../../MFFStyle}					    % Import stylu



\begin{document}

\section{Úvod}
    \begin{poznamka}[Co je diskrétní matematika]
        Zkoumá struktury z oddělených částí (protipólem je spojitá matematika).
        \begin{itemize}
            \item Množiny, relace
            \item Kombinatorika
            \item Teorie grafů
        \end{itemize}
    \end{poznamka}

    \begin{definice}[Značení]
        \begin{itemize}
            \item \N -- přirozená čísla od 1
            \item $\N_0$ -- přirozená čísla od 0
            \item \Z -- celá čísla
            \item \R -- reálná čísla
            \item $\(a_1; a_2; …; a_k\)$ -- uspořádaná $k$-tice
            \item $\{a_1; a_2; …; a_k\}$ -- množina s prvky $a_1; …; a_k$
        \end{itemize}
    \end{definice}

    \subsection{Množiny}
        \begin{poznamka}[Shodnost množin]
                $$ \X = \Y \Leftrightarrow (\forall z) (z \in \X \Leftrightarrow z \in \Y) $$ 
        \end{poznamka}

        \begin{poznamka}[Historie teorie množin]
            70. léta 19. století G. Cantor (zavedl např. porovnávání velikostí množin, viz bijekce, také přišel na to, že ne všechny nekonečné množiny jsou spočetné (stejně velké jako \N)).

            1901 Russelův paradox (B. Russel, rozbil naivní teorii množin, která platila do té doby) -- množina všech množin, které neobsahují sami sebe (ke každé vlastnosti by měla existovat množina všeho, co ji splňuje, ale to očividně nelze).

            Axiomatická teorie množin (opravuje teorii množin, aby zůstali výsledky, ale zmizel R. paradox) -- např. Zermelo-Frankelovy axiomy: všechny objekty jsou množiny, existuje predikát náležení ($x \in \X$ -- $x$ je prvkem \X), pomocí něhož lze definovat další predikáty, jako $\subseteq$. Axiomy jsou definované tak, aby $\X \notin \X$ (dokonce ani nemohou vznikat cykly $\in$). Také zajišťují, že výsledky klasických operací ($\cup$, $\cap$, $\\$, …) a klasické množiny (\N, \Z, \R, …) jsou množiny.

            Nikdo však ještě nedokázal, že další paradox neexistuje. Zároveň se nedařilo dokázat, zda některá tvrzení (např. axiom výběru) platí, nebo ne (dnes je často dokázáno, že se nedají dokázat).

            30. léta C. Gödel dokázal, že každá teorie má tvrzení, které v ní nelze dokázat.
        \end{poznamka}

        \begin{definice}[Operace s množinami]
            $$ \Z \in (\X \cup \Y) \Leftrightarrow (\Z \in \X) \lor (\Z \in \Y) $$
            $$ \Z \in (\X \cup \Y) \Leftrightarrow (\Z \in \X) \land (\Z \in \Y) $$

            Obě operace lze udělat s libovolným počtem (i nespočetným) množin.

            $$ \Z \in (\X \\ \Y) \Leftrightarrow (\Z \in \X \land \Z \notin \Y) $$

            Doplněk \Y, značený $\overline{Y}$ je zkratka pro $\X \\ \Y$, kde \X musí být jasná z kontextu.

            Kartézský součin $\X\times\Y = \{(a,b): a\in\X \land b\in\Y\}$.

            Potenční množina množiny \X : $2^\X - \P(\X) = \{\Y : \Y \subseteq \X\}$
        \end{definice}




\end{document}

