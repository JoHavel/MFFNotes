\documentclass[12pt]{article}					% Začátek dokumentu
\usepackage{../../MFFStyle}					    % Import stylu



\begin{document}

\section*{Organizační úvod}
    \begin{poznamka}[Zápočet]
        Za vypracování domácích úloh.
    \end{poznamka}
    \begin{poznamka}[Zkouška]
        Písemná, ale Covid?
    \end{poznamka}
\section*{Úvod}
    MA je na rovném prostoru $\R^n$
    Naším cílem je vybudovat analýzu na nerovném? prostoru, tzv. varietě.
    \begin{poznamka}[literatura]
        Skripta -- Krump, Souček, Těšínský: MA ve varietách\\
        Sborník příkladů -- Kopáček: Příklady z matematiky pro fyziky III.
    \end{poznamka}

\section{Opakování}
\noindent 'Odvozovali' (přes limity velikosti rozdělení jdoucí k nule) jsme si:

    Křivkový integrál 1. druhu, křivkový integrál 2. druhu. Integrální věty (pol. 19. stol, moderní formulace Cardan (1945)): Věta o potenciálu, Greenova věta

    Plošný integrál 1. druhu, plošný integrál 2. druhu. Integrální věty: Stokesova věta, Gauss-Ostrogradského věta





\section{Stokesova věta v $\R^n$, diferenciální formy v $\R^n$}
    \begin{veta}[Moderní (= obecná) formulace Stokesovy věty]
        $$ \int_{S} {d\omega} = \int_{\delta S}\omega $$
        Kde S je buď 'singulární' \k-plocha v $R^n$ (tato část) nebo \k-varieta s okrajem (3. část).
          
    \end{veta}


\section{Variety, Stokesova věta na varietách}

    
\end{document}
