\documentclass[12pt]{article}					% Začátek dokumentu
\usepackage{../../MFFStyle}					    % Import stylu



\begin{document}

\section*{Organizační úvod}
    \begin{poznamka}[Zápočet]
        Za vypracování domácích úloh.
    \end{poznamka}
    \begin{poznamka}[Zkouška]
        Písemná, ale Covid?
    \end{poznamka}
\section*{Úvod}
    MA je na rovném prostoru $\R^n$
    Naším cílem je vybudovat analýzu na nerovném? prostoru, tzv. varietě.
    \begin{poznamka}[literatura]
        Skripta -- Krump, Souček, Těšínský: MA ve varietách\\
        Sborník příkladů -- Kopáček: Příklady z matematiky pro fyziky III.
    \end{poznamka}

\section{Opakování}
\noindent 'Odvozovali' (přes limity velikosti rozdělení jdoucí k nule) jsme si:

    Křivkový integrál 1. druhu, křivkový integrál 2. druhu. Integrální věty (pol. 19. stol, moderní formulace Cardan (1945)): Věta o potenciálu, Greenova věta

    Plošný integrál 1. druhu, plošný integrál 2. druhu. Integrální věty: Stokesova věta, Gauss-Ostrogradského věta





\section{Stokesova věta v $\R^n$, diferenciální formy v $\R^n$}
\begin{veta}[Moderní (= obecná) formulace Stokesovy věty = Cíl (Cartan 1945)]
        $$ \int_{S} {d\omega} = \int_{\delta S}\omega $$
        Kde S je buď 'singulární' \k-plocha v $R^n$ (tato část) nebo \k-varieta s okrajem (3. část).
          
    \end{veta}

% 7. 10. 2020

    \subsection{Vnější algebra vektorového prostoru}
        Motivace: Jak násobit vektory z $\R^n$?

        \begin{poznamka}
                Násobení na $\R^n$ zachovává Eklidovskou normu (tzn. $||x\cdot y = ||x|| \cdot ||y||$) pouze v dimenzích 1, 2, 4, 8 (= \R, \C, kvaterniony, oktocosi).
        \end{poznamka}

        \begin{definice}[Algebra]
            Algebra nad tělesem k (= \R) je vektorový prostor $\A$ nad k s bilineárním zobrazením ….

            Algebra je asociativní, jestliže co asi.

            Algebra má jednotku, jestliže existuje co asi ;)
        \end{definice}

        \begin{definice}[]
            Nechť $\Lambda$ je vektorový prostor nad $\R$
        \end{definice}

% od 7. 10. 2020 nestihnuto spoustu věcí!

        \begin{poznamka}[Vlastnosti vnější algebry]
            $\dim \Lambda*(\V) = 2^n$, protože každý vektor je určen bázovými vektory, kterých je jako podmnožin $n$ prvkové množiny

            TODO

            \begin{align*}
                e_I \land e_J &= 0, \text{je-li} I \cap J \neq \O
                          &= sgn(\text{permutace})e_{I\cup J}, \text{je-li} I \cap J = \O
            \end{align*}
            
            Je-li $\omega \in \Lambda^k(\V)$ a $\tau \in \Lambda^l(\V)$, potom $\omega\land\tau = (-1)^{kl}\tau\land\omega \in \Lambda{k+l}(\V)$.
            \begin{dukazin}
                (Dokázat, že prohození je právě $k\cdot l$, následně z linearity násobení)
            \end{dukazin}
        \end{poznamka}

        \begin{veta}
            Nechť \V je vektorový prostor s bází $e_1,…e_n$. Nechť $v_1, v_2, … v_k \in \V$, kde $1≤k≤n$. Potom $v_i = \sum_{j=1}^n v_i^j e_j$ a označme $\W = \(v_i^j\)_{j = 1,…,n; i = 1,…,k}$ je matice $n \times k$ jejich souřadnice (sloupec $i$ je vektor $i$).
            Je-li $J$ k-prvková podmnožina $\{1,…,n\}$, označ $W_j:=(v^j_i)_{j \in I; i=1,…,k}$ (minor $k\times k$). Potom $v_1 \land v_2 \land … \v_k = \sum_{|J| = k}(\det(W_j))e_J$.
            \begin{dukazin}
                Posčítáním. A dokázáním, že to je definice determinantu.
            \end{dukazin}
        \end{veta}

        \begin{definice}[Skalární součin na $\Lambda*(\V)$]
            Nechť \V je vektorový prostor se skalárním součinem (?, symetrický) $<\cdot,\cdot>$ a $e_1,…,e_n$ je ortonormální báze \V.

            Definujeme skalární součin ve $\Lambda*(\V)$ jako:
            $$ \left\lbrace…\right\rbrace $$ 
            TODO!
        \end{definice}

        \begin{umluva}
            $\R^n$ chápeme jako Euklidovský prostor se standardní bází $e_1, … e_n$ a TODO!
        \end{umluva}

        \begin{priklady}
            Nechť $R$ je rovnoběžnostěn v $\R^n$ určený vektory $v_1,…,v_k$, kde $1≤k≤n$. Potom $k$-dimenzionální objem $R$ je roven:
            $$ \vol_k(R) = ||v_1\land …\land v_k||, $$
            kde $||x||$ je euklidovská norma.
            \begin{dukazin}
                TODO!
            \end{dukazin}
        \end{priklady}

        TODO TODO!

        \begin{definice}[Vektorový součin v $\R^n$]
            Nechť $v_1,…,v_{n-1} \in \R^n$. Potom jejich vektorový součin $v_1 \times v_2 \times\cdots\times v_{n-1} \in \R^n$ je definován jako $*(v_1 \times \cdots \times v_{n-1}) = v_1 \land v_2 \land … \land v_{n-1}$
            \begin{poznamkain}
                Ve skriptech označeno $\[v_1,…,v_{n-1}\]$.
            \end{poznamkain}
            \begin{poznamkain}[Platí]
                $$v_1 \times \cdots \times v_{n-1} = (-1)^{n-1}*(v_1\land … \land v_{n-1}) (z Cv.2 TODO)$$

            $$ \forall \omega \in \R^n: \<\omega, v_1 \times \cdots \times v_{n-1}\> = \det(\omega|v_1|cdots|v_{n-1}) $$
            \end{poznamkain}
        \end{definice}
    

    \subsection{Rozložitelné $k$-vektory}
        Nechť \V je vektorový prostor. Nechť $\omega \in \Lambda^k(\V)$. Položme
        $$ \ker\omega:= \{v\in\V|\omega\land v = 0\}. $$
        Platí 1. $\ker\omega$ je podprostor
        
        \begin{definice}[Rozložitelné $k$-vektory]
            $\omega \in \Lambda^k(\V)$ je rozložitelný, pokud existují $v_1,…,v_k\in\V$ takové, že $\omega = v_1\land …\land v_k$.
        \end{definice}

        Platí 2. $v_1 \land … \land v_k \neq 0 \Leftrightarrow$ vektory $v_1,…,v_k$ jsou lineárně nezávislé.

        Platí 3. Nechť $\omega = v_1\land … \land v_k \neq 0$. Potom
        $$ \ker\omega = \LO(v_1,…,v_k) $$ 

        \begin{definice}
            $$ R_k(\V) := \{\omega \in \Lambda^k(\V)| \omega \neq 0 \text{rozložitelný}\}$$
            $$G_k(\V) := \{L|L k\text{- dimenzionální podprostor \V}\} (\text{tzv. Grassmannian})$$
        \end{definice}

        Platí 4. Zobrazení $\phi: R_k(\V) \rightarrow G_k(\V)$: $\omega \rightarrow \ker\omega$ je na, ale není prosté. Skutečně máme
        $$ \ker\omega = \ker\omega' \Leftrightarrow \exists\alpha\in\R\\?: \omega' = \alpha\omega. $$

        \begin{priklady}[Nerozložitelné $k$-vektory]
            Platí 5. Pro $\V = \R^n$ jsou všechny $1$-vektory, $n$-vektory i $(n-1)$-vektory rozložitelné.
            \begin{prikladin}
                Rozložte $e_{123} + e_{124} + e_{234} \in \Lambda^3(\R^4)$, kde $e_{123} = e_{\{ 1, 2, 3 \}}$.
            \end{prikladin}

            Musíme tedy hledat v $\R^4$ a „výše“.
            \begin{prikladin}
                Najděte nerozložitelný 2-vektor $\omega \in \Lambda^2(\R^4)$
            \end{prikladin}
        \end{priklady}

        \begin{poznamka}[Projektivní prostor]
            Mezi nejdůležitější Grassmanniany patří projektivní prostor:

            Nechť \V je vektorový prostor. Polož $P(V) := \{1\text{-dimenzionální podprostor \V}\}$.

            Tvrdíme $P(\V) = G_1(\V)$.

            TODO?

        \end{poznamka}

        \begin{veta}[Plückerovo vnoření]
            $$ G_k(\V) \rightarrow ? P(\R^{(n nad k)}) $$, je-li $\dim\V = n$
        \end{veta}


\section{Variety, Stokesova věta na varietách}

    
\end{document}
