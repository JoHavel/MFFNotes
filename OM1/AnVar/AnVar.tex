\documentclass[12pt]{article}					% Začátek dokumentu
\usepackage{../../MFFStyle}					    % Import stylu



\begin{document}

\section*{Organizační úvod}
    \begin{poznamka}[Zápočet]
        Za vypracování domácích úloh.
    \end{poznamka}
    \begin{poznamka}[Zkouška]
        Písemná, ale Covid?
    \end{poznamka}
\section*{Úvod}
    MA je na rovném prostoru $\R^n$
    Naším cílem je vybudovat analýzu na nerovném? prostoru, tzv. varietě.
    \begin{poznamka}[literatura]
        Skripta -- Krump, Souček, Těšínský: MA ve varietách\\
        Sborník příkladů -- Kopáček: Příklady z matematiky pro fyziky III.
    \end{poznamka}

\section{Opakování}
\noindent 'Odvozovali' (přes limity velikosti rozdělení jdoucí k nule) jsme si:

    Křivkový integrál 1. druhu, křivkový integrál 2. druhu. Integrální věty (pol. 19. stol, moderní formulace Cardan (1945)): Věta o potenciálu, Greenova věta

    Plošný integrál 1. druhu, plošný integrál 2. druhu. Integrální věty: Stokesova věta, Gauss-Ostrogradského věta





\section{Stokesova věta v $\R^n$, diferenciální formy v $\R^n$}
\begin{veta}[Moderní (= obecná) formulace Stokesovy věty = Cíl (Cartan 1945)]
        $$ \int_{S} {d\omega} = \int_{\partial S}\omega $$
        Kde S je buď 'singulární' \k-plocha v $R^n$ (tato část) nebo \k-varieta s okrajem (3. část).
          
    \end{veta}

% 7. 10. 2020

    \subsection{Vnější algebra vektorového prostoru}
        Motivace: Jak násobit vektory z $\R^n$?

        \begin{poznamka}
                Násobení na $\R^n$ zachovává Eklidovskou normu (tzn. $||x\cdot y = ||x|| \cdot ||y||$) pouze v dimenzích 1, 2, 4, 8 (= \R, \C, kvaterniony, oktocosi).
        \end{poznamka}

        \begin{definice}[Algebra]
            Algebra nad tělesem k (= \R) je vektorový prostor $\A$ nad k s bilineárním zobrazením ….

            Algebra je asociativní, jestliže co asi.

            Algebra má jednotku, jestliže existuje co asi ;)
        \end{definice}

        \begin{definice}[]
            Nechť $\Lambda$ je vektorový prostor nad $\R$
        \end{definice}

% od 7. 10. 2020 nestihnuto spoustu věcí!

        \begin{poznamka}[Vlastnosti vnější algebry]
            $\dim \Lambda*(\V) = 2^n$, protože každý vektor je určen bázovými vektory, kterých je jako podmnožin $n$ prvkové množiny

            TODO

            \begin{align*}
                e_I \land e_J &= 0, \text{je-li} I \cap J \neq \O
                          &= sgn(\text{permutace})e_{I\cup J}, \text{je-li} I \cap J = \O
            \end{align*}
            
            Je-li $\omega \in \Lambda^k(\V)$ a $\tau \in \Lambda^l(\V)$, potom $\omega\land\tau = (-1)^{kl}\tau\land\omega \in \Lambda{k+l}(\V)$.
            \begin{dukazin}
                (Dokázat, že prohození je právě $k\cdot l$, následně z linearity násobení)
            \end{dukazin}
        \end{poznamka}

        \begin{veta}
            Nechť \V je vektorový prostor s bází $e_1,…e_n$. Nechť $v_1, v_2, … v_k \in \V$, kde $1≤k≤n$. Potom $v_i = \sum_{j=1}^n v_i^j e_j$ a označme $\W = \(v_i^j\)_{j = 1,…,n; i = 1,…,k}$ je matice $n \times k$ jejich souřadnice (sloupec $i$ je vektor $i$).
            Je-li $J$ k-prvková podmnožina $\{1,…,n\}$, označ $W_j:=(v^j_i)_{j \in I; i=1,…,k}$ (minor $k\times k$). Potom $v_1 \land v_2 \land … \v_k = \sum_{|J| = k}(\det(W_j))e_J$.
            \begin{dukazin}
                Posčítáním. A dokázáním, že to je definice determinantu.
            \end{dukazin}
        \end{veta}

        \begin{definice}[Skalární součin na $\Lambda*(\V)$]
            Nechť \V je vektorový prostor se skalárním součinem (?, symetrický) $<\cdot,\cdot>$ a $e_1,…,e_n$ je ortonormální báze \V.

            Definujeme skalární součin ve $\Lambda*(\V)$ jako:
            $$ \left\lbrace…\right\rbrace $$ 
            TODO!
        \end{definice}

        \begin{umluva}
            $\R^n$ chápeme jako Euklidovský prostor se standardní bází $e_1, … e_n$ a TODO!
        \end{umluva}

        \begin{priklady}
            Nechť $R$ je rovnoběžnostěn v $\R^n$ určený vektory $v_1,…,v_k$, kde $1≤k≤n$. Potom $k$-dimenzionální objem $R$ je roven:
            $$ \vol_k(R) = ||v_1\land …\land v_k||, $$
            kde $||x||$ je euklidovská norma.
            \begin{dukazin}
                TODO!
            \end{dukazin}
        \end{priklady}

        TODO TODO!

        \begin{definice}[Vektorový součin v $\R^n$]
            Nechť $v_1,…,v_{n-1} \in \R^n$. Potom jejich vektorový součin $v_1 \times v_2 \times\cdots\times v_{n-1} \in \R^n$ je definován jako $*(v_1 \times \cdots \times v_{n-1}) = v_1 \land v_2 \land … \land v_{n-1}$
            \begin{poznamkain}
                Ve skriptech označeno $\[v_1,…,v_{n-1}\]$.
            \end{poznamkain}
            \begin{poznamkain}[Platí]
                $$v_1 \times \cdots \times v_{n-1} = (-1)^{n-1}*(v_1\land … \land v_{n-1}) (z Cv.2 TODO)$$

            $$ \forall \omega \in \R^n: \<\omega, v_1 \times \cdots \times v_{n-1}\> = \det(\omega|v_1|cdots|v_{n-1}) $$
            \end{poznamkain}
        \end{definice}
    

    \subsection{Rozložitelné $k$-vektory}
        Nechť \V je vektorový prostor. Nechť $\omega \in \Lambda^k(\V)$. Položme
        $$ \ker\omega:= \{v\in\V|\omega\land v = 0\}. $$
        Platí 1. $\ker\omega$ je podprostor
        
        \begin{definice}[Rozložitelné $k$-vektory]
            $\omega \in \Lambda^k(\V)$ je rozložitelný, pokud existují $v_1,…,v_k\in\V$ takové, že $\omega = v_1\land …\land v_k$.
        \end{definice}

        Platí 2. $v_1 \land … \land v_k \neq 0 \Leftrightarrow$ vektory $v_1,…,v_k$ jsou lineárně nezávislé.

        Platí 3. Nechť $\omega = v_1\land … \land v_k \neq 0$. Potom
        $$ \ker\omega = \LO(v_1,…,v_k) $$ 

        \begin{definice}
            $$ R_k(\V) := \{\omega \in \Lambda^k(\V)| \omega \neq 0 \text{rozložitelný}\}$$
            $$G_k(\V) := \{L|L k\text{- dimenzionální podprostor \V}\} (\text{tzv. Grassmannian})$$
        \end{definice}

        Platí 4. Zobrazení $\phi: R_k(\V) \rightarrow G_k(\V)$: $\omega \rightarrow \ker\omega$ je na, ale není prosté. Skutečně máme
        $$ \ker\omega = \ker\omega' \Leftrightarrow \exists\alpha\in\R\\?: \omega' = \alpha\omega. $$

        \begin{priklady}[Nerozložitelné $k$-vektory]
            Platí 5. Pro $\V = \R^n$ jsou všechny $1$-vektory, $n$-vektory i $(n-1)$-vektory rozložitelné.
            \begin{prikladin}
                Rozložte $e_{123} + e_{124} + e_{234} \in \Lambda^3(\R^4)$, kde $e_{123} = e_{\{ 1, 2, 3 \}}$.
            \end{prikladin}

            Musíme tedy hledat v $\R^4$ a „výše“.
            \begin{prikladin}
                Najděte nerozložitelný 2-vektor $\omega \in \Lambda^2(\R^4)$
            \end{prikladin}
        \end{priklady}

        \begin{poznamka}[Projektivní prostor]
            Mezi nejdůležitější Grassmanniany patří projektivní prostor:

            Nechť \V je vektorový prostor. Polož $P(V) := \{1\text{-dimenzionální podprostor \V}\}$.

            Tvrdíme $P(\V) = G_1(\V)$.

            TODO?

        \end{poznamka}

        \begin{veta}[Plückerovo vnoření]
            $$ G_k(\V) \rightarrow ? P(\R^{(n nad k)}) $$, je-li $\dim\V = n$
        \end{veta}

% 14. 10. 2020

    \subsection{Diferenciální formy}
        $$x \in ®R^n = \(x_1, …, x_n\)$$
        Označme $T*(®R^n)$ reálný vektorový prostor, jehož bázi tvoří symboly $dx_1, …, dx_n$ tj.
        $$ T*(®R^4) = \{\sum_{i=1}^n\alpha_i\,dx_i|\alpha_i \in ®R\} $$ 
        \begin{definice}[Diferenciální fomrma]
            Diferenciální forma $\omega$ na otevřené množině $\Omega \subset ®R^n$ je zobrazení $\omega: \Omega \rightarrow \Lambda*(T*(®R^n))$ třídy $©S^∞$ (= je hladké).

            Označme $©E*(\Omega)$ vektorový prostor všech diferenciálních forem na $\Omega$. Každé $a \in ©E*(\Omega)$ laze jednoznačně psát jako
            \begin{equation}
                \omega(x) = \sum_{I}\omega_I(x)\,dx_I, 
            \end{equation}
            kde součet je přes všechny $I\subset\{1, …, n\}$, $\omega_I \in ©S^∞(\Omega)$ a $dx_I = dx_{i_1}\lor … \lor dx_{i_k}$ jsou-li prvky $i_1, …, i_k$ množiny $I$ uspořádány postupně
        \end{definice}

        \begin{definice}[Stupeň diferenciální formy]
            Dále $\omega \in ©E*(\Omega)$ má stupeň $k$ (tzv. $k$-forma), pokud $\omega : \Omega \rightarrow \Lambda^k(T*(®R^n))$ je hladké zobrazení. Označme $©E^k(\Lambda)$ vektorový prostor všech $k$-forem na $\Omega$.
            \begin{poznamkain}
                Každá $\omega \in ©E^k(\Omega)$ má tvar (1), kde je součet přes všechny $|I| = k$.

                Zřejmě $©E*(\Omega) = \bigoplus_{k=0}^n ©E^k(\Omega)$ a $©E^0(\Omega) = ©S^∞(\Omega)$.
            \end{poznamkain}
        \end{definice}

        \begin{definice}[Vnější násobení]
            Na $©E*(\Omega)$ definujeme vnější násobení
            $$(\omega \lor \tau)(x):= \omega(x)\lor \tau(x), x\in \Omega, \omega, \tau \in ©E*(\Omega).$$
        \end{definice}

        \begin{definice}[Vnější (de Rhammův) diferenciál]
            Nechť $\Omega \subset ®R^n$ je otevřená. Potom definujeme zobrazení $d: ©E*(\Omega) \rightarrow ©E(\Omega)$ následovně:
            (i)Je-li $f \in ©E^0(\Omega)$, potom
            $$ (df)(x) := \sum_{i=1}^{n} \frac{\partial f}{\partial x_i}(x)\,dx_i, x \in \Omega $$

            (ii) Nechť $\omega \in ©E*(\Omega)$ je tvaru (1). Potom
            $$ dw := \sum_{I} (d\omega_I)\lor dx_I $$

            \begin{prikladyin}
                $$ \omega = e^{xy}\,dx + \cos(x+y)\,dy $$
                $$ d(e^{xy}) = e^{xy}y\,dx + e^{xy}x\,dy $$ 
                $$ (\cos(x+y)) = -\sin(x+y)\,dx - \sin(x+y)\,dy $$ 
                $$ d\omega = e^{xy}x \,dy\lor dx - sin(x+y)dx \lor dy = -(xe^{xy} + \sin(x+y))\,dx\lor dy $$ 
            \end{prikladyin}
        \end{definice}

        \begin{poznamka}
            Nechť $\phi_i: ®R^n \rightarrow ®R$ je $i$-tá souřadnice funkce, tzn. $\phi_i(x) := x_i$, $x=(x_1, …, x_n)\in ®R^n$. Potom
            $$ d\phi_i = \sum_{j=1}^n \frac{\partial \phi_i}{\partial x_j}\, dx_j = dx_i. $$
        \end{poznamka}

        \begin{poznamka}
            V „rovném“ prostoru $®R^n$:
            \begin{itemize}
                \item tečný prostor $T_x(®R^n) \simeq ®R^n$.
                \item kotečný prostor $T_x*(®R^n) := (T_x(®R^n))* \simeq (®R^n)*$
            \end{itemize}
        \end{poznamka}

        \begin{veta}
            Nechť $\Omega \subset ®R^n$ je otevřená, $\omega, \tau \in ©E*(\Omega)$ a $p = 0, …, n$. Potom platí
            \begin{itemize}
                \item[(i)] $d(\omega + \tau) = d\omega + d\tau$ a $\forall \omega \in ©E^p(\Omega): d\omega \in ©E^{p+1}(\Omega)$, kde $©E^{n+1}(\Omega) := \O$.
                \item[(ii)] Je-li $\omega \in ©E^p(\Omega)$, potom $d(\omega \lor \tau) = d\omega \lor \tau + (-1)^p\omega\lor d\tau$.
                \item[(iii)] $d(d\omega) = 0$.
            \end{itemize}
            
            \begin{dukazin}
                \begin{itemize}
                    \item[(i)] plyne z linearity $\lor$ a definice.
                    \item[(ii)] Vzhledem k (i) stačí dokázat pro $\omega = \omega_I\,dx_I$ a $\tau = \tau_J\,dx_J$, kde $I, J \subset \{1, …, n\}$, $I$ je $p$-prvková a $I \cap J = \O$.

                            Potom $d(\omega \lor \tau) = d(\omega_I\tau_J)\lor dx_I \lor dx_J.$ Dále $d(\omega_I\tau_J) = \sum_{i=1}^n \frac{\partial(\omega_I\tau_J)}{\partial x_i}\,dx_i = \sum_{i=1}^n\(\frac{\partial\omega_I}{\partial x_i}\tau_J + \omega_I\frac{\partial \tau_J}{}\partial x_i\)\,dx_i$.

                            Tedy $d(\omega \lor \tau) = \sum_{i=1}^n\frac{\partial \omega_I}{\partial x_i}\tau_J\,dx_i\lor dx_I \lor dx_J + \sum_{i=1}^n\omega_I\frac{\partial \tau_J}{\partial x_i}\,dx_i\lor dx_I \lor dx_J$, kde musím v druhém členu posunout „$d\tau$“, k jeho $dx_J$

                    \item[(iii)] Pro $f\in ©E^0(\Omega)$ si to roznásobím a popáruji prohozené bázové vektory.

                            Díky (i) stačí rozbrat pro $\omega = \omega_I\,dx_I$, kde $I \subset\{1, …, n\}$. Potom $d(d\omega) = d(\omega_I\,dx_I)$, (dvojkou rozepíšu) a z první části a $d1 = 0$ je to rovno 0.
                \end{itemize}
            \end{dukazin}
        \end{veta}

    \subsection{Přenášení diferenciálních forem pomocí zobrazení}
        $x = (x_1, …, x_n) \in ®R^n$ a $u = (u_1, …, u_k) \in ®R^k$. V této části předpokládejme, že $\Omega \subset ®R^n$ je otevřená, $U \subset ®R^k$ je otevřená a $\phi: U \rightarrow \Omega$ je hladké zobrazení.

        Je tedy $x = \phi(u), u \in U$ a $x_i = \phi_i(u_1, …, u_k)$, kde $\phi_i$ je $i$-tá složka $\phi$.

        \begin{definice}
            Za předpokladů výše definujeme zobrazení $\phi*:©E*(\Omega) \rightarrow ©E*(U)$ předpisem $\phi*(\omega) := \sum_I (\omega_I \circ \phi)\,d\phi_I$, kde $\omega = \sum_I \omega_I\,dx_I$ je tvaru (1) a $d\phi_I = d\phi_{i_1}\lor … \lor d\phi_{i_k}$, jsou-li prvky $i_1, …, i_k$ uvnitř $I$ uspořádány vzestupně.

            \begin{poznamkain}
                V souladu s definicí plošného integrálu 2. druhu.
            \end{poznamkain}
        \end{definice}

        \begin{veta}
            Nechť $\phi$ je jako výše a $\omega, \tau \in ©E*(\Omega)$. Potom:
            \begin{itemize}
                \item[(i)] $\phi*(\omega + \tau) = \phi*(\omega) + \phi*(\tau)$,
                \item[(ii)] $\phi*(\omega \lor \tau) = \phi*(\omega) \lor \phi*(\tau)$,
                \item[(iii)] $\phi*(d\omega) = d(\phi*(\omega))$.
                \item[(iv)] Je-li $V \subset ®R^l$ otevřená a $\psi: V \rightarrow U$ je hladká, potom $(\phi\circ\psi)*(\omega) = (\psi*\circ\phi*)(\omega)$.
                \item[v] Je-li $k=n$, $\omega \in ©E^n(\Omega)$ a $\omega = f\,dx_1\lor …\lor dx_n$, potom $\phi*(\omega) = (f\circ \phi) \det(\Jac\phi)\, du_1 \lor … \lor du_n$, kde $\Jac\phi=\(\frac{\partial\phi_i}{\partial u_j}\)_{i, j = 1,…,n}$ je Jacobiho matice $x = \phi(u)$.
            \end{itemize}

            \begin{dukazin}
                Jednoduchý.
            \end{dukazin}
        \end{veta}

        \begin{definice}[Uzavřené a exaktní formy]
            Formule $\omega \in ©E^k(\Omega)$ se nazývá uzavřená, je-li $d\omega = 0$ a exaktní, existuje-li $\tau \in ©E^{k-1}(\Omega)$ takové, že $d\tau = \omega$.
        \end{definice}

        \begin{poznamka}[Platí]
            Je-li $\omega$ exaktní, potom je uzavřená.
        \end{poznamka}

        \begin{lemma}[Poincarého lemma]
            Nechť $\Omega$ je otevřená koule v $®R^n$. Potom pro $k>0$ každé $\omega \in ©E^k(\Omega)$, která je uzavřená, je i exaktní.
            
            \begin{poznamkain}
                    Platí i pro hvězdovité (znáte z analýzy) nebo jednoduše souvislé (dá se stáhnout do bodu) oblasti $\Omega \subset ®R^n$
            \end{poznamkain}
        \end{lemma}

        \begin{poznamka}[Poncarého lemma platí pouze pro dané oblasti]
            Nechť $\Omega := ®R^2 \setminus \{(0, 0)\}$. Potom $\omega := \frac{x}{x^2 + y^2}\,dy - \frac{y}{x^2 + y^2}\,dx \in ©E^1(\Omega)$ je uzavřená, ale není exaktní.
        \end{poznamka}

        \begin{definice}[De Rhanův komplex]
            Nechť $\Omega \subset ®R^n$ je otevřená, potom
            $$ ©E^0(\Omega) \overset{d}{\rightarrow} ©E^1(\Omega) \overset{d}{\rightarrow} … \overset{d}{\rightarrow} ©E^n(\Omega) $$
            je komplex (tzn. posloupnost vektorových prostorů a lineární zobrazení mezi nimi s vlastností, že každá složka dvou po sobě jdoucích zobrazení je triviální (zde, $d\circ d = 0$, splněno))
        \end{definice}

        TODO!

% 21. 10. 2020
    
    \subsection{Stokesova věta pro řetězce}
        \begin{poznamka}[Cíl]
            $$ \int_{C\text{ $k$-dim. řet. v $®R^n$}} d\omega = \int_{\partial C\text{ $(k-1)$-dim.}} \omega $$ 
        \end{poznamka}

        \begin{definice}
                Nechť $E \in ®R^k$ (je libovolná). Potom zobrazení $\phi: E \rightarrow ®R^k$ nazvěme hladké, pokud existuje otevřená $O \subset ®R^k$ a $\Phi \rightarrow R®k$ hladké zobrazení takové, že $E \subset O$ a $\phi = \Phi|_E$. Navíc $\Phi$ nazveme hladkým rozšířením $\phi$. ($\leftarrow$ Whitneyho rozšiřovací věta.)
        \end{definice}

        \begin{definice}
            Nechť $I_k = \[0, 1\]^k \subset ®R^k$. Potom k-dimenzionální singulární krychle v $®R^n$ rozumíme hladké zobrazení $I_k \rightarrow ®R^n$. Píšeme $\<\phi\> = \phi(I_k)$.
        \end{definice}

        \begin{poznamka}
            $\<\phi\>$ je 'hladká deformace' $k$-dimenzionální krychle, může být singulární, např. bod (je-li $\phi$ konet).
        \end{poznamka}

        \begin{definice}
            Nechť $\Omega \subset ®R^n$ je otevřená.

            (i) Nechť $\omega \in É^n(\Omega)$ a $\omega = f\, dx_1 \land … \land dx_n$, kde $f \in Č^∞(\Omega)$. Je-li $E \subset \Omega$, potom definujeme
            $$ \int_E \omega = \int_E fd\lambda^n, $$
            pokud int. vpravo existuje jako Lebesgueův vůči Lebesgueově míře $\lambda^n$ na $®R^n$.

            Pro $n=0$ definujeme $\int f = 0$

            (ii) Nechť $k = 0, …, n$ a $\omega \in ©E^k(\Omega)$. Nechť $\phi$ je $k$-dimenzionální supul? krychle v $\Omega$ (tzn. $\<\phi\> \in \Omega$).

            Položme $\int_\phi \omega := \int_{I_k}\Phi*(\omega)$, je-li $\Phi: O \rightarrow \Omega$ hladké rozšíření $\phi$.
        \end{definice}

        \begin{poznamka}
            Definice (ii) je v pořádku, protože takové $\Phi$ vždy existuje (jinak $\phi|_{\sigma_n\Phi_{-1}(\Omega)}$) a hodnota $\int_\phi \omega$ nezávisí na hladkém rozšíření $\phi$. Skutečně pro jiné takové hladké rozšíření $\Phi$ mějmé?, že
            $$ \Phi = \phi = \psi \text{na} I_k^0 $$ 
            $$ \Phi*(\omega) = \psi*(\omega) \text{na} I^0_k $$
            $$ \Phi*(\omega) = \psi*(\omega) \text{na} I_k \text{ze spojitosti funkcí $\Phi, \psi$ a jejich 1. parciálních derivací.} $$ 
        \end{poznamka}

        \begin{umluva}
            Často budeme ztotožňovat $\phi$ s $\Phi$.
        \end{umluva}

        \begin{veta}[Integrál nezávisí na parametrizaci, jen na orientaci]
            Nechť $\Omega \subset ®R^n$ je otevřená a $\omega \in ©E^k(\Omega)$. Nechť $I_k \subset O, O' \subset ®R^n$ jsou otevřené a $\alpha: O' \rightarrow O$ (na) je hladký difomorfismus (tzn. $\alpha$ i $\alpha_{-1}$ jsou hladká zobrazení), $\alpha(I_k) = I_k$. Nechť $\phi: O \rightarrow \Omega$ je hladké a $\phi' := \phi\circ\alpha$.

            Potom $\int_{\phi'}\omega = \Theta \int_{\phi}\omega$, kde $\Theta = +1$, je-li $J_\alpha := \det(\Jac(\alpha)) > 0 \text{na} I_k$, $\Theta = -1$, je-li $J_\alpha := \det(\Jac(\alpha)) < 0 \text{na} I_k$.

            \begin{dukazin}
                    Víme, že $J_\alpha ≠ 0$ na $O'$. Tedy $J_\alpha$ (spojité funkce) nemění na $I_k$ znaménko.
                    TODO!
            \end{dukazin}
        \end{veta}

        TODO!

        \begin{veta}[Stokes]
            Nechť $\Omega \subset ®R^n$

            \begin{dukaz}
                Nechť $k = n$ a $C = I_n$. Potom $\omega \in ©E^{n-1}(\Omega)$ má tvar $\omega = \sum_{i=1}^n\omega_i$, kde $\omega_i = (-1)^{i+1} f_i dx_1 \land … \land dx_{i-1} \land dx_{i+1} \land … \land dx_n$ a $f_i\in ©C^∞(\Omega)$.

                Potom $d\omega_i = \frac{\partial f_i}{\partial x_i} dx_1 \land … \,dx_n$ a
                $$\int_{I_n} d\omega_i = \int_{\[0, 1\]^n}\frac{\partial f_i}{\partial x_i}\, dx_1 … dx_n \overset{\text{Fubki (věta) + newtonův vzorec v $x_i$}}{=}$$
                $$ = \int_{\[0, 1\]^{n-1}} (f_i(x_1, …, x_{i-1}, 1, x_{i+1}, …, x_n) - f_i(x_1, …, x_{i-1}, 0, x_{i+1}, …, x_n))\, dx_1 … dx_{i-1} dx_{i+1} … dx_n = $$
                $$ (-1)^{i+1}(\int_{I^n_{(i, 1)}}\omega_i - \int_{I^n_{(i, 0)}}\omega_i) = \int_{\partial I_n}\omega_i, $$ 
                protože $\int_{I^n_{j, \alpha}} \omega_i = 0$ pro $j≠i$. Tedy $\int_{I_n} d\omega = \int_{\partial I_n} \omega$.


                (2.) Nechť $c = cosi$. Potom cosi
            \end{dukaz}
        \end{veta}



        \begin{priklad}[Singulární homologie $\Omega$]
            Nechť $\Omega \subset ®R^n$ je otevřená. Dokažte, že pro každý $k$-řetězec $c \in C_k(\Omega)$ je $\partial c \in C_{k-1}(\Omega)$ a $\partial(\partial c) = 0$.

            \begin{dukaz}
                Nahlédneme, že každá část potenciální hranice se jednou „přičte“ a jednou „odečte“.
            \end{dukaz}
        \end{priklad}

        \begin{veta}[De Rhamova (hluboká!)]
            Máme tedy $C_0(\Omega) \overset{\partial}{\leftarrow} C_1(\Omega) \overset{\partial}{\leftarrow} \cdots \overset{\partial}{\leftarrow} C_n(\Omega)$.

            Označme
            $$ Z_k(\Omega) := \{c \in C_k(\Omega) | \partial c = 0\} \text{tzv. k-cykly} $$
        \end{veta}

        \begin{poznamka}[Ze cvičení]
            Nechť $S$ je libovolná množina (i nekonečná). Potom volnou Abelovou grupou $®Z(S)$ generovanou $S$ rozumíme grupu
            $$ ®Z(S) := \{f': S\rightarrow ®Z| f(s)≠0 \text{pro konečně mnoho $s\in S$}\} $$
            s operací $(f + g)(s) := f(s) + g(s)$, $s \in S$.

            Zřejmě každá $f \in ®Z(S)$ lze jednoznačně psát jako $f = \sum_{s \in S}n_s z_s$, kde $n_s \in ®Z$, $n_s ≠ 0$ pro konečně $s\in S$ a $z_s(t):=1, t=s$ $z_s(t):= 0, t≠s$. Píšeme často $s$ místo $z_s$
        \end{poznamka}

        \begin{poznamka}[Ze cvičení]
            Nechť $S$ je libovolná množina. Položme $®R(S) := \{f: S \rightarrow ®R| f(s) ≠ 0 \text{pro konečně $s\in S$}\}$. Potom $®R(S)$ je vektorový prostor nad ®R s operacemi $(f+g)(s) := f(s) + g(s)$ a $(r·f)(s) := r·f(s), f, g \in ®R(S), s\in S \text{a} r\in ®R$.

            Dále $®R(S)$ má bázi $\{z_s | s \in S\}$, kde $z_s(t) := 1, t=s; z_s(t) ;= 0, t≠s$. 
        \end{poznamka}

\section{Variety, Stokesova věta na varietách}

% 4. 11. 2020

    \subsection{Tenzory}
        \begin{umluva}
            Všechny vektorové prostory budou nad reálnými čísly a konečnědimenzionální.
        \end{umluva}

        \begin{definice}[]
            Nechť ¦V je vektorový prostor.

            Jeho $k$-tou tenzorovou mocninou $¦V^k$ definujeme jako $¦V^k:=©L(¦V*, …, ¦V*)$, kde položíme $¦V^0 = ®R$ a ztotožňujeme $¦V^1 = ¦V**$ s ¦V.

            Jeho tenzorovou algebru definujeme jako $T(¦V) := \bigoplus_{k=0}^∞ ¦V^k$. Násobení $\otimes$ na $T(¦V)$ definujeme následovně:

            a) Je-li $\alpha \in ¦V^k$ a $\beta \in ¦V^m$, TODO

            b) násobení $\otimes$ rozšíříme na $T(¦V)$ bilineárně TODO
        \end{definice}

        \begin{tvrzeni}[Vlastnosti $T(¦V)$]
            Je to $∞$-dimenzionální, nekomutativní, asociativní algebra s jednotkou $1\in ¦V^0 = ®R$.

            Nechť ¦V má bázi $e_1, …, e_n$. Pottom $¦V^k$ má bázi $e_A:= e_{a_1} \otimes … \otimes e_{a_k}$, kde $A = \(a_1, …, a_k\) \in \{1, 2, …, n\}^k$. Speciálně $\dim ¦V^k = (\dim ¦V)^k = n^k$.

            \begin{dukazin}
                Triviální.

                Nechť $\epsilon^1, …, \epsilon^n$ je duální báze $¦V*$, tzn. $\epsilon^i(e_j) = \delta_j^i = 1, i = j; = 0, i≠j$.

                Nechť $\sum_A \alpha^Ae_A = ¦o$ s $\alpha^A \in ®R$. Potom $¦o = \sum_A \alpha_A (\epsilon^{b_1}, …, \epsilon^{b_k}) = \alpha_B$ TODO.
            \end{dukazin}
        \end{tvrzeni}

        \begin{umluva}[Einsteinova sumační konvence]
            V tenzorovém počtu vynecháváme symbol $\sum$ pro každý index od 1 do $n$, který je „nahoře i dole“. Ale příliš ji nebudeme používat.
        \end{umluva}

        \begin{tvrzeni}[Změna souřadnic tenzoru při změně báze]
            Nechť $e_1', …, e_n'$ je jiná báze ¦V a nechť $E = (E_b^a)$ je matice přechodu od $e_1, …, e_n$ k $e_1', …, e_n'$, tzn.
            $$ e_b' = E_b^a e_a. $$ 

            \begin{dukazin}
                Dosadíme (1) a (2) do
                $$ \alpha'_{b_1, …, b_r}^{a_1, …, a_s} = \alpha(\epsilon'^{a_1}, …, \epsilon'^{a_s}, e_{b_1}', …, e_{b_r}') $$ 
            \end{dukazin}
        \end{tvrzeni}

        \begin{definice}[Symetrická a vnější algebra]
            Nechť ¦V je vektorový prostor. Nechť $S_k$ je grupa permutací $\{1, 2, …, k\}$.

            Potom $\alpha \in ¦V^k$ je symetrický, resp. antisymetrický, pokud $\forall f^1, …, f^k \in ¦V* \forall \pi \in S_k: \alpha(f^{\pi(1)}, …, f^{\pi(k)}) = \alpha(f^1, …, f^k)$ (resp. přenásobené $\sgn\pi$). Označme $Sym_k$ TODO.
        \end{definice}

        \begin{definice}[Symetrická algebra]
            Symetrickou algebrou vektorového prostoru ¦V rozumíme algebru $\Sym(¦V) = \bigoplus_{k=0}^∞ \Sym^k(¦V)$ s násobením definovaným následovně:

            Je … TODO
        \end{definice}

        \begin{definice}[Vnější algebra podruhé]
            TODO
        \end{definice}

% 11. 11. 2020

    \subsection{Topologické prostory}
        Viz Topologie: Definice topologie, topologie generovaná metrikou, uzavřená množina, okolí, vnitřek, uzávěr, hranice, topologický podprostor, Hausdorffův prostor (a to, že metrický prostor je Hausdorffův), indiskrétní $\tau_0$ a diskrétní $\tau_1$ topologie, $\tau_0$ není pro více jak jednoprvkovou množinu hausdorffova, báze, spojité zobrazení, homeomorfismus, kompaktnost, souvislost, spojitost funkce v bodě, spojitý obraz kompaktního (resp. souvislý) prostoru je kompaktní (resp. souvislý)

    \subsection{Variety}
        \begin{definice}[Varieta]
            TODO jestliže
            \begin{enumerate}
                \item ®X je lokálně homeomorfní s $®R^n$, tzn. pro každé $x \in ®X$ existuje otevřená $x \in U \subset ®X$ a homeomorfismus $\phi$ množiny $U$ na $\phi(U) \subset ®R^n$
                \item ®X je Hausdorffův a
                \item ®X má spočetnou bázi a TODO.
            \end{enumerate}
        \end{definice}

    \begin{priklady}
        
        Tvorba základních dvourozměrných variet: viz témátko Topologie v M\&M.
    \end{priklady}

    \begin{definice}[Mapa]
        Nechť $(®X, \tau)$ je topologická varieta dimenze $n$. Potom mapou (lokálním souřadnicovým systémem) na $(®X, \tau)$ nazveme $(U, \phi)$, kde $U \subset ®X$ je otevřené a $\phi$ homeomorfizmus? TODO.

        Nechť $(V, \psi)$ je jiná mapa na ®X. Potom buď $U \cap V = \O$, nebo $U \cap V ≠ \O$ a tzv. přechodové funkce $\psi \circ \phi^{-1}$ TODO
    \end{definice}

    \begin{definice}[Atlas]
        Systém map $©A = \{\}$ TODO
    \end{definice}

    TODO

    \begin{poznamka}
        Každý topologický prostor ®X, který se dá pokrýt spočetně mnoha mapami, má spočetnou bázi otevřených množin.

        Obecněji: Nechť $(®X, ©A)$ je varieta dimenze $n$. Je-li $Y \subset ®X$ otevřená, potom
        $$ ©A_Y := \{(U \cap Y, \phi|_{U \cap Y})| (U, \phi) \in ©A\} $$
        je atlas na $Y$ a $(Y, ©A_Y)$ je varieta dimenze $n$.
    \end{poznamka}

    \begin{priklad}
        Rozmysli si, že 2rozměrné plochy v $®R^3$ zavedené a studované v přednášce z Geometrie v LS jsou příklady hladkých variet dimenze 2. (Ale s trochu jiným pojmem mapy.)
    \end{priklad}

% 18. 11. 2020

% 25. 11. 2020
    
\end{document}
