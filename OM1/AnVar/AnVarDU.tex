\documentclass[12pt]{article}					% Začátek dokumentu
\usepackage{../../MFFStyle}					    % Import stylu

\setlength{\parskip}{0em}

\begin{document}

    \begin{priklad}[1.]
        Ověřte přímým výpočtem Stokesovu větu pro singulární krychli $c$ a formu $\omega$:
        $$ c: \begin{cases} x = r·s \\ y = s·t \\ z = r·t \\ w = r + s + t \end{cases}, \hskip 2em (r, s, t) \in [0, 1]^3, \hskip 2em \omega = w \, dx \wedge dz + z \, dy \wedge dw. $$ 

        \begin{reseni}
            Stokesova věta pro singulární krychli $c$ a formu $\omega$ vypadá následovně
            $$ \int_{\partial c}\omega = \int_{c} d\omega. $$
            Tudíž chceme dokázat, že integrál vlevo se rovná integrálu vpravo. Nejdřív si zvolíme pořadí souřadnic, tj. např. $r, s, t$ a začneme integrálem vlevo:

            Chceme spočítat parametrizaci $\partial c$, a jelikož máme trojrozměrnou parametrizaci $c$, $\partial c$ je
            $$ \partial c = - c_{r, s, 0} + c_{r, s, 1} + c_{r, 0, t} - c_{r, 1, t} - c_{0, s, t} + c_{1, s, t} $$
            $$ \int_{\partial c}d\omega = \int_{- c_{r, s, 0}}d\omega + \int_{c_{r, s, 1}}d\omega + \int_{c_{r, 0, t}}d\omega + \int_{- c_{r, 1, t}}d\omega + \int_{- c_{0, s, t}}d\omega + \int_{c_{1, s, t}}d\omega $$
            $$ \int_{\partial c}d\omega = - \int_{c_{r, s, 0}}d\omega + \int_{c_{r, s, 1}}d\omega + \int_{c_{r, 0, t}}d\omega - \int_{c_{r, 1, t}}d\omega - \int_{c_{0, s, t}}d\omega + \int_{c_{1, s, t}}d\omega $$
            Tedy najdeme parametrizace jednotlivých 'stěn' krychle a následně dosadíme do příslušných integrálů (viz další strana) a vypočítáme
            $$ \int_{\partial c}d\omega = - 0 + \(- \frac{19}{12}\) + 0 - \frac{10}{12} - 0 + \frac{25}{12} = -\frac{4}{12} = -\frac{1}{3}. $$

            Nyní pravý integrál: stačí najít $d\omega$ a poté dosadit do integrálu:
            $$ d\omega = dw \wedge dx \wedge dz + dz \wedge dy \wedge dw, $$
            $$ dx = r\,ds + s\,dr, \hskip 2em dy = s\,dt + t\,ds, \hskip 2em dz = r\,dt + t\,dr, \hskip 2em dw = dr + ds + dt, $$
            $$ d\omega = (dr + ds + dt)·(r\,ds + s\,dr)·(r\,dt + t\,dr) + (r\,dt + t\,dr)·(s\,dt + t\,ds)·(dr + ds + dt) = $$
            $$ = \(r^2 - s·r - r·t - r·t - t·s + t^2\)\, dr \wedge ds \wedge dt, $$ 
            $$ \int_{c}d\omega = \int_{[0, 1]^3}\(r^2 - s·r - r·t - r·t - t·s + t^2\)\, dr \wedge ds \wedge dt = $$
            $$ = \frac{1}{3} - \frac{1}{4} - \frac{1}{4} - \frac{1}{4} - \frac{1}{4} + \frac{1}{3} = -\frac{1}{3} $$

            $-\frac{1}{3} = -\frac{1}{3}$, tedy Stokesova věta pro tuto krychli $c$ a tuto formu $\omega$ platí.    
        \end{reseni}

        \begin{reseni}[Výpočty levého integrálu]
            $$ c_{r, s, 0} = \begin{cases} x = r·s \\ y = 0 \\ z = 0 \\ w = r + s + 0 \end{cases}, \hskip 2em
            c_{r, s, 1} = \begin{cases} x = r·s & dx = r\,ds + s\,dr \\ y = s & dy = ds \\ z = r & dz = dr \\ w = r + s + 1 & dw = dr + ds \end{cases} $$
            $$ \int_{c_{r, s, 0}}\omega = \int_{c_{r, s, 0}}… …\wedge 0 + 0\,0\wedge… = 0 $$ 
            $$ \int_{c_{r, s, 1}}\omega = \int_{c_{r, s, 1}}(r+s+1)·(-r) + r·(-1)\, dr\wedge ds = -\frac{1}{3} - \frac{1}{4} - \frac{1}{2} - \frac{1}{2} = -\frac{19}{12} $$ 

            $$ c_{r, 0, t} = \begin{cases} x = 0 \\ y = 0 \\ z = r·t \\ w = r + 0 + t \end{cases}, \hskip 2em
            c_{r, 1, t} = \begin{cases} x = r & dx = dr \\ y = t & dy = dt \\ z = r·t & dz = r\,dt + t\,dr \\ w = r + 1 + t & dw = dr + dt \end{cases} $$
            $$ \int_{c_{r, 0, t}}\omega = \int_{c_{r, 0, t}}… 0\wedge … + … 0\wedge… = 0 $$ 
            $$ \int_{c_{r, 1, t}}\omega = \int_{c_{r, 1, t}}(r+1+t)·(r) + r·t·(-1)\, dr\wedge dt = \frac{1}{3} + \frac{1}{2} + \frac{1}{4} - \frac{1}{4} = \frac{5}{6} = \frac{10}{12} $$

            $$ c_{0, s, t} = \begin{cases} x = 0 \\ y = s·t \\ z = 0 \\ w = 0 + s + t \end{cases}, \hskip 2em
            c_{1, s, t} = \begin{cases} x = s & dx = ds \\ y = s·t & dy = s\,dt + t\,ds \\ z = t & dz = dt \\ w = 1 + s + t & dw = ds + dt \end{cases} $$
            $$ \int_{c_{0, s, t}}\omega = \int_{c_{0, s, t}}… 0\wedge 0 + 0 \wedge… = 0 $$ 
            $$ \int_{c_{1, s, t}}\omega = \int_{c_{1, s, t}}(1+s+t)·(1) + t·(t-s)\, ds\wedge dt = 1 + \frac{1}{2} + \frac{1}{2} + \frac{1}{3} - \frac{1}{4} = \frac{25}{12} $$
        \end{reseni}
    \end{priklad}

    \pagebreak

    \begin{priklad}[2.]
        Definujte mapy na Grassmanniánu
        $$ Gr_{k, n} := \{L \subseteq ®R^n \middle| L \text{ podprostor dimenze } k \} $$
        a pro $G_{2, 3}$ ukažte, že přechodové funkce jsou difeomorfismy. Ukažte, že
        $$ \dim Gr_{k, n} = k·(n-k). $$

        \vskip -1.5em \begin{reseni}[Definování map]
            Každé $L \in Gr_{k, n}$ má nějakou bázi $B = \(¦b_1, ¦b_2, …, ¦b_k\)$. Jelikož jsou vektory báze lineárně nezávislé, existuje posloupnost $l = \(l_i\)_{i=1}^k$ přirozených čísel $1 ≤ l_1 < l_2 < … < l_k ≤ n$ tak, že $B$ lze lineárními kombinacemi převést na $B'$ (bázi, jelikož má $k$ zjevně nezávislých vektorů)
            $$ B' = \(¦b'_1, ¦b'_2, …, ¦b'_k\) = \(\begin{pmatrix} c_{1, 1} \\ \vdots \\ c_{l_1-1, 1} \\ 1 \\ c_{l_1, 1} \\ \vdots \\ c_{l_2 - 2, 1} \\ 0 \\ c_{l_2 - 1, 1} \\ \vdots \\ c_{l_k - k, 1} \\ 0 \\ c_{l_k - k + 1, 1} \\ \vdots \\ c_{n-k, 1} \end{pmatrix}, \begin{pmatrix} c_{1, 2} \\ \vdots \\ c_{l_1-1, 2} \\ 0 \\ c_{l_1, 2} \\ \vdots \\ c_{l_2 - 2, 2} \\ 1 \\ c_{l_2 - 1, 2} \\ \vdots \\ c_{l_k - k, 2} \\ 0 \\ c_{l_k - k + 1, 2} \\ \vdots \\ c_{n-k, 2} \end{pmatrix}, …, \begin{pmatrix} c_{1, k} \\ \vdots \\ c_{l_1-1, k} \\ 0 \\ c_{l_1, k} \\ \vdots \\ c_{l_2 - 2, k} \\ 0 \\ c_{l_2 - 1, k} \\ \vdots \\ c_{l_k - k, k} \\ 1 \\ c_{l_k - k + 1, k} \\ \vdots \\ c_{n-k, k} \end{pmatrix}\) \text{, řádky } \begin{array}{c} 1 \\ \vdots \\ l_1-1 \\ l_1 \\ l_1 + 1 \\ \vdots \\ l_2-1 \\ l_2 \\ l_2 + 1 \\ \vdots \\ l_k - 1 \\ l_k \\ l_k + 1 \\ \vdots \\ n  \end{array} $$

            \vskip -1em (tedy, že na každém řádku $l_i$ je ve všech vektorech 0 kromě vektoru $¦b'_i$, kde je v tomto řádku 1, zbytek míst je vyplněn $c_{\alpha, \beta}$ pro všechna $\alpha = 1, 2, …, n-k$ a $\beta = 1, 2, …, k$).

            Pro pevnou posloupnost $l$ a každé $L$, které lze vyjádřit v tomto tvaru při této volbě $l$ (označme množinu těchto $L$ jako $©L_l$), jsou čísla $c_{\alpha, \beta}$ určena jednoznačně, jelikož kdybychom našli jinou takovou bázi $B'' = (¦b''_1, …, ¦b''_k)$, která splňuje 'polohu' jedniček a nul, tak každý její prvek musí jít vyjádřit jako lineární kombinace prvků $B'$, ale to lze zjevně jen nějako $¦b''_i = 1·¦b'_i$ právě díky 'poloze' jedniček a nul. Tudíž můžeme definovat zobrazení $\phi_l: ©L_l \rightarrow ®R^{k·(n-k)}$:
            $$ \phi_l(L) = \(c_{1, 1}, c_{1, 2}, …, c_{1, k}, c_{2, 1}, …, c_{2, k}, …, c_{n-k, 1}, …,  c_{n-k, k}\)^T. $$ 
            Mapy tedy budou $\(©L_l, \phi_l\)$ a atlas $\{\(©L_l, \phi_l\) | l \text{ jako v prvním odstavci}\}$. (Zde by bylo ještě nutné ověřit, že obrazy průniků $©L$ jsou otevřené a že přechodové funkce jsou difeomorfismy, aby byly mapy kompatibilní, ale jelikož víme, že $\forall L$ existuje $l$ tak, abychom byly schopni získat $c_{\alpha, \beta}$, tak alespoň $\bigcup_l ©L_l = Gr_{k, n}$, což je druhá podmínka na atlas…)
        \end{reseni}

        \begin{reseni}[Difeomorfismy]
            Předpokládejme dvě posloupnosti s prvního odstavce definice map $\lambda_1$ a $\lambda_2$. BÚNO $\lambda_1 = (0, 1)$ a $\lambda_2 = (0, 2)$. Tedy $L \in ©L_{\lambda_1} \cap ©L_{\lambda_2}$ má báze:
            $$ B'_1 = \(\begin{pmatrix} 1 \\ 0 \\ a \end{pmatrix}, \begin{pmatrix} 0 \\ 1 \\ b \end{pmatrix}\) \text{ a } B'_2 = \(\begin{pmatrix} 1 \\ c \\ 0 \end{pmatrix}, \begin{pmatrix} 0 \\ d \\ 1 \end{pmatrix}\). $$
            Z báze $B'_1$ do báze $B'_2$ lze přejít tak, že k prvnímu vektoru přičteme $-\frac{a}{b}$ násobek druhého a druhý vektor vynásobíme $\frac{1}{b}$ (pokud $b = 0$, tak zjevně $B'_2$ není báze $L$, tj. $L \notin ©L_{\lambda_2}$). Tedy $d = \frac{1}{b}$ a $c = -\frac{a}{b}$ a naopak $b = \frac{1}{d}$ a $a = -\frac{c}{d}$. Tedy přechodová funkce (BÚNO) $\psi = \phi_{\lambda_2} \circ \phi^{-1}_{\lambda_1}$ je:
            $$ \psi((a, b)) = \(-\frac{a}{b}, \frac{1}{b}\), \hskip 4em \psi^{-1}((c, d)) = \(-\frac{c}{d}, \frac{1}{d}\). $$ 
            Tedy je zřejmě prostá (má inverzi) a hladká (derivováním podle $b$ se zvyšuje mocnina u $b$ a přenásobuje se $-1, -2, -3, …$, derivováním podle $a$ se $d$ změní na 0 a $c$ nejdříve na zápornou mocninu $b$ a pak na nulu), stejně tak její inverze.
        \end{reseni}

        \begin{reseni}[Dimenze]
            Zobrazení $\phi_l$ je zřejmě na $®R^{k·(n-k)}$, jelikož každá nezávislá množina vektorů určuje nějaký prostor (a nezávislé jsou díky 'jedničkám a nulám'), tedy $Gr_{k, n}$ je dimenze $k·(n-k)$, jelikož mapy jsou této dimenze.

            %(Aby to byla přímo varieta dimenze $k·(n-k)$, tak $Gr_{k, l}$ s topologií danou tímto atlasem musí být Hausdorffův a musí mít spočetnou bázi. Hausdorffův je, jelikož buď jsou $L_1, L_2 \in Gr_{k, l}$ ze shodného ©L, pak je můžeme převést mapou a vzít disjunktní okolí v té souřadnici, kde se liší a převést zpět.

            %Spočetnou bázi má, jelikož pro libovolné $l$ je $\phi_l$ z definice topologie (a protože je bijekce) homeomorfismus mezi $®R^{k·(n-k)}$, jež má spočetnou bázi, a $©L_$)
        \end{reseni}
    \end{priklad}

    \pagebreak

    \begin{priklad}[3.]
        Ukažte, že tečný fibrovaný prostor $TX$ má přirozenou strukturu hladké variety. Definujte mapy na $TX$ a ukažte, že přechodové funkce jsou difeomorfismy.

        \begin{reseni}
            Nechť $M$ je varieta dimenze $n$. Víme, že $\forall m \in M: T_mX$ je lineární vektorový prostor dimenze $n$, který pro libovolnou mapu, jež definuje obraz i pro $m$ má bázi $(\frac{\partial}{\partial x_1}, …, \frac{\partial}{\partial x_n})$. Tzn. disjunktní sjednocení $T_mX$ přes nějakou mapu $(\phi, U)$ lze ztotožnit s $U \times ®R^n$.

            Tedy mějme daný atlas na $M$, libovolnou mapu $(U, \phi)$ z něho a pro každý bod $m \in U$ bázi $B_m = \(\frac{\partial}{\partial x_1}, …, \frac{\partial}{\partial x_n}\)$ tečného prostoru $T_mX$. Zobrazení $\psi_\phi: \coprod_{m \in U} T_mX \rightarrow ®R^{2n}$ potom můžeme definovat pro všechna $x \in T_mX$ jako\footnote{$\pi$ je projekce $\forall x \in T_mX: \pi(x) = m$ a $[x]_B$ je vyjádření vektoru při bázi $B$.}
            $$ \psi_\phi(x) = \binom{\phi(\pi(x))}{[x]_{B_{\pi(x)}}} = \(\phi_1(\pi(x)), …, \phi_n(\pi(x)), \([x]_{B_{\pi(x)}}\)_1, …, \([x]_{B_{\pi(x)}}\)_n\)^T $$ 
            (tedy pro bod z $T_mX$ se na prvních $n$ souřadnic zobrazuje $m$ pomocí $\phi$ a na dalších $n$ souřadnicích jsou souřadnice tohoto bodu při bázi $B_m$). $\(\coprod_{m \in U} T_mX, \psi_\phi\)$ je pak zřejmě mapa a lze těmito mapami pokrýt celý $TX$, jelikož každý bod $TX$ má projekci $m \in M$, která jistě musí být v nějakém $U$.

            Aby byly přechodové funkce difeomorfismy, musí být prosté, na a hladké. Můžeme si všimnout, že pro dvě různé mapy $(U_1, \phi_1)$ a $(U_2, \phi_2)$ zobrazuje přechodová funkce prvních $n$ souřadnic podle přechodové funkce $\phi_1 \circ \phi_2^{-1}$, o níž víme, že je prostá, na a hladká, jelikož $M$ je varieta a toto jsou mapy z jednoho jejího atlasu, tedy musí být kompatibilní.

            Naopak druhých $n$ souřadnic je vyjádření v bázi dané derivacemi, tedy se v bodě $¦y = \(y_1, …, y_n\) \in ®R^n$ při přechodové funkcí $\Phi(x_1, …, x_n)$ zobrazuje za pomoci koeficientů Jacobiho matice $\Phi$ (vizte skripta):
            $$ (z_1, …, z_n) \mapsto \(\sum_{k=1}^n z_k \frac{\partial f_1}{\partial x_k}(¦y), …, \sum_{k=1}^n z_k \frac{\partial f_n}{\partial x_k}(¦y)\), $$ 
            což je hladká funkce, protože součty, násobky a derivace hladkých funkcí jsou hladké. Zároveň je prostá a na, jelikož je to v každém bodě $¦y$ vlastně násobení maticí $\Jac(\Phi)$, která má nenulový determinant (jelikož $\Phi$ je difeomorfismus, tedy $\Phi^{-1}$ je hladká), tedy je regulární.

            Pro mapy $(U_1, \psi_{\phi_1})$ a $(U_2, \psi_{\phi_2})$ na $TX$ a označení $\Phi(x_1, …, x_n) = \phi_1 \circ \phi_2^{-1} (x_1, …, x_n)$ je přechodová funkce (z $\psi_{\phi_2}\(U_2\)$ do $\psi_{\phi_1}\(U_1\)$) pro všechna $(¦y, ¦z)^T \in \psi_{\phi_2}(U_2) \subseteq ®R^{2n}$ $\(¦y, ¦z \in ®R^n\)$ tvaru
            $$ \psi_{\phi_1} \circ \psi_{\phi_2}^{-1} \binom{¦y}{¦z} = \psi_{\phi_1} \circ \psi_{\phi_2}^{-1} \(y_1, …, y_n, z_1, …, z_n\) = \binom{\Phi(¦y)}{\(\Jac(\Phi)(¦y)\)·¦z} = $$
            $$ = \(\Phi_1(¦y), …, \Phi_n(¦y), \sum_{k=1}^n z_k · \frac{\partial \Phi_1}{\partial x_k}(¦y), …, \sum_{k=1}^n z_k · \frac{\partial \Phi_n}{\partial x_k}(¦y)\), $$
            tj. je to hladká prostá funkce na (z předchozích 2 odstavců).
        \end{reseni}
    \end{priklad}

\end{document}
