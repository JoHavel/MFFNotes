\documentclass[12pt]{article}					% Začátek dokumentu
\usepackage{../../MFFStyle}					    % Import stylu



\begin{document}
	
\section*{Organizační úvod}
    \begin{poznamka}[složení předmětu]
        Předmět má přednášku, cvičení a proseminář z Matematické analýzy.
    \end{poznamka}


    \begin{poznamka}[Motivace]
        TODO  
    \end{poznamka}

    \begin{poznamka}[Jak studovat]
        Studujte průběžně, ptejte se…   
    \end{poznamka}

    \begin{poznamka}[Literatura]
        \ 
        \begin{itemize}
            \item skripta -- viz homepage
            \item příklady -- Koláček \&\ spol. -- Příklady z matematické analýzy pro fyziky 1, 2, 3
            \item další příklady -- viz homepage
        \end{itemize}        
    \end{poznamka}

\section*{Motivace}
    \begin{poznamka}[Používá se v]
        \ 
        \begin{itemize}
            \item Fyzice (viz pohyb planet, např. Proč obíhají planety po elipsách?)
            \item Ekonomii (viz úroky, např. Lze předpovědět změny úroků v ekonomice a na burze?)
            \item Meteorologii (např. Jak ovlivňují teplé mořské proudy počasí v Evropě?)
            \item Lékařství (např. Jak dlouho bude koncentrace léku v krvi na účinné úrovni?)
            \item Epidemiologie (např. Proč se epidemie šíří nejprve rychle a potom pomalu?)
            \item Architektura (viz mosty (vibrace), např. Jak se ujistit, že most nespadne v bouři?)
            \item Inženýrství (viz letadla, např. Jak zkonstruovat nové křídlo pro letadlo? (Většina peněz i času při konstrukci jsou simulace.))
            \item Informatika (viz MP3, JPEG, např. Jak uchovat informaci o dané funkci v co nejmenším počtu čísel? (Používá se Fourierova transformace = převod funkce na siny a cosiny))
        \end{itemize}
    \end{poznamka}

    \begin{poznamka}[Proč se analýzu učíme my?]
        \ 
        \begin{itemize}
            \item Abychom uměli látku (hledat extrémy funkcí, umět integrovat)
            \item Měli solidní matematické základy
            \item Přečíst návod, jak používat nějaké matematické modely (např. Uplatnění v bance -- MFF nese mnoho peněz (nejvíce žádané práce jsou práce stylu statistik))
            \item Myslet, analyzovat, nedělat chyby (nezapomínat na další možnosti)
            \item Abychom se naučili způsob myšlení -- matematici / matematičky dělají činnosti lépe (ze dvou lidí, co ji neumí, je matematik ten, kdo ji zvládne lépe, nikoliv ten druhý.)
        \end{itemize}
    \end{poznamka}

\section{Úvod}
    Dosti možná spíše opakování střední
    \subsection{Výroky}
        \begin{definice}[Výrok]
            Tvrzení, o kterém má smysl říct, že je pravdivé, či ne.
            \begin{prikladyin}
                „Obloha je modrá.“\\
                „Vídeň je hlavní město ČR.“
            \end{prikladyin}
        \end{definice}
        \begin{poznamka}[Vytváření nových výroků]
            Děje se pomocí spojek a, nebo, implikací, ekvivalencí, atd.
            
                 \begin{tabular}{ccccccc}
                            $A$ & $B$ & \begin{tabular}[c]{@{}c@{}}konjunkce\\ $A\&B$\end{tabular} & \begin{tabular}[c]{@{}c@{}}disjunkce\\ $AvB$\end{tabular} & \begin{tabular}[c]{@{}c@{}}implikace\\ $A\implies B$\end{tabular} & \begin{tabular}[c]{@{}c@{}}ekvivalence\\ $A\Leftrightarrow B$\end{tabular} & \begin{tabular}[c]{@{}c@{}}negace $A$\\ $\neg A$\end{tabular} \\
                            0 & 0 & 0 & 0 & 1! & 1 & 1 \\
                            0 & 1 & 0 & 1 & 1! & 0 & 1 \\
                            1 & 0 & 0 & 1 & 0 & 0 & 0 \\
                            1 & 1 & 1 & 1 & 1 & 1 & 0
                \end{tabular}

            $A \implies B$ = $A$ je postačující podmínka pro $B$ = $B$ je nutná podmínka pro $A$.
            \begin{prikladyin}[Pravdivé výroky]
                $ 1=2 \implies 2=3 $\\
                já jsem papež $ \implies $ všechna letadla jsou modrá \\
            \end{prikladyin}

            \begin{prikladin}
                \ 
                \begin{itemize}
                    \item $$ (A \implies B) \Leftrightarrow (\neg B \implies \neg B $$
                    \item $$ (A \implies B) \Leftrightarrow (\neg (A \& \neg B)) $$ 
                    \item $$ \neg (A \& B) \Leftrightarrow (\neg A \lor \neg B) $$
                    \item $$ \neg (A \lor B) \Leftrightarrow (\neg A \& \neg B) $$                    
                    \item $$ (A \Leftrightarrow B) \Leftrightarrow ((A \implies B) \& (B \implies A) $$ 
                \end{itemize}
            \end{prikladin}
        \end{poznamka}
        \begin{definice}[Kvantifikátory]
            Dále existují tzv. kvantifikátory: Obecný (= pro všechna) $\forall$ a Existenční (= existuje) $\exists$.
        \end{definice}
            
        \begin{umluva}
            $\forall x \in ®N, x>10 \ A(x)$ značí  $\forall x \in ®N (x>10 \implies A(x))$
        \end{umluva}
        \begin{priklady}
            \ 
            \begin{itemize}
                \item Pro všechna $x \in M$ platí $A(x)$ je: $\forall x \in M: A(x)$ 
                \item Existuje $x \in M$ tak, že platí $A(x)$ je $\exists x \in M: A(x)$
                \item $ \forall n \in ®N\ \forall m \in ®N\ \exists k \in ®N: k> n+m $
            \end{itemize}
        \end{priklady}

        \begin{priklady}[Negace výroků]
            \ 
            \begin{itemize}
                \item $\neg (\forall x \in M: A(x)) \Leftrightarrow \exists x \in M: \neg A(x)$
                \item $ \neg (\exists x \in M: A(x) \Leftrightarrow \forall x \in M: \neg A(x)$
                \item $\neg$(Nikdo mě nemá rád.) $\Leftrightarrow$ Existuje alespoň jeden člověk, který mě má rád. 
                \item $\neg(\forall n \in ®N\ \forall m \in ®N\ \exists k \in ®N: k > n+m)$
                    $$ \exists n \in ®N\ \neg(\forall m \in ®N\ \exists k \in ®N: k > n+m) $$
                    $$ \exists n \in ®N\ \exists m \in ®N\ \neg(\exists k \in ®N: k > n+m) $$
                    $$ \exists n \in ®N\ \exists m \in ®N\ \forall k \in ®N: k \leq n+m) $$ 


            \end{itemize}
        \end{priklady}

        \begin{upozorneni}
            Na pořadí kvantifikátorů záleží!
            \begin{prikladyin}
                M…Muži\\
                Ž…Ženy\\
                L(m, ž)…muži m se líbí žena ž
                $$ \forall m \in M\ \exists ž \in Ž: L(m, ž) $$
                $$ \exists ž \in Ž\ \forall m \in M: L(m, ž) $$
                První je, že pro každého muže existuje nějaká žena, která se mu líbí, naopak druhá říká, že existuje žena, která se líbí všem mužům.
            \end{prikladyin}
        \end{upozorneni}

    \subsection{Metody důkazů tvrzení}
        \begin{definice}[Důkaz sporem]
            $$ (A \implies B) \Leftrightarrow \neg(A \& \neg B) $$
            \begin{prikladyin}[$\sqrt{2} \notin ®Q$]
                ($A: x = \sqrt{2}$, $B: x \notin ®Q$)
                \begin{dukazin}[Důkaz sporem:]
                    Nechť $x = \sqrt{2}$ a $x \in ®Q$. $x \in ®Q \implies x = \frac{p}{q}, p,q \in ®N$, nesoudělná.\\
                $x^2 = 2$,  $2 = x^2 = \frac{p^2}{q^2} \implies 2q^2 = p^2 \implies p = 2k \implies 2q^2 = 4k^2 \implies q^2 = 2k^2 \implies q = 2l$\\
                    $p=2k\ \&\ q=2l \implies$ $p$ a $q$ soudělná. \lightning 
                \end{dukazin}
                    
            \end{prikladyin}
        \end{definice}

        \begin{definice}[Přímý důkaz]
                $$ (A \implies B) \Leftrightarrow (A \implies C_1 \implies C_2 \implies … \implies C_n \implies B) $$ 
            \begin{prikladyin}
                $$ n^2 \text{liché} \implies n \text{liché} $$
                \begin{dukazin}
                    $$ n = p_1 \cdot … \cdot p_k \implies n^2 = p_1^2 \cdot … \cdot p_k^2 $$
                    $$ n^2 \text{liché} \implies 2\nmid p_1\ \&\ …\ \&\ 2\nmid p_k \implies n \text{liché} $$ 
                \end{dukazin}
            \end{prikladyin}
        \end{definice}

        \begin{definice}[Nepřímý důkaz]
            $$ (A \implies B) \Leftrightarrow (\neg B \implies \neg A) $$
            \begin{prikladyin}
                $$ n^2 \text{liché} \implies n \text{liché} $$ 
                \begin{dukazin}
                    $$ n \text{sudé} \Leftrightarrow n = 2k \implies n^2 = 4k^2 \implies n^2 \text{sudé} $$         
                \end{dukazin}
            \end{prikladyin}
        \end{definice}

        \begin{definice}[Matematická indukce]
            $$ (\forall n \in ®N: V(n)) \Leftrightarrow (V(1)\ \&\ \forall n \in ®N: V(n) \implies V(n+1)) $$ 
            \begin{prikladyin}
                $$ \sum_{k=1}^n k = 1+2+…+n = \frac{1}{2}n\cdot(n+1) $$
                \begin{dukaz}
                    1. $n = 1$: $1 = \frac{1}{2}1 \cdot 2$\\
                    2. 
                    $$ \sum_{k=1}^n k = 1+2+…+n = \frac{1}{2}n\cdot(n+1) \implies \sum_{k=1}^{n+1} k = \frac{1}{2}n\cdot(n+1) + (n + 1) = \frac{1}{2}(n+1)\cdot(n+2) $$ 
                \end{dukaz}
            \end{prikladyin}
            
            \begin{prikladin}
                Všechna auta mají stejnou barvu.
                \begin{dukazin}
                    1. $n = 1$: Jedno auto má stejnou barvu jako ono samo.\\
                    2. $n \rightarrow n+1$: vezmu prvních $n$ aut, ty mají stejnou barvu, vezmu posledních $n$ aut, ty mají také stejnou barvu. Tedy dohromady mají stejnou barvu.
                \end{dukazin}
                (Spoiler: $n = 2$)
            \end{prikladin}
        \end{definice}

    \subsection{Množina reálných čísel}
        \begin{poznamka}[Množiny čísel]
            $$ ®N = \{1, 2, 3, …\} $$ 
            $$ ®Z = \{… , -3, -2, -1, 0, 1, 2, 3, …\} $$ 
            $$ ®Q = \{\frac{p}{q}, q\in ®N, p \in ®Z\} $$ 
        \end{poznamka}

        \begin{definice}[Omezená množina]
            Nechť $M \subset ®R$. Řekněme, že $M$ je omezená shora (omezená zdola), jestliže existuje $a \in ®R$ = horní (dolní) závora tak, že pro všechna $x \in M$ platí $x ≤ a$ ($x≥a$).
        \end{definice}

        \begin{definice}[Supremum a infimum]
                Nechť $M \subset ®R$ je shora (zdola) omezená. Číslo $s \in ®R$ nazýváme supremem (infimem) $M$, pokud:
                $$ \forall x \in M: x≤s (s≥x) $$
                $$ \forall y \in ®R, y<s (y>s) \exists x \in M, y<x (x>y) $$ 
            \begin{prikladyin}
                \begin{itemize}
                        \item $\sup\[0,1\]=1$ (dokázat obě podmínky pro 1 ($x$ z druhé podmínky volím 1)…)
                        \item $\sup(0,1)=1$ (taktéž dokázat obě podmínky pro 1 (pozor na záporná y) ($x$ z druhé podmínky zvolíme často $\frac{s+y}{2}$))
                \end{itemize}
            \end{prikladyin}
        \end{definice}

        \begin{definice}[Reálná čísla]
            Na množině ®R je dána relace $≤$ ($\subset ®R \times ®R$), operace sčítání $+$ a operace násobení $\cdot$ a množina ®R obsahuje prvky 0 a 1 tak, že platí:
            
            Viz skripta (takové ty tělesové / grupové podmínky, podmínky uspořádání a \emph{existence suprema})
        \end{definice}

        \begin{veta}[o existenci infima]
            Nechť $M \subset ®R$ je neprázdná zdola omezená množina. Pak existuje $\inf M$.
            \begin{dukazin}
                Označme $-M = \{x\in®R:-x\in M\}$. Zřejmě $M \neq \O$. $M$ je zdola omezená $\implies$ $-M$ je shora omezená. ($\exists K \in ®R \forall x \in M x>K \implies \exists K \in ®R \forall x \in -M x<-K$). Z axiomů ®R tedy existuje $s = \sup -M$. Položme $i = -s$. Tvrdím $i = \inf M$. (Dokážeme z definice suprema a infima, viz skripta).
            \end{dukazin}
        \end{veta}

        \begin{veta}[Archimedova vlastnost]
            Ke každému $x \in ®R$ existuje $n \in ®N$ tak, že $x < n$
            \begin{dukazin}[Sporem]
                $$ \exists x \in ®R \forall n \in ®N x≥n $$
                Tedy $®N$ je omezená podmnožina ®R. Tedy existuje $x' \in ®R, x' = \sup®N$. Tedy $\forall n \in ®N: n≤x'$. Pak také $\forall n \in ®N: n+1 ≤ x'$. To ale tvrdí, že $x'-1$ je také $\sup®N$. To je ale spor, protože můžeme zvolit $y = x' - \frac{1}{2}$, pak $y<x'$, tedy z druhé vlastnosti suprema $\exists n \in ®N: x' - \frac{1}{2} < n$, ale zároveň už víme, že $\forall n \in ®N: n<x'-1$.
            \end{dukazin}
        \end{veta}

        \begin{veta}[Hustota ®Q a $®R\setminus ®Q$]
            Nechť $a, b \in®R, a<b$. Pak existují $q\in®Q$ a $r\in®R\setminus®Q$ tak, že $q\in(a, b)$ a $r\in(a, b)$.
            \begin{dukaz}
                Podle \label{Archimedova vlastnost} $\exists n \in ®N$ tak, že $\frac{1}{b-a}<n$, tedy $\frac{1}{n}<b-a$. Zvolme $q = \frac{\lceil an\rceil + 1}{n}$, pak jistě $a<q<b$ a $q\in®Q$. 

                Poté použijeme $q_1 \in (a, b)$ a $q_2 \in (q_1, b)$. Zvolme $r = q_1 + \frac{q_2-q_1}{\sqrt{2}}$. Pak (jelikož druhá část je kladná) $r>q_1$. A $r<q_2 \Leftrightarrow q_1 + \frac{q_2-q_1}{\sqrt{2}}< q_2 \Leftrightarrow \frac{q_2-q_1}{\sqrt{2}}<q_2 - q_1$.

                Tedy $r \in (a, b)$ a $r \in ®R\setminus ®Q$, jelikož $r = q_1 + \frac{q_2-q_1}{\sqrt{2}} = p \in®Q \implies \sqrt{2} = \frac{q_2-q_1}{p-q_1}$, ale levá strana je jistě iracionální a pravá racionální. Spor.
            \end{dukaz}
        \end{veta}

        \begin{veta}[O $n$-té odmocnině (BD = bez důkazu)]
            Nechť $n \in ®N$ a $x \in \[0, \infty\)$, pak existuje právě jedno $y \in \[0,\infty\)$ tak, že $y^n = x$.
            \begin{dukaz}
                Idea: Položme $M = \{z \in ®R\}$. Ukážeme, že $M \neq \O$ shora omezená $\implies$ $\exists s = \sup M$. Nyní ukážeme $s^n = x$.
            \end{dukaz}
        \end{veta}

% 6. 10. 2020

    \subsection{Krátký výlet do nekonečna}
        \begin{definice}[Mohutnost množin]
            Řekněme, že množiny ®A a ®B mají stejnou mohutnost, pokud existuje bijekce $®A \rightarrow ®B$. Značíme $®A \approx ®B$.

            Řekněme, že množina ®A má mohutnost menší, nebo rovnu mohutnosti ®B, pokud existuje prosté zobrazení $®A \rightarrow ®B$. Značíme $®A \preceq ®B$.

            Řekněme, že množina ®A má menší mohutnost než ®B, pokud $®A \preceq ®B$, ale neplatí $®B \preceq ®A$. Značíme $®A \prec ®B$.

            \begin{prikladyin}
                    1) ®N, ®Z: $®N \approx ®Z$ (prosté z ®N do ®Z je triviální, opačně si očísluji ®Z)\\
                    2) ®N, ®Q: $®N \approx ®Q$ (obdobně, čísluji diagonálně)\\
                    3) ®N, ®R: $®N \prec ®R$ (důkaz sporem, přes diagonálu, vezmu první desetinou cifru z $f(1)$, druhou z $f(2)$… a pozměním je…)
            \end{prikladyin}
        \end{definice}

        \begin{tvrzeni}[Viz proseminář]
            $®A \preceq ®B \land ®B \preceq ®A \implies ®A \approx ®B$
        \end{tvrzeni}

        \begin{definice}
            Řekněme, že množina ®A je konečná, má-li konečný počet prvků.

            Řekněme, že ®A je spočetná, jestliže $®A \approx ®N$, nebo je ®A konečná.

            Řekněme, že ®A je nespočetná, jestliže $®N \prec ®A$.
        \end{definice}

        \begin{tvrzeni}[Cantor]
            Nechť $X$ je množina, pak $X \prec ©P(X)$, kde $@P(X)$ je množina všech podmnožin $X$.
            \begin{dukazin}
                Zobrazení $\phi: X \rightarrow ©P(X)$ definované $\phi(x) = \{x\}$ je prosté.

                Tvrdím, že neplatí, že $X \approx ©P(X)$. Důkaz sporem: Nechť existuje bijekce $\phi: X \rightarrow \P(X)$. Označme $A = \{x \in X: x \notin \phi(x)\}$. $\phi$ je bijekce $\implies \exists a \in X: \phi(a) = A$.

                        Nyní buď a) $a \in A \implies a \notin \phi(a) = A \lightning$ nebo b) $a \notin A \implies a \in \phi(a) = A \lightning$.
            \end{dukazin}
        \end{tvrzeni}

        \begin{poznamka}[„Nebrali jsme“ Hypotéza kontinua]
            Otázka: Existuje $A \subset ®R$, že $®N \prec A$ a $A \prec ®R$?

            Odpověď: Může a nemusí. (Hypotéza kontinua je ze standardních axiomů teorie množin tzv. nerozhodnutelná.)
        \end{poznamka}

        \begin{tvrzeni}
            Nechť $A_n, n \in ®N$, jsou spočetné množiny, pak:
            $$ A = \bigcup_{n=1}^∞ A_n = A_1\cup A_2\cup … $$
            je spočetná.
            \begin{dukazin}
                Napíšu si množiny $A_i$ do matice a očísluji po diagonálách. Tím získám $®N \succeq A$.
            \end{dukazin}
        \end{tvrzeni}

\section{Posloupnost}
    \subsection{Úvod}
        \begin{definice}
            Jestliže ke každému $n \in ®N$ je přiřazeno $a_n \in ®R$, pak říkáme, že $\{a_n\}_{n=1}^∞ = \{a_1, a_2, a_3, …\}$ je posloupnost reálných čísel.
            \begin{prikladyin}
                \begin{itemize}
                    \item $\{\frac{1}{n}\}_{n = 1}^∞ = \{1, \frac{1}{2}, \frac{1}{3}, …\}$
                    \item $\{2^n\}_{n=1}^∞ = \{2, 4, 8, …\}$
                    \item $a_1 = 1$, $a_{n+1} = a_n^2 + 1$ (rekurentně zadaná posloupnost)
                \end{itemize}
            \end{prikladyin}
        \end{definice}

        \begin{definice}
            Řekněme, že posloupnost $\{a_n\}_{n=1}^∞$ je:
            \begin{itemize}
                \item neklesající, jestliže $\forall n \in ®N: a_n ≤ a_{n+1}$
                \item nerostoucí, jestliže $\forall n \in ®N: a_n ≥ a_{n+1}$
                \item klesající, jestliže $\forall n \in ®N: a_n < a_{n+1}$
                \item rostoucí, jestliže $\forall n \in ®N: a_n < a_{n+1}$
            \end{itemize}
            \begin{prikladyin}
                $\{\frac{1}{n}\}$ je klesající a nerostoucí\\
                $\{2^n\}$ je rostoucí a neklesající
            \end{prikladyin}
        \end{definice}

        \begin{definice}
            Řekněme, že posloupnost $\{a_n\}_{n = 1}^∞$ je omezená, jestliže množina členů posloupnosti $\{a_n\}_{n = 1}^∞$ je omezená podmnožina ®R. Analogicky definujeme omezenost shora a omezenost zdola.
            \begin{prikladyin}
                $\{\frac{1}{n}\}$ je omezená\\
                $\{2^n\}$ je pouze omezená zdola
            \end{prikladyin}
        \end{definice}

    \subsection{Vlastní limita posloupnosti}
        \begin{definice}[Limita]
            Nechť $A \in ®R$ a $\{a_n\}_{n = 1}^∞$ je posloupnost. Řekněme, že $A$ je (vlastní) limitou posloupnosti $\{a_n\}$, jestliže:
            $$ \forall \epsilon > 0\ \exists n_0 \in ®N\ \forall n≥ n_0, n \in ®N: |a_n - A|<\epsilon $$ 
            Značíme $\lim_{n \rightarrow ∞} a_n = A$.

            \begin{prikladyin}
                Ve videu, při pochopení limity nejsou moc zajímavé.
            \end{prikladyin}
        \end{definice}

% 9. 10. 2020

        \begin{priklad}[$\lim_{n \rightarrow ∞}$ \hbox{$\sqrt[n]{n}$} $= 1$]
                $$ \sqrt[n]{n} = 1+a_n \Leftrightarrow n = \(1+a_n\)^n = 1+ na_n + \frac{n\cdot(n-1)}{2} + … ≥ 1 + \frac{n\cdot(n-1)}{2}a_n^2 $$

                $$ \frac{2(n-1)}{n\cdot(n-1)} ≥ a_n^2 \implies \frac{\sqrt{2}}{\sqrt{n}} ≥ a_n ≥ 0 $$
        \end{priklad}

        \begin{veta}[Jednoznačnost vlastní limity (2.1)]
            Každá posloupnost má nejvýše jednu limitu.
            \begin{dukazin}[Sporem]
                    Nechť tedy existuje více limit. Dvě z nich označme $\lim_{n \rightarrow ∞} a_n = A$ a $\lim_{n \rightarrow ∞} a_n = B$, $A>B$. Zvolme $\epsilon = \frac{A-B}{3}$. Z definice limity k našemu $\epsilon$ existují $n_A, n_B \in ®N$ tak, že $\forall n≥n_A |a_n-A|<\epsilon$ a $\forall n≥n_B |b_n-B|<\epsilon$. Položme $n_0 = \max\{n_A, n_B\}$. Z trojúhelníkové nerovnosti\footnote{$\forall x,y \in ®R: |x+y| ≤ |x| + |y|$} $|A - B| = |(A-a_{n_0}) + (a_{n_0} - B)|≤ |A-a_{n_0}| + |a_{n_0} - B|<\epsilon + \epsilon = \frac{2}{3} (A-B)$. $\lightning$
            \end{dukazin}
        \end{veta}

        \begin{veta}[O omezenosti konvergentní posloupnosti]
            Nechť $\{a_n\}_{n = 1}^∞$ má vlastní limitu. Pak $\{a_n\}_{n = 1}^∞$ je omezená.
            \begin{dukazin}
                Nechť $\lim_{n \rightarrow ∞} a_n = A \in ®R$. Položme $\epsilon = 1$. K tomuto $\epsilon = 1 \exists n_0 \in ®N \forall n≥n_0: |a_n - A| < \epsilon \Leftrightarrow a_n \in \(A - \epsilon, A + \epsilon\) = \(A-1, A+1\)$. Množina $\{a_n| n = 1, 2, …, n_0\}$ je konečná, tedy omezená. Položme $K = \max\{|a_1|, |a_2|, …, |a_{n_0}|, |A| + 1\}$. Potom jistě $\forall n \in ®N |a_n| ≤ K$ (protože $\forall n≤n_0 |a_n| ≤ \max\{|a_i|; i≤n_0\}$ a $\forall n > n_0 a_n \in \(A-1, A+1\)\implies |a_n| ≤ |A| + 1≤K$).
            \end{dukazin}
        \end{veta}

        \begin{priklad}
            $$ \exists \lim_{n \rightarrow ∞} a_n = A \in ®R \implies \exists n_0 \forall n > n_0 a_n\text{ je monotónní} $$
        \end{priklad}

        \begin{definice}[Vybraná podposloupnost]
                Řekněme, že posloupnost $\{b_n\}_{n = 1}^∞$ je vybraná z posloupnosti $\{a_n\}_{n = 1}^∞$, jestliže existuje rostoucí posloupnost přirozených čísel $\{k_n\}_{n = 1}^∞$ tak, že $b_n = a_{k_n}$
        \end{definice}

        \begin{veta}[o limitě vybrané podposloupnosti]
            Nechť $\lim_{n \rightarrow ∞} a_n = A\in ®R$ a nechť $\{b_n\}_{n = 1}^∞$ je vybraná z $\{a_n\}_{n = 1}^∞$. Pak $\lim_{n \rightarrow ∞} b_n = A$
            \begin{dukazin}
                K $\epsilon > 0 \exists n \in ®N \forall n ≥ n_0: |a_n - A|<\epsilon$. Chceme dokázat $\lim_{n \rightarrow ∞} b_n = A$.

                K $\epsilon > 0$ zvolme $k_0$, kde $n_0$ je z definice $\lim_{n \rightarrow ∞}a_n$. Nechť $k≥k_0$, pak $n_k≥k≥k_0≥n_0$. Tedy $|b_k - A| = |a_{n_k}-A|<\epsilon$
            \end{dukazin}
        \end{veta}

        \begin{veta}[Aritmetika limit]
            Nechť $\lim_{n \rightarrow ∞} a_n \in ®R$ a $\lim_{n \rightarrow ∞} b_n = B\in®R$. Pak platí:

            $$ \lim_{n \rightarrow ∞} a_n + b_n = A + B $$
            $$ \lim_{n \rightarrow ∞} a_n b_n = A\cdot B $$
            $$ \forall b_n \neq 0 \land B \neq 0 \implies \lim_{n \rightarrow ∞} \frac{a_n}{b_n} = \frac{A}{B} $$

% 13. 10. 2020

            \begin{dukazin}
                Nechť $\epsilon > 0$. Z $\lim_{n \rightarrow ∞} a_n = A\exists n_A \in ®N \forall n>n_A |a_n - A| < \epsilon$, z $\lim_{n \rightarrow ∞} b_n = B\exists n_B \in ®N \forall n>n_B |b_n - B| < \epsilon$. Zvolme $n_0 = \max\{n_A, n_B\}$. Pak $\forall n≥ n_0$ platí $|(a_n + b_n) - (A + B)| = |(a_n - A) + (b_n - B)| ≤ |a_n - A| + |b_n - B| < \epsilon + \epsilon = 2\epsilon$. (A tady se použije lemmátko, které jsem ani nepsal a které je o tom, že $\epsilon$ můžeme na konci definice limity vynásobit libovolnou konstantou.)

                $\exists \lim_{n \rightarrow ∞} b_n = B \stackrel{\text{V2.2}}{\implies}$ b je omezená, tedy $\exists K \in ®R \forall n \in ®N |b_n|≤K$. Nechť $\epsilon > 0$. Z $\lim_{n \rightarrow ∞} a_n = A\exists n_A \in ®N \forall n>n_A |a_n - A| < \epsilon$, z $\lim_{n \rightarrow ∞} b_n = B\exists n_B \in ®N \forall n>n_B |b_n - B| < \epsilon$. Zvolme $n_0 = \max{n_A, n_B}$. Pak $\forall n>n_0$ platí $|a_nb_n - AB| = |a_nb_n - b_nA + b_nA - AB|≤|a_nb_n-b_nA| + |b_nA - AB|≤|a_n - A||b_n| + |b_n - B||A|≤\epsilon \cdot K + \epsilon \cdot |A| = \epsilon \cdot (K + |A|)$.

                K $\epsilon_1 = \frac{|B|}{2}$ z $\exists \lim_{n \rightarrow ∞} b_n = B \exists n_1 \in ®N \forall n>n_A |a_n - A| < \epsilon_1 = \frac{|B|}{2} \implies |b_n| > \frac{|B|}{2}$. Nechť $\epsilon > 0$. Z $\lim_{n \rightarrow ∞} a_n = A\exists n_A \in ®N \forall n>n_A |a_n - A| < \epsilon$, z $\lim_{n \rightarrow ∞} b_n = B\exists n_B \in ®N \forall n>n_B |b_n - B| < \epsilon$. Zvolme $n_0 = \max{n_A, n_B, n_1}$. Pak $\forall n>n_0$ platí $|\frac{a_n}{b_n} - \frac{A}{B}| = \frac{|a_nB - AB + AB - b_nA|}{|b_n|\cdot|B|} ≤ \frac{|a_nB - AB|}{|b_n|\cdot |B|} + \frac{|A-B-b_nA|}{|b_n| - |B|} = \frac{|a_n - A|\cdot |B|}{|b_n|\cdot |B|} + \frac{|A|\cdot |B-b_n|}{|b_n|\cdot |B|} < \epsilon \cdot (\frac{2}{|B|} + \frac{|A|}{|B|}\cdot\frac{2}{|B|})$.
            \end{dukazin}
        \end{veta}

        \begin{veta}[Limita a uspořádání]
            Nechť $\lim_{n \rightarrow ∞} a_n = A \in ®R$, $\lim_{n \rightarrow ∞} b_n = B \in ®R$. 

            Jestliže $A<B$, pak existuje $n_0 \in ®N \forall n≥n_0 a_n < b_n$.

            Jestliže $\exists n_0 \in ®N$ tak, že $\forall n > n_0$ platí $a_n≥b_n$, pak $A≥B$
            \begin{dukazin}
                Položme $\epsilon = \frac{B-A}{2}$. Z existence limit vyplývá $\exists n_A \forall n≥n_A |a_n - A| < \epsilon \implies a_n < A + \epsilon = A + \frac{B-A}{2} = \frac{B+A}{2}$ a $\exists n_B \forall n≥n_B |b_n - B| < \epsilon \implies b_n > B - \epsilon = B - \frac{B-A}{2} = \frac{B+A}{2}$. Zvolme $n_0 = \max\{n_A, n_B\}$. Pak $\forall n ≥ n_0$ platí $b_n > \frac{A+B}{2} > a_n$.
                
                Sporem. Nechť $A<B$. Pak podle předchozí části $\exists n_1 \forall n>n_1 a_n < b_n$. Zároveň z předpokladu $\forall n≥n_0 a_n≥b_n$. Pak pro libovolné $n≥n_1$ a $n≥n_0$ platí $(a_n<b_n) \land (b_n<a_n) \lightning$
            \end{dukazin}
        \end{veta}

        \begin{veta}[O dvou strážnících]
            Nechť $\{a_n\}_{n = 1}^∞, \{b_n\}_{n = 1}^∞, \{c_n\}_{n = 1}^∞$ jsou posloupnosti splňující $\exists n_0 \in ®N \forall n \in ®N n≥n_0: a_n ≤ c_n ≤ b_n$ a $\lim_{n \rightarrow ∞} a_n = \lim_{n \rightarrow ∞} b_n = A \in ®R$.

            Pak $\lim_{n \rightarrow ∞} c_n = A$.
            
% 16. 10. 2020

            \begin{dukazin}
                Nechť $\epsilon$ je kladné. Potom $\exists n_A \forall n≥n_A |a_n - A| < \epsilon$ a $\exists n_B \forall n≥n_B |b_n - A| < \epsilon$, tedy zvolme $n_C = \max\{n_A, n_B\}$, tudíž $\forall n≥n_C c_n \in (a_n, b_n) \subseteq (A-\epsilon, A+\epsilon)$.
            \end{dukazin}
        \end{veta}

        \begin{veta}[O limitě součinu omezené a mizející posloupnosti]
            Nechť $\lim_{n \rightarrow ∞} a_n = 0$ a $\{b_n\}$ je omezená. Pak $\lim_{n \rightarrow ∞} a_n\cdot b_n = 0$.
            \begin{dukazin}
                Posloupnost $\{b_n\}$ je omezená $\implies \exists K > 0 \forall n \in ®N |b_n|≤K$. Z $\lim_{n \rightarrow ∞} a_n = 0$ k zadanému $\epsilon > 0 \exists n_0 \forall n≥n_0 |a_0 - 0| < \epsilon$. K tomuto $\epsilon > 0$ volme stejné $n_0$, pak $\forall n≥n_0 |a_n · b_n-0| = |a_n · b_n|≤|a_n| · |b_n| ≤ \epsilon · K$
            \end{dukazin}

            \begin{prikladyin}
                $$ \lim_{n \rightarrow ∞} \frac{\sin(n)}{n} = 0 $$ 
            \end{prikladyin}
        \end{veta}

    \subsection{Nevlastní limita posloupnosti}
        \begin{definice}[Nevlastní limita]
            Řekneme, že posloupnost $\{a_n\}_{n = 1}^∞$, má (nevlastní) limitu $+∞$ (respektive $-∞$), pokud
            $$ \forall K \in ®R \exists n_0 \in ®N \forall n≥n_0, n \in ®N: a_n > K $$ 
            ($$ \forall K \in ®R \exists n_0 \in ®N \forall n≥n_0, n \in ®N: a_n < K $$)
            \begin{prikladyin}
                $$ \lim_{n \rightarrow ∞} n = +∞ $$ 
            \end{prikladyin}
        \end{definice}

        \begin{tvrzeni}
            Věty: jednoznačnost limity (2.1), limita vybrané posloupnosti (2.3), limita a uspořádání (2.5), o dvou strážnících (2.6, stačí jeden z nich).
            \begin{dukazin}
                Analogicky
            \end{dukazin}
        \end{tvrzeni}
        
        \begin{definice}[Rozšířená reálná osa]
            Rozšířená reálná osa je množina $®R = ®R\cup\{+∞, -∞\}$. S následujícími vlastnostmi:
            \begin{itemize}
                \item Uspořádání: $\forall a \in ®R: -∞<a<+∞$
                \item Absolutní hodnota: $|+∞| = |-∞| = +∞$
                \item Sčítání: $\forall a \in ®R*\setminus \{-∞\}: +∞ +a = +∞$\\$\forall a \in ®R*\setminus \{+∞\}: -∞ +a = -∞$
                \item Násobení: $\forall a \in ®R* a>0: a(±∞) = ±∞$\\
                        $\forall a \in ®R* a<0: a(±∞) = \mp∞$
                \item Dělení: $\forall a \in ®R: \frac{a}{±∞} = 0$.
                \item Výrazy $-∞+∞, 0(±∞), \frac{±∞}{±∞}, \frac{\text{cokoliv}}{0}$ nejsou definovány (z dobrého důvodu!).
            \end{itemize}
        \end{definice}

        \begin{poznamka}[Rozšířená definice suprema a infima]
            Je-li $®A≠\O$ shora neomezená, pak definujeme $\sup®A = +∞$.

            Je-li $®A≠\O$ zdola neomezená, pak definujeme $\inf®A = -∞$.

            $\sup\O = -∞, \inf\O = +∞$
        \end{poznamka}

        \begin{veta}[Aritmetika limit podruhé (L2.4)]
            Nechť $\lim_{n \rightarrow ∞} a_n = A\in ®R*$ a $\lim_{n \rightarrow ∞} b_n = B \in ®R*$. Pak platí:
            \begin{enumerate}
                \item $\lim_{n \rightarrow ∞} a_n + b_n = A + B$, pokud je výraz $A+B$ definován.
                \item $\lim_{n \rightarrow ∞} a_n · b_n = A·B$, pokud je výraz $A·B$ definován.
                \item Pokud $b_n ≠ 0, \forall n \in ®N$ a $B ≠ 0$, pak $\lim_{n \rightarrow ∞}\frac{a_n}{b_n} = \frac{A}{B}$, pokud je výraz $\frac{A}{B}$ definován.
            \end{enumerate}

            \begin{dukazin}[Část]
                 1. $A, B \in ®R$ víme. $A = +∞, B \in ®R$. Zvolme $K \in ®R$ libovolně. Z toho, že $\lim_{n \rightarrow ∞} = B \in ®R$ k $\epsilon = 1 \exists n_1 \forall n≥n_1 |b_n - B|<1 \implies b_n > B-1$. Z $\lim_{n \rightarrow ∞} a_n = +∞$ plyne, že k $K' = K - B + 1 \exists n_0 \forall n≥n_0: a_n > K'=K-B+1$. Pak $\forall n≥n_0'=\max\{n_0, n_1\}: a_n + b_n > K-B+1+B-1 = K$.
            \end{dukazin}
        \end{veta}

        \begin{veta}[Limita typu $\frac{A}{0}$]
            Nechť $\lim_{n \rightarrow ∞} a_n = A\in ®R*, A>0$, $\lim_{n \rightarrow ∞} b_n = 0$ a $\exists n_0 \forall n ≥ n_0$ platí $b_n>0$. Pak $\lim_{n \rightarrow ∞} \frac{a_n}{b_n} = +∞$
            \begin{dukazin}
                Zvolme $K \in ®R$.
                $$ \lim_{n \rightarrow ∞} a_n = A \< \begin{array}{l}
                    A = +∞:\exists n_1 \forall n≥n_1 a_n > 1\\
                    A \in ®R: \epsilon = \frac{A}{2} \exists n_1 \forall n≥ n_1 |a_n - A|<\epsilon = \frac{A}{2} \implies a_n > \frac{A}{2}
                    \end{array} \right. $$ 
                    Tedy položme $\tilde{A} = \min\{1, \frac{A}{2}\}$. Pak $\forall n≥ n_1: a_n > \tilde{A}$. Z $\lim_{n \rightarrow ∞} b_n = 0$ k $\epsilon = \frac{\tilde{A}}{K} \exists n_2 \forall n≥n_2 |b_n - 0|< \frac{\tilde{A}}{K} \implies 0<b_n<\frac{\tilde{A}}{K}$.

                    Položme $n_3 = \max\{n_0, n_1, n_2\}$ pak
                    $$ \forall n≥n_3 \frac{a_n}{b_n} > \tilde{A}·\frac{K}{\tilde{A}} = K $$ 
            \end{dukazin}
        \end{veta}

% 20. 10. 2020

    \subsection{Hlubší věty o limitách}
        \begin{veta}[O limitě monotónní posloupnosti (L2.9)]
            Každá monotónní posloupnost má limitu.

            \begin{dukazin}
                BÚNO $a_n$ je neklesající. Označme $A = \sup_{n\in®N}\{a_n\}$.

                1. $A = +∞$. Nechť $K \in ®R$, $\sup\{a_n\} = +∞ \implies a_n$ není shora omezená $\implies \exists n_0 a_{n_0}>K$. $a_n$ je neklesající $\implies \forall n ≥ n_0 a_n ≥ a_{n_0} > K$. To je ale definice $\lim_{n \rightarrow ∞} a_n = +∞$

                2. $A \in ®R$. Nechť $\epsilon > 0 A-\epsilon < A$. Z definice suprema musí existovat $n_0: a_{n_0} > A-\epsilon$. Jelikož $a_n$ je neklesající, je $\forall n≥n_0 a_n≥a_{n_0}>A-\epsilon$. Z definice suprema $a_n ≤ A < A+\epsilon$, tedy $\forall n≥ n_0 |a_n - A| < \epsilon$.
            \end{dukazin}
        \end{veta}

        \begin{poznamka}
            Monotónní posloupnost: neklesající (shora omezená = vlastní limita, shora neomezená = limita $+∞$), nerostoucí (sdola omezená = vlastní limita, sdola neomezená = limita $-∞$).
        \end{poznamka}

        \begin{priklad}
            $$a_1 = 10, a_{n+1} = 6-\frac{5}{a_n}$$
            \begin{reseni}
                Napíšu prvních pár členů a tipneme, že je klesající a $a_n≥5$. Pak vše dokážeme. A použijeme aritmetiku limit.
            \end{reseni}
        \end{priklad}

        \begin{upozorneni}
            V předchozím příkladu je použití věty 2.9 nutné!.
        \end{upozorneni}

        \begin{veta}[Cantorův princip vložených intervalů]
            Nechť $\{\[a_n, b_n\]\}_{n=1}^∞$ je posloupnost uzavřených intervalů splňující $\[a_{n+1}, b_{n+1}\] \subset \[a_n, b_n\]$ a $\lim_{n \rightarrow ∞} b_n - a_n = 0$. Pak je množina $\bigcap_{n=1}^∞\[a_n, b_n\]$ jednobodová.

            \begin{dukazin}
                Z první podmínky na interval vidíme $a_{n+1}≥a_n$ a $b_{n+1} ≤ b_n$. Navíc $a_n$ je shora omezená $b_1$ a $b_n$ je sdola omezená $a_1$. Podle $V2.9 \exists \lim_{n \rightarrow ∞}a_n = A \in ®R$ a $\exists \lim_{n \rightarrow ∞} b_n = B \in ®R$.
                $$ 0 = \lim_{n \rightarrow ∞} b_n - a_n \overset{?}{=} \lim_{n \rightarrow ∞} b_n - \lim_{n \rightarrow ∞} a_n = B-A \implies A=B $$
                Tedy $\bigcap_{n=1}^∞ \[a_n, b_n\] = \{A\}$.
            \end{dukazin}    
        \end{veta}

        \begin{veta}[Bolzano-Weirstrass]
            Z každé omezené posloupnosti lze vybrat konvergentní podposloupnost.

            \begin{dukazin}[Tzv. půlením intervalu]
                $\{a_n\}$ je omezená, tedy $\exists c_1, d_1 \in ®R \forall n \in ®N c_1 ≤ a_n ≤ d_1$. Zvolme $a_{n_1} \in \[c_1, d_1\}$ libovolně. Rozdělme $\[c_1, d_1\]$ na $\[c_1, \frac{c_1 + d_1}{2}\]$ a $\[\frac{c_1 + d_1}{2}, d_1\]$. V alespoň jednom tomto intervalu je $∞$ mnoho $a_n$. Pokud $\#\{n: a_n \in \[c_1, \frac{c_1 + d_1}{2}\]\} = +∞$, položme $c_2 = c_1, d_2 = \frac{c_1 + d_1}{2}$. Jinak $\#\{n: a_n \in \[\frac{c_1 + d_1}{2}, d_1\]\} = +∞$, položme $c_2 = \frac{c_1 + d_1}{2}, d_2 = d_1$. Nalezneme $n_2 > n_1$ a $a_{n_2} \in [c_2, d_2]$. Dále pokračujeme indukcí.

                Nechť $\#\{n: a_n \in \[c_k, d_k\]\} = +∞$. Rozdělme $\[c_k, d_k\]$ na $\[c_k, \frac{c_k + d_k}{2}\]$ a $\[\frac{c_k + d_k}{2}, d_k\]$. V alespoň jednom tomto intervalu je $∞$ mnoho $a_n$. Pokud $\#\{n: a_n \in \[c_k, \frac{c_k + d_k}{2}\]\} = +∞$, položme $c_{k+1} = c_k, d_{k+1} = \frac{c_k + d_k}{2}$. Jinak $\#\{n: a_n \in \[\frac{c_k + d_k}{2}, d_k\]\} = +∞$, položme $c_{k+1} = \frac{c_k + d_k}{2}, d_{k+1} = d_k$. Nalezneme $n_{k+1} > n_k$ a $a_{n_{k+1}} \in [c_{k+1}, d_{k+1}]$.

                Nyní máme posloupnost intervalů $\[c_k, d_k\]$ a $a_{n_k} \in \[c_k, d_k\]$. Víme, že $\[c_{k+1}, d_{k+1}\] \subset \[c_k, d_k\]$ a $d_{k+1} - c_{k+1} = \frac{d_k-c_k}{2} = \frac{d_1-c_1}{2^k}$, tedy $\lim_{k \rightarrow ∞} d_k - c_k = 0$.

                Podle V2.10 $\exists A = \bigcap_{k=1}^∞\[c_k, d_k\]$ a $\lim_{k \rightarrow ∞} c_k = A = \lim_{k \rightarrow ∞} d_k$. Nyní $n_k$ je  rostoucí posloupnost, tedy $a_{n_k}$ je vybraná podposloupnost z $\{a_n\}$. Víme, že $a_{n_k}\in\[c_k, d_k\] \Leftrightarrow c_k ≤ a_{n_k} ≤ d_k$ a podle Věty o dvou strážnících je $\lim_{k \rightarrow ∞} a_{n_k} = A \in ®R$.
            \end{dukazin}
        \end{veta}

        \begin{definice}[Limes superior]
            Nechť $\{a_n\}$ je posloupnost a označme $b_n = \sup\{a_k: k≥n\}$ a $c_n = \inf\{a_k: k≥n\}$. (Pak $b_n$ je nerostoucí a $c_n$ je neklesající.)

            Je-li $\{a_n\}$ shora neomezená, pak klademe $\lim_{n \rightarrow ∞} b_n = +∞$. Je-li $\{a_n\}$ sdola neomezená, pak klademe $\lim_{n \rightarrow ∞} c_n = -∞$.

            Číslo $\lim_{n \rightarrow ∞} b_n$ nazýváme limes superior posloupnosti $\{a_n\}$ a značíme $\limsup\limits_{n \rightarrow ∞} a_n$. Číslo $\lim_{n \rightarrow ∞} c_n$ nazýváme limes inferior posloupnosti $\{a_n\}$ a značíme $\liminf\limits_{n \rightarrow ∞} a_n$.
            \begin{dukazin}
                Existence limit je zaručena V 2.9 (o limitě monotónní posloupnosti).
            \end{dukazin}

            \begin{poznamkain}
                Zřejmě $\forall c_n ≤ a_n ≤ b_n$.
            \end{poznamkain}

            \begin{prikladyin}
                $$ \limsup\limits_{n \rightarrow ∞} (-1)^n = 1 $$
                $$ \liminf\limits_{n \rightarrow ∞} (-1)^n = -1 $$
            \end{prikladyin}
        \end{definice}

% 23. 10. 2020

        \begin{veta}[Vztah limity, limes superior a limes inferior (T2.12)]
            Nechť $\{a_n\}_{n = 1}^∞$ je posloupnost reálných čísel a $A \in ®R*$. Pak
            $$ \lim_{n \rightarrow ∞} a_n = A \Leftrightarrow \limsup\limits_{n \rightarrow ∞} a_n = \liminf\limits_{n \rightarrow ∞} a_n = A. $$

            \begin{dukazin}
                Důkaz jen pro $A \in ®R$. Jinak by byl podobný.

                $\implies$ $\lim_{n \rightarrow ∞} a_n = A$, tedy posloupnost je omezená dle věty 2.2. Můžeme tedy definovat $b_n$ a $c_n \in ®R$. Posloupnosti $b_n$ a $c_n$ jsou monotónní a platí $c_n ≤ b_n$. Nechť $\epsilon > 0$. Z definice $\lim_{n \rightarrow ∞} a_n \exists n_0 \in ®N \forall n≥n_0 |a_n - A| < \epsilon$, tj. $\forall n_0 b_n = \sup\{a_n, a_{n+1}, …\}≤ A + \epsilon$ a $\forall n_0 c_n = \inf\{a_n, a_{n+1}, …\} ≥ A - \epsilon$.

                Podle V 2.5 o limitě a uspořádání $\lim_{n \rightarrow ∞} c_n \in \[A - \epsilon, A + \epsilon\]$, $\lim_{n \rightarrow ∞} b_n \in \[A - \epsilon, A + \epsilon\]$, pro libovolné $\epsilon$.

                $\Leftarrow$ Podle definice $b_n$ a $c_n$ je $\{a_n\}_{n = 1}^∞$ omezená, tedy můžeme definovat $b_n$ a $c_n$. Z poznámky víme, že $c_n≤a_n≤b_n$ a podle věty o dvou strážnících tedy $\exists \lim_{n \rightarrow ∞} a_n = A$.
            \end{dukazin}
        \end{veta}

        \begin{priklad}
            $$\liminf\limits_{n \rightarrow ∞} a_n ≤ \limsup\limits_{n \rightarrow ∞} a_n$$

            $$ \liminf\limits_{n \rightarrow ∞} a_n + \liminf\limits_{n \rightarrow ∞} b_n ≤ \liminf\limits_{n \rightarrow ∞} (a_n + b_n) ≤ \limsup\limits_{n \rightarrow ∞} (a_n + b_n) ≤ \limsup\limits_{n \rightarrow ∞} a_n + \limsup\limits_{n \rightarrow ∞} b_n $$
        \end{priklad}

        \begin{definice}[Hromadná hodnota]
            Nechť $\{a_n\}_{n = 1}^∞$ je posloupnost reálných čísel a $A \in ®R*$. Řekněme, že $A$ je hromadná hodnota posloupnosti $\{a_n\}$, jestliže existuje vybraná podposloupnost $\{a_{n_k}\}_{k = 1}^∞$ z $\{a_n\}_{n = 1}^∞$ tak, že $\lim_{k \rightarrow ∞} a_{n_k} = A$. Množinu hromadných hodnot značíme $H(\{a_n\})$.

            \begin{prikladyin}
                $H(\{(-1)^n\} = \{0, 1\})$
            \end{prikladyin}
        \end{definice}

        \begin{veta}[O hromadných hodnotách posloupnosti]
            Nechť $\{a_n\}_{n = 1}^∞$ je posloupnost reálných čísel, potom $\limsup\limits_{n \rightarrow ∞} a_n$ a $\liminf\limits_{n \rightarrow ∞} a_n$ jsou hromadnými hodnotami posloupnosti $\{a_n\}_{n = 1}^∞$ a pro každou hromadnou hodnotu $A \in ®R*$ této posloupnosti platí $\liminf\limits_{n \rightarrow ∞} a_n ≤ A ≤ \limsup\limits_{n \rightarrow ∞} a_n$.
            \begin{dukazin}
                Opět pouze pro $\limsup\limits_{n \rightarrow ∞} a_n = A \in ®R$. Pak $a_n$ je shora omezená. Označme $b_n$ jako vždy, $b_n$ je nerostoucí a $\lim_{n \rightarrow ∞} b_n = A$.

                Z $\lim_{n \rightarrow ∞} b_n = A$ $\epsilon = 1 \exists m_1$ tak, že $|b_{m_1} - A|<1$. Nyní z $b_{m_1} = \sup\{a_{m_1}, a_{m_1 + 1}, …\}$ existuje $n_1 ≥ m_1$ tak, že $b_{m_1} - 1 < a_{n_1} ≤ b_{m_1} \implies |a_{n_1} - b_{m_1}| < 1 \implies |a_{n_{1}} - A|≤|a_{n_1} - b_{n_1}| + |b_{m_1} - A| < 2$. Dále indukcí.

                Mějme $m_1, …, m_k$ a $n_1, …, n_k$. Z $\lim_{n \rightarrow ∞} b_n = A, \epsilon = \frac{1}{k+1} \exists m_k > n_k |b_{m_k} - A| < \frac{1}{n+1}$. Z $b_{m_{k+1}} = \sup\{ a_{m_{k+1}}, … \} \exists n_{k+1} ≥ m_{k+1}$ tak, že $b_{m_{k+1}} - \frac{1}{k+1} < a_{n_{k+1}} ≤ b_{m_{k+1}} \implies |a_{n_{k+1} - b_{m_{k+1}}}|<\frac{1}{k+1} \implies |a_{n_{k+1}} - A|≤|a_{n_{k+1}} - b_{m_{k+1}}| + |b_{m_{k+1}}-A| < \frac{2}{k+1}$.
                
                Tedy jsme dostali rostoucí posloupnost $n_k$ tak že $|a_{n_k} - A| < \frac{2}{k} \implies \lim_{k \rightarrow ∞} a_{n_k} = A$. Tedy $\limsup\limits_{n \rightarrow ∞} a_n = A \in H(\{a_n\})$.

                Úplně stejně pro $\liminf\limits_{n \rightarrow ∞} a_n$.

                Stačí dokázat $\forall A \in H(\{a_n\})$ $\liminf\limits_{n \rightarrow ∞} a_n ≤ A ≤ \limsup\limits_{n \rightarrow ∞} a_n$. Nechť $n_k$ je rostoucí posloupnost taková, že $\lim_{k \rightarrow ∞} a_{n_k} = A$. Z poznámky víme $c_{n_k} ≤ a_{n_k} ≤ b_{n_k}$, tedy podle V2.5 $\liminf\limits_{n \rightarrow ∞} a_n ≤ \lim_{k \rightarrow ∞} a_{n_k} ≤ \limsup\limits_{n \rightarrow ∞} a_n$.

            \end{dukazin}
        \end{veta}

% 27. 10. 2020

        \begin{dusledek}
            Nechť $\{a_n\}_{n = 1}^∞$ je posloupnost reálných čísel a $A \in ®R$. Pak
            \begin{itemize}
                \item $H(\{a_n\})≠\O$,
                \item $\liminf\limits_{n \rightarrow ∞} a_n = \min H(\{a_n\}_{n = 1}^∞), \limsup\limits_{n \rightarrow ∞} a_n = \max H(\{a_n\}_{n = 1}^∞)$,
                \item Je-li $\lim_{n \rightarrow ∞}a_n = A$, pak $H(\{a_n\}_{n = 1}^∞) = \{A\}$.
            \end{itemize}
        \end{dusledek}

        \begin{veta}[Bolzano-Cauchyova podmínka (T2.14)]
            Posloupnost $\{a_n\}_{n = 1}^∞$ má vlastní limitu právě tehdy, když splňuje Bolzano-Cauchyovu podmínku, tedy
            $$ \forall \epsilon > 0 \exists n_0 \in ®N \forall m, n ≥ n_0: |a_n - a_m| < \epsilon $$

            \begin{dukazin}
                $\implies$: Z $\lim_{n \rightarrow ∞} a_n = A \in ®R$ vyplývá $\frac{\epsilon}{2} > 0 \exists n_0 \in ®N \forall n≥n_0: |a_n - A| < \frac{\epsilon}{2}$. Tedy $\forall m,n \in ®N, m≥n_0, n≥n_0$ platí $|a_m - a_n| ≤ |a_m - A| + |a_n - A| < \frac{\epsilon}{2} + \frac{\epsilon}{2} = \epsilon$.

                Opačná implikace: Nechť $\{a_n\}$ splňuje Bolzano-Cauchyovu podmínku. Nechť $\epsilon > 0$. Pak $\exists n_0 \in ®N \forall m, n ≥ n_0: |a_n - a_m| < \epsilon$. Toto použiji pro $m = n_0$ a dostanu $a_{n_0} - \epsilon < a_n < a_{n_0} + \epsilon$. Tedy $a_n$ je omezená posloupnost a definujeme $b_n = \sup\{a_n, a_{n+1}, …\}$ a $c_n = \inf\{a_n, a_{n+1}, …\}$. Z definice $b_n$ a $c_n$ dostaneme $\forall n > n_0:$ $a_{n_0} - \epsilon ≤ c_n ≤ b_n ≤ a_{n_0} + \epsilon$. Tedy podle věty o uspořádání a …, $a_{n_0} - \epsilon ≤ \liminf\limits_{n \rightarrow ∞} a_n = \lim_{n \rightarrow ∞} c_n ≤ \lim_{n \rightarrow ∞} b_n = \limsup\limits_{n \rightarrow ∞} a_n ≤ a_{n_0} + \epsilon$.

                Odtud $\limsup\limits_{n \rightarrow ∞} a_n - \liminf\limits_{n \rightarrow ∞} a_n ≤ 2\epsilon$. Toto platí $\forall \epsilon > 0$, tedy $\limsup\limits_{n \rightarrow ∞} a_n = \liminf\limits_{n \rightarrow ∞} a_n \in ®R$. Podle V2.12 $\exists \lim_{n \rightarrow ∞} a_n \in ®R$.
            \end{dukazin}
        \end{veta}

        \begin{poznamka}[Organizační věci ke zkoušce]
                Písemná část (příklady, stačí udělat jednou i na více pokusů, 50b, minimum 25, libovolná literatura, žádná elektronika) $\rightarrow$ ústní část (teoretická, 40b (+10 pro jedničkáře), minimum 25b, lze se na ni dostat nejenom s úspěšnou písemnou částí, ale i s 3 neúspěšnými písemnými částmi).

                3 pokusy. (Jeden termín i v září.) Nezapomenout si s sebou doklad totožnosti.

                RADA 1 (písemná): Začněte příklady 1 (limita posloupnosti 10b), 2 (limita funkce 10b), 3 (průběh funkce 20b), až potom udělat 4 (teoretický příklad, 10b).

                Ústní část = klíčový pojem (0b), 3 definice nebo znění vět ($3\times 4$b = 12b), LV+důkaz (4+8b = 12b), TV+důkaz(4+12 = 16b). Nemá časový limit.

                RADA 2 (ústní): Musím umět důkazy? ANO! Stačí lehké? ANO!

                $<50$b :(, $\[50, 59\)$ (3), $≥60$b (bonusová otázka 10b): $≥70$b (2), $≥ 85$b (1). 
        \end{poznamka}

\section{Funkce jedné reálné proměnné - limita a spojitost}
    \subsection{Základní definice}
        \begin{definice}[Funkce jedné reálné proměnné]
            Funkcí jedné reálné proměnné rozumíme zobrazení $f: M \rightarrow ®R$, kde $M \subseteq ®R$.
        \end{definice}

        \begin{definice}[Sudá, lichá a periodická funkce]
                Řekněme, že funkce $f: M \rightarrow ®R, M\subseteq ®R$ je sudá, jestliže $\forall x \in M: (-x \in M) \land (f(x) = f(-x))$, je lichá, jestliže $\forall x \in M: (-x \in M) \land (f(x) = -f(-x))$, je periodická, jestliže $\exists p>0, \forall x \in M: (x+p \in M) \land (f(x) = f(x+p))$.
        \end{definice}

        \begin{definice}[Funkce omezená, omezená shora, omezená zdola]
            Řekněme, že funkce $f: M \rightarrow ®R$, $M \subseteq ®R$, je omezená (omezená shora, omezená zdola), jestliže $f(M)$ je omezená (omezená shora, omezená zdola).
        \end{definice}

% 30. 10. 2020

        \begin{definice}[Prstencové okolí bodu, okolí bodu (+ levé a pravé)]
            Nechť $\delta > 0$ a $a \in ®R$. Prstencové okolí bodu $\P(a, \delta) = (a-\delta, a+\delta) \setminus \{a\}$, $\P(+∞, \delta) = \(\frac{1}{\delta}, +∞\)$, $\P(-∞, \delta) = \(-∞, -\frac{1}{\delta}\)$.
            
            Pravé a levé prstencové okolí bodu $a$ je $\P_+(a, \delta) = (a, a+\delta), \P_-(a, \delta) = (a-\delta, a)$.


            Okolí bodu $\B(a, \delta) = (a-\delta, a+\delta)$, $\B(+∞, \delta) = \(\frac{1}{\delta}, +∞\)$, $\B(-∞, \delta) = \(-∞, -\frac{1}{\delta}\)$.

            Pravé a levé okolí bodu $a$ je $\B_+(a, \delta) = \[a, a+\delta\), \B_-(a, \delta) = \(a-\delta, a\]$.
        \end{definice}

        \begin{definice}[Limita funkce]
            Nechť $f: M \rightarrow ®R, M \subseteq ®R$. Řekneme, že $f$ má v bodě $a \in ®R*$ limitu rovnou $A \in ®R*$, jestliže platí
            $$ \forall \epsilon > 0 \exists \delta > 0 \forall x \in \P(a, \delta): f(x) \in \B(A, \epsilon),  $$
            značíme $\lim_{x \rightarrow a} f(x) = A$.

            \begin{poznamkain}
                Pro reálné $a$ lze zapsat definici podobně jako limitu posloupnosti.

                Definici lze také zapsat jako $\lim_{x \rightarrow a}f(x) = A ≡ \forall \epsilon > 0 \exists \delta > 0: f(\P(a, \delta)) \subseteq \B(A, \epsilon)$.
            \end{poznamkain}
        \end{definice}

        \begin{definice}[Limita zprava / zleva]
            Nechť $f: M \rightarrow ®R, M \subseteq ®R$. Řekneme, že $f$ má v bodě $a \in ®R$ limitu zprava (zleva) rovnou $A \in ®R*$, jestliže platí
            $$ \forall \epsilon > 0, \exists \delta > 0 \forall x \in \P_{+ (-)}: f(x) \in B(A, \epsilon), $$
            značíme $\lim_{x \rightarrow a_{+ (-)}} f(x) = A$.
        \end{definice}

        \begin{poznamka}
            $$ \lim_{x \rightarrow a} f(x) = A \Leftrightarrow \lim_{x \rightarrow a+} f(x) = A \land \lim_{x \rightarrow a-} f(x) = A $$
        \end{poznamka}

        \begin{definice}[Spojitost v bodě (+ zprava a zleva)]
            Nechť $f: M \rightarrow ®R, M \subseteq ®R, a \in M$. Řekneme, že $f$ je v bodě $a \in ®R$ spojitá (spojitá zprava, spojitá zleva), jestliže
            $$ \lim_{x \rightarrow a} f(x) = f(a) \(\lim_{x \rightarrow a+} f(x) = f(a), \lim_{x \rightarrow a-} f(x) = f(a)\) $$
            
            \begin{prikladyin}
                    $f(x) = x$ je spojitá. Dirichletova funkce (D($x$) je 1 pro $x \in ®Q$, je 0 pro $x \in ®R \setminus ®Q$) není spojitá (a nemá nikde limitu). Riemanova funkce ((R($x$) je $\frac{1}{q}$ pro $x \in ®Q, x = \frac{p}{q}, p \in ®Z, q \in ®N$ nesoudělná, je 0 pro $x \in ®R \setminus ®Q$) je spojitá v každém iracionálním čísle a nespojitá v každém racionálním.
            \end{prikladyin}
        \end{definice}

    \subsection{Věty o limitách}
        \begin{veta}[Heineho věta (T3.1)]
            Nechť $A \in ®R*, f: M \rightarrow ®R$ a $f$ je definována na prstencovém okolí bodu $a \in ®R*$. Následující podmínky jsou ekvivalentní (NPJE):\\
            (i) $\lim_{x \rightarrow a} = A$,\\
            (ii) pro každou posloupnost $\{x_n\}_{n = 1}^∞$ takovou, že $x_n \in M, \forall n \in ®N, x_n ≠ a, \lim_{n \rightarrow ∞} = a$ platí $\lim_{n \rightarrow ∞} f(x_n) = A$.

            \begin{dukazin}
                ($\implies$): Mějme posloupnost $\{x_n\}_{n = 1}^∞$ splňující předpoklady bodu (ii). Nechť $\epsilon > 0$ Podle (i) $\exists \delta > 0 \forall x \in \P(a, \delta): f(x) \in B(A, \epsilon)$. $\lim_{n \rightarrow ∞} x_n = a$, tedy k tomuto $\delta > 0 \exists n_0 \forall n≥ n_0: x_n \in \B(c, \delta)$. Dále $x_n ≠ a$, tedy $x_n \in \P(a, \delta)$. Tudíž dostáváme $f(x_n) \in B(A, \epsilon)$.
                ($\Leftarrow$): Dokažme $\neg (i) \implies \neg (ii)$. $\neg(i): \exists \epsilon > 0 \forall \delta > 0 \exists x \in \P(a, \delta): \neg(f(x) \in B(A, \epsilon))$. Toto použijeme pro $\delta_n = \frac{1}{n}, n \in ®N$. Dostaneme $\exists x_n \in \P(a, \frac{1}{n}): f(x_n) \notin B(A, \epsilon)$. Nyní $x_n \rightarrow a$, $\forall n \in ®N: x_n ≠ a \implies \{x_n\}$ splňuje podmínky (ii) a dostaneme $\lim_{n \rightarrow ∞} = A$. Toto je spor s $f(x_n) \notin \B(A, \epsilon)$.
            \end{dukazin}
        \end{veta}

% 3. 11. 2020

        \begin{poznamka}
            Existují i varianty pro limitu zprava a zleva a pro Spojitost
        \end{poznamka}

        \begin{veta}[O jednoznačnosti limity]
            Funkce $f$ má v bodě $a$ nejvýše jednu limitu.

            \begin{dukazin}
                    Sporem. Nechť $\lim_{x \rightarrow a} f(x) = A$ a $\lim_{x \rightarrow a} f(x) = B$, $A≠B$. Podle Heineho věty $\lim_{n \rightarrow ∞} f(x_n) = A$ a $\lim_{n \rightarrow ∞} f(x_n) = B$. Toto je spor s jednoznačností limity posloupnosti. $\lightning$.
            \end{dukazin}
        \end{veta}

        \begin{veta}[limita a omezenost]
            Nechť $f$ má vlastní limitu v bodě $a \in ®R$. Pak existuje $\delta > 0$ tak, že $f$ je na $\P(a, \delta)$ omezená.

            \begin{dukazin}
                Označme $A = \lim_{x \rightarrow a} f(x)$. Víme $A \in ®R$. Pro $\epsilon = 1$ nalezneme $\delta > 0$ tak, že
                $$ \forall x \in \P(a, \delta) : f(x) \in \B(A, 1) = (A-1, A+1). $$
                Tedy $f(\P(a, \delta)) \subseteq (A-1, A+1)$. Z toho plyne, že $f$ je na $\P(a, \delta)$ omezená.
            \end{dukazin}
        \end{veta}

        \begin{veta}[O aritmetice limit]
            Nechť $a \in ®R*$, $\lim_{x \rightarrow a} f(x) = A \in ®R*$ a $\lim_{x \rightarrow a} g(x) = B \in ®R*$. Pak platí (má li pravá strana smysl):
            $$ \lim_{x \rightarrow a} (f(x) + g(x)) = A+B $$
            $$ \\lim_{x \rightarrow a} f(x)·g(x) = A·B $$ 
            $$ \lim_{x \rightarrow a} \frac{f(x)}{g(x)} = \frac{A}{B} $$ 
            $$  $$ 
            \begin{dukazin}
                Heineho věta.
            \end{dukazin}
        \end{veta}

        \begin{dusledek}
            Nechť jsou funkce $f$ a $g$ spojité v bodě $a \in ®R$. Pak jsou funkce $f+g, f·g$ spojité v $a$. Pokud navíc $g(a) ≠ 0$, pak $f/g$ je spojitá v $a$.

            Speciálně polynomy jsou spojité na ®R a racionální lomené funkce $\frac{P(x)}{Q(q)}$ jsou spojité ve všech $x: Q(x)≠0$.
        \end{dusledek}

        \begin{veta}[O limitě a uspořádání a o dvou policajtech]
            Nechť $a \in ®R*$. Nechť $\lim_{x \rightarrow a} f(x) > \lim_{x \rightarrow a} g(x)$. Pak existuje prstencové okolí $\P(a, \delta)$ tak, že $\forall x \in \P(a, \delta): f(x) > g(x)$.

            Nechť existuje prstencové okolí bodu $\P(a, \delta)$ tak, že $\forall x \in \P(a, \delta): f(x)≤g(x)$. Nechť existují $\lim_{x \rightarrow a} f(x), \lim_{x \rightarrow a} g(x)$. Potom platí $\lim_{x \rightarrow a} f(x) ≤ \lim_{x \rightarrow a} g(x)$.

            Nechť na nějakém prstencovém okolí $\P(a, \eta)$ platí $f(x)≤ h(x) ≤ g(x)$. Nechť $\lim_{x \rightarrow a} f(x) = \lim_{x \rightarrow a} g(x)$. Pak existuje $\lim_{x \rightarrow a} h(x)$ a rovná se jim.

            \begin{dukazin}
                $\lim_{x \rightarrow a} f(x) > \lim_{x \rightarrow a} g(x) \implies f(x) > g(x)$ na $\P(a, \delta)$. Označme $A = \lim_{x \rightarrow a} f(x)$ a $B = \lim_{x \rightarrow a} g(x)$. Víme $A > B$. Nalezneme $\epsilon > 0$ tak, aby $\B(A, \epsilon) \cap \B(B, \epsilon) = \O$. Navíc $\forall c \in \B(A, \epsilon) \forall d \in \B(B, \epsilon)$ platí $c>d$. K tomuto $\epsilon$ nalezneme $\delta_1>0$ a $\delta_2>0$ tak, že $\forall x \in \P(c, \delta_1): f(x) \in \B(A, \epsilon)$ a $\forall x \in \P(c, \delta_2): g(x) \in \B(B, \epsilon)$. Položme $\delta = \min\{\delta_1, \delta_2\}$. Pak $\forall x \in \P(c, \delta): f(x) \in \B(A, \epsilon) \ land g(x) \in \B(B, \epsilon) \implies f(x) > g(x)$.

                Sporem z první části.

                Označme $A = \lim_{x \rightarrow a} f(x) = \lim_{x \rightarrow a} g(x)$. Nechť nejprve $A \in ®R$. Nechť $\epsilon > 0$. Nalezneme $\delta_1 > 0$ tak, že $\forall x \in \P(a, \delta_1)$ platí $A - \epsilon < f(x) < A + \epsilon$, $A - \epsilon < g(x) < A + \epsilon$. Nechť $\delta = \min\{\delta_1, \eta\}$. Pak $\forall x \in \P(a, \delta): h(x) \in \B(a, \epsilon)$.

                Pro $A = ±∞$ obdobně.
            \end{dukazin}
        \end{veta}

% 6. 11. 2020

        \begin{veta}[Limita složené funkce]
            Nechť funkce $f$ a $g$ splňují $\lim_{x \rightarrow a} g(x) = A$ a $\lim_{y \rightarrow A} f(x) = B$. Je-li navíc splněna alespoň jedna z podmínek: (S) $f$ je spojitá v $A$, (P) $\exists \eta > 0 \forall x \in \P(a, \eta): g(x)≠A$. Pak platí $\lim_{x \rightarrow a} f(g(x)) = B$.

            \begin{dukazin}
                (S) Nechť $\epsilon > 0$. Z $\lim_{y \rightarrow A} f(x) = B$. $\exists \psi > 0$ tak, že $f(\P(A, \psi)) \subseteq \B(B, \epsilon)$. $f$ je spojitá v $A \implies B = \lim_{y \rightarrow A} f(x) = f(A) \implies$ dokonce $f(\B(A, \psi)) \subseteq \B(B, \epsilon)$.

                K našemu $\psi > 0$ z $\lim_{x \rightarrow a} g(x) = A \exists \delta > 0: g(\P(a, \delta)) \subseteq \B(A, \psi)$. Nyní $f(g(\P(a, \delta))) \subseteq f(\B(A, \psi)) \subseteq \B(B, \epsilon) \implies \lim_{x \rightarrow a} f(g(x)) = B$.

                (P) Nechť $\epsilon > 0$. Z $\lim_{y \rightarrow A} f(x) = B$. $\exists \psi > 0$ tak, že $f(\P(A, \psi)) \subseteq \B(B, \epsilon)$. K $\psi>0$ z $\lim_{x \rightarrow a} g(x) = A \exists \delta > 0: g(\P(a, \delta)) \B(A, \psi)$. BÚNO $\delta < \eta$. Z (P) $g(x)≠A \forall x \in \P(a, \delta) \subseteq \P(a, \eta)$. Tedy dokonce $g(P(a, \delta)) \subseteq \P(A, \psi)$.

                Nyní $f(g(\P(a, \delta))) \subseteq f(\P(A, \psi)) \subseteq \B(B, \epsilon) \implies \lim_{x \rightarrow a} f(g(x)) = B$.
            \end{dukazin}
        \end{veta}

        \begin{veta}[Limita monotónní funkce]
            Nechť $f$ je monotónní na intervalu $(a, b)$, $a, b \in ®R*$. Potom existuje $\lim_{x \rightarrow a+} f(x)$ a $\lim_{x \rightarrow b-} f(x)$.

            \begin{dukazin}
                Nechť $f$ je neklesající, pro nerostoucí je důkaz analogický. Označme $m = \inf_{x\in (a, b)} f(x)$. Dokážeme $\lim_{x \rightarrow a+} f(x) = m$ v případě $m \in ®R$. Případ $m = -∞$ a důkaz pro $\lim_{x \rightarrow b-} f(x) = \sup_{x \in (a, b)}$ je analogický.

                Nechť $\epsilon > 0$. Z vlastnosti infima $\exists y \in f((a, b))$ tak, že $y < m+\epsilon$. Z definice $f((a, b)) \exists x' \in (a, b): f(x') = y$. $f$ je neklesající, a proto $\forall x \in (a, x'): f(x)≤f(x') = y < m + \epsilon$. $m$ je také dolní závora $f((a, b))$  tedy $\forall x \in (a, b): m-\epsilon < m ≤ f(x)$.

                Celkem $\forall x \in (a, x'): m-\epsilon < f(x) < m + \epsilon$. Tedy $\lim_{x \rightarrow a+} f(x) = m$.
            \end{dukazin}
        \end{veta}

        \begin{veta}[T3.8, Bolzano-Cauchyova podmínka pro funkce]
            Nechť $a \in ®R*$ a $\delta_0 > 0$. Nechť $f$ je funkce definovaná alespoň na $\P(a, \delta_0)$. Potom existuje vlastní $\lim_{x \rightarrow a} f(x)$ právě tehdy, když je splněna následující (tzv. Bolzano-Cauchyova) podmínka:
            $$ \forall \epsilon > 0 \exists \delta > 0 \forall x, y \in \P(a, \delta): |f(x) - f(y)| < \epsilon.$$ 

% 10. 11. 2020

            \begin{dukazin}
                $(\implies)$ Nechť $\epsilon > 0$ a $\lim_{x \rightarrow a} f(x) = A \in ®R$. Z definice limity $\exists \delta > 0 \forall x \in \P(o, \delta): f(x) \in \B(A, \epsilon)$. Toto $\delta$ použijeme pro BC podmínku a $\forall x, y \in \P(a, \delta): |f(x) - f(y)| ≤ |f(x) - A| + |A - f(x)| < \epsilon + \epsilon = 2\epsilon$. Toto je ekvivalentní BC podmínce.

                $(\Leftarrow)$ Použijeme BC podmínku pro posloupnosti a Heineho větu. Víme BC a chceme $\exists A \in ®R: \lim_{x \rightarrow a} f(x) = A$. Z heineho věty je tato „limita“ ekvivalentní s $\forall x_n \rightarrow a, x_n ≠ a: \lim_{n \rightarrow ∞} f(x_n) = A$. Nechť tedy $x_n \rightarrow a, x_n ≠ a$. Tvrdím, že $a_n = f(x_n)$ splňuje BC podmínku pro posloupnost. Nechť $\epsilon > 0$, nalezneme $\delta > 0$ z BC podmínky pro funkci. K tomuto $\delta > 0 \exists n_0 \forall n ≥ n_0: x_n \in \P(a, \delta)$, neboť $\lim_{n \rightarrow ∞} x_n = a$, tedy $\forall m, n ≥ n_0: |a_n - a_m| = |f(x_n) - f(x_m)| \overset{\text{z BC podmínky}}{<} \epsilon$. Tedy $a_n = f(x_n)$ splňuje BC podmínku pro posloupnosti, tedy $\exists \lim_{n \rightarrow ∞} a_n = \lim_{n \rightarrow ∞} f(x_n) \in ®R$.

                Nyní nechť $x_n \rightarrow a, x_n≠a$ a $y_n \rightarrow a, y_n ≠ a$. Podle předchozího $\exists A, B \in ®R: \lim_{n \rightarrow ∞} f(x_n) = A \land \lim_{n \rightarrow ∞} b(y_n) = B$. Nechť $x_1, y_1, x_2, y_2, …$ je nová posloupnost splňující podmínky. Pak existuje její limita $\implies A=B$ podle věty o limitě podposloupnost.
            \end{dukazin}
        \end{veta}

    \subsection{Funkce spojité na intervalu (3.3)}
        \begin{definice}[Vnitřní body]
            Vnitřními pody intervalu $J$ rozumíme ty body $J$, které nejsou krajními.
        \end{definice}

        \begin{definice}[Funkce spojitá na intervalu]
            Nechť $f$ je funkce a $J$ je interval. Řekneme, že $f$ je spojitá na $J$, jestliže je spojitá v každém vnitřním bodě. Je-li počáteční bod $J$ prvkem $J$, tak požadujeme i spojitost zprava v tomto bodě. Analogicky pro koncový bod požadujeme spojitost zleva.
        \end{definice}

        \begin{definice}[Darboux]
            Nechť $f$ je spojitá na intervalu $\[a, b\]$ a platí $f(a) < f(b)$. Pak pro každé $y \in \(f(a), f(b)\)$ existuje $x_0 \in (a, b)$ tak, že $f\(x_0\) = y$.

            \begin{poznamkain}
                Pro $f(a) > f(b)$ platí analogie.
            \end{poznamkain}

            \begin{dukazin}
                Nechť $y \in (f(a), f(b))$ a položme $M = \{z \in \[a, b\]: f(z) < y\}$. Množina $M$ je neprázdná (obsahuje např. $a$), shora omezená ($b$), tedy existuje $x_0 = \sup M$. Zřejmě $x_0 \in \[a, b\]$. Dokážeme, že $f(x_0) = y$ vyloučením případů:

                $f(x_0) > y$, potom ze spojitosti ($\lim_{x \rightarrow x_0} f(x) = f(x_0) > y$) plyne $\exists \delta > 0 \forall x \in \(x_0 - \delta, x_0\): f(x) > y$ ($0 < \epsilon < f(x_0) - y$). Tedy z definice $M$, že $(x_0-\delta, x_0)$ neleží v $M \implies x_0$ není nejmenší horní závora $M$, $\lightning$.

                $f(x_0) < y$, potom ze spojitosti plyne $\exists \delta \forall x \in \(x_0 - \delta, x_0 + \delta\): f(x) < y$, tedy $\exists x>x_0: f(x) < y$, tedy $x$ není horní závora, $\lightning$.

                Tedy $f(x_0) = y$ (zřejmě $b ≠ x_0 ≠ a$, tedy $x_0 \in \[a, b\]$).
            \end{dukazin}
        \end{definice}

        \begin{dusledek}
            Nechť $J$ je interval a funkce $f: J \rightarrow ®R$ je spojitá. Pak je $f(J)$ interval.
        \end{dusledek}

        \begin{definice}[Maximum (minimum)]
            Nechť $f: M \rightarrow ®R, m \subseteq ®R$. Řekneme, že funkce $f$ nabývá v bodě $a \in M$ maxima (resp. minima, ostrého maxima, ostrého minima) na $M$, jestliže $forall x \in M: f(x) \overset{(≥, <, >)}{≤} f(a)$ (samozřejmě při ostrých $x≠a$).

            Řekneme, že $f$ nabývá v bodě $a \in M$ lokálního maxima (ostrého lokálního maxima, ostrého likoálního minima, lokálního minima), jestliže $\exists \delta > 0$ tak, že $f$ nabývá na $M \cap \B(a, \delta)$ svého maxima (…).
        \end{definice}

        \begin{veta}[Spojitost funkce a nabývání extrémů]
            Nechť $f$ je spojitá na $\[a, b\]$. Pak $f$ nabývá na $\[a, b\]$ svého maxima a minima.

% 13. 11. 2020

            \begin{dukazin}
                Použijeme Bolzano-Weirstrassovu větu a Heineho větu pro spojitost.

                Označme $G = \sup f([a, b])$. Z definice suprema $\exists y_n \in f(\[a, b\])$ tak, že $y_n \rightarrow G$. Z definice $f(\[a, b\]) \exists x_n \in \[a, b\]: f(x_n) = y_n$. Podle Bolzano-Weirstrass $\exists x_{n_k} \rightarrow x* \in [a, b]$ (díky strážníkům). Podle Heineho věty (pro spojitost) $x_{n_k} \rightarrow x* \implies f(x_{n_k}) = y_{n_k} \rightarrow f(x*)$. Ale $y_n \rightarrow G \implies y_{n_k} \rightarrow G$.

                Tedy $G = f(x*) \implies$ v $x*$ je nabyto maximum. Minimum lze dokázat analogicky.
            \end{dukazin}
        \end{veta}

        \begin{dusledek}
            Nechť $f$ je spojitá funkce na intervalu $\[a, b\]$. Pak funkce $f$ je na $\[a, b\]$ omezená.
        \end{dusledek}

        \begin{definice}[Prostá a inverzní funkce na intervalu]
            Nechť $f$ je funkce a $J$ je interval. Řekneme, že $f$ je prostá na $J$, jestliže pro všechna $x, y \in J$ platí $x≠y \implies f(x) ≠ f(y)$.

            Pro prostou funkci $f: J \rightarrow ®R$ definujeme funkci $f^{-1}: f(J) \rightarrow ®R$ předpisem $f^{-1}(y) = x \Leftrightarrow f(x) = y$.
        \end{definice}

        \begin{veta}[O inverzní funkci]
            Nechť $f$ je spojitá a rostoucí (klesající) funkce na intervalu $J$. Potom je $f^{-1}$ spojitá a rostoucí (klesající) na intervalu $f(J)$.

            \begin{dukazin}
                BÚNO $f$ je spojitá a rostoucí. Víme, že $f^{-1}$ je definováno na $f(J)$. Tvrdím, že $f^{-1}$ je rostoucí: Sporem: Nechť $y_1 = f(x_1) < y_2 = f(x_2)$, ale $f^{-1}(y_1) = x_1 ≥ f^{-1}(y_2) = x_2$. Pak ale $x_1 ≥ x_2 \overset{\text{$f$ rostoucí}}{\implies} f(x_1) = y_1 ≥ f(x_2) = y_2 \lightning$.

                $y_0 \in f(J)$ vnitřní bod: spojitost: Víme, že $f^{-1}(y_0) = x_0$ a $x_0$ je vnitřní bod $J$. Nechť $\epsilon > 0$. Existují $x_1, x_2 \in \(x_0 - \epsilon, x_0 + \epsilon\) \cap J$. Pak $f(x_1) < f(x_0) = y_0 < f(x_2)$. Zvolme $\delta = \min \{f(x_2) - f(x_0), f(x_0) - f(x_1)\}$. Pak $\B(f(x_0), \delta) = \B(y_0, \delta) \subseteq (f(x_1), f(x_2))$. Nyní $f^{-1}(\B(y_0, \delta)) \subseteq (x_1, x_2) \subseteq (x_0 - \epsilon, x_0 + \epsilon) \implies \lim_{y \rightarrow y_0} f^{-1}(y) = f^{-1}(y_0)$.

                $y_0 \in f(J)$ levý krajní bod: spojitost zprava: $f^{-1}(y_0) = x_0$ a $x_0$ je levý krajní bod $J$. Nechť $\epsilon > 0$. Existuje $x_1 \in (x_0, x_0 + \epsilon)$ tak, že $x_0 < x_1 \implies f(x_0) < f(x_1)$. Položme $\delta = f(x_1) - f(x_0)$, pak $\B_+(y_0, \delta) = \[y_0, y_0 + \delta\) = \[y_0, f(x_1)\)$. Nyní $f^{-1}\(\B_+(y_0, \delta)\) = \[x_0, x_1\) \subset (x_0 - \epsilon, x_0 + \epsilon)$.

                $y_0 \in f(J)$ pravý krajní bod: spojitost zleva: Analogicky.
            \end{dukazin}
        \end{veta}

        \begin{priklady}
            Funkce $x \rightarrow x^n$ je spojitá a rostoucí na $\[0, ∞\)$, proto je i funkce $x \rightarrow \sqrt[n]{x}$ spojitá a rostoucí na $\[0, ∞\)$.
        \end{priklady}

    \subsection{Elementární funkce (3.4)}
        \begin{veta}[Zavedení exponenciely]
            Existuje funkce $\exp : ®R \rightarrow ®R$ splňující: a) $\exp(x)$ je rostoucí na ®R, b) $\forall x, y \in ®R: \exp(x+y) = \exp(x)·\exp(y)$, c) $\exp(0) = 1$, d) $\lim_{x \rightarrow 0} \frac{\exp(x) - 1}{x} = 1$, e) $\exp(x)$ je spojitá na ®R.

            \begin{poznamkain}[Další možnosti zavedení]
                $$ e^x = \lim_{n \rightarrow ∞}\(1+\frac{x}{n}\)^n $$ 
                $$ \log x = \int_1^x \frac{1}{y}\, dy $$
                $$ e^x = \lim_{k \rightarrow ∞} \sum_{n=1}^k \frac{x^n}{n!} = \sum_{n=0}^∞ \frac{x^n}{n!} $$ 
            \end{poznamkain}

% 20. 11. 2020

            \begin{dukazin}
                Položme $\exp x = \lim_{k \rightarrow ∞} \sum_{n = 0}^k \frac{x^n}{n!}$. Nejdříve chceme ukázat $\exists \lim_{k \rightarrow ∞} \sum_{n=0}^k \frac{x^n}{n!} \in ®R$ pro každé $x \in ®R$. Nechť $x$ je pevné.\footnote{$\exists \lim_{k \rightarrow ∞} \frac{1}{n·(n-1)} = \lim_{k \rightarrow ∞} 1 - \frac{1}{k} = 1$. Pak existuje $k_0 > |x|$ ($k_0 = \lfloor x \rfloor + 1$). $\forall n ≥ 2k_0$ platí
                        $$ \left|\frac{x^n}{n!}\right| = |x|^{k_0}·\frac{1}{k_0!} · \frac{|x|}{k_0 + 1} · … · \frac{|x|}{n} ≤ |x|^{k_0}\frac{|x|^2}{(n-1)·n}. $$}

                Podle věty Bolzano-Cauchyova podmínka pro posloupnost stačí ověřit BC podmínku. Nechť $\epsilon > 0$, zvolme $n_0 ≥ 2k_0 \land \frac{1}{n_0} < \epsilon$, kde $k_0 = \lfloor|x|\rfloor + 1$. Nechť $m, k ≥ n_0$. BÚNO $m > k$:
                $$ \left| \sum_{n=0}^m \frac{x^n}{n!} - \sum_{n = 0}^k \frac{x^n}{n!} \right| = \left| \sum_{n=k+1}^m \frac{x^n}{n!} \right| ≤ \sum_{n=k+1}^m \left| \frac{x^n}{n!} \right| ≤ \sum_{n=k+1}^m |x|^{k_0 + 2} · \frac{1}{n·(n-1)} = |x|^{k_0+2} · \frac{1}{k} < |x|^{k_0 + 2} · \epsilon, $$
                což je přesně BC podmínka, tedy naše posloupnost je dobře definována.

                a) $0 ≤ x ≤ y \implies \frac{x^n}{n!} ≤ \frac{y^n}{n!} \implies \sum_{n = 0}^k \frac{x^n}{n!} ≤ \sum_{n = 0}^k \frac{y^n}{n!} \implies \lim_{k \rightarrow ∞} \sum_{n = 0}^k \frac{x^n}{n!} ≤ \lim_{k \rightarrow ∞} \sum_{n = 0}^k \frac{y^n}{n!} \implies \exp x ≤ \exp y$.

                $x ≤ y ≤ 0 \implies -x ≥ -y ≥ 0 \implies \exp(-x)  ≥ \exp(-y) \overset{\text{z b), viz dále}}{\implies} \exp x = \frac{1}{\exp (-x)} ≤ \frac{1}{\exp(-y)} = \exp y$.
                
                $x ≤ 0 ≤ y \implies \exp x ≤ \exp 0 ≤ \exp y$

                b)\footnote{Myšlenka: $$ \exp (x + y) = \sum_{k=0}^∞ \frac{1}{k!}·(x+y)^k = \sum_{k=0}^∞ \frac{1}{k!}· \sum_{j=0}^k \binom{k}{j}·x^{k-j}·y^j = \sum_{k=0}^∞ \sum_{j=0}^k \frac{x^{k-j}}{(k-j)!}·\frac{y^j}{j!} \overset{\text{chceme}}{=} \(\sum_{i=0}^∞ \frac{x^i}{i!}\) · \(\sum_{j=0}^∞ \frac{y^j}{j!}\), $$ ale poslední rovnost nemůžeme, protože nekonečně mnoho členů nemůžeme přeuspořádat.} Označme $s_n = \sum_{i=0}^n \frac{1}{i!}x^i \overset{n\rightarrow ∞}{\rightarrow} s = \exp x$, $\sigma_n = \sum_{j=0}^n \frac{1}{j!}y^j \overset{n\rightarrow ∞}{\rightarrow} \sigma = \exp y$, $\rho_n = \sum_{k=0}^n \frac{(x+y)^k}{k!} = \sum_{k=0}^n \sum_{j=0}^k \frac{x^{k-j}}{(k-j)!}·\frac{y^j}{j!} \overset{n\rightarrow ∞}{\rightarrow} \rho = \exp (x + y)$.

                Nechť $\epsilon > 0$. Pak z BC podmínky $\exists n_0$, že $\sum_{i = n_0}^∞ \frac{|x|^i}{i!} < \epsilon$, $\sum_{j = n_0}^∞ \frac{|y|^j}{j!} < \epsilon$ a zároveň $|s_{n_0}·\sigma_{n_0} - s·\sigma| < \epsilon$. Nechť $n ≥ 2n_0$, pak
                $$|\rho_n - s·\sigma| ≤ |\rho_n - s_{n_0}\sigma_{n_0}| + |s_{n_0}\sigma_{n_0} - s·\sigma| ≤ \sum_{j=0}^∞ \sum_{i=n_0}^∞ \frac{|x|^j}{j!}·\frac{|y|^i}{i!} + \sum_{j=n_0}^∞ \sum_{i=0}^∞ \frac{|x|^j}{j!}·\frac{|y|^i}{i!} + \epsilon ≤, $$
                jelikož všechny členy, které se neodečetly, mají alespoň jeden index $≥ n_0$ jinak se vyskytují v $s_{n_o}\sigma_{n_0}$. Pokračujeme:
                $$ ≤ \sum_{j=0}^∞ \frac{|x|^j}{j!}· \sum_{i=n_0}^∞ \frac{|y|^i}{i!} + \sum_{j=n_0}^∞ \frac{|x|^j}{j!}· \sum_{i=0}^∞ \frac{|y|^i}{i!} + \epsilon < e^{|x|}·\epsilon + \epsilon e^{|y|} + \epsilon. $$ 

                c) $\exp 0 = \lim_{k \rightarrow ∞} \sum_{n=0}^k \frac{0^n}{n!} = \lim_{k \rightarrow ∞} 1 = 1$.

                d) $\lim_{x \rightarrow 0} \frac{\exp x - 1}{x} = \lim_{x \rightarrow 0} \frac{\sum_{n=1}^∞ \frac{x^n}{n!}}{x} = \lim_{x \rightarrow 0} 1 + x\(\sum_{n=2}^∞ \frac{x^{n-2}}{n!}\) \overset{\text{další rovnost}}{=} 1$
                $$ \left|x·\sum_{n=2}^∞ \frac{x^{n-2}}{n!}\right| \overset{|x|≤1}{≤} |x|·\sum_{n=2}^∞ \frac{1}{n!} ≤ |x|·e \rightarrow 0 $$

                e) Chceme $ \lim_{h \rightarrow 0} \exp (x + h) = \exp x $ (tj. spojitost v $x$).
                $$ \lim_{h \rightarrow 0} \exp (x + h) - \exp x + \exp x = \lim_{h \rightarrow 0} \frac{\exp (x)·\exp(h) - \exp x}{h}·h + \exp x = \lim_{h \rightarrow 0} \exp(x)·\frac{\exp(h) - 1}{h}·h + \exp x \overset{\text{AL}}{=} \exp (x)·1·0 + \exp x = \exp x. $$ 
            \end{dukazin}
        \end{veta}

% 13. 11. 2020

        \begin{poznamka}[Vlastnosti exp]
            \ 
            \begin{itemize}
                \item $\exp(n·x) = (\exp x)^n$ (Matematická indukce).
                \item $\exp x > 0 \forall x \in ®R$ ($\exp x = \(\exp \frac{x}{2}\)^2 ≥ 0$, kdyby $\exp x = 0$, pak $\exp \frac{x}{2^i} = 0 \implies \exp 0 = 0 \lightning$).
                \item Z a) je exponenciela rostoucí rostoucí $\implies \lim_{x \rightarrow ∞} \exp x$ existuje. $\lim_{n \rightarrow ∞} \exp(n) = \lim_{n \rightarrow ∞} \(\exp 1\)^n = +∞ \implies \lim_{x \rightarrow a} \exp(x) = +∞$.
                \item $\lim_{x \rightarrow -∞} \exp(x) = 0$ ($ = \lim_{x \rightarrow -∞} \exp(x)·\frac{\exp -x}{\exp -x} = \lim_{x \rightarrow -∞} \frac{\exp 0}{\exp -x} = \frac{1}{+∞} = 0$).
                \item $\exp ®R = \(0, ∞\)$ (funkce nabývá mezihodnot).
            \end{itemize}
        \end{poznamka}

% 20. 11. 2020

        \begin{priklad}[$e$ je iracionální]
            \begin{reseni}
                Nechť pro spor $e = \frac{p}{q}$, kde $p, q \in ®N$ nesoudělná.
                $$ \sum_{n=0}^q \frac{1}{n!} < e = \frac{p}{q} = \sum_{n=0}^q \frac{1}{n!} + \sum_{n=q+1}^∞ \frac{1}{n!}. $$
                $$ \sum_{n=q+1}^∞ \frac{1}{n!} ≤ \sum_{n=q+1}^∞ \frac{1}{(q+1)!}·\frac{1}{(q+1)^{n-(q+1)}} = \frac{1}{(q+1)!} · \sum_{j=0}^∞ \frac{1}{(q+1)^j} = \frac{1}{(q+1)!}·\frac{1}{1-\frac{1}{q+1}} = \sum_{n=0}^q \frac{1}{n!} + \frac{1}{q!·q}. (/·q!q) $$ 
                $$ m < pq! < m+1,\ \ m \in ®N, \lightning. $$
            \end{reseni}
        \end{priklad}

% 24. 11. 2020

        \begin{definice}[Logaritmus]
            Funkci inverzní k exponenciele ($\exp$) je logaritmus ($\log$)
        \end{definice}

        \begin{veta}[Vlastnosti logaritmu]
            Funkce $\log$ splňuje: a) $\log: \(0, ∞\) \rightarrow ®R$ je spojitá a rostoucí funkce, b) $\forall x, y > 0: \log(x·y) = \log x + \log y$, c) $\lim_{x \rightarrow 1} \frac{\log x}{x-1} = 1$.
            \begin{dukazin}
                a) $\exp$ je spojitá a rostoucí $\implies$ existuje inverzní funkce, $\exp: ®R \rightarrow (0, ∞) \implies \log (0, ∞) \rightarrow ®R$. Podle věty o inverzní funkci je $\log$ spojitá a rostoucí funkce.

                b) $\log x = A, \log y = B \Leftrightarrow \exp A = x, \exp B = y$. Z definice exponenciely $x·y = \exp A · \exp B = \exp(A + B) \implies \log(x·y) = A + B = \log x + \log y$.

                c) označme $f(y) = \frac{\exp y - 1}{y}, y ≠ 0$ a $g(x) = \log x$, pak $g(x) ≠ 0 \forall x ≠ 1$ víme $\lim_{y \rightarrow 0} f(y) = 1$ z definice exponenciely $\lim_{x \rightarrow 1} g(x) = 0$. Podle VOLSF (P) dostaneme $1 = \lim_{x \rightarrow 1} f(g(x)) = \lim_{x \rightarrow 1} \frac{\exp(\log x) - 1}{\log x} = \lim_{x \rightarrow 1} \frac{x-1}{\log x}$.
            \end{dukazin}
        \end{veta}

        \begin{definice}
            Nechť $a>0$ a $b \in ®R$. Pak definujeme $a^b = \exp(b\log(a))$. Je-li $b>0$, pak definujeme $\log_b a = \frac{\log a}{\log b}$.
            \begin{poznamkain}
                $x·x = x^2 = x^2 = \exp(2·\log x) = \exp(\log x)·\exp(\log x) = x·x$. Obdobně pro $x^n, n \in ®N$.
            \end{poznamkain}
        \end{definice}

        \begin{priklad}
            $\lim_{n \rightarrow ∞} \(1 + \frac{1}{n}\)^n \overset{\text{Heine}}{=} \lim_{x \rightarrow 0} (1+x)^{\frac{1}{x}} = \lim_{x \rightarrow 0} e^{\frac{1}{x}\log(1+x)} \overset{\text{VOLSF (S)}}{=} e^1 = e$.
        \end{priklad}

        \begin{veta}[Zavedení sinu a cosinu]
            Existují funkce $\sin: ®R \rightarrow ®R$ a $\cos: ®R \rightarrow ®R$ splňující: a) $\forall x, y \in ®R: \sin(x+y) = \sin x · \cos y + \cos x · \sin y$, $\cos(x+y) = \cos x · \cos y - \sin x · \sin y$, $\cos(-x) = \cos x$, $\sin(-x) = - \sin x$, b) existuje kladné číslo $\pi$ tak, že $\sin$ je rostoucí na $\[0, \frac{\pi}{2}\]$ a $\sin\(\frac{\pi}{2}\) = 1$, c) $\lim_{x \rightarrow 0} \frac{\sin x}{x} = 1$.

            \begin{dukazin}[Bez důkazu, jen nástřel]
                $\sin x = x - \frac{x^3}{3!} + \frac{x^5}{5!} + … = \sum_{n=0}^∞ \frac{x^{2n+1}}{(2n+1)!}·(-1)^n$.
                $\cos x = 1 - \frac{x^2}{2!} + \frac{x^4}{4!} + … = \sum_{n=0}^∞ \frac{x^{2n}}{(2n)!}·(-1)^n$.
            \end{dukazin}
        \end{veta}

        \begin{definice}[tan a cotg]
            Pro $x \in ®R \setminus \{\frac{\pi}{2} + k \pi, k \in ®Z\}$ a $y \in ®R \setminus \{k \pi, k \in ®Z\}$ definujeme funkce tangens a cotangens předpisem
            $$ \tan x = \frac{\sin x}{\cos x},\ \cotg x = \frac{\cos y}{\sin y}. $$ 
        \end{definice}

        \begin{veta}[Spojitost sinu a cosinu]
            Funkce $\sin$, $\cos$, $\tan$, $\cotg$ jsou spojité na svém definičním oboru.
            \begin{dukazin}
                a) $\sin$, pak b) $\cos x = \sin\(\frac{\pi}{2} - x\)$ a složení spojitých funkcí je spojitá funkce, c) $\tan x = \frac{\sin x}{\cos x}$ a $\cotg x = \frac{\cos x}{\sin x}$ jsou spojité z aritmetiky limit.

                a) $\lim_{x \rightarrow a} \sin x = \sin a$: 1. v 0: $\lim_{x \rightarrow 0} \sin x = \lim_{x \rightarrow 0} \frac{\sin x}{x} · x \overset{\text{AL}}{=} 1·0 = \sin 0$. 2. $\lim_{x \rightarrow a} (\sin x - \sin a) = \lim_{x \rightarrow a} f(x) 2· \sin \frac{x-a}{2} · \cos \frac{x+a}{2} \overset{\text{AL}}{=} 2· \lim_{x \rightarrow a} \frac{\sin \frac{x-a}{2}}{\frac{x-a}{2}} · \lim_{x \rightarrow a} \cos \frac{x+a}{2} \overset{\text{VOLSF (P)}}{=} 2 · 0 · \cos a = 0$.
            \end{dukazin}
        \end{veta}

        \begin{definice}[Inverzní funkce]
            Nechť $\sin* x = \sin x$ pro $x \in \[-\frac{\pi}{2}, \frac{\pi}{2}\]$. Nechť $\cos* x = \cos x$ pro $x \in \[0, \pi\]$. Nechť $\tan* x = \tan x$ pro $x \in \(-\frac{\pi}{2}, \frac{\pi}{2}\)$. Nechť $\cotg* x = \cotg x$ pro $x \in \(0, \pi\)$.

            Definujeme $\arcsin$ (respektive $\arccos$, $\arctan$, $\arccotg$) jako inverzní funkce k $\sin*$ (respektive $\cos*$, $\tan*$, $\cotg*$).
        \end{definice}

    \subsection{Derivace a Taylorův polynom}
        \begin{definice}[Derivace]
            Nechť $f$ je reálná funkce a $a \in ®R$. Pak derivací $f$ v bodě $a$ budeme rozumět
            $$ f'(a) = \lim_{h \rightarrow 0} \frac{f(a+h) - f(a)}{h}, $$ 
            derivací $f$ v bodě $a$ zprava budeme rozumět
            $$ f_+'(a) = \lim_{h \rightarrow 0+} \frac{f(a+h) - f(a)}{h} $$
            a derivací $f$ v bodě $a$ zleva budeme rozumět
            $$ f_-'(a) = \lim_{h \rightarrow 0-} \frac{f(a+h) - f(a)}{h}. $$
        \end{definice}

        \begin{poznamka}
            Limita buď existuje a pak je vlastní nebo nevlastní, nebo vůbec neexistuje.

            Ekvivalentní definice je
            $$ f'(a) = \lim_{x \rightarrow a} \frac{f(x) - f(a)}{x-a}. $$

            $$ f'(a) = A \Leftrightarrow (f_+'(a) = A \land f_-'(a) = A). $$ 
        \end{poznamka}

        \begin{veta}[Vztah derivace a spojitosti]
            Nechť má funkce $f$ v bodě $a \in ®R$ derivaci $f'(a) \in ®R$. Pak je $f$ v bodě $a$ spojitá.
            \begin{dukazin}
                $$ \lim_{x \rightarrow a} f(x) = \lim_{x \rightarrow a} \frac{f(x) - f(a)}{x - a} (x-a) + f(a) \overset{\text{AL}}{=} f'(a)·0 + f(a) = f(a). $$ 
            \end{dukazin}
        \end{veta}

        \begin{veta}[Aritmetika derivací (T4.2)]
            Nechť $f'(a)$ a $g'(a)$ existují. Pak (pokud mají pravé strany smysl)
            $$ (f+g)'(a) = f'(a) + g'(a). $$
            Nechť $g$ je spojitá v $a$, pak 
            $$(f·g)'(a) = f'(a)·g(a) + f(a)·g'(a).$$
            Nechť je $g$ spojitá v $a$ a $g(a)≠0$, pak
            $$ \(\frac{f}{g}\)'(a) = \frac{f'(a)·g(a) - f(a)·g'(a)}{g^2(a)}. $$
            \begin{dukazin}
                $$ (f+g)'(a) = \lim_{h \rightarrow 0} \frac{f(a+h) + g(a+h) - (f(a)+g(a))}{h} = \lim_{h \rightarrow 0} \frac{f(a+h) - f(a)}{h} + \lim_{h \rightarrow 0} \frac{g(a+h) - g(a)}{h} = f'(a) + g'(a). $$

                $$ (f·g)'(a) = \lim_{h \rightarrow 0} \frac{f(a+h)·g(a+h) - f(a)·g(a) - f(a)·g(a+h) + f(a)·g(a+h)}{h} = \lim_{h \rightarrow 0} \frac{f(a+h)·g(a+h) - f(a)·g(a+h)}{h} + \lim_{h \rightarrow 0} \frac{f(a)·g(a+h) - f(a)·g(a)}{h} = \lim_{h \rightarrow 0} g(a+h)·\lim_{h \rightarrow 0} \frac{f(a+h) - f(a)}{h} + \lim_{h \rightarrow 0} f(a)·\lim_{h \rightarrow 0} \frac{g(a+h) - g(a)}{h} \overset{g\text{ spojitá}}{=} g(a)·f'(a) + g'(a)·f(a). $$

                Víme, že $g(a) ≠ 0$ a $g$ je spojitá v $a$. Tedy $\exists \delta > 0$ tak, že $g(a+h)≠0 \forall h \in \B(0, \delta)$.
                $$ \(\frac{f}{g}\)'(a) = \lim_{h \rightarrow 0} \frac{\frac{f(a+h)}{g(a+h) - \frac{f(a)}{g(a)}}}{h} = \lim_{h \rightarrow 0} \frac{f(a+h)·g(a) - f(a)·g(a+h)}{g(a+h)·g(a)·h} \overset{\text{AL}}{=} \lim_{h \rightarrow 0} \frac{1}{g(h+a)·g(a)} · \lim_{h \rightarrow 0} \frac{f(a+h)·g(a) - f(a)·g(a+h) + f(a)·g(a) - f(a)·g(a)}{h} \overset{g\text{ spojitá}}{=} \frac{1}{g^2(a)}·(\lim_{h \rightarrow 0} g(a) · \lim_{h \rightarrow 0}\frac{f(a+h)-f(a)}{h} + \lim_{h \rightarrow 0}f(a)·\lim_{h \rightarrow 0} \frac{g(a) - g(a+h)}{h})  = \frac{f'(a)·g(a) - f(a)·g'(a)}{g^2(a)} $$  
            \end{dukazin}
        \end{veta}

        \begin{veta}[Derivace složené funkce]
            Nechť má $f$ derivaci v bodě $y_0$, $g$ má derivaci v $x_0$, je v $x_0$ spojitá a $y_0 = g(x_0)$. Pak
            $$ (f\circ g)'(x_0) = f'(y_0)·g'(x_0) = f'(g(x_0))·g'(x_0). $$ 
            \begin{dukazin}
                Myšlenka:
                $$ \lim_{h \rightarrow 0} \frac{f(g(x_0 + h)) - f(g(x_0))}{h} = \lim_{h \rightarrow 0} \frac{f(g(x_0 + h)) - f(g(x_0))}{g(x_0 + h) - g(x_0)}·\frac{g(x_0 + h) - g(x_0)}{h} \overset{\text{VOLSF}}{=} \lim_{y \rightarrow y_0} \frac{f(y) - f(y_0)}{y - y_0} · g'(x_0) = f'(g(x_0)) · g'(x_0)  $$ 
            \end{dukazin}
        \end{veta}
\end{document}
