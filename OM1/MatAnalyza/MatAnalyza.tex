\documentclass[12pt]{article}					% Začátek dokumentu
\usepackage{../../MFFStyle}					    % Import stylu



\begin{document}
	
\section*{Organizační úvod}
    \begin{poznamka}[složení předmětu]
        Předmět má přednášku, cvičení a proseminář z Matematické analýzy.
    \end{poznamka}


    \begin{poznamka}[Motivace]
        TODO  
    \end{poznamka}

    \begin{poznamka}[Jak studovat]
        Studujte průběžně, ptejte se…   
    \end{poznamka}

    \begin{poznamka}[Literatura]
        \ 
        \begin{itemize}
            \item skripta -- viz homepage
            \item příklady -- Koláček \&\ spol. -- Příklady z matematické analýzy pro fyziky 1, 2, 3
            \item další příklady -- viz homepage
        \end{itemize}        
    \end{poznamka}

\section*{Motivace}
    \begin{poznamka}[Používá se v]
        \ 
        \begin{itemize}
            \item Fyzice (viz pohyb planet, např. Proč obíhají planety po elipsách?)
            \item Ekonomii (viz úroky, např. Lze předpovědět změny úroků v ekonomice a na burze?)
            \item Meteorologii (např. Jak ovlivňují teplé mořské proudy počasí v Evropě?)
            \item Lékařství (např. Jak dlouho bude koncentrace léku v krvi na účinné úrovni?)
            \item Epidemiologie (např. Proč se epidemie šíří nejprve rychle a potom pomalu?)
            \item Architektura (viz mosty (vibrace), např. Jak se ujistit, že most nespadne v bouři?)
            \item Inženýrství (viz letadla, např. Jak zkonstruovat nové křídlo pro letadlo? (Většina peněz i času při konstrukci jsou simulace.))
            \item Informatika (viz MP3, JPEG, např. Jak uchovat informaci o dané funkci v co nejmenším počtu čísel? (Používá se Fourierova transformace = převod funkce na siny a cosiny))
        \end{itemize}
    \end{poznamka}

    \begin{poznamka}[Proč se analýzu učíme my?]
        \ 
        \begin{itemize}
            \item Abychom uměli látku (hledat extrémy funkcí, umět integrovat)
            \item Měli solidní matematické základy
            \item Přečíst návod, jak používat nějaké matematické modely (např. Uplatnění v bance -- MFF nese mnoho peněz (nejvíce žádané práce jsou práce stylu statistik))
            \item Myslet, analyzovat, nedělat chyby (nezapomínat na další možnosti)
            \item Abychom se naučili způsob myšlení -- matematici / matematičky dělají činnosti lépe (ze dvou lidí, co ji neumí, je matematik ten, kdo ji zvládne lépe, nikoliv ten druhý.)
        \end{itemize}
    \end{poznamka}

\section{Úvod}
    Dosti možná spíše opakování střední
    \subsection{Výroky}
        \begin{definice}[Výrok]
            Tvrzení, o kterém má smysl říct, že je pravdivé, či ne.
            \begin{prikladyin}
                „Obloha je modrá.“\\
                „Vídeň je hlavní město ČR.“
            \end{prikladyin}
        \end{definice}
        \begin{poznamka}[Vytváření nových výroků]
            Děje se pomocí spojek a, nebo, implikací, ekvivalencí, atd.
            
                 \begin{tabular}{ccccccc}
                            $A$ & $B$ & \begin{tabular}[c]{@{}c@{}}konjunkce\\ $A\&B$\end{tabular} & \begin{tabular}[c]{@{}c@{}}disjunkce\\ $AvB$\end{tabular} & \begin{tabular}[c]{@{}c@{}}implikace\\ $A\implies B$\end{tabular} & \begin{tabular}[c]{@{}c@{}}ekvivalence\\ $A\Leftrightarrow B$\end{tabular} & \begin{tabular}[c]{@{}c@{}}negace $A$\\ $\neg A$\end{tabular} \\
                            0 & 0 & 0 & 0 & 1! & 1 & 1 \\
                            0 & 1 & 0 & 1 & 1! & 0 & 1 \\
                            1 & 0 & 0 & 1 & 0 & 0 & 0 \\
                            1 & 1 & 1 & 1 & 1 & 1 & 0
                \end{tabular}

            $A \implies B$ = $A$ je postačující podmínka pro $B$ = $B$ je nutná podmínka pro $A$.
            \begin{prikladyin}[Pravdivé výroky]
                $ 1=2 \implies 2=3 $\\
                já jsem papež $ \implies $ všechna letadla jsou modrá \\
            \end{prikladyin}

            \begin{prikladin}
                \ 
                \begin{itemize}
                    \item $$ (A \implies B) \Leftrightarrow (\neg B \implies \neg B $$
                    \item $$ (A \implies B) \Leftrightarrow (\neg (A \& \neg B)) $$ 
                    \item $$ \neg (A \& B) \Leftrightarrow (\neg A \lor \neg B) $$
                    \item $$ \neg (A \lor B) \Leftrightarrow (\neg A \& \neg B) $$                    
                    \item $$ (A \Leftrightarrow B) \Leftrightarrow ((A \implies B) \& (B \implies A) $$ 
                \end{itemize}
            \end{prikladin}
        \end{poznamka}
        \begin{definice}[Kvantifikátory]
            Dále existují tzv. kvantifikátory: Obecný (= pro všechna) $\forall$ a Existenční (= existuje) $\exists$.
        \end{definice}
            
        \begin{umluva}
            $\forall x \in \N, x>10 \ A(x)$ značí  $\forall x \in \N (x>10 \implies A(x))$
        \end{umluva}
        \begin{priklady}
            \ 
            \begin{itemize}
                \item Pro všechna $x \in M$ platí $A(x)$ je: $\forall x \in M: A(x)$ 
                \item Existuje $x \in M$ tak, že platí $A(x)$ je $\exists x \in M: A(x)$
                \item $ \forall n \in \N\ \forall m \in \N\ \exists k \in \N: k> n+m $
            \end{itemize}
        \end{priklady}

        \begin{priklady}[Negace výroků]
            \ 
            \begin{itemize}
                \item $\neg (\forall x \in M: A(x)) \Leftrightarrow \exists x \in M: \neg A(x)$
                \item $ \neg (\exists x \in M: A(x) \Leftrightarrow \forall x \in M: \neg A(x)$
                \item $\neg$(Nikdo mě nemá rád.) $\Leftrightarrow$ Existuje alespoň jeden člověk, který mě má rád. 
                \item $\neg(\forall n \in \N\ \forall m \in \N\ \exists k \in \N: k > n+m)$
                    $$ \exists n \in \N\ \neg(\forall m \in \N\ \exists k \in \N: k > n+m) $$
                    $$ \exists n \in \N\ \exists m \in \N\ \neg(\exists k \in \N: k > n+m) $$
                    $$ \exists n \in \N\ \exists m \in \N\ \forall k \in \N: k \leq n+m) $$ 


            \end{itemize}
        \end{priklady}

        \begin{upozorneni}
            Na pořadí kvantifikátorů záleží!
            \begin{prikladyin}
                M…Muži\\
                Ž…Ženy\\
                L(m, ž)…muži m se líbí žena ž
                $$ \forall m \in M\ \exists ž \in Ž: L(m, ž) $$
                $$ \exists ž \in Ž\ \forall m \in M: L(m, ž) $$
                První je, že pro každého muže existuje nějaká žena, která se mu líbí, naopak druhá říká, že existuje žena, která se líbí všem mužům.
            \end{prikladyin}
        \end{upozorneni}

    \subsection{Metody důkazů tvrzení}
        \begin{definice}[Důkaz sporem]
            $$ (A \implies B) \Leftrightarrow \neg(A \& \neg B) $$
            \begin{prikladyin}[$\sqrt{2} \notin \Q$]
                ($A: x = \sqrt{2}$, $B: x \notin \Q$)
                \begin{dukazin}[Důkaz sporem:]
                    Nechť $x = \sqrt{2}$ a $x \in \Q$. $x \in \Q \implies x = \frac{p}{q}, p,q \in \N$, nesoudělná.\\
                $x^2 = 2$,  $2 = x^2 = \frac{p^2}{q^2} \implies 2q^2 = p^2 \implies p = 2k \implies 2q^2 = 4k^2 \implies q^2 = 2k^2 \implies q = 2l$\\
                    $p=2k\ \&\ q=2l \implies$ $p$ a $q$ soudělná. \lightning 
                \end{dukazin}
                    
            \end{prikladyin}
        \end{definice}

        \begin{definice}[Přímý důkaz]
                $$ (A \implies B) \Leftrightarrow (A \implies C_1 \implies C_2 \implies … \implies C_n \implies B) $$ 
            \begin{prikladyin}
                $$ n^2 \text{liché} \implies n \text{liché} $$
                \begin{dukazin}
                    $$ n = p_1 \cdot … \cdot p_k \implies n^2 = p_1^2 \cdot … \cdot p_k^2 $$
                    $$ n^2 \text{liché} \implies 2\nmid p_1\ \&\ …\ \&\ 2\nmid p_k \implies n \text{liché} $$ 
                \end{dukazin}
            \end{prikladyin}
        \end{definice}

        \begin{definice}[Nepřímý důkaz]
            $$ (A \implies B) \Leftrightarrow (\neg B \implies \neg A) $$
            \begin{prikladyin}
                $$ n^2 \text{liché} \implies n \text{liché} $$ 
                \begin{dukazin}
                    $$ n \text{sudé} \Leftrightarrow n = 2k \implies n^2 = 4k^2 \implies n^2 \text{sudé} $$         
                \end{dukazin}
            \end{prikladyin}
        \end{definice}

        \begin{definice}[Matematická indukce]
            $$ (\forall n \in \N: V(n)) \Leftrightarrow (V(1)\ \&\ \forall n \in \N: V(n) \implies V(n+1)) $$ 
            \begin{prikladyin}
                $$ \sum_{k=1}^n k = 1+2+…+n = \frac{1}{2}n\cdot(n+1) $$
                \begin{dukaz}
                    1. $n = 1$: $1 = \frac{1}{2}1 \cdot 2$\\
                    2. 
                    $$ \sum_{k=1}^n k = 1+2+…+n = \frac{1}{2}n\cdot(n+1) \implies \sum_{k=1}^{n+1} k = \frac{1}{2}n\cdot(n+1) + (n + 1) = \frac{1}{2}(n+1)\cdot(n+2) $$ 
                \end{dukaz}
            \end{prikladyin}
            
            \begin{prikladin}
                Všechna auta mají stejnou barvu.
                \begin{dukazin}
                    1. $n = 1$: Jedno auto má stejnou barvu jako ono samo.\\
                    2. $n \rightarrow n+1$: vezmu prvních $n$ aut, ty mají stejnou barvu, vezmu posledních $n$ aut, ty mají také stejnou barvu. Tedy dohromady mají stejnou barvu.
                \end{dukazin}
                (Spoiler: $n = 2$)
            \end{prikladin}
        \end{definice}

    \subsection{Množina reálných čísel}
        \begin{poznamka}[Množiny čísel]
            $$ \N = \{1, 2, 3, …\} $$ 
            $$ \Z = \{… , -3, -2, -1, 0, 1, 2, 3, …\} $$ 
            $$ \Q = \{\frac{p}{q}, q\in \N, p \in \Z\} $$ 
        \end{poznamka}

        \begin{definice}[Omezená množina]
                Nechť $\M \subset \R$. Řekněme, že $\M$ je omezená shora (omezená zdola), jestliže existuje $a \in \R$ = horní (dolní) závora tak, že pro všechna $x \in \M$ platí $x ≤ a$ ($x≥a$).
        \end{definice}

        \begin{definice}[Supremum a infimum]
                Nechť $\M \subset \R$ je shora (zdola) omezená. Číslo $s \in R$ nazýváme supremem (infimem) $\M$, pokud:
                $$ \forall x \in \M: x≤s (s≥x) $$
                $$ \forall y \in \R, y<s (y>s) \exists x \in \M, y<x (x>y) $$ 
            \begin{prikladyin}
                \begin{itemize}
                        \item $\sup\[0,1\]=1$ (dokázat obě podmínky pro 1 ($x$ z druhé podmínky volím 1)…)
                        \item $\sup(0,1)=1$ (taktéž dokázat obě podmínky pro 1 (pozor na záporná y) ($x$ z druhé podmínky zvolíme často $\frac{s+y}{2}$))
                \end{itemize}
            \end{prikladyin}
        \end{definice}

        \begin{definice}[Reálná čísla]
            Na množině \R je dána relace $≤$ ($\subset \R \times \R$), operace sčítání $+$ a operace násobení $\cdot$ a množina \R obsahuje prvky 0 a 1 tak, že platí:
            
            Viz skripta (takové ty tělesové / grupové podmínky, podmínky uspořádání a \emph{existence suprema})
        \end{definice}

        \begin{veta}[o existenci infima]
            Nechť $\M \subset \R$ je neprázdná zdola omezená množina. Pak existuje $\inf\M$.
            \begin{dukazin}
                    Označme $-\M = \{x\in\R:-x\in\M\}$. Zřejmě $\M \neq \O$. $\M$ je zdola omezená $\implies$ $-\M$ je shora omezená. ($\exists K \in \R \forall x \in \M x>K \implies \exists K \in \R \forall x \in -\M x<-K$). Z axiomů \R tedy existuje $s = \sup -\M$. Položme $i = -s$. Tvrdím $i = \inf M$. (Dokážeme z definice suprema a infima, viz skripta).
            \end{dukazin}
        \end{veta}

        \begin{veta}[Archimedova vlastnost]
            Ke každému $x \in \R$ existuje $n \in \N$ tak, že $x < n$
            \begin{dukazin}[Sporem]
                $$ \exists x \in \R \forall n \in \N x≥n $$
                Tedy $\N$ je omezená podmnožina \R. Tedy existuje $x'\R x" = \sup\N$. Tedy $\forall n \in \N: n≤x'$. Pak také $\forall n \in \N: n+1 ≤ x'$. To ale tvrdí, že $x'-1$ je také $\sup\N$. To je ale spor, protože můžeme zvolit $y = x' - \frac{1}{2}$, pak $y<x'$, tedy z druhé vlastnosti suprema $\exists n \in \N: x' - \frac{1}{2} < n$, ale zároveň už víme, že $\forall n \in \N: n<x'-1$.
            \end{dukazin}
        \end{veta}

        \begin{veta}[Hustota \Q a $\R\\\Q$]
            Nechť $a, b \in\R, a<b$. Pak existují $q\in\Q$ a $\r\in\R\\\Q$ tak, že $q\in(a, b)$ a $r\in(a, b)$.
            \begin{dukaz}
                Podle \label{Archimedova vlastnost} $\exists n \in \N$ tak, že $\frac{1}{b-a}<n$, tedy $\frac{1}{n}<b-a$. Zvolme $q = \frac{\lceil an\rceil + 1}{n}$, pak jistě $a<q<b$ a $q\in\Q$. 

                Poté použijeme $q_1 \in (a, b)$ a $q_2 \in (q_1, b)$. Zvolme $r = q_1 + \frac{q_2-q_1}{\sqrt{2}}$. Pak (jelikož druhá část je kladná) $r>q_1$. A $r<q_2 \Leftrightarrow q_1 + \frac{q_2-q_1}{\sqrt{2}}< q_2 \Leftrightarrow \frac{q_2-q_1}{\sqrt{2}}<q_2 - q_1$.

                Tedy $r \in (a, b)$ a $r \in \R\\\Q$, jelikož $r = q_1 + \frac{q_2-q_1}{\sqrt{2}} = p \in\Q \implies \sqrt{2} = \frac{q_2-q_1}{p-q_1}$, ale levá strana je jistě iracionální a pravá racionální. Spor.
            \end{dukaz}
        \end{veta}

        \begin{veta}[O $n$-té odmocnině (BD = bez důkazu)]
            Nechť $n \in \N$ a $x \in \[0, \infty\)$, pak existuje právě jedno $y \in \[0,\infty\)$ tak, že $y^n = x$.
            \begin{dukaz}
                Idea: Položme $\M = \{z \in \R\}$. Ukážeme, že $\M \neq \O$ shora omezená $\implies$ $\exists s = \sup\M$. Nyní ukážeme $s^n = x$.
            \end{dukaz}
        \end{veta}


\end{document}
