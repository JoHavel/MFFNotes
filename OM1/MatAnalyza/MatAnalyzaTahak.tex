\documentclass[12pt]{article}					% Začátek dokumentu
\usepackage{../../MFFStyle}					    % Import stylu
\usepackage{cancel}

\newcommand{\its}{\item[$\square\ \square\ \square$]}

\makeatother%
\def\@fnsymbol#1{\ensuremath{\ifcase#1\or *\or \dagger\or \ddagger\or%
\mathsection\or \mathparagraph\or \|\or **\or \dagger\dagger%
\or \ddagger\ddagger \else\@ctrerr\fi}}%
\renewcommand{\thefootnote}{\fnsymbol{footnote}}%
\makeatletter%

\begin{document}

\section*{\hfill Zápočtová písemka 3. 11. 2020 \hfill\ }

\noindent\kern -1em\kern 1.4pt 1\ \ 2\ \ 3\\[-2.2em]
\begin{itemize}
    \its Přečíst si zadání.
    \its Přečíst si zadání.
    \its Znovu a pořádně přečíst zadání.
    \its Odhadnout výsledek.
    \its Zamyslet se nad podmínkami.
    \its Vyřešit příklad.
    \its Znovu si přečíst zadání a zkontrolovat, zda jsem opravdu řešil úlohu ze zadání.
    \its Zkontrolovat, že je všude $\lim_{n \rightarrow ∞}$.
    \its Doplniti vše tam, kde má být\footnote{AL, RŠ, věta o Spojitosti, \cancel{L'Hopital}, POLICIE (strážníci), (Anděl), (Ďábel), omezená krát nulová = nulová, bezejmenná věta o dělení nulou, limita a uspořádání, Bolzano-Cauchyova podmínka, limita vybrané posloupnosti, \cancel{známé limity}, věta o monotónní posloupnosti, věty o hromadných bodech a limsup s liminf.}.
    \its Nezapomenout udělat všechny domácí úkoly a naučit se na topologii a analýzu na varietách, už v nich příliš plavu.


\end{itemize}

\noindent\kern -1em Co nedělat:\\[-2.2em]

\begin{itemize}
    \item Nel'hopitalit.
    \item Nezkracovat zápisy.
    \item Nedělat více kroků najednou.
    \item Nepoužívat známé limity, které nejsou známé\footnote{Vždy pamatuj, že žádná limita, o které víme, že platí, není nedokazatelná!}.
    \item Vyhnout se částečnému limitění.
\end{itemize}

\end{document}
