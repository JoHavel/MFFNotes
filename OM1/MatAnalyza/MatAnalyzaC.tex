\documentclass[12pt]{article}					% Začátek dokumentu
\usepackage{../../MFFStyle}					    % Import stylu



\begin{document}

\section{Cvičení 29. 9. 2020}
    Zápočet bude za 2 zápočtové písemky + docházka (připojení se k cvičení / odpověď na mail, alespoň 50\%), scan přes mobil např. přes Adobe scan, nebo Clear scanner.

    \subsection{Rovnice a nerovnice}
        \begin{priklad}
            $$ \frac{x-2}{2x - 8} \geq 1 $$
            \begin{reseni}
                $$ \frac{x-2-(2x-8}{2x-8} \geq 0 $$ 
                $$ \frac{-x+6}{2(x-4)} \geq 0 $$
                (nulové body 4, 6)
                $$ x \in \(4, 6\] $$
            \end{reseni}
        \end{priklad}

        \begin{priklad}
            $$ |1-|x-1||<3 $$ 
            \begin{reseni}
                (nulové body 0, 1, 2)
                1. $x > 1$
                $$ |1-(x-1)|<3 $$
                $$ x \in \[1, 5\) $$
                2. $x < 1$
                $$ |1+x-1)|<3 $$
                $$ x \in (-3, 1) $$
                Celkově:
                $$ x \in (-3, 5) $$
            \end{reseni}
        \end{priklad}

        \begin{priklad}
            $$ \sqrt{2x-1} \geq x $$
            \begin{reseni}
                Podmínka: $ x \geq \frac{1}{2} $
                $$ 2x-1 \geq x^2 $$
                $$ 0 \geq x^2 - 2x + 1 $$
                $$ 0 \geq (x - 1)^2 $$ 
                $$ x = 1 $$
                Splňuje podmínky
            \end{reseni}
        \end{priklad}

        \begin{priklad}[$x^y$]
            $$ 2^x > 3 $$
            \begin{reseni}[Aplikuji $\log_2()$ na obě strany]
                $$ 2 > \log_2(3) $$ 
            \end{reseni}
        \end{priklad}


    \subsection{Samovýpočet}
        \begin{priklad}
            $$ \frac{x+2}{x+3} > \frac{2x+3}{x+6} $$
            \begin{reseni}
                Podmínky: $x \neq -3 \land x \neq -6$
                1. $-6 < x < -3$
                $$ (x+2)(x+6) < (2x+3)(x+3) $$
                $$ x^2 + 8x + 12 < 2x^2 + 9x + 9 $$
                $$ 0 < x^2 + x - 3 $$ 
                Což má kořeny mimo interval $(-6, -3)$ a třeba pro $x = -4$ je nerovnost splněna, tedy je splněna pro všechna $x \in (-6, 6)$\\
                2. $x < -6 \lor x > -3$
                $$ (x+2)(x+6) > (2x+3)(x+3) $$
                $$ x² + 8x + 12 > 2x² + 9x + 9 $$
                $$ 0 > x² + x - 3 $$
                $$ x_{1,2} = \frac{-1 \pm \sqrt{13}}{2} $$
                A v -1 je nerovnost splněna, tedy je splněna v intervalu $(\frac{-1 - \sqrt{13}}{2}, \frac{-1 + \sqrt{13}}{2})$\\
                Celkově: $x \in (-6, -3)\cup(\frac{-1 - \sqrt{13}}{2}, \frac{-1 + \sqrt{13}}{2})$
            \end{reseni}
        \end{priklad}

        \begin{priklad}
            $$ \log_{\frac{1}{3}}(x^2 - 3x + 2) ≥ 0 $$
            \begin{reseni}
                Podmínky: $x^2 - 3x + 2 > 0$
                $$ x^2 - 3x + 2 ≤ 1 $$
                $$ x^2 - 3x + 1 ≤ 0 $$
                $$ x_{1,2} = \frac{3 \pm \sqrt{5}}{2} $$
                $$ x \in \(\frac{3 - \sqrt{5}}{2}, \frac{3 + \sqrt{5}}{2}\) $$ 
                Podmínky splňuje:
                $$ x_{1,2} = \frac{3 \pm \sqrt{1}}{2} $$
                $$ x \in (-\infty, 1)\cup (2, \infty) $$
                Celkově tedy:
                $$ x \in \[\frac{3 - \sqrt{5}}{2}, 1\)\cup \(2, \frac{3 + \sqrt{5}}{2}\] $$
            \end{reseni}
        \end{priklad}

        \begin{priklad}
            $$ |x-|x+1|| ≤ 2x $$
            \begin{reseni}
                Nulové body (-1)\\
                1. $ x ≤ -1$
                $$ |x+x-1| ≤ 2x $$ 
                $$ -2x + 1 ≤ 2x $$ 
                $$ 1 ≤ 4x $$
                Nesplňuje žádné $x ≤ -1$\\
                2. $ x > -1 $
                $$ |x-x-1| ≤ 2x $$
                $$ 1 ≤ 2x $$ 
                $$ x ≥ \frac{1}{2} $$
                Celkově: $x ≥ \frac{1}{2}$
            \end{reseni}
        \end{priklad}

        \begin{priklad}
            $$ \sin 2x < \cos x $$
            \begin{reseni}
                 $$ \sin 2x = 2\cdot \sin x \cos x < \cos x $$
                 1. $\cos > 0$
                 $$ 2 \sin x < 1 $$ 
                 $$ \sin x < \frac{1}{2} $$
                 Nedořešeno
            \end{reseni}
                
        \end{priklad}


\section{Cvičení 2. 10. 2020}
    \subsection{Výroky + formální důkazy}
        \begin{priklad}
            Máme formuli:
            $$ \forall x \in \M \exists y \in \M \exists z \in \M: x = y + z $$ 
            Je splněna pro $\M = \N$? Dokažte.
            \begin{dukazin}
                $$ \text{platí} \Leftrightarrow \forall x \in \M \exists y \in \M: x - y \in\M $$
                Tvrdíme: výraz neplatí. Dokazujeme negaci:
                $$ \neg\text{platí} \Leftrightarrow \exists x \in \N \forall y \in \N: x - y \notin\M $$
                Zvolme $x = 1$. Víme, že $\forall n \in \N: n>0$. Ale $\forall y \in \N: x-y ≤ 1-1 = 0 \implies x-y \notin \N$.
            \end{dukazin}
        \end{priklad}

        \begin{priklad}
            Je formule z předchozího příkladu splněna pro $\M = (0,1)$?
            \begin{dukazin}
                Tvrdím, že je. Zafixuji $x \in (0, 1)$, zvolím $y = z = \frac{x}{2}$. Jistě $y, z \in (0,1)$ a $x = y + z = \frac{x}{2} + \frac{x}{2} = x$.
            \end{dukazin}
        \end{priklad}

        \begin{priklad}
            $$  \forall x \in \R \forall \epsilon>0 \exists \delta>0 \forall y\in\R (|y-x|<\delta \implies y < x + \frac{\epsilon}{3}) $$
            \begin{dukazin}
                Zafixujeme $x \in \R, \epsilon > 0$. Položme $\delta = \frac{\epsilon}{100}$. Zafixujeme $y \in \R$, že $|y-x|<\delta$ (jinak implikace platí). Tj. $x - \delta < y < x + \delta$. Pak $y < x + \delta = x + \frac{\epsilon}{100} < x + \frac{\epsilon}{3}$.
            \end{dukazin}
        \end{priklad}


\end{document}
