\documentclass[12pt]{article}					% Začátek dokumentu
\usepackage{../../MFFStyle}					    % Import stylu



\begin{document}

\section{Organizační úvod}
    Přesun nebyl odhlasován.

    \begin{poznamka}[Literatura]
        \ 
        \begin{itemize}
            \item Engelking: General Topology (spíš taková příručka, hodně obtížná)
            \item Čech: Bodová topologie
            \item Kelley: General Topology
            \item Willard: General Topology
        \end{itemize}
        Doporučené jsou poslední dvě.
    \end{poznamka}

    \begin{poznamka}[Podmíny zakončení]
        Zkouška (ústní) + úkoly ze cvičení (a účast na cvičení)
    \end{poznamka}

\section{Úvod}

    \begin{poznamka}[Historie]
        \ 
        \begin{itemize}
            \item Euler: mosty ve městě Královec (7 mostů, Eulerovský tah)
            \item Listing (1847): pojem topologie (bez rigorózních definic)
            \item Poincaré (1895): Analysis Situs (Poincarého hypotéza )
            \item Fréchet (1906): definuje metrický prostor (až dodnes)
            \item Hausdorff (1914): tzv. Hausdorffův TP
            \item Kuratowski (1922): TP, jak jej známe dnes (formálně)
        \end{itemize}
    \end{poznamka}

    \begin{poznamka}[TOPOSYM]
        V Praze se každých 5 let koná významná konference topologů -- TOPOSYM.
    \end{poznamka}



\section{Základní pojmy}
    Topos = umístění (řečtina).

    \subsection{Topologický prostor, báze, subbáze, váha, charakter}

    \begin{definice}[Topologický prostor (TP)]
        Uspořádaná dvojice $(®X, \tau)$ se nazývá topologický prostor, pokud ®X je množina, $\tau \subseteq ©P(®X)$ a platí:\\
        (T1) $\O, ®X \in \tau$\\
        (T2) jsou-li $®U, ®V \in \tau$, pak $®U \cap ®V \in \tau$\\
        (T3) je-li $©U \in \tau$, pak $\bigcup ©U \in \tau$.
    \end{definice}

    \begin{definice}[Topologie]
        Systém $\tau$ se nazývá topologie na ®X. Prvky množiny ®X se nazývají body. Prvky $\tau$ se nazývají otevřené množiny.
    \end{definice}

    \begin{definice}[Okolí bodu]
        Množina $®V \subseteq ®X$ se nazývá okolí bodu $x$, pokud existuje $®U \in \tau$, že $x \in ®U \subseteq ®V$. Množina všech okolí bodu $x$ značíme $©U(x) = ©U_\tau(x)$.
    \end{definice}

    \begin{definice}[Báze a subbáze]
        Soubor množin $©B \subseteq \tau$ se nazývá báze topologie $\tau$, pokud pro každé $®U\in \tau$ existuje $©U \subseteq ®B: \bigcup ©U = ®U$. Soubor $©S \subseteq \tau$ se nazývá subbáze topologie $\tau$, pokud $\{\bigcap ©F: ©F \subseteq ©S \text{konečná}\}$ je báze topologie $\tau$.
    \end{definice}

    \begin{tvrzeni}[Charakterizace otevřené množiny pomocí okolí]
        Ať $(®X, \tau)$ je TP a $®U \in ®X$. Pak $®U \in \tau$, právě když $\forall x \in ®U \exists ®V\in ©U(x): ®V \subseteq ®U$
        \begin{dukazin}
            Důkaz ($\implies$) vidíme $®U = ®V$.

            Opačně víme $\forall x \in ®U \exists ®V_x\in©U(x): ®V_x \subseteq ®U$. $\exists ®W_x \in \tau: x\in ®W_x \subseteq ®U_x$. $®U = \bigcup_{x\in ®U}®W_x \in \tau$. Tedy $®U \in \tau$.
        \end{dukazin}
    \end{tvrzeni}

    \begin{priklad}
        Je-li $(®X, \rho)$ metrický prostor (MP), pak soubor všech $\rho$-otevřených množin tvoří topologii na množině ®X.
    \end{priklad}

    \begin{definice}[Metrizovatelný TP]
        TP $(®X, \tau)$ se nazývá metrizovatelný, pokud na množině ®X existuje metrika $\rho$ tak, že topologie odvozené z $(®X, \rho)$ splývá s topologií $\tau$.
    \end{definice}

    \begin{priklad}
        Je-li $(®X, \rho)$ MP, pak systém všech otevřených koulí tvoří bázi topologie $\tau_\rho$.
        \begin{prikladyin}
            Všechny otevřené intervaly tvoří bázi topologie na ®R.
            
            Systém $\{(-\infty, b), (a, \infty): a,b \in ®R\}$ je subbáze topologie na ®R.
        \end{prikladyin}
    \end{priklad}

    \begin{priklad}[Diskrétní a indiskrétní TP]
        Je-li ®X množina, pak $(®X, ©P(®X))$ je TP, nazývá se diskrétní TP (a vždy je metrizovatelný). Naopak $(®X, \{\O, ®X\})$ se nazývá indiskrétní TP. (Pokud $|®X| ≥ 2$, pak indiskrétní TP není metrizovatelný.)
    \end{priklad}

    \begin{tvrzeni}[Vlastnosti báze]
        Je-li $(®X, \tau)$ TP a ©B jeho báze, pak\\
        (B1) $\forall ®U, ®V \in ©B \forall x \in ®U \cap ®V \exists ®W \in ®B: x \in ®W \subseteq ®U \cap ®V$,\\
        (B2) $\bigcup ©B = ®X$.

        Je-li ®X libovolná množina a $©B \subseteq ®P(®X)$ splňuje podmínky (B1), (B2), pak na ®X existuje jediná topologie, jejíž báze je $®B$.

        \begin{dukazin}
            První část je snadná (průnik 2 množin báze je otevřený, tj. prvkem topologie, tedy se dá zapsat jako sjednocení podmnožiny báze).

            Druhá část: Mějme tedy ®X a ©B z věty splňující obě podmínky. Definujme $\tau := \{\bigcup ©U: ©U \subseteq ©B\}$. $\tau$ je topologie na ®X (ověříme, že $\tau$ splňuje podmínky topologie).

            Zároveň volba $\tau$ je jediná množná, jelikož každý její prvek se musí dát vyjádřit jako sjednocení báze a opačně.
        \end{dukazin}

        \begin{dusledekin}
            Je-li ®X množina, $©S \subseteq ©P(®X)$ a $\bigcup©S=®X$, pak $©S$ je subbáze jednoznačně určené topologie na ®X.
            \begin{dukazin}
                    $ ©B = \{\cap ©F: ©F \subseteq ©S \text{konečná}\} $ splňuje podmínky (B1) a (B2) předchozího tvrzení (B2 \ definice ©S, B1 protože $®U, ®V \in ©B, ®U = \bigcup ©F_1, ®V = \bigcap ©F_2, ©F_1, ©F_2 \subseteq ©S \text{konečné}$. $ ®U \cap ®V = \bigcap (©F_1 \cup ©F_2) \in ©B $. (Dokonce celý průnik je prvkem ©B, nejenom pro každý prvek existuje množina, která ho obsahuje, je podmnožinou průniku a je v ©B).
            \end{dukazin}
        \end{dusledekin}
    \end{tvrzeni}

    \begin{tvrzeni}[Vlastnosti systému všech okolí]
        Je-li $(®X, \tau)$ TP, pak soubory všech okolí $©U_\tau(x), x \in ®X$ splňují\\
        (U1) $ \forall x \in ®X: ©U(x) ≠ \O, x \in \bigcap©U(x)$,\\
        (U2) $\forall ®U \in ©U(x) \forall ®V: ®U \subseteq ®V \subseteq ®X \implies ®V \in ©U(x)$,\\
        (U3) $\forall ®U, ®V \in ©U(x): ®U \cap ®V \in ©U(x)$,\\
        (U4) $\forall ®U \in ©U(x) \exists ®V \in ©U(x) \forall y \in ®V: ®U \in ©U(y)$

        Je-li ®X množina a systémy množin $©U(x) \subseteq ©P(®X), x \in ®X$ splňující podmínky (U1-4), pak na množině ®X existuje jediná topologie $\tau$, že $©U(x) = ©U_\tau(x), x \in ®X$.
        \begin{dukazin}
            První část snadná. (Domácí cvičení.)

            Položme $\tau = \{®U \in ©P(®X): \forall x \in ®U, ®U\in©U(x)\}$. $\tau$ je topologie na $®X$. Z (U1) a (U2) vyplyne (T1). Atd…
        
        \end{dukazin}
    \end{tvrzeni}
    
    \begin{definice}[Báze okolí]
        Ať $(®X, \tau)$ je TP. Systém množin $©B(x) \subseteq ©P(®X)$ se nazývá báze okolí v bodě $x$, pokud $©B(x) \subseteq ©U_\tau(x)$ a pro každé $®V \in ©U_\tau(x)$ existuje $®U \in ©B(x)$, že $®U \in V$???. Indexovaný soubor $\{©B(x): x \in ®X\}$ se nazývá báze okolí prostoru $®X$, pokud $\forall x \in ®X: ©B(x)$ je báze okolí v bodě $x$.
    \end{definice}

    \begin{tvrzeni}[Vlastnosti báze okolí]
        Je-li $(®X, \tau)$ TP a $\{©B(x): x \in ®X\}$ báze okolí, pak\\
        (O1) $©B(x) ≠ \O, x \in \bigcap ©B(x), x \in ®X$,\\
        (O2) $\forall ®U, ®V \in ©B(x) \exists ®W \in ©B(x): ®W \subseteq ®U \cap ®V$,\\
        (O3) $\forall ®U \in ©B(x) \exists ©B(x) \forall y \in ®V \exists ®W \in ©B(y): ®W \subseteq ®U$.

        Je-li ®X množina a $©B(x)\subseteq ©P(®X), x \in ®X$ soubory splňující (O1), (O2), (O3), pak na množině ®X existuje jediná topologie, jejíž báze okolí je $\{©B(x): x \in ®X\}$.

        \begin{dukazin}
            První část je snadná. 

            Položme $©U(x) = \{®U \in ©P(x): \exists ®B \in ©B(x): ®B \subseteq ®U\}, x \in ®X$. Ověříme, že splňuje (U1-4). (U1) z (O1). (U2) z definice $©U$. (U3) z (O2), (U4) z (O3).
        \end{dukazin}
    \end{tvrzeni}

    \begin{definice}[Váha prostoru]
        Ať $(®X, \tau)$ je TP. Pak váha prostoru $(®X, \tau)$ je nejmenší mohutnost báze prostoru $(®X, \tau)$. Značíme ji $\weight(®X) = \weight(®X, \tau)$

        Charakter v bodě $x$ je nejmenší mohutnost báze okolí bodu $x$. Značíme ho $\chi(x, ®X)$.

        Charakter prostoru ®X je $\sup\{\chi(x, ®X): x \in ®X\}$.

        \begin{prikladyin}
            $\weight(®R) = \omega$ (®R má spočetnou bázi).

            $\weight(®X, ©P(®X)) = |®X|$ ($\{\{x\}: x \in ®X\}$ je báze $(®X, ©P(®X))$)

            $ \weight(®X, \{\O,\{®X\}\}) = 1 $
        \end{prikladyin}

        \begin{prikladyin}
            Je-li $(®X, \tau)$ metrizovatelný, pak $\chi(x, ®X) ≤ \omega$
        \end{prikladyin}
    \end{definice}

    \begin{tvrzeni}
            Ať $(®X, \tau)$ je TP a $x \in ®X$. Pak $\chi(x, ®X)≤\weight(®X)$
        \begin{dukazin}
            Ať ©B je báze $(®X, \tau)$, že $|©B| = \weight(®X)$. Položme $©B(x):=\{®U \in ©B: x \in ®U\}$. $©B(x)$ je báze okolí v bodě $x$.

            $|©B(x)|≤|©B|$, protože $©B(x) \subseteq ©B$. $\chi(x, ®X)≤|©B(x)|≤|©B|=\weight(®X)$.
        \end{dukazin}
    \end{tvrzeni}

    \subsection{Vnitřek, Uzávěr, hranice}
        \begin{definice}[Uzavřená množina]
            Ať $(®X, \tau)$ je TP. Množina $®F \subseteq ®X$ se nazývá uzavřená, pokud její doplněk je otevřená množina (neboli $®X \setminus ®F \in \tau$).
        \end{definice}

        \begin{definice}[Obojetná množina (clopen set)]
            Množina se nazývá obojetná, pokud je uzavřená a otevřená zároveň.
        \end{definice}

        \begin{definice}[Uzávěr]
            Je-li $®A \subseteq ®X$, pak uzávěr ®A je $\cl(®A) = \overline{®A} = \bigcap\{®F \subseteq ®X, ®A \subseteq ®F, ®F \text{je uzavřená}\}$.
        \end{definice}

        \begin{definice}[Vnitřek množiny]
            Vnitřek množiny $®A$ je $\Int®A = ®A^0 = \bigcup\{®U \in \tau: ®U \subseteq ®A\}$.
        \end{definice}

        \begin{definice}[Hranice množiny]
                Hranice množiny ®A je $\delta ®A = \overline{®A} \cap \overline{®X \setminus ®A}$
        \end{definice}

        \begin{tvrzeni}[Vztah vnitřku a uzávěru]
                Ať $(®X, \tau)$ je TP, $®A\subseteq®X$, pak $®X \setminus \overline{®A} = \Int(®X \setminus ®A)$ a $®X \setminus \Int®A = \overline{®X \setminus ®A}$.

            \begin{dukazin}
                $® \setminus \overline{®A}$ je otevřená, navíc $®X \setminus \overline{®A} \subseteq ®X \setminus ®A$. Tedy $®X \setminus \overline{®A} \subseteq \Int(®X\setminus ®A)$. $\Int(®X \setminus ®A)®X \setminus ®A$, přechodem k doplňku $®A \subseteq ®X \setminus \Int(®X \setminus ®A)$. Tedy $\overline{®A} \subseteq ®X \setminus \Int(®X)???$. Přechodem k doplňku: $\Int(®X \setminus ®A) \subseteq ®X \setminus \overline{®A}$.

                Druhou část můžeme dokázat přechodem k doplňku a převedením na první část.
            \end{dukazin}
        \end{tvrzeni}

        \begin{tvrzeni}[Charakterizace uzávěru]
                Buď $(®X, \tau)$ TP, $x \in ®X, ®A \subseteq ®X$ a $©B(x)$ báze okolí v bodě $x$. Pak následující podmínky jsou ekvivalentní\\
            1) $x \in \overline{®A}$,\\
            2) $\forall ®U \in ©U(x): ®U\cap ®A ≠ \O$,\\
            3) $\forall ®U \in ©B(x): ®U\cap ®A ≠ \O$.

            \begin{dukazin}
                1) -> 2) sporem: Kdyby pro nějaké $®U \in ©U(x): ®U \cap ®A = \O$, pak existuje ®V otevřené: $x \in ®V \subseteq ®U$. $®V \cap®A = \O$. $®X \setminus ®V$ je uzavřená a $®A \subseteq ®X \setminus ®V$. Pak $x \in \overline{®A} \subseteq ®X \setminus ®V$, neobsahuje $x$. $\lightning$.

                2) -> 3) triviální

                3) -> 1) sporem: $x \notin \overline{®A}$ pak $x \in ®X\setminus \overline{®A}$. Pak existuje $®U \in ©B(x): x \in ®U \subseteq ®X \setminus \overline{®A}$. Pak ???
            \end{dukazin}
        \end{tvrzeni}

        Jako speciální důsledky dostáváme následující. Je-li ®U otevřená, pak $®U \cap ®A = \O$ právě když $®U \cap \overline{®A} = \O$. Jsou-li ®U, ®V otevřené disjunktní množiny, pak $®U \cap \overline{®V} = \O = \overline{®U} \cap ®V$.

        \begin{tvrzeni}[Vlastnosti uzávěru]
            Pro množiny ®A, ®B v TP $(®X, \tau)$ platí\\
            (C1) $\overline{\O} = \O$,\\
            (C2) $®A \subseteq \overline{®A}$,\\
            (C3) $\overline{\overline{®A}} = \overline{®A}$
            (C4) $\overline{®A \cup ®B} = \overline{®A} \cup \overline{®B}$,\\
            (C5) $\overline{®A \cap ®B} \subseteq \overline{®A} \cap \overline{®B}$.

            \begin{dukazin}
                První dvě jsou jednoduché, 3. plyne z uzavřenosti uzávěru. 4. dokážeme inkluzemi. Shrnutím dostaneme (C5).
            \end{dukazin}
        \end{tvrzeni}

        \begin{priklad}
                Zobrazení z podmnožin do podmnožin, které splňuje podmínky (C1-C4) jednoznačně určuje topologii.
        \end{priklad}

        \begin{tvrzeni}[Vlastnosti vnitřku]
            Obdobně jako vlastnosti uzávěru.
        \end{tvrzeni}

        \begin{tvrzeni}[Charakterizace hranice]
            Ať $®A \subseteq ®X$ a $x \in ®X$. Pak $x \in \delta ®A$, právě když každé okolí bodu $x$ protíná jak ®A, tak $®X \setminus ®A$.
            \begin{dukazin}
                Plyne okamžitě z definice hranice $\delta ®A = \overline{®A} \cap \overline{®X \setminus ®A}$ a charakterizace uzávěru.
            \end{dukazin}
        \end{tvrzeni}

        \begin{tvrzeni}[Vlastnosti hranice]
            12. bodů viz skripta. Stejně tak důkaz.
        \end{tvrzeni}
    
    \subsection{Husté a řídké množiny, hromadné a izolované body}
        \begin{definice}[Hustá a řídká množina, hustota, separabilní prostor]
                Ať ®X je TP. Množina $®A \subseteq ®X$ se nazývá hustá (v ®X), pokud $\overline{®A} = ®X$. ®A se nazývá řídká, pokud $®X \setminus \overline{®A}$ je hustá.

                Hustota prostoru ®X je nejmenší mohutnost husté podmnožiny, značí se $\d(®X)$ (d…density). Prostor se spočetnou hustotou se nazývá separabilní.
        \end{definice}

        \begin{tvrzeni}[Charakterizace hustých a řídkých množin]
            Ať ®X je TP. Množina $®A \subseteq ®X$ je hustá v ®X, právě když $\forall ®U$ otevřená neprázdná v ®X protíná ®A. Množina ®A je řídká (v ®X), právě když $\forall ®V$ otevřená neprázdná $\exists ®U$ otevřená neprázdná, že $®U \subseteq ®V \setminus ®A$, což je právě když $\Int(\overline{®A}) = \O$.
            \begin{dukazin}
                Označme $\tau* = \tau \setminus \O$. Z charakterizace uzávěru: $\overline{®A} = ®X \Leftrightarrow \forall x \in ®X \forall ®V \in ©U(x): ®V\cap ®A ≠ \O$. $A$ je řídká $\Leftrightarrow$ $®X\setminus \overline{®A}$ je hustá $\Leftrightarrow \forall ®U \in \tau*: ®U \cap (®X \setminus \overline{®A}) ≠ \O \Leftrightarrow \forall ®U \in \tau*: ®U \setminus \overline{®A} ≠ \O$.

                První část dostaneme ekvivalencí z předchozího: $ \forall ®U \in \tau* \exists ®V \in \tau*: ®V \subseteq ®U \setminus \overline{®A}$.

                Druhá část pak plyne z $\Int \overline{A} = \O$
            \end{dukazin}
        \end{tvrzeni}
        
        \begin{tvrzeni}[Vztah váhy a hustoty]
            Ať ®X je TP. Pak $\d(®X) ≤ \weight(®X)$. Speciálně každý prostor se spočetnou bází je separabilní.
            \begin{dukazin}
                Ať ©B je báze TP ®X. (BÚNO $\O \notin ©B$). $forall ®B \in ©B$ fixujeme $x_B \in B, ®D:=\{x_B : B \in ©B\}$. Zřejmě $|®D|≤|©B|$, ®D je hustá v ®X. (Když tedy volíme $©B$ nejmenší, získáme výraz.)
            \end{dukazin}
        \end{tvrzeni}

        \begin{poznamka}
            Pro metrizovatelný TP ®X platí $\d(®X) = \weight(®X)$.
        \end{poznamka}

        \begin{definice}[Izolovaný a hromadný bod]
            Ať ®X je TP. Bod $x \in ®A \subseteq ®X$ se nazývá izolovaným bodem množiny $A$, pokud existuje otevřená množina $®U \subseteq ®X$, že $®U\cap ®A = \{x\}$. Bod $x$ se nazývá hromadným bodem množiny ®A, pokud každé okolí bodu $x$ protíná množinu $®A\subseteq \{x\}$
            \begin{prikladyin}
                V diskrétním prostoru jsou všechny body izolované. Naopak je-li $®X = ®R$ a $®A = ®Q$, pak každý bod ®X je hromadným bodem množiny ®A. Žádný bod z ®A není izolovaným bodem ®A.
            \end{prikladyin}

        \end{definice}

        \begin{definice}[Derivace množiny]
            Množina hromadných bodů množiny ®A se značí $®A'$. Někdy se nazývá derivace ®A.
        \end{definice}

        \begin{tvrzeni}[Vlastnosti derivace]
            $\overline{®A} = ®A \cup ®A'$, $(®A \cup ®B)' = ®A' \cup ®B'$
            \begin{dukazin}
                Domácí cvičení (je jednoduchý).
            \end{dukazin}
        \end{tvrzeni}

    \subsection{Spojitá zobrazení}
        \begin{definice}[Spojité zobrazení, homeomorfizmus a spojitost v bodě]
            Ať $(®X, \tau)$ a $(®Y, \sigma)$ jsou TP. Ať $f: ®X \rightarrow ®Y$. Zobrazení $f$ se nazývá spojité, pokud $\forall ®U \in \sigma: f^{-1}(®U) \in \tau$.

            $f$ se nazývá homeomorfizmus, pokud $f$ je bijekce a $f$ i $f^{-1}$ jsou spojitá.

            $f$ je spojité v bodě $x$, pokud $\forall ®V \in ©U_\sigma(f(x)) \exists ®U \in ©U_\tau(x): f(U) \subseteq ®V$.
        \end{definice}

        \begin{priklady}
            ®R, $(0, 1)$ jsou homeomorfní (ale nejsou izometrické)
        \end{priklady}

        \begin{poznamka}
            Vlastnosti, TP, které se zachovávají homeomorfizmem se nazývají topologické vlastnosti.

            (Úplnost není topologický pojem.)
        \end{poznamka}

        \begin{prikladyin}
            Zobrazení z diskrétního prostoru je vždy spojité.

            Zobrazení do indiskrétního prostoru je taktéž vždy spojité.
        \end{prikladyin}

        \begin{tvrzeni}[Charakterizace spojitých zobrazení]
            Ať $(®X, \tau)$, $(®Y, \sigma)$ jsou TP, $f: ®X \rightarrow ®Y$ zobrazení. Pak následující je ekvivalentní:\\
            1) $f$ je spojité\\
            2) vzory množin z nějaké subbáze jsou otevřené\\
            3) vzory množin z nějaké báze jsou otevřené\\
            4) $f$ je spojité v každém bodě\\
            5) vzory uzavřených množin jsou uzavřené\\
            6) $\forall ®A \subseteq ®X: f(\overline{®A}) \subseteq \overline{f(®A)}$\\
            7) $\forall ®B \subseteq ®Y: \overline{f^{-1}(®B)} \subseteq f^{-1}(\overline{®B})$\\
            8) $\forall ®B \subseteq ®Y: f^{-1}(\Int ®B) \subseteq \Int\(f^{-1}(®B)\)$

            \begin{dukazin}
                1->2 Triviální (z definice).

                2->3 Ať $©B$ je nějaká báze. Dle 2 pro nějakou subbázi ©S toho $(®Y, \sigma)$ platí, že $f^{-1}(®S)$ je otevřená pro $®S \in ©S$. Ať $®B \in ©B$. ®B lze vyjádřit jako sjednocení konečných průniků prvků ©S. (Vzor průniku je průnik vzorů, vzor sjednocení je sjednocení vzorů.) $f^{-1}(®B)$ je sjednocením konečných průniků prvků tvaru $f^{-1}(®S), ®S \in ©S$. Tedy $f^{-1}(®B)$ je otevřená.

                3->4 Ať $x \in ®X$, ®V okolí bodu $f(x)$. ©B báze z 3. podmínky. $\exists ®B \in ©B$, že $f(x) \in ©B \subseteq ®V$. $®U = f^{-1}(®B)$ otevřená, $x \in ®U f(®U) \subseteq ®B \subseteq ®V$.

                4->5 Ať $®F \subseteq ®Y$ je uzavřená. Ať $x \in \overline{f^{-1}(F)}$. Chceme, že $x \in f^{-1}(®F)$ (tj. že $f(x) \in ®F$). Z 4 pro každé okolí ®V bodu $f(x)$ existuje ®U okolí $x$, že $f(x) \subseteq V$. Z definice uzávěru platí, že každé takové ®U protíná $f^{-1}(®F)$, tedy $f(®U) \cap ®F ≠ \O$, tedy $®V \cap ®F ≠ \O$. Tedy podle charakterizace uzávěru $f(x) \in \overline{®F} = ®F$.

                5->6 $f^{-1}(\overline{f(®A)})$ je uzavřená dle 5 a obsahuje ®A, tedy obsahuje i $\overline{®A}$. Pak $f(\overline{®A}) \subseteq f\(f^{-1}(\overline{f(®A)})\)\subseteq \overline{f(®A)}$.

                6->7 Ať $®B \subseteq Y$, $A := f^{-1}(®B)$. Dle 6 $f(\overline{f^{-1}(®B)}) \subseteq \overline{f(f^{-1}(®B))}\subseteq \overline{®B}$. $\overline{f^{-1}(®B)} \subseteq f^{-1}(\overline{®B})$ (aplikováním vzoru? na předchozí).

                7->8 Vztah vnitřku a uzávěru. $f^{-1}(\Int ®B) = f^{-1}(®Y \setminus \overline{®Y \setminus ®B}) = ®X \setminus f^{-1}(\overline{®Y \setminus ®B}) \stackrel{\text{dle 7}}{\subseteq} ®X \setminus \overline{®Y \setminus ®B} = ®X \setminus \overline{®X \setminus f^{-1}(®B)} = ®X \setminus \(®X \setminus \Int f^{-1}(®B)\) = \Int f^{-1}(®B)$.

                8->1 Je-li $®V \subseteq ®Y$ otevřená, pak ze 7: $f^{-1}(®V) \subseteq \Int(f^{-1}(®V))$. Triviálně $\Int f^{-1}(®V) \subseteq f^{-1}(®U)$. Tedy $f^{-1}(®V) = \Int f^{-1}(®V)$, tedy $f^{-1}(®V)$ je otevřená.
            \end{dukazin}
        \end{tvrzeni}

        \begin{tvrzeni}[Skládání spojitých zobrazení]
            Ať $®X, ®Y, ®Z$ jsou TP, $f: ®X \rightarrow ®Y, g: ®Y \rightarrow ®Z$ zobrazení. Jsou li $f, g$ spojitá, pak $g \circ f: ®X \rightarrow ®Z$ je spojité.

            Pokud $f$ je spojité v bodě $x$ a $g$ spojité v $f(x)$, pak $g \circ f$ je spojité v $x$.

            \begin{dukazin}
                $(g\circ f)^{-1}(®V) = f^{-1}(g^{-1}(®V))$

                Je-li ®V okolí $gf(x)$, pak $g^{-1}(®V)$
            \end{dukazin}
        \end{tvrzeni}

% 19. 10. 2020

    \subsection{Oddělovací axiomy}
        \begin{definice}
            TP ®X se nazývá:
            \begin{itemize}
                \item $T_0$, pokud $\forall x, y \in ®X \exists ®U \text{otevřená} : |U \cap \{x, y\}| = 1$.
                \item $T_1$, pokud $\forall x, y \in ®X, x≠y \exists ®U \text{otevřená} : x\in ®U, y\notin ®U$.
                \item $T_2$ (Hausdorffův), pokud $\forall x, y \in ®X \exists ®U, V \text{otevřené disjunktní} : x \in ®U, y \in ®V $.
                \item regulární, pokud $\forall ®F \subseteq ®X$ uzavřenou $\forall \in ®X \setminus ®F \exists ®U, ®V$ otevřené disjunktní: $x \in ®U, ®F \subseteq ®V$.
                \item normální, pokud $\forall ®E, ®F$ uzavřené disjunktní $\exists ®U, ®V$ otevřené disjunktní: $®E \subseteq ®U, ®F \subseteq ®V$.
                \item úplně regulární, pokud $\forall ®F \subseteq ®X$ uzavřenou $\forall x \in ®X \setminus ®F \exists f: ®X \rightarrow \[0, 1\]$ spojitá, že $f(x) = 0, f(®F)\subseteq \{1\}$.
                \item $T_3$, pokud je regulární a $T_1$.
                \item $T_{3\frac{1}{2}}$ nebo $T_\pi$ (Tichonovův), pokud je úplně regulární a $T_1$.
                \item $T_4$, pokud je normální a $T_1$.
            \end{itemize}

            \begin{poznamka}
                $$ \text{normální} \implies \text{úplně regulární} \overset{\text{rozpůlení intervalu $\[0, 1\]$}}{\implies} \text{regulární} $$ 
                $$ T_4 \implies T_\pi \implies T_3 \implies T_2 \implies T_1 \implies T_0 $$ 
                (Platí pouze tímto směrem, ne opačně!)
                $$ T_0 \not\implies T_1: \(\{0, 1\}, \{\O, \{0, 1\}, \{0\}\}\) … (\text{Sierpinského TP}) $$
                $$ T_1 \not\implies T_2: \(®N, \{\O\} \cup \{®N \setminus K: K \text{je konečná}\}\) (\text{Topologie kokonečných (doplněk konečných) množin}) $$
            \end{poznamka}
        \end{definice}

        \begin{tvrzeni}[Metrizovatelné prostory jsou $T_4$]
            Je-li ®X metrizovatelný prostor a $®E, ®F \subseteq ®X$ uzavřené disjunktní množiny, pak existuje spojitá funkce $f: ®X \rightarrow \[0, 1\]$, že $f(®E) \subseteq \{0\}, f(®F) \subseteq \{1\}$.

            \begin{dukazin}
                    ®X je metrizovatelný, tedy existuje metrika $\rho$ kompatibilní s topologií na ®X. Položme $f(x) = \frac{\rho(x, ®E)}{\rho(x, ®E) + \rho(x, ®F)}, x \in ®X$. $f$ je dobře definovaná a jistě spojitá. $f(x) = 0, x \in ®E$, $f(x) = 1, x \in ®F$.
            \end{dukazin}
        \end{tvrzeni}

        \begin{lemma}
            Ať ®X je TP. Pak
            \begin{itemize}
                \item[a)] ®X je $T_1 \Leftrightarrow$ každá jednoprvková množina je uzavřená $\Leftrightarrow$ každá konečná množina je uzavřená.
                \item[b)] ®X je $T_2$ $\implies \forall x, y \in ®X, x≠y \exists ®U \in ©U(x): y\notin \overline{®U}$.
                \item[c)] ®X je regulární $\Leftrightarrow \forall x \in ®X \forall ®U \in ©U(x) \exists ®V \in ©U(x): \overline{®V} \subseteq ®U$.
                \item ®X je normální $\Leftrightarrow \forall ®V \subseteq ®X \text{otevřenou} \forall ®E \in ®V \text{uzavřenou} \exists U \subseteq ®X \text{otevřená}: ®E \subseteq ®U \subseteq \overline{®U} \subseteq V$.
            \end{itemize}

            \begin{dukazin}
                Jednoduché.
            \end{dukazin}
        \end{lemma}

        \begin{veta}[Urysohnovo lemma]
            TP ®X je normální $\Leftrightarrow$ pro každé dvě disjunktní uzavřené ®E, ®F existuje spojitá funkce $f:®X \rightarrow [0, 1]$, že $f(®E) \subseteq \{0\}, f(®F) \subseteq \{1\}$
            
            \begin{dukazin}
                Implikace zprava doleva je snadná -- uvažujeme $\{x \in ®X: f(x) < \frac{1}{2}\}$ a $\{x \in ®X: f(x)>\frac{1}{2}\}$.

                $\implies$ Označme $D:=®Q \cap \[0, 1\]$, $D= \{r_n: n \in ®N \cup \{0\}\}$, $r_0 = 0, r_1 = 1$ ($r_n$) prostá posloupnost. Indukcí najdeme otevřené množiny $®V_q: q \in D$, že pro $p, q \in D, p<q \implies ®V_p \subseteq ®V_q$ a navíc $®E \subseteq ®V_0, ®V_1 \subseteq ®X \setminus ®F$.

                Z normality najdeme otevřenou množinu ®U, že $®E \subseteq ®U \subseteq \overline{®U}\subseteq ®X \setminus \overline{r}$. Položíme $®V_0 = ®U$, $®V_1 = ®X \setminus ®F$.

                Nyní předpokládejme, že $®V_{r_0}, ®V_{r_1}, …, ®V_{r_n}, n≥1$. Už známe a platí, že pro $p, q \in \{r_0, …, r_n\}: p<q \implies \overline{®V_p} \subseteq ®V_q$. Chceme najít $®V_{r_{n+1}}$. Ať $i,j≤n$ jsou taková, že $r_i = \max\{r_k: r_k < r_{n+1}\}$ a $r_j = \min\{r_k: r_k > r_{n+1}\}$. $r_i < r_j$. Z 1P: $\overline{V_{r_i}} \subseteq V_{r_j}$. Z normality existuje otevřená $®V_{r_{n+1}}$, že $\overline{®V_{r_i}} \subseteq ®V_{r_{n+1}} \subseteq \overline{®V_{r_{n+1}}} \subseteq ®V_{r_j}$.

                Položme $f(x) = 1, x \in ®X \setminus ®V_1| f(x) = \inf{r\in D: x \in ®V_r}, x \in ®V_1$. $f:®X \rightarrow \[0, 1\]$. Nyní stačí ověřit spojitost: vzory subbázových (nějaké subbáze) podmnožin jsou otevřené. Zvolím si subbázi $\{\[0, b\), \(a,1\], a, b \in \(0, 1\)\}$. $f^{-1}(\[0, b\)) = \{x \in ®X: f(x)<b\} = \{x \in ®X: \exists r< b: x \in ®V_r\} = \bigcap_{r<b}®V_r…\text{otevřené}$. $f^{-1}(\(a, 1\]) = \{x \in ®X: f(x) > a\} = \{x \in ®X: \exists r>a: x\notin ®V_r\} = \{x \in ®X: \exists s>a: x \notin \overline{®V_s}\} = \bigcup_{s>a}®X \setminus \overline{®V_s}…\text{otevřené}$
            \end{dukazin}
        \end{veta}

        \begin{poznamka}[$T_4 \implies T_{3.5}$, normalita $\implies$ úplná regularita]
        
        \end{poznamka}
    

    \subsection{Konvergence v topologických prostorech}
        \begin{definice}[Usměrněné množiny]
            Dvojice $(®I, ≤)$ se nazývá usměrněná množina, pokud ®I je množina a $≤$ je binární relace na ®I, která je reflexivní, tranzitivní a pro $i, j \in ®I$, pak existuje $k\in ®I$, že $i≤k, j≤k$.

            \begin{prikladyin}
                $(®N, ≤)$
            \end{prikladyin}
        \end{definice}

        \begin{definice}[Net]
            Net v TP ®X je libovolné zobrazení z usměrněné množiny do ®X.
        \end{definice}

        \begin{definice}[Konvergence netu]
            Řekneme, že net $(x_i)_{i \in ®I}$ konverguje k bodu $x$, pokud $\forall ®U \in ©U(x) \exists i_0 \in ®I \forall i \in ®I, i≥ i_0: x_i \in ®U$. Pokud existuje právě jeden, značíme $x= \lim_{i\in®I}x_i$.

            Bod $x$ se nazývá hromadným bodem netu $(x_i)_{i \in ®I}$, pokud $\forall ®U \in ©U(x) \forall i \in ®I \exists j≥i: x_j \in ®U$.
        \end{definice}

        \begin{tvrzeni}[Jednoznačnost limity netu]
            Prostor ®X je Hausdorffův $\Leftrightarrow$ každý net má nejvýše jednu limitu.

% 26. 10. 2020

            \begin{dukazin}
                $(\implies)$: Ať $(x_i)_{i \in I}$ je net mající dvě různé limity $x, y \in ®X$. ®X je Hausdorffův, tedy existuje disjunktní okolí $U, V$ bodů $x, y$. Pak existuje $i \in I$, že $\forall j\in I, j≥i: x_j \int ®U$ a existuje $k \in I$, že $\forall j\in I, j≥k: x_j \int ®V$. $(I, ≤)$ je usměrněná množina, tedy existuje $l \in I$, že $l≥i$, $l≥k$. $x_l \in ®U \cap ®V$. $\lightning$.

                Opačně: Ať ®X není Hausdorffův. Ať $x, y \in ®X$ je dvojice různých bodů, které nejdou oddělit otevřenými disjunktními množinami. Uvažme otevřenou množinu $\(©U(x)\times ©V(y), ≤\)$, kde $(®A, ®B) ≤ (®U, ®V) ≡ (®U\subseteq ®U \land ®V \subseteq ®B)$. Pro každé $(®U, ®V) \in ©U(x)\times ©V(y)$ vezměme nějaký bod $x_{(®U, ®V)} \in ®U \cap ®V$. $\(x_{(®U, ®V)}\)_{(®U, ®V)\in ©U(x) \times ©U(y)}$ je net v $X$, který konverguje k $x$ a zároveň konverguje k $y$.
            \end{dukazin}
        \end{tvrzeni}

        \begin{tvrzeni}[Charakterizace uzávěru pomocí konvergence netů]
            Ať ®X je TP a $®A \subseteq ®X$. Pak $x\in \overline{®A}$, právě když existuje net $(x_i)_{i \in I}$ tvořený body z $®A$, který konverguje k $x$.

            \begin{dukazin}
                ($\implies$): Ať $x \in \overline{A}$. $\forall ®U\in ©U(x): ®U\cap ®A ≠ \O$. Fixujme $x_{®U}\in ®U \cap ®A$, pro $®U \in ©U(x)$. $\(©U, \supseteq\)$ je usměrněná množina. $(x_{®U})_{®U \in ©U(x)}$ je net tvořený prvky z ®A, který konverguje k $x$.

                ($\Rightarrow$): Ať $x\in ®X, (x_i)_{i\in I}$ je net z prvků $®A$, který konverguje k $x$. Chceme, $x\in \overline{®A}$. Ať ®U je okolí $x$. Chceme, že $U\cap®A ≠ \O$. $(x_i)$ konverguje k $x$, tedy existuje $j \in I: x_j \in ®U$. Navíc $x_j \in ®A$. $x_j \in ®A \cap ®U$.
            \end{dukazin}
        \end{tvrzeni}

        \begin{tvrzeni}[Charakterizace spojitosti pomocí netů]
            Ať ®X, ®Y jsou TP. $f: ®X \rightarrow Y$ je zobrazení, $x \in ®X$. Pak $f$ je spojité v bodě $x$ právě tehdy, když pro každý net $(x_i)_{i \in I}$ konvergující k bodu $x \in ®X$ konverguje net $(f(x_i))_{i\in I}$ k bodu $f(x)$.

            \begin{dukazin}
                ($\implies$): Ať $®V \in ©U(f(x))$. Pak ze spojitosti $\exists ®U \in ©U(x): f(®U)\subseteq ®V$. Net $(x_i)$ konverguje k $x$, tedy existuje $i_0 \in I$, že pro $i≥i_0: x_i \in ®U$. Pak zřejmě pro $i ≥ i_0: f(x_i)\in ®V$.

                ($\Rightarrow$): Ať $f$ není spojité v bodě $x$. Tedy existuje $®V \in ©U(f(x))$, že $\forall ®U \in ©U(x): f(®U)\setminus ®V ≠ \O$. Zvolme $x_{®U} \in ®U$, že $f(x_{®U}) \notin ®V$. $(x_{®U})_{®U \in ©U(x)}$ je net v ®X, zřejmě $(x_{®U})$ konverguje k $x$. $(f(x_{®U}))_{®U \in ©U(x)}$ zřejmě tedy nekonverguje k bodu $f(x)$. $\lightning$
            \end{dukazin}
        \end{tvrzeni}

\section{Operace s TP a zobrazeními}
    \subsection{Obecné konstrukce}
        \begin{definice}[Větší a menší topologie]
            Ať ®X je množina, $\tau, \sigma$ dvě topologie na ®X. Řekněme, že $\tau$ je větší (jemnější, silnější) než $\sigma$, pokud $\tau \supseteq \sigma$. Topologie $\sigma$ se pak nazývá menší (hrubší, slabší).
        \end{definice}

        \begin{poznamka}
            Topologie $\tau$ je větší než $\sigma$ $\Leftrightarrow$ $id_{®X}: (®X, \tau) \rightarrow (®X, \sigma)$ je otevřená.

            Jsou-li $\tau_i: i \in I$ topologie na ®X, pak $\bigcap_{i \in I}\tau_i$ je opět topologie na ®X. Navíc je největší topologií, která je menší než všechny $\tau_i$. $\bigcap_{i \in I} \tau_i$ je subbáze nějaké topologie, která je nejmenší topologie, která je větší než všechny $\tau_i$.
        \end{poznamka}

        \begin{definice}[Projektivní a induktivní vytváření]
            Ať ®X je množina a $(®X_i, \tau_i), i \in I$, jsou TP a $f_i: ®X \rightarrow ®X_i$ zobrazení.

            Topologie $\tau$ na množině ®X se nazývá projektivně vytvořená, pokud $\tau$ je nejmenší topologie, při níž jsou všechna zobrazení $f_i: (®X, \tau) \rightarrow (®X_i, \tau_i)$ spojitá.

            Jsou-li $f_i:®X_i \rightarrow ®X$ zobrazení, topologie $\tau$ na ®X se nazývá induktivně vytvořená, pokud $\tau$ je největší topologie na ®X, při které jsou všechna $f_i: (®X_i, \tau_i) \rightarrow (®X, \tau)$ spojitá.
        \end{definice}

        \begin{veta}[Charakterizace spojitosti zobrazení do projektivně definovaného TP]
            Ať $(®X, \tau)$ je projektivně vytvořen souborem zobrazení $f_i: ®X \rightarrow (®X_i, \tau_i)$. Zobrazení $g: (®Y, \sigma) \rightarrow (®X, \tau)$ je spojité $\Leftrightarrow \forall i \in I: f_i \circ g: (®Y, \sigma) \rightarrow (®X_i, \tau_i)$ je spojité.

            \begin{dukazin}
                Doprava je jednoduché, složení spojitých zobrazení je spojité.

                Opačně: Ať $\tau'$ je největší topologie na ®X, při které je zobrazení $g$ spojité: $\tau' = \{ ®U \subseteq ®X: g^{-1}(®U) \in \sigma\}$. Stačí, že $\tau \subseteq \tau'$. $\tau$ je nejmenší topologie, která obsahuje množiny $f^{-1}_i (®V), ®V \in \tau_i, i \in I$. Tedy stačí ukázat, že $f^{-1}_i (®V) \in \tau'$ pro $®V \in \tau_i, i \in I$. $g^{-1}(f^{-1}(®V)) = (f_i \circ g)^{-1}(®V)\in \sigma$. Tedy opravdu $f_i^{-1} \in \tau'$.
            \end{dukazin}
        \end{veta}
        
    \subsection{Podprostor, suma, součin, kvocient}
        \begin{definice}
            Je-li $(®X, \tau)$ TP a $A \subseteq ®X$, pak $(A, \sigma)$ se nazývá podprostor $(®X, \tau)$, pokud topologie $\sigma$ je projektivně vytvořená zobrazením identitou na $A$.

            Jsou-li $(®X_i, \tau_i)$ TP, pak je jejich součin TP $(®X, \tau)$, kde $X = \prod_{i \in I} ®X_i$ a $\tau$ je projektivně vytvořená zobrazeními $\pi_i: ®X \rightarrow ®X_i, \pi_i(…) = x_i$

            Zobrazení $f: (®X, \tau) \rightarrow (®Y, \sigma)$ se nazývá vnořené zobrazení, pokud $f$ je prosté a topologie $\tau$ na ®X je projektivně vytvořená zobrazením $f$.
        \end{definice}

        \begin{poznamka}
            Ať $(®X, \tau)$ je TP. $A\subseteq X$. $\tau_A := \{®U \cap A: ®U \in \tau\}$ je topologie podprostoru na $A$.

            Ať $(®X_i, \tau_i)$ jsou TP, $i \in I$. Ať $®X = \prod_{i \in I} ®X_i$, součinová topologie na ®X má subbázi: $©S := \{\pi^{-1}(®U): i \in I, ®U \in \tau_i\}$. 

            Konvergence netú v součinové topologii: Net $(x_j)_{j \in J}$ konverguje k $x \in X \Leftrightarrow \forall i \in I: (\pi_i(x_j))_{j \in J}$ konverguje k $\pi_i (c)$.

            Jsou-li $A_i \subseteq X_i$, pak $\overline{\prod A_i} = \prod \overline{A_i}$.
        \end{poznamka}

        \begin{priklad}
            $C(\[0, 1\], ®R) \subseteq ®R^{\[0, 1\]} := \{f:[0, 1] \rightarrow ®R, f \text{zobrazení}\}$.

            Topologie podprostoru $C…$ = „topologie bodové konvergence“.
        \end{priklad}

        \begin{definice}
            Ať $(®X, \tau)$ je TP, $E \subseteq ®X \times ®X$ ekvivalence. Uvažme $®X \setminus E = \{\[x\]_E : x \subseteq ®X\}$, $\pi: ®X \rightarrow ®X \setminus E, x\rightarrow [x]_E$. Kvocientová topologie na $®X \setminus E$ je induktivně vytvořená zobrazením $\pi$.

            Jsou-li $(®X_i, \tau_i)$ TP, $i \in I$, ($®X_i$ jsou po dvou disjunktní) pak topologie sumy na $\bigcap_{i \in I} ®X_i$ je topologie, která je induktivně vytvořena zobrazeními $j_i: ®X_i \rightarrow \bigcup_{k \in I}®X_k, j_i(x) = x$. Sumu TP značíme $\bigoplus_{i \in I}®X_i$.

            Zobrazení $f: (®X, \tau) \rightarrow (®Y, \sigma)$ se nazývá kvocientové, pokud je na a topologie $\sigma$ je induktivně vytvořená zobrazením $f$.
        \end{definice}

        \begin{priklad}
            $®X = ®R, E: xEy \Leftrightarrow x-y \in ®Z$. $®R \setminus E$ homeomorfní s kružnicí.
        \end{priklad}

        \begin{poznamka}
            Množina ®U v kvocientovém prostoru $®X \setminus E$ je otevřená $\Leftrightarrow \pi^{-1}(®U)$ je otevřená v ®X.

            $®X = \bigoplus ®X_i, ®U \subseteq ®X$. ®U je otevřená $\Leftrightarrow ®U \cap ®X_i$ je otevřená v $®X_i$.
        \end{poznamka}

% 2.11.2020

        \begin{priklad}
            Je-li $®X$ TP a $®Y \subseteq ®X$, $®M \subseteq ®Y$, pak $\overline{®M}^{®Y} = \overline{®M}^{®X}\cap ®Y$.
        \end{priklad}

        \begin{tvrzeni}[Charakterizace vnoření a kvocientových zobrazení]
            Ať $(®X, \tau)$ a $(®Y, \sigma)$ jsou TP a $f:®X \rightarrow ®Y$ zobrazení. Zobrazení $f:(®X, \tau) \rightarrow (®Y, \sigma)$ je vnoření $\Leftrightarrow f:(®X, \tau) \rightarrow (f(®X), \sigma|_{f(®X)})$ je homeomorfizmus. Zobrazení $f:(®X, \tau) \rightarrow (®Y, \sigma)$ je kvocientové zobrazení $\Leftrightarrow f$ je na a $\forall V \subseteq ®Y: V \in \sigma \Leftrightarrow f^{-1}(V) \in \tau$.

            \begin{dukazin}
                $f$ je vnoření $\Leftrightarrow$ $f$ je prosté a $\tau$ je projektivně vytvořená zobrazením $f: ®X \rightarrow (®Y, \sigma) \Leftrightarrow f: ®X \rightarrow f(®X)$ je bijekce a obě zobrazení $f: (®X, \tau) \rightarrow (f(®X), \sigma|_{f(®X)})$ a $f^{-1}: (f(x), \sigma|_{f(®X)}) \rightarrow (®X, \tau)$ jsou spojitá $\Leftrightarrow f:(®X, \tau) \rightarrow (f(®X), \sigma|_{f(®X)})$ je homeomorfismus.

                $f$ je kvocientové zobrazení $\Leftrightarrow$ $f$ je na a $\sigma = \sigma' \rightarrow (f$ je na a $\forall V \subseteq ®Y: V \in \sigma \Leftrightarrow f^{-1}(V)\in \tau)$.
            \end{dukazin}
        \end{tvrzeni}

        \begin{tvrzeni}[Postačující podmínka pro kvocientové zobrazení]
            Je-li $f: ®X \rightarrow ®Y$ spojité a otevřené (tj. obraz otevřené je otevřená) (nebo uzavřené, tj. obraz uzavřené je uzavřená) a na, pak $f$ je kvocientové zobrazení.

            \begin{dukazin}
                Použijeme přechozí charakterizaci kvocientového zobrazení. Ať $V \subseteq ®Y$. Pak 1) $V$ je otevřená v $®Y$, pak $f^{-1}$ je otevřená v ®X ze spojitosti. 2) $f^{-1}(V)$ otevřená v ®X. Pak z otevřenosti zobrazení $f$ máme, že $f(f^{-1}(V))$ (= $V$, protože $f$ je na) je otevřená v ®Y.

                Pro uzavřená zobrazení přes doplňky.
            \end{dukazin}
        \end{tvrzeni}

        \begin{poznamka}
            Jsou-li ®X, ®Y Banachovy prostory a $f: ®X \rightarrow ®Y$ lineární spojité a na, pak $f$ je otevřené.
        \end{poznamka}

        \begin{tvrzeni}[Charakterizace Hausdorffových prostorů]
            TP ®X je Hausdorffův $\Leftrightarrow \{(x, x) \in ®X\times ®X, x \in ®X\}$ je uzavřená v $®X \times ®X$.
        \end{tvrzeni}

        \begin{poznamka}
            Operace s TP jsou tranzitivní (součet, součin, kvocient, podprostor, …).
        \end{poznamka}

    \subsection{Zachovávání konstrukcemi}
        \begin{definice}
            Jsou-li $®X_i$ a $®Y_i$ TP, $i \in I$ a $f_i: ®X_i \rightarrow ®Y_i$ zobrazení, pak definujeme
            $$ \bigoplus_{i \in I}f_i : \bigoplus_{i \in I} ®X_i \rightarrow \bigoplus_{i \in I} ®Y_i, x \rightarrow f_i(x), \text{ pokud } x \in ®X. $$
            $$ \prod_{i \in I}f_i : \prod_{i \in I} ®X_i \rightarrow \prod_{i \in I} ®Y_i, \(x_i\)_{i \in I} \rightarrow \(f_i(x_i)\)_{i \in I}. $$
            Jsou-li $®X_i = ®X$, $i \in I$, pak definujeme tzv. diagonální zobrazení
            $$ \triangle_{i \in I} f_i: ®X \rightarrow \prod_{i \in I} ®Y_i, x \rightarrow \(f_i(x)\)_{i \in I}. $$ 
        \end{definice}

        \begin{tvrzeni}
            Součinové, součtové a diagonální zobrazení odvozené od spojitých je spojité.

            \begin{dukazin}
                Plyne z charakterizace spojitého zobrazení do projektivně vytvořeného prostoru.
            \end{dukazin}
        \end{tvrzeni}

        \begin{dusledek}
            Ať ®X je TP a $f, g: ®X \rightarrow ®R$ spojité, pak $f+g, f-g, f·g, \max\{f, g\}, \min\{f, g\}, |f|, \frac{f}{g} (g≠0)$ jsou spojitá.

            \begin{dukazin}
                $f\triangle g: ®X \rightarrow ®R^2, (f \triangle g)(x) = (f(x), g(x))$ je spojité. Následně toto zobrazení spojíme s $+, -, …$, která jsou spojitá, tedy i výsledek je spojitý.
            \end{dukazin}
        \end{dusledek}

        \begin{tabular}{|c|c|c|c|c|c|c|c|c|c|c|}
            \hline
                               & $T_0$ & $T_1$ & $T_2$ & $T_3$ & $T_\pi$ & $T_4$ & Separabilní & Spoč. báze & Spoč. charakter & Metrizovatelnost \\ \hline
                podprostor     & Ano   & Ano   & Ano   & Ano   & Ano     & Ne    & Ne          & Ano        & Ano             & Ano              \\ \hline
                (spoč.) suma   & Ano   & Ano   & Ano   & Ano   & Ano     & Ano   & (Ano) Ne    & (Ano) Ne   & Ano             & Ano              \\ \hline
                kvocient       & Ne    & Ne    & Ne    & Ne    & Ne      & Ne    & Ano         & Ne         & Ne              & Ne               \\ \hline
                (spoč.) součin & Ano   & Ano   & Ano   & Ano   & Ano     & Ne    & (Ano) Ne    & (Ano) Ne   & (Ano) Ne        & (Ano) Ne         \\ \hline
        \end{tabular}

        \begin{tvrzeni}[(Úplná) regularita se zachovává součinem]
            Jsou-li TP $X_i,\ i\in I$ (úplně) regulární, pak $\prod_{i \in I} X_i$ je (úplně) regulární.

            \begin{dukazin}
                $X:= \prod_{i \in I} X_i$. Ať $F \subseteq X$ je uzavřená a $x \in X \setminus F$. Z definice součinové topologie existuje $K \setminus I$ konečná a otevřené $U_i \subseteq X_i,\ i \in K$, že $x \in \cap_{i \in K} \pi_i^{-1} (U_i) \subseteq X \setminus F$.

                Tedy $x_i \in U_i,\ i \in K$. $X_i$ regulární, tedy existuje $G_i \subseteq X_i$ otevřená, že $x_i \in G_i \subseteq \overline{G_i} \subseteq U_i$. TODO dlouhý vzorec.
            \end{dukazin}
        \end{tvrzeni}

    \subsection{Rozšiřování spojitých funkcí}
        \begin{tvrzeni}
            Ať $X, Y$ jsou TP, $f, g: X \rightarrow Y$ spojitá. Pokud $Y$ je Hausdorffův, pak $M:=\{x\in X: f(x) = g(x)\}$ je uzavřená v $X$.

            \begin{dukazin}
                Ať $x \in X\setminus M$. Pak $f(x)≠g(x) \in Y$. $Y$ je Hausdorffův, tedy existují otevřené disjunktní $U, V$, že $f(x) \in U, g(x) \in V$. Ať $W := f^{-1}(U) \cap g^{-1}(V)$ je otevřená množina a $x \in W$. $W \cap M = \O$, protože $U$ a $V$ jsou disjunktní, tedy $X \setminus M$ je otevřená, $M$ je uzavřená.
            \end{dukazin}
        \end{tvrzeni}

        \begin{poznamka}
            Je-li $f: X \rightarrow Y$ spoj., $Y$ Hausdorffův a $S \subseteq X$ hustá, pak $fcosiS$ má  jediné spojité rozšíření.
        \end{poznamka}

        \begin{tvrzeni}
            Je-li $X$ TP a $f_n: X \rightarrow ®R$ spoj. zobrazení $(f_n)$ konverguje stejnoměrně k $f: X \rightarrow ®R$, pak $f$ je spojité.

            \begin{dukazin}
                TODO!
            \end{dukazin}
        \end{tvrzeni}

        \begin{veta}[Tietze-Urysohnova]
            Je-li $®X$ normální TP a $F \subseteq ®X$ uzavřená, pak lze každou spojitou funkci $f: F \rightarrow ®R$ spojitě rozšířit na celé $®X$, tedy existuje spojitá funkce $\overline{f}: ®X\rightarrow ®R$, že $\overline{f}cosiF = f$.

% 9. 11. 2020

            \begin{dukazin}
                Pozorování: Ke každé spojité funkci $g: F \rightarrow \[-c, c\]$ existuje spojitá funkce $\overline{g}: X \rightarrow \[-\frac{c}{3}, \frac{c}{3}\]$, že $|g(x) - \overline{g}(x)|≤ \frac{2}{3}c$ pro každé $x \in F$.

                Důkaz pozorování: Ať $E := \{x \in F: g(x)≤-\frac{c}{3}\}$ a $H := \{x \in F: g(x)≥\frac{c}{3}\}$. $E, H$ uzavřené v $F$ a disjunktní. Tedy $E, H$ uzavřené v $®X$. Tedy z Urysohnova lemmatu existuje spojitá $h: ®X \rightarrow \[-1, 1\]$, že $h(E) \subseteq \{-1\}, h(H) \subseteq \{1\}$. Položme $\overline{g} := \frac{c}{3}·h$. Jednoduše nahlédneme, že vzdálenosti z pozorování teď fungují.

                Nejprve dokažme pro $f: F \rightarrow \[-1, 1\]$ (a rozšíříme jí na spoj. $\overline{f}: ®X \rightarrow \[-1, 1\]$). Indukcí najdeme posloupnost spojitých funkcí $g_n: ®X \rightarrow ®R$, že $||g_n|| ≤ \frac{1}{3}·\(\frac{2}{3}\)^{n-1}$ a pro každé $x \in F$ a $n \in ®N: |f(x) - \sum_{i=1}^n g_i(x)| ≤ \(\frac{2}{3}\)^n$.

                Položme $g_1 = \overline{f}$ z pozorování, tedy $g_1: ®X \rightarrow \[-\frac{1}{3}, \frac{1}{3}\]$. Máme-li $g_1, …, g_n$ zkonstruované a splňující předpoklady indukce, pak uvažujme funkci $f' := f - \sum_{i = 1}^n g_i: F \rightarrow \[-\(\frac{2}{3}\)^n, \(\frac{2}{3}\)^n\]$ a aplikujeme na ni pozorování, tedy existuje spojitá funkce $g_{n+1}: ®X \rightarrow \[-\frac{1}{3}\(\frac{2}{3}\)^n, \frac{1}{3}\(\frac{2}{3}\)^n\]$, že $|f'(x) - g_{n+1}(x)|≤ \frac{2}{3}\(\frac{2}{3}\)^n = \(\frac{2}{3}\)^{n+1},\ x\in F$. Položme $\tilde{f_n}:= \sum_{i = 1}^n g_i(x)$ a $\tilde{f} := \sum_{i=1}^∞ g_i(x) = \lim_{n \rightarrow ∞}\tilde{f_n}(x)$. Stejnoměrná konvergence zachovává spojitost. A jelikož $\tilde{f_n}$ z Waierstrassova kriteria konverguje stejnoměrně, tak $\tilde{f}$ je spojitá. Zároveň $|\tilde{f}(x) - f(x)| = 0$, tedy $\tilde{f}$ je rozšířením $f$.

                Ať nyní $f: F \rightarrow ®R$. Ať $h: ®R \rightarrow (-1, 1)$ je homeomorfismus. $h\circ f: F\rightarrow (-1, 1) \subseteq \[-1, 1\]$ podle předchozí části existuje spojité $v: ®X \rightarrow \[-1, 1\]$, že pro $x \in F$ je $v(x) = h \circ f(x)$. Ať $E := v^{-1}(\{-1, 1\})$ uzavřená v ®X. $E$ je disjunktní s $F$. Z Urysohnova lemmatu existuje spojité $m: ®X \rightarrow \[0, 1\], m(E) \subseteq \{0\}, m(F)\subseteq \{1\}$. $m \circ v: ®X \rightarrow (-1, 1)$. Tudíž $h^{-1} \circ (m \circ v): ®X \rightarrow ®R$ je spojité a navíc $(h^{-1}\circ (m \circ v))(x) = f(x)$ pro $x \in F$.
            \end{dukazin}
        \end{veta}

\section{Kompaktnost}
    \begin{definice}
        Systém množin ©S se nazývá pokrytí ®X, pokud $\bigcup ©S = ®X$. Každý podsystém ©S, který je také pokrytí, se nazývá podpokrytí.

        Pokrytí se nazývá otevřené, pokud všechny jeho prvky jsou otevřené množiny.

        TP ®X se nazývá kompaktní, pokud každé jeho otevřené pokrytí má konečné podpokrytí.

        TP ®X se nazývá spočetně kompaktní, pokud každé spočetné pokrytí má konečné podpokrytí.

        TP ®X se nazývá Lindelöfův, pokud každé otevřené pokrytí má spočetné pokrytí.

        Řekneme, že systém $©F \subseteq ©P(®X)$ je centrovaný, pokud pro každé $n \in ®N$ a $F_1, …, F_n \in ©F$ je $F_1 \cap … \cap F_n ≠ \O$.
    \end{definice}

    \begin{veta}[Charakterizace kompaktnosti]
        Pro TP ®X je ekvivalentní: a) ®X je kompaktní. b) Každý centrovaný systém sestávající z uzavřené množiny má neprázdný průnik. c) Každý net má limitu? TODO

        \begin{dukazin}
            $(a \implies b)$ Ať $©F \subseteq ©P(®X)$ sestává z uzavřených množin a je centrovaný. Položme $©U := \{®X \setminus F: F \in ©F\}$ (systém otevřených množin). Ať pro spor $\bigcap ©F = \O$. Pak ©U je pokrytí ®X. ®X je kompaktní, tedy existuje $U_1, …, U_n \in ©U$, že $U_1\cup … \cup U_n = ®X$. $U_i = ®X \setminus F_i$ pro něj $F_i \in ©F$. Pak $\bigcup_{i = 1}^n F_i = \O$. Tedy $©F$ není centrovaný, $\lightning$.

            $(b \implies c)$ Ať $(x_i)_{i \in I}$ je net v ®X, $(I, ≤)$ usměrněná množina. Položme $F_i = \{x_j: j ≥ i\}$ je uzavřená, $i \in I$. $©F = \{F_i| i \in I\}$ je centrovaný. Tedy dle b) $\bigcap ©F ≠ \O$. Ať $x_0 \in \bigcap ©F$. Pak $x_0$ je hromadným bodem netu $(x_i)_{i \in I}$.

            $(c \implies a)$. Ať ©U je otevřená podmnožina ®X. Předpokládejme pro spor, že neexistuje konečné pokrytí. Tedy pro $©F \subseteq ©U$ konečnou existuje bod $x_{©F} \in ®X \setminus \bigcup ©F$. $(x_{©F})_{©F \subseteq ©U}$ je net v ®X. Podle c) existuje hromadný bod $x$ tohoto netu. Existuje $U \in ©U: x \in U$. Z definice hromedného bodu existuje $©F \subseteq ©U$ konečné, že $©F \supseteq \{U\}$ a $x_{®F} \in U$. Ale $x_{®F} \notin \bigcup ©F. To je spor.$
        \end{dukazin}
    \end{veta}

    \begin{tvrzeni}[Zachovávání vlastností]
        Kompaktnost, spočetná kompaktnost i lindelöfovost se dědí na uzavřené podprostory a spojité obrazy.

        \begin{dukazin}
            Ukážeme pouze pro kompaktnost: Ať ®X je kompaktní a $F \subseteq ®X$ uzavřená. Ať tedy ©U je otevřené pokrytí $F$. Pro každé $U \in ©U$ existuje $\tilde{U}$ otevřená v ®X, že $\tilde{U} \cap F = U$. Označme $\tilde{©U} = \{\tilde{U}: U \in ©U\} \cup \{X \setminus F\}$. $\tilde{©U}$ otevřené pokrytí ®X, tedy z kompaktnosti ®X existuje konečné podpokrytí $\{\tilde{U_1}, …, \tilde{U_n}, ®X \setminus F\}$. Pak $\{U_1, …, U_n\}$ je pokrytí $F$ vybrané z ©U. Tedy $F$ je kompaktní.

            Ať $f: ®X \rightarrow ®Y$ na, spojité a ®X kompaktní. Ať ©U je otevřené pokrytí ®Y. TODO otevřené pokrytí ®X. ®X je kompaktní, tedy existuje TODO. Pak TODO pokrývá ®Y.
        \end{dukazin}
    \end{tvrzeni}

    \begin{dusledek}[Nabývání extrému]
        Spojitá reálná funkce na (spočetně) kompaktním neprázdném prostoru nabývá maxima a minima.

        \begin{dukazin}
            $f: ®X \rightarrow ®R$, spojitá, ®X spočetně kompaktní. $f(®X)$ je spočetně kompaktní $\Leftrightarrow$ kompaktní (v metrických prostorech). Tedy $f(®X)$ uzavřená omezená v ®R, tedy má minimum a maximum.
        \end{dukazin}
    \end{dusledek}

    \begin{veta}[Postačující podmínky pro normalitu]
        Regulární Lindelöfův TP je normální.

        Hausdorffův kompaktní TP je normální (tedy $T_4$).

        \begin{dukazin}
                a) Ať $E$, $F$ jsou uzavřené disjunktní. $\forall x \in E \exists$ otevřené $U_x \in ©U(x)$, že $\overline{U_x} \cap F = \O$. $\{U_x: x\in ®X\}$ je otevřené pokrytí $E$. Z lindelöfovosti $E$ (uzavřený podprostor ®X) existuje $C \subseteq ®X$ spočetné, že $\{U_c: c \in C\}$ pokrývá $E$. Přeindexujeme systém $\{U_c: c \in C\}$ na $\{U_i: i \in ®N\}$. Analogicky najdeme $\{V_j: j \in ®N\}$ systém otevřených množin pokrývající $F$, $\overline{V_j} \cap E = \O,\ k \in ®N$. TODO

                $U := \bigcup_{i \in ®N} U_i*, $ otevřené. Kdyby $x \in U \cap V$, pak existují $i, j \in …®N: x \in U_i*\cap V_j*$. Búno: $i ≥ j: x \in U_i \setminus \bigcup_{k ≤ i} \overline{V_k}$, $x \notin \overline{V_j}, x \notin V_j* \lightning$. Tedy ®X je normální.

                b) Ať ®X je Hausdorffův kompaktní. Stačí ukázat, že ®X je regulární a použít a). Ať $F \subseteq ®X$ je uzavřená, $x \in ®X \setminus F$. Pro $y \in F$ existují otevřené disjunktní $V_y$, $U_y$, že $x \in U_y, y \in V_y$. $F$ je kompaktní, $\{V_y : y \in F\}$ je otevřená podmnožina $F$. Tedy existuje konečné podpokrytí $\{V_{y_1}, …, V_{y_n}\}$. $U := \bigcap_{i = 1}^n U_{y_i}$, $V := \bigcap_{i = 1}^n V_{y_i}$ otevřené, tedy ®X regulární.
        \end{dukazin}
    \end{veta}

% 16. 11. 2020

    \begin{tvrzeni}
            Kompaktní podprostory (®K) jsou uzavřené v Hausdorffových prostorech (®X).
        \begin{dukazin}
                Pro $x \in ®X \setminus ®K$ fixované a $y \in ®K$ existují disjunktní $U_y$ a $V_y$ v ®x, že $x \in U_y$ a $y \in V_y$. $\{V_y: y \in ®K\}$ je otevřené pokrytí ®K. ®K je kompaktní, tedy $exists y_1, …, y_n \subseteq ®K$, že $V_{y_1} \cup … \cup V_{y_n} \supseteq ®K$. $x \in \bigcap_{i=1}^n U_{y_i}$ otevřené je disjunktním s $\bigcup V_{y_i}$, tedy i disjunktní s ®K. Tedy $®X \setminus ®K$ je otevřená, tj. ®K je uzavřená.
        \end{dukazin}
    \end{tvrzeni}

    \begin{tvrzeni}[Automatický homeomorfismus]
        Ať ®X, ®Y jsou kompaktní Hausdorffovy TP a $f: ®X \rightarrow ®Y$ spojitá. a) Pokud je $f$ na, pak $f$ je kvocientové. b) Pokud je $f$ bijekce, pak $f$ je homeomorfismus.

        \begin{dukazin}
            a) Stačí ukázat, že $f$ je uzavřené zobrazení. Ať $F \subseteq ®X$ je uzavřená. Pak $F$ je kompaktní, $f$ spojitá, tedy $f(F)$ je kompaktní. Podle předchozího tvrzení je $f(F)$ uzavřená v ®Y.

            b) Okamžitý důsledek a).
        \end{dukazin}
    \end{tvrzeni}

    \begin{poznamka}
        Ať $\tau$ je kompaktní Hausdorffova topologie na ®X. Pak $\tau$ je maximální kompaktní topologie a minimální Hausdorffova.
    \end{poznamka}

    \begin{lemma}[Alexandrovo]
        Ať ®X je TP a ©S jeho subbáze. Předpokládejme, že z každého pokrytí $©U \subseteq ©S$ lze vybrat konečné podpokrytí. Pak ®X je kompaktní.
        \begin{dukazin}[Sporem]
            Předpokládejme pro spor, že existuje otevřené pokrytí ®X, které nemá konečné podpokrytí. Označme $©P$ množinu všech takových pokrytí. $©P ≠ \O$. Ať $©L \subseteq ©P$ je řetězec vzhledem k $\subseteq$. Pak $\bigcup ©L \in ©P:$ Zřejmě $\bigcup ©L$ je otevřené pokrytí ®X. Kdyby existovalo $©F \subseteq \bigcup ©L$ konečné podpokrytí, pak existuje $©U \in ©L$, že $©F \subseteq ©U$. Tedy ©U má konečné podpokrytí $\lightning$.

            Tedy podle Zormova lemmatu existuje maximální prvek $©U \in ©P$. Ukážeme, že $©U \cap ©S$ je pokrytí ®X: Ať $x \in ®X$. Pak existuje $U \in ©U: x \in U$. Zároveň existuje $S_1, …, S_n, n \in ®N$, že $x \in S_1 \cap … \cap S_n \subseteq U$. Tvrdíme, že pro nějaké $i ≤ n: S_i \in ©U$: Kdyby ne, pak $\forall i ≤ n: S_i \notin ©U$. Tedy $©U \cup \{S_i\} \notin ©P$ (©U byl maximální prvek ©P). Tedy existuje $©F_i \subseteq ©U$ konečná, že $©F_i \cap \{S_i\}$ je pokrytí ®X. Pak $\bigcup_{i ≤ n} ©F_i \cup \{U\}$ je pokrytí ®X, je konečné, je to podpokrytí ©U. Spor.

            Tedy $x \in S_i \in ©S \cap ©U$. Tedy $©U \cap ©S$ je pokrytí ®X. Podle předpokladu má $©S \cap ©U$ konečné podpokrytí, tedy ©U má konečné podpokrytí. Spor s volbou ©U.
        \end{dukazin}
    \end{lemma}

    \begin{veta}[Tichonova]
        Součin kompaktních prostorů je kompaktní.

        \begin{dukazin}
            Ať $(®X_i, \tau_i)$, $i \in I$, jsou kompaktní TP. Ať $®X = \prod ®X_i$. $\tau$ součinová topologie na ®X. Ať $©S := \{\pi_i^{-1}(U): U \in \tau_i, i \in I\}$. ©S je subbáze $\tau$. Ověříme, že ©S splňuje předpoklady Alexandrova lemmatu. Ať $©U \subseteq ©S$ je pokrytí ®X. Pro $i \in I$ označme $©U_i = \{U \in \tau_i: \pi_i^{-1}(U) \in ©U\}$. Tvrdíme, že existuje $i \in I$, že $©U_i$ je pokrytí $®X_i$: Kdyby ne, pak $\forall i \in IL ©U_i$ nepokrývá $®X_i$, tedy existuje $x_i \in ®X_i \setminus \bigcup \{V: V \in ©U_i\}$. Nyní $x:=(x_i)_{i \in I} \in ®X$, ale $x\notin \bigcup ©U$. Tedy ©U nepokrývá ®X, spor.

            $®X_i$ je kompaktní, tedy existuje $©K \subseteq ©U_i$ konečná, že $\bigcup ©K = ®X_i$. Zřejmě $\{\pi_i^{-1}(U): U \in ©K\}$ je konečné podpokrytí ®X prvky z ®U. Podle Alexandrova lemmatu je ®X kompaktní.
        \end{dukazin}
    \end{veta}

    \begin{tvrzeni}[Spojitý obraz kompaktu nezvýší váhu]
        Ať ®X je kompaktní a ®Y Hausdorffův. Ať $f: ®X \rightarrow ®Y$ je spojité a na. Potom $\weight(®Y) ≤ \weight(®X)$.
        
        \begin{dukazin}
            Ať ©B je báze ®X. Můžeme předpokládat, že ©B je uzavřená na konečné sjednocení (tím nezvýšíme mohutnost nekonečné báze). Definujeme $©C:=\{®Y \setminus f\(®X \setminus B\): B \in ©B\}$. ©C sestává z otevřené množiny. Ukážeme, že ©C je báze ®Y. Ukážeme, že ©C je báze ®Y: Ať $y \in ®Y$ a $V \in ©U(y)$ otevřená, pak $f^{-1}(y) \subseteq f^{-1}(V)$. ©B je báze, tedy $\forall x \in f^{-1}(y) \exists B_x \in ©B: x \in B_x \subseteq f^{-1}(V)$. $f^{-1}(y)$ je kompaktní a $\{B_x: x \in f^{-1}(y)\}$  je otevřené pokrytí $f^{-1}(y)$. Tedy existuje $x_1, …, x_n \in f^{-1}(y): B_{x_1} \cup … \cup B_{x_n} \supseteq f^{-1}(y)$.

            Navíc $B \subseteq TODO$.
        \end{dukazin}
    \end{tvrzeni}

\section{Prostory spojitých funkcí na kompaktech}
    \begin{definice}
        Pro TP ®X, ®Y značíme symbolem $C(®X, ®Y)$ množinu všech spojitých funkcí ®x do ®Y. Pokud $®Y = ®R$, pak píšeme pouze $C(®X)$.

        Topologie bodové konvergence: $C(®X, ®Y) \subseteq ®Y^{®X}$ se součinovou topologií.

        Pro ®X kompaktní je $C(X)$ se supremovou normou Banachův prostor.
    \end{definice}

    \begin{tvrzeni}[Diniho kriterium pro stejnoměrnou konvergenci]
        Ať ®X je kompaktní TP a $f_n: ®X \rightarrow ®R$ spojitá, že $f_{n+1} ≥ f_n, n \in ®N$ a $f_n$ bodově konverguje ke spojité funkci $f$. Pak $f_n$ konverguje stejnoměrně k $f$.
        \begin{dukazin}
                Ať $\epsilon > 0$ ať $D_n = \{x \in ®X: f(x) - f_n(x) < \epsilon\}$. Pak $D_n$ je otevřená pro $n \in ®N$. Navíc z monotonie $D_1 \subseteq D_2 \subseteq …$. $\bigcup_{n=1}^∞ = ®X$. ®X kompaktní, tedy existuje $n_0$, že $D_{n_0} = ®X$. Tedy $|f(x) - f_{n_0}(x)| < \epsilon$ pro libovolné $x \in ®X$. Pak pro $n ≥ n_0: f(x) - f_n(x) ≤ f(x) - f_{n_0}(x) < \epsilon$.
        \end{dukazin}
    \end{tvrzeni}

    \begin{lemma}[O odmocnině]
        Existuje posloupnost polynomů, která na intervalu $\[0, 1\]$ konverguje stejnoměrně k $\sqrt{t}$.

        \begin{dukazin}
                Položme $p_0(t) = 0$, $p_{n+1}(t) = p_n(t) + \frac{t - p_n^2(t)}{2}, n≥0$. Každé $p_n$ je polynom v proměnné $t$. $\forall n \in ®N: p_n(t) ≤ \sqrt{t}$ a $p_1(t) ≤ p_{n+1}(t), t \in \[0, 1\]$ (dokazatelné indukcí). Tedy pro každé $t \in \[0, 1\]$ je posloupnost $(p_n(t))_{n=1}^∞$ neklesající a shora omezená, tedy má vlastní limitu $L = L + \frac{t - L^2}{2} \implies L = \sqrt{t}$.
        \end{dukazin}
    \end{lemma}

% 23. 11. 2020

    \begin{definice}
        Ať ©F je systém funkcí $®X \rightarrow ®Y$. Řekneme, že ©F odděluje body, pokud pro $x, y \in ®X, x≠y$ existuje $f\in©F: f(x)≠f(y)$. Řekneme, že ©F odděluje body a uzavřené množiny, pokud $\forall F \subseteq ®X$ uzavřená $\forall x \in ®X \setminus F \exists f \in ©F: f(x) \notin \overline{f(F)}$.
    \end{definice}

    \begin{poznamka}[Připomínáme svaz a okruh.]
        $\O ≠ A \subseteq C(K)$, $A$ je okruh $\Leftrightarrow A$ je uzavřená na násobení, sčítání a odčítání funkcí.

        $A$ je svaz $\Leftrightarrow A$ je uzavřená na minimum a maximum dvou funkcí.
    \end{poznamka}

    \begin{veta}[Stone-Weierstrass]
        Ať ®K je kompaktní TP a $©B \subseteq C(®K)$ je vektorový podprostor obsahující konstanty a oddělující body. Je-li $©B$ okruh nebo svaz, pak $©B$ je hustá v $C(®K)$ (s topologií stejnoměrné konvergence).

        \begin{dukazin}[Svaz]
            Pozorování: pro $x, y \in ®K, x≠y, a, b \in ®R$ existuje $h \in ©B: h(x) = a \land h(y) = b$. ($©B$ odděluje body, tedy $\exists u \in ©B: u(x) ≠ u(y)$. $h(z) := \frac{b-a}{u(y) - u(x)}·(u(z) - u(x)) + a \in ©B$.)

            Chceme, že $©B$ je hustý. Ať $f \in C(®K)$ a $\epsilon > 0$. Pro každou dvojici $x, y \in ®K$ fixujeme $f_{x, y} \in ©B$, že $f_{x, y}(x) = f(x)$ a $f_{x, y}(y) = f(y)$ (to můžeme díky pozorování pro $x≠y$, pro $x=y$ položíme $f_{x, y} = f(x)$.) Ať $x \in ®K$ je pevné. $\forall y \in ®K\ \exists \text{otevřené} U_y \in ©U(y): |f_{x, y}(z) - f_{x, y}(y)| < \frac{\epsilon}{2}$ (ze spojitosti $f_{x, y}$) a zároveň $|f(z) - f(y)| < \frac{\epsilon}{2}$ pro $z \in U_y$ (ze spojitosti $f$). Systém $\{U_y: y \in ®K\}$ je otevřené pokrytí ®K, tedy existuje $y_1, …, y_k \in ®K: U_{y_1} \cup … \cup U_{y_k} = ®K$. Definujeme $f_x := \min\{f_{x, y_1}, …, f_{x, y_k}\} \in ©B$ (©B je svaz). Navíc $f_x ≤ f+\epsilon$ na celém $®K$ a také $f_x(x) = f(x)$. Nyní $\forall x \in ®K\ \exists\text{otevřené} V_x \in ©U(x): |f_x(z) - f_x(x)|<\frac{\epsilon}{2}$ a $|f(z) - f(x)| < \frac{\epsilon}{2}$ pro $z \in V_x$. $\{V_x: x \in ®K\}$ otevřené pokrytí ®K. Tedy existují $x_1, …, x_l: V_{x_1}\cup … \cup V_{x_l} = ®K$. $g:=\max\{f_{x_1}, …, f_{x_l}\}, g \in ©B$ a $f-\epsilon ≤ g ≤ f + \epsilon$, tj. $||f-g||≤\epsilon$.
        \end{dukazin}

        \begin{dukazin}[Okruh]
            Ukážeme, že je-li $©A \subseteq C(®K)$ VP a okruh obsahující konstanty a oddělující body, pak $©B = \overline{©A}$ je svaz. Zřejmě $©B$ je VP obsahující konstanty a odděluje body (je nadmnožinou, proto vše obsahuje). Ukážeme, že pro $f \in ©B$ je $|f| \in ©B$: $f$ je omezená, tedy existuje $c>0$, že $c·f^2: ®K \rightarrow \[0, 1\]$. Z lemmatu o odmocnině existují polynomy $p_n: \[0, 1\] \rightarrow ®R$, že $p_n$ stejnoměrně konverguje k odmocnině. Tedy $p_n \circ (c·f^2) \in \overline{©A}$ stejnoměrně konverguje k $\sqrt{c}·|f| \in \overline{©A}$ (z uzávěru se nedá vykonvergovat). Tedy $|f|\in \overline{©A}$, tedy i $\max\{f, g\} = \frac{f+g+|f-g|}{2}$, tedy $\overline{A}$ je uzavřená na maxima (minima podobně).  Z první části $\overline{\overline{©A}} = C(®K)$, tedy $\overline{©A} = C(®K)$.
        \end{dukazin}
    \end{veta}

    \begin{definice}[Kompaktifikace]
        Ať ®X je TP. Dvojice $(j, ®Y)$ se nazývá kompaktifikací ®X, pokud ®Y je kompaktní Hausdorfův prostor a $j: ®X \rightarrow ®Y$ je vnoření a $j(®X)$ je hustá v ®Y.

        Prostor ®X a $j(®X)$ se často ztotožňují, tedy pak zapomínáme na $j$ a mluvíme pouze o ®Y, jakožto kompaktifikaci.

        Řekneme, že kompaktifikace $(j_1, ®Y_1)$ prostoru ®X je větší než kompaktifikace $(j_2, ®Y_2)$ prostoru ®X, pokud existuje spojité zobrazení $f: ®Y_1 \rightarrow ®Y_2$, že $f\circ j_1 = j_2$.

        Řekneme, že 2 kompaktifikace $(j_1, ®Y_1)$ a $(j_2, ®Y_2)$ jsou ekvivalentní, pokud existuje homeomorfismus $h: ®Y_1 \rightarrow ®Y_2$, $h \circ j_1 = j_2$.
    \end{definice}

    \begin{poznamka}
        Jsou-li $(j_1, ®Y_1)$, $(j_2, ®Y_2)$ kompaktifikace prostoru ®X a $(j_1, ®Y_1)$ je větší než $(j_2, ®Y_2)$ a naopak, pak jsou již ekvivalentní.
        \begin{dukazin}
            $(f_2 \circ f_1)\circ j_1 = f_2 \circ (f_1  \circ j_1) = f_2 \circ j_2 = j_1$. Obdobně $f_1 \circ f_2 \circ j_2$. Tedy $f_2 \circ f_1 = id$ na $j_1(®X)$ a $f_1 \circ f_2 = id$ na $j_2(®X)$. Jenže $j_1(®X)$ je hustá v $®Y_1$, tedy nutně $f_2 \circ f_1 = id$ na celém $®Y_1$. Úplně analogicky $f_1 \circ f_2 = id$ na $®Y_2$, tedy $f_1$ a $f_2$ jsou vzájemně inverzní a navíc spojité, tedy jsou homeomorfismy, tj. kompaktifikace jsou ekvivalentní.
        \end{dukazin}
    \end{poznamka}

    \begin{priklady}
        $®Y_1 = [0, 1]$ je kompaktifikací $®X = (0, 1)$ ($j_1: (0, 1) \rightarrow [0, 1], j_1(x) = x$). $®Y_2 = \{z \in ®C: |z| = 1\}$. $j_2: ®X \rightarrow ®Y_2, j_2(x) = e^{2 \pi i x}$. $(j_2, ®Y_2)$ je kompaktifikací $(0, 1)$. Zřejmě $(j_1, ®Y_1)$ je větší než $(j_2, ®Y_2)$. Naopak zřejmě nejsou ekvivalentní.
    \end{priklady}

    \begin{definice}[Lokální kompaktnost]
        TP ®X se nazývá lokálně kompaktní, pokud každý jeho bod má kompaktní okolí.
    \end{definice}

    \begin{tvrzeni}[Alexandrovova kompaktifikace]
        Každý Hausdorfův lokálně kompaktní prostor ®X má kompaktifikaci $(e, ®Y)$, při které je $®Y \setminus e(®X)$ nejvýše jednoprvková. Ta je určena jednoznačně.
        \begin{dukazin}
            Je-li ®X kompaktní, pak má až na ekvivalenci jedinou kompaktifikaci: nechť $(e, ®Y)$ je kompaktifikace ®X. Potom $e(®X)$ je kompaktní (spojitý obraz kompaktu), $e(x)$ je uzavřená a hustá v ®Y, tedy $e(®X) = ®Y$. Tedy $e$ je homeomorfismus a tato kompaktifikace je tedy ekvivalentní s libovolnou další.

            Předpokládejme, že $®X$ není kompaktní. BÚNO $∞ \notin ®X$. Ať $\tau$ je topologie na ®X. $®Y = ®X \cup \{∞\}$. Na ®Y definujeme topologii $\sigma$:
            $$\sigma := \tau \cup \{\{∞\} \cup (®X \setminus K): K \subseteq\text{ je kompaktní}\}.$$
            Snadno ověříme, že $\sigma$ je topologie na ®Y. $(®Y, \sigma)$ je kompaktní: Ať ©U je otevřené pokrytí ®Y. $\exists U_0 \in ©U: ∞ \in U_0\ \exists K \subseteq ®X$ kompaktní: $U_0 = \{∞\} \cup (®x \setminus K)$. $K$ je pokryto otevřeným pokrytím $\{K \cap U: U\in ©U\}$, tedy z kompaktnosti existují $U_1, …, U_n \in ©U: K \subseteq U_1 \cup … \cup U_n$. $\{U_0, U_1, …, U_n\}$ je tedy konečné podpokrytí ®Y.

            $(®Y, \sigma)$ je Hausdorffův: Pro $x, y \in ®X\ \exists U, V …$. Zajímavější je to pro $∞$ a $x \in ®X$. ®X je likálně kompaktní, tedy existuje $K \subset ®X$ kompaktní, $K \in ©U(x)$, $x \in \Int(K)$, $∞ \in \{∞\} \cup (®X\setminus K)$ (dvě disjunktní množiny).

            Jednoznačnost: v Hausdorffově prostoru jsou uzavřené množiny právě ty kompaktní, tedy nebyla jiná volba $\sigma$.
        \end{dukazin}
    \end{tvrzeni}

    \begin{poznamka}
        Jednobodová (tzn. Alexandrovova) kompaktifikace (pokud existuje) je nejmenší kompaktifikace mezi všemi kompaktifikacemi daného prostoru ®X.

        Lokálně kompaktní Hausdorffovy prostory jsou právě otevřené podmnožiny Hausdorffových kompaktů.
    \end{poznamka}

% 30. 11. 2020

    \begin{lemma}[Tichonovovo vnoření]
        Ať ®X je TP, $®Y_i, i \in I$, jsou TP. Ať $©F = \{f_i: ®X \rightarrow ®Y_i: i \in I\}$ je soubor spojitých zobrazení. Pokud ©F odděluje body, pak $f:=\triangle ©F: ®X \rightarrow \prod_{i \in I}®Y_i$ ($f(x) = (f_i(x))_{i \in I}$) je prosté. Pokud navíc ©F odděluje body a uzavřené množiny, pak $f$ je vnoření.

        \begin{dukazin}
            Ať $x, y \in ®X, x≠y$. ©F odděluje body, tedy existuje $i \in I: f_i(x) ≠ f_i(y)$. $f(x) ≠ f(y)$. Tedy $f$ je prosté.
            Chceme ukázat, že $f: ®X \rightarrow f(®X)$ je homeomorfismus. Tedy stačí, že je uzavřené. Ať $F\subseteq ®X$ je uzavřené a $x \in ®X \setminus F$. $\exists i \in I: f_i(x) \notin f_i(F) = \overline{\pi_i(f(F))} \supseteq \pi_i(\overline{f(F)})$. Tedy $f_i(x) \notin \pi_i(\overline{f(F)})$. Proto $\pi_i(f(x)) = f(x) \notin \overline f(x) \notin \overline{f(F)}$. Tedy f je uzavřené.
        \end{dukazin}
    \end{lemma}

    \begin{tvrzeni}[Tichonovova krychle]
        Každý Tichonovův prostor lze vnořit do nějaké Tichonovovy krychle, tj. $\[0, 1\]^I$, pro vhodnou množinu $I$.
        
        \begin{poznamkain}
            Za $I$ lze volit bázy.)
        \end{poznamkain}
        
        \begin{dukazin}
            Označme jako $I:=C(®X, \[0, 1\])$ a uvažme diagonální zobrazení $e:=\triangle \{f: f\in I\}: ®X \rightarrow \[0, 1\]^I$. $I$ odděluje body a odděluje body a uzavřené množiny, protože ®X je Tichonovův. Tedy $e$ je vnoření podle předchozího lemmatu.
        \end{dukazin}
    \end{tvrzeni}

    \begin{dusledek}
        Každý Tichonovův prostor má nějakou kompaktifikaci.
        \begin{dukazin}
            Podle předchozího tvrzení existuje vnoření $e$ do Tichonovovy krychle. $(e, \overline{e(®X)})$ je kompaktifikace ®X.
        \end{dukazin}
    \end{dusledek}

    \begin{definice}
            Kompaktifikace z důkazu předchozího důsledku (nebo kterákoliv s ní ekvivalentní) se nazývá Čechova-Stoneova (nebo beta-obal) a značí se $\beta®X$.
    \end{definice}

    \begin{veta}[Charakterizace beta obalu]
        Ať ®X je Tichonovův TP a $Y$ kompaktifikace ®X. Pak je ekvivalentní a) $Y$ je Čechova-Stoneova kompaktifikace ®X. b) Každou spojitou funkci $f: ®X \rightarrow \[0, 1\]$ lze spojitě rozšířit na $Y$. c) Každou spojitou funkci $f:®X \rightarrow ®Z$ do libovolného kompaktního Hausdorffova prostoru ®Z lze spojitě rozšířit na $Y$. d) $Y$ je největší kompaktifikace ®X.
        \begin{dukazin}
            $a) \implies b):$ $Y = \overline{e(X)}, e = \triangle\{f: f:®X\rightarrow \[0, 1\]\text{ spojitá}\}$. $e: ®X \rightarrow \[0, 1\]^I$. Ať $f: ®X \rightarrow \[0, 1\]$ je spojité $f \in I, \pi_f:\[0, 1\]^I\rightarrow \[0, 1\], \pi_f \circ e = f$, tedy $\pi_f$ rozšiřuje $f$, $\pi_f|Y$ je hledané spojité rozšíření.

            $b) \implies c)$ Podle lemmatu o Tichonovově vnoření\footnote{Z kompaktní Hausdorfův $\implies T_4 \implies T_{\pi}$} můžeme předpokládat, že $Z \subseteq [0, 1]^J$ pro nějakou množinu $J$. $\pi_j \circ f: ®X \rightarrow \[0, 1\]$ lze spojitě rozšířit na $g_j: Y \rightarrow \[0, 1\], j \in J$. $\triangle\{g_j: j \in J\}: Y \rightarrow \[0, 1\]^J$ je hledané spojité rozšíření toho zobrazení $f=\triangle_{j \in J}(\pi_j \circ f)$.

            $c) \implies d)$ Triviální z definice uspořádání na kompaktifikaci (Je-li Z nějaká kompaktifikace ®X, pak $id_{®X}:®X \rightarrow Z$ lze podle c) spojitě rozšířit na $Y$. Tedy $Y$ je větší kompaktifikace než $Z$.)

            $d) \implies a)$ $Y$ je největší, tedy je větší než Čechova-Stoneova. Zbývá ukázat, že Čechova-Stoneova kompaktifikace je větší než $Y$. To již víme z důkazu implikace $a \implies c$ volbou $f = id_{®X}$. Podle poznámky o ekvivalenci kompaktifikací jsou tyto ekvivalentní.
        \end{dukazin}
    \end{veta}
    
    \begin{poznamka}
            Je-li ®X Tichonovův prostor a $f: ®X \rightarrow ®R$ spojitá omezená, pak $f$ lze spojitě rozšířit na $\overline{f}: \beta®X \rightarrow ®R$. Banachovy prostory $e_∞$ a $C(\beta®N, ®R)$ lze stotožnit.
    \end{poznamka}

\section{Metrizovatelnost}
    \begin{poznamka}
        Je-li $(®X, \rho)$ metrický prostor, tak $\sigma:=\min\{\rho, 1\}$ je opět metrika na ®X, která generuje stejnou topologii jako $\rho$.
    \end{poznamka}

    \begin{tvrzeni}
        Jsou-li $®X_n$ metrizovatelné TP, pak $\prod_{n \in ®N}®X_n$ je metrizovatelný.

        \begin{dukazin}
                Ať $\rho_n$ je metrika na $®X_n$ kompatibilní s topologií $\tau_n$. BÚNO $\rho_n ≤ 1$. $®X = \prod ®X_n$. $\tau$ součinová topologie na ®X. Definujeme $\rho$ metriku na $®X: \rho(x, y):= \sum_{n \in ®N} 2^{-n}\rho_n(x_n, y_n)$. Ověříme, že topologie generovaná metrikou $\rho$ splývá s $\tau: \pi_n: ®X \rightarrow ®X_n$ je $2^n$-lipschitzovské, tedy spojité. Tedy topologie generovaná $\rho$ je větší než $\tau$. Ať $\epsilon > 0$ a $x \in ®X$, najdeme $n \in ®N: 2·2^{-n}<\epsilon$ a ať $\delta = 2^{-n}$. $B_{\rho_1}(x_1, \delta)\times … \times B_{\rho_n}(x_n, \delta) \times \prod_{i=n+1}^∞ ®X_i$ je prvek $\tau$, navíc je podmnožinou $B_{\rho}(x, \epsilon)$. Tedy topologie generovaná, metrikou $\rho$ je menší než $\tau$.
        \end{dukazin}

        \begin{poznamkain}
            Alternativně lze za $\rho$ brát
            $$ \sum_{n=1}^∞ \frac{\rho_n (x_n, y_n)}{2^n (1+\rho_n(x_n, y_n))}. $$ 
        \end{poznamkain}
    \end{tvrzeni}

    \begin{veta}[Urysohnova metrizační]
        Každý $T_3$ prostor se spočetnou bází je metrizovatelný.
        
        \begin{dukazin}
            Prostor se spočetnou bází je Lindelöfův. Každý regulární Lindelöfův prostor je normální. Tedy $®X$ je $T_4$, tedy i Tichonovův. Ať $©B$ je spočetná báze $®X$. Označme $©A = \{(U, V): \overline{U} \subseteq V, U, V \in ©B\}$. Pro $(U, V) \in ©A$ existuje spojitá funkce $f_{U, V}$ existuje spojitá funkce $f_{U, V}: ®X \rightarrow \[0, 1\]$, že $f_{U, V}|\overline{U} = 0, f_{U, V}|(®X\setminus V) = 1$ (lze z normality). $©F := \{f_{U, V}: (U, V) \in ©A\}$ odděluje body a odděluje body a uzavřené množiny. Podle lemmatu o Tichonovově vnoření je $e:= \triangle ©F: ®X \rightarrow \[0, 1\]^{©F}$ je vnoření. $\[0, 1\]^{©F}$ je metrizovatelný, protože ©F je spočetná. $e: ®X \rightarrow \[0,1\]^{©F}, e: ®X \rightarrow e(®X)$ je homeomorfismus. $e(®X)$ je metrizovatelný, tedy ®X je metrizovatelný.
        \end{dukazin}
    \end{veta}

    \begin{dusledek}
        Každý kompaktní Hausdorffův prostor se spočetnou bází je metrizovatelný.
        \begin{dukazin}
            Kompaktní Hausdorffův prostor je $T_4$, tedy je i $T_3$. Tedy podle Urysohnovy metrizační věty metrizovatelný.
        \end{dukazin}
    \end{dusledek}

    \begin{dusledek}
        Ať $f: ®X \rightarrow ®Y$ je spojitá a na, ®X kompaktní metrizovatelný a ®Y Hausdorffův. Pak ®Y je metrizovatelné.
        \begin{dukazin}
            Víme, že spojitý obraz kompaktu nezvýší jeho váhu. ®X kompaktní, netriviální $\implies$ má spočetnou bázi. Tedy ®Y má spočetnou bázi. Navíc ®Y je kompakt. Podle předchozího důsledku je ®Y metrizovatelný. 
        \end{dukazin}
    \end{dusledek}
    
    \begin{poznamka}
        Z důkazu Urysohnovy metrizační věty lze odvodit, že každý separabilní metrizovatelný prostor má metrizovatelnou kompaktifikaci a lze ho vnořit do Hilbertovy kostky $\[0, 1\]^{®N}$.
    \end{poznamka}

\end{document}
