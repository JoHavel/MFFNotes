\documentclass[12pt]{article}					% Začátek dokumentu
\usepackage{../../MFFStyle}					    % Import stylu



\begin{document}

\section{Organizační úvod}
    Přesun nebyl odhlasován.

    \begin{poznamka}[Literatura]
        \ 
        \begin{itemize}
            \item Engelking: General Topology (spíš taková příručka, hodně obtížná)
            \item Čech: Bodová topologie
            \item Kelley: General Topology
            \item Willard: General Topology
        \end{itemize}
        Doporučené jsou poslední dvě.
    \end{poznamka}

    \begin{poznamka}[Podmíny zakončení]
        Zkouška (ústní) + úkoly ze cvičení (a účast na cvičení)
    \end{poznamka}

\section{Úvod}

    \begin{poznamka}[Historie]
        \ 
        \begin{itemize}
            \item Euler: mosty ve městě Královec (7 mostů, Eulerovský tah)
            \item Listing (1847): pojem topologie (bez rigorózních definic)
            \item Poincaré (1895): Analysis Situs (Poincarého hypotéza )
            \item Fréchet (1906): definuje metrický prostor (až dodnes)
            \item Hausdorff (1914): tzv. Hausdorffův TP
            \item Kuratowski (1922): TP, jak jej známe dnes (formálně)
        \end{itemize}
    \end{poznamka}

    \begin{poznamka}[TOPOSYM]
        V Praze se každých 5 let koná významná konference topologů -- TOPOSYM.
    \end{poznamka}



\section{Základní pojmy}
    Topos = umístění (řečtina).

    \subsection{Topologický prostor, báze, subbáze, váha, charakter}

    \begin{definice}[Topologický prostor (TP)]
        Uspořádaná dvojice $(\X, \tau)$ se nazývá topologický prostor, pokud ®X je množina, $\tau \subseteq ©P(®X)$ a platí:\\
        (T1) $\O, ®X \in \tau$\\
        (T2) jsou-li $®U, ®V \in \tau$, pak $®U \cap ®V \in \tau$\\
        (T3) je-li $©U \in \tau$, pak $\bigcup ©U \in \tau$.
    \end{definice}

    \begin{definice}[Topologie]
        Systém $\tau$ se nazývá topologie na ®X. Prvky množiny ®X se nazývají body. Prvky $\tau$ se nazývají otevřené množiny.
    \end{definice}

    \begin{definice}[Okolí bodu]
        Množina $®V \subseteq ®X$ se nazývá okolí bodu $x$, pokud existuje $®U \in \tau$, že $x \in ®U \subseteq ®V$. Množina všech okolí bodu $x$ značíme $©U(x) = ©U_\tau(x)$.
    \end{definice}

    \begin{definice}[Báze a subbáze]
        Soubor množin $©B \subseteq \tau$ se nazývá báze topologie $\tau$, pokud pro každé $®U\in \tau$ existuje $©U \subseteq ®B: \bigcup ©U = ®U$. Soubor $©S \subseteq \tau$ se nazývá subbáze topologie $\tau$, pokud $\{\bigcap ©F: ©F \subseteq ©S \text{konečná}\}$ je báze topologie $\tau$.
    \end{definice}

    \begin{tvrzeni}[Charakterizace otevřené množiny pomocí okolí]
        Ať $(®X, \tau)$ je TP a $®U \in ®X$. Pak $®U \in \tau$, právě když $\forall x \in ®U \exists ®V\in ©U(x): ®V \subseteq ®U$
        \begin{dukazin}
            Důkaz ($\implies$) vidíme $®U = ®V$.

            Opačně víme $\forall x \in ®U \exists ®V_x\in©U(x): ®V_x \subseteq ®U$. $\exists ®W_x \in \tau: x\in ®W_x \subseteq ®U_x$. $®U = \bigcup_{x\in ®U}®W_x \in \tau$. Tedy $®U \in \tau$.
        \end{dukazin}
    \end{tvrzeni}

    \begin{priklad}
        Je-li $(®X, \rho)$ metrický prostor (MP), pak soubor všech $\rho$-otevřených množin tvoří topologii na množině ®X.
    \end{priklad}

    \begin{definice}[Metrizovatelný TP]
        TP $(®X, \tau)$ se nazývá metrizovatelný, pokud na množině ®X existuje metrika $\rho$ tak, že topologie odvozené z $(®X, \rho)$ splývá s topologií $\tau$.
    \end{definice}

    \begin{priklad}
        Je-li $(®X, \rho)$ MP, pak systém všech otevřených koulí tvoří bázi topologie $\tau_\rho$.
        \begin{prikladyin}
            Všechny otevřené intervaly tvoří bázi topologie na ®R.
            
            Systém $\{(-\infty, b), (a, \infty): a,b \in ®R\}$ je subbáze topologie na ®R.
        \end{prikladyin}
    \end{priklad}

    \begin{priklad}[Diskrétní a indiskrétní TP]
        Je-li ®X množina, pak $(®X, ©P(®X))$ je TP, nazývá se diskrétní TP (a vždy je metrizovatelný). Naopak $(®X, \{\O, ®X\})$ se nazývá indiskrétní TP. (Pokud $|®X| ≥ 2$, pak indiskrétní TP není metrizovatelný.)
    \end{priklad}

    \begin{tvrzeni}[Vlastnosti báze]
        Je-li $(®X, \tau)$ TP a ©B jeho báze, pak\\
        (B1) $\forall ®U, ®V \in ©B \forall x \in ®U \cap ®V \exists ®W \in ®B: x \in ®W \subseteq ®U \cap ®V$,\\
        (B2) $\bigcup ©B = ®X$.

        Je-li ®X libovolná množina a $©B \subseteq ®P(®X)$ splňuje podmínky (B1), (B2), pak na ®X existuje jediná topologie, jejíž báze je $®B$.

        \begin{dukazin}
            První část je snadná (průnik 2 množin báze je otevřený, tj. prvkem topologie, tedy se dá zapsat jako sjednocení podmnožiny báze).

            Druhá část: Mějme tedy ®X a ©B z věty splňující obě podmínky. Definujme $\tau := \{\bigcup ©U: ©U \subseteq ©B\}$. $\tau$ je topologie na ®X (ověříme, že $\tau$ splňuje podmínky topologie).

            Zároveň volba $\tau$ je jediná množná, jelikož každý její prvek se musí dát vyjádřit jako sjednocení báze a opačně.
        \end{dukazin}

        \begin{dusledekin}
            Je-li ®X množina, $©S \subseteq ©P(®X)$ a $\bigcup©S=®X$, pak $©S$ je subbáze jednoznačně určené topologie na ®X.
            \begin{dukazin}
                    $ ©B = \{\cap ©F: ©F \subseteq ©S \text{konečná}\} $ splňuje podmínky (B1) a (B2) předchozího tvrzení (B2 \ definice ©S, B1 protože $®U, ®V \in ©B, ®U = \bigcup ©F_1, ®V = \bigcap ©F_2, ©F_1, ©F_2 \subseteq ©S \text{konečné}$. $ ®U \cap ®V = \bigcap (©F_1 \cup ©F_2) \in ©B $. (Dokonce celý průnik je prvkem ©B, nejenom pro každý prvek existuje množina, která ho obsahuje, je podmnožinou průniku a je v ©B).
            \end{dukazin}
        \end{dusledekin}
    \end{tvrzeni}

    \begin{tvrzeni}[Vlastnosti systému všech okolí]
        Je-li $(®X, \tau)$ TP, pak soubory všech okolí $©U_\tau(x), x \in ®X$ splňují\\
        (U1) $ \forall x \in ®X: ©U(x) ≠ \O, x \in \bigcap©U(x)$,\\
        (U2) $\forall ®U \in ©U(x) \forall ®V: ®U \subseteq ®V \subseteq ®X \implies ®V \in ©U(x)$,\\
        (U3) $\forall ®U, ®V \in ©U(x): ®U \cap ®V \in ©U(x)$,\\
        (U4) $\forall ®U \in ©U(x) \exists ®V \in ©U(x) \forall y \in ®V: ®U \in ©U(y)$

        Je-li ®X množina a systémy množin $©U(x) \subseteq ©P(®X), x \in ®X$ splňující podmínky (U1-4), pak na množině ®X existuje jediná topologie $\tau$, že $©U(x) = ©U_\tau(x), x \in ®X$.
        \begin{dukazin}
            První část snadná. (Domácí cvičení.)

            Položme $\tau = \{®U \in ©P(®X): \forall x \in ®U, ®U\in©U(x)\}$. $\tau$ je topologie na $®X$. Z (U1) a (U2) vyplyne (T1). Atd…
        
        \end{dukazin}
    \end{tvrzeni}
    
    \begin{definice}[Báze okolí]
        Ať $(®X, \tau)$ je TP. Systém množin $©B(x) \subseteq ©P(®X)$ se nazývá báze okolí v bodě $x$, pokud $©B(x) \subseteq ©U_\tau(x)$ a pro každé $®V \in ©U_\tau(x)$ existuje $®U \in ©B(x)$, že $®U \in V$???. Indexovaný soubor $\{©B(x): x \in ®X\}$ se nazývá báze okolí prostoru $®X$, pokud $\forall x \in ®X: ©B(x)$ je báze okolí v bodě $x$.
    \end{definice}

    \begin{tvrzeni}[Vlastnosti báze okolí]
        Je-li $(®X, \tau)$ TP a $\{©B(x): x \in ®X\}$ báze okolí, pak\\
        (O1) $©B(x) ≠ \O, x \in \bigcap ©B(x), x \in ®X$,\\
        (O2) $\forall ®U, ®V \in ©B(x) \exists ®W \in ©B(x): ®W \subseteq ®U \cap ®V$,\\
        (O3) $\forall ®U \in ©B(x) \exists ©B(x) \forall y \in ®V \exists ®W \in ©B(y): ®W \subseteq ®U$.

        Je-li ®X množina a $©B(x)\subseteq ©P(®X), x \in ®X$ soubory splňující (O1), (O2), (O3), pak na množině ®X existuje jediná topologie, jejíž báze okolí je $\{©B(x): x \in ®X\}$.

        \begin{dukazin}
            První část je snadná. 

            Položme $©U(x) = \{®U \in ©P(x): \exists ®B \in ©B(x): ®B \subseteq ®U\}, x \in ®X$. Ověříme, že splňuje (U1-4). (U1) z (O1). (U2) z definice $©U$. (U3) z (O2), (U4) z (O3).
        \end{dukazin}
    \end{tvrzeni}

    \begin{definice}[Váha prostoru]
        Ať $(®X, \tau)$ je TP. Pak váha prostoru $(®X, \tau)$ je nejmenší mohutnost báze prostoru $(®X, \tau)$. Značíme ji $\weight(®X) = \weight(®X, \tau)$

        Charakter v bodě $x$ je nejmenší mohutnost báze okolí bodu $x$. Značíme ho $\chi(x, ®X)$.

        Charakter prostoru ®X je $\sup\{\chi(x, ®X): x \in ®X\}$.

        \begin{prikladyin}
            $\weight(®R) = \omega$ (®R má spočetnou bázi).

            $\weight(®X, ©P(®X)) = |®X|$ ($\{\{x\}: x \in ®X\}$ je báze $(®X, ©P(®X))$)

            $ \weight(®X, \{\O,\{®X\}\}) = 1 $
        \end{prikladyin}

        \begin{prikladyin}
            Je-li $(®X, \tau)$ metrizovatelný, pak $\chi(x, ®X) ≤ \omega$
        \end{prikladyin}
    \end{definice}

    \begin{tvrzeni}
            Ať $(®X, \tau)$ je TP a $x \in ®X$. Pak $\chi(x, ®X)≤\weight(®X)$
        \begin{dukazin}
            Ať ©B je báze $(®X, \tau)$, že $|©B| = \weight(®X)$. Položme $©B(x):=\{®U \in ©B: x \in ®U\}$. $©B(x)$ je báze okolí v bodě $x$.

            $|©B(x)|≤|©B|$, protože $©B(x) \subseteq ©B$. $\chi(x, ®X)≤|©B(x)|≤|©B|=\weight(®X)$.
        \end{dukazin}
    \end{tvrzeni}

    \subsection{Vnitřek, Uzávěr, hranice}
        \begin{definice}[Uzavřená množina]
            Ať $(®X, \tau)$ je TP. Množina $®F \subseteq ®X$ se nazývá uzavřená, pokud její doplněk je otevřená množina (neboli $\x \setminus ®F \in \tau$).
        \end{definice}

        \begin{definice}[Obojetná množina (clopen set)]
            Množina se nazývá obojetná, pokud je uzavřená a otevřená zároveň.
        \end{definice}

        \begin{definice}[Uzávěr]
            Je-li $®A \subseteq ®X$, pak uzávěr ®A je $\cl(®A) = \overline{®A} = \bigcap\{®F \subseteq ®X, ®A \subseteq ®F, ®F \text{je uzavřená}\}$.
        \end{definice}

        \begin{definice}[Vnitřek množiny]
            Vnitřek množiny $®A$ je $\Int®A = ®A^0 = \bigcup\{®U \in \tau: ®U \subseteq ®A\}$.
        \end{definice}

        \begin{definice}[Hranice množiny]
                Hranice množiny ®A je $\delta ®A = \overline{®A} \cap \overline{®X \setminus ®A}$
        \end{definice}

        \begin{tvrzeni}[Vztah vnitřku a uzávěru]
                Ať $(®X, \tau)$ je TP, $®A\subseteq®X$, pak $®X \setminus \overline{®A} = \Int(®X \setminus ®A)$ a $®X \setminus \Int®A = \overline{®X \setminus ®A}$.

            \begin{dukazin}
                $® \setminus \overline{®A}$ je otevřená, navíc $®X \setminus \overline{®A} \subseteq ®X \setminus ®A$. Tedy $®X \setminus \overline{®A} \subseteq \Int(®X\setminus ®A)$. $\Int(®X \setminus ®A)®X \setminus ®A$, přechodem k doplňku $®A \subseteq ®X \setminus \Int(®X \setminus ®A)$. Tedy $\overline{®A} \subseteq ®X \setminus \Int(®X)???$. Přechodem k doplňku: $\Int(®X \setminus ®A) \subseteq ®X \setminus \overline{®A}$.

                Druhou část můžeme dokázat přechodem k doplňku a převedením na první část.
            \end{dukazin}
        \end{tvrzeni}

        \begin{tvrzeni}[Charakterizace uzávěru]
                Buď $(®X, \tau)$ TP, $x \in ®X, ®A \subseteq ®X$ a $©B(x)$ báze okolí v bodě $x$. Pak následující podmínky jsou ekvivalentní\\
            1) $x \in \overline{®A}$,\\
            2) $\forall ®U \in ©U(x): ®U\cap ®A ≠ \O$,\\
            3) $\forall ®U \in ©B(x): ®U\cap ®A ≠ \O$.

            \begin{dukazin}
                1) -> 2) sporem: Kdyby pro nějaké $®U \in ©U(x): ®U \cap ®A = \O$, pak existuje ®V otevřené: $x \in ®V \subseteq ®U$. $®V \cap®A = \O$. $®X \setminus ®V$ je uzavřená a $®A \subseteq ®X \setminus ®V$. Pak $x \in \overline{®A} \subseteq ®X \setminus ®V$, neobsahuje $x$. $\lightning$.

                2) -> 3) triviální

                3) -> 1) sporem: $x \notin \overline{®A}$ pak $x \in ®X\setminus \overline{®A}$. Pak existuje $®U \in ©B(x): x \in ®U \subseteq ®X \setminus \overline{®A}$. Pak ???
            \end{dukazin}
        \end{tvrzeni}

        Jako speciální důsledky dostáváme následující. Je-li ®U otevřená, pak $®U \cap ®A = \O$ právě když $®U \cap \overline{®A} = \O$. Jsou-li ®U, ®V otevřené disjunktní množiny, pak $®U \cap \overline{®V} = \O = \overline{®U} \cap ®V$.

        \begin{tvrzeni}[Vlastnosti uzávěru]
            Pro množiny ®A, ®B v TP $(®X, \tau)$ platí\\
            (C1) $\overline{\O} = \O$,\\
            (C2) $®A \subseteq \overline{®A}$,\\
            (C3) $\overline{\overline{®A}} = \overline{®A}$
            (C4) $\overline{®A \cup ®B} = \overline{®A} \cup \overline{®B}$,\\
            (C5) $\overline{®A \cap ®B} \subseteq \overline{®A} \cap \overline{®B}$.

            \begin{dukazin}
                První dvě jsou jednoduché, 3. plyne z uzavřenosti uzávěru. 4. dokážeme inkluzemi. Shrnutím dostaneme (C5).
            \end{dukazin}
        \end{tvrzeni}

        \begin{priklad}
                Zobrazení z podmnožin do podmnožin, které splňuje podmínky (C1-C4) jednoznačně určuje topologii.
        \end{priklad}

        \begin{tvrzeni}[Vlastnosti vnitřku]
            Obdobně jako vlastnosti uzávěru.
        \end{tvrzeni}

        \begin{tvrzeni}[Charakterizace hranice]
            Ať $®A \subseteq ®X$ a $x \in ®X$. Pak $x \in \delta ®A$, právě když každé okolí bodu $x$ protíná jak ®A, tak $®X \setminus ®A$.
            \begin{dukazin}
                Plyne okamžitě z definice hranice $\delta ®A = \overline{®A} \cap \overline{®X \setminus ®A}$ a charakterizace uzávěru.
            \end{dukazin}
        \end{tvrzeni}

        \begin{tvrzeni}[Vlastnosti hranice]
            12. bodů viz skripta. Stejně tak důkaz.
        \end{tvrzeni}
    
    \subsection{Husté a řídké množiny, hromadné a izolované body}
        \begin{definice}[Hustá a řídká množina, hustota, separabilní prostor]
                Ať ®X je TP. Množina $®A \subseteq ®X$ se nazývá hustá (v ®X), pokud $\overline{®A} = ®X$. ®A se nazývá řídká, pokud $®X \setminus \overline{®A}$ je hustá.

                Hustota prostoru ®X je nejmenší mohutnost husté podmnožiny, značí se $\d(®X)$ (d…density). Prostor se spočetnou hustotou se nazývá separabilní.
        \end{definice}

        \begin{tvrzeni}[Charakterizace hustých a řídkých množin]
            Ať ®X je TP. Množina $®A \subseteq ®X$ je hustá v ®X, právě když $\forall ®U$ otevřená neprázdná v ®X protíná ®A. Množina ®A je řídká (v ®X), právě když $\forall ®V$ otevřená neprázdná $\exists ®U$ otevřená neprázdná, že $®U \subseteq ®V \setminus ®A$, což je právě když $\Int(\overline{®A}) = \O$.
            \begin{dukazin}
                Označme $\tau* = \tau \setminus \O$. Z charakterizace uzávěru: $\overline{®A} = ®X \Leftrightarrow \forall x \in ®X \forall ®V \in ©U(x): ®V\cap ®A ≠ \O$. $A$ je řídká $\Leftrightarrow$ $®X\setminus \overline{®A}$ je hustá $\Leftrightarrow \forall ®U \in \tau*: ®U \cap (®X \setminus \overline{®A}) ≠ \O \Leftrightarrow \forall ®U \in \tau*: ®U \setminus \overline{®A} ≠ \O$.

                První část dostaneme ekvivalencí z předchozího: $ \forall ®U \in \tau* \exists ®V \in \tau*: ®V \subseteq ®U \setminus \overline{®A}$.

                Druhá část pak plyne z $\Int \overline{A} = \O$
            \end{dukazin}
        \end{tvrzeni}
        
        \begin{tvrzeni}[Vztah váhy a hustoty]
            Ať ®X je TP. Pak $\d(®X) ≤ \weight(®X)$. Speciálně každý prostor se spočetnou bází je separabilní.
            \begin{dukazin}
                Ať ©B je báze TP ®X. (BÚNO $\O \notin ©B$). $forall ®B \in ©B$ fixujeme $x_B \in B, ®D:=\{x_B : B \in ©B\}$. Zřejmě $|®D|≤|©B|$, ®D je hustá v ®X. (Když tedy volíme $©B$ nejmenší, získáme výraz.)
            \end{dukazin}
        \end{tvrzeni}

        \begin{poznamka}
            Pro metrizovatelný TP ®X platí $\d(®X) = \weight(®X)$.
        \end{poznamka}

        \begin{definice}[Izolovaný a hromadný bod]
            Ať ®X je TP. Bod $x \in ®A \subseteq ®X$ se nazývá izolovaným bodem množiny $A$, pokud existuje otevřená množina $®U \subseteq ®X$, že $®U\cap ®A = \{x\}$. Bod $x$ se nazývá hromadným bodem množiny ®A, pokud každé okolí bodu $x$ protíná množinu $®A\subseteq \{x\}$
            \begin{prikladyin}
                V diskrétním prostoru jsou všechny body izolované. Naopak je-li $®X = ®R$ a $®A = ®Q$, pak každý bod ®X je hromadným bodem množiny ®A. Žádný bod z ®A není izolovaným bodem ®A.
            \end{prikladyin}

        \end{definice}

        \begin{definice}[Derivace množiny]
            Množina hromadných bodů množiny ®A se značí $®A'$. Někdy se nazývá derivace ®A.
        \end{definice}

        \begin{tvrzeni}[Vlastnosti derivace]
            $\overline{®A} = ®A \cup ®A'$, $(®A \cup ®B)' = ®A' \cup ®B'$
            \begin{dukazin}
                Domácí cvičení (je jednoduchý).
            \end{dukazin}
        \end{tvrzeni}

    \subsection{Spojitá zobrazení}
        \begin{definice}[Spojité zobrazení, homeomorfizmus a spojitost v bodě]
            Ať $(®X, \tau)$ a $(®Y, \sigma)$ jsou TP. Ať $f: ®X \rightarrow ®Y$. Zobrazení $f$ se nazývá spojité, pokud $\forall ®U \in \sigma: f^{-1}(®U) \in \tau$.

            $f$ se nazývá homeomorfizmus, pokud $f$ je bijekce a $f$ i $f^{-1}$ jsou spojitá.

            $f$ je spojité v bodě $x$, pokud $\forall ®V \in ©U_\sigma(f(x)) \exists ®U \in ©U_\tau(x): f(U) \subseteq ®V$.
        \end{definice}

        \begin{priklady}
            ®R, $(0, 1)$ jsou homeomorfní (ale nejsou izometrické)
        \end{priklady}

        \begin{poznamka}
            Vlastnosti, TP, které se zachovávají homeomorfizmem se nazývají topologické vlastnosti.

            (Úplnost není topologický pojem.)
        \end{poznamka}

        \begin{prikladyin}
            Zobrazení z diskrétního prostoru je vždy spojité.

            Zobrazení do indiskrétního prostoru je taktéž vždy spojité.
        \end{prikladyin}

        \begin{tvrzeni}[Charakterizace spojitých zobrazení]
            Ať $(®X, \tau)$, $(®Y, \sigma)$ jsou TP, $f: ®X \rightarrow ®Y$ zobrazení. Pak následující je ekvivalentní:\\
            1) $f$ je spojité\\
            2) vzory množin z nějaké subbáze jsou otevřené\\
            3) vzory množin z nějaké báze jsou otevřené\\
            4) $f$ je spojité v každém bodě\\
            5) vzory uzavřených množin jsou uzavřené\\
            6) $\forall ®A \subseteq ®X: f(\overline{®A}) \subseteq \overline{f(®A)}$\\
            7) $\forall ®B \subseteq ®Y: \overline{f^{-1}(®B)} \subseteq f^{-1}(\overline{®B})$\\
            8) $\forall ®B \subseteq ®Y: f^{-1}(\Int ®B) \subseteq \Int\(f^{-1}(®B)\)$

            \begin{dukazin}
                1->2 Triviální (z definice).

                2->3 Ať $©B$ je nějaká báze. Dle 2 pro nějakou subbázi ©S toho $(®Y, \sigma)$ platí, že $f^{-1}(®S)$ je otevřená pro $®S \in ©S$. Ať $®B \in ©B$. ®B lze vyjádřit jako sjednocení konečných průniků prvků ©S. (Vzor průniku je průnik vzorů, vzor sjednocení je sjednocení vzorů.) $f^{-1}(®B)$ je sjednocením konečných průniků prvků tvaru $f^{-1}(®S), ®S \in ©S$. Tedy $f^{-1}(®B)$ je otevřená.

                3->4 Ať $x \in ®X$, ®V okolí bodu $f(x)$. ©B báze z 3. podmínky. $\exists ®B \in ©B$, že $f(x) \in ©B \subseteq ®V$. $®U = f^{-1}(®B)$ otevřená, $x \in ®U f(®U) \subseteq ®B \subseteq ®V$.

                4->5 Ať $®F \subseteq ®Y$ je uzavřená. Ať $x \in \overline{f^{-1}(F)}$. Chceme, že $x \in f^{-1}(®F)$ (tj. že $f(x) \in ®F$). Z 4 pro každé okolí ®V bodu $f(x)$ existuje ®U okolí $x$, že $f(x) \subseteq V$. Z definice uzávěru platí, že každé takové ®U protíná $f^{-1}(®F)$, tedy $f(®U) \cap ®F ≠ \O$, tedy $®V \cap ®F ≠ \O$. Tedy podle charakterizace uzávěru $f(x) \in \overline{®F} = ®F$.

                5->6 $f^{-1}(\overline{f(®A)})$ je uzavřená dle 5 a obsahuje ®A, tedy obsahuje i $\overline{®A}$. Pak $f(\overline{®A}) \subseteq f\(f^{-1}(\overline{f(®A)})\)\subseteq \overline{f(®A)}$.

                6->7 Ať $®B \subseteq Y$, $A := f^{-1}(®B)$. Dle 6 $f(\overline{f^{-1}(®B)}) \subseteq \overline{f(f^{-1}(®B))}\subseteq \overline{®B}$. $\overline{f^{-1}(®B)} \subseteq f^{-1}(\overline{®B})$ (aplikováním vzoru? na předchozí).

                7->8 Vztah vnitřku a uzávěru. $f^{-1}(\Int ®B) = f^{-1}(®Y \setminus \overline{®Y \setminus ®B}) = ®X \setminus f^{-1}(\overline{®Y \setminus ®B}) \stackrel{\text{dle 7}}{\subseteq} ®X \setminus \overline{®Y \setminus ®B} = ®X \setminus \overline{®X \setminus f^{-1}(®B)} = ®X \setminus \(®X \setminus \Int f^{-1}(®B)\) = \Int f^{-1}(®B)$.

                8->1 Je-li $®V \subseteq ®Y$ otevřená, pak ze 7: $f^{-1}(®V) \subseteq \Int(f^{-1}(®V))$. Triviálně $\Int f^{-1}(®V) \subseteq f^{-1}(®U)$. Tedy $f^{-1}(®V) = \Int f^{-1}(®V)$, tedy $f^{-1}(®V)$ je otevřená.
            \end{dukazin}
        \end{tvrzeni}

        \begin{tvrzeni}[Skládání spojitých zobrazení]
            Ať $®X, ®Y, ®Z$ jsou TP, $f: ®X \rightarrow ®Y, g: ®Y \rightarrow ®Z$ zobrazení. Jsou li $f, g$ spojitá, pak $g \circ f: ®X \rightarrow ®Z$ je spojité.

            Pokud $f$ je spojité v bodě $x$ a $g$ spojité v $f(x)$, pak $g \circ f$ je spojité v $x$.

            \begin{dukazin}
                $(g\circ f)^{-1}(®V) = f^{-1}(g^{-1}(®V))$

                Je-li ®V okolí $gf(x)$, pak $g^{-1}(®V)$
            \end{dukazin}
        \end{tvrzeni}

% 19. 10. 2020

    \subsection{Oddělovací axiomy}
        \begin{definice}
            TP ®X se nazývá:
            \begin{itemize}
                \item $T_0$, pokud $\forall x, y \in ®X \exists ®U \text{otevřená} : |U \cap \{x, y\}| = 1$.
                \item $T_1$, pokud $\forall x, y \in ®X, x≠y \exists ®U \text{otevřená} : x\in ®U, y\notin ®U$.
                \item $T_2$ (Hausdorffův), pokud $\forall x, y \in ®X \exists ®U, V \text{otevřené disjunktní} : x \in ®U, y \in ®V $.
                \item regulární, pokud $\forall ®F \subseteq ®X$ uzavřenou $\forall \in ®X \setminus ®F \exists ®U, ®V$ otevřené disjunktní: $x \in ®U, ®F \subseteq ®V$.
                \item normální, pokud $\forall ®E, ®F$ uzavřené disjunktní $\exists ®U, ®V$ otevřené disjunktní: $®E \subseteq ®U, ®F \subseteq ®V$.
                \item úplně regulární, pokud $\forall ®F \subseteq ®X$ uzavřenou $\forall x \in ®X \setminus ®F \exists f: ®X \rightarrow \[0, 1\]$ spojitá, že $f(x) = 0, f(®F)\subseteq \{1\}$.
                \item $T_3$, pokud je regulární a $T_1$.
                \item $T_{3\frac{1}{2}}$ nebo $T_\pi$ (Tichonovův), pokud je úplně regulární a $T_1$.
                \item $T_4$, pokud je normální a $T_1$.
            \end{itemize}

            \begin{poznamka}
                $$ \text{normální} \implies \text{úplně regulární} \overset{\text{rozpůlení intervalu $\[0, 1\]$}}{\implies} \text{regulární} $$ 
                $$ T_4 \implies T_\pi \implies T_3 \implies T_2 \implies T_1 \implies T_0 $$ 
                (Platí pouze tímto směrem, ne opačně!)
                $$ T_0 \not\implies T_1: \(\{0, 1\}, \{\O, \{0, 1\}, \{0\}\}\) … (\text{Sierpinského TP}) $$
                $$ T_1 \not\implies T_2: \(®N, \{\O\} \cup \{®N \setminus K: K \text{je konečná}\}\) (\text{Topologie kokonečných (doplněk konečných) množin}) $$
            \end{poznamka}
        \end{definice}

        \begin{tvrzeni}[Metrizovatelné prostory jsou $T_4$]
            Je-li ®X metrizovatelný prostor a $®E, ®F \subseteq ®X$ uzavřené disjunktní množiny, pak existuje spojitá funkce $f: ®X \rightarrow \[0, 1\]$, že $f(®E) \subseteq \{0\}, f(®F) \subseteq \{1\}$.

            \begin{dukazin}
                    ®X je metrizovatelný, tedy existuje metrika $\rho$ kompatibilní s topologií na ®X. Položme $f(x) = \frac{\rho(x, ®E)}{\rho(x, ®E) + \rho(x, ®F)}, x \in ®X$. $f$ je dobře definovaná a jistě spojitá. $f(x) = 0, x \in ®E$, $f(x) = 1, x \in ®F$.
            \end{dukazin}
        \end{tvrzeni}

        \begin{lemma}
            Ať ®X je TP. Pak
            \begin{itemize}
                \item[a)] ®X je $T_1 \Leftrightarrow$ každá jednoprvková množina je uzavřená $\Leftrightarrow$ každá konečná množina je uzavřená.
                \item[b)] ®X je $T_2$ $\implies \forall x, y \in ®X, x≠y \exists ®U \in ©U(x): y\notin \overline{®U}$.
                \item[c)] ®X je regulární $\Leftrightarrow \forall x \in ®X \forall ®U \in ©U(x) \exists ®V \in ©U(x): \overline{®V} \subseteq ®U$.
                \item ®X je normální $\Leftrightarrow \forall ®V \subseteq ®X \text{otevřenou} \forall ®E \in ®V \text{uzavřenou} \exists U \subseteq ®X \text{otevřená}: ®E \subseteq ®U \subseteq \overline{®U} \subseteq V$.
            \end{itemize}

            \begin{dukazin}
                Jednoduché.
            \end{dukazin}
        \end{lemma}

        \begin{veta}[Urysohnovo lemma]
            TP ®X je normální $\Leftrightarrow$ pro každé dvě disjunktní uzavřené ®E, ®F existuje spojitá funkce $f:®X \rightarrow [0, 1]$, že $f(®E) \subseteq \{0\}, f(®F) \subseteq \{1\}$
            
            \begin{dukazin}
                Implikace zprava doleva je snadná -- uvažujeme $\{x \in ®X: f(x) < \frac{1}{2}\}$ a $\{x \in ®X: f(x)>\frac{1}{2}\}$.

                $\implies$ Označme $D:=®Q \cap \[0, 1\]$, $D= \{r_n: n \in ®N \cup \{0\}\}$, $r_0 = 0, r_1 = 1$ ($r_n$) prostá posloupnost. Indukcí najdeme otevřené množiny $®V_q: q \in D$, že pro $p, q \in D, p<q \implies ®V_p \subseteq ®V_q$ a navíc $®E \subseteq ®V_0, ®V_1 \subseteq ®X \setminus ®F$.

                Z normality najdeme otevřenou množinu ®U, že $®E \subseteq ®U \subseteq \overline{®U}\subseteq ®X \setminus \overline{r}$. Položíme $®V_0 = ®U$, $®V_1 = ®X \setminus ®F$.

                Nyní předpokládejme, že $®V_{r_0}, ®V_{r_1}, …, ®V_{r_n}, n≥1$. Už známe a platí, že pro $p, q \in \{r_0, …, r_n\}: p<q \implies \overline{®V_p} \subseteq ®V_q$. Chceme najít $®V_{r_{n+1}}$. Ať $i,j≤n$ jsou taková, že $r_i = \max\{r_k: r_k < r_{n+1}\}$ a $r_j = \min\{r_k: r_k > r_{n+1}\}$. TODO! Z normality existuje otevřená $®V_{r_{n+1}}$, že $\overline{®V_{r_i}} \subseteq ®V_{r_{n+1}} \subseteq \overline{®V_{r_{n+1}}} \subseteq ®V_{r_j}$.

                Položme $f(x) = 1, x \in ®X \setminus ®V_1| f(x) = \inf{r\in D: x \in ®V_r}, x \in ®V_1$. $f:®X \rightarrow \[0, 1\]$. Nyní stačí ověřit spojitost: vzory subbázových (nějaké subbáze) podmnožin jsou otevřené. Zvolím si subbázi $\{\[0, b\), \(a,1\], a, b \in \(0, 1\)\}$. $f^{-1}(\[0, b\)) = \{x \in ®X: f(x)<b\} = \{x \in ®X: \exists r< b: x \in ®V_r\} = \bigcap_{r<b}®V_r…\text{otevřené}$. $f^{-1}(\(a, 1\]) = \{x \in ®X: f(x) > a\} = \{x \in ®X: \exists r>a: x\notin ®V_r\} = \{x \in ®X: \exists s>a: x \notin \overline{®V_s}\} = \bigcup_{s>a}®X \setminus \overline{®V_s}…\text{otevřené}$
            \end{dukazin}
        \end{veta}

        \begin{poznamka}[$T_4 \implies T_{3.5}$, normalita $\implies$ úplná regularita]
        
        \end{poznamka}
    

    \subsection{Konvergence v topologických prostorech}
        \begin{definice}[Usměrněné množiny]
            Dvojice $(®I, ≤)$ se nazývá usměrněná množina, pokud ®I je množina a $≤$ je binární relace na ®I, která je reflexivní, tranzitivní a pro $i, j \in ®I$, pak existuje $k\in ®I$, že $i≤k, j≤k$.

            \begin{prikladyin}
                $(®N, ≤)$
            \end{prikladyin}
        \end{definice}

        \begin{definice}[Net]
            Net v TP ®X je libovolné zobrazení z usměrněné množiny do ®X.
        \end{definice}

        \begin{definice}[Konvergence netu]
            Řekneme, že net $(x_i)_{i \in ®I}$ konverguje k bodu $x$, pokud $\forall ®U \in ©U(x) \exists i_0 \in ®I \forall i \in ®I, i≥ i_0: x_i \in ®U$. Pokud existuje právě jeden, značíme $x= \lim_{i\in®I}x_i$.

            Bod $x$ se nazývá hromadným bodem netu $(x_i)_{i \in ®I}$, pokud $\forall ®U \in ©U(x) \forall i \in ®I \exists j≥i: x_j \in ®U$.
        \end{definice}

        \begin{tvrzeni}[Jednoznačnost limity netu]
            Prostor ®X je Hausdorffův $\Leftrightarrow$ každý net má nejvýše jednu limitu.
        \end{tvrzeni}






\end{document}
