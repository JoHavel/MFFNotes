\documentclass[12pt]{article}					% Začátek dokumentu
\usepackage{../../MFFStyle}					    % Import stylu



\begin{document}

% 16. 02. 2023

\begin{poznamka}
	Credit for giving 'small lecture'. Oral exam.
\end{poznamka}

\section{Meromorphic functions}
\begin{definice}
	We say that a function $f$ is holomorphic in a set $F \subset ®C$ if there is an open $G \supseteq F$ such that $f$ is holomorphic on $G$.

	In particular, $f$ is holomorphic at $z_0 \in ®C$ if $f$ is holomorphic in some neighbour (= $U(z_0) = U(z_0, \epsilon)$) of $z_0$.
\end{definice}

\begin{definice}
	Function $f$ has at $∞$ a removable singularity, if $f\(\frac{1}{z}\)$ has a removable singularity at 0. Similarly pole and essential singularity.

	Function $f$ is holomorphic at $∞$ if $f\(\frac{1}{z}\)$ is holomorphic at $0$.

	Let $G \subset ®S$ be open. Then $f$ is holomorphic on $G$ if $f$ is holomorphic at any $z_0$. Denote $©H(G) := \{f: G \rightarrow ®C | f \text{ holomorphic}\}$.

	\begin{prikladyin}
		From Liouville theorem $®H(®S) =$ constant functions. So $®H(G)$ is interesting only for $G \subsetneq ®S$, so WLOG $G \subset ®C$.
	\end{prikladyin}
\end{definice}

\begin{definice}[Meromorphic function]
	Let $G \subset ®S$ be open. Then a function $f$ on $G$ is called meromorphic if at any $z_0 \in G$ the function $f$ is either holomorphic at $z_0$ or has a pole at $z_0$.

	Denote $©M(G)$ the set of meromorphic functions on $G$.
\end{definice}

\begin{dusledek}
	\ 

	\begin{itemize}
		\item $©H(G) \subset ©M(G)$.
		\item Denote $P_f := \{z_0 \in G | f \text{ has a pole at } z_0\}$. Then $P_f$ has no limit points in $G$.
		\item If $f = ∞$ on $P_f$, then $f: G \rightarrow ®S$ is continuous. (We always assume, that $f \in ©H(G)$ has this property.)
	\end{itemize}
\end{dusledek}

\begin{priklady}
	$$ \frac{\pi}{\sin(\pi z)} \in ©M(®C), \qquad e^{\frac{1}{z}} \notin ©M(®C), \qquad \Gamma \in ©M(®C), \qquad \zeta \in ©M(®C). $$

	$©M(®S) =$ rational functions. (One inclusion is clear, second: Let $f \in ©M(®S)$, then because $®S$ is compact it holds that $P_f$ is finite (has no limit point), $P_f \cap ®C = \{z_1, …, z_n\}$, so from theorem from last semester there exists $h \in ©H(®C)$ such that $f(z) = h(z) + \sum_{j=1}^n p_j \(\frac{1}{z - z_j}\)$ for some polynomials $p_j$. $f$ has removable singularity or pole at infinity and $p_j$ and $\frac{1}{z - z_j}$ have removable singularity there, so $h(z)$ is polynomial, otherwise $h(z)$ has infinity Taylor polynom and $h\(\frac{1}{z}\)$ has essential singularity at $0$.)

	So $©M(G)$ is interesting for $G \subsetneq ®S$, WLOG $G \subset ®C$.

	If $G \subset ®C$ is domain, $f, g \in ®H(G)$ and $g ≡ 0$, then $f / g \in ©M(G)$. The inverse is also true (we will prove it) (but not for $G = ®S$).
\end{priklady}

\begin{lemma}
	Let $®G \subset ®C$ be open. Then there are compacts $K_n$, $n \in ®N$, in $G$ such that $G = \bigcup_{n=1}^∞ K_n$, $K_n \subset \Int (K_{n+1})$ and for any compact $K$ in $G$, $\exists n \in ®N: K \in K_n$.

	\begin{dukazin}
		Set $K_n := \{z \in G | \dist(z, ®C \setminus G) ≥ \frac{1}{n}\} \cap U(0, n)$.
	\end{dukazin}
\end{lemma}

\begin{tvrzeni}
	Let $G \subset ®S$ be open and $M \subset G$ has no limit point in $G$. Then

	\begin{itemize}
		\item $G \setminus M$ is open;
		\item if $K$ is a compact in $G$, then $K \cap M$ is finite. In particular for $G = ®S$ we have $M$ is finite;
		\item $M$ is at most countable. If $M$ is infinite, then $\O ≠ M' \subset \partial G$;
		\item if $G \subset ®C$ is domain (connected), then $G \setminus M$ is domain.
	\end{itemize}
\end{tvrzeni}

\begin{veta}[Uniqueness of meromorphic functions]
	Let $G \subset ®C$ be a domain, $f \in ©M(G)$ and $f \not≡ 0$. Then $N_f := \{z \in G | f(z) = 0\}$ has no limit points in $G$.

	\begin{dukazin}
		We know this holds for holomorphic functions. Set $G_0 := G \setminus P_f$. Then $G_0 \subset ®C$ is also domain and $f \in ©H(G)$ and $f \not≡ 0$ on $G_0$. Then $N_f \subset G_0$ has no limit points in $G_0$, nor in $P_f$.
	\end{dukazin}
\end{veta}

\begin{veta}[Residue theorem]
	Let $G \subset ®C$ be open, $\phi$ be a closed curve (or cycle) in $G$ and
	$$ \Int \phi := \{z_0 \in ®C \setminus \<\phi\> | \ind_\phi z_0 ≠ 0\} \subset G. $$
	Let $M \subset G \setminus \<\phi\>$ be finite and $f \in ©H(G \setminus M)$. Then $\int_\phi f = 2\pi i·\sum_{s \in M} \ind_\phi s · \res_s f$.

	\begin{poznamkain}
		This holds true even if instead of finiteness of $M$, we assume only that $M \subset G \setminus \<\phi\>$ has no limit points in $G$. Indeed, we have $M_0 = M \cap \Int \phi$ is finite, because $\<\phi\> \cup \Int \phi$ is compact and $G_0 := G \setminus (M \setminus M_0)$ is open and $f$ is holomorphic on $G_0 \setminus M_0$ and by R. theorem for $G_0$ and $M_0$ we get $\int_\phi f = 2 \pi i \sum_{s \in M_0} \res_s f · \ind_\phi s$.
	\end{poznamkain}
\end{veta}

\subsection{Logarithmic integrals}

\begin{definice}[Logarithmic integral]
	Let $\phi: [a, b] \rightarrow ®C$ be a (regular) curve and let $f$ be a non-zero holomorphic function on $\<\phi\>$. Then we define logarithmic integrals integral as
	$$ I := \frac{1}{2\pi i} \int_\phi \frac{f'}{f} = \frac{1}{2\pi i} \int_a^b \frac{f'(\phi(t)) \phi'(t)}{f(\phi(t))} dt = \frac{1}{2 \pi i} \int_a^b \frac{(f(\phi(t)))'}{f(\phi(t))} dt = \frac{1}{2\pi i} \int_{f \circ \phi} \frac{dz}{z} = $$
	$$ = \frac{1}{2\pi i}(\Phi(b) - \Phi(a)), $$
	where $\Phi$ is a branch (jednoznačná větev) of logarithm of $f \circ \phi$. If $\phi$ is, in addition, closed, then $I = \ind_{f \circ \phi} 0 = \frac{1}{2\pi}(\Theta(b) - \Theta(a)) \in ®Z$, where $\Theta$ is a branch of argument of $f \circ \phi$.

	($\frac{f'}{f}$ is called logarithmic derivative of $f$, because '$(\log f)' = \frac{f'}{f}$'.)
\end{definice}

\begin{veta}[Argument principle]
	Let $G \subseteq ®C$ be a domain, $\phi$ be a closed curve in $G$ and $f \in ©M(G)$. Let $\Int \phi \subset G$ and $\<\phi\> \cap N_f = \O$, $\<\phi\> \cap P_f = \O$. Then
	$$ \frac{1}{2\pi i} \int_\phi \frac{f'}{f} = \sum_{s \in \Int \phi, f(s) = 0} n_f(s) · \ind_\phi s - \sum_{s \in \Int \phi, f(s) = ∞} p_f(s) · \ind_\phi s, $$
	where $n_f(s)$ is multiplicity of the zero point $s$ of $f$ and $p_f(s)$ is multiplicity of the pole $s$ of $f$.

	\begin{dukazin}
		By Residua theorem, we have
		$$ \frac{1}{2\pi i} \int_\phi \frac{f'}{f} = \sum_{s \in \Int \phi, s \in N_f \cup P_f} \res_s \(\frac{f'}{f}\)·\ind_\phi s. $$
		If $s \in N_f$ then on $P(s)$:
		$$ \frac{f'(z)}{f(z)} = \frac{p·c_p(z - s)^{p - 1} + …}{c_p(z - s)^p + …} = \frac{p}{z - s}·\frac{1 + …}{1 + …} \implies \res_s \(\frac{f'}{f}\) = p = n_f(s). $$
		If $s \in P_f$ then on $P(s)$
		$$ \frac{f'(z)}{f(z)} = \frac{p·c_p(z - s)^{p - 1} + …}{c_p(z - s)^p + …} = \frac{p}{z - s}·\frac{1 + …}{1 + …} \implies \res_s \(\frac{f'}{f}\) = p = -p_f(s). $$
	\end{dukazin}
\end{veta}

% 23. 02. 2023

\begin{definice}
	$$ \Sigma(f, \phi) := \sum_{s \in \Int \phi, f(s) = 0} n_f(s)·\ind_\phi s - \sum_{s \in \Int \phi, f(s) = ∞} p_f(s) · \ind_\phi s. $$
\end{definice}

\begin{lemma}
	Let $\phi_1, \phi_2: [a, b] \rightarrow ®C$ be closed curve and $s \in ®C \setminus (\<\phi_1\> \cup \<\phi_2\>)$. Assume, for $t \in [a, b]$, $|\phi_1(t) - \phi_2(t)| < |\phi_1(t) - s|$. Then $\ind_{\phi_1} s = \ind_{\phi_2} s$.

	\begin{dukazin}
		For $t \in [a, b]$, we have $|(\phi_1(t) - s) - (\phi_2(t) - s)| < |\phi_1(t) - s|$. Divide by $|\phi_1(t) - s|$:
		$$ |1 - \psi(t)| < 1, \qquad \psi(t) := \frac{\phi_2(t) - s}{\phi_1(t) - s}. $$
		Then $\psi$ is a closed curve, $<\psi> \subset U(1, 1)$, and so
		$$ 0 = \ind_\psi 0 = \frac{1}{2\pi i} \int_a^b \frac{\psi'}{\psi} = \frac{1}{2\pi i} \int_a^b \frac{\frac{\phi_2'(\phi_1 - s) - \phi_1'(\phi_2 - s)}{(\phi_1 - s)^2}}{\frac{\phi_2 - s}{\phi_1 - s}} = \frac{1}{2\pi i}\int_a^b \frac{\phi_2'}{\phi_2 - s} - \frac{1}{2\pi i}\int_a^b \frac{\phi_1'}{\phi_1 - s} = $$
		$$ = \ind_{\phi_2} s - \ind_{\phi_1} s. $$
	\end{dukazin}
\end{lemma}

\begin{veta}[Rouché]
	Let $G \subset ®C$ be a domain, $f_1, f_2 \in ©M(G)$ and $\phi$ be closed curve in $G$ such that $\Int \phi \subset G$. Assume $\forall z \in \<\phi\>$:
	$$ |f_1(z) - f_2(z)| < |f_1(z)| < +∞ $$
	Then $\Sigma(f_1, \phi) = \Sigma(f_2, \phi)$.

	\begin{dukazin}
		Set $\phi_j = f_j \circ \phi$. Then
		$$ \ind_{\phi_j} 0 = \frac{1}{2\pi i} \int_\phi \frac{f_j'}{f_j} = \Sigma(f_j, \phi). $$
		By previous lemma we have for $s = 0$: $\ind_{\phi_1} 0 = \ind_{\phi_2} 0$.
	\end{dukazin}
\end{veta}

\begin{dusledek}
	Let $f_1, f_2$ be holomorphic functions on $\overline{U(z_0, r)}$ and $\forall z \in \partial U(z_0, r): |f_1(z) - f_2(z)| < |f_1(z)|$. Then $\Sigma_1 = \Sigma_2$, where $\Sigma_j := \sum_{s \in U(z_0, r), f(s) = 0} n_{f_j}(s)$.

	\begin{dukazin}
		Apply Rouché's theorem to $\phi(t) := z_0 + r·e^{it}$, $t \in [0, 2\pi]$.
	\end{dukazin}
\end{dusledek}

\begin{priklad}
	$f_2 = p$, $f_1(z) = a_0 z^n$ and big enough $U(0, r)$.
\end{priklad}

\begin{definice}[Notation]
	Let $f$ be a function holomorphic at $z_0 \in ®C$. We say that $f(z_0) = w_0 \in ®C$ $p$~times for $p \in ®N$ if $z_0$ is a zero point of $f - w_0$ of order $p$.

	\begin{poznamkain}
		Following statements are equivalent to each other:

		\begin{itemize}
			\item $f(z_0) = w_0$ $p$~times;
			\item $f(z_0) = w_0$, $f'(z_0) = 0 = … = f^{(p - 1)}(z_0)$, $f^{(p)}(z_0) ≠ 0$;
			\item $f(z) = w_0 + \sum_{k=p}^{+∞} c_k(z - z_0)^k$ on some neighbourhood of $z_0$ and $c_p ≠ 0$.
		\end{itemize}
	\end{poznamkain}

	We say that $f(z_0) = ∞$ $p$~times if $z_0$ is a zero point of $\frac{1}{f}$ of order $p$. (It's the same as $z_0$ is pole of $f$ of order $p$.) And we say that $f(∞) = w_0 \in ®S$ $p$~times if $f(1/z)$ attains $w_0$ $p$~times at 0.
\end{definice}

\begin{veta}[On a multiple value]
	Let $z_0, w_0 \in ®S$, $f$ be a holomorphic function on a $P(z_0)$ and $f(z_0) = w_0$ $p$ times for some $p \in ®N$. Let $\delta_0 > 0$. Then there are $\epsilon > 0$ and $\delta \in (0, \delta_0)$ such that, for any $w \in P(w_0, \epsilon)$ there are just $p$ different points $z_1, …, z_p$ in $P(z_0, \delta)$ with $f(z_j) = w$. In addition, $f(z_j) = 0$ once.

	\begin{dukazin}
		WLOG, assume $z_0 = 0 = w_0$.Then $z_0 = 0$ is a zero point of $f$ of order $p$. Choose $\delta \in (0, \delta_0)$ such that $f ≠ 0$ and $f' ≠ 0$ on $P(0, 2\delta)$. Set $\epsilon := \min_{|z| = \delta} |f(z)| > 0$.

		Let $w \in P(0, \epsilon)$. Use Rouché's theorem for $f_1 := f$, $f_2 := f - w$ and $\phi := \delta e^{it}$, $t \in [0, 2\pi]$. Of course, $|f_1 - f_2| = |w| < \epsilon < |f_1|$ on $\<\phi\>$.

		Since in $U(0, \delta)$ the function $f = f_1$ has the only zero point of order $p$ at origin, $f - w = f_2$ has just $p$ simple zero points in $P(0, \delta)$.
	\end{dukazin}
\end{veta}

\begin{dusledek}
	Let $G \subset ®S$ be a domain, $f \in ©M(G)$ and $f$ be not constant on $G$. Then $f: G \rightarrow ®S$ is an open map (for any open $\Omega \subset G, f(\Omega)$ is open).

	\begin{dukazin}
		Let $\Omega \subset G$ be open and $w_0 \in f(\Omega)$. Then there is a $z_0 \in \Omega$ and $p \in ®N$ such that $f(z_0) = w_0$ $p$ times. Choose $\delta_0 > 0$ such that $U(z_0, \delta_0) \subset \Omega$. By the previous theorem, there is $\epsilon > 0$, $\delta \in (0, \delta_0)$ such that $P(w_0, \epsilon) \subset f(P(z_0, \delta))$, so $U(w_0, \epsilon) \subset f(U(z_0, \delta)) \subset f(\Omega)$.
	\end{dukazin}

	\begin{poznamkain}
		This is true for $©H(G)$ too.
	\end{poznamkain}
\end{dusledek}

\begin{dusledek}
	Let $f$ be a function holomorphic at $z_0 \in ®C$. Then $f'(z_0) ≠ 0$ if and only if there is $U(z_0)$ such that $f|_{U(z_0)}$ is one-to-one.

	\begin{dukazin}
		„$\implies$“: Let $f'(z_0) ≠ 0$. Then $f(z_0) = w_0$ once, so we choose $\delta_0 > 0$ such that $f ≠ w_0$ on a $P(z_0, \delta_0)$. By the previous theorem choose $\epsilon > 0$, $\delta \in (0, \delta_0)$. Moreover, due to the continuity of $f$ at $z_0$ choose $\delta_1 \in (0, \delta)$ such that $f(U(z_0, \delta_1)) \subset U(w_0, \epsilon)$. Then $f|_{U(z_0, \delta_1)}$ is one-to-one.

		„$\impliedby$“: Let $f'(z_0) = 0$ and let $f$ be not constant on any neighbourhood of $z_0$. Then $f(z_0) = w_0$ $p$ times ($p \in ®N \setminus \{1\}$). By the previous theorem $f$ is not one-to-one on any neighbourhood of $z_0$.
	\end{dukazin}
\end{dusledek}

\begin{veta}[On holomorphic inverse]
	Let $G \subset ®C$ be open and $f: G \rightarrow ®C$ be a one-to-one holomorphic\footnote{One-to-one holomorphic function is sometimes called conformal.} function, then $f' ≠ 0$ on $G$, $\Omega := f(G)$ is open and $f_{-1}: \Omega \overset{\text{onto}}\rightarrow G$ is holomorphic.

	In addition, $(f_{-1})' = \frac{1}{f' \circ f_{-1}}$ on $\Omega$.

	\begin{dukazin}
		WLOG, $G \subset ®C$ is a domain. By first „dusledek“ of previous theorem $f$ is an open map, so $\Omega := f(G)$ is open and $f_{-1}: \Omega \rightarrow G$ is continuous. Let $z_0 \in G$ and $w_0 = f(z_0)$. By second „dusledek“ we have $f'(z_0) ≠ 0$, and
		$$ \frac{1}{f'(z_0)} = \lim_{z \rightarrow z_0} \frac{z - z_0}{f(z) - f(z_0)} \overset*= \lim_{w \rightarrow w_0} \frac{f_{-1}(w) - f_{-1}(w_0)}{w - w_0} = f_{-1}'(w_0). $$
		The equality * follows from theorem on limits of composite functions because $f_{-1}$ is continuous and $f_{-1}(w) ≠ f_{-1}(w_0)$ for $w ≠ w_0$.
	\end{dukazin}
\end{veta}

\begin{veta}[Hurwitz]
	Let $G \subset ®C$ be a domain, $f_n \in ©H(G)$, $f_n \overset{\text{loc.}}\rightrightarrows f$ on $G$ and $f \not≡ 0$. Let $z_0 \in G$ be a zero point of $f$. Then $\exists$ $\{z_n\}_{n = 1}^∞ \subset G$ and a subsequence $\{f_{k_n}\}$ of $\{f_n\}$ such that $z_n \rightarrow 0$ and $f_{k_n}(z_n) = 0$.

	\begin{poznamkain}
		Not true in ®R! The assumption $f \not≡ 0$ is important! ($f_n(z) := z/n$)
	\end{poznamkain}
\end{veta}

\begin{dusledek}
	Let $G \subset ®C$ be a domain, $f_n$ be one-to-one holomorphic functions on $G$ and $f_n \overset{\text{loc}}\rightrightarrows f$ on $G$. Then $f$ is either one-to-one and holomorphic, or constant.
\end{dusledek}

% 02. 03. 2023

\begin{dukaz}[Hurwitz theorem]
	Choose $\delta > 0$ such that $U(z_0, \delta) \subset G$ and $f ≠ 0$ on $P(z_0, \delta)$. For $n \in ®N$ put $\rho_n := \frac{\delta}{n + 1}$ and $\phi_n(t) := z_0 + \rho_n e^{it}$, $t \in [0, 2\pi]$. Of course, $\tau_n := \min_{\<\phi_n\>} |f| > 0$. For a given $n$, there is (from uniformly convergence) $k_n \in ®N$ such that $\forall z \in \<\phi_n\>: |f_{k_n}(z) - f(z)| < \tau_n ≤ |f|$.

	By Rouché's theorem there is $z_n \in U(z_0, \rho_n)$ such that $f_{k_n}(z_n) = 0$. Of course, we can choose $\{k_n\}$ to be increasing.
\end{dukaz}

\begin{dukaz}[Corollary]
	Assume that there is $w_0 \in ®C$ such that $f ≠ w_0$ but, for different $z', z'' \in G$ we have $f(z') = w_0 = f(z'')$. WLOG $w_0 = 0$. Choose $\delta > 0$ such that $U(z', \delta) \cap U(z'', \delta) = \O$. By Hurwitz, there are $\{z_n'\} \subset U(z', \delta)$ and $\{f_{k_n'}\}$ of $\{f_n\}$ such that $z_n' \rightarrow z'$ and $f_{k_n'}(z_n') = 0$. By Hurwitz, there are also $\{z_n''\} \subset U(z'', \delta)$ and $\{f_{k_n''}\} \subset \{f_{k_n'}\}$ such that $z_n'' \rightarrow z''$ and $f_{k_n''}(z_n'') = 0$.

	Every $f_{k_n''}$ has at least two different zero points which is contradiction.
\end{dukaz}

\begin{veta}[Mittag-Leffler]
	Let $\{s_j\} \subset ®C$ be one-to-one, $s_j \rightarrow ∞$ and $s_0 := 0 < |s_1| ≤ |s_2| ≤ |s_3| ≤ … ≤ |s_j| ≤ …$

	Let $P_0, P_1, …, P_j, …$ be polynomials such that $P_j(0) = 0$. Then the function
	$$ f(z) := P_0\(\frac{1}{z}\) + \sum_{j=1}^∞ \(P_j\(\frac{1}{z - s_j}\) - Q_j(z)\) $$
	for some polynomials $Q_j$ satisfies:
	\begin{enumerate}
		\item series in definition converges locally uniformly on ®C, i. e., on any compact $K \subset ®C$, the series converges uniformly if we omit finitely many terms which have poles.
		\item $f \in ©M(®C)$ and $f$ has poles just at $s_0, s_1, …, s_j, …$, while at $s_j$ the function $f$ has its principal part equal to $P_j\(\frac{1}{z - s_j}\)$.
		\item If $g \in ©M(®C)$ satisfies previous property, then there is $h \in ©H(®C)$ such that $g = f + h$ on $G$.
	\end{enumerate}

	\begin{dukazin}
		Let $k \in ®N$. Then $H_k(z) := P_k\(\frac{1}{z - s_k}\) \in ©H(U(0, |s_k|))$, $H_k(z) = \sum_{n=0}^∞ c_n^k z^n$ for $|z| < |s_k|$. There is $n_k \in ®N$ such that $Q_k(z) = \sum_{n=1}^{n_k} c_n^k z^n$ satisfies $|H_k(z) - Q_k(z)| < \frac{1}{2^k}$, $|z| ≤ \frac{|s_k|}{2}$ (*).

		Let $K \subset ®C$ be a compact. Choose $k_0 \in ®N$ such that $K \subset \overline{U(0, |s_{k_0}| / 2)}$. If $k > k_0$, (*) holds on $K$ which implies 1. obviously, 2. is valid.

		3. follow from the fact that $g - f \in ©M(®C)$ has all isolated singularities removable.
	\end{dukazin}
\end{veta}

\section{Zero points of holomorphic functions}
\begin{tvrzeni}
	Let $f$ be non-zero holomorphic function on a simply connected domain ($G$ is domain, and $®S \setminus G$ is connected) $G \subset ®C$. Then there is $L \in ©H(G)$ such that $f = e^L$ on $G$.

	\begin{dukazin}
		1) Let $L \in ©H(G)$ and $f = e^L$ on $G$. Then $f' = L'·e^L$ and $f'/f = L'$.

		2) Since $G$ is a simply connected domain and $f'/f \in ©H(G)$, by Cauchy theorem, there is $L_0 \in ©H(G)$ such that $L_0' = f'/f$.

		3) On $G$ we have $\(f·e^{-L_0}\)' = e^{-L_0}·(f' - L_0'·f) = 0$ on $G$, hence $f·e^{-L_0} = e^c$ is constant, i. e. $c \in ®C$. Put $L := L_0 + c$.
	\end{dukazin}
\end{tvrzeni}

\begin{poznamka}
	Polynomial $f(z) = \prod_{j=1}^n (z - z_j)$ has zero points just at $z_1, …, z_n$ and their multiplicity corresponds to their occurrence.

	Let $g \in ©H(®C)$ have the same zero points including multiplicity as $f$. Then there is $L \in ©H(®C)$ such that $g = f·e^L$ on ®C. (Proof: use previous tvrzeni for $g / f$.)
\end{poznamka}

\begin{poznamka}[Notation]
	Let $\{a_j\} \subset ®C$. Then we define
	$$ \prod_{j=1}^∞ a_j := \lim_{n \rightarrow ∞} \prod_{j=1}^n a_j, $$
	if the limit on the right-hand side exists.
\end{poznamka}

\begin{tvrzeni}
	Let $0 ≠ z_j \rightarrow ∞$ and $k \in ®N_0$ (multiplicity of 0 as zero point). Then consider
	$$ f(z) := z^k \prod_{j=1}^∞ \(1 - \frac{z}{z_j}\). $$
	It sometimes converges and then $f$ has zero points in $z_j$ with right multiplicities.
\end{tvrzeni}

\begin{veta}[On infinite product]
	Let $M$ be a set (in ®C), $u_j : M \rightarrow ®C$ be bounded and $\sum_{j=1}^∞ |u_j|$ converges uniformly on $M$. Then $p_n := \prod_{j=1}^n(1 + u_j)$ converge uniformly to a function $f: M \rightarrow ®C$, and it holds that $f = \prod_{j=1}^∞ (1 + u_{n(j)})$ on $M$, where $n$ is bijection onto ®N.

	If $z_0 \in M$, then $f(z_0) = 0$ if and only if $u_{j_0}(z_0) = -1$ for some $j_0 \in ®N$.

% 09. 03. 2023

	\begin{dukazin}
		Denote $p_n^* := \prod_{j=1}^n (1 + |u_j|)$. Then $p_n^* ≤ \exp\(\sum_{j=1}^n |u_j|\)$ and $|p_n - 1| ≤ p_n^* - 1$ (from $1 + x ≤ e^x$ and the second inequality by induction on $n$: $n = 1$ yes, $p_{n+1} - 1 = p_n(1+u_{n+1}) - 1 = (p_n - 1)·(1 + u_{n+1}) + u_{n+1}$ so $|p_{n+1} - 1| ≤ (p_n^* - 1)·(1 + |u_{n+1}|) + |u_{n+1}| = p_{n+1}^* - 1$).

		$\sum_{j=1}^∞ |u_j|$ is bounded on $M$, because there is $n_0 \in ®N$ such that $\sum_{j=n_0 + 1}^∞ |u_j| < 1$. By inequalities there is $C \in (0, +∞)$ such that $|p_n| ≤ C$ $\forall n \in ®N$.

		Let $0 < \epsilon < \frac{1}{2}$. Choose $n_0 \in ®N$ such that $\sum_{n=n_0}^∞ |u_n| < \epsilon$ on $M$. Let $\{n_1, n_2, …\}$ be a permutation of $®N$ and $q_m := \prod_{j=1}^m (1 + u_{n_j})$, $m \in ®N$. Let $n ≥ n_0$ and $m \in ®N$ be such that $\{n_1, …, n_m\} \supseteq [n]$. Then
		$$ |q_m - p_n| = |p_n·\(\prod_{n_j > n, j \in [m]} (1 + u_{n_j}) - 1\) ≤ |p_n|\(\prod_{…} (1 + |u_{n_j}|) - 1\) ≤ $$
		$$ ≤ |p_n|·\(e^{\sum_{…} |u_{n_j}|} - 1\) ≤ |p_n|·\(e^\epsilon - 1\) ≤ |p_n|·2\epsilon ≤ 2C\epsilon. $$

		If $n_j = j$ $\forall j \in ®N$, then $q_m = p_m$ and we get $\forall m > n: |q_m - p_n| < 2C\epsilon$, so $p_n \rightrightarrows f$ on $M$. Moreover we have, for $n ≥ n_0$, $|p_n - p_{n_0}| ≤ 2 \epsilon |p_{n_0}|$, so $|p_n| ≥ |p_{n_0}| - |p_n - p_{n_0}| ≥ (1 - 2\epsilon)|p_{n_0}|$. For $n \rightarrow ∞$: $|f| ≥ (1 - 2\epsilon) |p_{n_0}|$, hence $f(z_0) = 0 \Leftrightarrow p_{n_0}(z_0) = 0$.

		If $n_j$ is any, then $q_m \rightrightarrows f$ on $M$.
	\end{dukazin}
\end{veta}

\begin{dusledek}
	Let $G \subset ®C$ be open, $f_n \in ©H(G)$ and $f_n \not≡ 0$ on any component of $G$. We assume $\sum_{n=1}^∞ |1 - f_n|$ converges locally uniformly on $G$. Then $f = \prod_{n=1}^∞ f_n$ converges locally uniformly on $G$, $f \in ©H(G)$ and the resulting infinite product $f$ does not depend on the order of functions $f_n$. Moreover, we have
	$$ n_f(s) = \sum_{k=1}^∞ n_{f_k}(s), \qquad s \in G $$
	where $n_f(s)$ is multiplicity of a zero point $s$ of $f$. Here we put $n_f(s) = 0$ if $f(s) ≠ 0$.

	\begin{poznamkain}
		Moreover the ? in previous sum contains only finitely many non-zero terms for any $s \in G$.
	\end{poznamkain}

	\begin{dukazin}
		Sufficient to prove previous equality. Let $s \in G$. There is a neighbourhood $V$ of $s$ such that $f_n \rightrightarrows 1$ on $V$. Choose $n_0 \in ®N$ such that $f_n ≠ 0$ on $V$ for $n > n_0$. By previous theorem, we get $\prod_{n=n_0 + 1}^∞ f_n ≠ 0$ on $V$. Since $f = \(\prod_{n=1}^{n_0} f_n\)·\(\prod_{n=n_0+1}^∞ f_n\)$ we get $n_f(s) = \sum_{k=1}^{n_0} n_{f_k}(s) = \sum_{k=1}^∞ n_{f_k}(s)$.
	\end{dukazin}
\end{dusledek}

\begin{priklad}[Homework]
	Under the assumption of previous corollary prove that
	$$ \frac{f'}{f} = \sum_{n=1}^∞ \frac{f_n'}{f_n} \text{ on } G \setminus N_f. $$
\end{priklad}

\begin{priklady}[Euler formula]
	$$ \sin(\pi z) = \pi z · \prod_{k=1}^∞ \(1 - \frac{z^2}{k^2}\). $$
\end{priklady}

\begin{lemma}[Weierstrass's factor]
	Let $E_0(z) := (1 - z)$ and $E_m(z) := (1 - z)·e^{z + … + \frac{z^m}{m}}$, $z \in ®C$, $m \in ®N$. Then $|1 - E_m(z)| ≤ |z|^{m + 1}$, $|z| ≤ 1$.

	\begin{dukazin}
		$$ E_m'(z) = e^{z + … + \frac{z^m}{m}}·(-1 + (1 - z)·(1 + … + z^m)) = -z^m·e^{z + … + \frac{z^m}{m}} = -z^m·\sum_{k=0}^∞ b_kz^k, $$
		where $b_0 = 1$, $b_k ≥ 0$, $k \in ®N$. Hence
		$$ E_m(0) - E_m(z) = 1 - E_m(z) = - \int_{[0, z]} E_m'(w) dw = + \sum_{k=0}^∞ c_k z^{k + m + 1} $$
		with $c_k = \frac{b_k}{m + k + 1} ≥ 0$.

		By this, if $|z| ≤ 1$, $z ≠ 0$, then $\left|\frac{1 - E_m(z)}{z^m}\right| ≤ \sum_{k=0}^∞ c_k = 1 - E_m(1) = 1$.
	\end{dukazin}
\end{lemma}

\begin{veta}[Weierstrass factorization in ®C]
	Let $k \in ®N_0$ and $0 ≠ z_i \rightarrow ∞$. Then there is $\{m_j\} \subset ®N_0$ such that
	$$ f(z) = z^k·\prod_{j=1}^∞ E_{m_j}\(\frac{z}{z_j}\) $$
	converges locally uniformly on ®C, $f \in ©H(®C)$ and $f$ has at $0$ zero point of multiplicity $K$ and 'non-zero' zero points just at $z_1, z_2, …, z_j, …$, and their multiplicity corresponds to their occurrence in $\{z_j\}$. We can always take $m_j := j-1$, $j \in ®N$.

	If $g \in ©H(®C)$ has the same zero points as $f$ including multiplicities, then there is $L \in ©H(®C)$ such that $g = f·e^L$ on ®C.

	\begin{dukazin}
		By the previous corollary, we know the product converges locally uniformly in ®C if
		$$ \sum_{j=1}^∞ \left|1 - E_{m_j} \(\frac{z}{z_j}\)\right| $$
		converges locally uniformly on ®C. By lemma, this is true if $\sum_{j=1}^∞ \left|\frac{z}{z_j}\right|^{m_j + 1}$ converges locally uniformly on ®C.

		Let $r > 0$ and $|z| ≤ r$. Choose $j_0 \in ®N$ such that $\frac{r}{|z_j|} < \frac{1}{2}$ for $j ≥ j_0$. If $m_j := j - 1$, then $\left|\frac{z}{z_j}\right|^j ≤ \frac{1}{2^j}$, $j ≥ j_0$ and $|z| ≤ r$. So, for $m_j := j - 1$, sum converges uniformly on $|z| ≤ r$.
	\end{dukazin}
\end{veta}

\begin{poznamka}
	If $\sum_{j=1}^∞ \frac{1}{|z_j|} < +∞$, take $m_j = 0$. If $\sum_{j=1}^∞ \frac{1}{|z_j|^2} < +∞$, take $m_j = 1$. Etc.
\end{poznamka}

\begin{veta}[Weierstrass factorization in a general open set]
	Let $G \subsetneq ®S$ be open, $N \subset G$ have no limit points in $G$ and $n: N \rightarrow ®N$. Then there is $f \in ©H(G)$ such that $N_f = N$ and $n_f(s) = n(s)$, $s \in N_f$.

	\begin{dukazin}
		WLOG $∞ \in G \setminus N$. Then $K := ®S \setminus G = ®C \setminus G$ is compact in ®C. For a finite $N$ it is obvious. Assume that $N$ is (infinite) countable. We put points of $N$ into the sequence $s_1, s_2, …, s_n$ such that any $s \in N$ occurs in $\{s_n\}$ just $n(s)$ times. For any $n$, take $t_n \in K$ such that $|s_n - t_n| = \dist(s_n, K)$, $n \in ®N$.

		Then „$|s_n - t_n| \rightarrow 0$“: Let $\epsilon > 0$ and $\{n_k\} \subset ®N$ such that $|s_{n_k} - t_{n_k}| ≥ \epsilon$, i. e., $\dist(s_{n_k}, K) ≥ \epsilon$. If $s_∞$ is a limit point of $s_{n_k}$, then $\dist (s_∞, K) ≥ \epsilon$. Hence $s_∞ \in G$, a contradiction.

		Put $f(z) := \prod_{n=1}^∞ E_n\(\frac{s_n - t_n}{z - t_n}\)$, $z \in G$. The infinite product converges locally uniformly on $G$. In fact, let $L$ be a compact in $G$. Put $r_n := 2·|s_n - t_n|$. Since $\dist(L, K) > 0$, there is $n_0 \in ®N$ such that $|z - t_n| > r_n$, $\forall z \in L$, $\forall n ≥ n_0$. So
		$$ \left|\frac{s_n - t_n}{z - t_n}\right| < \frac{1}{2} \qquad \forall z \in L\ \forall n ≥ n_0. $$
		By lemma on Weierstrass factors, we get
		$$ \left|1 - E_n\(\frac{s_n - t_n}{z - t_n}\)\right| < \frac{1}{2^n} \qquad \forall z \in L\ \forall n ≥ n_0. $$
		Now use theorem on infinite product.
	\end{dukazin}
\end{veta}

% 16. 03. 2023

\begin{lemma}
	If $G \subseteq ®C$ is open and $f \in ©M(G)$, then there are $g, h \in ©H(G)$ such that $f = \frac{g}{h}$ on $G$.

	\begin{dukazin}
		Let $P_f$ be the set of poles of $f$. By Weierstrass factorization, we construct $h \in ©H(G)$ such that $N_h = P_f$ and $n_h = p_f$ on $P_f$. Put $g := f·h$. Then $g \in ©H(G)$ because at the points of $P_f$ $g$ has a removable singularities.
	\end{dukazin}
\end{lemma}

\section{The space H(G)}
\begin{poznamka}[Arzela–Ascoli theorem]
	Let $©F \subset ©C(K)$ and let the functions of ©F be equibounded (i.e. $\exists M \in (0, +∞)\ \forall f \in ©F: |f| ≤ M$ on $K$) and equicontinuous (i.e. $\forall ε > 0\ \exists δ > 0\ \forall f \in ©F\ \forall x, y \in K: ρ(x, y) < δ \implies |f(x) - f(y)| < ε$, where $ρ$ is metric on $K$).  Then every $\{f_n\} \subset ©F$ has $\{f_{n_k}\}$ which is uniformly convergent on $K$. 
\end{poznamka}

\subsection{The space C(G)}
\begin{definice}
	Let $G \subseteq ®C$, then $©C(G) := \{f: G \rightarrow ®C | f \text{ continuous}\}$.
\end{definice}

\begin{tvrzeni}
	For $f_n, f \in ©C(G)$ and $K_m$ compact in $G$ such that $\bigcup_{m=1}^∞ K_m = G$ and $\forall m \in ®N: K_m \subseteq \Int K_{m+1}$, TSAE:
	
	\begin{itemize}
		\item $f_n \overset{\text{loc.}}\rightrightarrows$ on $G$;
		\item for any compact $K$ in $G$, $\|f_n - f\| \rightarrow 0$, where $\|f\|_K := \sup_K |f|$ is a seminorm on $©C(G)$;
		\item $\forall m \in ®N: \|f_n - f\|_{K_m} \rightarrow 0$ for $u \rightarrow ∞$;
		\item $ρ(f_n, f) \rightarrow 0$, where $ρ(f_n, f) := \sum_{m=1}^∞ \frac{1}{2^m}·\frac{\|f_n - f\|_{K_m}}{1 + \|f_n - f\|_{K_m}}$.
	\end{itemize}

	\begin{dukazin}
		„$1 \Leftrightarrow 2 \implies 3$“ is obvious. „$2 \impliedby 3$“: Let $K$ be a compact in $G$. Then $K \subset K_{m_0}$ for come $m_0 \in ®N$. Then $\|f_n - f\|_K ≤ \|f_n - f\|_{K_{m_0}}$. „$3 \Leftrightarrow 4$“ homework.
	\end{dukazin}
\end{tvrzeni}

\begin{poznamka}
	$(©C(G), ρ)$, where $ρ$ is defined in previous tvrzeni, is complete metric space and $©H(G)$ is closed subspace.

	$ρ$ is not canonical, it depends on the choice of $\{K_m\}$.

	The convergence / the topology on $©C(G)$ is given by the system of seminorms $\|·\|_K$ for any compact $K$ in $G$.
\end{poznamka}

\begin{veta}[Moore–Osgood, Montöl]
	Let $G \subset ®C$ be open and let $\{f_n\} \subset ©H(G)$ be locally equibounded (i.e. on every compact $K$ in $G$ $\{f_n\}$ is equibounded). Then there is $\{f_{n_k}\}$ which converges locally uniformly on $G$.

	\begin{dukazin}
		First step: Let $\overline{U(z_0, 2r)} \subset G$ and $\phi(t) := z_0 + 2 r e^{it}$, $t \in [0, 2π]$. Let $z_1, z_2 \in \overline{U(z_0, r)}$. Then by the Cauchy formula we get $f_n(z_j) = \frac{1}{2 π i} \int_φ \frac{f_n(z)}{z - z_j} dz$. There is $M \in (0, +∞)$ such that $\forall n \in ®N$ $|f_n| ≤ M$ on $\<φ\>$. Then we have
		$$ |f_n(z_1) - f_n(z_2)| = \frac{1}{2π}\left| \int_φ f_n(z)·\(\frac{1}{z - z_1} - \frac{1}{z - z_2}\) dz \right| ≤ $$
		$$ ≤ \frac{2π·2r}{2π}·M·\frac{|z_1 - z_2|}{r^2} $$
		($\left|\frac{1}{z - z_1} - \frac{1}{z - z_2}\right| = \left|\frac{z_1 - z_2}{(z - z_1)·(z - z_2)}\right|≤ \frac{|z_1 - z_2|}{r^2}$).

		By this $\{f_n\}$ are equicontinuous on $\overline{U(z_0, r)}$, and by Arzela–Ascoli, there is $\{f_{n_k}\}$ which is uniformly convergent on $\overline{U(z_0, r)}$.

		Second step: Let us cover the set $G$ by $U_j = U(z_j, r_j)$, $j \in ®N$, such that $\overline{U(z_j, 2r_j)} \subset G$. Then use a diagonal choice: 1. By first step choose $\{f_{n_k^1}\}$ of $\{f_n\}$ such that $\{f_{n_k^1}\}$ converges uniformly on $\overline{U_1}$. 2. By first step choose $\{f_{n_k^2}\}$ subsequence of $\{f_{n_k^1}\}$ such that $\{f_{n_k^2}\}$ converges uniformly on $\overline{U_2}$ and so on.

		Then $\{f_{n_k^k}\}_{k=1}^∞$ converges uniformly on any $\overline{U_j}$, i.e., locally uniformly on $G$.
	\end{dukazin}
\end{veta}

\begin{definice}[]
	Let $E$ be a (complex) linear space and let ©P be a system of seminorms on $E$. Then $(E, ©P)$ is called locally convex space (LCS). In $(E, ©P)$ we define:

	\begin{itemize}
		\item convergence: $f_n \rightarrow f \Leftrightarrow \forall p \in ©P: p(f_n - f) \rightarrow 0$;
		\item topology $τ$ is the weakest topology on $E$ for which all $p \in ©P$ are continuous;
		\item $©F \subset E$ is bounded if $©F$ is bounded with respect to any $p \in ©P$, i.e.,
			$$ \forall p \in ©P\ \exists C \in (0, +∞): p(f) ≤ C\ \forall f \in ©F; $$
		\item the dual space to $(E, ®P)$ is defined as
			$$ E^* := \{L: E \rightarrow ®C | L \text{ linear and continuous}\}. $$
	\end{itemize}
\end{definice}

\begin{poznamka}
	$©C(G)$ is the so-called Fréchet space, i.e., completely metrizable LCS. So is $©H(G)$ because $©H(G)$ is closed subspace of $©C(G)$.

	Topology $τ$ on $©C(G)$ is generated by the system of seminorms
	$$ ©P := \{\|·\|_K | K \text{ is compact in $G$}\}. $$
	$U \subset ©C(G)$ is neighbourhood of $f \in ©C(G)$ iff there are a compact $K \in G$ and $ε > 0$ such that
	$$ U \supset U_{K, ε}(f) := \{g \in ©C(G) | \|g - f\|_K < ε\}. $$

	\begin{dukazin}
		„$\impliedby$“: obvious. „$\implies$“: There are $m \in ®N$, compact, $K_1, …, K_m$ in $G$ and $ε_1, …, ε_m > 0$ such that
		$$ U \supset \bigcap_{j=1}^m U_{K_j, ε_j}(f) \supset U_{K, ε}(f), $$
		where $K := K_1 \cup … \cup K_m$ and $ε := \min\{ε_1, …, ε_m\} > 0$.
	\end{dukazin}
\end{poznamka}

\begin{poznamka}
	Let $X = ©H(G)$. Then in the sense of (LCS) $©F \subset ©H(G)$ is bounded iff in the functions of $©F$ are locally equibounded on $G$. By the Montal theorem, we get $\overline{©F}$ is a compact in $©H(G)$. Easily we get that $©F \subset X$ is compact iff ©F is closed and bounded in $X$.
\end{poznamka}

% 23. 03. 2023

\section{\texorpdfstring{The dual space $©H^*(G)$}{The dual space H*(G)}}
\begin{poznamka}
	1. Let $G = ®D := \{z \in ®C | |z| < 1\}$. Let $L \in ©H^*(®D)$. Let $f \in ©H(®D)$, $f(z) = \sum_{n=0}^∞ a_n z^n$, $z \in ®D$, and $R := \frac{1}{\limsup_{n \rightarrow +∞} \sqrt[n]{|a_n|}} ≥ 1$. Then
	$$ L(f) = L(\sum_{n=0}^∞ a_n z^n) = L\(\lim_{n \rightarrow ∞} \sum_{k=0}^n a_kz^k\) = \lim_{n \rightarrow ∞} \sum_{k=0}^n a_k L(z^k) = \sum_{n=0}^∞ a_n·b_n, $$
	where $b_n := L(z^n) \in ®C$. We show $r := \limsup_{n \rightarrow ∞} \sqrt[n]{|b_n|} < 1$:

	If $r > 1$, then for $a_n := 1$, $n \in ®N_0$, we get $\sum_{n=0}^∞ a_n·b_n$ is divergent. If $r = 1$, then there is $\{n_k\}$ such that such that $0 ≠ \sqrt[n_k]{|b_{n_k}|} \rightarrow 1$. Putting $a_n = \frac{1}{b_{n_k}}$, $n = n_k$, we get $\sum_{n=0}^∞ a_n b_n$ is divergent.

	Conclusion: $L \in ©H^*(®D)$ iff there is a unique $\{b_n\} \subset ®C$ such that $\limsup_{n \rightarrow ∞} \sqrt[n]{|b_n|} < 1$ and $L(f) = \sum_{n=0}^∞ a_n b_n$ for $f(z) = \sum_{n=0}^∞ a_n z^n \in ©H(®D)$. In addition, $b_n = L(z^n)$, $n \in ®N_0$. ($\impliedby$ obvious, HW.)
\end{poznamka}

\begin{poznamka}[Integral form of $L$]
	Let $\{b_n\} \subset ®C$ and $r:= \limsup_{n \rightarrow ∞} \sqrt[n]{|b_n|} < 1$. Define
	$$ λ(z) := \sum_{n=0}^∞ \frac{b_n}{z^{n + 1}}, \qquad |z| > r. $$
	Of course, $λ \in ©H(®S \setminus \overline{U(0, r)})$, $λ(∞) = 0$ and $b_n = \frac{λ^{(n+1)}(∞)}{(n + 1)!}$, $n \in ®N_0$. Here $λ^{(k)}(∞) := \(λ\(\frac{1}{z}\)\)^{(k)} (0)$.

	Let $R \in (r, 1)$ and $φ(t) := R e^{it}$, $t \in [0, 2π]$. Let $f \in ©H(®D)$ and $f(z) = \sum_{n=0}^∞ a_n z^n$, $z \in ®D$. Then
	$$ \frac{1}{2πi} \int_φ f(z)·λ(z) dz = \frac{1}{2πi} \int_φ \(\sum_{n=0}^∞ a_n·z^n\)·\(\sum_{m=0}^∞ \frac{b_m}{z^{m+1}}\) dz = $$
	$$ = \frac{1}{2πi} \int_φ \sum_{n,m = 0}^∞ a_n b_m z^{n - m - 1} dz = \sum_{n, m = 0}^∞ a_n·b_m· \frac{1}{2πi} \int_φ z^{n - m - 1} dz = \sum_{n=0^∞} a_n·b_n = L(f). $$
\end{poznamka}

\begin{definice}[Notation]
	Let $A \subset ®S$. Then a function $f$ is holomorphic on $A$ if $f$ is holomorphic on some open superset $U \supset A$. Let $f_1, f_2$ be holomorphic function on $A$. We say that $f_1$ \textasciitilde $f_2$ if there are open $U_1, U_2 \subset ®S$ such that $A \subset U_1 \cap U_2$, $f_1 \in ©H(U_1)$, $f_2 \in ©H(U_2)$ and $f_1 = f_2$ on $U_1 \cap U_2$. Denote $©H(A) := \{[f] | f \text{ is holomorphic on } A\}$, where $[f]$ is an equivalence class for \textasciitilde. As usual, we do not often distinguish between $[f]$ and $f$.

	We have that $λ \in ©H_0(®S \setminus ®D) := \{μ \in ©H(®S \setminus ®D) | μ(∞) = 0\}$. Moreover, we have
	$$ (*) L(f) = \frac{1}{2πi} \int_φ f(z)·λ(z) dz, \qquad f \in ©H(®D); $$
	$$ L(z^n) = \frac{λ^{(n + 1)}(∞)}{(n + 1)!}, \qquad n \in ®N_0; $$
	$$ λ(w) = L\(\frac{1}{w - z}\), \qquad |w| ≥ 1. $$

	\begin{dukazin}
		In fact, we have
		$$ L\(\frac{1}{w - z}\) = L\(\sum_{n=0}^∞ \frac{z^n}{w^{n + 1}}\) = \sum_{n=0}^∞ \frac{b_n}{w^{n+1}} = λ(w), $$
		because $\frac{1}{w - z} = \frac{1}{w}·\frac{1}{1 - \frac{1}{w}} = \sum_{n=0}^∞ \frac{z^n}{w^{n+1}}$, $z \in ®D$.
	\end{dukazin}
\end{definice}

\begin{poznamka}[Conclusion]
	$$ ©H^*(®D) = ©H_0(®S \setminus ®D). $$
	In particular, $L \in ©H^*(®D)$ iff there is a unique $λ \in ©H_0(®S \setminus ®D)$ such that $(*)$ hold true.
\end{poznamka}

\begin{priklad}[Birkhoff]
	There is a universal entire function, i.e., $f \in ©H(®C)$ such that $\overline{\{τ_γ(f) | γ \in ®C\}} = ©H(®C)$, where $τ_γ(f) := f(z - γ)$, $z, γ \in ®C$.

	\begin{reseni}
		HW.
	\end{reseni}
\end{priklad}

\begin{poznamka}
	2. Let $G = \bigcup_{j=1}^n D_j$ with $D_j = U(z_j, r_j)$ and $D_j \cap D_k = \O$ for $j = k$.

	Let $L \in ©H^*(G)$. For $j \in [n]$, put $L_j(d) := L(\tilde f)$ for $f \in ©H(D_j)$ and $\tilde f := f$ on $D_j$ and $\tilde f := 0$ on $D_k$, $k ≠ j$. Then
	$$ L(f) = \sum_{j=1}^n L_j(f|_{D_j}), \qquad f \in ©H(G). $$

	By 1., for each $j \in [n]$, there are $\tilde r_j \in (0, r_j)$ and $λ_j \in ©H_0(®S \setminus \overline{U(z_j, \tilde r_j)})$ such that
	$$ L_j(f) = \frac{1}{2πi} \int_{φ_j} f(z)·λ_j(z) dz, \qquad f \in ©H(D_j), $$
	where $φ_j(t) := z_j + R_j e^{it}$, $t \in [0, 2π]$ for some $R_j \in (\tilde r_j, r_j)$.

	In addition, we have
	$$ L_j(z^n) = \frac{λ^{(n+1)}(∞)}{(n+1)!}, \qquad n \in ®N_0. $$

	If $f \in ©H(G)$, then $L(f) = \sum_{j=1}^n \frac{1}{2πi} \int_{φ_j} f(z)·λ_j(z) dz$.

	$\overset?\implies$ $L(f) = \frac{1}{2πi} \int_Γ f(z)·λ(z) dz$, where $Γ := \{φ_1, …, φ_n\}$ and $λ := \sum_{j=1}^n λ_j$.

	$?$ holds true because $\int_{φ_j} f(z) · λ_k(z) dz = 0$ for $k ≠ j$ by Cauchy ($f(z)·λ_k(z) \in ©H(D_j)$).

	We have $L(z^n) = \frac{λ^{(n+1)}(∞)}{(n+1)!}$, $n \in ®N_0$.
\end{poznamka}

\begin{poznamka}[Conclusion]
	($G = \bigcup_{j=0}^n D_j$.) $©H^*(G) = ©H_0(®S \setminus G)$. Indeed, $L \in ©H^*(G)$ iff there is a unique $λ \in ©H_0(®S \setminus G)$ such that last 2 equation hold true.
\end{poznamka}


\section{Hahn–Banach theorem}
\begin{lemma}
	Let $L: E \rightarrow ®C$ be linear. Then $L \in E^*$ iff there is a compact $K$ in $G$ and $M \in [0, +∞)$ such that $|L(f)| ≤ M·\|f\|_K$, $f \in E$.

	\begin{dukazin}
		„$\impliedby$“ from continuity of $\|·\|_K$. „$\implies$“: Since $U := L^{-1}(®D)$ is a neighbourhood of $¦o$ in $E$, there are a compact $K$ in $G$ and $ε > 0$ such that $U \supseteq U_{K, ε}(0) := \{f \in E | \|f\|_K < ε\}$. Let $f \in E$.

		1. Let $\|f\|_K ≠ 0$. Then
		$$ \left| L\(\frac{f}{\|f\|_K}·\frac{ε}{2}\)\right| < 1, $$
		hence $|L(f)| < \frac{2}{ε}\|f\|_K$. Put $M := \frac{2}{ε}$.

		2. Let $\|f\|_K = 0$. Then for any $n \in ®N$, we have $\|n f\|_K = 0$, so $|L(n·f)| < 1$, $|L(f)| < \frac{1}{n} \rightarrow 0$, $L(f) = 0$.
	\end{dukazin}
\end{lemma}

\begin{veta}[Hahn–Banach]
	Let $A$ be a linear subspace of $E$. Then

	\begin{itemize}
		\item if $L \in A^*$, then there is $\tilde L \in E^*$ such that $\tilde L |_A = L$;
		\item if $A$ is closed and $0 ≠ b \in E \setminus A$, then there is $L \in E^*$ such that $L(b) = 1$ and $L = 0$ on $A$;
		\item $\overline{A} = E$ iff ($L \in E^*$, $L = 0$ on $A$ $\implies$ $L = 0$ on E).
	\end{itemize}

	\begin{dukazin}
		„1.“ Use lemma and algebraic version of HB theorem.

		„2. + 3.“ can be proved as for Banach space.
	\end{dukazin}
\end{veta}

\begin{veta}[Runge (special)]
	Let $G \subset ®C$ be a finite union of pairwise open discs as in above "poznamka"s. Then for each $f \in ©H(G)$ there are polynomials $P_n$, $n \in ®N$, such that $P_n \overset{\text{loc.}}\rightrightarrows f$ on $G$.

	\begin{dukazin}
		Let $©P := \LO \{1, z, …\}$ be the space of polynomials. Then $©P \subset ©H(G)$. Let $L \in ©H^*(G)$ and $L = 0$ on $©P$. We know that there is $λ \in ©H_0(®S \setminus G)$ such that ? is valid. So, $λ^{(n)}(∞) = 0$, $n \in ®N_0$. By the uniqueness theorem, we get $λ ≡ 0$, so $L = 0$ on $©H(G)$ (because $L = 0$ fits and is uniquely determined by $λ$). By HB theorem, $\overline{©P} = ©H(G)$.
	\end{dukazin}
\end{veta}

% 30. 03. 2023 From notes of my colleague

\begin{veta}[Cauchy formula for compact]
	Let $G \subset ®C$ be open, $K \subset G$ compact. Then there is a cycle $Γ \subset G$, $K \subseteq \Int Γ \subseteq G$ and $\forall a \in \Int Γ: \ind_Γ a = 1$.

	In addition
	$$ \forall f \in ©H(G): \int_Γ f = 0 \land \forall a \in \Int Γ: f(a) = \frac{1}{2πi} \int_Γ \frac{f(z)}{z - a} dz. $$

	\begin{poznamkain}
		„In addition“ follows from the properties of $Γ$ and residue's theorem for cycles, but we prove it directly.
	\end{poznamkain}
\end{veta}

	\begin{dukaz}[Cauchy formula for compact]
		Choose $0 < δ < \frac{1}{2} \dist(K, ®C \setminus G)$, if $G \subsetneq ®C$, otherwise, if $G = ®C$, take $δ := 1$. For $m, n \in ®Z$ let $Q_{m, n}$ be the closed square with edges (parallel to the axes) with length $δ$, and such that $m δ + i n δ$ is the lower left vertex of $Q_{m, n}$.

		Denote $Q^* := \{Q_{n, m} | Q_{n, m} \cap K ≠ \O\}$, $U := \int(\bigcup Q^*)$. $Q^*$ is finite because of compactness of $K$. Of course, $K \subseteq U \subseteq \bigcup Q^* \subseteq G$ (by choice of $δ$).

		We understand $\partial Q_{m, n}$ as a positively oriented curve (piece-wise linear curve). Let $Γ$ be the system of all edges $Γ_1, …, Γ_k$ of squares of $Q^*$ when we omit those edges which occur twice ($±$). Of course, $U = \bigcup Q^* \setminus \im Γ$.

		a) Let $f \in ©H(G)$. Then $\int_Γ f := \sum_{j=1}^k \int_{Γ_j} f = \sum_{Q_{m, n} \subset Q^*} \int_{\partial Q_{m, n} f} = 0$.

		b) $Γ$ can be viewed as a cycle. In fact the edges $Γ_1, …, Γ_k$ form finitely many closed simple piece-wise linear curves.

		For $j \in [k]$ put $Γ_j =: [a_j, b_j]$.

		(*) „Every point $c \in ®C$ is the starting point of some edge of $Γ$ as many times as it is the ending point of some edge in $Γ$“:

		Take a polynomial $P$ such that $p(c) = 1$ and $p(a) = 0$, if $a ≠ c$ and $[a, b] \in Γ$ for some $b$. $p(b) = 0$, if $b ≠ c$ and $[a, b] \in Γ$ for some $a$. By a):
		$$ 0 = \int_Γ p' = \sum_{j=1}^k \int_{Γ_j} p' = \sum_{j=1}^k (p(b_j) - p(a_j)) = \sum_{j=1}^k p(b_1) - \sum_{j=1}^k p(a_j) = $$
		$$ = \# \text{ $c$ is the ending point} - \# \text{ $c$ is the starting point}. $$

		„$Γ$ can be viewed as a cycle“: Let $L$ be longest (one of the longest) simple piecewise linear curve consisting of edges of $Γ$ which begins with $Γ1$, i. e.,

		\begin{itemize}
			\item $L = [c_1, c_2, …, c_l] := [c_1, c_2] + [c_2, c_3] + … + [c_{l-1}, c_l]$;
			\item $Γ1 = [c_1, c_2]$;
			\item $c_i ≠ c_j$ for $i ≠ j$ (simple curve);
			\item $l$ is the biggest.
		\end{itemize}

		Since we have (*) there is an index $j \in [l - 2]$ such that $[c_l, c_j] \in Γ$ (otherwise we would have a longer curve).
		$$ L' := [c_j, c_{j+1}] + … + [c_{l-2}, c_l] + [c_l, c_j] \subseteq L $$
		$\implies$ $L'$ is simple closed piece-wise linear curve. The proper subset $Γ'$, which we get from $Γ_k$ by omitting the edges of $L'$ has again (*). We can process in this fashion for $Γ'$, by finitely many steps we get what we want.

		c) Let $f \in ©H(G)$ and $a \in U = \Int(\bigcup Q^*)$. c1) $a \in \Int\(\tilde Q\)$ for some $\tilde Q \in Q^*$. Then
		$$ \frac{1}{2πi} \int_Γ \frac{f(z)}{z - a} dz = \sum_{Q_{m, n} \in Q^*} \frac{1}{2πi} \int_{\partial Q_{m, n}} \frac{f(z)}{z - a} dz = f(a) $$
		(residue's theorem for $Q_{m, n} = \tilde Q$, Cauchy for $Q_{m, n} ≠ \tilde Q$).

		c2) $a \in \partial \tilde Q$, $\tilde Q \in Q^*$ ($a \notin \partial Q^* \implies a \notin \im(Γ)$). Take $a_j \in \Int \tilde Q$, $a_j \rightarrow a$. Then
		$$ \frac{1}{2πi} \int_Γ \frac{f(z)}{z - a_j} dt \rightarrow f(a) = \frac{1}{2πi} \int_Γ \frac{f(z)}{z - a} dz. $$

		d) „$U = \Int(Γ)$“: Let $a \in ®C \setminus(U \cup \im Γ)$ $\implies$ $a \in ®C \setminus \bigcup Q^*$. As in a), we show easily that $\ind_Γ a = 0$. If $a \in U$, then by c) we have that $\ind_p(1) = 1$.
	\end{dukaz}

\begin{veta}[Description of $©H^*(G)$]
	Let $G \subset ®C$ be open subset. Then $©H^*(G) \simeq ©H_0(®S \setminus G)$.

	In more detail, let $L \in ©H^*(G)$. Then there are a compact $K \subset G$ and $λ \in ©H_0(®S \setminus K)$ such that
	$$ L(f) = \frac{1}{2πi} \int_Γ f(z) λ(z) dz, \qquad f \in ©H(G), $$
	where $Γ$ is a cycle in $G \setminus K$ with $K \subset \Int Γ \subset G$ and $\forall z_0 \in \Int Γ: \ind_Γ z_0 = 1$.

	In addition, as an element of $©H_0(®S \setminus G)$, $λ$ is uniquely determined by
	$$ \frac{λ^{k + 1}(∞)}{(k+1)!} = L(z^k), k \in ®N_0, \frac{λ^{(k)}(z_0)}{k!} = - L\(\frac{1}{(z - z_0)^{k+1}}\), z_0 \in ®C \setminus G, k \in ®N_0. $$

	\begin{dukazin}[Step 1]
		Let $L \in ©H^*(G)$.

		Step 1: There are a compact $K \subset G$ and $L_1 \in (©C(K))^* =: ©C^*(K)$ such that $L(f) = L_1(f|_K), f \in ©H(G)$.

		We know that there are a compact $K \subseteq G$ and $C \in (0, +∞)$ such that $\forall f \in ©H(G): |L(f)| ≤ \|f\|_K · C$.

		By the Hahn–Banach theorem we can extend $L$ (from $©H^*(G)$ to $©C^*(G)$) to $\tilde L$ $\Leftrightarrow$ $\tilde L \in ©C^*(G)$ such that $\tilde L_2 |_©H(G) = L$ and $|L(f)| ≤ \|f\|_K · C$, $f \in ®C(G)$.

		For each $f \in ©C(K)$ put $L_1(f) := \tilde L_1(\tilde f)$, where $\tilde f \in ©C(G)$ and $\tilde f|_K = f$.

		Is definition of $L_1$ correct?

		i) by Tietze theorem: $f \in ©C(K)$ can be extended to $f \in ©C(G)$,
		$$ \forall f \in ©C(K)\ \exists \tilde f \in ©C(®C)\ (©C(G)): \tilde f|_K = f; $$

		ii) for any extension we want to get the same result. $\tilde f_1, \tilde f_2 \in ©C(G)$, $\tilde f_i|_U = f$, $i = 1, 2$.
		$$ \implies |\tilde L_1(\tilde f_1) - \tilde L_1(\tilde f_2)| = |\tilde L_1(\tilde f_1 - \tilde f_2)| ≤ C·\|\tilde f_1 - \tilde f_2\|_K = C\|f - f\|_K = 0. $$
	\end{dukazin}

% 06. 04. 2023

	\begin{poznamkain}[$©C^*(K)$]
		By the Riesz representation theorem, for each $L_1 \in ©C^*(K)$ there is a unique complex Borel measure $μ$ on $K$ such that
		$$ L_1(f) = \int_K f dμ, \qquad \forall f \in ©C(K). $$
	\end{poznamkain}

	\begin{dukazin}
		Step 2: By the Cauchy formula for compact, there is a cycle $Γ \subset G$ such that $K \subset \Int Γ \subset G$, $\forall a \in \Int Γ: \ind_Γ a = 1$ and we have, $\forall f \in ©H(G)$:
		$$ f(z_1) = \frac{1}{2πi} \int_Γ \frac{f(z_2) dz_2}{z_2 - z_1}, \qquad z_1 \in K. $$

		Denote
		$$ L_2(f) := \frac{1}{2πi} \int_Γ f(z_2) dz_2, f \in ©C(\<Γ\>), \qquad F(z_1, z_2) := \frac{f(z_2)}{z_2 - z_1}. $$
		Of course $L_2 \in ©C^*(\<Γ\>)$ and $f(z_1) = L_2(F(z_1, z_3))$, $z_1 \in K$.

		Step 3: For a given $f \in ©H(G)$,
		$$ L(f) = L_1(f(z_1)) = L_1(L_2(F(z_1, z_2))) \overset{\text{Fubini}} L_2(L_1(F(z_1, z_2))), $$
		hence
		$$ L(f) = \frac{1}{2πi} \int_Γ f(z_2)·λ(z_2) dz_2, $$
		where
		$$ λ(z_2) := L_1\(\frac{1}{z_2 - z_1}\), \qquad z_2 \in ®C \setminus K. $$

		Step 4: $λ \in ©H_0(®S \setminus K)$ satisfies „in addition“: Let $U(∞, ε) \subset ®S \setminus K$. For $u \in P(0, ε)$, we have
		$$ λ\(\frac{1}{u}\) = L_1(\frac{u}{1 - u·z_1}) = L_1\(\sum_{k=0}^∞ z_1^k u^{k+1}\) = \sum_{k=0}^∞ L_1(z_1^k) u^{k+1}, $$
		hence $λ(∞) = 0$ and
		$$ \forall k \in ®N_0: \frac{λ^{(k+1)}(∞)}{(k+1)!} = L_1\(z_1^k\). $$

		Let $U(z_0, ε) \subset ®C \setminus K$. Then $\forall z_2 \in U(z_0, ε)$:
		$$ λ(z_2) = L_1\(\frac{1}{z_2 - z_1}\) = -L_1\(\sum_{k=0}^∞ \frac{(z_2 - z_0)^k}{(z_1 - z_0)^{k+1}}\) = - \sum_{k=0}^∞ L_1 \(\frac{1}{(z_1 - z_0)^{k+1}}\)(z_2 - z_0)^k; $$
		$$ \forall z_1 \in K: \frac{1}{z_2 - z_1} = \frac{1}{(z_2 - z_0) - (z_1 - z_0)} = -\frac{1}{z_1 - z_0}·\frac{1}{1 - \frac{z_2 - z_0}{z_1 - z_0}} = -\sum_{k=0}^∞ \frac{(z_2 - z_0)^k}{(z_1 - z_0)^{k+1}}. $$
		Hence $\frac{λ^{(k)}(z_0)}{k!} = -L_1\(\frac{1}{(z_1 - z_0)^{k+1}}\)$, $k \in ®N_0$.

		Step 5: As an element of $©H_0(®S \setminus G)$, $λ$ is uniquely determined by „in addition“. (Proof below.)
	\end{dukazin}
\end{veta}

\begin{lemma}
	Let $G \subset ®C$ be open and $K$ be a compact in $G$. There is a compact $K_1$ such that $K \subset K_1 \subset G$ and each component of $®S \setminus K_1$ contains some component of $®S \setminus G$.

	\begin{dukazin}
		Take $n \in ®N$ such that $K_1 := \{z \in G | \dist(z, ®C \setminus G) ≥ \frac{1}{n}\} \cap \overline{U(0, n)} \supset K$. In addition, we have
		$$ ®S \setminus K_1 = \bigcup_{z_0 \in ®S \setminus G} U(z_0, \frac{1}{n}). $$

		Let $V$ be a component of $®S \setminus K_1$. There is $z_0 \in ®S \setminus G$ such that $U\(z_0, \frac{1}{n}\) \subset V$. If $W$ is a component of $®S \setminus G$ containing $z_0$, then $W \subset V$.
	\end{dukazin}
\end{lemma}

\begin{dukaz}[Step 5]
	Let $λ_1, λ_2 \in ©H_0(®S \setminus G)$ satisfying „in addition“. Then there is a compact $K \subset G$ such that $λ_1, λ_2 \in ©H_0(®S \setminus K)$.

	By the previous lemma, WLOG we assume that each component $V$ of $®S \setminus K$ intersect $®S \setminus G$. We show $λ_1 = λ_2$ on $®S \setminus K$.

	Let $V$ be any component of $®S \setminus K$ and $z_0 \in V \cap (®S \setminus G) ≠ 0$. By „in addition“ we have $λ_1^{(k)}(z_0) = λ_2^{(k)}(z_0)$ $\forall k \in ®N_0$. By the uniqueness theorem $λ_1 = λ_2$ on the domain $B$, so $λ_1 = λ_2$ on $®S \setminus K$.
\end{dukaz}

\begin{lemma}[Fubini]
	Let $K_1, K_2 \subset ®C$ be compact, $L_j \in ©C^*(K_j)$ for $j = 1, 2$ and $F \in ©C(K_1 \times K_2)$. Then we have
	$$ L_1(L_2(F(z_1, z_2))) = L_2(L_1(F(z_1, z_2))). $$

	\begin{dukazin}[Sketch]
		Obviously it holds true for the functions of the following form: $F(z_1, z_2) = f(z_1)·g(z_2)$ for $f \in ©C(K_1)$, $G \in ®C(K_2)$.

		Now we can use the Stone–Weierstrass theorem which show that the linear span of the functions of this form is dense in $©C(K_1 \times K_2)$.
	\end{dukazin}
\end{lemma}

\section{Runge's theorem}
\begin{definice}[Notation]
	Let $E \subset ®C$ and $m: E \rightarrow ®N \cup \{∞\}$. We call $m(e)$ the multiplicity of $e \in E$. We say that $(E, m)$ has a limit point $e \in ®S$ if $e$ is a limit point of $E$, or $e \in E$ with $m(e) = ∞$.

	Denote by $©F(E, m)$ system of functions which consists of
	\begin{itemize}
		\item $\frac{1}{z - e}$ if $e \in E \cap ®C$, $m(e) < ∞$;
		\item $\frac{1}{(z - e)^k}$, $k \in ®N$ if $e \in E \cap ®C$, $m(e) = ∞$;
		\item $z^k$, $k \in ®N_0$ if $∞ \in E$, $m(∞) = ∞$.
	\end{itemize}
\end{definice}

\begin{veta}[Runge]
	Let $G \subset ®C$ be open, $E \subset ®S \setminus G$ and $m: E \rightarrow N \cup \{∞\}$. If $(E, m)$ has a limit point in every component of $®S \setminus G$, then the linear span of $®F(E, m)$ is dense in $©H(G)$.

	\begin{dukazin}
		We shall use Hahn–Banach theorem. Let $L \in ©H^*(G)$ and $L = 0$ on $®F(E, m)$. We need to show $L = 0$ on $©H(G)$. Let $λ \in ©H_0(®S \setminus G)$ which represents $L$ in the sense of theorem describing $©H^*(G)$.

		If $e \in E \cap ®C$, $m(e) < ∞$, then $λ(e) = -L\(\frac{1}{z - e}\) = 0$. If $e \in E \cap ®C$, $m(e) = ∞$, then $\frac{λ^{(k)}(e)}{k!} = -L\(\frac{1}{(z - e)^k}\) = 0$ $\forall k \in ®N_0$. If $∞ \in E$, $m(∞) = ∞$, then $\frac{λ^{k + 1}(∞)}{(k+1)!} = L(z^k) = 0$ $\forall k \in ®N_0$.

		We show that $λ = 0$ in $©H_0(®S \setminus G)$. There is a compact $K \subset G$ such that $λ \in ©H_0(®S \setminus K)$ and every component of $®S \setminus K$ contains some component of $®S \setminus G$.

		Let $V$ be any component of $®S \setminus K$. Then $V$ is domain and $V$ contains a limit point $e$ of $(E, m)$. By the uniqueness theorem, we get $λ = 0$ on $V$, so on $®S \setminus K$.
	\end{dukazin}
\end{veta}

\begin{veta}[Runge, classical version]
	Let $G \subset ®C$ be open and $f \in ©H(G)$. Then there are rational functions $R_n$, $n \in ®N$ with poles outside $G$ such that $R_n \overset{\text{Loc.}} f$ on $G$.

	If, in addition, $®S \setminus G$ is connected, then there are polynomials $P_n$, $n \in ®N$, such that $P_n \overset{\text{Loc.}} f$ on $G$.

	\begin{dukazin}
		„Second part“: Let $E = \{∞\}$ and put $m(∞) = ∞$. Then
		$$ ®F(E, m) = \{1, z, …, z^k, …\} $$
		and by the previous theorem, the polynomials are dense in $©H(G)$.

		„First part“: Let $E \subset ®S \setminus G$ containing at least one point of every component of $®S \setminus G$. Put $m = ∞$ on $E$. Then $\LO(©F(E, m))$ is dense in $©H(G)$ and it is a subspace of rational functions with poles outside $G$.
	\end{dukazin}
\end{veta}

% 13. 04. 2023

\begin{dusledek}[Cauchy's theorem for simply connected domains]
	Let $G \subset ®C$ be open a nd $®S \setminus G$ be connected. If $f \in ©H(G)$ and $φ$ is a closed curve in $G$, then $\int_φ f = 0$.

	\begin{dukazin}
		By Runge, there are polynomials $P_n$ such that $P_n \overset{\text{Loc.}}\rightrightarrows f$ on $G$. Then ($P_n$ has a primitive function in ®C) $0 = \int_φ P_n \rightarrow \int_φ f$.
	\end{dukazin}
\end{dusledek}

\begin{dusledek}[Cauchy's theorem for cycles]
	Let $G \subset ®C$ be open and $Γ$ be a cycle in $G$ (i.e., $\<Γ\> \subset G$). Then $\(\forall f \in ©H(G): \int_Γ f = 0\) \Leftrightarrow \Int Γ \subset G$.

	\begin{dukazin}
		„$\implies$“: If $z_0 \in ®C \setminus G$, then $f(z) := \frac{1}{z - z_0} \in ©H(G)$ and $\ind_Γ z_0 = \frac{1}{2π i} \int_Γ f = 0$.

		„$\impliedby$“: Let $f \in ©H(G)$. By Runge, there are rational $R_n$ with poles outside $G$ such that $R_n \overset{\text{Loc.}}\rightrightarrows f$. Then $0 = \int_Γ R_n \rightarrow \int_Γ f$. (First equality is from: Let $Γ = \{φ_1, …, φ_m\}$, where $φ_j$ are closed curves in $G$. Then $\int_Γ R_n = \sum_{j=1}^m \int_{φ_j} R_n = \sum_{j=1}^m 2πi \sum_{R_n(s) = ∞} \res_s R_n \ind_{φ_j} s = 2πi·\sum_{R_n(s) = ∞} \res_s R_n · \ind_Γ s$, but $s$ lies outside of $G$, so it is equal to 0.)
	\end{dukazin}
\end{dusledek}

\begin{veta}[Runge, for compacts]
	Let $K$ be a compact in ®C and let $S \subset ®S \setminus K$ contain at least one point of any component of $®S \setminus K$. Let $f$ be a holomorphic function on $K$. Then there are rational functions $R_n$ with poles in $S$ such that $R_n \rightrightarrows f$ on $K$.

	\begin{poznamkain}[Technique: pushing poles]
		Each rational function $R$ can be uniquely expressed in the form (rational function has $n \in ®N$ poles, and we will write the principal part of Laurent expansion around the pole $z_k$):
		$$ R(z) = \sum_{k=1}^n \sum_{j=1}^{n_k} \frac{A_j^k}{(z - z_k)^j} + C_0 + C_1z + … + C_mz^m, $$
		where $n, m, n_k \in ®N$, $z_k \in ®C$ and $A^k_{n_k} ≠ 0$, $C_m ≠ 0$. Then $z_k$ is a pole of $R$ of multiplicity $n_k$ and $∞$ is a pole of $R$ of multiplicity $m$. A rational function $R$ is a polynomial iff $R$ has a pole at most at $∞$.
		
		Notation: Let $K$ be a compact in $®C$, $U \subset ®S$ and $U \cap K = \O$. Put $B(K, U) = \overline{\{R|_K | R \text{ is rational with poles in } U\}}^{©C(K)}$. ($B(K, U)$ is a closed subalgebra of $©C(K)$.)

		Theorem (pushing poles): Let $K$ be a compact in ®C, $U \subset ®S$ be a domain, $K \cap U = \O$ and $z_0 \in U$. If $R$ is rational function with poles in $U$, then $R \in R(K, \{z_0\})$.

		Corollary: By theorem, we have $B(K, U) = B(K, z_0)$.
	\end{poznamkain}

	\begin{poznamkain}[Technique: pushing poles]
		Proof: Put $V := \{ξ \in U | \frac{1}{z - ξ} \in B(K, x_0), \text{ for $ξ \in ®C$ and } z \in B(K, z_0) \text{for $ξ = ∞$}\}$. Of course $B(K, z_0) = B(K, V)$. Indeed, if $ξ \in V$, then $\frac{1}{(z - ξ)^k} \in B(K, z_0)$, for $ξ \in ®C$ and $k \in ®N$, and $z^k \in B(K, z_0)$ for $ξ = ∞$, $k \in ®N$.

		Then each rational $R$ with poles in $V$ is contained in $B(K, z_0)$. Hence $B(K, V) \subset B(K, z_0)$. Since $z_0 \in V$, we have $B(K, z_0) \subset B(K, V)$.

		„$V$ is closed in $U$“: Let $ξ_n \in V$, $ξ_n \rightarrow ξ_0$ and $ξ_0 \in U$. We need to show that $ξ_0 \in V$. WLOG $\forall n \in ®N: ξ_n \in ®C$.

		„$ξ_0 \in ®C$“. Then put $δ := \dist(ξ_0, K) > 0$. Choose $n_0 \in ®N$ such that $\dist(ξ_n, K) ≥ \frac{δ}{2}$ for $n > n_0$. Then $\frac{1}{z - ξ_n} \rightrightarrows \frac{1}{z - ξ_0}$, $z \in K$,
		$$ \impliedby \left|\frac{1}{z - ξ_n} - \frac{1}{z - ξ_0}\right| = \frac{|ξ_n - ξ_0|}{|z - ξ_n|·|z - ξ_0|} ≤ \frac{2}{δ^2}·|ξ_n - ξ_0| \rightarrow 0, $$
		if $n > n_0$ and $z \in K$. Hence $\frac{1}{z - ξ_n} \in B(K, z_0)$, so $ξ_0 \in V$.

		„$ξ_0 = ∞$“. Then
		$$ \frac{ξ_n z}{ξ_n - z} = -ξ_n \(\frac{ξ_n}{z - ξ_n} + 1\) \in B(K, z_0). $$
		Take $C > 0$ with $\forall z \in K: |z| ≤ C$. Take $n_0 \in ®N$ such that $\forall n > n_0: |ξ_n| > C$. Then $\forall z \in K: \frac{ξ_n z}{ξ_n - z} \rightrightarrows z$, because
		$$ \left|\frac{ξ_n z}{ξ_n - z} - z\right| = \frac{|z|^2}{|ξ_n - z|} ≤ \frac{C^2}{|ξ_n| - C} \rightarrow 0. $$
		if $n > n_0$ and $z \in K$. Hence $z \in B(K, z_0)$, so $∞ \in V$.

		„$V$ is open (so $V = U$)“: Let $ξ_0 \in V$.

		„$ξ \in ®C$“: Put $δ := \dist(ξ_0, K) > 0$. Let $ξ \in U(ξ \in U(ξ_0, δ/2))$. Then
		$$ \frac{1}{z - ξ} = \frac{1}{(z - ξ_0) - (ξ - ξ_0)} = \frac{1}{z - ξ_0}·\frac{1}{1 - \frac{ξ - ξ_0}{z - ξ_0}} = \sum_{k=0}^∞ \frac{(ξ - ξ_0)^k}{(z - ξ_0)^{k + 1}} $$
		converges uniformly for $z \in K$ because $\left|\frac{(ξ - ξ_0)^k}{(z - ξ_0)^{k + 1}}\right| ≤ \frac{\(\frac{δ}{2}\)^k}{δ^{k+1}} = \frac{1}{δ·2^k}$, for $z \in K$ and $k \in ®N_0$. Hence $\frac{1}{z - ξ} \in B(K, ξ_0) \subset B(K, V) = B(K, z_0)$. So $U(ξ_0, δ/2) \subset V$.

		„$ξ_0 = ∞$“: Take $C > 0$ with $\forall z \in K: |z| ≤ C$. Let $ξ \in ®C$ with $|ξ| > 2C$. Then $\frac{1}{z - ξ} = -\frac{1}{ξ}·\frac{1}{1 - \frac{z}{ξ}} = -\sum_{k=0}^∞ \frac{z^k}{ξ^{k+1}}$ converges uniformly for $z \in K$, because $|z^k / ξ^{k+1}| ≤ C^k / (2C)^{k+1} = \frac{1}{C·2^{k+1}}$ for $z \in K$ and $k \in ®N_0$. Hence $\frac{1}{z - ξ} \in B(K, ∞) \subset B(K, z_0)$. So $U\(∞, \frac{1}{2C}\) \subset V$.
	\end{poznamkain}

	\begin{dukazin}
		Let $f$ be a holomorphic function on an open set $G \supset K$. Using Runge's theorem for "open sets", there are rational functions $\tilde R_n$ with poles outside $G$ such that $\tilde R_n \rightrightarrows f$ on $K$.

		„$\tilde R_n \in B(K, S)$“: All poles of $\tilde R_n$ are contained in a finitely many components $C_1, …, C_k$ of $®S \setminus K$. Express $\tilde R_n = \tilde Q_1 + … + \tilde Q_k$, where $\tilde Q_j$ is a rational function with poles in the domain $C_j$. For $j \in [k]$ take $s_j \in S \cap C_j$. By pushing poles we have $\tilde Q_j \in B(K, s_j)$. For given $ε > 0$ and $j \in [k]$, there is a rational function $Q_j$ with a pole at $s_j$ such that $|\tilde Q_j - Q_j| ≤ \frac{ε}{k}$ on $K$. Put $R_n := Q_1 + … + Q_k \in B(K, S)$. Then $|R_n - \tilde R_n| ≤ ε$ on $K$. Hence $\tilde R_n \in B(K, S)$.
	\end{dukazin}
\end{veta}

% 20. 04. 2023

\section{Characterization of simple connectedness}
\begin{tvrzeni}
	Let $G \subset ®C$ be open. FSAE:

	\begin{itemize}
		\item[SC1] If $φ$ is closed (regular) curve in $G$, then $\Int φ \subset G$;
		\item[SC2] $®S \setminus G$ is connected;
		\item[SC3] $\forall f \in ©H(G)$ $\exists$ polynomials $P_n$: $P_n \overset{\text{loc.}}\rightrightarrows f$ on $G$;
		\item[SC4] $\forall f \in ©H(G): \int_φ f = 0$ for any closed regular curve $φ$ in $G$;
		\item[SC5] $\forall f \in ©H(G)\ \exists F \in ©H(G): F' = f$ on $G$;
		\item[SC6] $\forall f \in ©H(G)$, $f ≠ 0$ on $G$, $\exists g \in ©H(G)$: $f = e^g$ on $G$;
		\item[SC7] $\forall f \in ©H(G)$, $f≠0$ on $G$, $\exists h \in ©H(G)$: $h^2 = f$ on $G$.
	\end{itemize}

	\begin{dukazin}[SC1 $\implies$ SC2]
		Assume that $®S \setminus G$ is not connected. Then there are disjoint closed sets $\O ≠ K, L \subset ®S$ such that $®S \setminus G = K \cup L$. WLOG $∞ \notin K$. Then $K$ is compact in ®C, $G_0 := G \cup K$ is an open set in $®C$ and, by theorem Cauchy formula for compact we know, there is cycle $Γ$ in $G_0$ such that $K \subset \Int Γ \subset G_0$.

		Let $z_0 \in K$. Since $\ind_Γ z_0 ≠ 0$, there is $φ \in Γ$ with $\ind_φ z_0 ≠ 0$. Of course, $z_0 \in (®C \setminus G) \cap \Int φ$.
	\end{dukazin}

	\begin{dukazin}[SC2 $\implies$ SC3, SC3 $\implies$ SC4]
		See Runge's theorem (classical version).

		See the proof of the Cauchy theorem for simply connected domains.
	\end{dukazin}

	\begin{dukazin}[SC4 $\Leftrightarrow$ SC5, SC5 $\implies$ SC6, SC6 $\implies$ SC7]
		We know from introduction to complex analysis.

		See the proof of proposition about non-zero holomorphic function.

		Put $h := e^{\frac{1}{2} g}$.
	\end{dukazin}
\end{tvrzeni}

\subsection{The right topological definition}
\begin{definice}[Loop]
	Let $G \subset ®C$ be open. WLOG: We assume that all curves are defined on $[0, 1]$ (otherwise we can make linear reparametrization). A continuous closed curve $φ: [0, 1] \rightarrow G$ is called a loop in $G$.
\end{definice}

\begin{definice}[Homotopic loops]
	We say that two loops $φ$ and $ψ$ are homotopic (in $G$) provided that there is a continuous map $H: [0, 1]\times[0, 1] \rightarrow G$ such that $φ_0(t) = φ(t)$ and $φ_1(t) = ψ(t)$ and $φ_s(0) = φ_s(1)$, where $φ_s(t) := H(s, t)$.
\end{definice}

\begin{tvrzeni}[Continuation of the previous tvrzeni]
	SC8: Every loop $φ$ in $G$ is homotopic in $G$ to a constant loop.

	\begin{dukazin}[SC7 $\implies$ SC8]
		Let $φ$ be a loop in $G$. Let $G_0$ be a component of $G$ containing $\<φ\>$. If $G_0 = ®C$, then (all star-like domain has the property SC8) $φ$ is homotopic to a constant loop. So assume $G_0 \subsetneq ®C$. Then $\O ≠ G_0 \subsetneq ®C$ is a domain with the property SC7. By Riemann (next) theorem, $G_0$ is homeomorphic to ®D and hence $G_0$ has SC8 property (all star-like domains have SC8 and homomorphism preserve homotopic loops).
	\end{dukazin}

	\begin{dukazin}[SC8 $\implies$ SC1]
		Of course, every constant loop $ψ$ has $\Int ψ = \O$. Hence this implication follows from the theorem after Riemann theorem.
	\end{dukazin}
\end{tvrzeni}

\begin{veta}[Riemann]
	Let $\O ≠ G_0 \subsetneq ®C$ be a domain with SC7. Then there is a one-to-one holomorphic function $h: G_0 \overset{\text{onto}}\rightarrow ®D := U(0, 1)$.
\end{veta}

\begin{poznamka}[Recall]
	Proposition: Let $φ_1, φ_2: [0, 1] \rightarrow ®C$ be closed (regular) curves and $z_0 \in ®C \setminus (\<φ_1\> \cup \<φ_2\>)$. If $\forall t \in [0, 1]$:
	$$ |φ_1(t) - φ_2(t)| < |φ_1(t) - z_0|, $$
	then $\ind_{φ_1} z_0 = \ind_{φ_2} z_0$.
\end{poznamka}

\begin{veta}
	Let $φ, ψ$ be two loops homotopic in an open set $G \subset ®C$. Then $\ind_φ z_0 = \ind_ψ z_0$ $\forall z_0 \in ®C \setminus G$.

	\begin{definicein}[Index of (non-regular) loop]
		Let $φ: [0, 1] \rightarrow ®C$ be a loop and $z_0 \in ®C \setminus \<φ\>$. There are regular closed curves $φ_n: [0, 1] \rightarrow ®C$ such that $φ_n \rightrightarrows φ$. Indeed using the uniform continuity of $φ$, $φ$ can be uniformly approximated by piecewise linear closed curves with vertices on $φ$ given by sufficiently fine partitions of $[0, 1]$.

		Define $\ind_φ z_0 := \lim_{n \rightarrow ∞} \ind_{φ_n} z_0$.

		By recalled proposition, the definition is correct because there is $n_0 \in ®N$ such that $\ind_{φ_n} z_0$, $n ≥ n_0$, are constant and $\ind_φ z_0$ does not depend of the choice of $\{φ_n\}$.
	\end{definicein}

	\begin{poznamkain}
		„Proposition from recall holds true for (non-regular) loops $φ_1, φ_2$.“: Indeed, let loops $φ_1$ and $φ_2$ satisfy
		$$ \forall t \in [0, 1]: |φ_1(t) - φ_2(t)| < |φ_1(t) - z_0|. $$
		Then, by definition, there are approximations $\tildeφ_1, \tildeφ_2$ which are regular, satisfy the assumptions of proposition from recall and $\ind_{φ_j} z_0 = \ind_{\tildeφ_j}$, $j = 1, 2$.
	\end{poznamkain}

	\begin{dukazin}
		Let $H: [0, 1] \times [0, 1] \rightarrow G$ be continuous, $φ_0 = φ$, $φ_1 = ψ$ and $φ_s(0) = φ_s(1)$, $\forall s \in [0, 1]$, where $φ_s(t) = H(s, t)$. Put $ε := \dist(z_0, H([0, 1]^2)) > 0$ ($H([0, 1]^2)$ is compact).

		Since $H$ is uniformly continuous, there is $n \in ®N$ such that for each $k \in [n - 1]$ and $t \in [0, 1]$ we have
		$$ \left| φ_{\frac{k}{n}}(t) - φ_{\frac{k+1}{n}}(t)\right| = \left|H(\frac{k}{n}, t) - H(\frac{k+1}{n}, t)\right| < ε. $$
		In particular, $φ_{\frac{k}{n}}$ and $φ_{\frac{k+1}{n}}$ satisfy the assumptions of proposition. Hence
		$$ \ind_{φ_0} z_0 = \ind_{φ_{\frac{1}{n}}} z_0 = \ind_{φ_{\frac{2}{n}}} z_0 = … = \ind_{φ_1} z_0. $$
	\end{dukazin}
\end{veta}

% 27. 04. 2023

\begin{veta}[The Schwarz lemma]
	Let $f \in ©H(®D)$, $f(®D) \subset ®D$ and $f(0) = 0$. Then $|f(z)| ≤ |z|$, $z \in ®D$ and $|f'(0)| ≤ 1$. If the equality occurs in first inequality got some $z \in ®D \setminus \{0\}$ or in second inequality, then $f$ is a rotation, i.e., $f(z) = λ z$, $z \in ®D$ for some $λ \in ®C, |λ| = 1$.

	\begin{dukazin}
		Put $g(z) := \frac{f(z)}{z}$, for $z \in ®D \setminus \{0\}$, and $f'(0)$, $z = 0$. Then $g \in ©H(®D)$. Let $0 < r < 1$. Then $|g(r)| ≤ \frac{1}{r}$, $|z| = r$. By the maximum modulus theorem, we get $|g(z)| ≤ \frac{1}{z}$, $|z| ≤ r$.

		Let $z \in ®D$. Then, for $r$ close enough to $1-$, we have this inequality and letting $r \rightarrow 1-$ we get $|g(z)| ≤ 1$. If $|g(z)| = 1$ for some $z \in ®D$, then by maximum modulus theorem $g$ is constant on ®D.
	\end{dukazin}
\end{veta}

\begin{lemma}
	For $α \in ®D$ put $φ_α(z) := \frac{z - α}{1 - \overline{α}z}$. Then

	\begin{itemize}
		\item $φ_α \in ©H\(®C \setminus \{\frac{1}{\overline{α}}\}\)$, $φ_α$ is one-to-one, $φ_α(®D) = ®D$, $φ_α(®T) = ®T := \{z \in ®C | |z| = 1\}$;
		\item $(φ_α)_{-1} = φ_{-α}$;
		\item $φ_α(α) = 0$, $φ_α'(α) = \frac{1}{1 - |α|^2}$, $φ_α'(0) = 1 - |α|^2$.
	\end{itemize}

	\begin{dukazin}[ii)]
		$$ w = \frac{z - α}{1 - \overline{α}z}: w - \overline{α}wz = z - α, \qquad w + α = z·(1 + \overline{α}w), \qquad z = \frac{w + α}{1 + \overline{α}w} = φ_{-α}(w). $$
	\end{dukazin}

	\begin{dukazin}[i)]
		If $z \in ®T$, then
		$$ \left|\frac{z - α}{1 - \overline{α}z}\right| = \frac{|z - α|}{|\overline{z - α}|·|z|} = 1. $$
		Hence $φ_α(®T) \subseteq ®T$. The same is true for $(φ_α)_{-1} = φ_{-α}$, so $φ_α(®T) = ®T$. By the fact $φ_α(®T) = ®T$ and maximum modulus theorem, we get $φ_α(®D), φ_{-α}(®D) \subset ®D$, so $φ_α(®D) = ®D$.
	\end{dukazin}

	\begin{dukazin}[iii)]
		$$ φ_α'(α) = \lim_{z \rightarrow α} \frac{φ_α(z)}{z - α} = \frac{1}{1 - |α|^2}. $$
		$$ φ_α'(0) = \left.\frac{1 - \overline{α}z + (z - α)\overline{α}}{(1 - \overline{α}z)^2}\right|_{z = 0} = 1 - |α|^2. $$
	\end{dukazin}
\end{lemma}

\begin{veta}[Conformal transformations of ®D]
	A function $f$ is one-to-one holomorphic map of ®D onto ®D, iff there are $θ \in ®R$ and $α \in ®D$ such that
	$$ f(z) = e^{iθ}\frac{z - α}{1 - \overline{α}z} (= rot_θ · φ_α), \qquad z \in ®D. $$

	\begin{dukazin}
		„$\impliedby$“: by the previous lemma. „$\implies$“: Let $α \in ®D$ and $f(α) = 0$. Then $g := f \circ φ_{-α}$ is a one-to-one holomorphic map of ®D onto ®D and $g(0) = 0$. By the Schwarz lemma, for $z \in ®D$, $|g(z)| ≤ |z|$, $|g_{-1}(z)| ≤ |z|$, so $|g(z)| = |z|$. By Schwarz lemma, $g$ is a rotation.
	\end{dukazin}
\end{veta}

\begin{lemma}[Schwarz–Pick]
	Let $F \in ©H(®D)$, $F(®D) \subset ®D$ and $F(α) = β$. Then $|F'(α)| ≤ \frac{1 - |β|^2}{1 - |α|^2}$. If the equality occurs, then $F(z) = φ_{-b}(λ φ_α(z))$, $z \in ®D$, for $λ \in ®C$, $|λ| = 1$.

	In particular $F'(0) < 1$ unless $F$ is a rotation.

	\begin{dukazin}
		Use the Schwarz lemma for $f := φ_β ∘ F ∘ φ_{-α}$. Then $|f'(0)| ≤ 1$ and
		$$ f'(0) = φ_β'(β)∘F'(α)∘φ_{-α}'(0) = \frac{1}{1 - |β|^2} ∘ F'(α) ∘ (1 - |α|^2) $$
		If $α = 0 = β$ and $F$ is not rotation, then $|F'(0)| < 1$.
	\end{dukazin}
\end{lemma}

\begin{veta}[Riemann]
	Let $\O ≠ G \subsetneq ®C$ be a simply connected domain. Then there is a one-to-one holomorphic map $f: G \overset{\text{onto}}\rightarrow ®D$.

	\begin{poznamkain}
		By this, we finish our proof that conditions (SC1)-(SC8) are equivalent with each other. (In proof we use SC7.)
	\end{poznamkain}

	\begin{dukazin}
		Let $\O ≠ G \subsetneq ®C$ be a domain with (SC7). Take a point $z_0 \in G$. Denote by $Σ$ the set of all one-to-one holomorphic maps $ψ: G \rightarrow ®D$. Then we have (we will prove it later)
		\begin{itemize}
			\item $Σ ≠ \O$;
			\item If $ψ \in Σ$ and $ψ(G) ≠ ®D$, then there is $\tilde ψ \in Σ$ such that $|\tilde ψ'(z_0)| > |ψ'(z_0)|$.
		\end{itemize}

		Put $η := \sup_{ψ \in Σ} |ψ'(z_0)|$. Take $ψ \in Σ ≠ \O$. Since $ψ$ is one-to-one we have $ψ'(z_0) ≠ 0$ and hence $η > 0$.

		By the definition of $η$, there are $ψ_n \in Σ$, $n \in ®N$, such that $|ψ_n'(z_0)| \rightarrow η$. Since $φ_n$, $n \in ®N$, are uniformly bounded, by the Montol theorem, there is a subsequence $\{ψ_{n_k}\}$ such that $ψ_{n_k} \overset{\text{loc.}}\rightrightarrows f$ on $G$. By the Weierstrass theorem, $f \in ©H(G)$ and $|f'(z_0)| = η > 0$ thus $f$ is not constant (but it is limit of one-to-one functions), so by Hurwitz theorem $f$ is one-to-one.

		Of course, $f(G) \subset \overline{®D}$ but by openness of $f$ we have $f(G) \subset ®D$. Hence $f \in Σ$ and $f(G) = ®D$.
	\end{dukazin}

	\begin{dukazin}[First property of $Σ$]
		Let $w_0 \in ®C \setminus G$. Then by (SC7), there is $φ \in ©H(G)$ such that $z - w_0 = φ^2(z)$, $z \in G$. „If $φ(z_1) = ± φ(z_2)$, then $z_1 = z_2$.“: Indeed, $z_1 - w_0 = φ^2(z_1) = φ^2(z_2) = z_2 - w_0$.

		By this, $φ$ is one-to-one and $0 ≠ w \in φ(G) \implies -w \notin φ(G)$. Since $\O ≠ φ(G)$ is open, there is $0 \notin U(a, r) \subset φ(G)$. By previous implication, we have $U(-a, r) \cap φ(G) = \O$, i.e., $|φ(z) + a| ≥ r$, $\forall z \in G$.

		Put $ψ := \frac{r}{2(φ(z) + a)}, z \in G$. Then $|ψ| ≤ \frac{1}{2}$ on $G$, so $ψ \in Σ$.
	\end{dukazin}

	\begin{dukazin}[Second property of $Σ$]
		Let $ψ \in Σ$ and $α \in ®D \setminus ψ(G)$. Consider the map $φ_α(z) := \frac{z - α}{1 - \overline{α}z}$, $z \in ®D$. Then $φ_α ∘ ψ \in Σ$ and $φ_α ∘ ψ ≠ 0$ on $G$.

		By (SC7), there is $g \in ©H(G)$ such that $φ_α ∘ ψ = g^2$ on $G$. Then $g$ is one-to-one, because
		$$ g(z_1) = g(z_2) \implies g^2(z_1) = g^2(z_2) \implies φ_α ∘ ψ(z_1) = φ_α ∘ ψ(z_2) \implies z_1 = z_2. $$
		Hence $g \in Σ$. If $β := g(z_0)$, then put $\tilde ψ := φ$. Of course, $\tilde ψ \in Σ$ and $\tilde ψ(z_0) = 0$.

		Denoting $s(w) := w^2$, $w \in ®C$, we have that $Ψ = (φ_{-α} ∘ s ∘ φ_{-β}) ∘ z\tilde ψ = F ∘ \tilde ψ$, where $F := φ_{-α} ∘ s ∘ φ_{-β}$. We have $F \in ©H(®D)$, $F(®D) \subset ®D$ and $F$ is not a rotation (because $F$ is not one-to-one). By the Schwarz–Pick lemma, we have $|F'(0)| < 1$. Since $ψ'(z_0) = F'(0)·\tilde Ψ'(z_0)$, we have $0 < |ψ'(z_0)| < |\tilde ψ'(z_0)|$.
	\end{dukazin}
\end{veta}

\begin{definice}
	Let $G \subset ®S$ be open. We say that $f: G \rightarrow ®S$ is a conformal map if $f$ is one-to-one meromorphic on $G$.
\end{definice}

\begin{definice}
	Let $Ω, G \subset ®S$ be open. We say that $G$ and $Ω$ are conformally equivalent (we write $G \sim Ω$) if there is a conformal map $f: G \overset{\text{onto}} Ω$.
\end{definice}

\begin{priklad}
	Show that there are just 4 classes of conform equivalent simply connected domains in ®S, namely:
	$$ \O, \qquad ®S, \qquad [®C] := \{®S \setminus \{z_0\} | z_0 \in ®S\}, \qquad [®D]. $$
\end{priklad}

% 04. 05. 2023

\section{Preservation of angles}

\begin{definice}
	For $z \in ®C \setminus \{0\}$, put $A(z) := \frac{z}{|z|}$.
\end{definice}

\begin{definice}[Map preserving angles]
	Let $G \subseteq ®C$ be open, $f: G \rightarrow ®C$, $z_0 \in G$ have $P(z_0) \subset G$ such that $f(z) ≠ f(z_0)$ $\forall z \in P(z_0)$. Then we say that $f$ preserves angles (with orientation) at $z_0$ if $\forall θ \in ®R$:
	$$ \lim_{r \rightarrow 0_+} e^{-k*θ}·A[f(z_0 + r e^{iθ}) - f(z_0)] $$
	exists and is independent of $θ$.
\end{definice}

\begin{definice}[Notation]
	Let $f: ®R^2 \rightarrow ®R^2$, where $®R^2 = ®C$, have the total differential $df(z_0)$ at $z_0 \in ®R^2 = ®C$, i.e.,
	$$ \lim_{h \rightarrow 0} \frac{f(z_0 + h) - f(z_0) - df(z_0)h}{|h|} = 0. $$
	Then
	$$ df(z_0) h = \frac{\partial f}{\partial x}(z_0)h_1 + \frac{\partial f}{\partial y}(z_0) h_2, $$
	$$ h = (h_1, h_2) = h_1 + ih_2 \in ®R^2 = ®C. $$
	We have $h_1 = \frac{h + \overline{h}}{2}$, $h_2 = \frac{h - \overline{h}}{2 i}$ and
	$$ df(z_0)h = \partial f(z_0) h + \overline{\partial} f(z_0) \overline{h}, $$
	where
	$$ \partial f(z_0) := \frac{1}{2}\(\frac{\partial f}{\partial x}(z_0) - i·\frac{\partial f}{\partial y}(z_0)\), $$
	$$ \overline{\partial} f(z_0) := \frac{1}{2}\(\frac{\partial f}{\partial x}(z_0) + i·\frac{\partial f}{\partial y}(z_0)\). $$

	\begin{poznamkain}
		We know $f'(z_0)$ exists iff $df(z_0)$ exists and $\overline{\partial} f(z_0) = 0$, in this case $f'(z_0) = \partial f(z_0)$.
	\end{poznamkain}
\end{definice}

\begin{veta}
	Let $G, \Omega \subset ®C$ be open. Then $f: G \rightarrow \Omega$ is conformal iff $f$ is a diffeomorphism of $G$ onto $\Omega$ preserving angles at any point of $G$.
\end{veta}

TODO!!!

% 11. 05. 2023

\section{Harmonic functions}
\begin{poznamka}
	We study $f: ®C \rightarrow ®C$. Since $®C = ®R^2$ we have $z = x + iy$, $x = \Re z$, $y = \Im z$ and $f = u + iv$ with $u = \Re f$, $v = \Im f$.
\end{poznamka}

\begin{poznamka}[Observation]
	If $G \subseteq ®C$ is a domain, $f, g \in ©H(G)$ and $\Re f = \Re g$ on $G$, then there is $c \in ®C$ such that $\Im f = \Im g + c$ on $G$.

	\begin{dukazin}
		It follows from CR condition:
		$$ \frac{\partial v}{\partial y} = + \frac{\partial u}{\partial x}, \qquad \frac{\partial v}{\partial x} = -\frac{\partial u}{\partial y}. $$
	\end{dukazin}
	What are the real parts of holomorphic functions?
\end{poznamka}

\begin{lemma}
	If $f \in ®C^2(G)$, $G \subseteq ®C$ open, then $\partial \overline{\partial} f = \overline{\partial} \partial f = \frac{1}{4}Δf$, where $Δ := \frac{\partial^2}{\partial x^2} + \frac{\partial^2}{\partial y^2}$ is the Laplace operator.

	\begin{dukazin}
		$$ \partial \overline{\partial} f = \frac{1}{4}\(\frac{\partial}{\partial x} - i\frac{\partial}{\partial y}\)\(\frac{\partial}{\partial x} + i \frac{\partial}{\partial y}\)f = \frac{1}{4}Δf. $$
	\end{dukazin}
\end{lemma}

\begin{definice}
	If $G \subseteq ®C$ is open, we say that $u \in ®C^2(G)$ is harmonic if $Δu = 0$ on $G$.
\end{definice}

TODO example?

\begin{priklad}
	If $G \subset ®C$ is a simply connected domain, $f \in ©H(G)$ and $f ≠ 0$ on $G$, then $\log |f| = \Re F$ for some $F \in ©H(G)$. In particular, $\log |f|$ is harmonic.

	\begin{dukazin}
		We know that there is a holomorphic branch $F$ of $\log f$, i.e., $F \in ©H(G)$ and $f = e^F$. Then $|f| = e^{\Re F}$.
	\end{dukazin}
\end{priklad}

\begin{dusledek}[From examples]
	If $G \subset ®C$ is open, $0 ≠ f \in ©H(G)$, then $\log |f|$ is harmonic.

	\begin{dukazin}
		By the previous exercise, $\log |f|$ is harmonic on any open ball $U$ in $G$.
	\end{dukazin}
\end{dusledek}

TODO example?

\begin{veta}
	If $G \subset ®C$ is a simply connected domain and $u: G \rightarrow ®R$ is harmonic, then there is $f \in ©H(G)$ such that $\Re f = u$.

	\begin{poznamkain}
		Every harmonic function is locally (but not globally) the real part of a holomorphic function.

		If $f \in ©H(G)$, then $f' = \partial f = \partial (f + \overline{f}) = 2\partial (\Re f)$, because $\partial \overline{f} = \overline{(\overline{\partial} f)} = 0$.
	\end{poznamkain}

	\begin{dukazin}
		We have $\partial u \in ©H(G)$, because $\overline{\partial}(\partial u) = \frac{1}{4}Δu = 0$. Then there is $f_0 \in ©H(G)$ such that $f_0' = 2\partial u$.

		By remark, $2\partial (\Re f_0) = f_0' = 2\partial u$, $\partial (\Re f_0 - u) = 0$. $\implies$ $\frac{\partial}{\partial x}(\Re f_0 - u) = 0$, $\frac{\partial}{\partial y}(\Re f_0 - u) = 0$ on the domain $G$.
	\end{dukazin}
\end{veta}

\begin{dusledek}
	Let $G \subset ®C$ be open and $u: G \rightarrow ®R$ be harmonic. Then $u \in ©C^∞(G)$ and $u$ satisfies the mean value property:
	$$ \frac{1}{2π}\int_0^{2π} u(z_0 + re^{it}) dt = u(z_0), $$
	whenever $\overline{U(z_0, r)} \subset G$.

	\begin{dukazin}
		Let $\overline{U(z_0, r)} \subseteq G$. Take $R \in (r, +∞)$ such that $U(z_0, R) \subseteq G$. Then  $u = \Re F$ for some $F \in ©H(U(z_0, R))$. So $u \in ©C^∞(U(z_0, R))$ and, by the Cauchy integral formula,
		$$ F(z_0) = \frac{1}{2πi} \int_φ \frac{F(z)}{z - z_0} dz = \frac{1}{2πi} \int_0^{2π} F(z_0 + re^{it})·\frac{ire^{it}}{re^{it}} dt, $$
		where $φ(t) = z_0 + re^{it}$, $t \in [0, 2π]$. This implies mean value property.
	\end{dukazin}
\end{dusledek}

\begin{veta}[The maximum principle]
	Let $G \subset ®C$ be a domain and $u: G \rightarrow ®R$ be continuous and $u$ satisfies mean value property. If $u$ is not constant, then $u$ does not attain an extreme in $G$.

	\begin{dukazin}
		Let us assume that $z_0 \in G$ and, say, $u(z_0) ≥ u$ on $G$. Put $M:=\{z \in G | u(z) = u(z_0)\}$. Obviously, $\O ≠ M$ is closed in $G$. If we show that $M$ is open, then $M = G$. Let $z_1 \in M$ and $U(z_1, r) \subset G$. We show that $U(z_1, r) \subset M$. Assume that there is $z_2 \in U(z_1, r) \setminus M$.

		By mean value property, $u(z_0) = u(z_1) = \frac{1}{2π} \int_0^{2π} u(z_1 + ρ e^{it}) dt < u(z_0)$, where $ρ := |z_1 - z_2|$. The Part inequality is strict because $u ≤ u(z_0)$ on $G$ and $u < u(z_0)$ in a neighbourhood of $z_2$.
	\end{dukazin}
\end{veta}

\begin{dusledek}
	Let $G \subset ®C$ be bounded and  open, $u \in ©C(\overline{G})$ and $u$ be harmonic on $G$. Then
	$$ \min_{\partial G}u ≤ u ≤ \max_{\partial G} u \text{ on } G. $$

	\begin{poznamkain}
		The assumption of harmonicity can be replaced with mean value property.
	\end{poznamkain}

	\begin{dukazin}
		Let $z_0 \in G$ and $u(z_0) = \max_{\overline{G}} u$. Then $u$ is constant on the component $G_0$ of $G$ containing $z_0$. So $u$ attains the maximum at the boundary.
	\end{dukazin}
\end{dusledek}

\section{Poisson integral}
\begin{poznamka}
	Let $u: \overline{®D} \rightarrow ®R$ be harmonic, i.e. $u$ is harmonic on $U(0, r)$ for some $r > 1$. Then there is $f \in ©H(\overline{®D})$ such that $u = \Re f$ on $\overline{®D}$ and $f(z) = \sum_{n=0}^∞ a_n z^n$, $|z| ≤ 1$.

	For $|z| ≤ 1$, $z = re^{iθ}$, we have
	$$ u(z) = \Re f(z) = \Re a_0 + \sum_{n=0}^{+∞} \frac{1}{2}(a_n r^n e^{inθ} + \overline{a_n} r^n e^{-inθ}). $$

	Hence $u(z) = \sum_{n=-∞}^∞ b_n·r^{|n|} e^{inθ}$, where $b_n = \Re a_0$ for $n = 0$, $b_n = \frac{1}{2}a_n$ for $n > 0$ and $b_n = \frac{1}{2}\overline{a_{-n}}$ for $n < 0$.

	In addition, we have
	$$ \frac{1}{2π}\int_{-π}^{π} u(e^{it})e^{-imt} dt = \frac{1}{2π} \int_{-π}^π \(\sum_{n=-∞}^∞ b_n e^{int}\) e^{-imt} dt = \frac{1}{2π} \int_{-π}^π \sum_{n=-∞}^∞ b_n e^{i(n - m)t} dt = $$
	$$ = \sum_{n=-∞}^∞ b_n · \frac{1}{2π} \int_{-π}^π e^{i(n - m)t}dt = b_m. $$

	Putting this equations together, we get, for $z = re^{iθ}$ with $r\in[0, 1)$,
	$$ u(z) = \sum_{n=-∞}^∞ r^{|n|}·\frac{1}{2π} \int_{-π}^π u(e^{it})e^{in(θ - t)} dt = \frac{1}{2π} \int_{-π}^π \(\sum_{n=-∞}^∞ r^{|n|} e^{in(θ - t)}\) u(e^{it}) dt. $$
\end{poznamka}

\begin{definice}[Poisonn kernel, poisson integral]
	For $0 ≤ r < 1$, $θ \in ®R$, put
	$$ P_r(θ) := \sum_{n=-∞}^∞ r^{|n|} e^{inθ}. $$

	$$ [Pu](re^{iθ}) := \frac{1}{2π} \int_{-π}^π P_r(θ - t) u(e^{it})dt $$
	for $0 ≤ r < 1$, $θ \in ®R$.
\end{definice}

\begin{priklad}
	$$ P_r(θ) = \Re \(\frac{1 + re^{iθ}}{1 - re^{iθ}}\) = \frac{1 - r^2}{1 + r^2 - 2r \cos θ}. $$

	\begin{dukazin}
		TODO?
	\end{dukazin}
\end{priklad}

\begin{veta}
	If $u$ is harmonic on $\overline{®D}$, then $u = Pu$ on ®D.

	\begin{poznamkain}
		The Poisson formula for harmonic functions is an analogue of the Cauchy integral formula.

		$P1 = 1$.
	\end{poznamkain}
\end{veta}

\end{document}
