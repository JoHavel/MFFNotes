\documentclass[12pt]{article}					% Začátek dokumentu
\usepackage{../../MFFStyle}					    % Import stylu



\begin{document}

% 16. 02. 2023

\begin{poznamka}
	Credit for giving 'small lecture'. Oral exam.
\end{poznamka}

\section{Meromorphic functions}
\begin{definice}
	We say that a function $f$ is holomorphic in a set $F \subset ®C$ if there is an open $G \supseteq F$ such that $f$ is holomorphic on $G$.

	In particular, $f$ is holomorphic at $z_0 \in ®C$ if $f$ is holomorphic in some neighbour (= $U(z_0) = U(z_0, \epsilon)$) of $z_0$.
\end{definice}

\begin{definice}
	Function $f$ has at $∞$ a removable singularity, if $f\(\frac{1}{z}\)$ has a removable singularity at 0. Similarly pole and essential singularity.

	Function $f$ is holomorphic at $∞$ if $f\(\frac{1}{z}\)$ is holomorphic at $0$.

	Let $G \subset ®S$ be open. Then $f$ is holomorphic on $G$ if $f$ is holomorphic at any $z_0$. Denote $©H(G) := \{f: G \rightarrow ®C | f \text{ holomorphic}\}$.

	\begin{prikladyin}
		From Liouville theorem $®H(®S) =$ constant functions. So $®H(G)$ is interesting only for $G \subsetneq ®S$, so WLOG $G \subset ®C$.
	\end{prikladyin}
\end{definice}

\begin{definice}[Meromorphic function]
	Let $G \subset ®S$ be open. Then a function $f$ on $G$ is called meromorphic if at any $z_0 \in G$ the function $f$ is either holomorphic at $z_0$ or has a pole at $z_0$.

	Denote $©M(G)$ the set of meromorphic functions on $G$.
\end{definice}

\begin{dusledek}
	\ 

	\begin{itemize}
		\item $©H(G) \subset ©M(G)$.
		\item Denote $P_f := \{z_0 \in G | f \text{ has a pole at } z_0\}$. Then $P_f$ has no limit points in $G$.
		\item If $f = ∞$ on $P_f$, then $f: G \rightarrow ®S$ is continuous. (We always assume, that $f \in ©H(G)$ has this property.)
	\end{itemize}
\end{dusledek}

\begin{priklady}
	$$ \frac{\pi}{\sin(\pi z)} \in ©M(®C), \qquad e^{\frac{1}{z}} \notin ©M(®C), \qquad \Gamma \in ©M(®C), \qquad \zeta \in ©M(®C). $$

	$©M(®S) =$ rational functions. (One inclusion is clear, second: Let $f \in ©M(®S)$, then because $®S$ is compact it holds that $P_f$ is finite (has no limit point), $P_f \cap ®C = \{z_1, …, z_n\}$, so from theorem from last semester there exists $h \in ©H(®C)$ such that $f(z) = h(z) + \sum_{j=1}^n p_j \(\frac{1}{z - z_j}\)$ for some polynomials $p_j$. $f$ has removable singularity or pole at infinity and $p_j$ and $\frac{1}{z - z_j}$ have removable singularity there, so $h(z)$ is polynomial, otherwise $h(z)$ has infinity Taylor polynom and $h\(\frac{1}{z}\)$ has essential singularity at $0$.)

	So $©M(G)$ is interesting for $G \subsetneq ®S$, WLOG $G \subset ®C$.

	If $G \subset ®C$ is domain, $f, g \in ®H(G)$ and $g ≡ 0$, then $f / g \in ©M(G)$. The inverse is also true (we will prove it) (but not for $G = ®S$).
\end{priklady}

\begin{lemma}
	Let $®G \subset ®C$ be open. Then there are compacts $K_n$, $n \in ®N$, in $G$ such that $G = \bigcup_{n=1}^∞ K_n$, $K_n \subset \Int (K_{n+1})$ and for any compact $K$ in $G$, $\exists n \in ®N: K \in K_n$.

	\begin{dukazin}
		Set $K_n := \{z \in G | \dist(z, ®C \setminus G) ≥ \frac{1}{n}\} \cap U(0, n)$.
	\end{dukazin}
\end{lemma}

\begin{tvrzeni}
	Let $G \subset ®S$ be open and $M \subset G$ has no limit point in $G$. Then

	\begin{itemize}
		\item $G \setminus M$ is open;
		\item if $K$ is a compact in $G$, then $K \cap M$ is finite. In particular for $G = ®S$ we have $M$ is finite;
		\item $M$ is at most countable. If $M$ is infinite, then $\O ≠ M' \subset \partial G$;
		\item if $G \subset ®C$ is domain (connected), then $G \setminus M$ is domain.
	\end{itemize}
\end{tvrzeni}

\begin{veta}[Uniqueness of meromorphic functions]
	Let $G \subset ®C$ be a domain, $f \in ©M(G)$ and $f \not≡ 0$. Then $N_f := \{z \in G | f(z) = 0\}$ has no limit points in $G$.

	\begin{dukazin}
		We know this holds for holomorphic functions. Set $G_0 := G \setminus P_f$. Then $G_0 \subset ®C$ is also domain and $f \in ©H(G)$ and $f \not≡ 0$ on $G_0$. Then $N_f \subset G_0$ has no limit points in $G_0$, nor in $P_f$.
	\end{dukazin}
\end{veta}

\begin{veta}[Residue theorem]
	Let $G \subset ®C$ be open, $\phi$ be a closed curve (or cycle) in $G$ and $\Int \phi := \{z_0 \in ®C \setminus \<\phi\> | \ind_\phi z_0 ≠ 0\} \subset G$. Let $M \subset G \setminus \<\phi\>$ be finite and $f \in ©H(G \setminus M)$. Then $\int_\phi f = 2\pi i·\sum_{s \in M} \ind_\phi s · \res_s f$.

	\begin{poznamkain}
		This holds true even if instead of finiteness of $M$, we assume only that $M \subset G \setminus \<\phi\>$ has no limit points in $G$. Indeed, we have $M_0 = M \cap \Int \phi$ is finite, because $\<\phi\> \cup \Int \phi$ is compact and $G_0 := G \setminus (M \setminus M_0)$ is open and $f$ is holomorphic on $G_0 \setminus M_0$ and by R. theorem for $G_0$ and $M_0$ we get $\int_\phi f = 2 \pi i \sum_{s \in M_0} \res_s f · \ind_\phi s$.
	\end{poznamkain}
\end{veta}

\subsection{Logarithmic integrals}

\begin{definice}[Logarithmic integral]
	Let $\phi: [a, b] \rightarrow ®C$ be a (regular) curve and let $f$ be a non-zero holomorphic function on $\<\phi\>$. Then we define logarithmic integrals integral as
	$$ I := \frac{1}{2\pi i} \int_\phi \frac{f'}{f} = \frac{1}{2\pi i} \int_a^b \frac{f'(\phi(t)) \phi'(t)}{f(\phi(t))} dt = \frac{1}{2 \pi i} \int_a^b \frac{(f(\phi(t)))'}{f(\phi(t))} dt = \frac{1}{2\pi i} \int_{f \circ \phi} \frac{dz}{z} = \frac{1}{2\pi i}(\Phi(b) - \Phi(a)), $$
	where $\Phi$ is a branch (jednoznačná větev) of logarithm of $f \circ \phi$. If $\phi$ is, in addition, closed, then $I = \ind_{f \circ \phi} 0 = \frac{1}{2\pi}(\Theta(b) - \Theta(a)) \in ®Z$, where $\Theta$ is a branch of argument of $f \circ \phi$.

	($\frac{f'}{f}$ is called logarithmic derivative of $f$, because '$(\log f)' = \frac{f'}{f}$'.)
\end{definice}

\begin{veta}[Argument principle]
	Let $G \subseteq ®C$ be a domain, $\phi$ be a closed curve in $G$ and $f \in ©M(G)$. Let $\Int \phi \subset G$ and $\<\phi\> \cap N_f = \O$, $\<\phi\> \cap P_f = \O$. Then
	$$ \frac{1}{2\pi i} \int_\phi \frac{f'}{f} = \sum_{s \in \Int \phi, f(s) = 0} n_f(s) · \ind_\phi s - \sum_{s \in \Int \phi, f(s) = ∞} p_f(s) · \ind_\phi s, $$
	where $n_f(s)$ is multiplicity of the zero point $s$ of $f$ and $p_f(s)$ is multiplicity of the pole $s$ of $f$.

	\begin{dukazin}
		By Residua theorem, we have
		$$ \frac{1}{2\pi i} \int_\phi \frac{f'}{f} = \sum_{s \in \Int \phi, s \in N_f \cup P_f} \res_s \(\frac{f'}{f}\)·\ind_\phi s. $$
		If $s \in N_f$ then on $P(s)$:
		$$ \frac{f'(z)}{f(z)} = \frac{p·c_p(z - s)^{p - 1} + …}{c_p(z - s)^p + …} = \frac{p}{z - s}·\frac{1 + …}{1 + …} \implies \res_s \(\frac{f'}{f}\) = p = n_f(s). $$
		If $s \in P_f$ then on $P(s)$
		$$ \frac{f'(z)}{f(z)} = \frac{p·c_p(z - s)^{p - 1} + …}{c_p(z - s)^p + …} = \frac{p}{z - s}·\frac{1 + …}{1 + …} \implies \res_s \(\frac{f'}{f}\) = p = -p_f(s). $$
	\end{dukazin}
\end{veta}

% 23. 02. 2023

\begin{definice}
	$$ \Sigma(f, \phi) := \sum_{s \in \Int \phi, f(s) = 0} n_f(s)·\ind_\phi s - \sum_{s \in \Int \phi, f(s) = ∞} p_f(s) · \ind_\phi s. $$
\end{definice}

\begin{lemma}
	Let $\phi_1, \phi_2: [a, b] \rightarrow ®C$ be closed curve and $s \in ®C \setminus (\<\phi_1\> \cup \<\phi_2\>)$. Assume, for $t \in [a, b]$, $|\phi_1(t) - \phi_2(t)| < |\phi_1(t) - s|$. Then $\ind_{\phi_1} s = \ind_{\phi_2} s$.

	\begin{dukazin}
		For $t \in [a, b]$, we have $|(\phi_1(t) - s) - (\phi_2(t) - s)| < |\phi_1(t) - s|$. Divide by $|\phi_1(t) - s|$:
		$$ |1 - \psi(t)| < 1, \qquad \psi(t) := \frac{\phi_2(t) - s}{\phi_1(t) - s}. $$
		Then $\psi$ is a closed curve, $<\psi> \subset U(1, 1)$, and so
		$$ 0 = \ind_\psi 0 = \frac{1}{2\pi i} \int_a^b \frac{\psi'}{\psi} = \frac{1}{2\pi i} \int_a^b \frac{\frac{\phi_2'(\phi_1 - s) - \phi_1'(\phi_2 - s)}{(\phi_1 - s)^2}}{\frac{\phi_2 - s}{\phi_1 - s}} = \frac{1}{2\pi i}\int_a^b \frac{\phi_2'}{\phi_2 - s} - \frac{1}{2\pi i}\int_a^b \frac{\phi_1'}{\phi_1 - s} = \ind_{\phi_2} s - \ind_{\phi_1} s. $$
	\end{dukazin}
\end{lemma}

\begin{veta}[Rouché]
	Let $G \subset ®C$ be a domain, $f_1, f_2 \in ©M(G)$ and $\phi$ be closed curve in $G$ such that $\Int \phi \subset G$. Assume $\forall z \in \<\phi\>$:
	$$ |f_1(z) - f_2(z)| < |f_1(z)| < +∞ $$
	Then $\Sigma(f_1, \phi) = \Sigma(f_2, \phi)$.

	\begin{dukazin}
		Set $\phi_j = f_j \circ \phi$. Then
		$$ \ind_{\phi_j} 0 = \frac{1}{2\pi i} \int_\phi \frac{f_j'}{f_j} = \Sigma(f_j, \phi). $$
		By previous lemma we have for $s = 0$: $\ind_{\phi_1} 0 = \ind_{\phi_2} 0$.
	\end{dukazin}
\end{veta}

\begin{dusledek}
	Let $f_1, f_2$ be holomorphic functions on $\overline{U(z_0, r)}$ and $\forall z \in \partial U(z_0, r): |f_1(z) - f_2(z)| < |f_1(z)|$. Then $\Sigma_1 = \Sigma_2$, where $\Sigma_j := \sum_{s \in U(z_0, r), f(s) = 0} n_{f_j}(s)$.

	\begin{dukazin}
		Apply Rouché's theorem to $\phi(t) := z_0 + r·e^{it}$, $t \in [0, 2\pi]$.
	\end{dukazin}
\end{dusledek}

\begin{priklad}
	$f_2 = p$, $f_1(z) = a_0 z^n$ and big enough $U(0, r)$.
\end{priklad}

\begin{definice}[Notation]
	Let $f$ be a function holomorphic at $z_0 \in ®C$. We say that $f(z_0) = w_0 \in ®C$ $p$~times for $p \in ®N$ if $z_0$ is a zero point of $f - w_0$ of order $p$.

	\begin{poznamkain}
		Following statements are equivalent to each other:

		\begin{itemize}
			\item $f(z_0) = w_0$ $p$~times;
			\item $f(z_0) = w_0$, $f'(z_0) = 0 = … = f^{(p - 1)}(z_0)$, $f^{(p)}(z_0) ≠ 0$;
			\item $f(z) = w_0 + \sum_{k=p}^{+∞} c_k(z - z_0)^k$ on some neighbourhood of $z_0$ and $c_p ≠ 0$.
		\end{itemize}
	\end{poznamkain}

	We say that $f(z_0) = ∞$ $p$~times if $z_0$ is a zero point of $\frac{1}{f}$ of order $p$. (It's the same as $z_0$ is pole of $f$ of order $p$.) And we say that $f(∞) = w_0 \in ®S$ $p$~times if $f(1/z)$ attains $w_0$ $p$~times at 0.
\end{definice}

\begin{veta}[On a multiple value]
	Let $z_0, w_0 \in ®S$, $f$ be a holomorphic function on a $P(z_0)$ and $f(z_0) = w_0$ $p$ times for some $p \in ®N$. Let $\delta_0 > 0$. Then there are $\epsilon > 0$ and $\delta \in (0, \delta_0)$ such that, for any $w \in P(w_0, \epsilon)$ there are just $p$ different points $z_1, …, z_p$ in $P(z_0, \delta)$ with $f(z_j) = w$. In addition, $f(z_j) = 0$ once.

	\begin{dukazin}
		WLOG, assume $z_0 = 0 = w_0$.Then $z_0 = 0$ is a zero point of $f$ of order $p$. Choose $\delta \in (0, \delta_0)$ such that $f ≠ 0$ and $f' ≠ 0$ on $P(0, 2\delta)$. Set $\epsilon := \min_{|z| = \delta} |f(z)| > 0$.

		Let $w \in P(0, \epsilon)$. Use Rouché's theorem for $f_1 := f$, $f_2 := f - w$ and $\phi := \delta e^{it}$, $t \in [0, 2\pi]$. Of course, $|f_1 - f_2| = |w| < \epsilon < |f_1|$ on $\<\phi\>$.

		Since in $U(0, \delta)$ the function $f = f_1$ has the only zero point of order $p$ at origin, $f - w = f_2$ has just $p$ simple zero points in $P(0, \delta)$.
	\end{dukazin}
\end{veta}

\begin{dusledek}
	Let $G \subset ®S$ be a domain, $f \in ©M(G)$ and $f$ be not constant on $G$. Then $f: G \rightarrow ®S$ is an open map (for any open $\Omega \subset G, f(\Omega)$ is open).

	\begin{dukazin}
		Let $\Omega \subset G$ be open and $w_0 \in f(\Omega)$. Then there is a $z_0 \in \Omega$ and $p \in ®N$ such that $f(z_0) = w_0$ $p$ times. Choose $\delta_0 > 0$ such that $U(z_0, \delta_0) \subset \Omega$. By the previous theorem, there is $\epsilon > 0$, $\delta \in (0, \delta_0)$ such that $P(w_0, \epsilon) \subset f(P(z_0, \delta))$, so $U(w_0, \epsilon) \subset f(U(z_0, \delta)) \subset f(\Omega)$.
	\end{dukazin}

	\begin{poznamkain}
		This is true for $©H(G)$ too.
	\end{poznamkain}
\end{dusledek}

\begin{dusledek}
	Let $f$ be a function holomorphic at $z_0 \in ®C$. Then $f'(z_0) ≠ 0$ if and only if there is $U(z_0)$ such that $f|_{U(z_0)}$ is one-to-one.

	\begin{dukazin}
		„$\implies$“: Let $f'(z_0) ≠ 0$. Then $f(z_0) = w_0$ once, so we choose $\delta_0 > 0$ such that $f ≠ w_0$ on a $P(z_0, \delta_0)$. By the previous theorem choose $\epsilon > 0$, $\delta \in (0, \delta_0)$. Moreover, due to the continuity of $f$ at $z_0$ choose $\delta_1 \in (0, \delta)$ such that $f(U(z_0, \delta_1)) \subset U(w_0, \epsilon)$. Then $f|_{U(z_0, \delta_1)}$ is one-to-one.

		„$\impliedby$“: Let $f'(z_0) = 0$ and let $f$ be not constant on any neighbourhood of $z_0$. Then $f(z_0) = w_0$ $p$ times ($p \in ®N \setminus \{1\}$). By the previous theorem $f$ is not one-to-one on any neighbourhood of $z_0$.
	\end{dukazin}
\end{dusledek}

\begin{veta}[On holomorphic inverse]
	Let $G \subset ®C$ be open and $f: G \rightarrow ®C$ be a one-to-one holomorphic\footnote{One-to-one holomorphic function is sometimes called conformal.} function, then $f' ≠ 0$ on $G$, $\Omega := f(G)$ is open and $f_{-1}: \Omega \overset{\text{onto}}\rightarrow G$ is holomorphic.

	In addition, $(f_{-1})' = \frac{1}{f' \circ f_{-1}}$ on $\Omega$.

	\begin{dukazin}
		WLOG, $G \subset ®C$ is a domain. By first „dusledek“ of previous theorem $f$ is an open map, so $\Omega := f(G)$ is open and $f_{-1}: \Omega \rightarrow G$ is continuous. Let $z_0 \in G$ and $w_0 = f(z_0)$. By second „dusledek“ we have $f'(z_0) ≠ 0$, and
		$$ \frac{1}{f'(z_0)} = \lim_{z \rightarrow z_0} \frac{z - z_0}{f(z) - f(z_0)} \overset*= \lim_{w \rightarrow w_0} \frac{f_{-1}(w) - f_{-1}(w_0)}{w - w_0} = f_{-1}'(w_0). $$
		The equality * follows from theorem on limits of composite functions because $f_{-1}$ is continuous and $f_{-1}(w) ≠ f_{-1}(w_0)$ for $w ≠ w_0$.
	\end{dukazin}
\end{veta}

\begin{veta}[Hurwitz]
	Let $G \subset ®C$ be a domain, $f_n \in ©H(G)$, $f_n \overset{\text{loc.}}\rightrightarrows f$ on $G$ and $f \not≡ 0$. Let $z_0 \in G$ be a zero point of $f$. Then $\exists$ $\{z_n\}_{n = 1}^∞ \subset G$ and a subsequence $\{f_{k_n}\}$ of $\{f_n\}$ such that $z_n \rightarrow 0$ and $f_{k_n}(z_n) = 0$.

	\begin{poznamkain}
		Not true in ®R! The assumption $f \not≡ 0$ is important! ($f_n(z) := z/n$)
	\end{poznamkain}
\end{veta}

\begin{dusledek}
	Let $G \subset ®C$ be a domain, $f_n$ be one-to-one holomorphic functions on $G$ and $f_n \overset{\text{loc}}\rightrightarrows f$ on $G$. Then $f$ is either one-to-one and holomorphic, or constant.
\end{dusledek}

\end{document}
