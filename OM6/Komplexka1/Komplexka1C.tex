\documentclass[12pt]{article}					% Začátek dokumentu
\usepackage{../../MFFStyle}					    % Import stylu



\begin{document}

% 23. 02. 2023

TODO!!! % RUDIN 16.1

\begin{definice}[WLOG]
	$$ D := U(0, 1), \qquad T = \partial D. $$
\end{definice}

TODO!!! % RUDIN 16.2

\begin{definice}
	$f \in ©H(D)$. We say that the boundary $T$ is a natural boundary of $f$ if $R_f = \O$.
\end{definice}

\begin{priklady}
	$f(z) = \sum_{n=0}^∞ z^{2^n}$. Radius of convergence is equal to 1 and $f$ has natural boundary.

	\begin{dukazin}
		$K = \{\exp\(\frac{2 \pi i k}{n}\) | k, n \in ®N\}$ is dense in $T$. $f$ is "diverges on" this set, because $f(z^{2^N}) = f(z) - \sum_{n=1}^N$. For $\alpha \in (0, 1)$ we have parametrization of one "line" $\alpha·\exp\(\frac{2k\pi i}{2^n}\)$ (for $k$, $n$ fixed).
		$$ f\(\alpha^{2^N}\) = f\(\alpha \exp\(\frac{2k\pi i}{2^N}\)\) + p\(\alpha \exp\(\frac{2k\pi i}{2^N}\)\). $$
	\end{dukazin}

	For every domain $\Omega \subseteq ®C$, there exists $f \in ©H(\Omega)$ such that $\partial \Omega$ is natural boundary of $f$.

	\begin{dukazin}
		We use theorem (15.11 from Rudin or TODO from lecture).
	\end{dukazin}
\end{priklady}

% 02. 03. 2023

TODO!!! % Část ještě z minula, zbytek já.

% 09. 03. 2023

TODO!!! % Oprava mého.

\section{Eulerův vzorec}
$$ \sin \pi z = \pi z \prod_{k=1}^∞ \(1 - \frac{z^2}{k^2}\). $$

\begin{lemma}
	$$ z ≠ \frac{k}{2}, k \in ®Z: 2\pi \cotg(2\pi z) = \pi \cotg(\pi x) + \pi \cotg(\pi(z + \frac{1}{2})). $$

	\begin{dukazin}
		$$ 2\pi \cotg(2 \pi z) = 2\pi \frac{\cos(2\pi z)}{\sin(2\pi z)} = \pi \frac{\cos^2(\pi z) - \sin^2(\pi z)}{\sin(\pi z) \cos(\pi z)} = \pi\(\cotg(\pi z) - \frac{\sin(\pi z)}{\cos(\pi z)}\) = \pi\(\cotg(\pi z) + \frac{\cos \pi(z + \frac{1}{2})}{\sin \pi(z + \frac{1}{2})}\). $$
	\end{dukazin}
\end{lemma}

\begin{lemma}[Herglotz]
	$r > 1$, $G$ oblast, $G \supset [0, r)$, $h$ funkce holomorfní na $G$, $z, z + \frac{1}{2}, 2z \in [0, r): 2h(2z) = h(z) + h(z + \frac{1}{2})$. Pak $h$ je konstantní na $G$.

	\begin{dukazin}
		Zvol $t \in (1, r)$.
		$$ M := \max \{|h'(z)|, z \in [0, t]\}, \qquad 4|h'(2z)| ≤ |h'(z)| + |h'(z + \frac{1}{2})| \implies $$
		$$ \implies 4|h'(z)| ≤ |h'(\frac{z}{2})| + |h'(\frac{z}{2} + \frac{1}{2})| < 2M \implies 4M ≤ 2M \implies M = 0. $$
	\end{dukazin}
\end{lemma}

\begin{lemma}
	$g$ holomorfní funkce na $®C \setminus ®Z$, hlavní část Laurenotvy řady $g$ na $P(k)$, $k \in ®Z$, je rovna $\frac{1}{z - k}$, $g$ lichá, $2g(2z) = g(z) + g(z + \frac{1}{2})$, $z ≠ \frac{k}{2}$, $k \in ®Z$. Pak $g(z) = \pi \cotg(\pi z)$.

	\begin{dukazin}
		$h(z) := g(z) - \pi \cotg(\pi z)$. $h$ rozšíříme spojitě a holomorfně na ®C. Z Herglotzova lemmatu je $h$ konstantní na ®C (obě funkce splňují podmínky). Navíc $h(0) = 0$.
	\end{dukazin}
\end{lemma}

\begin{dusledek}[Eisenstein]
	$\pi \cotg(\pi z) = \frac{1}{2} + \sum_{k=1}^∞ \frac{2z}{z^2 - k^2}$, $z \notin ®Z$.

	\begin{dukazin}
		$z \mapsto \frac{2z}{z^2 - k^2}$ jsou holomorfní na $U(0, n)$, $k > n$.
		$$ \left|\frac{2z}{z^2 - k^2}\right| ≤ \frac{2 n}{k^2 - n^2} \qquad \land \qquad \sum_{k=n+1}^∞ \frac{2n}{k^2 - n^2} K \implies \sum \in ©M. $$

		$f$ je holomorfní na $®C \setminus ®Z$. Také je lichá. Nakonec
		$$ s_n(z) = \frac{1}{2} + \sum_{k=1}^n \frac{2z}{z^2 - k^2} = \frac{1}{2} + \sum_{k=1}^n \frac{1}{z + k} + \sum_{k=1}^n \frac{1}{z - k} = \sum_{-n}^n \frac{1}{z + k}, $$
		$$ s_n\(\frac{z}{2}\) + s_n\(\frac{z + 1}{2}\) = \sum_{k=-n}^n \frac{2}{z + 2k} + \frac{2}{z + 2k + 1} = 2\sum_{k=-2n}^{2n} \frac{1}{z + k} + 2·\frac{1}{z + 2n + 1} = 2 s_{2n}(k) + \frac{2}{z + 2n + 1}, $$
		$$ n \rightarrow ∞: f\(\frac{z}{2}\) + f\(\frac{z + 1}{2}\) = 2f(z) + 0. $$

		Z předchozího lemmatu vyplývá důkaz.
	\end{dukazin}
\end{dusledek}

% 16. 03. 2023

\begin{veta}[Euler]
	$$ \underbrace{\sin(πz)}_g = \underbrace{π z \prod_{k=1}^∞ \(1 - \frac{z^2}{k^2}\)}_f. $$

	\begin{dukazin}
		$\forall z \in ®C\ \exists$ neighbourhood $U$: $\sum_{k=1}^∞ \|\(z \mapsto \frac{z^2}{k^2}\)\|_∞$ is convergent $\implies$ $\prod_{k=1}^∞ \(1 - \frac{z^2}{k^2}\)$ is holomorphic.
		$$ f_k(z) := 1 - \frac{z^2}{k^2}, \qquad \frac{f'_k(z)}{f_k(z)} = \frac{\frac{-2z}{k^2}}{\frac{k^2 - z^2}{k^2}} = \frac{2z}{z^2 - k^2}. $$
		$$ \frac{f'(z)}{f(z)} = \frac{1}{2π \prod\(1 - \frac{z^2}{k^2}\)} \(π \prod_{k=1}^∞ \(1 - \frac{z^2}{k^2}\) + πz\(\prod_{k=1}^∞ …\)'\) = \frac{1}{2} = \frac{\prod'}{\prod} = \frac{1}{2} + \sum_{k=1}^∞ \frac{f_k'(z)}{f_k(z)} = \frac{1}{2} + \sum_{k=1}^∞ \frac{2z}{z^2 - k^2} = \pi \cotg(πz). $$
		($\frac{g'(z)}{g(z)} = \frac{π \cos(πz)}{\sin πz} = π \cotg(πz)$, $\(\frac{f}{g}\)' = \frac{f' g - f g'}{g^2} = 0 \implies \frac{f}{g}$ is constant).
		$$ \lim_{z \rightarrow 0} \frac{f(z)}{g(z)} = \lim_{z \rightarrow 0} \frac{π \sin (zπ)}{TODO} TODO(one line). $$
	\end{dukazin}
\end{veta}

\section{Gamma function}
\begin{definice}
	$$ Γ(x) = \int_0^∞ e^{-t} t^{\lambda - 1} dt, \qquad x \in (0, ∞). $$

	\begin{poznamka}[Notion]
		$$ I_n = \int_0^1 1·(-\log x)^n dx \overset{\text{IBP}}= [x (-\log x)^n]_0^1 - \int_0^1 x n(-\log x)^{n-1}\(-\frac{1}{x}\) dx = n·I_{n-1}. $$
		So $I_n = n!$. Set $\log x = t$, $e^t = x$:
		$$ I_n = \int_{-∞}^0 (-t)^n · e^t dt = \int_0^∞ t^n e^{-t} dt. $$
	\end{poznamka}
\end{definice}

\begin{lemma}
	$\forall n \in ®N$ we define $Γ_n(x) = \int_0^n t^{x - 1}\(1 - \frac{t}{n}\)^n dt$, then $Γ_n(x) \rightarrow Γ(x)$.

	\begin{dukazin}
		„$0 ≤ e^t - \(1 - \frac{t}{n}\)^n ≤ \frac{t^2}{n}e^{-t}$“:

		$$ 0 ≤ e^{-t}·\(1 - \frac{t}{n}\)^n = e^{-t}\(1 - e^t\(1 - \frac{t}{n}\)^n\) ≤ e^{-t} \(1 - \(1 - \frac{t}{n}\)^n·\(1 + \frac{t}{n}\)^n\) = e^{-t}\(1 - \(1 - \frac{t^2}{n^2}\)^n\) ≤ \frac{t^2}{n} e^{-t}. $$
		(Last inequality from Bernoulli: $(1 + x)^n ≥ 1 + n·x$, $x ≥ -1$.)
		$$ |Γ(x) - Γ_n(x)| ≤ |\int_0^n \(e^{-t}\(1 - \frac{t}{n}\)^n\)·t^{x - 1}| + |\int_n^∞ e^{-t} t^{x - 1} dt| ≤ $$
		$$ ≤ \frac{1}{n} \int_0^n e^{-t} t^{x + 1}dt + \int_n^∞ e^{-t} t^{x - 1} dt ≤ \frac{1}{n} \int_0^∞ e^{-t} t^{x + 1}dt + \int_n^∞ e^{-t} t^{x - 1} dt \rightarrow 0. $$
	\end{dukazin}
\end{lemma}

\begin{lemma}
	$$ Γ(x) = \lim_{n \rightarrow ∞} \frac{n!·n^x}{x·(x + 1)· … ·(x + n)}. $$
	
	\begin{dukazin}
		„$Γ_n(x) = \frac{n! n^x}{x·(x + 1)· … · (x + n)}$“: $Γ_n(x) = \int_0^n t^{x - 1} \(1 - \frac{t}{n}\)^n dt = \frac{1}{n^n}·\int_0^n t^{x - 1}(n - A^t) dt \overset{\text{IBP}}=$
		$$ = \(\frac{1}{n}\)^n \(\[\frac{1}{x}·t·(n - t)^n\]_0^n + \int_0^n \frac{n}{x} t^x (n - t)^{n - 1} dt\) = \frac{1}{n^n} \frac{n·(n-1)·(n-2)·…·2·1}{x·(x+1)·(x+2)·…·(x + n - 1)}· \int_0^n t^{x + n - 1} dt = \frac{n^x·n!}{x·(x+1)·…·(x+n)}. $$
	\end{dukazin}
\end{lemma}

% 23. 03. 2023

\section{Weierstrass function}
\begin{definice}
	$$ H(z) := z·\prod_{k=1}^∞ \(1 + \frac{z}{k}\) e^{-\frac{z}{k}}. $$
\end{definice}

\begin{lemma}
	$H \in ©H(®C)$ has simple zero points just in $®N_0$.

	$H(z)·H(-z) = -\frac{z}{π}\sin(π z)$.

	$H(1) = e^{-γ}$, where $γ := \lim_{n \rightarrow ∞} \(\sum_{k=1}^n \frac{1}{k} - \log n\)$ is the Euler–Mascheroni constant. It is known $γ \overset{.}= 0, 577$, but it isn't known if it is even irrational (much less transcendent).

	\begin{dukazin}
		$\(1 + \frac{z}{k}\)e^{-\frac{z}{k}}$ is $E_1(\frac{z}{k})$ (first Weierstrass factor). So from WF, we know that 1. hold because $\sum_{k=1}^∞ \frac{|z|^2}{k^2}$ converges locally uniformly on $®C$.

		$$ H(z)·H(-z) = -z^2 \prod_{k=1}^∞ \(1 - \frac{z^2}{k^2}\) = - \frac{z}{π} \sin(πz). $$

		$$ H(1) = \lim_{n \rightarrow ∞} \prod_{k=1}^n \(1 + \frac{1}{k}\) e^{-\frac{1}{k}} = e^{-γ}, $$
		because
		$$ \frac{2}{1}·\frac{3}{2}·…·\frac{n+1}{n}\exp\(-\sum_{k=1}^n \frac{1}{k}\) = \exp\(\log(n+1) - \sum_{k=1}^n \frac{1}{k}\) \rightarrow e^{-γ}. $$
	\end{dukazin}
\end{lemma}

\begin{definice}[Weierstrass]
	$Δ(z) := e^{γ z} H(z)$.
\end{definice}

\begin{lemma}
	$Δ \in ©H(®C)$ has simple zero points just in $®N_0$.

	$$ Δ(z) = \lim_{n \rightarrow ∞} \frac{z·(z + 1)·…·(z + n)}{n! n^z}, \qquad z \in ®C. $$

	$Δ(1) = 1$, $z·Δ(z + 1) = Δ(z)$ for $z \in ®C$.

	\begin{dukazin}
		First is similar to previous lemma.

		$$ Δ(z) = e^{γ z}·\lim_{n \rightarrow ∞} z · \prod_{k=1}^n \(1 + \frac{z}{k}\) e^{-\frac{z}{k}} = e^{γ z}·\lim_{n \rightarrow ∞} z·\frac{(z + 1)·(z + 2)·…·(z + n)}{1·2·…·n} e^{-z·\sum_{k=1}^n \frac{1}{k}} = $$
		$$ = e^{γ z}·\lim_{n \rightarrow ∞} \frac{z·(z + 1)·…·(z + n)}{n!·n^z}·e^{-z·\sum_{k=1}^n \frac{1}{k} - \log n} = \lim_{n \rightarrow ∞} \frac{z·(z + 1)·…·(z + n)}{n!·n^z}. $$

		$Δ(1) = 1$ is obvious. $z·Δ(z + 1) = Δ(z)·\lim_{n \rightarrow ∞} \frac{z + n + 1}{n}$ previous.
	\end{dukazin}
\end{lemma}

\begin{definice}
	$Γ := \frac{1}{Δ}$.
\end{definice}

% 30. 03. 2023 From notes of my colleague

\begin{lemma}
	$Γ \in ©M(®C)$ has simple poles just in $(-®N_0) =: ®N_0^-$, $Γ ≠ 0$ on ®C.

	Gauss formula: $Γ(z) = \lim_{n \rightarrow ∞} \frac{n! n^z}{z·(z + 1)·…·(z + n)}$, $z \in ®C \setminus ®N_0^-$.

	$$ Γ(1) = 1, \qquad Γ(z + 1) = z·Γ(z). $$

	$$ \res_{-n} Γ = \frac{(-1)^n}{n!}, n \in ®N_0. $$

	\begin{dukazin}[Only last proposition]
		We know that (with $z \notin ®N_0^-$ in limits)
		$$ \res_{-n} Γ = \lim_{z \rightarrow -n} (z + n) Γ(z) = \lim_{z \rightarrow -n} \frac{Γ(z + n + 1)}{z·(z - 1)·…·(z + n - 1)} \frac{Γ(1)}{(-1)^n·n!} = \frac{(-1)^n}{n!}. $$
	\end{dukazin}
\end{lemma}

\section{?}
\begin{poznamka}
	$Ω$ open, bounded, $f \in ©C(\overline{Ω})$, $f \in ©H(Ω)$ $\implies$ $\sup_Ω |f| = \max_{\partial Ω} |f|$.
\end{poznamka}

\begin{veta}
	$Ω = \{x + i y | x \in (a, b) \land y \in ®R\}$, $f \in ©C(\overline{Ω}) \cap ©H(Ω)$, $|f| < B < ∞$ on $Ω$. $M(x) := \sup_{y \in ®R} |f(x + i y)|$. Then $M(x)^{b - a} ≤ M(a)^{b - x} · M(b)^{x - a}$.

	\begin{dukazin}
		$M(a) = M(b) = 1$, $|f| ≤ 1$ on $Ω$. Let $ε > 0$, $h_ε(z) = \frac{1}{1 + ε(z - a)}$. $|h_ε(z)| ≤ 1$ on $\overline{Ω}$.

		$\Re\{1 + ε(z - a)\} = 1 + ε(x - a) ≥ 1$. $|1 + ε(z - a)| ≥ ε|y|$. $|f(z) h_ε(z)| ≤ \frac{B}{ε}·\frac{1}{y}$, $y ≠ 0$. $|f h_ε| ≤ 1$ on $\partial R$.
	\end{dukazin}
\end{veta}

% 06. 04. 2023

\section{Riemann zeta function}
\begin{poznamka}
	$$ ζ(z) = \sum_{n=1}^∞ n^{-z}, \qquad \Re z > 1. $$
\end{poznamka}

\begin{definice}[Riemann zeta function]
	Riemann zeta function is defined on $\{\Re z > 1\}$ by $ζ(z) = \sum_{n=1}^∞ n^{-z}$ and $ $

	\begin{poznamkain}
		$$ Γ(z) = \int_0^∞ e^{-t} t^{z - 1} dt = n^z \int_0^∞ e^{-n t} t^{z - 1} dt \implies $$
		$$ \implies n^{-t} Γ(z) = \int_0^∞ e^{-n t} t^{z - 1} dt. $$

		$$ ζ(z)·Γ(z) = \sum_{n=1}^∞ \int_0^∞ e^{-n t} t^{z - 1} dt. $$
	\end{poznamkain}
\end{definice}

\begin{lemma}
	Let $S = \{\Re z ≥ a\}$, $a > 1$. If $ε > 0$, there is $δ \in (0, 1)$ such that $\forall z \in S$:
	$$ \left|\int_α^β (e^t - 1)^{-1} t^{z - 1} dt\right| < ε, \qquad δ > β > α > 0. $$

	Let $S = \{\Re z ≤ A\}$, $A \in ®R$. If $ε > 0$, there is $κ > 1$ such that $\forall z \in S$:
	$$ \left|\int_α^β (e^t - 1)^{-1} t^{z - 1} dt\right| < ε, \qquad β > α > κ. $$

	\begin{dukazin}
		„First part“: $e^t - 1 ≥ t$, $t ≥ 0$, so for $0 < t ≤ 1$:
		$$ z \in S: \left|(e^t - 1)^{-1} t^{z - 1}\right| ≤ |t^{z - 2}| \implies $$
		$$ \implies \int_0^1 \left|(e^t - 1)^{-1} t^{z - 1}\right| dt ≤ \int_0^1 |t^{z - 2}| dt < ∞. $$

		„Second part“: $t ≥ 1$, $z \in S$:
		$$ \left|(e^t - 1)^{-1} t^{z - 1}\right| ≤ (e^t - 1)^{-1}·t^{A - 1} < C·e^{\frac{1}{2} t} (e^t - 1)^{-1} \implies $$
		$$ \implies \int_1^∞ \left|(e^t - 1)^{-1} t^{z - 1}\right| dt ≤ C·\int_1^∞ e^{\frac{1}{2} t}(e^t - 1)^{-1} dt < ∞. $$
	\end{dukazin}
\end{lemma}

\begin{dusledek}
	If $S = \{a ≤ \Re z ≤ A\}$, $1 < a < A < ∞$, then $\int_0^∞ (e^t - 1)^{-1} t^{z - 1} dt$ converges uniformly on $S$.

	If $S = \{\Re z ≤ A\}$, $A \in ®R$, then $\int_1^∞ (e^t - 1)^{-1} t^{z - 1} dt$ converges uniformly on $S$.
\end{dusledek}

\begin{tvrzeni}
	For $\Re z > 1$
	$$ ζ(z)·Γ(z) = \int_0^∞ (e^t - 1)^{-1} t^{z-1} dt. $$

	\begin{dukazin}
		By the previous lemma for $ε > 0$ there exist $0 < α < β < ∞$ such that
		$$ \int_0^α (e^t - 1)^{-1} t^{x - 1} dt < \frac{3}{4}, $$
		$$ \int_\beta^∞ (e^t - 1)^{-1} t^{x - 1} dt < \frac{3}{4}. $$

		$$ \sum_{k=1}^∞ e^{-k t} ≤ (e^t - 1)^{-1} \qquad \forall n ≥ 1: $$
		$$ \sum_{n=1}^∞ \int_0^α e^{-n t} t^{x - 1} dt < \frac{3}{4}, $$
		$$ \sum_{n=1}^∞ \int_β^∞ e^{-n t} t^{x - 1} dt < \frac{3}{4}. $$

		$$ \left|ζ(x)·Γ(x) - \int_0^∞ (e^t - 1)^{-1} t^{x - 1} dt\right| = \left|\sum\(\int_0^α + \int_α^β + \int_β^∞\) - \(\int_0^α + \int_α^β + \int_β^∞\)\right| ≤ ? + \left|\sum_{n=1}^∞ \int_α^β e^{-n t} t^{x - 1} dt - \int_α^β (e^t - 1)^{-1} t^{x - 1}\right| = ?, $$
		since on $[α, β]$ $\sum e^{-n t}$ converges uniformly to $(e^t - 1)^{-1}$.
	\end{dukazin}
\end{tvrzeni}

\begin{poznamka}
	Extend to $\{\Re z > -1\}$: Laurent expansion (in 0):
	$$ (e^z - 1)^{-1} = \frac{1}{z} - \frac{1}{2} + \sum_{n=1}^∞ a_n z^n. $$
\end{poznamka}

% 13. 04. 2023

TODO?

\end{document}
