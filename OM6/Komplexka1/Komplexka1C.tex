\documentclass[12pt]{article}					% Začátek dokumentu
\usepackage{../../MFFStyle}					    % Import stylu



\begin{document}

% 23. 02. 2023

TODO!!! % RUDIN 16.1

\begin{definice}[WLOG]
	$$ D := U(0, 1), \qquad T = \partial D. $$
\end{definice}

TODO!!! % RUDIN 16.2

\begin{definice}
	$f \in ©H(D)$. We say that the boundary $T$ is a natural boundary of $f$ if $R_f = \O$.
\end{definice}

\begin{priklady}
	$f(z) = \sum_{n=0}^∞ z^{2^n}$. Radius of convergence is equal to 1 and $f$ has natural boundary.

	\begin{dukazin}
		$K = \{\exp\(\frac{2 \pi i k}{n}\) | k, n \in ®N\}$ is dense in $T$. $f$ is "diverges on" this set, because $f(z^{2^N}) = f(z) - \sum_{n=1}^N$. For $\alpha \in (0, 1)$ we have parametrization of one "line" $\alpha·\exp\(\frac{2k\pi i}{2^n}\)$ (for $k$, $n$ fixed).
		$$ f\(\alpha^{2^N}\) = f\(\alpha \exp\(\frac{2k\pi i}{2^N}\)\) + p\(\alpha \exp\(\frac{2k\pi i}{2^N}\)\). $$
	\end{dukazin}

	For every domain $\Omega \subseteq ®C$, there exists $f \in ©H(\Omega)$ such that $\partial \Omega$ is natural boundary of $f$.

	\begin{dukazin}
		We use theorem (15.11 from Rudin or TODO from lecture).
	\end{dukazin}
\end{priklady}

% 02. 03. 2023

TODO!!! % Část ještě z minula, zbytek já.

% 09. 03. 2023

TODO!!! % Oprava mého.

\section{Eulerův vzorec}
$$ \sin \pi z = \pi z \prod_{k=1}^∞ \(1 - \frac{z^2}{k^2}\). $$

\begin{lemma}
	$$ z ≠ \frac{k}{2}, k \in ®Z: 2\pi \cotg(2\pi z) = \pi \cotg(\pi x) + \pi \cotg(\pi(z + \frac{1}{2})). $$

	\begin{dukazin}
		$$ 2\pi \cotg(2 \pi z) = 2\pi \frac{\cos(2\pi z)}{\sin(2\pi z)} = \pi \frac{\cos^2(\pi z) - \sin^2(\pi z)}{\sin(\pi z) \cos(\pi z)} = \pi\(\cotg(\pi z) - \frac{\sin(\pi z)}{\cos(\pi z)}\) = \pi\(\cotg(\pi z) + \frac{\cos \pi(z + \frac{1}{2})}{\sin \pi(z + \frac{1}{2})}\). $$
	\end{dukazin}
\end{lemma}

\begin{lemma}[Herglotz]
	$r > 1$, $G$ oblast, $G \supset [0, r)$, $h$ funkce holomorfní na $G$, $z, z + \frac{1}{2}, 2z \in [0, r): 2h(2z) = h(z) + h(z + \frac{1}{2})$. Pak $h$ je konstantní na $G$.

	\begin{dukazin}
		Zvol $t \in (1, r)$.
		$$ M := \max \{|h'(z)|, z \in [0, t]\}, \qquad 4|h'(2z)| ≤ |h'(z)| + |h'(z + \frac{1}{2})| \implies $$
		$$ \implies 4|h'(z)| ≤ |h'(\frac{z}{2})| + |h'(\frac{z}{2} + \frac{1}{2})| < 2M \implies 4M ≤ 2M \implies M = 0. $$
	\end{dukazin}
\end{lemma}

\begin{lemma}
	$g$ holomorfní funkce na $®C \setminus ®Z$, hlavní část Laurenotvy řady $g$ na $P(k)$, $k \in ®Z$, je rovna $\frac{1}{z - k}$, $g$ lichá, $2g(2z) = g(z) + g(z + \frac{1}{2})$, $z ≠ \frac{k}{2}$, $k \in ®Z$. Pak $g(z) = \pi \cotg(\pi z)$.

	\begin{dukazin}
		$h(z) := g(z) - \pi \cotg(\pi z)$. $h$ rozšíříme spojitě a holomorfně na ®C. Z Herglotzova lemmatu je $h$ konstantní na ®C (obě funkce splňují podmínky). Navíc $h(0) = 0$.
	\end{dukazin}
\end{lemma}

\begin{dusledek}[Eisenstein]
	$\pi \cotg(\pi z) = \frac{1}{2} + \sum_{k=1}^∞ \frac{2z}{z^2 - k^2}$, $z \notin ®Z$.

	\begin{dukazin}
		$z \mapsto \frac{2z}{z^2 - k^2}$ jsou holomorfní na $U(0, n)$, $k > n$.
		$$ \left|\frac{2z}{z^2 - k^2}\right| ≤ \frac{2 n}{k^2 - n^2} \qquad \land \qquad \sum_{k=n+1}^∞ \frac{2n}{k^2 - n^2} K \implies \sum \in ©M. $$

		$f$ je holomorfní na $®C \setminus ®Z$. Také je lichá. Nakonec
		$$ s_n(z) = \frac{1}{2} + \sum_{k=1}^n \frac{2z}{z^2 - k^2} = \frac{1}{2} + \sum_{k=1}^n \frac{1}{z + k} + \sum_{k=1}^n \frac{1}{z - k} = \sum_{-n}^n \frac{1}{z + k}, $$
		$$ s_n\(\frac{z}{2}\) + s_n\(\frac{z + 1}{2}\) = \sum_{k=-n}^n \frac{2}{z + 2k} + \frac{2}{z + 2k + 1} = 2\sum_{k=-2n}^{2n} \frac{1}{z + k} + 2·\frac{1}{z + 2n + 1} = 2 s_{2n}(k) + \frac{2}{z + 2n + 1}, $$
		$$ n \rightarrow ∞: f\(\frac{z}{2}\) + f\(\frac{z + 1}{2}\) = 2f(z) + 0. $$

		Z předchozího lemmatu vyplývá důkaz.
	\end{dukazin}
\end{dusledek}
\end{document}
