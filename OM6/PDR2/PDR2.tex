\documentclass[12pt]{article}					% Začátek dokumentu
\usepackage{../../MFFStyle}					    % Import stylu

\begin{document}

% 13. 02. 2023

\begin{poznamka}
	The previous semester we work with linear equation (L-M, Fredholm, Minimizing quadratic function). This semester we will have non-linear equations like $((\partial_t u)) - \Delta u + \arctg u = f$ or $f = -\Delta_p u := -\Div (|\nabla u|^{p - 2} \nabla u)$.

	We don't work with $\partial_{t t} u - \Delta_p u = f$, because nobody know how to proof it has solution (for $d ≥ 2, p > 2$).
\end{poznamka}

\begin{poznamka}[Credit]
	Two homework. -10 to 10 points to exam from each. (If we hand anything we get credit.)
\end{poznamka}

\section*{What we must know}
\begin{poznamka}
	Lebesgue spaces.

	Fixed point theorem: 1) Let $F$ be continuous mapping from $®R^d$ to $®R^d$. Assume that $\exists$ convex compact set in $®R^d$ such that $F(\Omega) \subseteq \Omega$. Then $\exists x \in \Omega$ such that $F(x) = x$. 2) Let $F: X \rightarrow X$, where $X$ is Banach space and $F$ is continuous and compact and let $\exists$ $\Omega \subseteq X$ convex and closed such that $F(\Omega) \subseteq \Omega$. Then $F(\Omega) \subseteq \Omega$. Then $\exists x \in X: F(x) = x$.

	Luzin: Let $\Omega$ be a measurable set and $f \in L_{loc}^1(\Omega)$. Then $\forall \epsilon > 0$ $\exists U \in \Omega$, $\mu(U) ≤ \epsilon$, $f \in C(\Omega \setminus U)$.

	Egorov: Let $\Omega$ be a measurable set and $f^n \rightarrow f$ in $L_{loc}^1(\Omega)$. Then $\forall \epsilon > 0$ $\exists U, \mu(U) ≤ \epsilon$ $f^n \rightarrow f$ in $C(\Omega \setminus U)$.

	Lebesgue dominated convergence theorem.

	Vitali convergence theorem: Let $\Omega \subseteq ®R^d$ be bounded measurable, $f^n$ a sequence of measurable functions, $f^n \rightarrow f$ almost everywhere in $\Omega$. Then $\lim_{n \rightarrow ∞} \int_\Omega f^n = \int_\Omega f$, provided $f^n$ is uniformly equi-integrable ($\forall \epsilon > 0$ $\exists \delta$ $\forall U, \mu(U) ≤ \epsilon$).

	Fatou lemma: $f^n \rightarrow f$ almost everywhere in $\Omega$ and $f^n ≥ 0$, then $\liminf_{n \rightarrow ∞} \int_\Omega f^n ≥ \int_\Omega f$.

	Regularization: $\eta \in C_0^∞(B_1(¦o))$ non-negative, radially symmetric and $\int_{®R^d} \eta(x) dx = 1$. Then $\forall f \in L_{loc}^1(\Omega)$ we extend $f$ by „0“ to $®R^d$ and $f_\epsilon := \eta_\epsilon * f$, where $\eta_\epsilon(x) = \frac{1}{\epsilon^d} \eta(x / \epsilon)$. Then $f_\epsilon \in C^∞(®R^d)$ and $\forall p \in [1, ∞)$ $f \in L^p(\Omega) \implies f_\epsilon \rightarrow f$ in $L^p(\Omega)$. (And for $p = ∞$: $f \in L^∞(\Omega) \implies f_\epsilon \rightarrow f$ in $L^q(\Omega)$ $\forall q \in \in [1, ∞)$).

	Reflexive and separable spaces. ($L^p(\Omega)$ is a Banach space, separable for $p \in [1, ∞)$, reflexive for $p \in (1, ∞)$.)

	Nemytsky operator: (Assume that for almost all $x \in \Omega$ and $ $, $|f(x, y)| ≤ g(x) + C \sum_{i=1}^N |y_1|^{p_i / p}$ for some $p_i \in [1, ∞), p \in (1, ∞), g \in L^p(\Omega)$. Then $\forall u_i \in L^{p_i}$, the function $f(·, u_1, …, u_n)$ is measurable, $(u_1, …, u_n) \mapsto f(·, u_1, …, u_n)$ is continuous $L^{p_1}(\Omega) \times … \times L^{p_N}(\Omega) \rightarrow L^p(\Omega)$. This mapping is called N.O.)
\end{poznamka}

\section{Sobolev spaces (and Bochner spaces)}
\begin{poznamka}
	$\Omega$ is open bounded subset of $®R^d$.
\end{poznamka}

\begin{veta}[Local approximation by smooth functions]
	Let $f \in W^{k, p}(\Omega)$ and extend it by „0“ outside. Define $f_\epsilon := \eta_\epsilon * f$ and set $\Omega_\epsilon := \{x \in \Omega | B(x, \epsilon) \subseteq \Omega\}$. Then $D^\alpha(f_\epsilon) = (D^\alpha f)_\epsilon$ almost everywhere in $\Omega_\epsilon$ $\forall \alpha, |\alpha| ≤ k$ and $\forall \Omega' \subseteq \overline{\Omega'} \subseteq \Omega$ and $p \in [1, ∞)$ $f_\epsilon \rightarrow f$ in $W^{k, p}(\Omega')$. (If $p = ∞$, then $f_\epsilon \rightarrow ^* f$ in $W^{1, ∞}(\Omega')$.)

	\begin{dukazin}
		$$ \frac{\partial}{\partial x_i} \(f_\epsilon(x)\) = \frac{\partial}{\partial x_i} \int_{®R^d} \eta_\epsilon(x - y)f(y) dy = $$
		$$ = \int_{®R^d} \frac{\partial}{\partial x_i}(\eta_\epsilon(x - y)) f(y) dy = - \int_{®R^d} \frac{\partial}{\partial y_i} (\eta_\epsilon(x - y)) f(y) dy = $$
		$$ = -\int_{B(x, \epsilon)} \frac{\partial}{\partial y_i} (\eta_\epsilon(x - y)) f(y) dy = -\int_{\Omega} \frac{\partial}{\partial y_i} (\eta_\epsilon(x - y)) f(y) dy = $$
		$$ = \int_\Omega \eta_\epsilon(x - y) \frac{\partial f(y)}{\partial y_i} dy = \int_{®R^d} \eta_\epsilon(x - y) \frac{\partial f(y)}{\partial y_i} = \(\frac{\partial f(y)}{\partial y_i}\)_\epsilon (x). $$

		Now we take sufficiently small $\epsilon$, such that $\Omega_\epsilon \subseteq \Omega'$. Then $D^\alpha f_\epsilon = \(D^\alpha f\)_\epsilon \rightarrow D^\alpha f$ in $L^p(\Omega')$.
	\end{dukazin}
\end{veta}

\begin{veta}[Composition of Lipschitz and Sobolev functions]
	Let $\Omega \subseteq ®R^d$ be open and $f: ®R \rightarrow ®R$ be Lipschitz. Assume that $u \in W^{1, p}(\Omega)$. Then $(f(u) - f(0)) \in W^{1, p}(\Omega)$ and $\nabla f(u) = f'(u) \nabla u \chi_{x, u(x) \notin S_f}$, where $S_f$ are points where $f'(s)$ doesn't exists.

	Moreover define $\Omega_a := \{x \in \Omega | u(x) = a\}$, then $\nabla u = 0$ almost everywhere in $\Omega_a$.

	\begin{dukazin}
		We know, that $f \in C^1(®R)$, $f_{lip} := \sup_{x ≠ y} \frac{|f(x) - f(y)|}{|x - y|} < ∞$.

		So $|f(u(x)) - f(0)|^p ≤ f_{lip}^p·|u(x)|^p$, if $u \in L^p(\Omega) \implies f(u) - f(0) \in L^p(\Omega)$.

		Next, $\frac{\partial f(u)}{\partial x_i} = f'(u) \frac{\partial u}{\partial x_i} \implies f(u) - f(0) \in W^{1, p}(\Omega)$.

		We take $\eta \in C_0^∞(\Omega)$ and $u \in W^{1, 1}(u)$.
		$$ \int_\Omega \frac{\partial \eta}{\partial x_i} f(u) = \lim_{\epsilon \rightarrow 0_+} \int_\Omega \frac{\partial \eta}{\partial x_i} f(u_\epsilon) \overset{\text{IBP, both are smooth}} -\lim_{\epsilon \rightarrow 0_+} \int_\Omega \eta \frac{\partial f(u_\epsilon)}{\partial x_i} = $$
		$$ = -\lim_{\epsilon \rightarrow 0_+} \int_\Omega \underbrace{\eta f'(u_\epsilon)}_{\rightarrow \eta f(u_\epsilon) \text{ in } L^1, \text{ so weakly in } L^∞} · \underbrace{\frac{\partial u_\epsilon}{\partial x_i}}_{\rightarrow \frac{\partial u}{\partial x_i} \text{ in } L^1}. $$

		TODO?
	\end{dukazin}
\end{veta}

\begin{veta}[Characterization of sobolev functions]
	Let $\Omega \subseteq ®R^d$ open, bounded. Define $\Omega_\delta := \{x \in \Omega | B(x, \delta) \subseteq \Omega\}$ and $u_i^h(x) := \frac{u(x + h·e_i) - u(x)}{h}$, $h > 0, i \in [d]$.

	\begin{itemize}
		\item If $u \in W^{1, p}(\Omega)$ then $\forall \delta\ \forall h < \frac{\delta}{2}: \|u_i^h\|_{L^p(\Omega_\delta)} ≤ \left\|\frac{\partial u}{\partial x_i}\right\|_{L^p}(\Omega)$.
		\item If $p \in (1, ∞]$ and $\sup_{\delta > 0} \sup_{h < \frac{\delta}{2}} \|u_i^h\|_{L^p(\Omega_\delta)} ≤ k$, then $\frac{\partial u}{\partial x_i}$ exists and $\left\|\frac{\partial u}{\partial x_i}\right\|_{L^p(\Omega)} ≤ k$.
		\item If $p \in [1, ∞)$ and if $u \in W^{1, p}(\Omega)$ then $u_i^h \rightarrow \frac{\partial u}{\partial x_i}$ in $L_{loc}^p(\Omega)$.
		\item[(*] If $p = 1$ and $\sup_{\delta > 0} \sup_{h < \frac{\delta}{2}} \|u_i^h\|_{L^p(\Omega_\delta)} ≤ k$, then $u \in BV(\Omega)$. Moreover if $≤k$ and $u_i^h$ is equiintegrable, then $u \in W^{1, 1}(\Omega)$.)
	\end{itemize}

	\begin{dukazin}
		„Second item“ Fix $\Omega_1 \subset\subset \Omega$. Fix $\delta_0$, $\Omega_1 \subseteq \Omega_{\delta_0} \implies \|u_i^h\|_{L^p(\Omega_1)} ≤ k$. $u_i^h \rightharpoonup \overline u$ in $L^p(\Omega_1)$ and $u_i^h \rightharpoonup^* \overline u$ in $L^∞(\Omega_1)$.
		We want $\|\overline{u}\|_{L^p(\Omega_1)} ≤ \liminf_{h \rightarrow 0_+} \|u_i^h\|_{L^p(\Omega_1)} ≤ k$.

		$$ \int_{\Omega_1} \overline{u} \phi dx = \lim_{h \rightarrow 0_+} \int_\Omega u_i^h \phi = \lim_{h \rightarrow 0_+} \int_{\Omega_1} \frac{u(x + h·e_i) - u(x)}{h} \phi(x) dx = $$
		$$ = \lim_{h \rightarrow 0_+} \int_{\Omega} \frac{u(x + h·e_i)}{h} \phi(x) - \frac{u(x)}{h}\phi(x) dx = $$
		$$ = -\lim_{h \rightarrow 0_+} \int_{\Omega} u(x) \frac{\phi(x) - \phi(x - h·e_i)}{h} dx = -\int_{\Omega_1} \frac{\partial \phi}{\partial x_i} u. $$

% 20. 02. 2023 lecture was canceled

% 27. 02. 2023

		„First item“: $u_\epsilon := u * \eta_\epsilon$ (where we extend $u$ by zero).
		$$ \frac{u_\epsilon(x + h·e_i) - u_\epsilon(x)}{h} = \frac{1}{h} \int_0^1 \frac{d}{dt} u_\epsilon(x + h e_i t) dt = \int_0^1 \frac{\partial u_\epsilon(x + h·e_i·t)}{\partial x_i} dt. $$
		$$ \left|\frac{u_\epsilon(x + h·e_i) - u_\epsilon(x)}{n}\right|^p ≤ \left| \int_0^1 \frac{\partial u^\epsilon}{\partial x_i}() dt \right|^p ≤ \int_0^1 \left|\frac{\partial u_\epsilon}{\partial x_i}()\right|^p dt. $$
		$$ \int_{\Omega_\delta} \left|\frac{u_\epsilon(x + h·e_i) - u_\epsilon(x)}{h}\right|^p ≤ \int_{\Omega_\delta} \int_0^1 \left| \frac{\partial u_\epsilon}{\partial x_i}(x + h·e_i·t) \right|^p dt dx =  $$
		$$ \int_0^1 \int_{\Omega_\delta} \left| \frac{\partial u_\epsilon}{\partial x_i}(x + h·e_i·t) \right|^p dx dt ≤ \int_0^1 \int_{\Omega_{\delta / 2}} \left|\frac{\partial u_\epsilon}{\partial x_i}(x)\right|^p dx dt = \int_{\Omega_{\delta / 2}} \left| \frac{\partial u^\epsilon}{\partial x_i} \right|^p dx $$
		$$ \epsilon \rightarrow 0_+: \int_{\Omega_\delta} \left|\frac{u(x + h·e_i) - u(x)}{h}\right|^p ≤ \int_{\Omega_{\delta / 2}} \left|\frac{\partial u}{\partial x_i} (x_i)\right|^p dx. $$

		„Third item“: It is enough to show „$u_i^h$ is Cauchy“: $\epsilon > 0$, $u_\epsilon := u * \eta_\epsilon$:
		$$ u_{\epsilon, i}^{h_1} - u_{\epsilon, i}^{h_2} = \frac{u_\epsilon(x + h_1 e_i) - u_\epsilon(x)}{h} - \frac{u_\epsilon(x + h_2·e_i) - u_\epsilon(x)}{h} = \int_0^1 \frac{\partial u_\epsilon}{\partial x_i}(x + h_1·e_1 t) - \frac{\partial u_\epsilon}{\partial x_i}(x + h_2 e_i t) dt. $$
		$$ \int_{\Omega_\delta} |…|^p ≤ \int_0^1 \int_{\Omega_\delta} \left|\frac{\partial u_\epsilon}{\partial x_i}(x + h_1·e_i·t) - \frac{\partial u_\epsilon}{\partial x_i}(x + h_2·e_i·t)\right|^p dx dt, $$
		$$ \int_{\Omega_\delta} |u_i^{h_1} - u_i^{h_2}|^p ≤ \int_0^1 \int_{\Omega_\delta} | no epsilon |^p dx dt $$
	\end{dukazin}
\end{veta}

\begin{veta}[Approximation by smmooth function]
	Let $\Omega \subseteq ®R^d$ be bounded and open and $p \in [1, ∞)$. Then $\forall u \in W^{k, p}(\Omega)$

	\begin{itemize}
		\item $\exists \{u^n\}_{n=1}^∞ \subset ©C^∞(\Omega)$ such that $\|u^n - u\|_{W^{k, p}(\Omega)} \rightarrow 0$;
		\item if $\Omega \in C^0$, then $\exists \{u^n\}_{n=1}^∞ \subset C^∞(\overline{\Omega})$ such that $\|u^n - u\|_{w^{k, p}(\Omega)} \rightarrow 0$.
	\end{itemize}

	\begin{dukazin}
		„First item“ at home. „Second item“: Lemma(Partition of unite): „Let $\{\Omega_r\}_{r=1}^{M+1}$ be open covering of $\overline{\Omega}$. Then $\exists \phi_r \in C_0^∞(\Omega_r)$, such thtat $\forall x \in \overline{\Omega}: \sum_{r=1}^{M+1} \phi_r(x) = 1$.“ Proof at home.

		Define $u_r(x) := u(x) \phi_r(x)$. TODO!!!

		1) $u_{M+1}$ is supported in $\Omega_{M+1} \subseteq \Omega$, so it can be extened by $0$. So $u_{M+1}^n = u_{M+1} * \eta_{\frac{1}{n}}$. 2) $u_r$ for $r \in [M]$. Set $u_i^h(x) := u_i(x_1, …, x_{d-1}, x_d - h)$, $u_i^n := u_i * \eta_{\frac{1}{n}}$. We need $h, \epsilon$ edepending on $n$: $\|u_i^n - u_i\|_{W^{k, p}(V_1^+)} \rightarrow 0$ (because $a_i$ is continuous).
		$\phi_i \in C_0^∞\ \exists \delta' > 0$. $\phi_1$ is positive on the set $x_d < a_1(x') + \beta - \delta' \land x_d > a_1(x') - \beta + \delta'$, $h < h_0 < \delta'$. Take $(x, …, x_{d-1}, x_d - h)$, where $(x_1, …, x_d) \in \partial \Omega$, $\dist((x_1, …, x_{d-1}, x_d - h), \partial \Omega) < \delta$. Denote this $h$ as $h_{max}$, so for $h < h_{max}$ $\dist(…) < \delta$.

		Give me $\delta > 0$, I find $h_0, h_{max}$ and define $u_i^h = u_i^h * \eta^\delta$, where $h < \min(h_0, h_{max})$. Then
		$\|u_i - u_i^h\|_{W^{k, p}(V_i^+)} \rightarrow 0$, $\|u_i^h - (u_i^h)_\delta\|_{W^{k, p}(V_i^+)} \rightarrow 0$
	\end{dukazin}
\end{veta}

\begin{veta}
	Let $\Omega \in C^{0, 1}$ and $p \in [1, ∞]$. Then there exists a linear operator $E: W^{1, p}(\Omega) \rightarrow W^{1, p}(®R^d)$ such that
	\begin{itemize}
		\item $E u = u$ in $\Omega$;
		\item $\exists B_R \subset ®R^d$ such that $E u ≡ 0$ in $®R^d \setminus B_R$;
		\item $\|E u\|_{W^{1, p}(®R^d)} ≤ c(p, \Omega)·\|u\|_{W^{1, p}(\Omega)}$.
	\end{itemize}

	\begin{dukazin}
		Focus only on $V_1$ (previous proof), $u = \sum u_r$ ($u_{M+1}$ is done) and only for $u_1$.

		TODO images!!!


	\end{dukazin}
\end{veta}

\section{Proof of $W^{1, p} \hookrightarrow C^{0, d}$}
\begin{lemma}[Morrey]
	Let $u \in W^{1, 1}(B_R(0))$ and $0$ be the Lebesgue point of $u$.
	$$ \left| \fint_{B_R} u(x) dx - u(0)\right| ≤ R^A c(A, d) \sup_{\rho ≤ R} \int_{B_\rho} \frac{|\nabla u(x)|}{\rho^{d - 1 + A}} dx \qquad A > 0. $$

	\begin{dukazin}
		$$ |\fint_{B_R} u - u(0)| = \lim_{r \rightarrow 0_+}|\fint_{B_R} u - \fint_{B_r} u| = \lim_{r \rightarrow 0_+} |\int_r^R \frac{d}{d\rho} \fint_{B_\rho} u(x) dx d\rho| = \lim_{r \rightarrow 0_+} | \int_r^R \frac{d}{d\rho} \fint_{B_1(0)} u(\rho x) dx d\rho| = \lim_{r \rightarrow 0_+}|\int_r^R \fint_{B_1(0)} x·\nabla u(\rho x) dx d\rho| ≤ $$
		$$ ≤ \lim_{r \rightarrow 0_+} \int_r^R \fint_{B_1(0)}|\nabla u(\rho x)| dx d\rho = \lim_{r \rightarrow 0_+} \int_r^R \fint_{B_\rho} |\nabla u(x)| dx d\rho = \lim \int_r^R \kappa_d \int_{B_\rho} \frac{|\nabla u(x)| dx}{\rho^{d - 1 + A}}\rho^{A - 1} d\rho ≤ $$
		$$ ≤ c(d) \sup_{\rho < R} \int_{B_\rho} \frac{|\nabla u|}{\rho^{d - 1 + A}} · \lim_{r \rightarrow 0_+} \int_r^R \rho^{A - 1} d\rho = c(d, A) R^A \sup … $$
	\end{dukazin}
\end{lemma}

\begin{poznamka}
	Replace $u$ by $E u$.
\end{poznamka}

\begin{dusledek}
	$x$, $y$ Lebesgue points of $u$.
	$$ |u(x) - u(y)| ≤ |x - y|^\alpha c(\alpha, \Omega, p)\max\{I_x, I_y\}, $$
	where
	$$ I_x := \sup_{r ≤ 2|x - y|} \int_{B_r(x)} \frac{|\nabla u|}{} $$

	\begin{dukazin}
		$R := |x - y|$
		$$ |u(x) - u(y)| ≤ |\fint_{B_R(x)} u - u(x)| + |\fint_{B_R(y)} u - u(y)| + |\int_{B_R(x)} u - \fint_{B_R(y)}u| ≤ c(\rho, \alpha) r^\alpha(I_x + I_y). $$

		$$ ?| | = |\int_0^1 \frac{d}{dt} \fint_{B_R(tx + (1 - t)y)} u(z) dz| = |\int_0^1 \frac{d}{dt} \fint_{B_R(0)} u(tx + (1 - t) y + z) dz| ≤ $$
		$$ ≤ \int_0^1 \fint_{B_R(0)} |\nabla u(…)·(x - y)| dz dt ≤ c(d) \int_0^1 \int_{B_R(0)} \frac{|\nabla u(…)|}{R^{d - 1}} dz dt ≤ $$
		$$ ≤ c(d) \int_0^1 \int_{B_{2R}(x)} \frac{|\nabla u(z)| dz}{R^{d - 1}} dt = c(d) \int_{B_{2R}(x)} \frac{|\nabla u(z)| dz}{(2R)^{d - 1 + \alpha}} (2R)^\alpha ≤ c(d, \alpha) I_x (2R)^\alpha. $$

		$$ \sup_{R > 0, x \in ®R^d} I_R(x) = \sup_{R, x} \int_{B_\rho(x)} \frac{|\nabla E u(z)|}{R^{d - 1 + \alpha}} dz ≤ \sup \(\int_{B_R} |\nabla E u|^p\)^{1 / p} \(\int_{B_R} R^{(1 - d - \alpha)p'}\)^{1 / p'} ≤ c \|u\|_{W^{1, p}(\Omega)} \(R^{(1 - d - \alpha) p' + d}\)^{1 / p'} = c·\|u\|_{W^{1, p}(\Omega)}·1. $$

		It remains $\|u\|_{L^∞(\Omega)}$:
		$$ |u(x)| ≤ |u(x) - u(y)| + |u(y)| $$
		$$ |u(x)| = \fint_\Omega |u(x)| dy ≤ \fint_\Omega |u(y)| dy + K \|u\|_{W^{1, p}(\Omega)} ≤ C(\Omega) \|u\|_{1, p}, $$
		$$ \|\frac{u(x) - u(y)}{|x - y|^\alpha}\| ≤ c \|u\|_{1, p}. $$

		$x, y$ Lebesgue points, so estimates TODO?
	\end{dukazin}
\end{dusledek}

\begin{poznamka}
	$W^{1, d}(\Omega) \not \hookrightarrow C^0(\overline{\Omega})$, but $W^{1, d}(\Omega) \hookrightarrow BMO(\Omega) (VMO)$.

	$W^{d, 1}(\Omega) \hookrightarrow W^{1, d} \not\hookrightarrow C^0(\overline{\Omega})$, but $W^{d, 1}(\Omega) \hookrightarrow C^0(\overline{\Omega})$.
\end{poznamka}

% 06. 03. 2023

\begin{veta}
	Let $\Omega \in C^{0, 1}$, $p \in [1, ∞)$. Then $W^{1, p}(\Omega) \hookrightarrow L^q(\Omega)$ if

	\begin{itemize}
		\item for $p \in [1, d)$ and $q ≤ \frac{dp}{d - p}$;
		\item for $p = d$ and $q \in [1, ∞)$;
		\item for $p > d$ and $q \in [1, ∞]$.
	\end{itemize}

	And $W^{1, p}(\Omega) \hookrightarrow \hookrightarrow L^q(\Omega)$ if previous holds, except for $p \in [1, d)$ and $q = \frac{dp}{d - p}$.

	\begin{dukazin}[Scheme of the proof]
		We use extension $W^{1, p}(\Omega) \rightarrow W^{1, p}(®R^d)$ compactly supported. Mollification $W^{1, p}(®R^d) \rightarrow ©C_0^∞(®R^d)$. Show all estimates only for smooth functions.
	\end{dukazin}

	\begin{lemmain}[Gagliardo–Nirenberg inequality]
		$\exists C(d), C(d, p)$ such that $\forall u \in C_0^∞(®R^d)$:
		$$ \|u\|_{L^{\frac{d}{d - 1}}(®R^d)} ≤ C(d) \|\nabla u\|_{L^1(®R^d)}, $$
		$$ \|u\|_{L^{\frac{dp}{d - p}}(®R^d)} ≤ C(d, p) \|\nabla u\|_{L^p(®R^d)}. $$
	\end{lemmain}

	\begin{dukazin}[Proof of lemma]
		Firstly we show that first inequality implies second. Define $v := |u|^q$ for some $q > 1$. Then from first inequality $\|v\|_{\frac{d}{d - 1}} ≤ c·\|\nabla v\|_1$:
		$$ \(\int_{®R^d} |u|^{\frac{qd}{d - 1}}\)^{\frac{d - 1}{d}} ≤ C(d) \int_{®R^d} |\nabla |u|^q| ≤ c(d, q) \int_{®R^d} |u|^{q - 1} |\nabla u| \overset{\text{Hölder}}≤ $$
		$$ ≤ c(d, q) \|\nabla u\|_p·\| |u|^{q - 1} \|_{p'} = c(d, q) \|\nabla u\|_p \(\int_{®R^d} |u|^{\frac{p(q - 1)}{p - 1}}\)^{\frac{p - 1}{p}}. $$
		Choose $q$ such that $\frac{qd}{d - 1} = \frac{p(q - 1)}{p - 1}$, i. e. $q = \frac{p(d - 1)}{d - p}$.
		$$ \(\int_{®R^d} |u|^{\frac{dp}{d - p}}\)^{\frac{d - 1}{d}} ≤ C(d, p) \|\nabla u\|_p \(\int_{®R^d} |u|^{\frac{dp}{d - p}}\)^{\frac{p - 1}{p}} \implies \|u\|_{\frac{dp}{d - p}} ≤ C(d, p) \|\nabla u\|_p. $$

		Then we proof first inequality by next lemma: ($u$ is smooth, compactly supported)
		$$ u(x) = \int_{-∞}^{x_i} \frac{\partial u}{\partial x_i}(x_1, …, x_{i-1}, s, x_{i + 1}, …, x_d) ds, $$
		$$ |u(x)| ≤ \int_{-∞}^∞ |\nabla u(x_1, …, x_{i-1}, s, x_{i+1}, …, x_d)| ds. $$
		$$ |u(x)|^d dx ≤ \prod_{i=1}^d \int_{-∞}^∞ |\nabla u(x_1, …, x_{i - 1}, s, x_{i+1}, …, x_d)| ds, $$
		$$ \int_{®R^d} |u(x)|^{\frac{d}{d - 1}} dx ≤ \int_{®R^d} \prod_{i=1}^d \(\int_{-∞}^∞ |\nabla u(x_1, …, x_{i - 1}, s, x_{i+1}, …, x_d)| ds\) dx =: \int_{®R^d} \prod_{i=1}^d v_i \overset{\text{Gagliardo}}≤ $$
		$$ ≤ \prod_{i=1}^d \( \int_{®R^{d - 1}} \[ \( \int_{-∞}^∞ |\nabla u(x_1, …, s, …, x_d)| ds\)^{\frac{1}{d - 1}} \]^{d - 1} dx_1…dx_{i-1}dx_{i+1}…dx_d \)^{\frac{1}{d - 1}} ≤ $$
		$$ ≤ \prod_{i=1}^d \|\nabla u\|_{L^1(®R^d)}^{\frac{1}{d - 1}} = \|\nabla u\|_1^{\frac{d}{d - 1}}. $$
	\end{dukazin}

	\begin{lemmain}[Gagliardo]
		Let $u_i \in ©C_0^∞(®R^{d - 1})$, $i \in [d]$. Define $v_i(x_1, …, x_d) := u_i(x_1, …, x_{i - 1}, x_{i + 1}, …, x_d)$. Then
		$$ \int_{®R^d} \prod_{i=1}^d |v_i(x)| dx ≤ \prod_{i=1}^d \|u_i\|_{L^{d - 1}(®R^{d - 1})}. $$
	\end{lemmain}

	\begin{dukazin}[Proof of lemma]
		By induction: 1) „$d = 2$“:
		$$ \int_{®R^d} \prod_{i=1}^2 |v_i(x)| dx = \int_{®R^2} |u_1(x_2)|·|u_2(x_1)| dx_1 dx_2 = \|u_1\|_{L^1(®R)}·\|u_2\|_{L^1(®R)}. $$

		2) „$d \implies d+1$“:
		$$ \int_{®R^{d + 1}} \prod_{i=1}^{d + 1} |v_i(x)| dx = \int_{®R^d} |v_{d + 1}(x)| · \(\int_®R \prod_{i=1}^d |v_i(x)| dx_{d+1}\) dx_1…dx_d ≤ $$
		$$ ≤ \|v_{d+1}\|_{L^d(®R^d)} \(\int_{®R^d} \(\int_{®R} \prod_{i=1}^d |v_i(x)| dx_{d+1}\)^{d'} dx_1…dx_d\)^{\frac{1}{d}} = RHS $$
		$$ (…)^{d'} = \(\int_{®R} |v_1|·…·|v_d| dx_{d+1}\)^{d'} \overset{\text{Hölder}}≤ \(\prod_{i=1}^d \(\int_{®R} |v_i|^d dx_{d+1}\)^{\frac{1}{d}}\)^{d'}. $$
		$$ RHS ≤ \|v_{d+1}\|_{L^d} \(\int_{®R^d} \( \prod_{i=1}^d \( \int_{®R} |v_i|^d dx_{d+1}\)^{\frac{1}{d}} \)^{\frac{d}{d - 1}} dx_1…dx_d\)^{\frac{d - 1}{d}} ≤ $$
		$$ ≤ \|u_{d+1}\|_d \(\int_{®R^d} \prod_{i=1}^d \(\int_{®R} |v_i|^d dx_{d+1}\)^{\frac{1}{d - 1}} dx_1…dx_d \)^{\frac{d - 1}{d}} \overset{\text{Induction step}}≤ $$
		$$ ≤ \|u_{d + 1}\|_d·\prod_{i=1}^d \|\(\int_{®R} |v_i|^d dx_{d+1}\)^{\frac{1}{d-1}}\|_{L^{d - 1}(®R^{d - 1})}^{\frac{d - 1}{d}} = \prod_{i=1}^d\|u_i\|_{L^d} $$
	\end{dukazin}

	\begin{dukazin}
		If $p < d$ Galiardo–Nirenberg finishes $W^{1, p} \hookrightarrow L^{\frac{dp}{d - p}}$. If $p = d$, $W^{1, d} \hookrightarrow W^{1, d - \epsilon} \hookrightarrow L^{\frac{d(d - \epsilon)}{d - (d - \epsilon)}} = L^{\frac{d(d - \epsilon)}{\epsilon}}$. If $p > d$ forget G–N and use $W^{1, p} \hookrightarrow C(\overline{\Omega}) \hookrightarrow L^∞(\Omega) \hookrightarrow L^q(\Omega)$.

		„Compact embeddings:“ 1. step: $W^{1, 1}(\Omega) \hookrightarrow \hookrightarrow L^1(\Omega)$. 2. step $W^{1, p}(\Omega) \hookrightarrow \hookrightarrow L^q(\Omega)$.

		„$1 \implies 2$“: $\|u\|_q ≤ \|u\|_{\frac{dp}{d - p}}^\alpha \|u\|_1^{1 - \alpha} ≤ c \|u\|_{1, p}^\alpha \|u\|_1^{1 - \alpha}$. Let $B$ be a bounded set in $W^{1, p}(\Omega)$. Use 1. step: (for $\frac{1}{q} = 1 - \alpha + \frac{\alpha(d - p)}{dp}$, i.e. $1 ≤ q < \frac{dp}{d - p}$)
		$$ \exists \{u_i\}_{i=1}^N \subseteq W^{1, p}(\Omega)\ \forall u \in B: \min_{i\in [N]} \|u - u_i\|_{L^1} ≤ \tilde\epsilon. $$
		$$ \|u - u_i\|_q ≤ c·\|u - u_i\|_{1,p}^\alpha·\|u - u_i\|_1^{1 - \alpha} ≤ c(\alpha, B)(\tilde \epsilon)^{1 - \alpha}. $$

		„1. step“: Let $B$ be a bounded set in $W^{1, 1}(\Omega)$, $EB$ be bounded set in $W^{1, 1}(®R^d)$ and compactly supported in $B_\Omega?$.
		$$ \forall u \in EB: u_\delta := u * \eta_\delta: $$
		$$ \int_{®R^d} |u(x) - u_\delta(x)| dx = \int_{®R^d} \left| \int_{®R^d} u(x) \eta_\delta(y) - u(x + y) \eta_\delta(y) dy\right| dx ≤ $$
		$$ ≤ \int_{®R^d} \int_{®R^d} \frac{|u(x) - u(x + y)|}{|y|} \eta_\delta(y) |y| dx dy \overset{\text{we already had it}}≤ $$
		$$ ≤ \int_{®R^d} \|\nabla u\|_1 \eta_\delta(y) |y| dy ≤ \|\nabla u\|_1 \delta \int_{®R^d} \eta_\delta(y) dy  = \|\nabla u\|_1 \delta ≤ C(B) \delta. $$

		Given $\epsilon > 0$ choose $\delta > 0$, $C(B) \delta < \frac{\epsilon}{2}$ and use Arsela–Ascoli.
	\end{dukazin}

	\begin{poznamkain}[Lipschitz domain is necessary]
		$$ u = \frac{1}{(1 + |x|)^{3 / 2}}. $$
	\end{poznamkain}
\end{veta}

\subsection{Compact embedding in Bochner spaces}
\begin{lemma}[Aubin–Lions]
	$V_1 \hookrightarrow \hookrightarrow V_2 \hookrightarrow V_3$ Banach spaces. $p \in [1, ∞)$. Then $U:=\{u \in L^p(0, T, V_1) | \partial_t u \in L^1(0, T, V_3)\} \hookrightarrow \hookrightarrow L^p(0, T, V_2)$.
\end{lemma}

\begin{lemma}[Ehrling (start of proof of Aubin–Lions)]
	Let $V_1 \hookrightarrow \hookrightarrow V_2 \hookrightarrow V_3$ be Banach spaces. Then
	$$ \forall \epsilon > 0\ \exists c > 0\ \forall u \in V_1: \|u\|_{V_2} ≤ \epsilon \|u\|_{V_1} + c·\|u\|_{V_3}. $$

	\begin{dukazin}[By contradiction]
		$$ \exists u^n \in V_1: \|u^n\|_{V_2} > \epsilon\|u^n\|_{V_1} + n\|u^n\|_{V_3}. $$
		$$ u^n ≠ 0 \implies v^n := \frac{u^n}{\|u^n\|_{V_2}} \implies 1 = \|v^n\|_{V_2} > \epsilon·\|v^n\|_{V_1} + n·\|v^n\|_{V_3}. $$
		$$ v^n \rightarrow \text{ in } V_3 \implies v^n \text{ bounded in } V_1 \hookrightarrow \hookrightarrow V_2 \implies V^n \rightarrow v \text{ in } V_2 \implies $$
		$$ \implies \|v\|_{V_2} = 1 \implies v ≠ 0. \text{\lightning} $$
	\end{dukazin}
\end{lemma}

\begin{dukaz}[Aubin–Lions]
	$M \subseteq U$ bounded set: $\exists c^*\ \forall u \in M: \int_0^T \|u\|_{V_1}^p + \|\partial_t u\|_{V_3} ≤ c^*$.
	
	We want $M$ is precompact in $L^p(0, T< V_2)$ $\Leftrightarrow$
	$$ \forall \epsilon > 0\ \exists \{w_i\}_{i=k}^N\ \forall u \in M\ \exists i \in [N]: \int_0^T \|u - w_i\|_{V_2}^p ≤ \epsilon. $$

	1. Mollify with respect to time and use Arsela–Ascoli for $C^1(0, T; V_1) \hookrightarrow\hookrightarrow C^0(0, T< V_2)$.

	2. Mollification is „not far“ from $u$ in $L^1(0, T; V_3)$.

	3, Ehrling interpolation to $L^p(0, T; V_2)$.

	\begin{dukazin}[1.]
		$u \in M$ extend to $(0, 2T)$ as $\tilde u(t) = u(t)$ if $t < T$ and $\tilde u(t) = u(2T - t)$ if $t > T$.
		$$ \int_0^{2T} \|\tilde u(t)\|_{V_1}^p + \|\partial_t \tilde u(t)\|_{V_3} = 2 \int_0^T \|u\|_{V_1} + \|\partial_t u\|_{V_3} ≤ 2C^*. $$
		$\forall 0 < \delta < T$ and $t \in (0, T)$, $u_\delta(t) = \int_0^\delta \tilde u(t + s) \phi_\delta(s) ds = \int_®R \tilde u(s) \phi_\delta(s - t)$, where $\phi \in C_0^∞(0, 1)$, $\phi ≥ 0$.

		$$ \|u_\delta(t)\|_{V_1} ≤ \frac{c}{\delta} \int_0^{2T} \|\tilde u\|_{V_1} ≤ \frac{c·c^*}{\delta}. $$
		$$ \|\partial_t u_\delta(t)\|_{V_1} ≤ \int_{®R} \|\tilde u\|_{V_1} |\phi_\delta'| ≤ c(\delta)·c^*. $$
		$M_\delta := \{u_\delta, u \in M\}$ $\implies$ $M_\delta$ is bounded in $C^1(0, T; V_1)$. $\forall \tilde \epsilon > 0\ \exists \{w_k\}_{k=1}^N \subseteq L^p(0, T; V_1)$ such that $\forall u_\delta \in M_\delta$ $\exists k: \int_0^T \|w_\delta - w_k\|_{V_2}^p < \tilde \epsilon$.
	\end{dukazin}

	\begin{dukazin}[2.]
		$$ u(t) - u_\delta(t) = u(t) - \int_0^\delta \tilde u(t + s) \phi_\delta(s) ds = $$
		$$ = \int_0^\delta (u(t) - \tilde u(t + s)) \phi_\delta(s) ds = - \int_0^\delta (u(t) - \tilde u(t + s)) \frac{d}{ds} \(\int_s^\delta \phi_\delta(\tau) d\tau\) ds = $$
		$$ = - \int_0^\delta \partial_t \tilde u(t + s) \int_s^\delta \phi_\delta(\tau) d\tau ds = - \int_0^\delta \int_0^\tau \partial_t \tilde u(t + s)\phi_\delta(\tau) ds d\tau. $$

		$$ \int_0^T \|u(t) - u_0(t)\|_{V_3} dt ≤ \int_0^T \int_0^\delta \int_0^\tau \|\partial_t \tilde u(t + s)\|_{V_3} \phi_\delta(\tau) ds d\tau dt ≤ \int_0^T \int_0^\delta \int_0^\delta \|\partial_t \tilde u(t + s)\|_{V_3} \phi_\delta(\tau) ds d\tau dt ≤ c^* \int_0^\delta \int_0^\delta \phi_\delta(\tau) ds d\tau = \delta c^*. $$
		$$ \|u(t) - u_\delta(t)\|_{V_3} ≤ \int_0^\delta \int_0^\tau \|\partial_t u(t + s)\|_{V_3} \phi_\delta(\tau) ds d\tau ≤ c^*. $$
	\end{dukazin}

	\begin{dukazin}[3.]
		$$ \int_0^T \|u - w_k\|_{V_2}^p \overset{\text{Ehrling}}≤ \tilde{\tilde \epsilon} \int_0^T \|u - w_k\|_{V_1}^p + c(\tilde{\tilde \epsilon}) \int_0^T \|u - w_k\|_{V_3}^p ≤ k(c^*) \tilde{\tilde\epsilon} + c(\tilde{\tilde\epsilon}) \int_0^T \|u - w_k\|_{V^3}^p ≤ $$
		$$ ≤ k(c^*) \tilde{\tilde\epsilon} + c(\tilde{\tilde\epsilon}, p) \int_0^T \|u - u_\delta\|_{V_3}^p ≤ $$
		$$ ≤ k(c^*) \tilde{\tilde\epsilon} + c(\tilde{\tilde\epsilon}, p) \sup_{t \in (0, T)}\{\|u(t) - u_\delta(t)\|_{V_3}^{p - 1}\} \int_0^T \|u - u_\delta\|_{V_3} + -||- ≤ $$
		$$ ≤ k(c^*) \tilde{\tilde\epsilon} + c(c^*, p, \tilde{\tilde\epsilon}) \delta + c(\tilde{\tilde\epsilon}, p, c^*) \int_0^T \|u_\delta - w_k\|_{V_3}^p. $$
		Find $\tilde{\tilde\epsilon}$ such that $k(c^*) \tilde{\tilde\epsilon} ≤ \frac{\epsilon}{3}$. Find $\delta > 0$ such that $c(c^*, p, \epsilon) < \frac{\epsilon}{3}$. Find $N$-covering $\{U_i\}_{i=1}^N$ such that $\min_k c(\tilde{\tilde\epsilon}, p, c^*) \int_0^T \|u_\delta - w_k\| ≤ \frac{\epsilon}{3}$.
	\end{dukazin}
\end{dukaz}

\subsection{Trace theorem}

\end{document}
