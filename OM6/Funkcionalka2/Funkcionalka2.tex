\documentclass[12pt]{article}					% Začátek dokumentu
\usepackage{../../MFFStyle}					    % Import stylu



\begin{document}

% 13. 02. 2023

\begin{poznamka}[Exam]
	Oral, similar as in FA1.
\end{poznamka}

\begin{poznamka}[Credit]
	Similar as in FA1.
\end{poznamka}

\section{Banach algebras}
\subsection{Basic properties}
\begin{definice}[Algebra]
	$(A, +, -, 0, ·_S, ·)$ is algebra over ®K, if

	\begin{itemize}
		\item $(A, +, -, 0, ·_S)$ is vector space over ®K;
		\item $(A, +, -, 0, ·)$ is ring (that is we have $a·(b + c) = a·b + a·c$, $(a + b)·c = a·c + b·c$ and $(a·b)·c = a·(b·c)$);
		\item $\forall \lambda \in ®K\ \forall x, y \in A: \lambda(x·y) = (\lambda x)y = x(\lambda y)$.
	\end{itemize}
\end{definice}

\begin{dusledek}
	1) $e \in A$ is left unit $≡$ $e·a = a$, right unit $≡$ $a·e = a$, unit $≡$ $a·e = e·a = a$ ($\forall a \in A$).

	If $e_1$ is left unit and $e_2$ is right unit, then $e_1 = e_2$ is unit. ($e_1 = e_1·e_2 = e_2$)

	2) (Algebra) homomorphism $\phi: A \rightarrow B$ $≡$ $\phi$ preserves $+, ·, ·_S$, that is $\phi(x + y) = \phi(x) + \phi(y)$, $\phi(x·y) = \phi(x)·\phi(y)$ and $\phi(\alpha·x) = \alpha·\phi(x)$.
\end{dusledek}

\begin{tvrzeni}
	Let $A$ be algebra over ®K. Put $A_e = A \times ®K$ with operations $A_e$ defined coordinate-wise and multiplication defined by
	$$ (a, \alpha)·(b, \beta) := (a·b + \alpha·b + \beta·a, \alpha·\beta), \qquad a, b \in A \land \alpha, \beta \in ®K. $$

	Then $A_e$ is algebra with a unit $(¦o, 1)$ and $A ≡ A \times \{0\} \subset A_e$. Moreover, if $A$ is commutative, then $A_e$ is commutative.

	\begin{dukazin}
		We have $A_e$ is vector space (from linear algebra). We easy proof from definition, that $A_e$ is algebra, $(¦o, 1)$ is a unit in $A_e$ and on $A \times \{0\}$ we have $(a, 0)·(b, 0) = (a·b, 0)$, so $a \mapsto (a, 0)$ is homomorphism. Commutativity is easy too.
	\end{dukazin}
\end{tvrzeni}

\begin{definice}[Normed algebra]
	$(A, \|·\|)$ is normed algebra $≡$ $A$ is algebra and $(A, \|·\|)$ is NLS and $\|a·b\| ≤ \|a\|·\|b\|$ ($\forall a, b \in A$).
\end{definice}

\begin{definice}[Banach algebra]
	$(A, \|·\|)$ is Banach algebra $≡$ $(A, \|·\|)$ is normed algebra nad Banach space.
\end{definice}

\begin{priklady}
	$l_∞(I)$ is commutative Banach algebra with a unit (all ones).

	If $T$ is Hausdorff topological space, then $©C_b(T) = \{f: T \rightarrow ®K | f \text{ is continuous and bounded}\} \subseteq l_∞(T)$ is closed subalgebra.

	If $T$ is locally compact, Hausdorff, not compact. Then $©C_0(T) = \{f: T \rightarrow ®K \text{ continuous } | \forall \epsilon > 0: \{t \in T | |f(t)| ≥ \epsilon\} is \text{ compact}\} \subseteq ©C_b(T)$ is closed subalgebra, which doesn't have unit.

	If $X$ is Banach, $\dim X > 1$, then $©L(X)$, with $S·T := S \circ T$, $S, T \in ©L(X)$, is Banach algebra with unit (identity), which isn't commutative.

	If $X$ is Banach, $\dim X = +∞$, then $©K(X) \subset ©L(X)$ is closed subalgebra which is not commutative and doesn't have unit.

	$(L_1(®R^d), *)$, where $*$ is convolution, is (commutative) Banach algebra (without unit).

	$(l_1(®Z), *)$, where $x*y(n) = \sum_{k=-∞}^{+∞} x_k y_{n - k}$ is (commutative) Banach algebra (with unit).
\end{priklady}

\begin{tvrzeni}
	If $(A, \|·\|)$ is normed algebra, then $·: A \oplus_∞ A \rightarrow A$ is Lipschitz on bounded sets.

	\begin{dukazin}
		$$ \forall r > 0: \forall (a, b) \in B_{A \oplus_∞ A}(¦o, r)\ \forall (c, d) \in B_{A \oplus_∞ A}(¦o, r): $$
		$$ \|ab - cd\| ≤ \|a(b - d)\| + \|(a - c)·d\| ≤ \|a\|·\|b - d\| + \|a - c\|·\|d\| ≤ R·(\|b - d\| + \|a - c\|) ≤ 2R \|(a, b) - (c, d)\|. $$
	\end{dukazin}
\end{tvrzeni}

\begin{tvrzeni}
	Let $(A, \|·\|)$ be a Banach algebra. On $A_e$ we consider the norm
	$$ \|(a, \alpha)\| := \|a\| + |\alpha|, \qquad (a, \alpha) \in A \times ®K = A_e. $$
	Then $(A_e, \|·\|)$ is Banach algebra.

	\begin{dukazin}
		It is a Banach space, because $A_e = A \oplus_1 ®K$. Now we need only check, that
		$$ \|(a, \alpha)·(b, \beta)\| ≤ \|(a, \alpha)\|·\|(b, \beta)\|, $$
		which is easy.
	\end{dukazin}
\end{tvrzeni}

\begin{poznamka}
	There is more (natural) ways to define norm on $A_e$ (unlike $·$ on $A_e$, which is natural).

	$A$ has a unit … we may still consider $A_e$.

	If $e \in A \setminus \{¦o\}$ is a unit, then $\|e\| ≥ 1$, because $\|e\| = \|e^2\| ≤ \|e\|^2$.
\end{poznamka}

\begin{veta}
	Let $A$ be a Banach algebra, for $a \in A$ consider $L_a \in ©L(A)$ defined as $L_a(x) := a·x$, $x \in A$. Then $I: A \rightarrow ®L(A)$, $a \mapsto L_a$ is continuous algebra homomorphism, $\|I\| ≤ 1$.

	Moreover, if $A$ has a unit $e$, then $I$ is isomorphism into and $I(e) = \id$.

	If $\|x^2\| = \|x\|^2$, $x \in A$, then $I$ is isometry into.

	\begin{dukazin}
		„$L_a \in ©L(A)$ and $I \in ©L(A, ©L(A))$, $\|I\| ≤ 1$“: Linearity is obvious, $\|L_a(x)\| = \|a·x\| ≤ \|a\|·\|x\|$, so $\|L_a\| ≤ \|a\|$ and so $\|I\| ≤ 1$. Since it is easily $I$ preserves multiplication, so we are left to prove the „Moreover“ part.

		„$A$ has a unit $e$“: WLOG $A ≠ \{¦o\}$.
		$$ \forall a \in A: \|I a\| = \|L_a\| ≥ \|L_a\(\frac{e}{\|e\|}\) = \frac{a}{\|e\|} = \frac{1}{\|e\|}·a. $$
		So $I$ is bounded from below, so $I$ is isomorphism.

		$$ I(e)(x) = L_e(x) = x, \text{ so } I(e) = \id. $$

		Finally, if $\|x^2\| = \|x\|^2$, $x \in A$, then $\forall a \in A:$
		$$ \|a\| ≥ \|I(a)\| = \|L_a\| ≥ \|L_a \(\frac{a}{\|a\|}\)\| = \frac{\|a^2\|}{\|a\|} = \|a\|. $$
		So $I$ is isometry.
	\end{dukazin}
\end{veta}

\begin{poznamka}
	$A ≠ \{¦o\}$ Banach algebra with a unit $\implies$ $\exists$ equivalent norm $\|·\|$ on $A$ such that $(A, \|·\|)$ is Banach algebra and $\|e\| = 1$.

	\begin{dukazin}
		Let $I: A \rightarrow ©L(A)$ be as before. Put $|\|x\| | := \|I(x)\|$, $x \in A$. Since $I$ is isomorphism, $|\|·\| |$ is equivalent norm. Moreover, $|\|x·y\| | = \|I(x·y)\| ≤ \|I(x)\|·\|I(y)\| = |\|x\| | · |\|y\| |$, $x, y \in A$. So $(A, |\|·\| |)$ is a Banach algebra. Finally
		$$ |\|e\| | = \|I(e)\| = \|\id\| = 1. $$
	\end{dukazin}
\end{poznamka}

\subsection{Inverse elements}
\begin{definice}
	$(M, ·, e)$ is monoid ($·$ is associative, $e$ is unit). Then invertible elements form a group ($e^{-1} = e, \exists x^{-1}, y^{-1} \implies (x·y)^{-1} = y^{-1}·x^{-1}$); if $x \in M$, and $y \in M$ is its left inverse and $z \in M$ is its right inverse, then $y = z$ is inverse:
	$$ y = y·e = y·x·z = e·z = z. $$

	We denote $M^\times := \{x \in M\ |\ \exists x^{-1}\}$
\end{definice}

% 14. 02. 2023

\begin{tvrzeni}
	If $(A, ·, e)$ is monoid and $x_1, …, x_n \in A$ commute, then $x_1·…·x_n \in A^x \Leftrightarrow \{x_1, …, x_n\} \subset A^x$.

	\begin{dukazin}
		It suffices to prove it for $n=2$ (and use induction). „If $x^{-1}$ and $y^{-1}$ exists, then $(xy)^{-1}$“ is easy from asociativity.

		If we have $(xy)^{-1}$. Put $z:= (xy)^{-1}x$. Then $zy = (xy)^{-1}(xy) = e$, so $z$ is left inverse to $y$. Next we show that there is also right inverse: Put $\tilde z := x(xy)^{-1}$: $y\tilde z = (xy)(xy)^{-1} = e$, so $\tilde z$ is right inverse. And we already know that if there is left and right inverse, then they are same and they are inverse.
	\end{dukazin}
\end{tvrzeni}

\begin{lemma}
	Let $A$ be a Banach algebra with a unit.

	\begin{itemize}
		\item $\|x\| < 1 \implies \exists(e - x)^{-1} \land (e - x)^{-1} = \sum_{n=0}^∞ x^n$;
		\item $\exists x^{-1} \land \|h\| < \frac{1}{\|x^{-1}\|} \implies \exists(x + e)^{-1} \land \|(x + h)^{-1} - x^{-1}\| ≤ \frac{\|x^{-1}\|^2·\|h\|}{1 - \|x^{-1}\|·\|h\|}$.
	\end{itemize}

	\begin{dukazin}
		„First item“: We have $\|x^n\| ≤ \|x\|^n$, so $\sum_{n=0}^∞ x^n$ is absolute convergent series, so $\sum_{n=0}|^∞ x^n \in A$. Moreover,
		$$ (e - x)·\(\sum_{n=0}^∞ x^n\) = \lim_{N \rightarrow ∞} (e - x)·(e + x + … + x^N) = \lim_{N \rightarrow ∞} e - x^{N+1} = e, $$
		because $\lim_{N \rightarrow ∞} \|x^{n+1}\| ≤ \lim_{N \rightarrow ∞} \|x\|^N = 0$. And similarly $\(\sum x^n\)·(e - x) = e$.

		„Second item“: $x + h = x·(e + x^{-1}h)$ we have $x^{-1}$ exists and $(e + x^{-1}h)^{-1}$ exists (from first item), so from previous fact $(x + h)^{-1}$ exists. Moreover
		$$ (x + h)^{-1} = (e + x^{-1} h)^{-1}·x^{-1} \overset{1)}= \sum_{n=0}^∞ \(-x^{-1} h\)^n x^{-1}, $$
		so
		$$ \|(x + h)^{-1} - x^{-1}\| = \|\sum_{n=1}^∞\(-x^{-1} h\)^n x^{-1}\| ≤ \|x^{-1}\|·\sum_{n=1}^∞ \|x^{-1} h\|^n ≤ $$
		$$ ≤ \|x^{-1}\| \sum_{n=1}^∞ \sum_{n=1}^∞\(\|x^{-1}\|·\|h\|\)^n = \|x^{-1}\| · \frac{\|x^{-1}\| \|h\|}{1 - \|x^{-1}\|·\|h\|}. $$
	\end{dukazin}
\end{lemma}

\begin{dusledek}
	$A$ Banach algebra with a unit $\implies$ $A^x \subset A$ is open and $A^x$ is topological group.

	\begin{dukazin}
		$A^x \subset A$ is open by previous lemma (second item). So it remains to prove $x \mapsto x^{-1}$ is continuous:
		$$ A^x \ni x_n \rightarrow x \in A^x \overset{?}\implies x_n^{-1} \rightarrow x^{-1}. $$
		$$ \|x_n^{-1} - x^{-1}\| \overset{h:= x_n - x}≤ \frac{\|x^{-1}\|^2·\|x_n - x\|}{1 - \|x^{-1}\|·\|x_n - x\|} \rightarrow 0. $$
	\end{dukazin}
\end{dusledek}

\subsection{Spectral theory}
\begin{definice}[Resolvent set, spectrum and resolvent]
	$A$ Banach algebra with a unit, $x \in A$. We define resolvent set of $x$ as $\rho_A(x) := \{\lambda \in ®K | \exists(\lambda·e - x)^{-1}\}$. Next we define spectrum of $x$ as $\sigma_A(x) := ®K \setminus \rho_A(x)$. Finally we define resolvent of $x$ as $R_x: \rho(x) \rightarrow A$, $R_x(\lambda) := (\lambda·e - x)^{-1}$.

	If $A$ doesn't have a unit, then notions above are defined with respect to $A_e$.
\end{definice}

\begin{tvrzeni}
	$A$ Banach algebra

	\begin{itemize}
		\item[a)] $\forall x \in A: 0 \in \sigma_{A_e}(x)$ (in particular, if $A$ has no unit, then $0 \in \sigma_A(x)$);
		\item[b)] $A$ has unit $\implies$ $\sigma_{A_e}(x) = \sigma_A(x) \cup \{0\}$.
	\end{itemize}

	\begin{dukazin}[a)]
		$$ \forall(b, \beta) \in A_e: (x, 0)·(b, \beta) = (…, 0) ≠ (¦o, 1) \implies \nexists (x, 0)^{-1} \implies 0 \in \sigma_{A_e}(x). $$
	\end{dukazin}

	\begin{dukazin}[b)]
		By a) we have $0 \in \sigma_{A_e}(x)$. So it suffices: $\forall \lambda ≠ 0: \lambda \in \rho_A(x) \Leftrightarrow \lambda \in \rho_{A_e}(x)$. First means $(\lambda·e - x)^{-1}$ exists in $A$ and second means that $((0, \lambda) - (x, 0))^{-1} = (-x, \lambda)^{-1}$ exists in $A$. We take „$x \rightarrow -x$“.

		„$\implies$“: find $(b, \beta) \in A_e$ such that $(x, \lambda)·(b, \beta) = (¦o, 1)$. So $(x·b + \lambda·b + \beta·x, \lambda·\beta) = (¦o, 1)$. So $\beta = \frac{1}{\lambda}$ and $b = -\frac{1}{\lambda}(\lambda e + x)^{-1}·x$. Similarly we find left inverse $\(-\frac{1}{\lambda}x(x + \lambda e)^{-1}, \frac{1}{\lambda}\)(x, \lambda)$. And next we prove that they are really inverses.

		„$\impliedby$“: Put $(b, \beta) := (x, \lambda)^{-1}$. Then $(\lambda e + x)^{-1} = b + \beta·e$. We have $(x, \lambda)·(b, \beta) = (¦o, 1)$, so $\lambda·\beta = 1$ and $x·b + \lambda·b + \beta·x = ¦o$. Then
		$$ (\lambda e + x)·(b + \beta·e) = \lambda·b + \lambda·\beta·e + x·b + \beta·x = e. $$
		Similarly second inverse.
	\end{dukazin}
\end{tvrzeni}

% 20. 02. 2023

\begin{veta}
	$\{¦o\} ≠ A$ complex Banach algebra, $x \in A$. Then $\sigma(x) \subseteq B_{®C}(0, \|x\|)$ is compact, nonempty.

	\begin{dukazin}
		After theory.
	\end{dukazin}
\end{veta}

\begin{definice}[Derivative]
	$Y$ Banach space, $\Omega \subset ®K$, $f: \Omega \rightarrow Y$, $a \in \Omega$. Then
	$$ f'(a) := \lim_{x \rightarrow a} \frac{f(x) - f(a)}{x - a} $$
	is the derivative of $f$ at $a$.
\end{definice}

\begin{tvrzeni}[Fact]
	$Y$ Banach, $\Omega \subset ®K$, $f: \Omega \rightarrow Y$, $a \in \Omega$. Then $f'(a)$ exists $\implies$ $f$ is continuous at $a$ $\land$ $\forall x^* \in Y^*$: $(x^* \circ f)'(a) = x^*(f'(a))$.

	\begin{dukazin}
		Continuity: $\lim_{x \rightarrow a} f(x) - f(a) = \lim_{x \rightarrow a} \frac{f(x) - f(a)}{x - a}·(x - a) = f'(a)·0 = 0$.

		$x^* \in Y^*$ given, then
		$$ \lim_{x \rightarrow a} \frac{x^*(f(x)) - x^*(f(a))}{x - a} = \lim_{x \rightarrow a} x^*\(\frac{f(x) - f(a)}{x - a}\) = x^*(f'(a)). $$
	\end{dukazin}
\end{tvrzeni}

\begin{tvrzeni}
	$A$ Banach algebra with a unit, $x \in A$. Then

	\begin{itemize}
		\item $\rho(x)$ is open set;
		\item $\forall |\lambda| > \|x\|: \lambda \in \rho(x) \land R_x(\lambda) = \sum_{n=0}^∞ \frac{x^n}{\lambda^{n + 1}}$;
		\item! $\rho(x) \ni \lambda \mapsto R_x(\lambda)$ has derivative at each $\lambda \in \rho(x)$;
		\item $\forall \mu, \nu \in \rho(x): R_x(\mu)·R_x(\nu) = R_x(\nu)·R_x(\mu)$;
		\item $\forall \mu, \nu \in \rho(x): R_x(\mu) - R_x(\nu) = (\nu - \mu)·R_x(\mu)·R_x(\nu)$.
	\end{itemize}

	\begin{dukazin}
		First is proved by lemma. Second by lemma we have
		$$ (\lambda e - x)^{-1} = \lambda^{-1}\(e - \frac{x}{\lambda}\)^{-1} = \lambda^{-1} \sum_{n=0}^∞ \(\frac{x}{\lambda}\)^n. $$

		For third we fix $\lambda \in \rho(x)$ and $t \in (0, \delta)$ for $\delta$ small enough ($\lambda + t \in \rho(x)$ and *). We shall prove that „$R_x'(\lambda) = -R_x(\lambda)^2$“:
		$$ 0 \overset?= \left\|\frac{R_x(\lambda + t) - R_x(\lambda)}{t} + R_x(\lambda)^2\right\| = \frac{1}{|t|}\left\|(\lambda e - x + t e)^{-1} - (\lambda e - x)^{-1} + (\lambda e - x)^{-1}·t·(\lambda e - x)^{-1}\right\| ≤ $$
		$$ \overset{* \text{ for existence of the inverse}}≤ \frac{1}{|t|}\|(\lambda e - x)^{-1}\|·\left\|(e + t(\lambda e - x)^{-1})^{-1} - e + (\lambda e - x)^{-1}·t\right\| = $$
		$$ = \frac{1}{|t|} \|(\lambda e - x)^{-1}\|·\left\|\sum_{n=0}^∞(-t)^n (\lambda e - x)^{-n} - e + (\lambda e - x)^{-1}·t\right\| ≤ $$
		$$ \overset{\|x^n\| ≤ \|x\|^n}≤ \frac{1}{|t|} \|(\lambda e - x)^{-1}\| · \sum_{n=2}^∞ \|t(\lambda e - x)^{-1}\|^n = $$
		$$ = \frac{1}{|t|} \|t(\lambda e - x)^{-1}\| · \frac{\|t(\lambda e - x)^{-1}\|^2}{1 - \|t(\lambda e - x)^{-1}\|} \overset{* \text{ for denominator }≤1/2}≤ \frac{2|t|^2}{|t|} \|t(\lambda e - x)^{-1}\| \rightarrow 0. $$

		Fourth: In general $u v = v u \implies u^{-1}v^{-1} = v^{-1}u^{-1}$ (proof: $u^{-1}v^{-1} = (v u)^{-1}$). And we apply it for $u = (\mu e - x)$ and $v = (\nu e - x)$.

		Fifth: In general $u·v = v·u \implies u^{-1}·v = v·u^{-1}$ (proof: $u^{-1} v = v·v^{-1}u^{-1}v = v·u^{-1} v^{-1} v = v u^{-1}$) so:
		$$ R_x(\mu) - R_x(\nu) = R_x(\mu)R_x(\nu)(R_x(\nu)^{-1}) - R_x(\mu)(R_x(\mu))^{-1}R_x(\nu) = R_x(\mu)R_x(\nu)(R_x(\nu)^{-1}) - R_x(\mu)(R_x(\mu))R_x(\nu)^{-1} = $$
		$$ = R_x(\mu) R_x(\nu) \(R_x(\nu)^{-1} - R_x(\mu)^{-1}\) = R_x(\mu) R_x(\nu) (\nu - \mu). $$
	\end{dukazin}
\end{tvrzeni}

\begin{veta}[Liouville for Banach space valued functions]
	$Y$ Banach space over ®C, $f: ®C \rightarrow Y$ has derivative at each point, $f$ is bounded ($≡$ $\|f\|$ is bounded). Then $f ≡ \const$.

	\begin{dukazin}
		Assume $f \not≡ \const$, so there are $a ≠ b \in ®C: f(a) ≠ f(b) \implies$ (by Hahn–Banach theorem) $\exists x^* \in Y^*: x^*(f(x)) ≠ x^*(f(x))$. From fact $x^* \in f: ®C \rightarrow ®C$ has derivative at each point is bounded, not constant which is in contradiction with Liouville theorem for complex valued functions.
	\end{dukazin}
\end{veta}

\begin{dukaz}[Theorem before theory]
	First case: „$A$ has a unit“: Then $\sigma(x) \subseteq B_{®C}(0, \|x\|)$ is closed, so $\sigma(x)$ is compact. Assume that $\rho(x) = ®C$. By previous tvrzeni we have $R_x: ®C \rightarrow A$ has derivative everywhere, and it is bounded because $\lim_|\lambda| \rightarrow ∞ R_x(\lambda) = \lim_{|\lambda|\rightarrow∞} \sum_{n=0}^∞ \frac{x^n}{\lambda^{n + 1}} = 0$. From previous theorem $R_x ≡ \const$ so $\lim_{|\lambda| \rightarrow ∞} R_x(\lambda) = 0 \implies R_x ≡ 0$. In particular $0 = R_x(0) = (-x)^{-1}$. \lightning (If $A ≠ \{0\}$ then $x^{-1} ≠ 0$ for $x \in A$.)

	Second case: „$A$ hasn't a unit“, then $\sigma(x) := \sigma_{A_e}((x, 0))$ so we apply the already proven case.
\end{dukaz}

\begin{poznamka}[Convention]
	If not said otherwise, in chapter about Banach algebras, all Banach spaces are complex.
\end{poznamka}

\begin{veta}[Gelfand–Mazur]
	$\{¦o\} ≠ A$ Banach algebra with a unit. Assume $\forall x \in A \setminus \{¦o\}: \exists x^{-1}$. Then $A$ is isomorphic to ®C. If moreover $e$ is a unit in $A$ and $\|e\| = 1$, then $A$ is isometrically isomorphic to ®C.

	\begin{dukazin}
		Consider $\psi: ®C \rightarrow A$ defined as $\psi(\lambda) := \lambda·e$. This is algebraic homomorphism and $\|\psi(\lambda)\| = |\lambda|·\|e\|$, so it is isomorphism (and isometry, if $\|e\| = 1$).

		It remains „$\phi$ is surjective“: Pick $a \in A$. From previously proved theorem $\exists \lambda \in \sigma(a)$, then $(\lambda e - a) \notin A^x$. So, $\lambda·e - a = 0$, then $\psi(\lambda) = a$.
	\end{dukazin}
\end{veta}

\begin{definice}[Spectral radius]
	A Banach algebra, $x \in A$. Then $r(x) := \sup\{|\lambda|, \lambda \in \sigma(x)\}$ is called spectral radius of $x$.
\end{definice}

\begin{veta}[Beurling–Gelfand]
	$A$ Banach algebra, $x \in A$ $\implies$ $r(x) = \inf_{n \in ®N} \sqrt[n]{\|x^n\|} = \lim_n \sqrt[n]{\|x^n\|}$.
\end{veta}

\begin{lemma}
	$A$ Banach algebra with a unit, $x \in A$. For $p(z) = \sum_{j=1}^n \alpha_j z^j \in ®C$ a polynom (with complex coefficients) we put $p(x) = \sum_{j=1}^n \alpha_j x^j \in A$. Then $\sigma(p(x)) = p(a(x))$.

	\begin{dukazin}
		Fix $\lambda \in ®C$ and write $(\lambda - p)(z) = c·\prod_{i=1}^m(z - z_i)$, where $z_1, …, z_m$ are roots of $\lambda - p$. Then $\lambda \in \sigma(p(x)) \Leftrightarrow (\lambda e - p(x))^{-1}$ does not exists. $(\lambda e - p(x))^{-1} = c·\prod_{i=1}^m (x - z_i·e)$, so it does'nt exists if and only if $\exists i \in [m]$, such that $(x - z_i·e)^{-1}$ doesn't exists $\Leftrightarrow$ $z_i \in \sigma(x)$ $\Leftrightarrow$ $\exists$ root $\nu$ of $\lambda - p$ such that $\nu \in \sigma(x)$ $\Leftrightarrow$ $\exists \nu \in \sigma(x): p(\nu) = \lambda$ $\Leftrightarrow$ $\lambda \in p(\sigma(x))$.
	\end{dukazin}
\end{lemma}

\begin{dukaz}[Beurling–Gelfand]
	WLOG $A$ has a unit. Step 1, „$r(x) ≤ \inf_n \sqrt[n]{\|x^n\|}$“: fix $\lambda \in \sigma(x)$. By previous lemma $\forall n: \lambda^n \in \sigma(x^n)$. By theorem 'Before theory' we have $\forall n: |\lambda|^n ≤ \|x^n\|$.

	Step 2, „$r(x) ≥ \limsup_n \sqrt[n]{\|x^n\|}$“:
%
% 21. 02. 2023
%
	Pick $r > r(x)$. Claim: „$\frac{x^n}{r^n} \rightarrow^w 0$“: Fix $x^* \in A^*$ and put $f(\lambda) := \lambda · x^*(R_x(\lambda))$. By fact and tvrzeni after it, $f$ has derivative at each $\lambda \in \rho(x)$. Moreover for $|\lambda| ≥ \|x\|$ we have $f(\lambda) = \lambda · x^*\(\sum_{n=0}^∞ \frac{x^n}{\lambda^{n + 1}}\) = \sum_{n=0}^∞ \frac{x^*(x^n)}{\lambda^n}$. Thus $f(\lambda) = \sum_{n=0}^∞ \frac{x^*(x^n)}{\lambda^n}$, $\lambda \in P(0, r(x), ∞)$. From Complex analysis $f \in H(P(0, r, ∞))$ is uniquely given by Laurent series. In particular $f(r) = \sum_{n=0}^∞ \frac{x^*(x^n)}{r^n}$, so $x^*\(\frac{x^n}{r^n}\) \rightarrow 0$.

	From princip of unique boundedness (last semester): $\frac{x^n}{r^n}$ if $\|·\|$-bounded, so $\exists c > 0:$ $\|x^n\| ≤ c r^n$, $\sqrt[n]{\|x^n\|} ≤ \sqrt[n]{c}·r \rightarrow r$. So $\limsup \sqrt[n]{\|x^n\|} ≤ r$.
\end{dukaz}

\begin{dusledek}
	$A$ Banach algebra, $x \in A$ and $|\lambda| > r(x)$. Then $\sum_{n=1}^∞ \frac{x^n}{\lambda^n}$ is absolutely convergent and $R_x(\lambda) = \sum_{n=0}^∞ \frac{x^n}{\lambda^{n + 1}}$.

	\begin{dukazin}
		Fix $q$, such that $\frac{r(x)}{|\lambda|} < q < 1$. By previous theorem, $\exists n_0\ \forall n ≥ n_0: \frac{\sqrt[n]{\|x^n\|}}{\lambda} < q$, so $\frac{\|x^n\|}{|\lambda|^n} < q^n$, $n ≥ n_0$. Thus $\sum \left\| \frac{x^n}{\lambda^n}\right\| ≤ ∞$, so the sum is absolutely convergent.

		Now we easily check that $(\lambda e - x)^{-1} = \sum_{n=0}^∞ \frac{x^n}{\lambda^{n + 1}}$.
	\end{dukazin}
\end{dusledek}

\subsection{Subalgebra}
\begin{veta}
	$A$ Banach algebra with a unit $e$, $B \subset A$ is closed subalgebra such that $e \in B$. Fix $x \in B$. Then

	\begin{itemize}
		\item $C \subset \rho_A(x)$ is component (maximum connected subset) $\implies$ $C \subseteq \sigma_B(x)$ or $C \cap \sigma_B(x) = \O$;
		\item $\partial \sigma_B(x) \subseteq \sigma_A(x) \subseteq \sigma_B(x)$;
		\item $\rho_A(x)$ is connected $\implies$ $\sigma_A(x) = \sigma_B(x)$;
		\item $\Int \sigma_B(x) = \O \implies \sigma_A(x) = \sigma_B(x)$.
	\end{itemize}

	\begin{dukazin}
		„$\sigma_A(x) \subseteq \sigma_B(x)$“: $(\lambda e - x)^{-1}$ exists in $B$ implies it exists (it's same) in $A$.

		„First item“: Let $C \subset \rho_A(x)$ be component. Pick $\lambda_0 \in C \cap \sigma_B(x)$. Wanted: „$C \setminus \sigma_B(x) = \O$“. Pick $x^* \in A^*: x^*|_B = 0 \land x^*(R_x(\lambda)) = 1$ (separate $B$ and $R_x(\lambda) \notin B$). Then $C \ni \lambda \mapsto x^*(R_x(\lambda))$ is holomorphic function on open (because maximum) connected set $C$. Which is zero\footnote{For $\lambda \in C \setminus \sigma_B(x)$, $(\lambda e - x)^{-1}$ exists in $B$ so $R_x(\lambda) \in B$ and therefore, $x^*(R_x(\lambda)) = 0$} on $C \setminus \sigma_B(x)$.

		Since $C \setminus \sigma_B(x)$ is open, if it is nonempty it contains a ball, so it has cluster point. Thus $C \ni \lambda \mapsto x^*(R_x(\lambda))$ is such that $\{\lambda \in C | x^*(R_x(\lambda))\} = 0$ has a cluster point, so from complex analysis (uniqueness theorem) it is constant zero. \lightning with $x^*(R_x(\lambda_0)) = 1$.

		„Second item“: Pick $\lambda \in \sigma_B(x) \setminus \sigma_A(x)$ and let $C \subset \rho_A(x)$ be a component containing $\lambda$. By first item, $C \subseteq \sigma_B(x)$, $C$ is open, so $\lambda \in C \subseteq \Int(\sigma_B(x))$.

		„Third item“: If $\rho_A(x)$ is connected, we can apply first item to $C = \rho_A(x)$, we have either $\rho_A(x) \subseteq \sigma_B(x)$ or $\rho_A(x) \cap \sigma_B(x) = \O$. But first is not possible, because $\rho_A(x)$ is unbounded and $\sigma_B(x)$ is bounded. Therefore $\sigma_B(x) \subseteq \sigma_A(x)$.

		„Fourth item“: If $\Int(\sigma_B(x)) = \O$, then (by second item) $\sigma_B(x) \subseteq \partial \sigma_B(x) \subseteq \sigma_A(x) \subseteq \sigma_B(x)$.
	\end{dukazin}
\end{veta}

\begin{dusledek}
	$A$ Banach algebra, $B \subseteq A$ closed subalgebra, $x \in B$. Then all items from previous theorem hold as well if we replace $\sigma_A(x)$ and $\sigma_B(x)$ by $\sigma_A(x) \cup \{0\}$ and $\sigma_B(x) \cup \{0\}$.
	
	\begin{dukazin}
		Without proof. (Basically same that previous; we add unit to $A$ and $B$, so this unit is same ($(¦o, 1)$), etc.)
	\end{dukazin}
\end{dusledek}

\end{document}
