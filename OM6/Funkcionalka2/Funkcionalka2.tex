\documentclass[12pt]{article}					% Začátek dokumentu
\usepackage{../../MFFStyle}					    % Import stylu



\begin{document}

% 13. 02. 2023

%\begin{poznamka}[Exam]
%	Oral, similar as in FA1.
%\end{poznamka}
%
%\begin{poznamka}[Credit]
%	Similar as in FA1.
%\end{poznamka}

\section{Banach algebras}
\subsection{Basic properties}
\begin{definice}[Algebra]
	$(A, +, -, 0, ·_S, ·)$ is algebra over ®K, if

	\begin{itemize}
		\item $(A, +, -, 0, ·_S)$ is vector space over ®K;
		\item $(A, +, -, 0, ·)$ is ring (that is we have $a·(b + c) = a·b + a·c$, $(a + b)·c = a·c + b·c$ and $(a·b)·c = a·(b·c)$);
		\item $\forall \lambda \in ®K\ \forall x, y \in A: \lambda(x·y) = (\lambda x)y = x(\lambda y)$.
	\end{itemize}
\end{definice}

\begin{dusledek}
	1) $e \in A$ is left unit $≡$ $e·a = a$, right unit $≡$ $a·e = a$, unit $≡$ $a·e = e·a = a$ ($\forall a \in A$).

	If $e_1$ is left unit and $e_2$ is right unit, then $e_1 = e_2$ is unit. ($e_1 = e_1·e_2 = e_2$)

	2) (Algebra) homomorphism $\phi: A \rightarrow B$ $≡$ $\phi$ preserves $+, ·, ·_S$, that is $\phi(x + y) = \phi(x) + \phi(y)$, $\phi(x·y) = \phi(x)·\phi(y)$ and $\phi(\alpha·x) = \alpha·\phi(x)$.
\end{dusledek}

\begin{tvrzeni}
	Let $A$ be algebra over ®K. Put $A_e = A \times ®K$ with operations $A_e$ defined coordinate-wise and multiplication defined by
	$$ (a, \alpha)·(b, \beta) := (a·b + \alpha·b + \beta·a, \alpha·\beta), \qquad a, b \in A \land \alpha, \beta \in ®K. $$

	Then $A_e$ is algebra with a unit $(¦o, 1)$ and $A ≡ A \times \{0\} \subset A_e$. Moreover, if $A$ is commutative, then $A_e$ is commutative.

	\begin{dukazin}
		We have $A_e$ is vector space (from linear algebra). We easy proof from definition, that $A_e$ is algebra, $(¦o, 1)$ is a unit in $A_e$ and on $A \times \{0\}$ we have $(a, 0)·(b, 0) = (a·b, 0)$, so $a \mapsto (a, 0)$ is homomorphism. Commutativity is easy too.
	\end{dukazin}
\end{tvrzeni}

\begin{definice}[Normed algebra]
	$(A, \|·\|)$ is normed algebra $≡$ $A$ is algebra and $(A, \|·\|)$ is NLS and $\|a·b\| ≤ \|a\|·\|b\|$ ($\forall a, b \in A$).
\end{definice}

\begin{definice}[Banach algebra]
	$(A, \|·\|)$ is Banach algebra $≡$ $(A, \|·\|)$ is normed algebra nad Banach space.
\end{definice}

\begin{priklady}
	$l_∞(I)$ is commutative Banach algebra with a unit (all ones).

	If $T$ is Hausdorff topological space, then
	$$ ©C_b(T) = \{f: T \rightarrow ®K | f \text{ is continuous and bounded}\} \subseteq l_∞(T) $$
	is closed subalgebra.

	If $T$ is locally compact, Hausdorff, not compact. Then
	$$ ©C_0(T) = \{f: T \rightarrow ®K \text{ continuous } | \forall \epsilon > 0: \{t \in T | |f(t)| ≥ \epsilon\} is \text{ compact}\} \subseteq ©C_b(T) $$
	is closed subalgebra, which doesn't have unit.

	If $X$ is Banach, $\dim X > 1$, then $©L(X)$, with $S·T := S \circ T$, $S, T \in ©L(X)$, is Banach algebra with unit (identity), which isn't commutative.

	If $X$ is Banach, $\dim X = +∞$, then $©K(X) \subset ©L(X)$ is closed subalgebra which is not commutative and doesn't have unit.

	$(L_1(®R^d), *)$, where $*$ is convolution, is (commutative) Banach algebra (without unit).

	$(l_1(®Z), *)$, where $x*y(n) = \sum_{k=-∞}^{+∞} x_k y_{n - k}$ is (commutative) Banach algebra (with unit).
\end{priklady}

\begin{tvrzeni}
	If $(A, \|·\|)$ is normed algebra, then $·: A \oplus_∞ A \rightarrow A$ is Lipschitz on bounded sets.

	\begin{dukazin}
		$$ \forall r > 0: \forall (a, b) \in B_{A \oplus_∞ A}(¦o, r)\ \forall (c, d) \in B_{A \oplus_∞ A}(¦o, r): \|ab - cd\| ≤ $$
		$$ ≤\|a(b - d)\| + \|(a - c)·d\| ≤ \|a\|·\|b - d\| + \|a - c\|·\|d\| ≤ R·(\|b - d\| + \|a - c\|) ≤ 2R \|(a, b) - (c, d)\|. $$
	\end{dukazin}
\end{tvrzeni}

\begin{tvrzeni}
	Let $(A, \|·\|)$ be a Banach algebra. On $A_e$ we consider the norm
	$$ \|(a, \alpha)\| := \|a\| + |\alpha|, \qquad (a, \alpha) \in A \times ®K = A_e. $$
	Then $(A_e, \|·\|)$ is Banach algebra.

	\begin{dukazin}
		It is a Banach space, because $A_e = A \oplus_1 ®K$. Now we need only check, that
		$$ \|(a, \alpha)·(b, \beta)\| ≤ \|(a, \alpha)\|·\|(b, \beta)\|, $$
		which is easy.
	\end{dukazin}
\end{tvrzeni}

\begin{poznamka}
	There is more (natural) ways to define norm on $A_e$ (unlike $·$ on $A_e$, which is natural).

	$A$ has a unit … we may still consider $A_e$.

	If $e \in A \setminus \{¦o\}$ is a unit, then $\|e\| ≥ 1$, because $\|e\| = \|e^2\| ≤ \|e\|^2$.
\end{poznamka}

\begin{veta}
	Let $A$ be a Banach algebra, for $a \in A$ consider $L_a \in ©L(A)$ defined as $L_a(x) := a·x$, $x \in A$. Then $I: A \rightarrow ®L(A)$, $a \mapsto L_a$ is continuous algebra homomorphism, $\|I\| ≤ 1$.

	Moreover, if $A$ has a unit $e$, then $I$ is isomorphism into and $I(e) = \id$.

	If $\|x^2\| = \|x\|^2$, $x \in A$, then $I$ is isometry into.

	\begin{dukazin}
		„$L_a \in ©L(A)$ and $I \in ©L(A, ©L(A))$, $\|I\| ≤ 1$“: Linearity is obvious, $\|L_a(x)\| = \|a·x\| ≤ \|a\|·\|x\|$, so $\|L_a\| ≤ \|a\|$ and so $\|I\| ≤ 1$. Since it is easily $I$ preserves multiplication, so we are left to prove the „Moreover“ part.

		„$A$ has a unit $e$“: WLOG $A ≠ \{¦o\}$.
		$$ \forall a \in A: \|I a\| = \|L_a\| ≥ \|L_a\(\frac{e}{\|e\|}\) = \frac{a}{\|e\|} = \frac{1}{\|e\|}·a. $$
		So $I$ is bounded from below, so $I$ is isomorphism.

		$$ I(e)(x) = L_e(x) = x, \text{ so } I(e) = \id. $$

		Finally, if $\|x^2\| = \|x\|^2$, $x \in A$, then $\forall a \in A:$
		$$ \|a\| ≥ \|I(a)\| = \|L_a\| ≥ \|L_a \(\frac{a}{\|a\|}\)\| = \frac{\|a^2\|}{\|a\|} = \|a\|. $$
		So $I$ is isometry.
	\end{dukazin}
\end{veta}

\begin{poznamka}
	$A ≠ \{¦o\}$ Banach algebra with a unit $\implies$ $\exists$ equivalent norm $\|·\|$ on $A$ such that $(A, \|·\|)$ is Banach algebra and $\|e\| = 1$.

	\begin{dukazin}
		Let $I: A \rightarrow ©L(A)$ be as before. Put $|\|x\| | := \|I(x)\|$, $x \in A$. Since $I$ is isomorphism, $|\|·\| |$ is equivalent norm. Moreover, $|\|x·y\| | = \|I(x·y)\| ≤ \|I(x)\|·\|I(y)\| = |\|x\| | · |\|y\| |$, $x, y \in A$. So $(A, |\|·\| |)$ is a Banach algebra. Finally
		$$ |\|e\| | = \|I(e)\| = \|\id\| = 1. $$
	\end{dukazin}
\end{poznamka}

\subsection{Inverse elements}
\begin{definice}
	$(M, ·, e)$ is monoid ($·$ is associative, $e$ is unit). Then invertible elements form a group ($e^{-1} = e, \exists x^{-1}, y^{-1} \implies (x·y)^{-1} = y^{-1}·x^{-1}$); if $x \in M$, and $y \in M$ is its left inverse and $z \in M$ is its right inverse, then $y = z$ is inverse:
	$$ y = y·e = y·x·z = e·z = z. $$

	We denote $M^\times := \{x \in M\ |\ \exists x^{-1}\}$
\end{definice}

% 14. 02. 2023

\begin{tvrzeni}
	If $(A, ·, e)$ is monoid and $x_1, …, x_n \in A$ commute, then $x_1·…·x_n \in A^x \Leftrightarrow \{x_1, …, x_n\} \subset A^x$.

	\begin{dukazin}
		It suffices to prove it for $n=2$ (and use induction). „If $x^{-1}$ and $y^{-1}$ exists, then $(xy)^{-1}$“ is easy from asociativity.

		If we have $(xy)^{-1}$. Put $z:= (xy)^{-1}x$. Then $zy = (xy)^{-1}(xy) = e$, so $z$ is left inverse to $y$. Next we show that there is also right inverse: Put $\tilde z := x(xy)^{-1}$: $y\tilde z = (xy)(xy)^{-1} = e$, so $\tilde z$ is right inverse. And we already know that if there is left and right inverse, then they are same and they are inverse.
	\end{dukazin}
\end{tvrzeni}

\begin{lemma}
	Let $A$ be a Banach algebra with a unit.

	\begin{itemize}
		\item $\|x\| < 1 \implies \exists(e - x)^{-1} \land (e - x)^{-1} = \sum_{n=0}^∞ x^n$;
		\item $\exists x^{-1} \land \|h\| < \frac{1}{\|x^{-1}\|} \implies \exists(x + e)^{-1} \land \|(x + h)^{-1} - x^{-1}\| ≤ \frac{\|x^{-1}\|^2·\|h\|}{1 - \|x^{-1}\|·\|h\|}$.
	\end{itemize}

	\begin{dukazin}
		„First“: We have $\|x^n\| ≤ \|x\|^n$, so $\sum_{n=0}^∞ x^n$ is absolute convergent series, so $\sum_{n=0}^∞ x^n \in A$. Moreover,
		$$ (e - x)·\(\sum_{n=0}^∞ x^n\) = \lim_{N \rightarrow ∞} (e - x)·(e + x + … + x^N) = \lim_{N \rightarrow ∞} e - x^{N+1} = e, $$
		because $\lim_{N \rightarrow ∞} \|x^{n+1}\| ≤ \lim_{N \rightarrow ∞} \|x\|^N = 0$. And similarly $\(\sum x^n\)·(e - x) = e$.

		„Second item“: $x + h = x·(e + x^{-1}h)$ we have $x^{-1}$ exists and $(e + x^{-1}h)^{-1}$ exists (from first item), so from previous fact $(x + h)^{-1}$ exists. Moreover
		$$ (x + h)^{-1} = (e + x^{-1} h)^{-1}·x^{-1} \overset{1)}= \sum_{n=0}^∞ \(-x^{-1} h\)^n x^{-1}, $$
		so
		$$ \|(x + h)^{-1} - x^{-1}\| = \|\sum_{n=1}^∞\(-x^{-1} h\)^n x^{-1}\| ≤ \|x^{-1}\|·\sum_{n=1}^∞ \|x^{-1} h\|^n ≤ $$
		$$ ≤ \|x^{-1}\| \sum_{n=1}^∞ \sum_{n=1}^∞\(\|x^{-1}\|·\|h\|\)^n = \|x^{-1}\| · \frac{\|x^{-1}\| \|h\|}{1 - \|x^{-1}\|·\|h\|}. $$
	\end{dukazin}
\end{lemma}

\begin{dusledek}
	$A$ Banach algebra with a unit $\implies$ $A^x \subset A$ is open and $A^x$ is topological group.

	\begin{dukazin}
		$A^x \subset A$ is open by previous lemma (second item). So it remains to prove $x \mapsto x^{-1}$ is continuous:
		$$ A^x \ni x_n \rightarrow x \in A^x \overset{?}\implies x_n^{-1} \rightarrow x^{-1}. $$
		$$ \|x_n^{-1} - x^{-1}\| \overset{h:= x_n - x}≤ \frac{\|x^{-1}\|^2·\|x_n - x\|}{1 - \|x^{-1}\|·\|x_n - x\|} \rightarrow 0. $$
	\end{dukazin}
\end{dusledek}

\subsection{Spectral theory}
\begin{definice}[Resolvent set, spectrum and resolvent]
	Let $A$ be a Banach algebra with a unit, $x \in A$. We define resolvent set of $x$ as $\rho_A(x) := \{\lambda \in ®K | \exists(\lambda·e - x)^{-1}\}$. Next we define spectrum of $x$ as $\sigma_A(x) := ®K \setminus \rho_A(x)$. Finally we define resolvent of $x$ as $R_x: \rho(x) \rightarrow A$, $R_x(\lambda) := (\lambda·e - x)^{-1}$.

	If $A$ doesn't have a unit, then notions above are defined with respect to $A_e$.
\end{definice}

\begin{tvrzeni}
	$A$ Banach algebra

	\begin{itemize}
		\item[a)] $\forall x \in A: 0 \in \sigma_{A_e}(x)$ (in particular, if $A$ has no unit, then $0 \in \sigma_A(x)$);
		\item[b)] $A$ has unit $\implies$ $\sigma_{A_e}(x) = \sigma_A(x) \cup \{0\}$.
	\end{itemize}

	\begin{dukazin}[a)]
		$$ \forall(b, \beta) \in A_e: (x, 0)·(b, \beta) = (…, 0) ≠ (¦o, 1) \implies \nexists (x, 0)^{-1} \implies 0 \in \sigma_{A_e}(x). $$
	\end{dukazin}

	\begin{dukazin}[b)]
		By a) we have $0 \in \sigma_{A_e}(x)$. So it suffices: $\forall \lambda ≠ 0: \lambda \in \rho_A(x) \Leftrightarrow \lambda \in \rho_{A_e}(x)$. First means $(\lambda·e - x)^{-1}$ exists in $A$ and second means that $((0, \lambda) - (x, 0))^{-1} = (-x, \lambda)^{-1}$ exists in $A$. We take „$x \rightarrow -x$“.

		„$\implies$“: find $(b, \beta) \in A_e$ such that $(x, \lambda)·(b, \beta) = (¦o, 1)$. So $(x·b + \lambda·b + \beta·x, \lambda·\beta) = (¦o, 1)$. So $\beta = \frac{1}{\lambda}$ and $b = -\frac{1}{\lambda}(\lambda e + x)^{-1}·x$. Similarly we find left inverse $\(-\frac{1}{\lambda}x(x + \lambda e)^{-1}, \frac{1}{\lambda}\)(x, \lambda)$. And next we prove that they are really inverses.

		„$\impliedby$“: Put $(b, \beta) := (x, \lambda)^{-1}$. Then $(\lambda e + x)^{-1} = b + \beta·e$. We have $(x, \lambda)·(b, \beta) = (¦o, 1)$, so $\lambda·\beta = 1$ and $x·b + \lambda·b + \beta·x = ¦o$. Then
		$$ (\lambda e + x)·(b + \beta·e) = \lambda·b + \lambda·\beta·e + x·b + \beta·x = e. $$
		Similarly second inverse.
	\end{dukazin}
\end{tvrzeni}

% 20. 02. 2023

\begin{veta}
	$\{¦o\} ≠ A$ complex Banach algebra, $x \in A$. Then $\sigma(x) \subseteq B_{®C}(0, \|x\|)$ is compact, nonempty.

	\begin{dukazin}
		After theory.
	\end{dukazin}
\end{veta}

\begin{definice}[Derivative]
	$Y$ Banach space, $\Omega \subset ®K$, $f: \Omega \rightarrow Y$, $a \in \Omega$. Then
	$$ f'(a) := \lim_{x \rightarrow a} \frac{f(x) - f(a)}{x - a} $$
	is the derivative of $f$ at $a$.
\end{definice}

\begin{tvrzeni}[Fact]
	$Y$ Banach, $\Omega \subset ®K$, $f: \Omega \rightarrow Y$, $a \in \Omega$. Then $f'(a)$ exists $\implies$ $f$ is continuous at $a$ $\land$ $\forall x^* \in Y^*$: $(x^* \circ f)'(a) = x^*(f'(a))$.

	\begin{dukazin}
		Continuity: $\lim_{x \rightarrow a} f(x) - f(a) = \lim_{x \rightarrow a} \frac{f(x) - f(a)}{x - a}·(x - a) = f'(a)·0 = 0$.

		$x^* \in Y^*$ given, then
		$$ \lim_{x \rightarrow a} \frac{x^*(f(x)) - x^*(f(a))}{x - a} = \lim_{x \rightarrow a} x^*\(\frac{f(x) - f(a)}{x - a}\) = x^*(f'(a)). $$
	\end{dukazin}
\end{tvrzeni}

\begin{tvrzeni}
	$A$ Banach algebra with a unit, $x \in A$. Then

	\begin{itemize}
		\item $\rho(x)$ is open set;
		\item $\forall |\lambda| > \|x\|: \lambda \in \rho(x) \land R_x(\lambda) = \sum_{n=0}^∞ \frac{x^n}{\lambda^{n + 1}}$;
		\item (important!) $\rho(x) \ni \lambda \mapsto R_x(\lambda)$ has derivative at each $\lambda \in \rho(x)$;
		\item $\forall \mu, \nu \in \rho(x): R_x(\mu)·R_x(\nu) = R_x(\nu)·R_x(\mu)$;
		\item $\forall \mu, \nu \in \rho(x): R_x(\mu) - R_x(\nu) = (\nu - \mu)·R_x(\mu)·R_x(\nu)$.
	\end{itemize}
	
	\begin{dukazin}
		First is proved by lemma. Second by lemma we have
		$$ (\lambda e - x)^{-1} = \lambda^{-1}\(e - \frac{x}{\lambda}\)^{-1} = \lambda^{-1} \sum_{n=0}^∞ \(\frac{x}{\lambda}\)^n. $$

		Fourth: In general $u v = v u \implies u^{-1}v^{-1} = v^{-1}u^{-1}$ (proof: $u^{-1}v^{-1} = (v u)^{-1}$). And we apply it for $u = (\mu e - x)$ and $v = (\nu e - x)$.

		Fifth: In general $u·v = v·u \implies u^{-1}·v = v·u^{-1}$ (proof: $u^{-1} v = v·v^{-1}u^{-1}v = v·u^{-1} v^{-1} v = v u^{-1}$) so:
		$$ R_x(\mu) - R_x(\nu) = R_x(\mu)R_x(\nu)(R_x(\nu)^{-1}) - R_x(\mu)(R_x(\mu))^{-1}R_x(\nu) = $$
		$$ = R_x(\mu)R_x(\nu)(R_x(\nu)^{-1}) - R_x(\mu)(R_x(\mu))R_x(\nu)^{-1} = $$
		$$ = R_x(\mu) R_x(\nu) \(R_x(\nu)^{-1} - R_x(\mu)^{-1}\) = R_x(\mu) R_x(\nu) (\nu - \mu). $$
	\end{dukazin}

	\begin{dukazin}
		For third we fix $\lambda \in \rho(x)$ and $t \in (0, \delta)$ for $\delta$ small enough ($\lambda + t \in \rho(x)$ and *). We shall prove that „$R_x'(\lambda) = -R_x(\lambda)^2$“:
		$$ 0 \overset?= \left\|\frac{R_x(\lambda + t) - R_x(\lambda)}{t} + R_x(\lambda)^2\right\| = $$
		$$ = \frac{1}{|t|}\left\|(\lambda e - x + t e)^{-1} - (\lambda e - x)^{-1} + (\lambda e - x)^{-1}·t·(\lambda e - x)^{-1}\right\| ≤ $$
		$$ \overset{* \text{ for existence of the inverse}}≤ \frac{1}{|t|}\|(\lambda e - x)^{-1}\|·\left\|(e + t(\lambda e - x)^{-1})^{-1} - e + (\lambda e - x)^{-1}·t\right\| = $$
		$$ = \frac{1}{|t|} \|(\lambda e - x)^{-1}\|·\left\|\sum_{n=0}^∞(-t)^n (\lambda e - x)^{-n} - e + (\lambda e - x)^{-1}·t\right\| ≤ $$
		$$ \overset{\|x^n\| ≤ \|x\|^n}≤ \frac{1}{|t|} \|(\lambda e - x)^{-1}\| · \sum_{n=2}^∞ \|t(\lambda e - x)^{-1}\|^n = $$
		$$ = \frac{1}{|t|} \|t(\lambda e - x)^{-1}\| · \frac{\|t(\lambda e - x)^{-1}\|^2}{1 - \|t(\lambda e - x)^{-1}\|} \overset{* \text{ for denominator }≤1/2}≤ \frac{2|t|^2}{|t|} \|t(\lambda e - x)^{-1}\| \rightarrow 0. $$
	\end{dukazin}
\end{tvrzeni}

\begin{veta}[Liouville for Banach space valued functions]
	$Y$ Banach space over ®C, $f: ®C \rightarrow Y$ has derivative at each point, $f$ is bounded ($≡$ $\|f\|$ is bounded). Then $f ≡ \const$.

	\begin{dukazin}
		Assume $f \not≡ \const$, so there are $a ≠ b \in ®C: f(a) ≠ f(b) \implies$ (by Hahn–Banach theorem) $\exists x^* \in Y^*: x^*(f(x)) ≠ x^*(f(x))$. From fact $x^* \in f: ®C \rightarrow ®C$ has derivative at each point is bounded, not constant which is in contradiction with Liouville theorem for complex valued functions.
	\end{dukazin}
\end{veta}

\begin{dukaz}[Theorem before theory]
	First case: „$A$ has a unit“: Then $\sigma(x) \subseteq B_{®C}(0, \|x\|)$ is closed, so $\sigma(x)$ is compact. Assume that $\rho(x) = ®C$. By previous tvrzeni we have $R_x: ®C \rightarrow A$ has derivative everywhere, and it is bounded because $\lim_|\lambda| \rightarrow ∞ R_x(\lambda) = \lim_{|\lambda|\rightarrow∞} \sum_{n=0}^∞ \frac{x^n}{\lambda^{n + 1}} = 0$. From previous theorem $R_x ≡ \const$ so $\lim_{|\lambda| \rightarrow ∞} R_x(\lambda) = 0 \implies R_x ≡ 0$. In particular $0 = R_x(0) = (-x)^{-1}$. \lightning (If $A ≠ \{0\}$ then $x^{-1} ≠ 0$ for $x \in A$.)

	Second case: „$A$ hasn't a unit“, then $\sigma(x) := \sigma_{A_e}((x, 0))$ so we apply the already proven case.
\end{dukaz}

\begin{poznamka}[Convention]
	If not said otherwise, in chapter about Banach algebras, all Banach spaces are complex.
\end{poznamka}

\begin{veta}[Gelfand–Mazur]
	$\{¦o\} ≠ A$ Banach algebra with a unit. Assume $\forall x \in A \setminus \{¦o\}: \exists x^{-1}$. Then $A$ is isomorphic to ®C. If moreover $e$ is a unit in $A$ and $\|e\| = 1$, then $A$ is isometrically isomorphic to ®C.

	\begin{dukazin}
		Consider $\psi: ®C \rightarrow A$ defined as $\psi(\lambda) := \lambda·e$. This is algebraic homomorphism and $\|\psi(\lambda)\| = |\lambda|·\|e\|$, so it is isomorphism (and isometry, if $\|e\| = 1$).

		It remains „$\phi$ is surjective“: Pick $a \in A$. From previously proved theorem $\exists \lambda \in \sigma(a)$, then $(\lambda e - a) \notin A^x$. So, $\lambda·e - a = 0$, then $\psi(\lambda) = a$.
	\end{dukazin}
\end{veta}

\begin{definice}[Spectral radius]
	A Banach algebra, $x \in A$. Then $r(x) := \sup\{|\lambda|, \lambda \in \sigma(x)\}$ is called spectral radius of $x$.
\end{definice}

\begin{veta}[Beurling–Gelfand]
	$A$ Banach algebra, $x \in A$ $\implies$ $r(x) = \inf_{n \in ®N} \sqrt[n]{\|x^n\|} = \lim_n \sqrt[n]{\|x^n\|}$.
\end{veta}

\begin{lemma}
	$A$ Banach algebra with a unit, $x \in A$. For $p(z) = \sum_{j=1}^n \alpha_j z^j \in ®C$ a polynom (with complex coefficients) we put $p(x) = \sum_{j=1}^n \alpha_j x^j \in A$. Then $\sigma(p(x)) = p(a(x))$.

	\begin{dukazin}
		Fix $\lambda \in ®C$ and write $(\lambda - p)(z) = c·\prod_{i=1}^m(z - z_i)$, where $z_1, …, z_m$ are roots of $\lambda - p$. Then $\lambda \in \sigma(p(x)) \Leftrightarrow (\lambda e - p(x))^{-1}$ does not exists. $(\lambda e - p(x))^{-1} = c·\prod_{i=1}^m (x - z_i·e)$, so it does'nt exists if and only if $\exists i \in [m]$, such that $(x - z_i·e)^{-1}$ doesn't exists $\Leftrightarrow$ $z_i \in \sigma(x)$ $\Leftrightarrow$ $\exists$ root $\nu$ of $\lambda - p$ such that $\nu \in \sigma(x)$ $\Leftrightarrow$ $\exists \nu \in \sigma(x): p(\nu) = \lambda$ $\Leftrightarrow$ $\lambda \in p(\sigma(x))$.
	\end{dukazin}
\end{lemma}

\begin{dukaz}[Beurling–Gelfand]
	WLOG $A$ has a unit. Step 1, „$r(x) ≤ \inf_n \sqrt[n]{\|x^n\|}$“: fix $\lambda \in \sigma(x)$. By previous lemma $\forall n: \lambda^n \in \sigma(x^n)$. By theorem 'Before theory' we have $\forall n: |\lambda|^n ≤ \|x^n\|$.

	Step 2, „$r(x) ≥ \limsup_n \sqrt[n]{\|x^n\|}$“:
%
% 21. 02. 2023
%
	Pick $r > r(x)$. Claim: „$\frac{x^n}{r^n} \rightarrow^w 0$“: Fix $x^* \in A^*$ and put $f(\lambda) := \lambda · x^*(R_x(\lambda))$. By fact and tvrzeni after it, $f$ has derivative at each $\lambda \in \rho(x)$. Moreover for $|\lambda| ≥ \|x\|$ we have $f(\lambda) = \lambda · x^*\(\sum_{n=0}^∞ \frac{x^n}{\lambda^{n + 1}}\) = \sum_{n=0}^∞ \frac{x^*(x^n)}{\lambda^n}$. Thus $f(\lambda) = \sum_{n=0}^∞ \frac{x^*(x^n)}{\lambda^n}$, $\lambda \in P(0, r(x), ∞)$. From Complex analysis $f \in H(P(0, r, ∞))$ is uniquely given by Laurent series. In particular $f(r) = \sum_{n=0}^∞ \frac{x^*(x^n)}{r^n}$, so $x^*\(\frac{x^n}{r^n}\) \rightarrow 0$.

	From princip of unique boundedness (last semester): $\frac{x^n}{r^n}$ if $\|·\|$-bounded, so $\exists c > 0:$ $\|x^n\| ≤ c r^n$, $\sqrt[n]{\|x^n\|} ≤ \sqrt[n]{c}·r \rightarrow r$. So $\limsup \sqrt[n]{\|x^n\|} ≤ r$.
\end{dukaz}

\begin{dusledek}
	$A$ Banach algebra, $x \in A$ and $|\lambda| > r(x)$. Then $\sum_{n=1}^∞ \frac{x^n}{\lambda^n}$ is absolutely convergent and $R_x(\lambda) = \sum_{n=0}^∞ \frac{x^n}{\lambda^{n + 1}}$.

	\begin{dukazin}
		Fix $q$, such that $\frac{r(x)}{|\lambda|} < q < 1$. By previous theorem, $\exists n_0\ \forall n ≥ n_0: \frac{\sqrt[n]{\|x^n\|}}{\lambda} < q$, so $\frac{\|x^n\|}{|\lambda|^n} < q^n$, $n ≥ n_0$. Thus $\sum \left\| \frac{x^n}{\lambda^n}\right\| ≤ ∞$, so the sum is absolutely convergent.

		Now we easily check that $(\lambda e - x)^{-1} = \sum_{n=0}^∞ \frac{x^n}{\lambda^{n + 1}}$.
	\end{dukazin}
\end{dusledek}

\subsection{Subalgebra}
\begin{veta}
	$A$ Banach algebra with a unit $e$, $B \subset A$ is closed subalgebra such that $e \in B$. Fix $x \in B$. Then

	\begin{itemize}
		\item $C \subset \rho_A(x)$ is component (maximum connected subset) $\implies$ $C \subseteq \sigma_B(x)$ or $C \cap \sigma_B(x) = \O$;
		\item $\partial \sigma_B(x) \subseteq \sigma_A(x) \subseteq \sigma_B(x)$;
		\item $\rho_A(x)$ is connected $\implies$ $\sigma_A(x) = \sigma_B(x)$;
		\item $\Int \sigma_B(x) = \O \implies \sigma_A(x) = \sigma_B(x)$.
	\end{itemize}

	\begin{dukazin}
		„$\sigma_A(x) \subseteq \sigma_B(x)$“: $(\lambda e - x)^{-1}$ exists in $B$ implies it exists (it's same) in $A$.

		„First item“: Let $C \subset \rho_A(x)$ be component. Pick $\lambda_0 \in C \cap \sigma_B(x)$. Wanted: „$C \setminus \sigma_B(x) = \O$“. Pick $x^* \in A^*: x^*|_B = 0 \land x^*(R_x(\lambda)) = 1$ (separate $B$ and $R_x(\lambda) \notin B$). Then $C \ni \lambda \mapsto x^*(R_x(\lambda))$ is holomorphic function on open (because maximum) connected set $C$. Which is zero\footnote{For $\lambda \in C \setminus \sigma_B(x)$, $(\lambda e - x)^{-1}$ exists in $B$ so $R_x(\lambda) \in B$ and therefore, $x^*(R_x(\lambda)) = 0$} on $C \setminus \sigma_B(x)$.

		Since $C \setminus \sigma_B(x)$ is open, if it is nonempty it contains a ball, so it has cluster point. Thus $C \ni \lambda \mapsto x^*(R_x(\lambda))$ is such that $\{\lambda \in C | x^*(R_x(\lambda))\} = 0$ has a cluster point, so from complex analysis (uniqueness theorem) it is constant zero. \lightning with $x^*(R_x(\lambda_0)) = 1$.

		„Second item“: Pick $\lambda \in \sigma_B(x) \setminus \sigma_A(x)$ and let $C \subset \rho_A(x)$ be a component containing $\lambda$. By first item, $C \subseteq \sigma_B(x)$, $C$ is open, so $\lambda \in C \subseteq \Int(\sigma_B(x))$.
	\end{dukazin}

	\begin{dukazin}
		„Third item“: If $\rho_A(x)$ is connected, we can apply first item to $C = \rho_A(x)$, we have either $\rho_A(x) \subseteq \sigma_B(x)$ or $\rho_A(x) \cap \sigma_B(x) = \O$. But first is not possible, because $\rho_A(x)$ is unbounded and $\sigma_B(x)$ is bounded. Therefore $\sigma_B(x) \subseteq \sigma_A(x)$.

		„Fourth item“: If $\Int(\sigma_B(x)) = \O$, then (by second item) $\sigma_B(x) \subseteq \partial \sigma_B(x) \subseteq \sigma_A(x) \subseteq \sigma_B(x)$.
	\end{dukazin}
\end{veta}

\begin{dusledek}
	$A$ Banach algebra, $B \subseteq A$ closed subalgebra, $x \in B$. Then all items from previous theorem hold as well if we replace $\sigma_A(x)$ and $\sigma_B(x)$ by $\sigma_A(x) \cup \{0\}$ and $\sigma_B(x) \cup \{0\}$.
	
	\begin{dukazin}
		Without proof. (Basically same that previous; we add unit to $A$ and $B$, so this unit is same ($(¦o, 1)$), etc.)
	\end{dukazin}
\end{dusledek}

% 27. 02. 2023

\subsection{Holomorphic calculus}
\begin{definice}
	$X$ Banach, $\gamma: [a, b] \rightarrow ®C$ path (continuous, piecewise smooth ($C^1$)), $f: \<\gamma\> \rightarrow X$ continuous. Then
	$$ \int_\gamma f := \int_{[a, b]} \gamma'(t) f(\gamma(t)) dt. \qquad (\text{As Bochner integral.}) $$
	If $\Gamma = \gamma_1 + … + \gamma_n$ is chain in ®C, $f: \<\Gamma\> \rightarrow X$ continuous, then
	$$ \int_\Gamma f := \sum_{i=1}^n \int_{\gamma_i} f. $$
\end{definice}

\begin{lemma}
	$\Gamma$ chain in ®C, $X$ Banach, $f: \<\Gamma\> \rightarrow X$, $x \in X$. Then
	$$ \int_\Gamma f = x \Leftrightarrow \forall x^* \in X^*: x^*(x) = \int_\Gamma x^* \circ f. $$

	\begin{dukazin}
		„$\impliedby$“ by Hahn–Banach theorem. „$\implies$“: (by previous semester $x^*$ and $\int$ "commutes")
		$$ x^*\(\int_\Gamma f\) = \sum_{i=1}^n x^*\(\int_{\gamma_i} f\) = \sum_{i=1}^n \int_{[a_i, b_i]} \gamma_i'(t) x^*(f(\gamma_i(t)))dt = \int_\Gamma x^* \circ f. $$
	\end{dukazin}
\end{lemma}

\begin{poznamka}[Recall]
	If $\Omega \subset ®C$ open, $K \subset \Omega$ compact. Then there is a cycle $\Gamma$ such that $\<\Gamma\> \subset \Omega \setminus K$ and $\ind_\Gamma z = 1$ if $z \in K$ and $0$ if $z \notin \Omega$.

	Then we say that $\Gamma$ circulates $K$ in $\Omega$.
\end{poznamka}

\begin{definice}
	Let $A$ be a Banach algebra with unit, $x \in A$, $\Omega \subset ®C$ open and $\sigma(x) \subset \Omega$, $f \in ©H(\Omega)$. Then $f(x) := \frac{1}{2\pi i} \int_\Gamma f·R_x$, where is any cycle which circulates $\sigma(x)$ in $\Omega$.
\end{definice}

\begin{poznamka}
	$f(x)$ exists ($f·R_x$ is continuous on $\<\Gamma\>$), $f(x)$ does not depend on the choice of $\Gamma$ (Pick $x^* \in X^*$, then $(x^* \circ f·R_x)(\lambda) = f(\lambda)·x^*(R_x(\lambda))$ is holomorphic. Pick $\Gamma_1, \Gamma_2$ cycles circulating $\sigma(x)$ in $\Omega$, then $\int_{\Gamma_1 - \Gamma_2} x^* \circ f·R_x = 0$ from Cauchy).
\end{poznamka}

\begin{veta}[Holomorphic calculus]
	$A$ Banach algebra with unit, $x \in A$, $\Omega \subset ®C$ open such that $\sigma(x) \subset \Omega$, $f \in ©H(\Omega)$. Then $\Phi: ©H(\Omega) \rightarrow A$ defined as $\Phi(f) = f(x)$ (from definition above) has the following properties:\vspace{-0.7em}
	\begin{itemize}
		\item $\Phi$ is algebra homomorphism, $\Phi(1) = e$, $\phi(\id) = x$;
		\item $f_n \overset{\text{loc.}}\rightrightarrows f$ in $H(\Omega)$, then $f_n(x) \rightarrow f(x)$;
		\item $f(x)^{-1}$ exists $\Leftrightarrow$ $f ≠ 0$ on $\sigma(x)$, in this case $f(x)^{-1} = \frac{1}{f}(x)$;
		\item $\sigma(f(x)) = f(\sigma(x))$;
		\item if $\Omega_1$ is open and $f(\sigma(x)) \in \Omega_1$, $g \in ©H(\Omega_1)$, then $(g \circ f)(x) = g(f(x))$;
		\item if $y \in A$ commutes with $x$, then $y$ commutes with $f(x)$.
	\end{itemize}

	Moreover, if $\psi: ©H(\Omega) \rightarrow A$ satisfy first two item, then $\psi = \Phi$.
\end{veta}

\begin{lemma}
	$(\Omega, \mu)$ complete measurable space, $A$ Banach algebra, $f \in L_1(\mu, A)$. Let $x \in A$ and $E \subset \Omega$ is measurable. Then
	$$ x·\(\int_E f(t) d\mu(t)\) = \int_E x·f(t) d\mu(t), \qquad \(\int_E f(t) d\mu(t)\)·x = \int_E f(t)·x d\mu(t). $$

	\begin{dukazin}
		Easy (by commutation of integral and linear operator from last semester), skipped.
	\end{dukazin}
\end{lemma}

\begin{dukaz}[Holomorphic calculus]
	„1st item“: „$\Phi$ is linear“ is easy, „$\Phi$ is multipicative“: Pick $f, g \in ©H(\Omega)$, open set $U$ such that $\sigma(x) \subset U \subset \overline{U} \subset \Omega$. Let $\Gamma$ cycle circulating $\sigma(x)$ in $U$, $\Lambda$ cycle circulating $\overline{U}$ in $\Omega$. Then
	$$ f(x)·g(x) = \(\frac{1}{2\pi i} \int_\Gamma f·R_x\)·g(x) \overset{\text{lemma}}= $$
	$$ = \frac{1}{2\pi i} \int_\Gamma f(t) R_x(t) g(x) dt = \frac{1}{2\pi i} \int_\Gamma f(t)·R_x(t)·\frac{1}{2\pi i} \int_\Lambda g(s)·R_x(s) ds dt \overset{\text{lemma}}= $$
	$$ = \frac{1}{2 \pi i} \int_\Gamma f(t)·\frac{1}{2\pi i} \int_\Lambda g(s)·R_x(t)·R_x(s) ds dt = $$
	because $\<\Lambda\> \cap \<\Gamma\> = \O$, we can use theorem after definition of $R_x$:
	$$ = \frac{1}{(2 \pi i)^2} \int_\Gamma \int_\Lambda f(t)·g(s)·\frac{R_x(t) - R_x(s)}{s - t} ds dt \overset{\text{Fubini to $x^*(…)$ and lemma}}=  $$
	$$ = \frac{1}{(2\pi i)^2} \int_\Gamma f(t) \(\int_\Lambda \frac{g(s)}{s - t} ds\) R_x(t) dt - \frac{1}{(2\pi i)^2} \int_\Lambda g(s) \(\int_\Gamma \frac{f(t)}{s - t}\) R_x(s) ds = $$
	(Now we use Cauchy theorem ($f(z) \ind_\Gamma z = \frac{1}{2\pi i} \int_\Gamma \frac{f(w)}{w - z} dw$). $\forall s \in \<\Lambda\>: (t \mapsto \frac{f(t)}{s - t}) \in ©H(U) \land \ind_\Gamma z = 0, z \notin U$, so $\int_\Gamma \frac{f(t)}{s - t} dt = 0$. $\forall t \in \<\Gamma\>: \ind_\Lambda t = 1 \land (s \mapsto g(s)) \in ©H(\Omega) \implies g(t) = \frac{1}{2\pi i} \int_\Lambda \frac{g(s)}{s - t} ds$.)
	$$ = \frac{1}{2\pi i} \int_\Gamma f(t)g(t) R_x(t) dt - 0. $$

	It remains that „if $f(z) = z^k$, $k \in ®N \cup \{0\}$ then $f(x) = x^k$“ (we want it for $k = 0$ and $k = 1$). Put $\Gamma(t) = r·e^{it}$, $t \in [0, 2\pi]$, where $r > \|x\|$ arbitrary. By some theorem:
	$$ R_x(\lambda) = \sum_{n=0}^∞ \frac{x^n}{\lambda^{n + 1}}, \qquad |\lambda| > \|x\|. $$
	Thus (we switch integral and sum, because later we realize that sum of integral of absolute value is finite)
	$$ \forall x^* \in A^*: x^*(f(x)) = \frac{1}{2 \pi i} \int_\Gamma \lambda^k x^*(\sum_{n=0}^∞ \frac{x^n}{\lambda^{n+1}}) d\lambda = \frac{1}{2\pi i} \int_\Gamma \sum_{n=0}^∞ \frac{x^*(x^n)}{\lambda^{n - k + 1}} d\lambda = $$
	$$ = \frac{1}{2\pi i} \sum_{n=0}^∞ \int_\Gamma \frac{x^*(x^n)}{\lambda^{n - k + 1}} d\lambda = \frac{1}{2\pi i} \sum_{n=0}^∞ x^*(x^n) \int_\Gamma \frac{1}{\lambda^{n - k + 1}} d\lambda = $$
	$$ = \frac{1}{2\pi i} \sum_{n=0}^∞ x^*(x^n) \int_0^{2\pi} i \frac{1}{\Gamma(t)^{n - k}} dt = x^*(x^k) + \sum 0, $$
	because $\Gamma$ (is $2\pi$ periodic).

	„2nd item“: For $\Gamma = \gamma_1 + … + \gamma_N$:
	$$ \|f_n(x) - f(x)\| = \frac{1}{2\pi i} \left\|\int_\Gamma (f_n(\lambda) - f(\lambda)) R_x(\lambda) d\lambda\right\| ≤ \frac{1}{2\pi} \int_\Gamma |f_n(\lambda) - f(\lambda)|·\|R_x(\lambda)\| d\lambda ≤ $$
	$$ ≤ \frac{1}{2\pi} \sum_{i=1}^N \int_{a_i}^{b_i} |\gamma_i'(t)| \sup_{z \in \<\Gamma\>} |f_n(z) - f(z)|·\|R_x(\gamma_i(t))\| dt = $$
	$$ = \sup_{z \in \<\Gamma\>} |f_n(z) - f(z)| · \frac{1}{2\pi} \sum_{i=1}^N \int_{a_i}^{b_i} \|R_x(\gamma_i(t))\|·|\gamma_i'(t)| dt \rightarrow 0. $$

	„Moreover part“: By Runge theorem (and second item) it is enough prove it for rational functions. If $R$ was polynom, then $\Phi(R) = \Psi(R)$ by second item. So it suffices „$\forall p$ polynom: $\frac{1}{p} \in ©H(\Omega) \implies \Phi(\frac{1}{p}) = \psi(\frac{1}{p})$“. Pick $p$ polynom. Then $e = \psi(1) = \psi(p·\frac{1}{p}) = \psi(p)·\psi(\frac{1}{p}) = \Phi(p)·\psi(\frac{1}{p})$ (similarly for $\frac{1}{p}·p$). So $\psi(\frac{1}{p}) = \Phi(p)^{-1} = \Phi(\frac{1}{p})$.

% 28. 02. 2023

	„3rd item“: „$\implies$“ Let $f(z) = 0$ for some $z \in \sigma(x)$. Then exists $g \in H(\Omega): f(u) = (z - u)g(z)$. By item one, we have $(z e - x)g(x) = f(x) = g(x) (z e - x)$. But $(z e - x)^{-1}$ does not exist, so $f(x)^{-1}$ does not exists.

	„$\impliedby$“ Suppose $f ≠ 0$ on $\sigma(x)$ by compactness. $\exists \Omega_1 \subset \Omega$ open: $\sigma(x) \subset \Omega_1$ and $f ≠ 0$ on $\Omega_1$. Then $\frac{1}{f} \in H(\Omega_1)$ and by first item we have $e = (f · \frac{1}{f})(x) = f(x) \frac{1}{f}(x) = … = \frac{1}{f}(x)·f(x) \implies f(x)^{-1} = \frac{1}{f}(x)$.
\end{dukaz}

\begin{poznamka}
	$f = g$ on a neighbourhood of $\sigma(x)$ $\implies$ $f(x) = g(x)$ (from definition), other implication doesn't hold!
\end{poznamka}

% 06. 03. 2023

\vspace{-1.8em}
\subsection{Multiplicative functionals}
\begin{definice}[Multiplicative functional]
	Let $A$ be a Banach algebra. We say $\phi: A \rightarrow ®C$ is multiplicative linear functional $≡$ $\phi$ preserves $+, ·, ·_S$.
	$$ \Delta(A) := \{\phi: A \rightarrow ®C | \phi \text{ multiplicative linear functional }, \phi \not≡ 0\}. $$
\end{definice}

\begin{tvrzeni}
	$A$ Banach algebra, $\phi \in \Delta(A) \cup \{0\}$. Then\vspace{-1.5em}
	\begin{itemize}
		\item $\exists! \tilde\phi \in \Delta(A_e): \tilde \phi((x, 0)) = \phi(x), \forall x \in A$. It is given by $\tilde \phi((x, \lambda)) = \phi(x) + \lambda$. Moreover, $\Delta(A_e) = \{\tilde\phi | \phi \in \Delta(A) \cup \{0\}\}$.
		\item $\forall x \in A: \phi(x) \in \sigma(x)$ whenever $\phi \not≡ 0$.
		\item $\Delta(A) \subseteq B_{A^*}$.
		\item $A$ has a unit, $\phi \not≡ 0$ $\implies$ $\|\phi\| ≥ \frac{1}{\|e\|}$. In particular if $\|e\| = 1$, then $\|\phi\| = 1$.
	\end{itemize}

	\begin{dukazin}
		„1. uniqueness“: For $\tilde \phi \in \Delta(A_e)$ such that $\tilde \phi((x, 0)) = \phi(x)$, $x \in A$:
		$$ \tilde \phi((x, \lambda)) = \phi(x) + \lambda \tilde\phi((¦o, 1)) = \phi(x) + \lambda, $$
		second equality by $\phi \in \Delta(A) \implies \phi(e) = \phi(e^2) = \phi^2(e)$. „1. existence“ is proven by check that defined $\tilde \phi$ is multiplicative linear functional (and it is nonzero, but $\tilde\phi((0, 1)) = 1 ≠ 0$). This is easy (omitted).

		„$\Delta(A_e) = \{\tilde\phi | \phi \in \Delta(A) \cup \{0\}\}$“: „$\subseteq$“: $\phi \in LHS$, put $\phi(x) := \psi((x, 0))$. Then $\phi \in \Delta(A) \cup \{0\}$ and $\tilde \phi = \psi$ became:
		$$ \tilde \phi((x, \lambda)) = \phi(x) + \lambda = \psi((x, 0)) + \lambda = \psi((x, \lambda)). $$
		„$\supseteq$“: We know already that $\tilde \phi \in \Delta(A_e)$.

		„2. with $A$ has unit $e$“: $\phi ≠ 0$, $\phi \in \Delta(A)$: If $\lambda \in \rho(x)$, then $\phi(\lambda e - x) ≠ 0$ ($\phi(x) ≠ 0$ if $x^{-1}$ exists). $0 ≠ \phi(\lambda e - x) = \lambda - \phi(x)$ $\implies$ $\lambda ≠ \phi(x)$. Thus $\phi(x) \notin \rho(x)$, so $\phi(x) \in \sigma(x)$. „2. with $A$ hasn't unit“, then $\phi(x) = \tilde\phi((x, 0)) \in \sigma_{A_e}((x, 0)) = \sigma_A(x)$.

		„3.“: $\phi \in \Delta(A)$. Then $\forall x \in A: \phi(x) \in \sigma(x) \subseteq B(¦o, \|x\|)$, so $|\phi(x)| ≤ \|x\|$.

		„4.“: $A$ has a unit $e$, then $\|\phi\| ≥ \left|\phi\(\frac{e}{\|e\|}\)\right| = \frac{1}{\|e\|}$.
	\end{dukazin}
\end{tvrzeni}

\begin{veta}
	$A$ Banach algebra, $M := \Delta(A) \cup \{0\}$. Then $M \subset (B_{A^*}, w^*)$ is compact, $\Delta(A)$ is locally compact and if $A$ has u unit, then $\Delta(A)$ is compact. The mapping $\Phi: M \rightarrow \Delta(A_e)$, $\Phi(\phi) = \tilde \phi$ is $w^*$–$w^*$ homeomorphism.

	\begin{dukazin}
		By previous proposition, $M \subset (B_{A^*}, w^*)$ ($(B_{A^*}, w^*)$ is compact by previous semester). So, it suffices to check that $M$ is $w^*$-closed.
		$$ M = \bigcap_{x, y \in A} \{\phi \in A^* | \phi(x·y) = \phi(x)·\phi(y)\}. $$
		Sets from RHS is closed by previous semester, so, $M$ is closed. Thus $M$ is compact.

		$\Delta \subset M$ is open, so $\Delta(A)$ is locally compact (and $M$ is 1-point compactification of $\Delta(A)$). If $\Delta$ has a unit, then $\Delta(A) = \{\phi \in M | \phi(e) = 1\}$ is $w^*$-closed, so $\Delta(A)$ is compact (and $0$ is isolated in $M$).

		Finally, by previous proposition, $\Phi$ is bijection. $\Phi$ is $w^*$-continuous:
		$$ \phi_i \overset{w^*}\rightarrow \phi \implies \forall (x, \lambda): \tilde\phi_i((x, \lambda)) = \phi_i(x) + \lambda \rightarrow \phi(x) + \lambda = \tilde \phi((x, \lambda)) \implies \tilde\phi_i \overset{w^*}\rightarrow \tilde\phi $$
		So, $\Phi$ is homeomorphism (continuous bijection on compact, last semester?).
	\end{dukazin}
\end{veta}

\begin{priklady}
	$\Delta(©C(K)) = \{\delta_x | x \in K\}$. ($f \mapsto f(x)$ is multiplicative. Suppose $\phi \in \Delta(©C(K))$, $\phi \notin \{\delta_x | x \in K\}$. So for $x \in K$ there is $g_x \in C(x): \phi(g_x) ≠ g_x(x)$. Consider $f_x = g_x - \phi(g_x)$. Then $\phi(f_x) = 0$, $f_x(x) ≠ 0$. So there is $U_x$ open neighbourhood of $x$ such that $f_x ≠ 0$ on $U_x$. Compactness implies $\exists x_1, …, x_n \in K: K \subset \bigcup_{i=1}^n U_{x_i}$. Consider $h := \sum_{i=1}^n |f_{x_i}|^2$. Then $h > 0$ on $K$, so $h^{-1}$ exists and therefore $\phi(h) ≠ 0$. But $\phi(h) = \sum_{i=1}^n \phi(f_{x_i}) \overline{\phi_{x_i}} = 0$.)

	$\Delta \{M_n\} = \O$, $n ≥ 2$, where $M_n$ is (non-commutative) algebra of $n\times n$ matrices. ($M_n = \LO \{E^{i, j}\}$. $E^{ij}·E^{kl} = E^{il}$ if $j = k$, else $0$. So $\phi(E^{ij})·\phi(E^{ij}) = \phi(E^{ij}·E^{ij}) = 0$ if $i ≠ j$. $\phi(E^{ii}) = \phi(E^{in}E^{ni}) = \phi(E^{in})\phi(E^{in}) = 0$. $\phi(E^{nn}) = \phi(E^{n1}E^{1n}) = 0$.)
\end{priklady}

\begin{definice}[Ideal, maximal ideal]
	$A$ Banach algebra. Ideal in $A$ is a subspace $I \subset A$ if $\forall x \in I\ \forall y \in A: x·y \in I \land y·x \in I$.

	Maximal ideal $≡$ proper ($I ≠ A$) ideal and it is maximal proper ideal with respect to inclusion.

	\begin{priklady}[2021, Johnson-Schetman, Acta mathematica]
		$©L(L_p)$ has $2^{2^\omega}$ non-isomorphic closed ideals.
	\end{priklady}
\end{definice}

\begin{tvrzeni}
	$A$ Banach algebra with a unit. Then:

	\begin{itemize}
		\item Any proper ideal is contained in a maximum ideal. (From Zorn's lemma. And $I \subset A$ ideal is proper $\Leftrightarrow$ $e \notin I$.)
		\item $I \subset A$ proper ideal $\implies$ $\overline{I} \in A$ is proper ideal. In particular, maximal ideals are closed. (Easy: $\overline{I}$ is ideal. Moreover, $I \cap A^* = \O$ (if $x \in I$ was invertible thus $e = x·x^{-1} \in I$, but $e \notin I$). So ($A^*$ is open) $\overline{I} \cap A^* = \O$ and therefore $e \notin \overline{I}$.)
	\end{itemize}
\end{tvrzeni}

\begin{tvrzeni}
	$A$ Banach algebra, $I \subseteq A$ closed ideal $\implies A / I$ is Banach algebra ($[x]·[y] := [x·y]$).

	\begin{dukazin}
		Straightforward from definition. (Omitted.)
	\end{dukazin}
\end{tvrzeni}

\begin{poznamka}
	From now on, $A$ will be commutative.

	Step 1: „Hahn-Banach“: $I \subset A$ closed ideal $\implies$ $\exists \phi \in \Delta(A): \phi / I ≡ …$.
\end{poznamka}

\begin{veta}
	$A$ commutative Banach algebra with a unit. Then $\Phi: \Delta(A) \rightarrow \{\text{maximal ideals in $A$}\}$, $\Phi(\phi) := \Ker \phi$, is bijection.

	\begin{dukazin}
		Pick $\phi \in \Delta(A)$. Then „$\Ker \phi$ is maximal ideal“: ideal: $y \in \Ker \phi, x \in A: \phi(x·y) = \phi(x)·\phi(y) = …·0 = 0$, proper: $\phi \not≡ 0$, maximal: $\codim \Ker \phi = 1$: pick $x_0: \phi(x_0) ≠ 0$, $a = a - \phi(a)·\frac{x_0}{\phi(x_0)} + \phi(a)·\frac{x_0}{\phi(x_0)} \in \Ker \phi \oplus ®R$.

		„$\Phi$ is one-to-one“: Pick $\phi, \psi \in \Delta(A)$: $\Ker \phi = \Ker \psi$. Then (by lemma from previous semester) $\phi = c·\psi$ for some $c \in ®K$. But $\phi(e) = 1 = \psi(1)$ so $\phi = \psi$.

		„$\Phi$ is surjective“: Let $I \subset A$ be maximal ideal ($\implies$ closed). Step 1 „Any nonzero element in $A / I$ is invertible“: For contradiction assume $\exists q(x) \in A / I$ ($q(x) = [x]$), $q(x) ≠ 0 \land q(x)^{-1}$ does not exist. By next lemma $q(x) (A / I)$ is proper ideal. Then $q^{-1}\(q(x) (A / I)\)$ is an ideal in $A$ which is proper and $I \subsetneq q^{-1}\(q(x) (A / I)\)$, which contradicts maximality of $I$. It follows from: ideal: follows from the fact that $q$ is algebra homomorphism; proper: $q(e) = [e] \notin q(x) A / I$; $I \subseteq q^{-1}(…)$: $0 \in q(x) A / I$; $I ≠ q^{-1}(…)$: $q(x) ≠ 0 \implies x \notin I$, but $q(x) = q(x)q(e) \in q(x) (A / I)$, so $x \in q^{-1}(…)$.

% 07. 03. 2023

		From Gelfand–Mazur theorem $\exists$ surjective isomorphism $j: A / I \rightarrow ®C$. Then $\phi := j \circ q \in \Delta(A)$. It remains „$I = \Ker \phi$“: $x \in \Ker \phi \Leftrightarrow j(q(x)) = 0 \Leftrightarrow q(x) = 0 \Leftrightarrow x \in I$.
	\end{dukazin}
\end{veta}

\begin{lemma}
	$A$ \emph{commutative} Banach algebra with a unit, $x \in A$ does not have inverse $\implies$ $xA$ is proper ideal.

	\begin{dukazin}
		$x A$ is ideal, because $A$ is commutative. Then $x A$ is proper ($e \notin x A$).
	\end{dukazin}
\end{lemma}

\begin{dusledek}[Hahn–Banach like theorem]
	$A$ is commutative Banach algebra with a unit, $I \subset A$ proper ideal. Then $\exists \phi \in \Delta(A): \phi / I ≡ 0$.

	\begin{dukazin}
		Let $\tilde I \supseteq I$ be maximal ideal. By previous theorem there is $\phi \in \Delta(A): \tilde I = \Ker \phi$.
	\end{dukazin}
\end{dusledek}

\begin{tvrzeni}
	$A$, $B$ Banach algebras, $Φ: A \rightarrow B$ algebraic isomorphism. Then $Φ^{\#}: \Delta(B) \rightarrow \Delta(A)$ defined as $Φ^{\#}(φ) := φ \circ Φ$ is homeomorphism.

	\begin{dukazin}
		„$Φ^{\#}(\phi) \in \Delta(A)$“: $Φ^{\#}(φ) = φ \circ Φ \in \Delta(A) \cup \{0\}$ and $φ \not≡ 0 \land Φ$ is onto $\implies φ \circ Φ ≠ 0$.

		„$Φ^{\#}$ is $w^*$-$W^*$ continuous“: $φ_i \overset{w^*}\rightarrow φ \implies φ_i \circ Φ \overset{w^*}\rightarrow φ \circ Φ$.

		Apply the proven part to $Φ^{-1}$, obtain that $(Φ^{-1})^{\#}: \Delta(A) \rightarrow \Delta(B)$ is $w^*$-$W^*$ continuous. Moreover we have $Φ^{\#} \circ (Φ^{-1})^{\#} = \id \land (Φ^{-1})^{\#} \circ Φ^{\#}$.
	\end{dukazin}
\end{tvrzeni}

\begin{tvrzeni}
	$L$ locally compact $T_2$. Then $\delta: L \rightarrow \Delta(C_0(L)), x \mapsto δ_x$ is homeomorphism onto.

	\begin{dukazin}
		„Case 1: $L$ is compact“: By example $δ$ is onto. Of course, $δ$ is one-to-one, continuous. So $δ$ is homeomorphism.

		„Case 2: $L$ is not compact“: Then there is $K = L \cup \{∞\}$, one-point compactification, and $\{f \in ©C(K) | f(∞) = 0\} \ni f \mapsto f|_L \in C_0(L)$ is isometric isomorphism. Moreover $Φ: ©C_0(L)_e \rightarrow ©C(K)$, $Φ(f, \lambda) := f + \lambda$, is algebraic isomorphism.

		So, we have $K \overset\eta\rightarrow Δ(C(K)) \overset{Φ^{\#}}\rightarrow Δ(C_0(L)_e) \overset\psi\rightarrow Δ(C_0(L)) \cup \{0\}$, where $\eta$ is homeomorphism from case 1 and $\psi(φ) := φ |_{C_0(L)}$.

		Thus $δ := \psi \circ Φ^{\#} \circ \eta$ is homeomorphism between $L \cup \{∞\}$ and $Δ(C_0(L)) \cup \{0\}$. Finally, for $x \in K$ and $f \in ©C_0(L)$:
		$$ Φ^{\#} \circ η(x)(f) = (η(x) \circ Φ)(f) = f(x), $$
		so $ψ \circ Φ^{\#} \circ η(x) = Φ^{\#} \circ η(x) |_{C_0(L)} = \delta_x |_{C_0(L)}$.
	\end{dukazin}
\end{tvrzeni}

\begin{veta}
	$K$, $L$ locally compact $T_2$. Then following is ekvivalent

	\begin{itemize}
		\item $©C_0(K) ≡ ©C_0(L)$ as Banach algebra;
		\item $©C_0(K) ≡ ©C_0(L)$ as algebras;
		\item $K ≈ L$ as topological spaces.
	\end{itemize}

	\begin{dukazin}
		„$1 \implies 2$“ trivial. „$2 \implies 3$“: $K ≈ Δ(©C_0(K)) ≈ Δ(©C_0(L)) ≈ L$ from previous two tvrzeni. „$3 \implies 1$“: Given $h: K \rightarrow L$ homeomorphism, $f \mapsto f \circ h$ is isometry between Banach algebras.
	\end{dukazin}
\end{veta}

% 13. 03. 2023

\begin{definice}[Semi-simple Banach algebra]
	$A$ commutative Banach algebra. It is semi-simple $≡$ $Δ(A)$ separates points of $A$. ($\Leftrightarrow \bigcap \{\Ker \phi | \phi \in Δ(A)\} = \{¦o\}$.)

	\begin{poznamkain}
		Semi-simple $\implies$ commutative. (Semi-simple and $x·y ≠ y·x \implies \exists \phi \in Δ(A): φ(x)·φ(y) = φ(x·y) ≠ φ(y·x) = φ(y)·φ(x)$ \lightning.)
	\end{poznamkain}
\end{definice}

\begin{veta}
	$A, B$ Banach algebras, $B$ is semi-simple, then every (algebra) homomorphism $Φ: A \rightarrow B$ is continuous.

	\begin{dukazin}
		Use Closed graph theorem. Pick $x_n \rightarrow x$, $φ(x_n) \rightarrow y$. Wanted „$Φ(x) = y$“ ($\Leftrightarrow \forall φ \in Δ(B): φ(Φ(x)) = φ(y)$). For $φ \in Δ(B)$ we have $φ(y) = \lim_n φ(Φ(x_n)) \overset{φ∘Φ \in \Delta(A) \subseteq A^*}= φ∘Φ(\lim_n x_n) = φ(Φ(x))$.
	\end{dukazin}
\end{veta}

\begin{dusledek}
	$(A, \|·\|)$ semi-simple Banach algebra and $(A, \| |·| \|)$ is Banach algebra (with other norm), then $\|·\|$ and $\| |·| \|$ are equivalent.

	\begin{dukazin}
		We have that $\id:(A, \| |·| \|) \rightarrow (A, \|·\|)$ is algebra homomorphism, so continuous by previous theorem. Similarly inverse is continuous (semi-simplicity doesn't depend on norm). So, $\id$ is isomorphism.
	\end{dukazin}
\end{dusledek}

\section{Gelfand transformation}
\begin{definice}[Gelfand transformation]
	$A$ Banach algebra. For $x \in A$ we define $\hat{x}: Δ(A) \rightarrow ®C$, $\hat{x}(φ) := φ(x)$. We say that $\hat{x}$ is Gelfand transformation of $x$.

	\begin{poznamka}
		$\hat{x} \in ©C_0(Δ(A))$.

		$A = ©C_0(L) \implies Δ(A) = \{δ_x | x \in L\} \implies \forall f \in A: \hat{f}(δ_x) = f(x), x \in L$. So, $\hat{f} = f$.

		$A = L_1(®R^d) \implies Δ(A) = \{e^{it·x}  | x \in ®R\} \subseteq L_∞(®R^d) = A^*$ and $\hat{f}$ is Fourier transformation.
	\end{poznamka}
\end{definice}

\begin{veta}
	$A$ commutative Banach algebra, $x \in A$. Then

	\begin{itemize}
		\item $A$ has a unit $\implies$ $ς(x) = \Rang \hat{x}$;
		\item $A$ doesn't have a unit $\implies$ $ς(x) = \Rang \hat{x} \cup \{0\}$;
		\item $\|\hat{x}\|_∞ = r(x) = \sup\{|λ| | λ \in \sigma(x)\}$.
	\end{itemize}

	\begin{dukazin}
		„a) $\subseteq$“: $λ \in ς(x) \Leftrightarrow (λ·e - x)^{-1}$ does not exists $\implies$ (Lemma above) $(λ e - x)A$ is proper ideal $\implies$ $\exists φ \in Δ(A): φ|_{(λ e - x)A} ≡ 0$ $\implies \exists φ \in Δ(A): 0 = φ(λ e - x) = λ - φ(x) = λ - \hat{x}(φ)$ $\implies λ \in \Rang \hat{x}$.

		„$\supseteq$“ follows from Tvrzeni above, $φ(x) \in ς(x)$ for $φ \in Δ(A)$.

		„b)“ For $x \in A:$
		$$ ς(x) = ς_{A_e}((x, 0)) \overset{\text{a)}}= \Rng \hat{(x, 0)} = \(\{\tilde φ | φ \in Δ(A) \cup \{0\}\}\) = $$
		$$ = \{φ(x) | φ \in Δ(A) \cup \{0\}\} = \Rang \hat{x} \cup \{0\}. $$

		„c)“ $\|\hat{x}\|_∞ = \sup\{|λ| | λ \in \Rang \hat{x}\} = \sup\{|λ| | λ \in \Rang \hat{x} \cup \{0\}\} = \sup\{|λ| | λ \in ς(x)\} = r(x)$.
	\end{dukazin}
\end{veta}

\begin{definice}[Gelfand transformation of algebra]
	$A$ Banach algebra, then $Γ: A \rightarrow ©C_0(Δ(A))$, $Γ(x) := \hat{x}$ is the Gelfand transformation of $A$.
\end{definice}

\begin{veta}
	$A$ commutative Banach algebra, $Γ$ Gelfand transformation. Then

	\begin{itemize}
		\item $Γ$ is algebra transformation, continuous, $\|Γ\| ≤ 1$;
		\item $Γ(A)$ separates the points of $Δ(A)$;
		\item $Γ$ is one-to-one $\Leftrightarrow$ $A$ is semi-simple;
		\item $Γ$ is an isomorphism into $\Leftrightarrow$ $\exists K > 0: \|x^2\| ≥ K·\|x\|^2$, $x \in A$; ($\Leftrightarrow$ $Γ$ is one-to-one and $Γ(A)$ is closed;)
		\item $Γ$ is an isometry into $\Leftrightarrow$ $\|x^2\| = \|x\|^2$, $x \in A$.
	\end{itemize}

	\begin{dukazin}
		„a)“: $Γ$ is linear (obvious), $Γ$ preserves multiplication (obvious). Finally, $\|Γ(x)\|_∞ = \|\hat{x}\|_∞ = r(x) ≤ \|x\|$. So $\|Γ\| ≤ 1$.

		„b)“: Let $φ ≠ ψ \in Δ(A)$ and $x \in A: \hat{x}(φ) = φ(x) ≠ ψ(x) = \hat{x}(ψ)$.

		„c)“: $Γ(x) = 0 \Leftrightarrow \hat{x}(φ) = 0, φ \in Δ(A) \Leftrightarrow φ(x) = 0, φ \in Δ(A)$. So, $Γ$ is one-to-one $\Leftrightarrow \forall x ≠ 0\ \exists φ \in Δ(A): φ(x) ≠ 0 \Leftrightarrow A$ is semi-simple.

		„d) second“: $Γ$ is isomorphism into $\Leftrightarrow$ $Γ$ is bijection between $A$ and $Γ(A)$ $\land$ $Γ(A)$ is closed. ($Γ(A)$ is closed, then we use Open mapping theorem; if $Γ$ is isomorphism, $Γ(A)$ is a Banach space.).

		„d) + e), $\implies$“: Suppose $\exists c > 0$: $\|Γ(x)\| ≥ c·\|x\|$, $x \in A$. Then $\forall x \in A: \|x^2\| \overset{\text{a)}}≥ \|Γ(x^2)\| = \|Γ(x)\|^2 ≥ c^2·\|x\|^2$.

		„d) + e), $\impliedby$“: Let d) hold with $K$ (this holds in every algebra). Then (proven by induction)
		$$ \forall x \in A: \|x^{2^n}\| ≥ K^{2^{n - 1}} \|x\|^{2^n}, \qquad n \in ®N. $$
		$$ \implies \sqrt[2^n]{\|x^{2^n}\|} ≥ K^{1 - 2^{-n}} \|x\|, $$
		where left side converges (by Beurling) to $r(x)$ and right side converges to $\|x\|$. So $r(x) ≥ K·\|x\|$ and from previous theorem $r(x) ≥ \|\hat{x}\|_∞ = \|Γ(x)\|$.
	\end{dukazin}
\end{veta}

\subsection{\texorpdfstring{$C^*$}{C*}-algebras}
\begin{definice}[Involution]
	$A$ is a Banach algebra. Involution is a mapping $*: A \rightarrow A$ such that
	$$ \forall x, y \in A\ \forall λ \in ®C: $$
	$$ (x + y)^* = x^* + y^*, \qquad (λx)^* = \overline{λ} x^*, \qquad (xy)^* = y^*·x^*, \qquad (x^*)^* = x. $$
\end{definice}

\begin{definice}[$C^*$-algebra]
	Banach algebra with involution $*$ is a $C^*$-algebra, if
	$$ \forall x \in A: \|x·x^*\| = \|x\|^2, x \in A. $$
\end{definice}

\begin{definice}[Self-adjoint element, normal element]
	For $A$ with involution $*$ and $x \in A$ we say that $x$ is self-adjoint $≡$ $x = x^*$, and $x$ is normal $≡$ $x·x^* = x^*·x$.
\end{definice}

\begin{tvrzeni}[Properties]
	$A$ Banach algebra with involution, $x \in A$. Then

	\begin{itemize}
		\item $e$ is left/right unit $\implies$ $e$ is unit and $e = e^*$. ($e$ is left unit $\Leftrightarrow$ $e^*$ is right unit. So there is unit.)
		\item $A$ is $C^*$-algebra $\Leftrightarrow$ $\|x·x^*\| ≥ \|x\|^2$, $x \in A$. Then $\|x^*\| = \|x\|$, $x \in A$. („$\implies$“: clear, „$\impliedby$“: Then $\forall x \in A: \|x\|^2 ≤ \|x·x^*\| ≤ \|x\|·\|x^*\|$, so $\|x\| ≤ \|x^*\|$, and applying to $x^*$ we get $\|x^*\| ≤ \|x\|$. But then we have $\|x·x^*\| ≤ \|x\|·\|x^*\| = \|x\|^2$.)
		\item Let $A$ has a unit. then $x^{-1}$ exists $\Leftrightarrow$ $(x^*)^{-1}$ exists. Then $(x^*)^{-1} = (x^{-1})^*$. („$\implies$“: $x^*·(x^{-1})^* = (x^{-1}x)^* = e^* = e$, analogically $(x^{-1})^* x^* = e$. „$\impliedby$“: Apply the proven part to $x^*$.)
		\item $λ \in ς(x) \Leftrightarrow \overline{λ} \in ς(x^*)$. ($A$ has a unit: $λ \notin ς(x) \Leftrightarrow \exists(λe - x)^{-1} \Leftrightarrow \exists\((λ e - x)^*\)^{-1} \Leftrightarrow \overline{λ} \notin ς(x^*)$. If $A$ has not a unit, then we use previous sentence and next theorem?)
		\item $x + x^*$, $x^*·x$, $x·x^*$, $i·(x - x^*)$ are self-adjoint. (Easy, omitted.)
		\item $\exists! u, v \in A$ self-adjoint: $x = u + i·v$. Then $x^* = u - i·v$, and $x$ is normal $\Leftrightarrow$ $uv = vu$.
			(„Existence“: $u := \frac{1}{2} (x + x^*), v:= \frac{1}{2i}(x - x^*)$. Then $x = u + iv$. „Formulas“: $(u + i·v)^* = u^* + \overline{i} v^*$. „Uniqueness“: Pick $a, b \in A_{sa}: x = a + i·b$. Then $a + i·b = x = u + i·v$, $a - i·b = x^* = u - i·v$. By subtracting or summing equation we get $a = u$ and $b = v$. „Normality“: $x$ normal $\Leftrightarrow$ $(u + i·v)(u - i·v) = (u - i·v)(u + i·v)$ $\Leftrightarrow$ $-i·u·v + i·v·u = i·u·v - i·v·u$ $\Leftrightarrow$ $u·v = v·u$.)
	\end{itemize}
\end{tvrzeni}

% 14. 03. 2023

\begin{veta}
	$A$ is $C^*$-algebra, $x \in A$ is normal. Then $r(x) = \|x\|$.

	\begin{dukazin}
		„Step 1: $\|x^2\| = \|x\|^2$“:
		$$ \|x\|^4 = \|x^* x\|^2 = \|(x^*x)^* (x^*x)\| = \|x^*x x^* x\| = \|x^*x^*x x\| = \|(x x)^* x x\| = \|x x\|^2 = \|x^2\|^2. $$
		Thus inductively, we obtain $\|x^{2^k}\| = \|x\|^{2^k}$, $k \in ®N$. Thus, Beurling gives
		$$ r(x) = \lim_k \sqrt[2^k]{\|x^{2^k}\|} = \|x\|. $$
	\end{dukazin}
\end{veta}

\begin{dusledek}
	$A$ (Banach) algebra with involution. Then there is at most one norm $\|·\|$ on $A$, such that $(A, \|·\|)$ is $C^*$-algebra.

	\begin{dukazin}
		If $\|·\|_1$ and $\|·\|_2$ are norms on $A$ such that $(A, \|·\|)$ is $C^*$-algebra, then by previous theorem
		$$ \forall x \in A: \|x\|_1^2 = \|x^* x\|_1 = r(x^* x) = \|x^* x\|_2 = \|x\|_2^2. $$
	\end{dukazin}
\end{dusledek}

\begin{veta}
	$(A, \|·\|)$ Banach algebra.

	\begin{itemize}
		\item $(a, \lambda)^* = (a^*, \overline{\lambda})$, $(a, \lambda) \in A_e$ defines an involution on $A_e$. (Trivial.)
		\item If $A$ is $C^*$-algebra, then on $A_e$ there exists a norm $\| |·| \|$ (equivalent to the norm from $A \oplus_1 ®K$) such that $(A_e, \| |·| \|)$ is $C^*$-algebra and $\| |(a, 0)| \| = \|a\|$, $a \in A$.
	\end{itemize}
\end{veta}

\begin{veta}
	$A$ is $C^*$-algebra, $x \in A$. Then
	
	\begin{itemize}
		\item $x = x^* \implies ς(x) \subseteq ®R$;
		\item $A$ has a unit and $x^* = x^{-1}$ (that is, $x$ is unitary) $\implies$ $ς(x) \subseteq \{λ | |λ| = 1\}$.
	\end{itemize}

	\begin{dukazin}
		By previous theorem, WLOG $A$ has a unit.

		„a)“: Let $\alpha + i\beta \in ς(x)$, $\alpha, \beta \in ®R$. We want $\beta = 0$. Trick: $x_t := x + i·t·e$, $t \in ®R$. Then
		$$ \alpha + i·(\beta + t) \in ς(x_t) (\impliedby (\alpha + i(\beta + t))e - x_t = (\alpha + i·\beta)e - x), $$
		$$ \alpha^2 + (\beta + t)^2 = |\alpha + i(\beta + t)|^2 ≤ \|x_t\|^2 = \|x_t^* x_t\| = \|(x - i·t·e)·(x + i·t·e)\| = \|x^2 + (t·e)^2\| ≤ $$
		$$ ≤ \|x^2\| + t^2. $$
		So, $\alpha^2 + (\beta + t)^2 - t^2 ≤ \|x^2\|$, $t \in ®R$ $\implies$ $\beta = 0$ (Otherwise $LHS \rightarrow +∞$ for $t \rightarrow ±∞$.)

		„b)“: ($\|e\| = \|e^2\| = \|e\|^2$.) $1 = \|e\| = \|x^*x\| = \|x\|^2$, so $\|x\| = 1$. Then, for $λ \in ς(x)$, we have $|λ| ≤ \|x\| = 1$. On the other hand $\frac{1}{λ} \in ς(x^{-1})$ (because if not, then $\frac{1}{λ}e - x^{-1}$ ha inverse $\implies$ $λ e - x = (λ e - x)x^{-1}x = (λ x^{-1} - e)x = -λ (\frac{1}{λ}e - x^{-1})x$ $\implies$ $λ e - x$ has inverse.) So
		$$ \left|\frac{1}{λ}\right| ≤ \|x^{-1}\| = \|x^*\| = \|x\| = 1. $$
	\end{dukazin}
\end{veta}

% 20. 03. 2023

\begin{definice}
	$A, B$ are $C^*$-algebras, then $Φ: A \rightarrow B$ is $*$-homomorphism if $Φ$ is homomorphism preserving $*$ (that is, $Φ(x^*) = (Φ(x))^*$).
\end{definice}

\begin{dusledek}
	Let $A$ be a $C^*$-algebra and $Φ \in Δ_A$. Then $Φ$ is $*$-homomorphism.

	\begin{dukazin}
		„If $x = x^*$“, then $Φ(x) \in ς(x) \subseteq ®R$, so $Φ(x^*) = Φ(x) = \overline{Φ(x)}$.

		„In general“, if $x = u + i·v$ ($u = u^*$, $v = v^*$), then $Φ(x^*) = Φ(u - i·v) = Φ(u) - i·Φ(v) = \overline{Φ(u) + i·Φ(v)} = \overline{Φ(x)}$.
	\end{dukazin}
\end{dusledek}

\begin{tvrzeni}[Automatical continuous]
	Let $A, B$ be $C^*$-algebras, $Φ: A \rightarrow B$ is $*$-homomorphism. Then $Φ$ is continuous and $\|Φ\| ≤ 1$.

	\begin{dukazin}
		$$ \forall x \in A: \|Φ(x)\|^2 = \|Φ(x)^*·Φ(x)\| = r(Φ(x^*)·Φ(x)) = r(Φ(x^*x)) \overset{*}= r(x^* x) = \|x^*x\| = \|x\|^2.  $$
		Thus it suffices to show that (by following lemma)
		$$ ς(Φ(x^* x)) \subseteq ς(x^* x) \cup \{0\}. $$
	\end{dukazin}
\end{tvrzeni}

\begin{lemma}
	Let $A, B$ be Banach algebras, $Φ: A \rightarrow B$ algebra homomorphism. Then $\forall x \in A: ς_B(φ(x)) \subseteq ς_A(x) \cup \{0\}$.

	\begin{dukazin}
		Consider $\tilde Φ: A_e \rightarrow B_e$ defined as $\tilde Φ(a, λ) := (Φ(a), λ)$. Then $\tilde Φ$ is algebra homomorphism preserving unit. Moreover $ς_B(Φ(x)) \subseteq ς_{B_e}((Φ(x), 0)) \cup \{0\}$ and $ς_{A_e}((x, 0)) \subseteq ς_A(x) \cup \{0\}$. Thus, WLOG $A, B$ have units and $Φ(e_A) = e_B$.

		But then for $λ ≠ 0$ and $x \in A: λ e - x$ has inverse in $A$, then $Φ(λ e - x) = λ Φ(e) - Φ(x)$ has inverse in $B$. So, $λ \notin ς_A(x) \cup \{0\} \implies \lambda \notin ς_B(Φ(x))$.
	\end{dukazin}
\end{lemma}

\begin{veta}[Gelfand–Naimark]
	$A$ commutative $C^*$-algebra. Then the Gelfand transformation $Γ: A \rightarrow ©C_0(Δ(A))$ is isometric $*$-isomorphism onto.

	\begin{dukazin}
		By proposition above, $Γ$ is algebra homomorphism, $\|Γ\| ≤ 1$ and from theorem above $\|Γ(x)\|_∞ = r(x)$, $x \in A$. „$Γ$ is $*$-homomorphism“:
		$$ \forall a \in A\ \forall φ \in Δ(A): Γ(a^*)(φ) = φ(a^*) = \overline{(φ(a))} = \overline{Γ(a)}(φ). $$
		„$Γ$ is isometry“:
		$$ \forall x \in A: \|Γ(x)\|^2 = \|\overline{Γ(x)}·Γ(x)\| = \|Γ(x^*x)\| = r(x^*x) = \|x^x\| = \|x\|^2. $$
		„$Γ$ is onto“: $Γ(A)$ is Banach space so $Γ(A) \subseteq ©C_0(Δ(A))$ is closed and $*$-subalgebra. And $Γ(A)$ separates points of $Δ(A)$. So from Stone–Weierstrass theorem ($A \subset ©C_0(K)$ is $*$-subalgebra separating the points, then $\overline{A}^{\|·\|} = ©C_0(K)$) $Γ(A) = ©C_0(Δ(A))$.
	\end{dukazin}
\end{veta}

\begin{dusledek}
	$A, B$ commutative $C^*$-algebras. Then the following items are equivalent:\vspace{-1em}
	\begin{itemize}
		\item $A$ and $B$ are isometrically $*$-isomorphic;
		\item $A$ and $B$ are algebraically isomorphic;
		\item $Δ(A)$ and $Δ(B)$ are homeomorphic.
	\end{itemize}

	\begin{dukazin}
		„2. $\Leftrightarrow$ 3.“ follows from theorem above (where it is proved for $©C_0(K)$-spaces). „1. $\implies$ 2.“: trivial.

		„3. $\implies$ 1.“: easy for $©C_0(K)$-spaces, because if $h: K \rightarrow L$ is homeomorphism, then $f \mapsto f \circ h$ is isometrical $*$-isomorphism.
	\end{dukazin}
\end{dusledek}

\begin{definice}
	$A$ Banach algebra, $M \subset A$. Then $alg(M) = \bigcap \{B \supseteq M | B \text{ is subalgebra of } A\}$.

	\begin{poznamka}[Easy]
		$$ = \{\sum_{i=1}^n \alpha_i \prod_{j=1}^m x_{ij} | n, m \in ®N, \alpha_i \in ®C, x_{ij} \in M\}. $$
	\end{poznamka}

	Moreover $\overline{alg} M = \bigcap \{B \supseteq M | B \text{ is closed subalgebra of } A\}$.

	\begin{poznamka}[Easy]
		$$ = \overline{alg M}^{\|·\|}. $$
	\end{poznamka}
\end{definice}

\begin{tvrzeni}[Fact]
	$A$ is $C^*$-algebra, $M \subset A$ is commutative and closed under $*$, then $\overline{alg} M$ is commutative $C^*$-subalgebra of $A$.
\end{tvrzeni}

\begin{veta}
	$A, B$ are $C^*$-algebras, $h: A \rightarrow B$ is $*$-homomorphism, one-to-one. Then $h$ is isometry.

	\begin{dukazin}
		WLOG $A$, $B$ have units and $h(e)$ is a unit ($(a, λ) \mapsto (h(a), λ)$ is 1-to-1 \hbox{$*$-homomorphism}). Suffices: $\forall x \in A$ self-adjoint: $\|x\| = \|h(x)\|$ ($\forall y \in A: \|h(y)\|^2 = \|h(y^*y)\| = \|y^* y\| = \|y\|^2$). Let $x \in A$ be self-adjoint. Put $A_0 := \overline{alg}\{e, x\} = \overline{\LO}\{e, x, x^2, x^3, …\}$ is commutative and $C^*$-subalgebra.

		$$ B_y := \overline{alg}\{e, h(x)\} = \overline{\LO}\{e, h(x), h(x^2), …\} $$
		is commutative and $C^*$-subalgebra. So, we have $A_0 \overset{h} \rightarrow B_0 \overset{Γ}\rightarrow ©C(Δ(B_0))$, $A_0 \overset{Γ}\rightarrow ©C(Δ(A_0))$. So there is $\tilde h: ©C(Δ(A_0)) \rightarrow ©C(Δ(B_0))$ one-to-one $*$-homeomorphism, $\tilde h(1) = 1$. So, it suffices to prove the following lemma.
	\end{dukazin}
\end{veta}

\begin{lemma}
	Let $K, L$ be $T_2$ compact spaces, $φ: ©C(K) \rightarrow ©C(L)$ $*$-homomorphism, $φ(1) = 1$. Then $\exists α: L \rightarrow K$ continuous mapping such that $φ(f) := f ∘ α$, $f \in ©C(K)$.

	Moreover, if $φ$ is one-to-one, then $α$ is onto and so $φ$ is isometry.

	\begin{dukazin}
		By proposition above $\|φ\| ≤ 1$ and $φ$ is continuous. Consider $φ^*: ©M(L) \rightarrow ©M(K)$. Then „$φ^*(Δ(©C(L)) \subseteq Δ(©C(K))$“:
		$$ \forall h \in Δ(©C(L))\ \forall f, g: φ^*h(f g) = h(φ(f g)) = h(φ(f))h (φ(g)) = φ^*h(f) φ^*h(g). $$

		So, we have: $L \overset{δ}\rightarrow Δ(©C(L)) \overset{φ^*}\rightarrow Δ(©C(K)) \overset{δ^{-1}} \rightarrow K$. So, $α(x) := δ^{-1}(φ^*(δ(x)))$, $x \in L$ is continuous from $L$ to $K$.

		For this $α$ we have:
		$$ \forall x \in L\ \forall f \in ©C(K): φ(f)(x) = δ_x(φ(f)) = (φ^* ∘ δ_x)(f) = f(δ^{-1} φ^* δ_x) = f(α(x)). $$

		Moreover, „if $φ$ is one-to-one, then $α$ is onto“: Suppose $α(L) \subsetneq K \implies \exists f \in C(K) \setminus \{0\}: f|_{\alpha(L)} ≡ 0$. But then $φ(f) ≡ 0$, but $f ≠ 0$. \lightning ($φ$ should be one-to-one.) Thus $φ$ is isometry.
	\end{dukazin}
\end{lemma}

\begin{poznamka}[GNS construction]
	$A$ is $C^*$-algebra $\implies$ $\exists H$ Hilbert $\exists φ: A \rightarrow B(H)$ $*$-isomorphism into.

	\begin{dukazin}[Sketch]
		$f ≥ 0$ ($\sigma(f) ≥ 0$) on $A|_{\{a | f(a*a) = 0\}}$ constructs inner product $\<[x], [y]\> := f(y^* x)$. Put $H := \overline{A|_{\{a | f(a*a) = 0\}}}$. Then $φ(a)([x]) = [a x]$.
	\end{dukazin}
\end{poznamka}

\section{Continuous calculus for formal elements of \texorpdfstring{$C^*$}{C*}-algebras}
\begin{poznamka}
	Idea: $φ(ς(x)) \ni f \mapsto f(x) \in A$.

	For $A = C(K)$:
	$$ g \in ©C(K), φ(ς(x)) \ni f \implies g \circ f \in C(K). $$

	Let $A$ be $C^*$-algebra with a unit, $x \in A$ normal. Consider
	$$ B = \overline{alg}\{e, x, x^*\} \in A \implies Γ_B: B \rightarrow ©C(Δ(B)) \land f(x) := Γ_B^{-1}(f∘Γ_B (x)), f \in ©C(ς_A(x)). $$
	Problem is when $Γ_B(x) \subseteq ς_A(x)$.
\end{poznamka}

% 21. 03. 2023

\begin{lemma}
	$A$ is $C^*$-algebra, $B \subset A$ is $C^*$-algebra. Then

	\begin{itemize}
		\item If $A$ and $B$ have the same unit $\implies$ $\forall x \in B: \exists x^{-1} \in B \Leftrightarrow \exists x^{-1} \in A$;
		\item $\forall x \in B:$ $B$ has a unit, which is not a unit in $A$ $\implies$ $ς_A(x) = ς_B(x) \cup \{0\}$, otherwise $ς_A(x) = ς_B(x)$;
		\item (In any case $ς_B(x) = ς_A(x)$).
	\end{itemize}

	\begin{dukazin}
		„1.“: Pick $x \in B$. „$\implies$“: easy. „$\impliedby$“: If $x^{-1}$ exists in $A$, then $(x^* x)^{-1}$ exists in $A$. So $0 \notin ς_A(x^* x) = ς_B(x^* x) \implies (x^*x)^{-1}$ exists in $B$. $x^{-1} = x^{-1}(x^*)^{-1} x^* = x^{-1} (x^* x)^{-1} x^*$.

		„2.“: If $A$ and $B$ have the same unit, we have $ς_A(x) = ς_B(x)$. WLOG $A$ has a unit $e \notin B$ (Because $B \in A_e$ and $ς_{A_e}(x) = ς_A(x)$ if $A$ has not unit). Then $ς_A(x) = ς_{B + \LO(e)}(x) \overset*= ς_{B_e}((x, 0)) =$ $ς_B(x)$ if $B$ has no unit and $ς_B(x) \cup \{0\}$ if $B$ has a unit.

		$*$: $φ: B + \LO(e) \rightarrow B_e, b + λ e \mapsto (b, λ)$ is algebra homomorphism.
	\end{dukazin}
\end{lemma}

\begin{veta}
	Let $A$ be a $C^*$-algebra with a unit, $x \in A$ normal, $f \in ©C(ς(x))$. Then the mapping
	$$ Φ: ©C(ς(x)) \rightarrow A, \qquad Φ(g) := g(x) := Γ_B^{-1}(g∘Γ_B(x)) $$
	has the following properties:

	\begin{enumerate}
		\item $Φ$ is isometric $*$-isomorphism onto $B = \overline{alg}\{e, x, x^*\}$, $Φ(1) = e$ and $Φ(\id) = x$.
		\item If $ψ: ©C(ς(x)) \rightarrow A$ is $*$-homomorphism, $ψ(1) = e$, $ψ(\id) = x$, then $ψ = Φ$.
		\item If $g \in ©H(Ω)$, where $Ω \subset ®C$ open, $ς(x) \subset Ω$, then $Φ(g|_{ς(x)}) = ψ(g)$, where $ψ$ is from holomorphic calculus.
		\item $f(x)^{-1}$ exists in $A$ $\Leftrightarrow$ $f ≠ 0$ on $ς(x)$. In this case $f(x)^{-1} = \(\frac{1}{f}\)(x)$.
		\item $ς(f(x)) = f(ς(x))$.
		\item $\forall g \in ©C(f(ς(x))): (g∘f)(x) = g(f(x))$.
		\item $\forall y \in A: yx = xy: y f(x) = f(x) y$.
	\end{enumerate}

	\begin{dukazin}
		„1.“: Recall theorem above $Γ_B(x): Δ(B) \rightarrow ©C$ continuous and onto $ς_B(x)$. And it is „one-to-one“:
		$$ \forall φ_1, φ_2 \in Δ(B): φ_1(x) = φ_2(x) \implies φ_1 = φ_2 \text{ on } B. $$

		So $Γ_B(x): Δ(B) \rightarrow ς(x)$ is homeomorphism, then $©C(ς(x)) \ni g \mapsto g∘Γ_B(x) \in ©C(Δ(A))$ is isometric $*$-isomorphism onto. Thus $©C(ς(x)) \ni g \mapsto Γ_B^{-1}(g∘Γ_B(x)) \in B$ is isometric $*$-isomorphism onto.

		Moreover, $Φ(1) = Γ_B^{-1}(1) = e$ ($\forall φ \in Δ(B): φ(e) = 1$). $Φ(\id) = Γ_B^{-1}(Γ_B(x)) = x$.

		„2.“: By theorem above, $ψ$ is continuous (because it is $*$-isomorphism), moreover $ψ = Φ$ on complex polynomials. Since complex polynomials are dense in $©C(ς(x))$ by (S-W), by continuity $Φ = ψ$ everywhere.

		„3.“: Omitted (on polynomials, on inverse, on rationals, rationals are dense in $©H$).

		„4.“: Since $f(x) \in B$, we have $f(x)^-1{}$ exists in $B$ $\Leftrightarrow$ $f(x)^{-1}$ exists in $A$ $\overset{Φ \text{ is ?}}\Leftrightarrow$ $f^{-1}$ exists in $©C(ς(x))$ $\Leftrightarrow$ $f ≠ 0$ on $ς(x)$. And if $f ≠ 0$ on $ς(x)$, then $f(x)^{-1} = Φ(f^{-1}) = Φ\(\frac{1}{f}\) = \(\frac{1}{f}\)(x)$.

		„5.“: $f(x) \in B$, so $ς_A(f(x)) \overset{\text{Lemma}}= ς_B(f(x)) = ς_B(Φ(f)) \overset{Φ \text{is isomorphism}}= ς_{©C(ς(x))} = \Rang f = f(ς(x))$.
		
		„6.“: Omitted.

% 27. 02. 2023 (I wasn't there) (From lecturer's notes)

		„7.“: TODO!!!
	\end{dukazin}
\end{veta}

\begin{veta}[Bent Fuglede (1950), Calvin R. Putnam (1951)]
	Let $A$ be complex $C^*$-algebra, $x \in A$ and $a, b \in A$ be complex such that $ax = xb$. Then $a^*x = xb^*$.

	\begin{dukazin}
		TODO!!! (Maybe omitted?)
	\end{dukazin}
\end{veta}

\section{Operators on Hilbert spaces}
\begin{definice}[Sesquilinear map, sesquilinear form]
	Let $X, Y$ be vector spaces over ®C. Map $S: X \times X \rightarrow Y$ is called sesquilinear, if it is linear in the first variable and conjugate-linear in the second one. If $Y = ®C$, $S$ is a sesquilinear form.
\end{definice}

\begin{tvrzeni}[Polarization identity]
	$X, Y$ vector spaces over ®C and $S: X \times X \rightarrow Y$ is a sesquilinear map. Then for all $x, y \in X$, it holds that
	$$ S(x, y) = \frac{1}{4}(S(x + y, x + y) - S(x - y, x - y) + iS(x + iy, x + iy) - iS(x - iy, x - iy)). $$

	\begin{dukazin}
		TODO!!!
	\end{dukazin}
\end{tvrzeni}

\begin{dusledek}
	$\{¦o\} ≠ H$ Hilbert space, $T, S \in ©L(H)$. Then $T = S$ iff $\forall x \in H: \<Tx, x\> = \<Sx, x\>$.

	\begin{dukazin}
		TODO!!!
	\end{dukazin}
\end{dusledek}

\begin{veta}
	$\{¦o\} ≠ H$ Hilbert space and $T \in ©L(H)$. Then
	
	\begin{itemize}
		\item $T$ is self-adjoint iff $\forall x \in H: \<Tx, x\> \in ®R$;
		\item $T$ is normal iff $\forall x \in H: \|Tx\| = \|T^*x\|$;
		\item $\forall x \in H: \<Tx, x\> ≥ 0$ iff $T$ is self-adjoint and $ς(T) \subseteq [0, ∞)$.
	\end{itemize}

	\begin{dukazin}
		TODO!!!
	\end{dukazin}
\end{veta}

\begin{definice}[Non-negative]
	$A$ $C^*$-algebra and $x \in A$. We say that $x$ is non-negative ($x ≥ 0$), if $x$ is self-adjoint and $ς(x) \subseteq [0, +∞)$.
\end{definice}

\begin{veta}
	$H$ Hilbert space and $T \in ©L(H)$ normal. Then\vspace{-1em}
	\begin{itemize}
		\item $\Ker T = \Ker T^*$ and $\Ker T = (\Rang T)^\perp$;
		\item $\Rang T$ is dense in $H$ iff $T$ is one-to-one;
		\item $λ \in ς_P(T)$ iff $\overline{λ} \in ς(T^*)$, eigenspace of $T$ associated with $λ$ is equal to eigenspace of $T^*$ associated with $\overline{λ}$;
		\item if $λ_1 ≠ λ_2$ are eigenvalues of $T$, then $\Ker(λ_1I - T) \perp \Ker(λ_2I - T)$.
	\end{itemize}

	\begin{dukazin}
		TODO!!!
	\end{dukazin}
\end{veta}

\begin{veta}[Hilbert–Schmidt]
	$H$ Hilbert space and $T \in ©K(H)$ nonzero normal. Then exists orthonormal basis $B$ of space $H$ consisting of eigenvectors of $T$. The set of vectors from $B$ associated with nonzero eigenvalues of $T$ is at most countable and we can arrange them to sequence $\{e_n\}_{n=1}^N$, $N \in ®N \cup \{∞\}$, then $\{e_n\}$ is orthonormal basis of $\overline{\Rang T}$ and for every $x \in H$:
	$$ Tx = \sum_{n=1}^N λ_n\<x, e_n\>e_n, $$
	where $λ_n$ is eigenvalue associated with the eigenvector $e_n$.

	\begin{dukazin}
		Omitted. „\verb|OM4/Funkcionalka.pdf|“
	\end{dukazin}
\end{veta}

\begin{veta}[Schmidt]
	$H$ Hilbert space and $T \in ©L(H)$ nonzero compact. Then exists $N \in ®N_0 \cup \{∞\}$, sequence of positive numbers $\{λ_n\}_{n=1}^N$ and orthonormal systems $\{u_n\}_{n=1}^N$ and $\{v_n\}_{n=1}^N$ such that for every $x \in H$:
	$$ Tx = \sum_{n=1}^N λ_n\<x, u_n\>v_n. $$

	\begin{dukazin}
		TODO!!!
	\end{dukazin}
\end{veta}

\begin{veta}
	$H$ Hilbert space and $P \in ©L(H)$ projection. Then following are equivalent: $P$ is orthogonal ($\Rang P \perp \Ker P$); $P ≥ 0$; $P$ is self-adjoint; $P$ is normal.

	Moreover, if $P, Q \in ©L(H)$ are orthogonal projections, then $\Rang(P) \perp \Rang(Q)$ iff $PQ = 0$.

	\begin{dukazin}
		TODO!!!
	\end{dukazin}
\end{veta}

\begin{definice}[Unitary operator]
	$H, K$ Hilbert spaces. Operator $T \in ©L(H, K)$ is called unitary, if $T^{-1} = T^*$, i.e., $T^*∘T = I_H$ and $T∘T^* = I_K$.
\end{definice}

\begin{tvrzeni}
	$H, K$ Hilbert spaces and $T \in ©L(H, K)$. Then $T$ is unitary $\Leftrightarrow$ $T$ is isometry onto $\implies$ $T$ is isometry $\Leftrightarrow$ $\<Tx, Ty\> = \<x, y\>$ for every $x, y \in H$. Moreover if $T$ is onto, then all propositions are equivalent.

	\begin{dukazin}[TODO!!!\vspace{-1em}]
	\end{dukazin}
\end{tvrzeni}

\begin{definice}[Partial isometry, initial subspace]
	$H$ Hilbert space. Operator $U \in ©L(H)$ is called partial isometry, if there is closed subspace $K \subset H$ (initial subspace of $U$) such that $U|_K$ is isometry and $U|_{K^\perp} ≡ 0$.
\end{definice}

\begin{veta}[Polararization decomposition]
	$H$ Hilbert space, $T \in ©L(H)$.

	Exists unique operators $P, U \in ©L(H)$ such that $P ≥ 0$, $U$ is partial isometry with initial subspace $\overline{\Rang P}$ and $T = UP$. Moreover $P = \sqrt{T^* T} = U^* T$.

	If $T$ is invertible, then exists unique $P, U \in ©L(H)$ such that $P ≥ 0$ is invertible, $U$ is unitary and $T = UP$.
\end{veta}

% 03. 04. 2023

\vspace{-2.5em}
\section{Borel measurable calculus}
\begin{lemma}[Lax–Milgram]
	$H$ Hilbert, $S: H \times H \rightarrow C$ sesquilinear, $\|S\| := \sup_{x, y \in S_H} |S(x, y)| < ∞$. Then $\exists! T \in ©L(H): \|T\| = \|S\| \land \<T x, y\> = S(x, y)$.

	\begin{dukazin}
		Fix $y \in H$. Then $H \ni x \mapsto S(x, y)$ is a point in $H^*$ $\implies$ $\exists! U(y) \in H: S(x, y) = \<x, U(y)\>$, $x \in H$. Then $U \in ©L(H)$, $\|U\| = \|S\|$.

		„Linearity“: Easy: $\forall y, z \in H, α \in ®K \implies$
		$$ \forall x \in H: \<x, U(α y + z)\> = S(x, α y + z) = \overline{α} S(x, y) + S(x, z) = \overline{α}\<x, U y\> + \<x, U z\>. $$

		„$\|U\| ≤ \|S\|$“:
		$$ \forall y \in H: \|Uy\|^2 = \<Uy, Uy\> = S(Uy, y) ≤ \|S\|·\|Uy\|·\|y\| \implies \|U y\| ≤ \|S\|·\|y\|. $$

		„$\|U\| ≥ \|S\|$“:\vspace{-0.4em}
		$$ \forall x, y \in S_H: \|S(x, y)| = |\<x, Uy\>| ≤ \|x\|·\|U\|·\|y\| = \|U\|.\vspace{-0.5em} $$

		„Uniqueness“: Bounded operator is given by values of $\<T x, y\>$.
	\end{dukazin}
\end{lemma}

\begin{definice}
	$H$ Hilbert, $T \in ©L(H)$ normal, $Φ: ©C(ς(T)) \rightarrow ©L(H)$ continuous ?.

	\begin{itemize}
		\item $\forall x, y \in H$: $μ_{x, y} \in M(ς(T))$ is the unique measure satisfying
			$$ \int_{ς(T)} f dμ_{x, y} = \<Φ(f) x, y\>, \qquad f \in ©C(ς(T)). $$
		\item $\forall f \in Bor_0(ς(T))$ (bounded, Borel) we get $Φ(f) \in ©L(H)$ be the unique operator such that
			$$ \<Φ(f) x, y\> = \int_{ς(T)} f dμ_{x, y}, \qquad x, y \in H $$
	\end{itemize}

	\begin{dukazin}
		„1.“: $f \mapsto \<Φ(f) x, y\>$ is linear and $|\<Φ(f) x, y\>| ≤ \|Φ(f)\|·\|x\|·\|y\|$. So $f \mapsto \<Φ(f) x, y\> \in ©C(ς(T))^* = M(ς(T))$ $\implies$ $μ$ exists by Riesz representation theorem.

		„2.“: $\forall x, x_2, y \in H\ \forall α \in ®K\ \forall f \in ©C(ς(T)):$
		$$ \<Φ(f)(α x_1 + x_2), y\> = α\<Φ(f)x_1, y\> + \<Φ(f) x_2, y\> = α μ_{x, y}(f) + μ_{x_2, y}(f). $$
		Thus $· \mapsto μ_{·, y}$ is linear (for each $y$). Analogously $· \mapsto μ_{x, ·}$ is conjugate-linear.

		Thus, $(x, y) \mapsto μ_{x, y}(f) \in ®C$ is sesquilinear form.
		$$ \forall x, y \in S_H: |μ_{x, y}(f)| ≤ \int |f| d|μ_{x, y}| ≤ \|f\|_∞·\|x\|·\|y\| = \|f\|_∞. $$
		And from Lax–Milgram:
		$$ \exists! Φ(f) \in ©L(H): \<Φ(f)x, y\> = μ_{x, y}. $$
		Moreover $\|Φ(f)\| ≤ \|f\|_∞$.
	\end{dukazin}
\end{definice}

\begin{poznamka}
	$H$ Hilbert, $T \in ©L(H)$ normal:

	\begin{itemize}
		\item Mapping $H \times H \ni (x, y) \mapsto μ_{x, y}$ is sesquilinear, so
			$$ μ_{x, y} = \frac{1}{4}\(μ_{x + y, x + y} - μ_{x - y, x - y} + i μ_{x + i y, x + iy} - i μ_{x - i y, x - i y}\). $$
		\item $\forall x \in H: μ_{x, x} ≥ 0$. (Proof: „$f ≥ 0 \implies μ_{x, x}(f) ≥ 0, f \in ©C(ς(T))$“: $f ≥ 0 \implies Φ(f) ≥ 0$ ($ς(Φ(f)) = f(ς(T)) \subseteq [0, ∞) \implies Φ(f) ≥ 0$.) So $\int_{ς(T)} f dμ_{x, x} = Φ(f)x, x ≥ 0$.)
		\item $Bor_b(ς(T)) \subseteq l_∞(ς(x)) \mapsto ©L(H)$ is $C^*$-subalgebra.
		\item The mapping $Φ: Bor_b(ς(x)) \rightarrow ©L(H)$ from previous definition, is extension of continuous calculus from theorem above.
	\end{itemize}
\end{poznamka}

\begin{veta}
	Let $P$ be a metric space, $Φ$ be the smallest system of functions such that $©C_b(P) \subset Φ$ and $Φ$ is closed under point-wise bounded convergent sequences. Then $Φ = Bor_b(P)$.

	\begin{dukazin}[Sketch]
		Suffices: „$\forall A \subset P$ Borel: $χ_A \in Φ$.“
		$$ ©F := \{A \subset P \text{ Borel } | χ_A \in Φ\} $$
		is $ς$-algebra containing closed sets $\implies$ $©F = Bor(P)$.
	\end{dukazin}
\end{veta}

\begin{definice}
	Let $X, Y$ be normed linear spaces. On $©L(X, Y)$ we define the following two Hausdorff locally convex topologies:

	\begin{itemize}
		\item $τ_{SOT}$ generated by pseudonorms $\{P_x(T) = \|T_x\| | x \in X\}$ (so, $T_i \overset{\text{SOT}}\rightarrow T \Leftrightarrow \forall x \in X: T_i x \overset{\|·\|}\rightarrow T x$);
		\item $τ_{WOT}$ generated by pseudonorms $\{P_{x, y^*}(T) = y^*(T x) | x \in X \land y^* \in Y^*\}$ (so, $T_i \overset{\text{WOT}}\rightarrow T \Leftrightarrow \forall x \in X: T_i x \overset{w} T x$) (in $X = Y = H$ Hilbert: $\Leftrightarrow \forall x, y \in H: \<T_i x, y\> \rightarrow \<T x, y\>$).
	\end{itemize}
\end{definice}

\begin{poznamka}
	$$ T_i \overset{\|·\|}\rightarrow T \implies T_i \overset{\text{SOT}}\rightarrow T \implies T_i \overset{\text{WOT}}\rightarrow T. $$

	\begin{prikladyin}
		$R_n x := (0, 0, …, 0, x_{n+1}, x_{n+2}, …)$, $x \in l_2$. Then $R_n \in ©L(l_2)$, $n \in ®N$, and $R_n \overset{\|·\|}{\not\rightarrow} 0$, because $\|R_n(e_{n+1})\| = 1$, $n \in ®N$. But $R_n \overset{\text{SOT}}\rightarrow 0$, because $\forall x \in l_2: \|R_n x\|_2^2 = \sum_{i=n+1}^∞ |x_i|^2 \rightarrow 0$.

		$S_n x := (0, 0, …, 0, x_1, x_2, …)$, $x \in l_2$. Then $S_n \in ©L(l_2)$ is isometry $\forall n \in ®N$. But $S_n \overset{\text{SOT}}{\not\rightarrow} 0$, because $\|S_n(e_1)\| = 1 \not\rightarrow 0$. But $S_n \overset{\text{WOT}}\rightarrow 0$, because $\forall x, y \in l_2$:
		$$ |\<S_n x, y\>| = |\sum_{i=1}^∞ x_i y_{n+i}| ≤ \|x\|_2 \sqrt{\sum_{i=n+1}^∞ |y_i|^2} \rightarrow 0. $$
	\end{prikladyin}
\end{poznamka}

\begin{veta}
	$H$ Hilbert, $T \in ©L(H)$ normal, $f \in Bor_b(ς(T))$, $Φ: Bor_b(ς(T)) \rightarrow ©L(H)$ as in definition above. Then

	$Φ$ is continuous $*$-homomorphism and $\|Φ\| = 1$;

	\begin{dukazin}
		$Φ$ is linear (easy from definition). $\|Φ\| ≤ 1$ follows from the second point of the previous theorem, and $\|Φ(1)\| = \|\id\| = 1$, so $\|Φ\| = 1$.

		„$Φ$ is multiplicative“: Step 1: „$©F \!\coloneq \{g \in\! Bor_b(ς(T)) | \forall f \in ©C(ς(t))\mathpunct{:} Φ(gf) = Φ(g)\!·\!Φ(f)\}$, then $©F = Bor_b(ς(T))$“: $©F \subseteq ©C(ς(T))$ follows from continuous calculus, „$©F$ closed under point-wise limits of bounded sequences“: Let $©F \ni g_n \rightarrow g$ and $f \in ©C(ς(T))$. Then $g_n f \rightarrow gf$ point-wise. So, for $x, y \in H$:
		$$ \<Φ(g, f)x, y\> = \int_{ς(T)} g f dμ_{x, y} = \lim \int g_n f dμ_{x, y} = \lim \<Φ(g_n) x, y\> = $$
		$$ = \lim \<Φ(g_n)(Φ(f)x), y\> = \lim \int g_n dμ_{Φ(f)x, y} = \<Φ(g)(Φ(f)x), y\>. $$
		$\implies ©F = Bor_b(ς(T))$.

% 04. 04. 2023

		Step 2: „$©H := \{g \in Bor_b(ς(T)) | \forall f \in Bor_b(ς(t)): Φ(gf) = Φ(g)·Φ(f)\}$, then $©H = Bor_b(ς(T))$“: „©H is closed under point-wise limits of bounded sequences“: $©H \ni f_n \overset{τ_p} f$, $f_n$ bounded, then $\forall x, y \in H$ $\forall g \in Bor_b(ς(T))$:
		$$ \<Φ(g f)x, y\> \overset{\text{Lebesgue}}= \lim_n \<φ(g f_n)x, y\> = \lim_n \<Φ(g)Φ(f_n)x, y\> = \lim_n \<Φ(f_n) x, Φ(g)^* y\> = $$
		$$ = \lim_n \int f_n dμ_{x, Φ(g)^* y} \overset{\text{Lebesgue}}= \int f dμ_{x, Φ(g)^*y} = \<Φ(f)x, Φ(g)^* y\> = \<Φ(g) Φ(f) x, y\>. $$
		Thus $Φ(g f) = Φ(g) Φ(f)$.

		„$Φ$ preserves $*$“: $©F := \{g \in Bor_b(ς(T)) | Φ(g)^* = Φ(\overline{g})\}$. Then $©F \subseteq ©C(ς)$ by continuous calculus and $©F$ is "closed under taking limits" analogously as above. $\implies$ $©F = Bor_b(ς(T))$.
	\end{dukazin}

	\pagebreak

	$(f_n) \in Bor_b(ς(T))^{®N}$ bounded and $f_n \overset{τ_p} f$, then $Φ(f_n) \overset{\text{SOT}}\rightarrow Φ(f)$.

	\begin{dukazin}
		Step 1: „$Φ(f_n) \overset{\text{WOT}} Φ(f)$“:
		$$ \forall x, y \in H: \<Φ(f_n) x, y\> \overset{\text{Lebesgue}}\rightarrow \<Φ(f)x, uy\>. $$

		Step 2: „$\|Φ(f_n) x\| \rightarrow \|Φ(f) x\|, x \in H$“:
		$$ \|Φ(f_n) x\|^2 = \<Φ(\overline{f_n}) Φ(f_n) x, x\> = \<Φ(\overline{f_n} f_n) x, x\> \overset{\text{Lebesgue}}\rightarrow \<Φ(\overline{f} f) x, x\> = \|Φ(f) x\|^2. $$

		Step 3: From steps 1 and 2:
		$$ \|Φ(f_n) x - Φ(f)x\|^2 \overset{\text{Cos. věta}}= \|Φ(f_n) x\|^2 + \|Φ(f) x\|^2 - 2\Re\<Φ(f_n) x, Φ(f) x\> \rightarrow 0. $$
	\end{dukazin}

	If $K \subset ®C$ is compact, $K \supseteq ς(T)$, $ψ: Bor_b(K) \rightarrow ©L(H)$ is continuous $*$-homomorphism, $ψ(1) = \id$, $ψ(\id) = T$ and $f_n \overset{τ_p}\rightarrow f \implies ψ(f_n) \overset{\text{WOT}}\rightarrow ψ(f)$. Then $ψ(g) = Φ(g |_{ς(T)})$, $g \in Bor_b(K)$.

	\begin{dukazin}
		Skipped. Using characterization of $Bor_b$.
	\end{dukazin}

	$Φ(f)$ is normal, $Φ(f)$ is self-adjoint $\Leftrightarrow$ $f$ is real.

	\begin{dukazin}
		Skipped. Easy from first part of theorem.
	\end{dukazin}

	$ς(Φ(f)) \subseteq \overline{f(ς(T))}$.

	$g \in Bor_b(\overline{\Rang f}) \implies (g∘f)(T) = g(f(T))$.

	$\forall S \in ©L(H), ST = TS: Sf(T) = f(T)S$.
\end{veta}

% 11. 04. 2023

\pagebreak

\section{Spectral decomposition of normal operator}
\begin{definice}[Spectral measure]
	$H$ Hilbert space, $(X, ©A)$ measurable space. Then $E: ©A \mapsto ©L(H)$ is spectral measure for $(X, ©A, H)$ if

	\begin{itemize}
		\item $\forall A \in ©A: E(A)$ is orthogonal projection;
		\item $E(X) = \id$, $E(\O) = 0$;
		\item if $\{A_n, n \in ®N\} \subset ©A$ is point-wise disjoint, then
			$$ E(\bigcup A_n) x = \sum_{n=1}^∞ E(A_n) x, x \in H. $$
	\end{itemize}
\end{definice}

\begin{tvrzeni}[Properties of spectral measure]
	$H$ Hilbert, $(X, ©A)$ measurable space, $E$ is spectral measure for $(X, ©A, H)$. Then
	
	\begin{enumerate}
		\item $\forall A, B \in ©A, A \subset B: E(A) ≤ E(B)$ (that's $E(B) - E(A) ≥ 0$);
		\item $\forall A, B \in ©A: E(A \cap B) = E(A)·E(B)$, in particular, if $A \cap B = \O$, then $E(A)·E(B) = \O$.
		\item $\forall x, y \in H: ©A \ni A \mapsto \<E(A) x, y\>$ is a complex measure (denoted by $E_{x, y}$), with total variation $\|E_{x, y}\| ≤ \|x\|·\|y\|$.
		\item $(x, y) \mapsto E_{x, y}$ is sesquilinear mapping.
		\item $\forall x, y \in H\ \forall A \in ©A$:
			$$ |E_{x, y}(A)| ≤ \frac{1}{2}\(E_{x, x}(A) + E_{y, y}(A)\). $$
		\item $\forall x, y \in H$:
			$$ E_{x + y, x + y} ≤ 2\(E_{x, x} + E_{y, y}\). $$
	\end{enumerate}

	\begin{dukazin}
		„1.“: $E(A) + E(B \setminus A) = E(B)$, so $E(B) - E(A) = E(B \setminus A) ≥ 0$.

		„2.“: „Step 1: $A \cap B = \O$“:
		$$ \id = E(X) = E(A) + E(A^c) ≥ E(A) + E(B), $$
		so $E(B) ≤ \id - E(A)$, which is orthogonal projection onto $(\Rang E(A))^\perp$. Thus („$P, Q \in ©L(A)$ orthogonal projections, $Q - P ≥ 0$, then $\Rang P \subset (\Rang Q)^\perp$“: $\|Px\|^2 =$
		$$ = \|QPx\|^2 + \|(\id - Q) Px\|^2 = \<QPx, Px\> + \|(\id - Q) Px\|^2 ≥ \underbrace{\<PPx, Px\>}_{\|Px\|^2} + \|(\id - Q) Px\|^2, $$
		thus, $(\id - Q) Px = 0$, so $\Rang P \subseteq \Ker(\id - Q) = \Rang Q$.) $\Rang E(B) \subseteq (\Rang E(A))^{\perp}$. Thus $\Rang E(A) \perp \Rang E(B)$, so $E(A)·E(B) = 0$.

		„Step 2: In general“:
		$$ E(A) = E(A \cap B) + E(A \setminus B), \qquad E(B) = E(A \cap B) + E(B \setminus) \implies $$
		$$ E(A)·E(B) = (E(A \setminus B) + E(A \setminus B))·(E(A \cap B) + E(B \setminus A)) = E^2(A \cap B) + 3·0 = E(A \cap B). $$

		„3.“: „$E_{x, y}$ is countably additive“ is easy. By this it is a complex measure. „Calculation of $\|E_{x, y}\|$“: Fix $A_1, …, A_n \in ©A$ disjoint such that $\bigcup_{i=1}^n A_i = X$. For $i \in [n]$ pick $\alpha_i \in S_{®C}: \alpha_i \<E(A_i) x, y\> = |\<E(A_i) x, y\>|$. Then
		$$ \sum_{i=1}^n |E_{x, y} (A_i)| = \sum_{i=1}^n \alpha_i \<E(A_i) x, y\> \overset{\text{Cauchy–Schwartz}}≤ \|\sum_{i=1}^n \alpha_i E(A_i) x\|·\|y\|. $$
		$$ \left\|\sum_{i=1}^n E(A_i)(\alpha_i x) \right\|^2 \overset{\text{Pythagoras}} = \sum_{i=1}^n \|E(A_i)(\alpha_i x)\| = \sum_{i=1}^n \|E(A_i)(x)\| = \sum_{i=1}^n \<E(A_i)x, x\> = $$
		$$ = \<E\(\bigcup A_i\)x, x\> = \<x, x\> = \|x\|^2. $$

		„4.“: Easy, using definition. „5.“:
		$$ |E_{x, y}(A)| = |\<E(A) x, y\>| = |\<E(A) x, E(A) y\>| \overset{\text{Cauchy-Schwartz}}≤ \|E(A) x\|·\|E(A)y\| = $$
		$$ = \sqrt{E_{x, x}(A)}·\sqrt{E_{y, y}(A)} \overset{\text{A–G}}≤ \frac{1}{2}\(E_{x, x}(A) + E_{y, y}(A)\). $$

		„6.“:
		$$ E_{x + y, x + y}(A) = E_{x, x}(A) + E_{y, x}(A) + E_{x, y}(A) + E_{y, y}(A) ≤ E_{x, x}(A) + 2\Re E_{y, x}(A) + E_{y, y}(A) ≤ $$
		$$ ≤ E_{x, x}(A) + 2·\frac{1}{2}\(E_{x, x}(A) + E_{y, y}(A)\) + E_{y, y}(A) = 2\(E_{x, x}(A) + E_{y, y}(A)\). $$
	\end{dukazin}
\end{tvrzeni}

\begin{poznamka}
	From 4. we get $E_{x, y}(A) = \frac{1}{4} \sum_{k=0}^3 i^k \<E(A)(x + i^k y), x + i k y\>$. Thus 3. is equivalent to $\forall x \in H: E_{x, x} ≥ 0$ is measure.
\end{poznamka}

\begin{definice}[Integral]
	$H$ Hilbert space, $(X, ©A)$ measurable space, $E$ spectral measure for $(X, ©A, H)$. $f: X \rightarrow ®C$ bounded ©A-measurable function. Then integral of $f$ with respect to $E$ is the operator $T \in ©L(H)$ such that
	$$ \<Tx, y\> = \int_X f dE_{x, y}, \qquad x, y \in H. $$

	Notation: Then $\int f dE := T$.

	\begin{poznamkain}
		It always exists due to Lax–Milgram: $(x, y) \mapsto \int f dE_{x, y}$ is sesquilinear and $\left|\int f d E_{x, y}\right| ≤ \|f\|_∞·\|E_{x, y}\| ≤ \|f\|_∞·\|x\|·\|y\|$. So $T$ exists and $\|T\| ≤ \|f\|_∞$.
	\end{poznamkain}
\end{definice}

% 17. 04. 2023

\begin{tvrzeni}
	$H$ Hilbert, $(X, ©A)$ measurable space, $E$ spectral measure for $(X, ©A, H)$, $f: X \rightarrow ®C$ bounded ©A-measurable. Then for $ε > 0$ pick $A_1, …, A_m \in ©A$ disjoint partition of $X$ such that $\diam f(A_i) < \epsilon$ and for $x_i \in A_i$, $i \in [n]$
	$$ \left\|\int f dE - \sum_{i=1}^n f(x_i) E(A_i)\right\| < ε. $$

	\begin{dukazin}
		Denote $T = \int f dE$. For $x, y \in H: |\<Tx, y\> - \<\sum f(x_i) E(A_i) x, y\>| =$
		$$ = \left|\sum_{i=1}^n \int_{A_i} \(f(t) - f(x_i)\) d E_{x, y}\right| ≤ \sum_{i=1}^n \int_{A_i} |f(t) - f(x_i)| d|E_{x, y}| ≤ ε\int_X d|E_{x, y}| ≤ ε·\|x\|·\|y\|. $$
		This finishes the proof. ($|\< S x, y\>| ≤ ε·\|x\|·\|y\| \implies \|S\| < ε$.)
	\end{dukazin}
\end{tvrzeni}

\begin{definice}[Notation]
	$(X, ©A)$ measurable space, $B(X, ©A) \subset l_∞(X)$ $c^*$ algebra consisting of bounded $f: X \rightarrow ©C$ ©A-measurable functions.
\end{definice}

\begin{tvrzeni}
	$H$ Hilbert, $(X, ©A)$ measurable space, $E$ spectral measure for $(X, ©A, H)$. Consider $ρ: B(X, ©A) \rightarrow ©L(H)$, $ρ(f) = \int f dE$. Then

	\begin{enumerate}
		\item $ρ$ is continuous $*$-homomorphism, $\|ρ\| = 1$, $ρ(1) = \id$.
		\item $\forall f \in B(X, ©A): ρ(f)$ is normal. $f$ is real $\implies$ $ρ(f)$ is self-adjoint, $f ≥ 0$ $\implies$ $ρ(f) ≥ 0$.
		\item $f_n \in B(X, ©A)^n$ bounded, $f_n \rightarrow f$ point-wise $\implies$ $ρ(f_n) \overset{\text{WOT}}\rightarrow ρ(f)$.
		\item $\forall f \in B(X, ©A)$ $\forall x \in H$: $\|ρ(f) x\| = \sqrt{\int |f|^2 d E_{x, x}}$.i
		\item $\int f dE$ is the unique $T \in ©L(H)$: $\<T x, y\> = \int f dE_{x, y}$, $x, y \in H$.
	\end{enumerate}

	\begin{dukazin}
		1.) „$ρ$ is linear“: easy. „$\|ρ\| ≤ 1$“: easy as well. „$ρ$ preserves $*$“:
		$$ \forall x \in H: \<ρ(f)^* x, x\> = \<x, ρ(f) x\> = \overline{\<ρ(f) x, x\>} = \overline{\int f dE_{x, x}} = \int \overline{f} dE_{x, x} = \<ρ(\overline{f}) x, x\>. $$
		„$ρ$ is multiplicative“: For $f, g \in B(X, ©A)$, $ε > 0$. Find disjoint partition $A_1, …, A_n \in ©A$ of $X$ such that for $ω \in \{f, g, f·g\}$ we have $\diam ω(A_i) < ε$ for $i \in [n]$. Pick $x_1 \in A_1, x_2 \in A_2, …, x_n \in A_n$. Thus using previous proposition we have
		$$ \left\|\int f g dE - \(\int f dE\)\(\int g dE\)\right\| ≤ ε + $$
		$$ + \left\|\sum_{i=1}^n (f·g)(x_i) E(A_i) - \(\sum f(x_i) E(A_i)\)\(\sum g(x_i) E(A_i)\)\right\| + $$
		$$ + \left\|\(\sum f(x_i) E(A_i)\)\(\sum g(x_i) E(A_i)\) - \(\int f dE\)\(\int g dE\)\right\| ≤ ε + 0 + $$
		$$ + \left\|\(\sum f(x_i) E(A_i)\)\(\sum g(x_i) E(A_i) - \int g dE\)\right\| + \|(\sum f(x_i) E(A_i)) TODO < $$
		$$ < \|f\|_∞·ε + ε·\|g\|_∞.\vspace{-0.4em} $$

		„$\|ρ\| = 1$“: TODO!!!

		„$ρ(1) = \id$“: $\forall x \in H: \<ρ(1) x, x\> = \int_X 1 dE_{x, x} = \<E(X)x, x\> = \<x, x\> = \<\id x, x\>$.

		2.)\vspace{-1.2em}
		$$ ρ(f)^* ρ(f) = ρ(\overline{f}f) = ρ(f \overline{f}) = ρ(f)ρ(f)^* \implies ρ(f) \text{ is normal}. $$
		$$ f \text{ is real } \implies f = \overline{f} \implies ρ(f) = ρ(f)^*. $$
		$$ f ≥ 0 \implies \forall x \in H: \<ρ(f) x, x\> = \int f dE_{x, x} ≥ 0 \implies ρ(f) ≥ 0. $$

		3.)\vspace{-1.7em}
		$$ \forall x, y \in H: \<ρ(f_n) x, y\> = \int f_n d E_{x, y} \overset{\text{Lebesgue}}\rightarrow \int f dE_{x, y} = \<ρ(f) x, y\>. $$

		4.)\vspace{-2.7em}
		$$ \|ρ(f) x\|^2 = \<ρ(f)x, ρ(f)x\> = \<ρ(\overline{f}f) x, x\> = \int \overline{f}f dE_{x, x} = \int |f|^2 dE_{x, x}.\vspace{-0.8em} $$
	\end{dukazin}
\end{tvrzeni}

\begin{dusledek}[Spectral decomposition of normal operator]
	$H$ Hilbert, $T \in ©L(H)$ normal $\implies$ $\exists!$ spectral measure $E$ for $(ς(T), Bor(ς(T)), H)$: $T = \int \id dE$. Moreover $E(A) = Φ(χ_A)$ for any $A \in Bor(ς(T))$, where $Φ: Bor_b(ς(T)) \rightarrow ©L(H)$ is borel calculus from definition above.

	\begin{dukazin}
		Whenever $E$ is spectral measure for $(ς(T), Bor(ς(T)), H)$ satisfying $T = \int \id dE$, then $\int f dE = Φ(f)$, $f \in ©B(ς(T), Bor(ς(T)))$. This proves uniqueness.

		„Existence“: Put $E(A) := Φ(A)$, $A \subset ς(T)$ borel. Then $E$ is spectral measure: $E(A)$ is orthogonal projection ($χ_A^2 = χ_A$, $χ_A$ is real), $E(ς(T)) = \id$, $E(\O) = 0$ ($χ_{ς(T)} = 1$ and $Φ(1) = \id$, $χ_{\O} = 0$), $A_i \in Borel(ς(T))$ disjoint, $x \in H$, then
		$$ \left\|E\(\bigcup A_n\)x - \sum E(A_i) x\right\| = \<E\(\bigcup A_i\) x, E\(\bigcup A_i\) x\> = \<E\(\bigcup A_i\) x, x\> = $$
		$$ = \int χ_{\bigcup A_i} dμ_{x, x} = \sum_{N+1}^∞ μ_{x, x}(A_i) \rightarrow 0. $$

		„$T = \int \id dE$“: $E_{x, y} = μ_{x, y}$ ($E_{x, y}(A) = \<E(A) x, y\> = \int χ_A dμ_{x, y} = μ_{x, y}(A)$). Thus
		$$ \<\int \id dE x, y\> = \int \id dE_{x, y} = \int \id dμ_{x, y} = \<Φ(\id) x, y\> = \<T x, y\>. $$
	\end{dukazin}
\end{dusledek}

\section{Unbounded operators}
\begin{definice}
	$X, Y$ Banach spaces. Operator from $X$ to $Y$ is a linear mapping defined on a linear space $D(T) \subset X$ with values in $R(T) \subset Y$. If $X = Y$, we say $T$ is operator on $X$. Then graph of $T$ is $G(T) = \{(x, Tx) | x \in D(T)\} \subseteq X \times Y$.

	We say that $T$ is densely defined $≡$ $\overline{D(T)} = X$. We say that $T$ is closed $≡$ $G(T) \subset X \times Y$ is closed.
\end{definice}

% 18. 04. 2023

\begin{definice}[Notations]
	$X, Y$ Banach spaces. If $T, S$ is operator from $X$ to $Y$, then $S + T$ is operator from $X$ to $Y$ defined as $(S + T)(x) = Sx + T(x)$ for $x \in D(S + T) = D(S) \cap D(T)$.

	If $T$ is operator from $X$ to $Y$ and $S$ is operator from $Y$ to a Banach space $Z$, then $ST$ is operator with $D(ST) = \{x \in D(T) | Tx \in D(S)\}$ defined as $(ST)x = S(Tx)$ for $x \in D(ST)$.

	Operator $S$ from $X$ to $Y$ is extension of $T$, if $G(S) \supset G(T)$ (and we write $T \subset S$).
\end{definice}

\begin{priklady}
	$D(T) = c_{00} \subset l_2 = X$, $Tx = \(\sum_{n=1}^∞ x_n, 0, 0, 0, 0, …\)$. Then $T$ is densely defined, butt it doesn't have closed extension.

	\begin{dukazin}
		Consider $x^n = \(\frac{1}{2^n}, …, \frac{1}{2^n}, 0, …\)$ then $(x_n, T x_n) \rightarrow (¦o, e_1)$, so if there is extension, then $(¦o, e_1) \in G(S)$, but $S ¦o = ¦o$, because of linearity.
	\end{dukazin}
\end{priklady}

\begin{poznamka}
	It is easy to check:
	$$ (S + T) + V = S + (T + V), $$
	$$ (ST)V = S(TV), $$
	$$ (S + T)V = SV + TV. $$
\end{poznamka}

\begin{upozorneni}
	$$ V (S + T) \supseteq VS + VT. $$
\end{upozorneni}

\begin{lemma}
	$X$, $Y$ Banach and $L \subseteq X \times Y$. Then $\exists$ operator $T$ from $X$ to $Y$ such that $L = G(T)$ $\Leftrightarrow$ $L$ is a subspace and $\{(x, y) \in L | x = 0\} = \{(0, 0)\}$.

	\begin{dukazin}
		„$\implies$“: Easy.

		„$\impliedby$“: Put $D(T) = \{x \in X | \exists y \in Y: (x, y) \in L\}$. Then $\forall x \in D(T)$ $\exists! y \in Y$: $(x, y) \in L$. ($(x, y_1), (x, y_2) \in L \implies (0, y_1 - y_2) \in L$.) So, we put $Tx := y$, where $y \in L$ is such that $(x, y) \in L$. Then $T$ is linear and $G(T) = L$
	\end{dukazin}
\end{lemma}

\begin{tvrzeni}
	$X$, $Y$ Banach spaces, $T$ operator from $X$ to $Y$.

	\begin{itemize}
		\item $D(T) = X$ $\land$ $T$ is closed $\implies$ $T \in ©L(X, Y)$.
		\item Equivalence:
			\begin{enumerate}
				\item $T$ has closed extension;
				\item $(x_n, T x_n) \rightarrow (0, y)$ in $D(T) \times Y$ $\implies$ $y = 0$;
				\item $\overline{G(T)} \subset X \times Y$ is graph of an operator from $X$ to $Y$.
			\end{enumerate}
		\item $T$ is one-to-one and closed $\implies$ $T^{-1}$ is closed.
	\end{itemize}

	\begin{dukazin}
		First point follows immediately from closed graph theorem.

		„$1.) \implies 2.)$“: Let $S \supset T$ be closed. If $(x_n, T x_n) \rightarrow (¦o, y)$, then $(¦o, y) \in G(S)$, so $¦o = S ¦o = y$.

		„$2.) \implies 3.)$“ We will show, using the previous lemma, that $G(T)$ is graph of an operator: $\overline{G(T)}$ is linear, because $G(T)$ is linear. If $(¦o, y) \in \overline{G(T)}$, then $\exists (x_n) \in D(T)^{®N}: (x_n, T x_n) \rightarrow (¦o, y)$, so $y = ¦o$ from $2).$.

		„$3.) \implies 1.)$“: Clear.

		Third point $Φ: X \times Y \rightarrow Y \times X$ defined as $(x, y) \mapsto (y, x)$ is homeomorphism, so, $G(T)$ is closed $\Leftrightarrow$ $Φ(G(T)) = G(T^{-1})$ is closed.
	\end{dukazin}
\end{tvrzeni}

\begin{definice}[Closure of operator]
	$X, Y$ Banach spaces, $T$ operator from $X$ to $Y$, $T$ has closed extension. Then $\overline{T}$ is operator satisfying $\overline{T} \supset T$ and $G(\overline{T}) = \overline{G(T)}$.
\end{definice}

\begin{tvrzeni}
	$X, Y, Z$ Banach spaces, $T$ operator from $X$ to $Y$, which is closed.

	\begin{itemize}
		\item If $S \in ©L(X, Y)$, then $S + T$ is closed and $D(S + T) = D(T)$.
		\item If $S \in ©L(Y, Z)$, then $D(ST) = D(T)$ and if $S$ is isomorphism into, then $ST$ is closed.
		\item If $S = ©L(Z, X)$, then $TS$ is closed.
	\end{itemize}

	\begin{dukazin}
		Of course $D(S + T) = D(S) \cap D(T) = D(T)$. If $(x_n, (S + T) x_n) \rightarrow (x, y)$, then $T x_n = (S + T) x_n - S x_n \rightarrow y - Sx$. So $(x_n, T x_n) \rightarrow (x, y - S x) \in G(T)$, so $Tx = y - Sx \implies y = (T + S)x$.

		$$ D(ST) = \{x \in D(T) | Tx \in D(S) = Y\} = D(T). $$
		Suppose $S$ is isomorphism into, $(x_n, S Tx_n) \rightarrow (x, z)$, then $Tx_n = S^{-1}STx_n \rightarrow S^{-1}z$. So $(x_n, T x_n) \rightarrow (x, S^{-1} z) \in G(T)$, so $T x = S^{-1} z$, then $S T x = z$.

		$(z_n, T S z_n) \rightarrow (x, y)$, then $S z_n \rightarrow S x$, so $(S z_n, T S z_n) \rightarrow (S x, y) \in G(T)$, thus $T S x = y$.
	\end{dukazin}
\end{tvrzeni}

% 24. 04. 2023

TODO example?

\begin{tvrzeni}
	$X, Y$ Banach, $T$ one-to-one closed operator from $X$ to $Y$. Then following statements are equivalent:\vspace{-0.8em}
	$$ \Rang T = Y \land T^{-1} \in ©L(Y, X); \quad \Rang T = Y; \quad \Rang T \text{ is dense and } T^{-1} \in ©L(\Rang T, X).\vspace{-1em} $$

	\begin{dukazin}
		„1) $\implies$ 2)“: trivial. „2) $\implies$ 3)“: $\Rang T$ is dense and $T^{-1}(\Rang T, X)$ due to previous proposition (by which $T^{-1}$ is closed).

		„3) $\implies$ 1)“: Let $S \in ©L(Y, X)$ be continuous extension of $T^{-1}$. Pick $y \in Y$. Since $\overline{\Rang T} = Y$, there is $(x_n) \in X^{®N}$ such that $T x_n \rightarrow y$. Then $S T x_n = T^{-1} T x_n = x_n \rightarrow S y$. So $(x_n, T x_n) \rightarrow (S y, y) \in G(T)$, thus $T S y = y \in \Rang T$.
	\end{dukazin}
\end{tvrzeni}

\begin{definice}[Resolvent set, resolvent function, spectrum of operator]
	$X$ Banach, $T$ linear operator on $X$. Then resolvent set is\vspace{-0.5em}
	$$ ρ(T) := \{λ \in ®K | λI - T \text{ has inverse which belongs to } ©L(X)\};\vspace{-0.5em} $$
	resolvent function is $R_T(λ) := (λI - T)^{-1}$, $λ \in ρ(T)$; spectrum of $T$ is $ς(T) := ®K \setminus ρ(T)$.
\end{definice}

\begin{veta}
	$X$ Banach, $T$ linear operator on $X$. Then $ρ(T)$ is open, $ρ(T)$ is closed and $R_T$ has derivative at each point of $ρ(T)$. (So, if $X$ is complex, then $R_t$ is holomorphic on $ρ(T)$).

	\begin{dukazin}
		„$ρ(T)$ is open“: Pick $λ \in ρ(T)$ and $h \in ®K$ small ($|·|$) enough: $|h| < \frac{1}{\|(λI - T)^{-1}\|}$. Then $h(λ I - T)^{-1} =: S \in ©L(X)$, $\|S\| < 1$. Thus, $(I + S)^{-1}$ exists, so $(λ + h)I - T = (I + S)·(λI - T)$ has inverse $(λI - T)^{-1} \circ (I + S)^{-1} \in ©L(X)$. $(λI - T)^{-1} ∘ (I + S)^{-1} \in ©L(X)$. So $U(λ, \frac{1}{\|(λI - T^{-1})\|}) \subset ρ(T)$.

		„$R_T$ has derivative at each $λ \in ρ(T)$“: $R_T'(λ) = -R_T(λ)^2$:
		$$ \forall h \text{ small enough}: \left\|\frac{R_t(λ + h) - R_t(λ)}{h} + R_T(λ)^2\right\| = $$
		$$ \frac{1}{h} \|R_T(λ + h) - R_T(λ) + R_T(λ) h R_T(λ)\| = \frac{\|R_T(λ)\|}{\|h\|}·\|(I + S)^{-1} - I + h R_T(λ)\| = $$
		$$ \((I + S)^{-1} = \sum_{n=0}^∞ (-S)^n = I - S + \sum_{n=2}^∞(-S)^n = I - h R_T(λ) + \sum_{n=2}^∞ (-h R_T(λ))^n\) $$
		$$ = \frac{\|R_T(λ)\|}{|h|}·\left\|\sum_{n=2}^∞ (-h R_T(λ))^n\right\| ≤ \frac{\|R_t(λ)\|}{|h|} \sum_{n=2}^∞ \|h R_t(λ)\|^n = \frac{\|R_T(λ)\|}{|h|}·\frac{\|h R_T(λ)\|^2}{1 - \|h R_T(λ)\|} ≤ $$
		$$ ≤ \frac{\|R_T(λ)\|}{|h|}·\frac{|h|^2 \|R_T(λ)\|^2}{1 / 2} = 2|h|·\|R_T(λ)\|^3 \rightarrow 0.\vspace{-1em} $$
	\end{dukazin}
\end{veta}

\begin{lemma}
	$X$ Banach space, $T$ operator in $X$, $0 \notin ς(T)$. Then $\forall λ ≠ 0: λ \in ς(T) \Leftrightarrow \frac{1}{λ} \in ς(T^{-1})$.

	\begin{dukazin}
		Since $0 \in ρ(T)$, so $T^{-1} \in ©L(X)$. Moreover, $T = (T^{-1})^{-1}$ is closed (by proposition above). In the same time, since $T$ is closed, we have $λ \in ρ(T) \Leftrightarrow λI - T$ is bijection („$\implies$“: trivial, „$\impliedby$“: $λI - T$ is bijection and closed operator, so by previous proposition $(λI - T)^{-1} \in ©L(X)$).

		So, it suffices: „$\forall λ ≠ 0$: $λI - T$ bijection $\Leftrightarrow$ $\frac{1}{λ}I - T^{-1}$ bijection“:
		$$ \frac{1}{λ}I - T^{-1} = -\frac{1}{λ}(λ I - T)T^{-1} \qquad \(\text{so $(λI - T)^{-1}$ exists $\implies$ $(\frac{1}{λ}I - T^{-1})^{-1}$ exists}\) $$
		$$ λI - T = -λ(\frac{1}{λ} I - T^{-1})T \qquad \(\text{so $(\frac{1}{λ}I - T^{-1})^{-1}$ exists $\implies$ $(λI - T)^{-1}$ exists}\). $$
	\end{dukazin}
\end{lemma}

\begin{dusledek}
	$X$ complex Banach, $T$ operator on $X$, $ς(T) = \O$. Then $T^{-1} \in ©L(X)$ and $ς(T^{-1}) = \{0\}$.

	\begin{dukazin}
		$0 \in ρ(T) \implies T^{-1} \in ©L(x)$. By previous lemma, $\forall λ ≠ 0: \frac{1}{λ} \notin ς(T^{-1})$. So $ς(T^{-1}) \subset \{0\}$. Since $ς(T^{-1}) ≠ \O$, we have $ς(T^{-1}) = \{0\}$.
	\end{dukazin}
\end{dusledek}

\subsection{Unbounded operators in Hilbert spaces}
\begin{definice}[Convention]
	From now, all Banach spaces are over $®K = ®C$ (if not said otherwise).
\end{definice}

\begin{definice}[Hilbert adjoint of operator]
	$H$ Hilbert, $T$ densely defined operator on $H$. Hilbert adjoint of $T$, denoted as $T^*$, is defined on $D(T^*) := \{y \in H | x \mapsto \<T x, y\> \text{ is continuous linear on $D(T)$}\}$. For $y \in D(T^*)$, $T^* y$ is the unique point from $H$ satisfying $\<Tx, y\> = \<x, T^* y\>$, $x \in D(T)$.

	\begin{dukazin}
		„$T^*y$ exists“: any $φ \in D(T)^*$ can be extended to $H^* = H$.
	\end{dukazin}
\end{definice}

% 25. 04. 2023

\pagebreak

\begin{tvrzeni}
	$H$ Hilbert, $S$ and $T$ densely defined in $H$.

	\begin{itemize}
		\item $S \subset T \implies T^* \subset S^*$.

			\begin{dukazin}
				$D(T^*) = \{y | x \mapsto\<Tx, y\> = \<Sx, y\> \text{ is continuous on } D(T) \supset D(S)\} \subset D(S^*)$. And for $y \in D(T^*)$:
				$$ \forall x \in D(S): \<x, T^* y\> = \<Tx, y\> = \<Sx, y\> = \<x, S^* y\> \implies T^* y = S^* y. $$
			\end{dukazin}
		\item $S+T$ is densely defined $\implies$ $S^* + T^* \subset (S + T)^*$ and if $S \in ©L(H)$, then there is equality.

			\begin{dukazin}
				For $y \in D(S^* + T^*) = D(S^*) \cap D(T^*)$ and $x \in D(S + T)$:
				$$ \<(S + T) x, y\> = \<x, S^* y\> + \<x, T^* y\> = \<x, (S^* + T^*) y\>. $$
				So, $y \in D((S + T)^*)$ and $(S+T)^* y = (S^* + T^*)(y)$. This proves the inclusion.

				„If $S \in ©L(H)$“ For $y \in D((S + T)^*)$ and for $x \in D(S + T) = D(T)$:
				$$ D(T) \ni x \mapsto \<Tx, y\> = \<(S + T)x, y\> - \<S x, y\> $$
				is constant on $D(T)$. So, $y \in D(T^*) = D(T^*) \cap D(S^*) = D(S^* + T^*)$. Thus, $D(S^* + T^*) = D((S + T)^*) \land S^* + T^* \subset (S + T)^*$, so $S^* + T^* = (S+T)^*$.
			\end{dukazin}

		\item $ST$ is densely defined $\implies$ $T^* S^* \subset (ST)^*$ and if $S \in ©L(H)$ then there is equality.

			\begin{dukazin}
				Pick $y \in D(T^* S^*)$. Then for $x \in D(ST)$:
				$$ \<STx, y\> = \<T x, S^* y\> = \<x, T^*S^* y\>. $$
				So, $y \in D((ST)^*)$ and $(ST)^* y = T^*S^* y$.

				„If $S \in ©L(H)$“: Then $D(ST) = D(T)$ and for $y \in D((ST)^*)$ we want „$S^*y \in D(T^*)$“ (then $y \in D(T^*S^*)$  and we are done):
				$$ D(T) \ni \mapsto \<Tx, S^* y\> = \<S T x, y\> = \<x, (ST)^* y\>. $$
				So, $x \mapsto \<Tx, S^* y\>$ is continuous on $D(T)$.
			\end{dukazin}
	\end{itemize}
\end{tvrzeni}

% 02. 05. 2023

\begin{tvrzeni}
	$H$ Hilbert, $T$ densely defined on $H$.

	\begin{itemize}
		\item $T^*$ is closed operator on $H$;
		\item $T$ has closed extension $\Leftrightarrow$ $T^*$ is densely defined. Then $(T^*)^* = \overline{T}$.
		\item $T$ is closed $\Leftrightarrow$ $T^*$ is densely defined and $T = (T^*)^*$.
	\end{itemize}
\end{tvrzeni}

\begin{lemma}
	$H$ Hilbert, $T$ densely defined on $H$. Consider $V \in ©L(H \oplus H)$ such that $V(x, y) := (-y, x)$. Then $V$ is unitary and $G(T^*) = V(G(T))^\perp$.

	\begin{dukazin}
		„$V$ is unitary:“ obvious ($V$ is isometry onto).

		„$G(T^*) \subseteq V(G(T))^\perp$“: Pick $y \in D(T^*)$ and $x \in D(T)$. Then
		$$ \<(y, T^* y), V(x, Tx)\> = \<(y, T^*y), (-Tx, x)\> = \<y, -Tx\> + \<T^* y, x\> = 0. $$

		„$V(G(T))^\perp \subseteq G(T^*)$“: Pick $(x, y) \in V(G(T))^\perp$. Then for $z \in D(T)$:
		$$ 0 = \<(x, y), (-Tz, z)\> = - \<x, Tz\> + \<y, z\>, $$
		so $\<x, Tz\> = \<y, z\>$, so $D(T) \ni z \mapsto \<Tz, x\>$ ($=\<z, y\>$) is continuous. So $x \in D(T^*)$ and $T^* x = y$, co $(x, y) \in G(T^*)$.
	\end{dukazin}
\end{lemma}

\begin{poznamka}
	$U \in ©L(H)$ unitary, $A \subset H$. Then $U(A^\perp) = U(A)^\perp$.

	\begin{dukazin}
		$x \in U(A)^\perp \Leftrightarrow \forall a \in A: 0 = \<x, Ua\> = \<U^*x, a\> \Leftrightarrow U^* x \in A^\perp \Leftrightarrow x \in U(A^\perp)$.
	\end{dukazin}
\end{poznamka}

\begin{dukaz}[Of the previous proposition]
	First point follows from the previous lemma.

	„Second point, $\implies$“: Pick $y_0 \in D(T^*)^\perp$. Wanted: $y_0 = 0$. We have $(y_0, 0) \in G(T^*)^\perp$ ($\forall z \in D(T^*): \<(z, T^*z), (y_0, 0)\> = 0$). $G(T^*)^\perp = V(G(T))^{\perp\perp} = \overline{V(G(T))} = V(\overline{G(T)})$. So $(0, -y_0) = V^*(y_0, 0) \in V^* V(\overline{G(T)}) = \overline{G(T)}$. Thus $y_0 = 0$ (because $T$ is closed).

	„Second point, $\impliedby$“: $T^*$ is densely defined. Then $(T^*)^*$ is defined and, by first point, it is closed. Moreover, „$T \subset (T^*)^*$“: Pick $x \in D(T)$. Then $D(T^*) \ni y \mapsto \<T^* y, x\> = \<y, Tx\>$, so $x \in D((T^*)^*)$ and $(T^*)^* x = Tx$.

	„Second point, then part“: $T \subseteq (T^*)^*$ is done, „$(T^*)^* \subseteq \overline{T}$“: it suffices to prove „$G((T^*)^*) = \overline{G(T)}$“: By previous lemma, $G((T^*)^*) = V(G(T^*))^\perp = V^*(G(T^*))^\perp = V^*(V(G(T))^\perp)^\perp = V^*V(G(T)^{\perp\perp}) = \overline{G(T)}$.

	„Third point“: „$\implies$“ follows directly from second point, „$\impliedby$“ by second point, $T$ has closed extension and $\overline{T} = (T^*)^* = T$, so ti is closed.
\end{dukaz}

\begin{tvrzeni}
	$H$ Hilbert, $T$ densely defined on $H$. Then

	\begin{itemize}
		\item $\Rang(T)^\perp = \Ker T^*$;
			\begin{dukazin}
				$y \in \Ker T^* \Leftrightarrow T^*y = 0 \Leftrightarrow \forall x \in D(T): \<Tx, y\> = 0 \Leftrightarrow y \in \Rang T^\perp$.
			\end{dukazin}
		\item If $T$ is moreover closed, then $\Ker T = (\Rang T^*)^\perp$.
			\begin{dukazin}
				By the previous proposition $T^*$ is densely defined and $T^{**} = T$. By the previous point, $\Ker T = \Ker T^{**} = (\Rang T^*)^\perp$.
			\end{dukazin}
	\end{itemize}
\end{tvrzeni}

\begin{tvrzeni}
	$H$ Hilbert, $T$ is one-to-one densely defined on $H$, $\overline{\Rang T} = H$. Then $T^*$ is one-to-one and $(T^*)^{-1} = (T^{-1})$.

	\begin{dukazin}
		Proof omitted (using the previous proposition and lemma).
	\end{dukazin}
\end{tvrzeni}

\begin{definice}[Self-adjoint operator, symmetric operator, maximally symetric operator]
	$H$ Hilbert, $T$ operator on $H$. $T$ is self-adjoint $≡$ $T = T^*$. $T$ is symmetric $≡$ $\forall x, y \in D(T): \<Tx, y\> = \<x, Ty\>$. $T$ is maximally symmetric $≡$ $T$ is symmetric, and there is no $S \supsetneq T$ symmetric.

	\begin{poznamkain}
		$T$ is self-adjoint $\implies$ $T$ is densely defined. $T$ is densely defined, then it is symmetric $\Leftrightarrow$ $T \subseteq T^*$. If $T$ is densely defined, then $T$ is self-adjoint $\implies$ symmetric. (And the other implication doesn't hold.)
	\end{poznamkain}
\end{definice}

\pagebreak

\begin{tvrzeni}
	$H$ Hilbert, $T$ densely defined and symmetric.

	\begin{itemize}
		\item $T$ has closed extension and $\overline{T}$ is symmetric;
		\item $R(T)$ is dense $\implies$ $T$ is one-to-one;
		\item $D(T) = H$ $\implies$ $T = T^*$ and $T \in ©L(H)$;
		\item $R(T) = H$ $\implies$ $T$ is one-to-one, self-adjoint and $T^{-1} \in ©L(H)$;
		\item $T$ is self-adjoint $\implies$ $T$ is maximally symmetric.
	\end{itemize}

	\begin{dukazin}
		Omitted.
	\end{dukazin}
\end{tvrzeni}

\begin{veta}
	$H$ Hilbert space, $H ≠ \{0\}$, $T$ is self-adjoint operator on $H$. Then $\O ≠ ς(T) \subseteq ®R$.

	\begin{dukazin}
		Let $T ≠ 0$ be self-adjoint. „$ς(T) ≠ 0$“: If $ς(T) = 0$, then by corollary above, $T^{-1} \in ©L(H)$ and $ς(T^{-1}) = \{0\}$. Moreover $T^{-1}$ is self-adjoint by the previous proposition (third point). So $0 = r(T^{-1}) = \|T^{-1}\|$, so $T^{-1} = 0$. \lightning.

		TODO? (Tady se něco zjednoduší: BÚNO $0 \not≡ T = T^*$. Kdyby $ς(T) = \O$, pak $T^{-1} \in ©L(H)$. Pak $T^{-1}$ je samoadjungovaný ($(T^{-1})^* = (T^*)^{-1} = T^{-1}$.).)

% 09. 05. 2023

		„$ς(T) \subseteq ®R$“: Let $λ \in ®C \setminus ®R$. Then
		$$ \overline{\Rang(λI - T)} = \Ker((λI - T)^*)^\perp = \Ker(\overline{λ}I - T^*)^\perp = \{¦o\}^\perp = H. $$
		By next lemma, $λI - T$ is onto. (Because $T$ is closed because $T$ is self-adjoint) and $(λI - T)^{-1}$ is continuous. Thus $λ \notin ς(T)$.

	\end{dukazin}
\end{veta}

\pagebreak

\begin{lemma}
	$T$ is symmetric on Hilbert $H$, $λ \in ®C \setminus ®R$. Then $(λ I - T)$ is one-to-one, $(λI - T)^{-1}$ is continuous on $R(λI - T)$, and moreover $T$ is closed $\Leftrightarrow$ $R(λI - T)$ is closed.

	\begin{dukazin}
		$λ = α + i·β$, $β ≠ 0$, $α, β \in ®R$. Then $αI - T$ is symmetric, so $\forall x \in D(T)$:
		$$ \|(λI - T)x\|^2 = \|(αI - T)x + i·βx\|^2 = \|i·β·x\|^2 + \|(αI - T)x\|^2 + 2\Re \<i·β·x, (αI - T)x\> = $$
		$$ = |β|^2·\|x\|^2 + \|(α I - T)x\|^2 + 0 ≥ |β|^2·\|x\|^2, $$
		cause $S$ is symmetric, then $\<Sx, x\> \in ®R$, $x \in D(S)$. So, $\|(λI - T)x\| ≥ |β|·\|x\|$, $x \in D(T)$, thus $(λI - T)$ is one-to-one. And $(λI - T)^{-1}$ is bounded on its domain, so continuous on its domain.

		It suffices: For $S := λI - T$: $S$ is closed $\Leftrightarrow$ $R(S)$ is closed. And proof of this is omitted.

% Důkaz po sem byl ještě na předchozí přednášce
		
		„Moreover“: Denote $S := λI - T$ ($S$ closed $\Leftrightarrow$ $T$ closed). „$\implies$“: Let $S$ be closed, then „$\Rang S$“ is closed: $\Rang S \ni y_n \rightarrow y \implies (S^{-1}(y_n))$ is Cauchy, so there is $x \in D(S): S^{-1}y_n \rightarrow x$. Then $(S^{-1}y_n, y_n) \rightarrow (x, y)$, so $Sx = y$.

		„$\impliedby$“: Let $\Rang S$ be closed. Then „$G(S)$ is closed“: $(x_n, S x_n) \rightarrow (x, y) \implies x_n = S^{-1}S x_n \rightarrow S^{-1}y$. So $S^{-1} y = x$.
	\end{dukazin}
\end{lemma}

\begin{dusledek}[Of the previous theorem]
	$H$ Hilbert, $T$ operator on $H$. Then next propositions are equivalent
	\begin{itemize}
		\item $T$ is self-adjoint;
		\item $T$ is densely defined, symmetric and $ς(T) \subseteq ®R$;
		\item $T$ is densely defined, symmetric and there is $λ \in ®C \setminus ®R: λ, \overline{λ} \in ς(T)$.
	\end{itemize}

	\begin{dukazin}
		„$1. \implies 2.$“ use the previous theorem. „$2. \implies 3.$“ easy. „$3. \implies 1.$“: $T \subset T^*$ by third point. Wanted: „$D(T^*) \subset D(T)$“: Pick $x \in D(T^*)$. Put
		$$ y := (λI - T)^{-1} \((λI - T^*)x\) \in \Rang((λI - T)^{-1}) = D(λ I - T). $$
		Then
		$$ (λI - T^*)x = (λI - T)y = λ y - T y = λy - T^*y = (λI - T^*)y. $$
		$λI - T^*$ is one-to-one ($\Ker(λI - T^*) = \Ker((\overline{λ}I - T)^*) = \Rang(\overline{λ}I - T)^\perp = H^\perp = \{¦o\}$). So, $x = y \in D(T)$.
	\end{dukazin}
\end{dusledek}

\section{Cayley transform}
\begin{poznamka}[Motivation]
	$T$ self-adjoint, then $ς(T) \subseteq ®R$ and $M(z) = \frac{z - i}{z + i}$, $z \in ®R$ is bijection between $®R$ and $®D \setminus \{1\}$.
\end{poznamka}

\begin{definice}[Cayley transform of operator]
	$H$ Hilbert, $T$ symmetric operator on $H$. Then Cayley transform of $T$ is the operator $©C(T) := (T - iI)·(T + i·I)^{-1}$.

	\begin{poznamkain}
		$©C(T)$ is well defined: $T + iI$ is one-to-one, $\Rang(T + iI)^{-1} = D(T + iI) = D(T - iI)$.

		$Tx + ix \overset{©C(T)} Tx - ix$.
	\end{poznamkain}
\end{definice}

\begin{veta}
	$H$ Hilbert, $T$ symmetric operator on $H$, $©C(T)$ Cauchy transform. Then
	\begin{itemize}
		\item $©C(T)$ is linear isometry $D(©C(T)) = R(T + iI)$ onto $R(©C(T)) = R(T - iI)$;
			\begin{dukazin}
				$D(©C(T)) = R(T + iI)$ by definition. $R(©C(T)) = R(T - iI)$ by definition too.

				For $y = Tx + ix \in D(©C(T))$ we have
				$$ \|©C(T)y\|^2 = \|Tx + ix\|^2 \overset{\text{COS}}= \|Tx\|^2 + \|x\|^2 + 2\Re\<Tx, -ix\> = \|Tx\|^2 + \|x\|^2 $$
				$$ \|y\|^2 = \|Tx + ix\|^2 = … = \|Tx\|^2 + \|x\|^2. $$
				So, $©C(T)$ is isometry.
			\end{dukazin}

		\item $I - ©C(T) = 2i(T + iI)^{-1}$, and so $I - ©C(T)$ is one-to-one and $R(I - ©C(T)) = D(T)$;
			\begin{dukazin}
				Let $y = Tx + ix \in D(©C(T))$, then
				$$ (I - ©C(T))y = y - ©C(T)y = Tx + ix - (Tx - ix) = 2ix = (T + iI)^{-1}y $$
				$\implies$ formula holds.

				Since $(T+iI)^{-1}$ is one-to-one, $I - ©C(T)$ is one-to-one. Moreover, $R(I - ©C(T)) = R((T + iI)^{-1}) = D(T + iI) = D(T)$.
			\end{dukazin}

% 15. 05. 2023

\pagebreak

		\item $T = i\(I + ©C(T)\)·\(I - ©C(T)\)^{-1}$.
			\begin{dukazin}
				We know $D(T) = R(I - ©C(T))$ and $R\((I - ©C(T))^{-1}\) = D(I - ©C(T)) = D(I + ©C(T))$. So operator on RHS is well-defined and LHS have same domain as RHS.

				Pick $y \in D(T)$ and $x \in D(©C(T))$ such that $(I - ©C(T)) x = y$. Then
				$$ y - (I - ©C(T))x = 2i(T + iI)^{-1} x, $$
				so $i(I + ©C(T))·\(I - ©C(T)\) y = i(I + ©C(T))x = i\(x + (T - iI)(T + iI)^{-1}x\) =$
				$$ = i\(x + \(T - iI\)·(y / 2i)\) = \frac{i}{2i}\(2ix + (T - iI)y\) = \frac{1}{2} ((T + iI)y + (T - iI)y) = Ty. $$
			\end{dukazin}

		\item $T$ closed $\Leftrightarrow$ $©C(T)$ closed $\Leftrightarrow$ $D(©C(T))$ closed $\Leftrightarrow$ $R(©C(T))$ closed.
			\begin{dukazin}[Omitted.\vspace{-1em}]
			\end{dukazin}
	\end{itemize}
\end{veta}

\begin{veta}
	Let $H$ be a Hilbert space and $U$ isometry form $D(U)$ onto $R(U)$. Let $I - U$ be one-to-one. Then $T := i(I + U)(I - U)^{-1}$ is symmetric and $©C(T) = U$. Moreover $T$ is densely defined if and only if $R(I - U)$ is dense.

	\begin{dukazin}
		$T$ is well-defined: $R((I - U)^{-1}) = D(I - U) = D(I + U)$. $D(T) = R(I - U)$, so $T$ is densely defined iff $R(I - U)$ is dense.

		„$T$ is symmetric“: Let $x = (I - U)x' \in D(T)$, $y = (I - U)y' \in D(T)$.
		$$ \<Tx, y\> = \<i(I + U)x', y\> = i\<x' + Ux', y' - Uy'\> \overset{U \text{ isometry}}= i\(-\<x' Uy'\> + \<Ux', y'\>\), $$
		$$ \<x, Ty\> = … = \<x, i(I + u)y'\> = -i\<x' - Ux'\> = -i\(\<x', Uy'\> - \<Ux', y'\>\). $$

		„$©C(T) = U$“: Let $x = (I - U)x' \in D(T)$:
		$$ (T - iI)x = i(I + U)x' - ix = i(x' + Ux') - i(x' - Ux') = 2i Ux', $$
		$$ (T + iI)x = … + ix = … + … = 2i x'. $$

		So, $x' \in R(T + iI) = D(©C(T))$ and $D(U) \subseteq D(©C(T))$ and $D(©C(T)) = R(T + iI) \subseteq D(U)$. Thus, $D(U) = D(©C(T))$. Finally, for $x \in D(T)$:
		$$ U(Tx + ix) = U(2i x') = 2i U x' = (T - iI)x = Tx - ix.\vspace{-1em} $$
	\end{dukazin}
\end{veta}

\begin{veta}
	$H$ Hilbert: \vspace{-1em}
	\begin{itemize}
		\item[a)] Let $T$ be a symmetric operator on $H$. Then $T$ is self-adjoint $\Leftrightarrow$ $©C(T)$ is unitary (i.e. $D(®C(T)) = H = R(©C(T))$).
		\item[b)] $U \in ©U(H)$ such that $I - U$ is one-to-one, then
			$$ T := i(I + U)(I - U)^{-1} $$ is self-adjoint and $©C(T) = U$.i
	\end{itemize}

	\begin{dukazin}
		„a) $\implies$“: Since $ς(T) \subseteq ®R$, we have $± i \in ρ(T)$, so $T ± iI$ are onto, so $D(©C(T)) = H = R(©C(T))$ by the theorem above.

		„a) $\impliedby$“: We have $D(T)^{\perp} = R(I - ©C(T))^{\perp} = \Ker (I - ©C(T))^* = \Ker(I - ©C(T)) = \{¦o\}$, co $T$ is densely defined. Moreover, $T ± iI$ is onto, so $I i \in ρ(T)$. Thus, from the corollary above, $T$ is self-adjoint.

		„b)“: $©C(T) = U$ by the previous theorem. Moreover $D(T)^\perp = R(I - U)^\perp = … = \{¦o\}$, so $T$ is densely-defined. It remains „$T ± iI$ is onto“: Fix $y \in H$, put $zi = (I - U)y$, then:
		$$ (T + iI)z = Tz + iz = i(I + U)y + i(I - U)y = 2iy, $$
		$$ (T - iI)z = Tz - iz = i(I + U)y - i(I - U)y = 2i Uy. $$
		So, (Since $D(U) = H = R(U)$), we have $T ± iI$ is onto.
	\end{dukazin}
\end{veta}

\begin{definice}[$n_+$ and $n_i$ (deficiency indices)]
	Let $T$ be a symmetric closed operator in a Hilbert space H. Then
	$$ n_+(T) = \dim(\Rang(T + iI))^\perp = \dim D(©C(T))^\perp, $$
	$$ n_-(T) = \dim(\Rang(T - iI))^\perp = \dim \Rang(©C(T))^\perp $$
	are called deficiency indices of the operator $T$.
\end{definice}

\begin{veta}
	$T$ symmetric, densely defined, closed operator on separable (we prove it only for separable) $H$. Then
	\begin{itemize}
		\item[a)] $T$ is self-adjoint $\Leftrightarrow$ $n_+(T) = n_-(T) = 0$;
		\item[b)] ($T$ is maximal symmetric $\Leftrightarrow$ $\min(n_+(T), n_-(T)) = 0$;)
		\item[c)] $T$ has self-adjoint extension $\Leftrightarrow$ $n_+(T) = n_-(T)$.
	\end{itemize}

	\begin{dukazin}
		„a)“: $T$ self-adjoint $\Leftrightarrow$ $©C(T)$ is unitary $\Leftrightarrow$ $D(©C(T)) = R(©C(T)) = H$ $\overset{*}\Leftrightarrow$ $n_+(T) = 0 = n_-(T)$. *) $T$ is closed, so $D(©C(T)) ≠ H \Leftrightarrow n_+(T) > 0$ and $R(©C(T)) ≠ 0$ $\Leftrightarrow n_-(T) > 0$ (from item d) from the theorem above).

		„b)“ omitted.

		„c) $\implies$“: Let $S \supseteq T$ be self-adjoint. Then $©C(S) \supseteq ©C(T)$ and $©C(S)$ is unitary and $©C(S)(D(©C(T))) = R(©C(T))$, $©C(S)(…^\perp) = R(©C(T))^\perp$ ($U$ unitary, $U(A) = B$ $\overset{\text{easy}}\implies$ $U(A^\perp) = B^\perp$). So,
		$$ n_+(T) = \dim D(©C(T))^\perp = \dim R(©C(T))^\perp = n_-(T) $$

		Since $H$ is separable, we have $n_+(T) = n_-(T)$ $\Leftrightarrow$ $\exists$ isometry between $D(©C(T))^\perp$ and $R(©C(T))^\perp$ (because Hilbert spaces are isometric to right $l_2$). Let $V \supseteq ©C(T)$ is unitary operator such that $V(R(©C(T))^\perp) = R(©C(T))^\perp$.

		Then „$R(I - V)$ is dense and $I - V$ is one-to one.“:
		$$ R(I - V) \supseteq R(I - ©C(T)) = D(T), $$
		so $R(I - V)$ is dense. Fix $x \in \Ker(I - V)$ and $y \in D(V)$. Then
		$$ \<x, (I - V)y\> = \<x, y\> - \<x, Vy\> = \<Vx, Vy\> - \<x, Vy\> = \<Vx - x, Vy\> = \<¦o, Vy\> = 0. $$
		Thus, $x \in R(I - V)^\perp = \{¦o\}$.

		$\implies$ $\exists S$ symmetric and densely defined such that $©C(S) = V \supseteq ©C(T)$, so $S \supseteq T$ ($S = i(I + V)(I - V)^{-1} \supseteq i(I + ©C(T))(I - ©C(T))^{-1} = T$).
	\end{dukazin}
\end{veta}

\section{Integral of unbounded function with respect to a spectral measure}
\begin{definice}
	$H$ Hilbert, $(X, ©A)$ is measurable space, $E$ spectral measure for $(X, ©A, H)$, $E$ spectral measure for $(X, ©A, H)$, $f: X \rightarrow ®C$ is ©A-measurable. Then $\int f dE$ is the operator on $H$ such that
	$$ D(\int f dE) := \{x \in H \middle| \int |f|^2 dE_{x, x} < ∞\}, \qquad \<Tx, y\> := \int_X f dE_{x, y}, \quad x, y \in D(T). $$
\end{definice}

\begin{veta}
	$H$ Hilbert, $(X, ©A)$ is measurable space, $E$ spectral measure for $(X, ©A, H)$, $E$ spectral measure for $(X, ©A, H)$, $f: X \rightarrow ®C$ is ©A-measurable. Then $D := \{x \in H | \int_X |f|^2 dE_{x, x} < ∞\}$ is dense subspace of $H$, $\int f dE$ exists (and it is unique).

	Moreover, $\|Tx\|^2 = \int_X |f(λ)| dE_{x, x}$, $x \in D(\int f dE)$.

	\begin{dukazin}
		„$D$ is subspace“: From proposition (basic properties of spectral measure) sixth item (addition) and fourth point (multiplication).

		„For $A_n := f^{-1}(B(¦o, n))$ we have $\Rang E(A_n) \subseteq D(\int f dE)$, $n \in ®N$“: $\forall x \in \Rang E(A_n)$:
		$$ E_{x, x}(A_n) = \<E(A_n)x, x\> = \<x, x\> = \<E(X)x, x\> = E_{x, x}(X). $$
		So, $E_{x, x}(X \setminus A_n) = 0$, so $|f| ≤ n$ $E_{x, x}$-almost everywhere, so
		$$ \int_X |f|^2 dE_{x, x} ≤ n^2 \int_X 1·E_{x, x} < ∞. $$

% 16. 05. 2023

		„$D$ is dense“: Pick $y \in H$, then $D \ni E(A_n)y \rightarrow y$ ($\|E(a_n) y - y\|^2 = \|E(X \setminus A_n)y\|^2 = E_{y, y}(X \setminus A_n) \rightarrow 0$.)

		„$\forall x, y \in D: \int f dE_{x, y} \in ®C$“: $(x, y) \mapsto E_{x, y}$ is sesquilinear, so it suffices to check it for $x = y$. But $f \in L^2(E_{x, x}) \subseteq L^1(E_{x, x})$, so $\int f dE_{x, x} \in ®C$.

		„Definition of $T$“: For $x \in D$ put $Tx := \lim_{n \rightarrow ∞}\(\int_X f χ_{A_n} dE\)x$. „$T$ well defined“: limit exists, because the sequence is cauchy:
		$$ \forall m < n: \|\int f χ_{A_n} dE x - \int f χ_{A_m} dE x\|^2 = \|\int f χ_{A_n \setminus A_m} dE x\|^2 = \int_{A_n \setminus A_m} |f|^2 dE_{x, x} \rightarrow 0. $$
		„$T$ linear“: easy (VAL + Linearity of the integral). „For $T$ equation holds“: By sesquilinearity, suffices to check for $x = y \in D$:
		$$ \<Tx, x\> = \lim \<\int_X f χ_{A_n} dE x, x\> = \lim \int f χ_{a_n} dE_{x, x} \!\overset{\text{Lebesgue}}=\! \int \lim f χ_{a_n} d E_{x, x} = \int f dE_{x, x}. $$
		„$\|Tx\| = \sqrt{…}$“:
		$$ \|Tx\|^2 = \lim \<\int f χ_{A_n} dE x, \int fχ_{A_n} dEx\> = \lim \int |f χ_{A_n}|^2 dE_{x, x} \overset{\text{Lebesgue}}= \int |f|^2 d E_{x, x}. $$
		„Uniqueness“: $\<Tx, y\> = \<z, y\>$, $y \in D$ $\implies$ $Tx = z$ on $H$, because $D$ is dense.
	\end{dukazin}
\end{veta}

\begin{veta}
	Let $H$ Hilbert space, $(X, ©A)$ measurable space, $E$ spectral measure for $(X, ©A, H)$ and $f, g: X \rightarrow ®C$ be ©A-measurable functions. Then the following assertions hold:
		
	$\int f dE + \int g dE \subset \int f + g dE$;
		\begin{dukazin}[Omitted. (From definition.)\vspace{-1em}]
		\end{dukazin}
	$\(\int f dE\)\(\int g dE\) \subset \int f g dE$ and $D\(\(\int f dE\)\(\int g dE\)\) = D\(\int g dE\) \cap D\(\int f g dE\)$;

		\begin{dukazin}[Omitted. (Technical, difficult, from definition of bounded version.)\vspace{-1em}]
		\end{dukazin}
	$(\int f dE)^* = \int \overline{f} dE$ and $\int f dE \(\int f dE\)^* = \int |f|^2 dE = (\int f dE)^* \int f dE$, that is, $\int f dE$ is normal;
		\begin{dukazin}[Omitted.\vspace{-1em}]
		\end{dukazin}
	$\int f dE$ is closed;
		\begin{dukazin}
			From the previous item: $\int f dE = \int \overline{\overline{f}} dE = \(\int \overline{f} dE\)^*$ $\implies$ (by the proposition above) $\int f dE$ is closed.
		\end{dukazin}
	$\int f dE \in ©L(H)$ $\Leftrightarrow$ $\exists A \in ©A$: $E(X \setminus A) = ¦o$ $\land$ $f$ is bounded on $A$.
		\begin{dukazin}
			„$\impliedby$“: „$D(\int f dE) = H$“: $\forall x \in H: \int_X |f|^2 dE_{x, x} = \int_A |f|^2 dE_{x, x} < ∞$. „$\forall x \in H: \|\int f dE x\|^2 ≤ C·\|x\|^2$“: from the previous theorem:
			$$ \|\int f dE x\|^2 = \int_X |f|^2 dE_{x, x} = \int_A |f|^2 dE_{x, x} ≤ \|f|_A\|_∞·E_{x, x}(X) ≤ \|f|_A\|_∞·\|x\|^2. $$
			„$\implies$“: Put $K := \|\int |f| dE\| < ∞$, $A := \{t | |f(t)| ≤ K+1\}$. Then „$E(X \setminus A) = 0$“: If not, $\exists x \in S_H \cap \Rang E(X \setminus A)$ and then
			$$ K + 1 = \int (K + 1) dE_{x, x} ≤ \int_{A^c} |f| dE_{x, x} = \int |f| χ_{A^c} d E_{x, x} = \<\int χ_{A^c} dE \int |f| dE x, x\> = $$
			$$ = \<E(A^c)·\int |f| dE x, x\> = \<\int |f| dE_{x, x}, E(A^c) x\> = \< \int |f| dE x, x\> ≤ $$
			$$ ≤ \left\|\int |f| dE x\right\|·1 ≤ \left\|\int |f| dE\right\|·1·1 = K. $$
		\end{dukazin}
\end{veta}

\begin{veta}
	Let $H$ be a Hilbert space, $(X, ©A)$ measurable space, $E$ spectral measure for $(X, ©A, H)$ and $f: X \rightarrow ®C$ be ©A-measurable function. Then
	$$ ς\(\int f dE\) = ess \Rang f:= \{λ \in ©C | \forall r > 0: E(f^{-1}(U(λ, r))) ≠ 0\}. $$
	Moreover, for $λ \in ®C$ we have $\Ker(λI - \int f dE) = \Rang(E(f^{-1}(\{λ\})))$. Thus $λ \in ς_P(\int TODO)$ TODO.
\end{veta}

\end{document}
