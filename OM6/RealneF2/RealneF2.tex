\documentclass[12pt]{article}					% Začátek dokumentu
\usepackage{../../MFFStyle}					    % Import stylu



\begin{document}

% 13. 02. 2023

\section{Area formula and coarea formula}
\begin{veta}
	Let $(P_1, \rho_1)$, $(P_2, \rho_2)$ be metric spaces, $s > 0$, and $f: P_1 \rightarrow P_2$ be $\beta$-Lipschitz. Then $\kappa^s(f(P_1)) ≤ \beta^s \kappa^s(P_1)$.

	\begin{dukazin}
		Choose $\delta > 0$. Let $P_1 = \bigcup_{i=1}^∞ A_j$, $\diam A_j < \delta$. Then we have $f(P_1) = \bigcup_{j=1}^∞ f(A_j)$, $\diam f(A_j) < \beta·\delta$.
		$$ \kappa^s(f(P_1), \beta·\delta) ≤ \sum_{j=1}^∞ (\diam f(A_j))^s ≤ \sum_{j=1}^∞ \beta^s·(\diam A_j)^s = \beta^s·\sum_{j=1}^∞ (\diam A_j)^s. $$
		It holds for all possible choices of $\(A_j\)$, so we can take infimum:
		$$ \kappa^s(f(P_1)) \leftarrow \kappa^s(f(P_1), \beta·\delta) ≤ \beta^s \inf_{\(A_j\)} \sum_{j=1}^∞ (\diam A_j)^s = \beta^s \kappa^s(P_1, \delta) \rightarrow \beta^s \kappa^s(P_1). $$
	\end{dukazin}
\end{veta}

\begin{lemma}
	Let $k, n \in ®N$, $k ≤ n$, and $L: ®R^k \rightarrow ®R^n$ be an injective linear mapping. Then for every $\lambda_k$-measurable set $A \subset ®R^k$ it holds $H^k(L(A)) = \sqrt{\det(L^TL) \lambda_k(A)}$.

	\begin{dukazin}[$\dim L(®R^k) = k$]
		We find linear isometry $Q$ of $®R^k$ onto $L(®R^k)$, from last semester
		$$ H^k(L(A)) = H^k(Q^{-1} \circ L(A)) = \lambda^k(Q^{-1} \circ L(A)) = |\det(Q^{-1}L)|·\lambda_k(A). $$
		$$ (\det(Q^{-1}L))^2 = \det((Q^{-1} L)^T)·\det(Q^{-1} L) = \det((Q^{-1} L)^T·(Q^{-1} L)) = \det((\<Q^{-1} L e^i, Q^{-1} L^T e^j\>)_{i,j}). $$
		And because $Q$ is isometry ($\implies Q^{-1}$ is isometry), we can remove $Q^{-1}$ from scalar product and we get $\det(L^T L)$.
	\end{dukazin}
\end{lemma}

\begin{lemma}
	Let $k, n \in ®N$, $k ≤ n$, $G \subset ®R^k$ be an open set, $\phi: G \rightarrow ®R^n$ be an injective regular mapping, $x \in G$, and $\beta > 1$. Then there exists a neighbourhood $V$ of the point $x$ such that

	\begin{itemize}
		\item the mapping $y \mapsto \phi(\phi'(x)^{-1}(y))$ is $\beta$-Lipschitz on $\phi'(x)(V)$;
		\item the mapping $z \mapsto \phi'(x)(\phi^{-1}(z))$ is $\beta$-Lipschitz on $\phi(V)$.
	\end{itemize}

	\begin{dukazin}
		$x$, $\beta$ fixed. We know, that there exists $\eta > 0$ such that
		$$ \forall v \in ®R^k: \|\phi'(x)(v)\| ≥ \eta·\|v\|. $$
		We find $\epsilon \in \(0, \frac{1}{2}\eta\)$ such that $\frac{2\epsilon}{\eta} + 1 < \beta$. We find a neighbourhood $V$ of $x$ such that $\forall y \in V: \|\phi'(x) - \phi'(y)\| ≤ \epsilon$.

		We show that for every $u, v \in V$ we have
		$$ \|\phi(u) - \phi(v) - \phi'(x)(u - v)\| ≤ \epsilon \|u - v\|. $$
		Fix $v \in V$ and consider the mapping
		$$ g: w \mapsto \phi(w) - \phi(v) - \phi'(x)(w - v). $$
		For $w \in V$ we have $g'(w) = \phi'(w) - \phi'(x)$:
		$$ \|\phi(u) - \phi(v) - \phi'(x)(u - v)\| = \|g(w) - g(v)\| ≤ \sup \{\|g'(w)\|\ |\ w \in V\}·\|u - v\| ≤ \epsilon·\|u - v\|. $$
		
		Further we show that for every $u, v \in V$ we have
		$$ \|\phi(u) - \phi(v)\| ≥ \frac{1}{2} \eta \|u - v\|. $$
		For $u - v \in V$ we compute
		$$ \|\phi(u) - \phi(v)\| ≥ - \|\phi(u) - \phi(v) - \phi'(x)(u - v)\| + \|\phi'(x)(u - v)\| ≥ - \epsilon \|u - v\| + \eta \|u - v\| ≥ \frac{1}{2}\eta \|u - v\|. $$

		„First point“: TODO (řádek nebyl k přečtení)
		$$ \|\phi(\phi^{-1}(x)(a)) - \phi(\phi^{-1}(x)(b))\| = \|\phi(u) - \phi(v)\| ≤ $$
		$$ ≤ \|phi(u) - \phi(v) - \phi'(x)(u - v)\| + \|\phi'(x)(u - v)\| ≤ $$
		$$ ≤ \epsilon·\|u - v\| + \|\phi'(x)(y - v)\| ≤ \epsilon \frac{1}{\eta} \|a - b\| + \|a - b\| = \(\frac{\epsilon}{\eta} + 1\)\|a - b\| ≤ \beta·\|a - b\|. $$

		„Second point“: $k, q \in \phi(V)$. We find $u, v \in V$ such that $\phi(u) = p$ and $\phi(v) = q$:
		$$ \|\phi'(x)(\phi^{-1}(p)) - \phi'(x)(\phi^{-1}(q))\| = \|\phi'(x)(u) - \phi'(x)(v)\| = $$
		$$ = \|\phi'(x)(u - v)\| ≤ \|\phi(u) - \phi(v) - \phi'(x)(u - v)\| + \|\phi(u) - \phi(v)\| ≤ $$
		$$ ≤ \epsilon·\|u - v\| + \|p - q\| ≤ \frac{2\epsilon}{\eta}\|\phi(u) - \phi(v)\| + \|p - q\| = \(\frac{2\epsilon}{\eta} + 1\)\|p - q\| ≤ \beta \|p - q\|. $$
	\end{dukazin}
\end{lemma}

% 20. 02. 2023

\begin{lemma}
	Let $k, n \in ®N$, $k ≤ n$, $G \subset ®R^k$ be an open set, $\phi: G \rightarrow ®R^n$ be an injective regular mapping, $x \in G$, and $\alpha > 1$. Then there exists a neighbourhood of $x$ such that for every $\lambda^k$-measurable $E \subset V$ we have
	$$ \alpha^{-1} \int_E \vol \phi'(t) d\lambda^k(t) ≤ H^k(\phi(E)) ≤ \alpha \int_E \vol \phi'(t) d\lambda^k(t). $$

	\begin{dukazin}
		Find $\beta > 1$, $\tau > 1$ such that $\beta^k \tau < \alpha$. By previous lemma we find a neighbourhood $V_1$ of $x$ such that the conclusion of the lemma holds for $\beta$. We find a neighbourhood $V_2$ of $x$ such that
		$$ \forall t \in V_2: \tau^{-1} \vol \phi'(t) ≤ \vol \phi'(t) ≤ \tau \vol \phi'(x). $$
		Set $V = V_1 \cap V_2$.

		Assume that $E \subset V$ is a $\lambda^k$-measurable set. We have
		$$ \tau^{-1} \vol \phi'(x)·\lambda^k(E) ≤ \int_E \vol \phi'(t) d \lambda^k(t) ≤ \tau \vol \phi'(t) \lambda^k(E). $$
		By lemma above we have $\vol \phi'(t) \lambda^k(E) = H^k(\phi'(x)(E))$:
		$$ \tau^{-1} H^k(\phi'(x)(E)) ≤ \int_E \vol \phi'(t) d\lambda^k(t) ≤ \tau H^k(\phi'(x)(E)). $$

		By previous lemma we get 
		$$ H^k(\phi(E)) = H^k\(\(\phi \circ (\phi'(x))^{-1}\circ \phi'(x)\) (E)\) ≤ \beta^k H^k(\phi'(x)(E)) ≤ \beta^k H^k(\phi'(x)(E)) ≤ $$
		$$ ≤ \beta^k \tau \int_E \vol \phi'(t) d\lambda^k(t) ≤ \alpha \int_E \vol \phi'(t) d\lambda^k(t). $$

		By lemma above we get
		$$ H^k(\phi(E)) ≥ \beta^{-k} H^k\(\(\phi'(x) \circ \phi^{-1} \circ \phi\)(E)\) = \beta^{-k} H^k(\phi'(x)(E)) ≥ $$
		$$ ≥ \beta^{-k} \tau^{-1} \int_E \vol \phi'(t) d\lambda^k(t) ≥ \alpha^{-1} \int_E \vol \phi'(t) d\lambda^k(t). $$
	\end{dukazin}
\end{lemma}

\begin{veta}
	Let $k, n \in ®N$, $k ≤ n$, $G \subset ®R^k$ be an open set, $\phi: G \rightarrow ®R^n$ be an injective regular mapping and $f: \phi(G) \rightarrow ®R$ be $H^k$-measurable. Then we have
	$$ \int_{\phi(G)} f(x) dH^k(x) = \int_G f(\phi(t)) \vol \phi'(t) d\lambda^k(t), $$
	if the integral at the right side converges.

	\begin{dukazin}
		„$\phi^{-1}$ is well defined“: If $H \subset G$ is open, then we can write $H = \bigcup_{n=1}^∞ K_n$, where $K_n$ is compact for every $n \in ®N$. Then we have $\phi(H) = \bigcup_{n=1}^∞ \underbrace{\phi(K_n)}_{\text{compact}}$ is $F_\sigma$. This implies that $\phi^{-1}$ is Borel. The mappings $\phi$, $\phi^{-1}$ are locally Lipschitz by lemma above. ($\phi(G)$ is Borel.) $\phi(G)$ is $H^k$-$\sigma$-finite.

		1. „$f = \chi_L$, $L \subset \phi(G)$ is $H^k$-measurable“: We show $H^k(L) = \int_{\phi^{-1}(L)} \phi'(t) d\lambda^k(t)$. Choose $\alpha > 1$. By previous lemma we find for every $y \in G$ neighbourhood $V_y \subset G$ of the point $y$ such that for every $\lambda^k$-measurable set $E \subset V_y$ we have
		$$ \alpha^{-1} \int_E \vol \phi'(t) d\lambda^k(t) ≤ H^k(\phi(E)) ≤ \alpha \int_E \vol \phi'(t) d\lambda^k(t). $$
		We have $\subset\{V_y | y \in G\} = G$. There exists a sequence $\{y_j\}_{j=1}^∞$ such that $\bigcup_{i=1}^∞ V_{y_j} = G$. Using lemma from previous semester we find Borel sets $B, N \subset \phi(G)$ such that $B \subset L \subset B \cup N$, $H^k(N) = 0$.

		$\lambda^k(\phi^{-1}(N)) = 0$. $\phi^{-1}(B) \subset \phi^{-1}(L) \subset \phi^{-1}(B) \cup \phi^{-1}(N)$ $\implies$ $\phi^{-1}(L)$ is $\lambda^k$-measurable. We set
		$$ A_j = \phi^{-1}(L) \cap \(V_{y_j} \setminus \bigcup_{i=1}^{j-1} V_{y_j}\). $$
		Then we have

		\begin{itemize}
			\item $A_j$ is $\lambda^k$-measurable;
			\item $A_j \subset V_{y_j}$ for every $j \in ®N$;
			\item $\forall j, j' \in ®N, j ≠ j': A_j \cap A_{j'} = \O$;
			\item $\bigcup_{j=1}^∞ A_j = \phi^{-1}(L)$;
			\item for every $j \in ®N$ we have
				$$ \alpha^{-1} \int_{A_j} \vol \phi'(t) d\lambda^k(t) ≤ H^k(\phi(A_j)) ≤ \alpha \int_{A_j} \vol \phi'(t) d\lambda^k(t). $$
		\end{itemize}

		From all except for second point we have
		$$ \alpha^{-1} \int_{\phi^{-1}(L)} \vol \phi'(t) d\lambda^k(L) ≤ \underbrace{\sum_{j=1}^∞ H^k(\phi(A_j))}_{= H^k\(\bigcup_{j=1}^∞ \phi(A_j)\) = H^k(L)} ≤ \alpha \int_{\phi^{-1}(L)} \vol \phi'(t) d\lambda^k(t). $$

		2. „$f ≥ 0$ simple $H^k$-measurable“: From linearity of integrals. 3. „$f ≥ 0$ $H^k$-measurable“: we approximate $f$ by $0 ≤ f_j ≤ f_{j+1}$ simple functions and from Levi
		$$ \lim_{j \rightarrow ∞} \int_{\phi(G)} f_j(x) dH^k(x) = \int_{\phi(G)} f(x) dH^k(x), \qquad \lim_{j \rightarrow ∞} \int_G f_j(\phi(t))\vol\phi'(t) d\lambda^k(t) = \int_G f(\phi(t)) \vol \phi'(t) d\lambda^k(t). $$

		3. „$f$ $H^k$-measurable“: We add positive and negative part.
	\end{dukazin}
\end{veta}

\end{document}
