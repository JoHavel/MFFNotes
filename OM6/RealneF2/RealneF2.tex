\documentclass[12pt]{article}					% Začátek dokumentu
\usepackage{../../MFFStyle}					    % Import stylu



\begin{document}

% 13. 02. 2023

\section{Area formula and coarea formula}
\begin{veta}
	Let $(P_1, \rho_1)$, $(P_2, \rho_2)$ be metric spaces, $s > 0$, and $f: P_1 \rightarrow P_2$ be $\beta$-Lipschitz. Then $\kappa^s(f(P_1)) ≤ \beta^s \kappa^s(P_1)$.

	\begin{dukazin}
		Choose $\delta > 0$. Let $P_1 = \bigcup_{i=1}^∞ A_j$, $\diam A_j < \delta$. Then we have $f(P_1) = \bigcup_{j=1}^∞ f(A_j)$, $\diam f(A_j) < \beta·\delta$.
		$$ \kappa^s(f(P_1), \beta·\delta) ≤ \sum_{j=1}^∞ (\diam f(A_j))^s ≤ \sum_{j=1}^∞ \beta^s·(\diam A_j)^s = \beta^s·\sum_{j=1}^∞ (\diam A_j)^s. $$
		It holds for all possible choices of $\(A_j\)$, so we can take infimum:
		$$ \kappa^s(f(P_1)) \leftarrow \kappa^s(f(P_1), \beta·\delta) ≤ \beta^s \inf_{\(A_j\)} \sum_{j=1}^∞ (\diam A_j)^s = \beta^s \kappa^s(P_1, \delta) \rightarrow \beta^s \kappa^s(P_1). $$
	\end{dukazin}
\end{veta}

\begin{lemma}
	Let $k, n \in ®N$, $k ≤ n$, and $L: ®R^k \rightarrow ®R^n$ be an injective linear mapping. Then for every $\lambda_k$-measurable set $A \subset ®R^k$ it holds $H^k(L(A)) = \sqrt{\det(L^TL) \lambda_k(A)}$.

	\begin{dukazin}[$\dim L(®R^k) = k$]
		We find linear isometry $Q$ of $®R^k$ onto $L(®R^k)$, from last semester
		$$ H^k(L(A)) = H^k(Q^{-1} \circ L(A)) = \lambda^k(Q^{-1} \circ L(A)) = |\det(Q^{-1}L)|·\lambda_k(A). $$
		$$ (\det(Q^{-1}L))^2 = \det((Q^{-1} L)^T)·\det(Q^{-1} L) = \det((Q^{-1} L)^T·(Q^{-1} L)) = \det((\<Q^{-1} L e^i, Q^{-1} L^T e^j\>)_{i,j}). $$
		And because $Q$ is isometry ($\implies Q^{-1}$ is isometry), we can remove $Q^{-1}$ from scalar product and we get $\det(L^T L)$.
	\end{dukazin}
\end{lemma}

\begin{lemma}
	Let $k, n \in ®N$, $k ≤ n$, $G \subset ®R^k$ be an open set, $\phi: G \rightarrow ®R^n$ be an injective regular mapping, $x \in G$, and $\beta > 1$. Then there exists a neighbourhood $V$ of the point $x$ such that

	\begin{itemize}
		\item the mapping $y \mapsto \phi(\phi'(x)^{-1}(y))$ is $\beta$-Lipschitz on $\phi'(x)(V)$;
		\item the mapping $z \mapsto \phi'(x)(\phi^{-1}(z))$ is $\beta$-Lipschitz on $\phi(V)$.
	\end{itemize}

	\begin{dukazin}
		$x$, $\beta$ fixed. We know, that there exists $\eta > 0$ such that
		$$ \forall v \in ®R^k: \|\phi'(x)(v)\| ≥ \eta·\|v\|. $$
		We find $\epsilon \in \(0, \frac{1}{2}\eta\)$ such that $\frac{2\epsilon}{\eta} + 1 < \beta$. We find a neighbourhood $V$ of $x$ such that $\forall y \in V: \|\phi'(x) - \phi'(y)\| ≤ \epsilon$.

		We show that for every $u, v \in V$ we have
		$$ \|\phi(u) - \phi(v) - \phi'(x)(u - v)\| ≤ \epsilon \|u - v\|. $$
		Fix $v \in V$ and consider the mapping
		$$ g: w \mapsto \phi(w) - \phi(v) - \phi'(x)(w - v). $$
		For $w \in V$ we have $g'(w) = \phi'(w) - \phi'(x)$:
		$$ \|\phi(u) - \phi(v) - \phi'(x)(u - v)\| = \|g(w) - g(v)\| ≤ \sup \{\|g'(w)\|\ |\ w \in V\}·\|u - v\| ≤ \epsilon·\|u - v\|. $$
		
		Further we show that for every $u, v \in V$ we have
		$$ \|\phi(u) - \phi(v)\| ≥ \frac{1}{2} \eta \|u - v\|. $$
		For $u - v \in V$ we compute
		$$ \|\phi(u) - \phi(v)\| ≥ - \|\phi(u) - \phi(v) - \phi'(x)(u - v)\| + \|\phi'(x)(u - v)\| ≥ - \epsilon \|u - v\| + \eta \|u - v\| ≥ \frac{1}{2}\eta \|u - v\|. $$

		„First point“: TODO (řádek nebyl k přečtení)
		$$ \|\phi(\phi^{-1}(x)(a)) - \phi(\phi^{-1}(x)(b))\| = \|\phi(u) - \phi(v)\| ≤ $$
		$$ ≤ \|phi(u) - \phi(v) - \phi'(x)(u - v)\| + \|\phi'(x)(u - v)\| ≤ $$
		$$ ≤ \epsilon·\|u - v\| + \|\phi'(x)(y - v)\| ≤ \epsilon \frac{1}{\eta} \|a - b\| + \|a - b\| = \(\frac{\epsilon}{\eta} + 1\)\|a - b\| ≤ \beta·\|a - b\|. $$

		„Second point“: $k, q \in \phi(V)$. We find $u, v \in V$ such that $\phi(u) = p$ and $\phi(v) = q$:
		$$ \|\phi'(x)(\phi^{-1}(p)) - \phi'(x)(\phi^{-1}(q))\| = \|\phi'(x)(u) - \phi'(x)(v)\| = $$
		$$ = \|\phi'(x)(u - v)\| ≤ \|\phi(u) - \phi(v) - \phi'(x)(u - v)\| + \|\phi(u) - \phi(v)\| ≤ $$
		$$ ≤ \epsilon·\|u - v\| + \|p - q\| ≤ \frac{2\epsilon}{\eta}\|\phi(u) - \phi(v)\| + \|p - q\| = \(\frac{2\epsilon}{\eta} + 1\)\|p - q\| ≤ \beta \|p - q\|. $$
	\end{dukazin}
\end{lemma}

% 20. 02. 2023

\begin{lemma}
	Let $k, n \in ®N$, $k ≤ n$, $G \subset ®R^k$ be an open set, $\phi: G \rightarrow ®R^n$ be an injective regular mapping, $x \in G$, and $\alpha > 1$. Then there exists a neighbourhood of $x$ such that for every $\lambda^k$-measurable $E \subset V$ we have
	$$ \alpha^{-1} \int_E \vol \phi'(t) d\lambda^k(t) ≤ H^k(\phi(E)) ≤ \alpha \int_E \vol \phi'(t) d\lambda^k(t). $$

	\begin{dukazin}
		Find $\beta > 1$, $\tau > 1$ such that $\beta^k \tau < \alpha$. By previous lemma we find a neighbourhood $V_1$ of $x$ such that the conclusion of the lemma holds for $\beta$. We find a neighbourhood $V_2$ of $x$ such that
		$$ \forall t \in V_2: \tau^{-1} \vol \phi'(t) ≤ \vol \phi'(t) ≤ \tau \vol \phi'(x). $$
		Set $V = V_1 \cap V_2$.

		Assume that $E \subset V$ is a $\lambda^k$-measurable set. We have
		$$ \tau^{-1} \vol \phi'(x)·\lambda^k(E) ≤ \int_E \vol \phi'(t) d \lambda^k(t) ≤ \tau \vol \phi'(t) \lambda^k(E). $$
		By lemma above we have $\vol \phi'(t) \lambda^k(E) = H^k(\phi'(x)(E))$:
		$$ \tau^{-1} H^k(\phi'(x)(E)) ≤ \int_E \vol \phi'(t) d\lambda^k(t) ≤ \tau H^k(\phi'(x)(E)). $$

		By previous lemma we get 
		$$ H^k(\phi(E)) = H^k\(\(\phi \circ (\phi'(x))^{-1}\circ \phi'(x)\) (E)\) ≤ \beta^k H^k(\phi'(x)(E)) ≤ \beta^k H^k(\phi'(x)(E)) ≤ $$
		$$ ≤ \beta^k \tau \int_E \vol \phi'(t) d\lambda^k(t) ≤ \alpha \int_E \vol \phi'(t) d\lambda^k(t). $$

		By lemma above we get
		$$ H^k(\phi(E)) ≥ \beta^{-k} H^k\(\(\phi'(x) \circ \phi^{-1} \circ \phi\)(E)\) = \beta^{-k} H^k(\phi'(x)(E)) ≥ $$
		$$ ≥ \beta^{-k} \tau^{-1} \int_E \vol \phi'(t) d\lambda^k(t) ≥ \alpha^{-1} \int_E \vol \phi'(t) d\lambda^k(t). $$
	\end{dukazin}
\end{lemma}

\begin{veta}
	Let $k, n \in ®N$, $k ≤ n$, $G \subset ®R^k$ be an open set, $\phi: G \rightarrow ®R^n$ be an injective regular mapping and $f: \phi(G) \rightarrow ®R$ be $H^k$-measurable. Then we have
	$$ \int_{\phi(G)} f(x) dH^k(x) = \int_G f(\phi(t)) \vol \phi'(t) d\lambda^k(t), $$
	if the integral at the right side converges.

	\begin{dukazin}
		„$\phi^{-1}$ is well defined“: If $H \subset G$ is open, then we can write $H = \bigcup_{n=1}^∞ K_n$, where $K_n$ is compact for every $n \in ®N$. Then we have $\phi(H) = \bigcup_{n=1}^∞ \underbrace{\phi(K_n)}_{\text{compact}}$ is $F_\sigma$. This implies that $\phi^{-1}$ is Borel. The mappings $\phi$, $\phi^{-1}$ are locally Lipschitz by lemma above. ($\phi(G)$ is Borel.) $\phi(G)$ is $H^k$-$\sigma$-finite.

		1. „$f = \chi_L$, $L \subset \phi(G)$ is $H^k$-measurable“: We show $H^k(L) = \int_{\phi^{-1}(L)} \phi'(t) d\lambda^k(t)$. Choose $\alpha > 1$. By previous lemma we find for every $y \in G$ neighbourhood $V_y \subset G$ of the point $y$ such that for every $\lambda^k$-measurable set $E \subset V_y$ we have
		$$ \alpha^{-1} \int_E \vol \phi'(t) d\lambda^k(t) ≤ H^k(\phi(E)) ≤ \alpha \int_E \vol \phi'(t) d\lambda^k(t). $$
		We have $\subset\{V_y | y \in G\} = G$. There exists a sequence $\{y_j\}_{j=1}^∞$ such that $\bigcup_{i=1}^∞ V_{y_j} = G$. Using lemma from previous semester we find Borel sets $B, N \subset \phi(G)$ such that $B \subset L \subset B \cup N$, $H^k(N) = 0$.

		$\lambda^k(\phi^{-1}(N)) = 0$. $\phi^{-1}(B) \subset \phi^{-1}(L) \subset \phi^{-1}(B) \cup \phi^{-1}(N)$ $\implies$ $\phi^{-1}(L)$ is $\lambda^k$-measurable. We set
		$$ A_j = \phi^{-1}(L) \cap \(V_{y_j} \setminus \bigcup_{i=1}^{j-1} V_{y_j}\). $$
		Then we have

		\begin{itemize}
			\item $A_j$ is $\lambda^k$-measurable;
			\item $A_j \subset V_{y_j}$ for every $j \in ®N$;
			\item $\forall j, j' \in ®N, j ≠ j': A_j \cap A_{j'} = \O$;
			\item $\bigcup_{j=1}^∞ A_j = \phi^{-1}(L)$;
			\item for every $j \in ®N$ we have
				$$ \alpha^{-1} \int_{A_j} \vol \phi'(t) d\lambda^k(t) ≤ H^k(\phi(A_j)) ≤ \alpha \int_{A_j} \vol \phi'(t) d\lambda^k(t). $$
		\end{itemize}

		From all except for second point we have
		$$ \alpha^{-1} \int_{\phi^{-1}(L)} \vol \phi'(t) d\lambda^k(L) ≤ \underbrace{\sum_{j=1}^∞ H^k(\phi(A_j))}_{= H^k\(\bigcup_{j=1}^∞ \phi(A_j)\) = H^k(L)} ≤ \alpha \int_{\phi^{-1}(L)} \vol \phi'(t) d\lambda^k(t). $$

		2. „$f ≥ 0$ simple $H^k$-measurable“: From linearity of integrals. 3. „$f ≥ 0$ $H^k$-measurable“: we approximate $f$ by $0 ≤ f_j ≤ f_{j+1}$ simple functions and from Levi
		$$ \lim_{j \rightarrow ∞} \int_{\phi(G)} f_j(x) dH^k(x) = \int_{\phi(G)} f(x) dH^k(x), \qquad \lim_{j \rightarrow ∞} \int_G f_j(\phi(t))\vol\phi'(t) d\lambda^k(t) = \int_G f(\phi(t)) \vol \phi'(t) d\lambda^k(t). $$

		3. „$f$ $H^k$-measurable“: We add positive and negative part.
	\end{dukazin}
\end{veta}

% 27. 02. 2023

\begin{veta}[Coarea formula]
	Let $k, n \in ®N$, $k > n$, $\phi: ®R^k \rightarrow ®R^n$ be Lipschitz mapping, $f: ®R^k \rightarrow ®R$ be $\lambda^k$-integrable function. Then we have
	$$ \int_{®R^k} f(x) \sqrt{\det(\phi'(x)·(\phi'(x))^T)} d\lambda^k(x) = \int_{®R^n} \int_{\phi^{-1}(\{y\})} f(x)  dH^{k - n}(x) d\lambda^k(y) $$
\end{veta}

\begin{veta}
	Let $f: ®R^k \rightarrow ®R$ be $\lambda^k$-integrable function. Then we have
	$$ \int_{®R^k} f(x) d\lambda^k(x) = \int_0^∞ \(\int_{x \in ®R^k, \|x\| = z} f(x) dH^{k - 1}(x)\) d\lambda^1(z). $$

	\begin{dukazin}
		By Coarea formula.
	\end{dukazin}
\end{veta}

\section{Semicontinuous functions}
\begin{definice}
	Let $X$ be a topological space and $f: X \rightarrow ®R^*$. We say that $f$ is lower semicontinuous (lsc), if the set $\{x \in X | f(x) > a\}$ is open for every $a \in ®R$. We say that $f$ is upper semicontinuous (usc) if the set $\{x \in X | f(x) < a\}$ is open for every $a \in ®R$.
\end{definice}

\begin{tvrzeni}[Fact]
	$f: ®R \rightarrow ®R$:
	$$ f \text{ is lsc } \Leftrightarrow \forall x \in ®R: \liminf_{t \rightarrow x} f(t) ≥ x. $$
\end{tvrzeni}

\begin{veta}
	Let $X$ be a metrizable topological space and $f: X \rightarrow ®R^*$ be a function bounded from below. Then $f$ is lsc if and only if there exists a sequence $\{f_n\}$ of continuous functions from $X$ to ®R such that $f_0 ≤ f_1 ≤ …$ and $f_n \rightarrow f$.

	\begin{dukazin}
		„$\impliedby$“: Choose $a \in ®R$. Assume that $f(x_0) > a$. There exists $k \in ®N$ such that $f_k(x_0 > a)$. Then there is an open set $G \subset X$ such that $x_0 \in G$ and $f_k|_G > a$. Thus we have $f|_G ≥ f_k|_G > a$. So $\{x \in X | f(x) > a\}$ is open.

		„$\implies$“ The case „$f ≡ ∞$“: Then we consider $f_n ≡ n$. The case „$f \not≡ ∞$“. Fix a compatible metric $\rho$ on $X$. We set $f_n(x) = \inf \{f(y) + n·\rho(x, y) | y \in X\}$. Then we have $f_n: X \rightarrow ®R$ and $f_0 ≤ f_1 ≤ …$. We have
		$$ |f_n(x) - f_n(z)| ≤ n·\rho(x, z) \impliedby $$
		$$ \impliedby f_n(x) - f_n(z) ≤ f(y) + n·\rho(x, y) - (f(y) + n·\rho(y, z)) + \epsilon = n(\rho(x, y) - \rho(y, z)) + \epsilon ≤ n·\rho(x, z) + \epsilon. $$
		So $f_n$ is continuous.

		„$f_n \rightarrow f$“: There exists $K \in ®R$ such that $f(x) ≥ K$ for every $x \in X$. Fix $x \in X$. Choose $\epsilon > 0$. For every $n \in ®N$ we find $y_n \in X$ such that $f(y_n) ≤ f(y_n) + n·\rho(x, y_n) ≤ f_n(x) + \epsilon$. Then we have
		$$ \rho(x, y_n) ≤ \frac{1}{n}\(f_n(x) + \epsilon - f(y_n)\) ≤ \frac{1}{n} \(f_n(x) + \epsilon - K\). $$
		$f_n(x) \rightarrow ∞ \implies f(x) = ∞$, since $f_n(x) ≤ f(x)$. $f_n(x)$ is bounded $\implies$ $y_n \rightarrow x$, so we can find $n_0 \in ®N$ such that $\forall n ≥ n_0: f(y_n) > f(x) - \epsilon$. Then we have $f(x) < f(y_n) + \epsilon ≤ f_n(x) + 2\epsilon$, $\lim f_n(x) ≤ f(x) ≤ \lim f_n(x) + 2\epsilon$, thus $\lim f_n(x) = f(x)$.
	\end{dukazin}
\end{veta}

\section{Function of Baire class 1}
\begin{definice}
	Let $X$ and $Y$ be metrizable topological spaces, a function $f: X \rightarrow Y$ is of Baire class 1 ($B_1$-function) if for every open set $U \subset Y$ the set $f^{-1}(U)$ is $F_\sigma$.
\end{definice}

\begin{veta}[Lebesgue–Hasudorff–Banach]
	Let $X$ be a metrizable topological space and $f: X \rightarrow ®R$ be a $B_1$–function. Then there exists a sequence $\{f_n\}$ of continuous functions from $X$ to ®R with $f_n \rightarrow f$.
\end{veta}

\begin{lemma}
	Let $X$ be a metrizable topological space and $A \subset X$ be $G_\delta$ and $F_\sigma$. Then $\chi_A$ is point-wise limit of a sequence of continuous functions.

	\begin{dukazin}
		$A = \bigcup_{n \in ®N} F_n$, $X \setminus A = \bigcup_{n \in ®N} H_n$, $F_n \subseteq F_{n+1}$, $H_n \subseteq H_{n+1}$. By Urysohn lemma there exists continuous function $f_n: X \rightarrow [0, 1]$ such that $f_n|_{H_n} = 0$ and $f_n|_{F_n} = 1$. Then $f_n(x) \rightarrow f(x)$.
	\end{dukazin}
\end{lemma}

% 06. 03. 2023

\begin{lemma}
	Let $X$ be a metrizable topological space, $p_n: X \rightarrow ®R$, $n \in \omega$, be a point-wise limit of a sequence of continuous functions. If the sequence $\{p_n\}$ converges uniformly to $p$, then $p$ is point-wise limit of continuous functions.

	\begin{dukazin}
		Claim: If $q_n: X \rightarrow ®R$, $n \in \omega$, is point-wise limit of continuous functions, $\|q_n\|_∞ ≤ 2^{-n}$, then $\sum_{n=0}^∞ q_n$ is a point-wise limit of continuous functions.

		Corollary: One can assume $\|p - p_n\|_∞ ≤ 2^{-(n+1)}$. $p = p_0 + \sum_{n=0}^∞ (p_{n+1} - p_n)$
		$$ \|p_{n+1} - p_n\|_∞ ≤ \|p_{n+1} - p\| + \|p - p_n\| < 2^{-(n+2)} + 2^{-(n+1)} < 2^{-n}. $$

		Proof of claim: For every $n \in \omega$, there exists a sequence of continuous functions $\{q_i^n\}_{i=0}^∞$ such that $q_i^n \rightarrow q_n$ and moreover we may assume $\|q_i^n\|_∞ ≤ 2^{-n}$. We set $r_i = \sum_{n=0}^∞ q_i^n$. The sum converges uniformly, so $r_i$ is continuous for every $i \in \omega$.

		Set $x \in X$ and $\epsilon > 0$. We find $N \in \omega$ such that
		$$ \left|\sum_{n=N+1}^∞ q_i^n(x)\right| < \frac{1}{2}\epsilon, \left|\sum_{n=N+1}^∞ q_n(x)\right| < \frac{1}{2}\epsilon. $$
		Then we have
		$$ \left|r_i(x) - \sum_{n=0}^∞ q_n(x)\right| = \left|\sum_{n=0}^∞ q_i^n(x) - \sum_{n=0}^∞ q_n(x)\right| ≤ \left|\sum_{i=0}^N q_i^n(x) - q_n(x)\right| + \left|\sum_{n=N+1}^∞ q_i^n(x) - \sum_{n=N+1}^∞ q_n(x)\right| ≤ \left|\sum_{n=0}^N (q_i^n(x)) - q_n(x) \right| + \epsilon. $$
		$$ \limsup_{i \rightarrow ∞} |r_i(x) \sum_{n=0}^∞ q_n(x)| ≤ \epsilon \implies r_i(x) \rightarrow \sum_{n=0}^∞ q_n(x). $$
	\end{dukazin}
\end{lemma}

\begin{lemma}[Reduction theorem for $F_\sigma$ sets]
	Let $X$ be a metrizable topological space, $A_n \subset X$ be an $F_\sigma$ set for every $n \in ω$. Then there are $F_ς$ sets $A_n^* \subset A_n$, such that $A_n^* \cap A_m^* = \O$, whenever $n, m \in ω$, $n ≠ m$, and $\bigcup_{n=0}^∞ A_n = \bigcup_{n=0}^∞ A_n^*$.

	\begin{dukazin}
		$A_n = \bigcup_{j=0}^∞ A_{n, j}$, $A_{n, j}$ is closed. $k \mapsto (k', k'')$ bijection of $ω$ onto $ω\times ω$. $Q_k = A_{(k)_0, (k)_j} \setminus \bigcup_{l < k} A_{(l)_0, (k)_1}$. $(Q_k)_{k \in ω}$ is sequence of $F_ς$ sets, which is disjoint. $A_n^* := \bigcup \{Q_k | (k)_0 = n\} \subseteq A_n$ is $F_ς$ set, $A_n^* \cap A_m^* = \O$ if $n ≠ m$ and $\bigcup_{n=0}^∞ A_n^* = \bigcup_{k=0}^∞ Q_k = \bigcup_{n=0}^∞ A_n$.
	\end{dukazin}
\end{lemma}

\begin{dukaz}[Of Lebesgue–Hasudorff–Banach theorem]
	It is sufficient to prove result for $g: X \rightarrow (0, 1)$. Because if $f \in B_1$, then we set $g = k \circ f$ where $k: ®R \rightarrow (0, 1)$ is homeomorphism. We find $g_n: X \rightarrow ®R$, continuous and $g_n \rightarrow g$. $\tilde g_n := \min\{\max\{\frac{1}{n}, g_n\}, 1 - \frac{1}{n}\}$. $\tilde g_n(X) \subset (\frac{1}{n}, 1 - \frac{1}{n})$.

	Let $g: X \rightarrow (0, 1)$ be $B_1$. For $N \in ω$, $N ≥ 2$, and $i \in [N-2]$ we set
	$$ A_i^N := g^{-1}\(\frac{i}{N}, \frac{i+2}{n}\) … F_ω, \qquad \bigcup_{i=0}^{N-2} A_i^N = X. $$

	$B_i^N \subset A_i^N$ such that $\bigcup_{i=0}^{N-2} B_i^N = X$, $B_i^N$ is $F_ς$ and $B_i^N \cap B_{i'}^N = \O$, whenever $i ≠ i'$. $g_N(x) := \sum_{i=0}^{N-2} \frac{1}{N} \chi_{B_i^n}(x)$. $g_N \rightrightarrows g$ ($\|g - g_N\|_∞ ≤ \frac{2}{N}$).
\end{dukaz}

\begin{veta}[Baire]
	Let $X$ be a metrizable topological space, $Y$ be separable metrizable topological space, and $f: X \rightarrow Y$ be $B_1$-function. Then the set of points of continuity of $f$ is $G_δ$ and residual.

	\begin{dukazin}
		$\{V_n\}$ open countable basis of $Y$. $f$ is not continuous at $x$ $\Leftrightarrow$ $\exists n \in ω: x \in f^{-1}(V_n) \setminus \Int f^{-1}(V_n)$. $D(f) = \{x \in X | f \text{ is not continuous at $x$}\} = \bigcup_{n \in ω} \underbrace{(f^{-1}(V_n) \Int f^{-1}(V_n))}_{\in F_ω}$.

		$B = (f^{-1}(V_n) \Int f^{-1}(V_n)) = \bigcup_{k \in ω} F_{n, k}$ is closed and $\Int F_{n, k} = \O$, so $F_{n, k}$ is nowhere dense. So $B$ is meager. And complement of meager is residual.
	\end{dukazin}
\end{veta}

\end{document}
