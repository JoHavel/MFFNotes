\documentclass[12pt]{article}					% Začátek dokumentu
\usepackage{../../MFFStyle}					    % Import stylu
\usetikzlibrary{arrows}

\begin{document}
\definecolor{xfqqff}{rgb}{0.4980392156862745,0.,1.}
\definecolor{qqffqq}{rgb}{0.,1.,0.}
\definecolor{ffffqq}{rgb}{1.,1.,0.}
\definecolor{qqqqff}{rgb}{0.,0.,1.}
\definecolor{ttzzqq}{rgb}{0.2,0.6,0.}
\definecolor{qqffff}{rgb}{0.,1.,1.}
\definecolor{ffxfqq}{rgb}{1.,0.4980392156862745,0.}
\definecolor{yqqqqq}{rgb}{0.5019607843137255,0.,0.}
\definecolor{ffqqqq}{rgb}{1.,0.,0.}
\definecolor{bfffqq}{rgb}{0.7490196078431373,1.,0.}
\definecolor{ffxfqq}{rgb}{1.,0.4980392156862745,0.}
\begin{priklad}[1]
    Nakreslete na torus $K_n$ pro co největší $n$.

    \begin{reseni}
        Nakreslíme si torus topologicky jako čtyřúhelník se správně ztotožněnými protějšími stranami. A zkoušíme. Dojdeme např. k:

        \noindent\hfill
        \begin{tikzpicture}[line cap=round,line join=round,>=triangle 45,x=1.0cm,y=1.0cm]
                \clip(-1.,-1.) rectangle (7.,7.);
                \draw [->,line width=2.8pt] (0.,0.) -- (6.,0.);
                \draw [->,line width=2.8pt] (0.,0.) -- (0.,6.);
                \draw [->,line width=2.8pt] (0.,6.) -- (6.,6.);
                \draw [->,line width=2.8pt] (6.,0.) -- (6.,6.);
                \draw [line width=2.pt] (1.,3.)-- (2.,5.);
                \draw [line width=2.pt] (2.,5.)-- (4.,5.);
                \draw [line width=2.pt] (4.,5.)-- (5.,3.);
                \draw [line width=2.pt] (5.,3.)-- (4.,1.);
                \draw [line width=2.pt] (4.,1.)-- (2.,1.);
                \draw [line width=2.pt] (2.,1.)-- (1.,3.);
                \draw [line width=2.pt] (2.,1.)-- (3.,3.);
                \draw [line width=2.pt] (3.,3.)-- (1.,3.);
                \draw [line width=2.pt] (2.,5.)-- (3.,3.);
                \draw [line width=2.pt] (4.,5.)-- (3.,3.);
                \draw [line width=2.pt] (5.,3.)-- (3.,3.);
                \draw [line width=2.pt] (3.,3.)-- (4.,1.);
                \draw [line width=2.pt,color=ffqqqq] (5.,3.)-- (6.,3.);
                \draw [line width=2.pt,color=ffqqqq] (1.,3.)-- (0.,3.);
                \draw [line width=2.pt,color=yqqqqq] (2.,1.)-- (2.,0.);
                \draw [line width=2.pt,color=yqqqqq] (2.,6.)-- (2.,5.);
                \draw [line width=2.pt,color=ffxfqq] (4.,5.)-- (4.,6.);
                \draw [line width=2.pt,color=ffxfqq] (4.,1.)-- (4.,0.);
                \draw [line width=2.pt,color=qqffff] (2.,1.)-- (3.,0.);
                \draw [line width=2.pt,color=qqffff] (3.,6.)-- (4.,5.);
                \draw [line width=2.pt,color=ttzzqq] (4.,1.)-- (6.,0.);
                \draw [line width=2.pt,color=ttzzqq] (0.,6.)-- (2.,5.);
                \draw [line width=2.pt,color=qqqqff] (4.,1.)-- (5.,0.);
                \draw [line width=2.pt,color=qqqqff] (5.,6.)-- (6.,5.);
                \draw [line width=2.pt,color=qqqqff] (0.,5.)-- (1.,3.);
                \draw [line width=2.pt,color=ffffqq] (2.,1.)-- (0.,2.);
                \draw [line width=2.pt,color=ffffqq] (6.,2.)-- (5.,3.);
                \draw [line width=2.pt,color=qqffqq] (4.,5.)-- (6.,4.);
                \draw [line width=2.pt,color=qqffqq] (0.,4.)-- (1.,3.);
                \draw [line width=2.pt,color=xfqqff] (2.,5.)-- (1.16,6.);
                \draw [line width=2.pt,color=xfqqff] (1.,0.)-- (0.,1.);
                \draw [line width=2.pt,color=xfqqff] (6.,1.)-- (5.,3.);
                \begin{scriptsize}
                        \draw [fill=black] (2.,1.) circle (2.5pt);
                        \draw [fill=black] (4.,1.) circle (2.5pt);
                        \draw [fill=black] (1.,3.) circle (2.5pt);
                        \draw [fill=black] (5.,3.) circle (2.5pt);
                        \draw [fill=black] (4.,5.) circle (2.5pt);
                        \draw [fill=black] (2.,5.) circle (2.5pt);
                        \draw [fill=black] (3.,3.) circle (2.5pt);
                \end{scriptsize}
        \end{tikzpicture}
        \hfill\hfill\  
    \end{reseni}
\end{priklad}

\pagebreak

\begin{priklad}[2]
    Nakreslete na torus graf, jehož všechny stěny budou ohraničeny čtyřcykly, ale jeho barevnost bude alespoň 3.

    \begin{reseni}
        Víme, že pokud bude graf obsahovat cyklus liché délky, pak jeho barevnost bude alespoň 3. Čtyřcykly ohraničené stěny má například čtvercová mřížka, tedy nakreslíme si torus zase jako čtyřúhelník a vložíme do něj čtvercovou mřížku „liché“ šířky (tj. vodorovná „přímka“ je cyklus liché délky), například:

        \noindent\hfill
        \begin{tikzpicture}[line cap=round,line join=round,>=triangle 45,x=1.0cm,y=1.0cm]
                \clip(-1.,-1.) rectangle (7.,7.);
                \draw [->,line width=2.8pt] (0.,0.) -- (0.,6.);
                \draw [->,line width=2.8pt] (0.,0.) -- (6.,0.);
                \draw [->,line width=2.8pt] (6.,0.) -- (6.,6.);
                \draw [->,line width=2.8pt] (0.,6.) -- (6.,6.);
                \draw [line width=2.pt] (1.,1.)-- (1.,3.);
                \draw [line width=2.pt,color=qqffqq] (1.,3.)-- (0.,3.);
                \draw [line width=2.pt,color=ffqqqq] (1.,1.)-- (0.,1.);
                \draw [line width=2.pt] (1.,1.)-- (3.,1.);
                \draw [line width=2.pt] (3.,1.)-- (3.,3.);
                \draw [line width=2.pt] (3.,3.)-- (1.,3.);
                \draw [line width=2.pt] (3.,1.)-- (5.,1.);
                \draw [line width=2.pt] (5.,1.)-- (5.,3.);
                \draw [line width=2.pt] (5.,3.)-- (3.,3.);
                \draw [line width=2.pt,color=qqffqq] (5.,3.)-- (6.,3.);
                \draw [line width=2.pt,color=ffqqqq] (5.,1.)-- (6.,1.);
                \draw [line width=2.pt] (1.,3.)-- (1.,5.);
                \draw [line width=2.pt,color=qqqqff] (1.,5.)-- (0.,5.);
                \draw [line width=2.pt] (1.,5.)-- (3.,5.);
                \draw [line width=2.pt] (3.,5.)-- (3.,3.);
                \draw [line width=2.pt] (3.,5.)-- (5.,5.);
                \draw [line width=2.pt] (5.,5.)-- (5.,3.);
                \draw [line width=2.pt,color=qqqqff] (5.,5.)-- (6.,5.);
                \draw [line width=2.pt,color=qqffff] (5.,5.)-- (5.,6.);
                \draw [line width=2.pt,color=bfffqq] (3.,5.)-- (3.,6.);
                \draw [line width=2.pt,color=ffxfqq] (1.,5.)-- (1.,6.);
                \draw [line width=2.pt,color=ffxfqq] (1.,1.)-- (1.,0.);
                \draw [line width=2.pt,color=bfffqq] (3.,1.)-- (3.,0.);
                \draw [line width=2.pt,color=qqffff] (5.,1.)-- (5.,0.);
                \begin{scriptsize}
                        \draw [fill=black] (1.,1.) circle (2.5pt);
                        \draw [fill=black] (1.,3.) circle (2.5pt);
                        \draw [fill=black] (3.,1.) circle (2.5pt);
                        \draw [fill=black] (3.,3.) circle (2.5pt);
                        \draw [fill=black] (5.,1.) circle (2.5pt);
                        \draw [fill=black] (5.,3.) circle (2.5pt);
                        \draw [fill=black] (1.,5.) circle (2.5pt);
                        \draw [fill=black] (3.,5.) circle (2.5pt);
                        \draw [fill=black] (5.,5.) circle (2.5pt);
                \end{scriptsize}
        \end{tikzpicture}
        \hfill\hfill\ 
    \end{reseni}
\end{priklad}

\begin{priklad}[3]
    Dokažte větu o 4 barvách pro rovinné grafy bez trojúhelníku.

    \begin{reseni}
        Na přednášce jsme si ukazovali, jak z Eulerovy věty dokázat, že pro rovinný graf bez trojúhelníků platí $|E|<2|V| - 4$, tudíž každý rovinný graf bez trojúhelníků obsahuje vrchol stupně menšího $2·2 = 4$. Tedy obdobně jako ve větě o 5 barvách pokud vrcholů bude nejvýše 4, tak je obarvení triviální. Jinak odebereme vrchol stupně nejvýše 3, ten, jak už víme existuje, a obarvíme zbytek grafu podle indukčního předpokladu. Následně tento vrchol přidáme, a jelikož měl nejvýše 3 sousedy, tj. 3 sousední barvy, máme alespoň 1 barvu volnou pro tento vrchol.
    \end{reseni}
\end{priklad}
\end{document}
