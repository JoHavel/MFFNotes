\documentclass[12pt]{article}					% Začátek dokumentu
\usepackage{../../MFFStyle}					    % Import stylu

\parskip = 0.8\parskip

\begin{document}

\begin{priklad}[1]
    Na množině ®N najděte co nejvíce vzájemně neizomorfních lineárních uspořádání.

    \begin{reseni}[Nespočetně mnoho]
        Další 4 odstavce pouze popisují konstrukci, pátý výslednou strukturu těchto uspořádání.\looseness = -1

        \looseness = -1 Množinu ®N rozdělíme na nekonečně mnoho nekonečných množin (například podle nejmenšího prvočíselného dělitele, 1 přidáme k libovolné z nich, třeba k sudým číslům). Jednu z těchto množin vybereme jako „číslicovou“, zbylé budou „oddělovače“.

        V každé z těchto zbylých množin vytvoříme lineární uspořádání, ve kterém má každý prvek svého předchůdce a následovníka\footnote{Speciálně tedy nemá žádný minimální ani maximální prvek. Také se dá popsat tak, že je izomorfní s uspořádáním $≤$ na množině ®Z.} (např. každé druhé číslo změníme na záporné a uspořádáme pomocí $≤$ zúženého na tuto množinu). Zároveň uspořádáme tyto množiny tak, že mají minimální nebo maximální prvek, nechť třeba minimální (např. uspořádám podle prvočísel, podle kterých jsem tyto množiny vybral).

Nyní uspořádejme i „číslicovou“ množinu tak, aby měla buď minimum, nebo maximum a každý prvek krom tohoto maxima / minima měl svého předchůdce a následníka, z m / m vyberme třeba minimum. Následně tuto množinu rozdělíme na nekonečně neprázdných\footnote{Tato podmínka není nutná, ale špatně by se ošetřovalo, že jsou všechny prázdné…} disjunktních podmnožin tak, aby v každé této podmnožině byl právě jeden prvek, který v ní nemá předchůdce, a právě jeden, který v ní nemá následovníka.

        Nakonec vytvoříme posloupnost, kde první prvek bude nejmenší „číslicová množina“, druhý prvek nejmenší „oddělovač“, každý další lichý bude následovník předchozího lichého a každý další sudý bude následovník předchozího sudého. Řekněme, že všechny prvky množiny, která je v posloupnosti dříve, jsou menší než všechny prvky množiny, která je později. Tak jsme konečně uspořádali všechny prvky ®N.

        Dostáváme tím ®N s uspořádáním, kde když odebereme $a_0 \in ®N$ nejmenších prvků, dostaneme množinu bez minima. Následně když odebereme podmnožinu, jejíž všechny prvky jsou menší než prvky, které do ní nepatří, a jejíž uspořádání je izomorfní s $≤$ na ®Z, dostaneme množinu s minimem. Když následně odebereme $a_1 \in @N$ dostaneme zase množinu bez minima atd.

        Dostáváme tedy posloupnost $\{a_n\}_{n=1}^∞$, které nám popisuje jednu ekvivalenční třídu (vzhledem k izomorfizmu) uspořádání a zjevně pro dvě různé posloupnosti je to jiná třída. Vzhledem k tomu, že můžeme ze všech „číselných“ množin od nějakého $1≠n_0 \in ®N$ odebrat nejmenší prvek a přidat ho k předchozí množině, umíme vytvořit libovolnou posloupnost, tedy každé nekonečné posloupnosti přirozených čísel lze přiřadit jedno takové uspořádání.

        Posloupností na přirozených číslech je nespočetně mnoho. Pokud to nevíme, lze si vybrat jen posloupnosti, kde $\forall 1≠n \in ®N: a_n≤10$. První člen nám pak reprezentuje celou část reálného čísla z intervalu $(0, ∞)$ a zbytek zápis desetinné části, kde cifry jsou o 1 vyšší (z množiny $\{1, 2, …, 10\}$). A jelikož každé reálné číslo z intervalu $(0, ∞)$ lze reprezentovat takto nějakou posloupností přirozených čísel menších rovno 10, nalezli jsme bijekci z vybraných posloupností do intervalu $(0, ∞)$, který má jistě nespočetnou velikost.\looseness = -1

    \end{reseni}

\end{priklad}

\pagebreak

\begin{priklad}[2]
    Pro každou funkci $f: X \rightarrow Y$, kde $X$ a $Y$ jsou neprázdné množiny, ukažte, že relace $R$ na množině $X$ definovaná jako $xRy \Leftrightarrow f(x) = f(y)$ je ekvivalence.

    \begin{reseni}
        Vyjdeme z definice, že ekvivalence je relace, která je reflexivní, tranzitivní a symetrická. Reflexivní zřejmě je, neboť $f(x) = f(x)$ (funkce přiřazují každému vzoru právě jeden obraz). Symetrická je také, neboť $=$ je symetrické, tedy $f(x) = f(y) \Leftrightarrow f(y) = f(x)$. Zároveň $=$ je navíc i tranzitivní, tudíž $f(x) = f(y) \land f(y) = f(z) \implies f(x) = f(z)$, tedy i relace $R$ je tranzitivní. $R$ má všechny vlastnosti z definice ekvivalence, tedy $R$ je ekvivalence.
    \end{reseni}
\end{priklad}

\pagebreak

\begin{priklad}[3]
        Vyvraťte následující tvrzení: Mějme částečné uspořádání $\preceq$ a lineární uspořádání $≤$. Pak relace $$\trianglelefteq: a \trianglelefteq b \Leftrightarrow a \preceq b \lor \(\(a\text{ a }b\text{ jsou neporovnatelné pomocí }\preceq\) \land a≤b \)$$ je nutně také lineární uspořádání.

        \begin{reseni}
            Ukážeme, že $\trianglelefteq$ nemusí být tranzitivní, tedy to vůbec nemusí být uspořádání, natož lineární uspořádání. Nechť obě uspořádání jsou na množině $\{1, 2, 3\}$, $≤$ definované jako $≤$ na $®R$ zúžené na naší množinu, $\preceq$ definované jako $\{(3, 1), (1, 1), (2, 2), (3, 3)\}$.

            To znamená, že $1 \trianglelefteq 2 \trianglelefteq 3$ (jelikož ty $\preceq$ neporovnává, takže používáme $≤$), ale není $3 \trianglelefteq 1$, jelikož zrovna dvojici $(1, 3)$ $\preceq$ porovnává a říká opak ($3\preceq 1$).
        \end{reseni}
\end{priklad}

\pagebreak

\begin{priklad}[4]
    Mějme množinu $\{1, 2, …, 100\}^2$ uspořádanou tak, že $(a, b)\preceq(c, d) \Leftrightarrow a≤c \land b≤d$.
    
    \begin{reseni}[Řetězec]
            Jeden z množiny nejdelších řetězců je $\{(1, y)| y \in \{1, …, 100\}\} \cup \{(x, 100)| x \in \{1, …, 100\}\}$ a má délku 199 (zřejmě sjednocuji 2 množiny se 100 prvky, které mají právě jeden společný: $(1, 100)$). Řetězec je to zjevně, jelikož prvky pravé množiny ze sjednocení jsou větší než z levé ($x≥1$ a $y≤100$) a obě množiny mají jedno z čísel konstantní, tedy se porovnávají lineárním uspořádáním „$≤$“. Zbývá dokázat, že neexistuje delší.

        Delší řetězec neexistuje, jelikož když seřadím řetězec od nejmenšího prvku po největší, každý následující prvek musí zvýšit součet čísel alespoň o jedna a nejmenší součet množiny ze zadání je $1+1 = 2$ a největší $100+100 = 200$, tedy takových „posunů“ lze udělat maximálně $200-2 = 198$, tedy řetězec může mít maximálně 199 prvků.

    \end{reseni}

    \begin{reseni}[Antiřetězec]
        Naopak nejdelší antiřetězec je např. $\{(x, 100-x)| x \in \{1, …, 100\}\}$. Antiřetězec je, neboť $\forall x, y \in \{1, …, 100\}$ buď $x=y$, nebo $x<y$, ale potom $100-x>100-y$, nebo $y<x$, ale potom $100-x<100-y$. Navíc je délky 100, tedy stačí ukázat, že 100 je maximum.

        Delší antiřetězec neexistuje, protože $\forall x \in \{1, …, 100\}$ jsou všechny dvojice tvaru $(x, y),y \in \{1, …, 100\}$ porovnatelné, tedy $\forall x$ můžeme zvolit nejvýše 1 $y$. A $x$ nabývá 100 hodnot, tedy můžeme zvolit nejvýše 100 neporovnatelných prvků.
    \end{reseni}
\end{priklad}


\end{document}
