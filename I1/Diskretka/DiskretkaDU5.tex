\documentclass[12pt]{article}					% Začátek dokumentu
\usepackage{../../MFFStyle}					    % Import stylu


\begin{document}

\begin{priklad}[1]
    Na konferenci potkal matematik 5 svých dobrých známých. Jelikož program byl bohatý, setkávali se pouze u obědů. Kolik dní trvala konference, pokud:
        
    \begin{enumerate}
        \item s každým jednotlivcem obědval 10 krát,
        \item s každou dvojicí 5 krát,
        \item s každou trojicí 3 krát,
        \item s každou čtvericí 2 krát,
        \item s celou pěticí právě jednou,
        \item vždy obědval alespoň s jedním z těchto pěti kamarádů.
    \end{enumerate}

    \begin{reseni}
        Rozmyslíme si, kolikrát obědval přesně s 1, 2, 3, 4, 5 lidmi (s více nemohl, protože jich tam více není, s 0 nemohl podle bodu 6.). Začal bych s pěticí, protože s pěticí se mohl setkat jen jedním způsobem. Nechť se tedy 1 den setkal se všemi 5 (1 den).

        Následně mě zajímá, kolikrát se setkal se čtveřicí. S každou čtveřicí už se setkal, když se setkal se všemi, tedy s každými právě 4 se potká 1 (5 dní).

        S každou trojicí se setkal již jednou při setkání s pěticí a dvakrát ve čtveřici (s každým z dalších dvou jednou), tedy s právě 3 neobědval (0 dní).

        S dvěma obědval jednou za pětici a třikrát za čtveřici (za každého chybějícího z ostatních jednou), tedy s každými přesně 2 obědval 1 ($\frac{5·4}{2}=10$ dní).

        Ve všech předchozích případech se dohromady potkal s $1·5 + 5·4 + 0·3 + 10 · 2 = 45$ obědolidmi. Ale s každým obědval desetkrát, což je 50 obědolidí, tedy musel ještě 5 krát obědvat s právě jedním člověkem (5 dní).

        Dohromady tedy $1+5+0+10+5 = 21$ dní.
    \end{reseni}

\end{priklad}

\pagebreak

\begin{priklad}[2]
    Ve volbách se o post prezidenta ucházeli Alice, Bob a Charlie. Dezinformační web přinesl zprávu, že:

    \begin{enumerate}
        \item 65 procent voličů by bylo spokojeno, kdyby byla zvolena Alice,
        \item 57 procent voličů by bylo spokojeno, kdyby byl zvolena Bob,
        \item 58 procent voličů by bylo spokojeno, kdyby byl zvolena Charlie,
        \item 28 procent voličů by bylo spokojeno, kdyby byl zvolen kdokoli z dvojice Alice, Bob,
        \item 30 procent voličů by bylo spokojeno, kdyby byl zvolen kdokoli z dvojice Alice, Charlie,
        \item 27 procent voličů by bylo spokojeno, kdyby byl zvolen kdokoli z dvojice Bob, Charlie,
        \item 12 procent voličů bude s prezidentem spokojeno, ať bude zvolen kterýkoli kandidát.
    \end{enumerate}

    David chtěl spočítat, kolik procent voličů nebude spokojeno s žádným z možných výsledků voleb, ale dospěl k názoru, že neumí počítat. Ukažte, že chyba nebyla (jen) na jeho straně a že web, který tuto zprávu přinesl, je skutečně dezinformační. 


    \begin{reseni}
        „spokojeno, kdyby byl zvolen kdokoli z dvojice / kterýkoli“ značí přesně průnik voličů jednotlivých lidí. Tedy počet voličů, kteří budou spokojeni z nějakého výsledku voleb je podle PIE: $$\underbrace{65 + 57 + 58}_\text{Voliči jednotlivých lidí.} \underbrace{- 28 - 30 - 27}_\text{Průniky voličů 2 lidí.} \underbrace{+ 12}_\text{Průnik voličů 3 lidí.} = \underbrace{107\%}_\text{Sjednocení lidí.}$$, což je jaksi více než kolik je voličů. Tudíž web je jistě dezinformační (nebo idnes.cz).
    \end{reseni}
\end{priklad}

\pagebreak

\begin{priklad}[3]
    Nechť $M$ je množina přirozených čísel menších nebo rovných 4200, která jsou dělitelná 2, 3 nebo 7. Každý z vás si jistě dokáže programem na pět řádků včetně výpisu spočítat, že součet čísel v množině $M$ je 6302100. Dokážete to ale spočítat i bez počítače pomocí PIE?

    \begin{reseni}
        Ano.

        \baselineskip=1.5em

        Důkaz: Čísel dělitelných 2 menších než 4200 je $\frac{4200}{2} = 2100$, tedy jejich součet je $2·\frac{2100·2101}{2}$, dělitelných 3 je $\frac{4200}{3} = 1400$, tedy součet $3·\frac{1400 · 1401}{2}$, dělitelných 7 je $\frac{4200}{7} = 600$, součet $7·\frac{600·601}{2}$.

        Pokud je ale sečteme, tak jsme všechna dělitelná 6, 14 nebo 21 započítali dvakrát, tedy odečteme $\frac{4200}{6} = 700$, součet $6·\frac{700·701}{2}$, $\frac{4200}{14} = 300$, součet $14·\frac{300·301}{2}$, a $\frac{4200}{21} = 200$, součet $21·\frac{200·201}{2}$.

        Nyní ale zase nepočítáme ty dělitelná 42, tedy přičteme $\frac{4200}{42} = 100$, součet $42·\frac{100·101}{2}$. Tedy výsledek je:
$$ \frac{2·2100·2101 + 3·1400·1401 + 7·600·601}{2} +$$  $$ + \frac{- 6·700·701 - 14·300·301 - 21·200·201 + 100·101·42}{2} = 6302100.$$ 
        
    \end{reseni}

\end{priklad}


\end{document}
