\documentclass[12pt]{article}					% Začátek dokumentu
\usepackage{../../MFFStyle}					    % Import stylu



\begin{document}
    
\begin{priklad}
    Dokažte výroky "Má-li relace $R$ na konečné množině vlastnost $X$, pak má inverzní relace $R^{-1}$ už nutně také vlastnost $X$," kde $X \in \{ \text{reflexivita, tranzitivita, symetrie, antisymetrie}\}$. Pokud nevíte jak na to, nebojte – podobné příklady budeme dělat na začátku příštího cvika.

    Zapišme si definici $R^{-1}$ takto: $ \forall x,y: xRy \Leftrightarrow yR^{-1}x$. A pomocí ní dokážeme, že definice vlastnosti $X$ pro $R$ je ekvivalentní definici vlastnosti $X$ pro $R^{-1}$.

    \begin{dukazin}[Reflexivita]
            $$ (\forall x: xRx) \overset{\raise 3pt \hbox{\framebox{\text{\tiny Z definice $R^{-1}$ $xRx \Leftrightarrow xR^{-1}x$}}}}{\Leftrightarrow} (\forall x: xR^{-1}x) $$
    \end{dukazin}

    \begin{dukazin}[Tranzitivita]
            $$ \forall x, y, z: xRy \land yRz \implies xRz \overset{\raise 3pt \hbox{\framebox{\text{\tiny Z definice $R^{-1}$}}}}{\Leftrightarrow} \forall x, y, z: yR^{-1}x \land zR^{-1}y \implies zR^{-1}xi $$
    $$ \overset{\raise 3pt \hbox{\framebox{\text{\tiny Přeznačíme $x \leftrightarrow z$}}}}{\Leftrightarrow} \forall x, y, z: yR^{-1}z \land xR^{-1}y \implies xR^{-1}z $$
    \end{dukazin}

    \begin{dukazin}[Symetrie]
            $$ \forall x, y: xRy \implies yRx \overset{\raise 3pt \hbox{\framebox{\text{\tiny Z definice $R^{-1}$}}}}{\Leftrightarrow} \forall x, y: yR^{-1}x \implies xR^{-1}y $$
        $$ \overset{\raise 3pt \hbox{\framebox{\text{\tiny Přeznačíme $x \leftrightarrow y$}}}}{\Leftrightarrow} \forall x, y: xR^{-1}y \implies yR^{-1}x$$ 
    \end{dukazin}

    \begin{dukazin}[Antisymetrie]
            $$ \forall x, y: xRy \land yRx \implies x = y \overset{\text{\raise 3pt \hbox{\framebox{\tiny Z definice $R^{-1}$}}}}{\Leftrightarrow} \forall x, y: yR^{-1}x \land xR^{-1}y \implies x = y$$
    \end{dukazin}
\end{priklad}


\end{document}
