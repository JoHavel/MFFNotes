\documentclass[12pt]{article}					% Začátek dokumentu
\usepackage{../../MFFStyle}					    % Import stylu

\begin{document}

\begin{priklad}[1]
    Lovci Antonín, Bedřich a Cecil spolu loví. Antonín umí zasáhnout zajíce s pravděpodobností 0.2, Bedřich 0.4 a Cecil 0.6. V jednu chvíli všichni tři zamířili na toho samého zajíce a vystřelili. Jakou mají jednotliví lovci podmíněnou pravděpodobnost, že zasáhli cíl, pokud víme, že zajíce zasáhl právě jeden z nich?

    \begin{reseni}
        Nejdříve spočítáme pravděpodobnosti, že zasáhl právě Antonín $P(A)$, právě Bedřich $P(B)$ a právě Cecil $P(C)$. Jeden z nich vždy zasáhne, další dva nezasáhnou, tj. násobíme pravděpodobnost zásahu jednoho s 1 mínus pravděpodobnost druhého a 1 mínus pravděpodobnost třetího (předpokládám, že se navzájem neovlivňují):
        $$ P(A) = 0.2 · (1-0.4) · (1-0.6) = 0,048, $$
        $$ P(B) = (1 - 0.2) · 0.4 · (1-0.6) = 0,128, $$
        $$ P(C) = (1-0.2) · (1-0.4) · 0.6 = 0,288. $$

        Pravděpodobnost $P(Y)$, že byl zasažen právě jedním (což je druhá hodnota pro výpočet podmíněné pravděpodobnosti), je součet těchto pravděpodobností, jelikož je to dělení jevu zasažen právě jednou, tj.
        $$ P(Y) = P(A) + P(B) + P(C) = 0,048 + 0,128 + 0,288 = 0,464. $$
        Podmíněná pravděpodobnost je pak dána jako $P(X|Y) = \frac{P(X)}{P(Y)}$ (jelikož zde $X \setminus Y = \O$), to jest:
        $$ P(A|Y) = \frac{0,048}{0,464} \approx 0,10, $$ 
        $$ P(A|Y) = \frac{0,128}{0,464} \approx 0,28, $$ 
        $$ P(A|Y) = \frac{0,288}{0,464} \approx 0,62. $$ 
    \end{reseni}
\end{priklad}

\begin{priklad}[2]
    Koupili jsme si soupravu vláčků. Tyto vláčky se spojují pomocí magnetů, tj. mašinka má vzadu magnet, který je otočen ven buď kladným nebo záporným pólem, a vagonky mají podobný magnet na každém svém konci. Máme-li v krabici 1 mašinku a $n$ vagonků, jaká je pravděpodobnost, že nám půjdou pospojovat (tj. ve vlaku vždy navazuje kladný pól na záporný), pokud v továrně přiřazují polarizace magnetů rovnoměrně nezávisle náhodně (ani polarizace magnetů na různých koncích téhož vagonku se nijak neovlivňují), pokud
    
    \begin{enumerate}
        \item Máme určené pořadí i orientaci jednotlivých vagonků?
        \item Máme určené pořadí vagonků?
        \item Můžeme vagonky libovolně přeskládat i pootáčet?
    \end{enumerate}

    \begin{reseni}
        V prvních dvou případech postavíme vlak tak, že budeme přidávat za lokomotivu vagóny jeden po druhém. Tedy vždy máme danou nějakou polarizaci posledního přidaného vagónu (na začátku lokomotivy) a hledáme pravděpodobnost, s jakou bude mít nějaký magnet „správnou“ (tedy opačnou tomuto) polarizaci. Tu každý z $m$ magnetů nebude mít s pravděpodobností $\frac{1}{2}$, tedy pravděpodobnost, že žádný z dostupných $m$ magnetů nebude mít správnou polarizaci je $\frac{1}{2^m}$, tudíž pravděpodobnost, že všechny budou mít správnou je $1 - \frac{1}{2^m}$. Tyto pravděpodobnosti budeme násobit, jelikož jsou nezávislé (viz dále).

        V prvním případě máme vždy k dispozici jen 1 magnet, tedy pravděpodobnost, že jedno spojení bude fungovat je $1 - \frac{1}{2} = \frac{1}{2}$, tedy že budou fungovat všechny je
        $$\underbrace{\frac{1}{2}·\frac{1}{2}·…·\frac{1}{2}}_{n\text{-krát}} = \frac{1}{2^n}. $$
        Jelikož jsou magnety nezávislé a my se ptáme pouze na ten, který hned použijeme, jsou jevy, jež násobíme, opravdu nezávislé.

        V případě druhém máme na výběr 2 magnety, tedy pravděpodobnost úspěchu u jednoho spojení je $1 - \frac{1}{2^2} = \frac{3}{4}$. Tj. u všech je to
        $$ \underbrace{\frac{3}{4}·\frac{3}{4}·…·\frac{3}{4}}_{n\text{-krát}} = \(\frac{3}{4}\)^n $$
        A jelikož jsou vagónky nezávislé a my se vždy ptáme jen na ten, který použijeme, spojení jsou opravdu nezávislá (můžeme předpovědět, jaký magnet bude na konci již spojeného vlaku, ale to nás netrápí, protože je jedno, která polarizace bude „správná“).

        Ve třetím případě to takhle nefunguje, ale můžeme si všimnout, že vláček půjde pospojovat právě tehdy, když se počet magnetů jedné polarizace a počet magnetů druhé budou lišit o 1. Tedy ze všech $2^{2n+1}$ různých vláčků funguje $\binom{2n}{n+1} + \binom{2n}{n+1}$ vláčků, tedy pravděpodobnost je
        $$ \frac{\binom{2n}{n+1} + \binom{2n}{n+1}}{2^{2n+1}} = \frac{2\binom{2n}{n+1}}{2^{2n+1}} = \frac{\binom{2n}{n+1}}{4^n}. $$ 
    \end{reseni}
\end{priklad}

\end{document}
