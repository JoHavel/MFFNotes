\documentclass[12pt]{article}					% Začátek dokumentu
\usepackage{../../MFFStyle}					    % Import stylu



\begin{document}
\section{Úvod}
    \begin{poznamka}[Co je diskrétní matematika]
        Protipól matematiky spojité. Souhrnný název pro matematické disciplíny, zabývající se diskrétními objekty.
    \end{poznamka}

    \begin{poznamka}[Co je potřeba]
        Cvičení + zkouška z věcí z přednášky.
    \end{poznamka}

    \begin{poznamka}[literatura]
        Kapitoly z diskrétní matematiky od Matouška.
    \end{poznamka}

    \begin{definice}[Důkaz (neformální)]
        Rozebírání tvrzení na tvrzení, která už jsou zřejmá.
    \end{definice}

    \begin{definice}[Definice (neformální)]
        Definujeme objekty pomocí jednodušších a jednodušších, až axiomů.
    \end{definice}
    
    \begin{definice}[Důkaz sporem]
        Dokážeme $\phi$ tím, že vyvrátíme $\phi$
    \end{definice}

    \begin{definice}[Důkaz matematickou indukcí]
            Dokážeme $\phi(n), \forall n \in ®N$ tak, že dokážeme $\phi(0)\land(\forall n \in ®N)(\phi(n)\implies\phi(n+1))$
    \end{definice}

    \begin{definice}[Dolní a horní celá část]
        $\left\lceil x\right\rceil$ je nejbližší nižší celé číslo k $x$

        $\left\lfloor x\right\rfloor$ je nejbližší vyšší celé číslo k $x$
    \end{definice}

    \begin{definice}[Sčítání mnoha čísel]
        $\sum_{i=13}^n x_i = x_{13} + x_{14} + … + x_n$ = Sčítání $x$ od indexu 13 do indexu $n$

        $\sum_\O = 0$
    \end{definice}

    \begin{definice}[Sčítání mnoha čísel]
        $\prod_{i=13}^n x_i = x_{13} \cdot x_{14} \cdot … \cdot x_n$ = Násobení $x$ od indexu 13 do indexu $n$

        $\prod_\O = 1$
    \end{definice}

    \begin{poznamka}[Klasické množiny]
        ®N ®Z ®Q ®R ®C
    \end{poznamka}

    \begin{poznamka}[Klasické množinové operace]
        $$ x \in ®A $$
        $$ ®A \subseteq ®B $$
        $$ ®A \cap ®B $$
        $$ ®A \cup ®B $$ 
        $$ ®A \setminus ®B $$
        $$ ®A\triangle®B = (®A\setminus ®B)\cup(®B\setminus ®A) = \text{disperze} $$
        $$ 2^®A = ©P(®A) $$
    \end{poznamka}

    \begin{definice}[Uspořádaná dvojice]
        Uspořádaná dvojice je $(x, y)$ nebo $\{\{x\},\{x, y\}\}$.

        Vytváří se např. kartézským součinem $®A\times ®B := \{(a,b)|a\in ®A, b \in ®B\}$.

        Uspořádaná trojice je $(x, y, z) = ((x, y), z) = (x, (y, z))$. Atd. pro n-tice.
    \end{definice}

    \begin{definice}[Relace]
        ®A je relace (binární) mezi množinami ®X a ®Y $≡ ®A \subseteq ®X \times ®Y.$

        ®A je relace (binární) na množině ®X $≡$ mezi ®X a ®X.

        Inverze je relace mezi ®Y a ®X: $R^{-1} := \{(y, x) | (x, y) \in R\}$.

        Skládání $T = R \circ S = \{(x, z)| \exists y: xRy \land ySz\}$

        Diagonála = diagonální relace: $\triangle x := \{(x, x) \in ®X\}$
    \end{definice}

    \begin{definice}[Funkce = zobrazení]
        Funkce z množiny ®X do množiny ®Y je relace $A$ mezi ®X a ®Y taková, že $\forall x \in ®X \exists! y\in ®Y: xAy$    
    \end{definice}

    \begin{definice}[Vlastnosti funkcí]
        Funkce $f: ®X \rightarrow ®Y$ je:
        \begin{itemize}
            \item prostá (injektivní) $≡ \not\exists x, x' \in ®X: x≠x' \land f(x) = f(x')$
            \item na ®Y (surjektivní) $≡ \forall y \in ®Y \exists x \in ®X: f(x) = y$
            \item vzájemně jednoznačná (bijektivní, 1-1 (jedna ku jedné)) $\forall y \in ®Y \exists! x \in ®X: f(x) = y$
        \end{itemize}
    \end{definice}

    \begin{definice}[Vlastnoti relací]
        Relace $R$ na ®X je:
        \begin{itemize}
            \item reflexivní $≡ \forall x \in ®X: xRx$
            \item symetrická $≡ \forall x, y \in ®X: xRy \implies yRx (\Leftrightarrow R = R^{-1})$
            \item antisymetrická $≡ \forall x, y \in ®X: xRy \land yRx \implies x = y$
            \item tranzitivní $≡ \forall x, y, z \in ®X: xRy \land yRz \implies xRz$
        \end{itemize}
    \end{definice}

    \begin{definice}[Ekvivalence]
        Relace se nazývá ekvivalence, pokud je tranzitivní, reflexivní a symetrická.
    \end{definice}

    \begin{definice}[Ekvivalenční třídy]
        $$ R[x] = \{y \in ®X | xRy\} $$
    \end{definice}

    \begin{veta}
        $$ 1) \forall x \in ®X R[x] ≠ \O $$
        $$ 2) \forall x, y \in ®X: R[x] = R[Y] XOR R[x]\cap R[y] = \O $$
        $$ 3) \{R[x]| x \in ®X\} \text{určuje ekvivalenci $R$ jednoznačně} $$ 
        \begin{dukazin}
            1) triviální

            2) Dokážeme: pokud $R[x] \cap R[y] ≠ \O$, pak $R[x] = R[y]$. (Tranzitivita).

            3)
        \end{dukazin}
    \end{veta}

    \begin{definice}[Rozklad množiny]
        Množinový systém $©S \subseteq 2^{®X}$ je rozklad množiny ®X tehdy, když\\
        (R1) $\forall ®A \in ©S: ®A ≠ \O$,\\
        (R2) $\forall ®A, ®B \in ©S: ®A ≠ ®B \implies ®A\cap ®B = \O$,\\
        (R3) $\bigcup_{®A \in ©S} = ®X$.
    \end{definice}

    \begin{definice}[Uspořádání]
        Relace $R$ na množině $®X$ je uspořádání $≡ R$ je reflexivní, antisymetrická a tranzitivní.

        \begin{poznamkain}
            Někdy se říká částečné uspořádání a částečně uspořádaná množina (čum), aby se zdůraznilo, že nemusí být lineární.
        \end{poznamkain}
    \end{definice}

    \begin{definice}[Uspořádaná množina]
        Dvojice $(X, R)$, kde $X$ je množina a $R$ je uspořádání na ní.
    \end{definice}

    \begin{definice}[Porovnatelné prvky a lineární uspořádání]
        $xy \in X$ jsou porovnatelné $≡ xRy \lor yRx$

        Uspořádání $R$ je lineární $≡ \forall x,y \in X$ porovnatelné.
    \end{definice}

    \begin{definice}[Ostrá nerovnost]
        $(X, ≤)$ ČUM $\rightarrow (X, <): x<y≡x≤y \land x≠y$
    \end{definice}

    \begin{definice}[Hasseův diagram]
        \ 
        \begin{poznamkain}
            Splňuje následující:
            1. To, co je nahoře je větší než to, co je dole\\
            2. Nezakreslujeme tranzitivitu\\
        \end{poznamkain}
        \ 
        \begin{definicein}[Bezprostřední předchůdce ($x \triangleleft y$)]
                $x$ je bezprostřední předchůdce $y$ v uspořádání $≤ ≡ x<y \land (\not\exists z: x<z \land z<y)$
        \end{definicein}

        V hasseově diagramu jsou mezi vrcholy (prvky množiny) hrany pouze, pokud dolní vrchol je bezprostředním předchůdcem toho nahoře.
    \end{definice}
    
    \begin{definice}[Nejmenší, minimální, největší a maximální prvek]
        \ 
        \begin{itemize}
            \item $x \in ®X$ je nemenší $≡ \forall y \in ®X: x≤y$
            \item $x \in ®X$ je minimální $≡ \nexists y \in ®X: y<x$
            \item největší a maximální obdobně
        \end{itemize}
    \end{definice}

    \begin{lemma}
        Každá konečná neprázdná ČUM má minimální prvek.
        \begin{dukazin}[Důkazík]
            $x_1 \in ®X$ zvolíme libovolně, pokud $x_1$ není minimální $\exists x_2 < x_1$… $\exists k \in ®N x_k$ je minimální.
        \end{dukazin}
    \end{lemma}

    \begin{definice}[Řetězec]
        Pro $(X, ≤)$ ČUM $A \subseteq X$ je řetězec $≡ \forall a, b \in A: a,b$ jsou porovnatelné.

        Naopak $A \subseteq X$ je antiřetězec (nezávislá množina) $≡ \nexists a, b \in A$ různé a porovnatelné.
    \end{definice}

    \begin{definice}[Délka nejdelšího řetězce]
            $$ \omega(X, ≤) := \text{maximum z délek řetězců („výška uspořádání“)} $$
            $$ \alpha(X, ≤) := \text{maximum z „délek“ (velikostí) antiřetězců („šířka uspořádání“)} $$
    \end{definice}

    \begin{veta}[O dlouhém a Širokém]
        $$ \forall (X, ≤) \text{ČUM}: \alpha(X, ≤)·\omega(X, ≤) ≥ |X| $$ 
        (Neboli buď $\alpha ≥ \sqrt{|X|}$ nebo $\omega ≥ \sqrt{|X|}$.)

        \begin{dukazin}
            Sestrojíme $X_1:= \{x \in X|x \text{je minimální}\}$.

            Když máme $X_1, …, X_i$, $Z_i:= X\setminus \(\bigcup_{j=1}^i x_j\)$. Pokud $Z_i = \O$, tak jsme skončili, jinak $X_{i+1} := \{x\in Z_i | x \text{je minimální v} Z_i\}$.

            Přitom $\forall i$ $X_i$ je antiřetězec, $\{X_1, …, X_k\}$ tvoří rozklad $X$ a $\exists \{r_j \in X_j\}_{j = 1}^k$, $\{r_j\}_{j=1}^k$ je řetězec. ($r_k \in X_k$ zvolíme libovolně, $r_j \notin X_{j-1} \implies \exists r_{j-1} \in X_{j-1}: r_{j-1}<r_j$.)

            $$ |X| = \sum_{i=1}^k |X_i| ≤ k·\max_{1≤i≤k}|X_i| ≤ \omega·\alpha. $$ 

        \end{dukazin}
    \end{veta}

% 26. 10. 2020

    \begin{veta}
        $\# f: N \rightarrow M = m^n, |N| = n, |M| = m, m > 0, n > 0$

        \begin{dukazin}[Indukcí]
            $$ n=1: \# f = m = m^1 $$
            $$ n \rightarrow n+1: f \text{jednoznačně určena} f(x) \text{a} f':N\setminus \{x\} \rightarrow M \implies \# f = m · m^n = m^{n+1} $$ 
        \end{dukazin}
    \end{veta}

    \begin{veta}
        Je-li $N$ $n$-prvková množina, pak $|2^N| = 2^n$.

        \begin{dukazin}
            $$ \text{charakteristická funkce: } A \subseteq N \rightarrow C_A: N \rightarrow \{0, 1\} C_A(x) = 0, x \notin A, C_A(x) = 1, x \in A $$ 
        \end{dukazin}
    \end{veta}

    \begin{veta}
        Nechť $X ≠ \O$ je konečná množina, $©S := \{S \subseteq X| |S| \text{je sudá}\}$, $©L := \{L \subseteq X| |L| \text{je lichá}\}$. Potom $|©S| = |©L| = 2^{n-1}$.
        
        \begin{dukazin}
            Víme, že $©S \cup ©L = 2^X$. Stačí tedy $|©S| = |©L|$. Zvolíme si $a \in X$. Pak $f(S):= S \triangle \{a\}$ je bijekce z ©S do ©L.
        \end{dukazin}
    \end{veta}

    \begin{veta}
        Nechť $N$ je $n$ prvková, $M$ je $m$-prvková. Potom $\# f: N \rightarrow M$ prostých $= m·(m-1)·…·(m-n+1)$.

        \begin{poznamka}[Možná značení]
            $$ \[n\]:= \{0, 1, …, \} $$
            $$ m^{\underline{n}} = \frac{m!}{(m-n)!} \text{($m$ na $n$ klesající)} $$ 
        \end{poznamka}
    \end{veta}

    \begin{poznamka}[Kódování funkcemi]
        \ 
        \begin{itemize}
            \item $X \rightarrow \{0, 1\} … 2^X$
            \item $\{1, 2\} \rightarrow X … (x, y)\in X^2$
            \item $\{1, …, k\} \rightarrow X … \text{uspořádané $k$-tice … $X^k$}$
            \item $®N \rightarrow X$ … nekonečné posloupnosti prvků $X$
            \item permutace na $X$, tj. počet bijekcí nebo počet lineárních uspořádání na konečném $X$ … $|X|!$ ($0!=1$)
        \end{itemize}
    \end{poznamka}

    \begin{definice}[Kombinační číslo]
        Kombinační číslo / binomický koeficient ($n$ nad $k$) je $\binom{n}{k} := \frac{n^{\underline{k}}}{k!} = \frac{n!}{k!(n-k)!}$.
    \end{definice}

    \begin{definice}
        Pro množinu $X$ a $k≥0$ definujeme $\binom{X}{k}:= \{A \subseteq X: |A| = k\}$.
    \end{definice}

    \begin{veta}
        Pro každou množinu $X$ a $k≥0$: $\left|\binom{X}{k}\right| = \binom{|X|}{k}$.
    \end{veta}

    \begin{poznamka}[Vlastnosti kombinačních čísel]
        $$ \binom{n}{0} = \binom{n}{n} = 1 $$ 
        $$ \binom{n}{1} = \binom{n}{n-1} = n $$ 
        $$ \binom{n}{k} = \binom{n}{n-k} $$
        $$ \binom{n}{k} = \binom{n-1}{k} + \binom{n-1}{k-1} \text{ (Lze upočítat / nebo rozdělit na případ vybereme / nevybereme konkrétní prvek.}$$
        $$ \sum_{k=0}^n \binom{n}{k} = 2^n \text{ BV $A=1$, $B=1$} $$ 
        $$ \sum_{k=0}^n (-1)^k \binom{n}{k} = 0 \text{ BV $A=1$, $B=-11$} $$
    \end{poznamka}

    \begin{poznamka}
        Vlastnosti se dají vykoukat v tzv. Pascalově trojúhelníku.
    \end{poznamka}

    \begin{veta}[Binomická]
        $$ (A + B)^n = \sum_{k=0}^n A^k·B^{n-k}·\binom{n}{k} $$

        \begin{dukazin}
            Vybírá se $k$ z $n$ členů, ze kterých bude $A$…
        \end{dukazin}
    \end{veta}

% 2. 11. 2020
    
    \begin{veta}[Princip inkluze a exkluze]
        Pro konečné množiny $A_1-A_n$:
        $$ \left| \bigcup_{i=1}^n \right| = \sum_k^n (-1)^{k+1} \sum_{I \in \binom{\{1, 2, …, n\}}{k}} \left|\bigcap_{i \in I} A_i\right| $$

        Nebo alternativně:
        $$ \left| \bigcup_{i=1}^n A_i \right| = \sum_{\O ≠ I \subseteq \{1, …, n\}} (-1)^{|I| + 1} \left| \bigcap_{i \in I} A_i \right| $$ 

        \begin{dukazin}
            Pro každý prvek $x \in \bigcup_i A_i$ spočítáme příspěvky k levé (vždy 1) a k pravé straně. Nechť $x$ patří právě j množin z $A_1, …, A_n$. Průniky $k$-tic: (1) $k>j$ přispěje 0. (2) $k≤j$ přispěje $(-1)^{k+1} \binom{j}{k}$. Součet toho je alternující řada kombinačních čísel „bez 1“, tedy součet je 1.
        \end{dukazin}

        \begin{dukazin}[Druhý]
            Vyjdeme z
            $$ \prod_{i=1}^n (1+x_i) = \sum_{I \subseteq \{1, …, n\}} \prod_{i\in I} x_i. $$
            Definujeme si charakteristickou funkci a zjistíme, že ch. f. průniku je součin, doplňku je 1-ch. f. původního, sjednocení je doplněk průniku doplňků a velikost je součet ch. funkce. Tedy dosadíme za $x_i$ mínus charakteristické funkce (1 nám vypadla z prázdné podmnožiny):
            $$ 1-c_{\bigcup_i A_i} = \(\sum_{\O≠I\subseteq \{1, …, n\}} (-1)^{|I|}·c_{\bigcap_{i \in I} A_i}\) + 1 $$
            Následně ještě přeformulujeme do velikostí a získáme princip inkluze a exkluze.
        \end{dukazin}
    \end{veta}

    \begin{priklad}[Šatnářka]
        Šatnářka náhodně vydala klobouky gentlemanům. Jaká je pravděpodobnost, že se ani jeden klobouk nedostal k majiteli?

        Tj. $S_n := \{\pi | \pi \text{ permutace na } \{ 1, …, n \}\}$, $\pi(i) = i \implies i$ je pevný bod:
        $$ \text{Š}_n := \{ \pi \in S_n | \nexists i: \pi(i) = i \}. $$ 
        Příklad se tedy ptá na $\frac{\text{Š}_n}{n!}$.

        \begin{reseni}
            Lepší je počítat doplněk: $A := \{\pi \in S_n | \pi \text{ má pevný bod}\}$. Definujeme si $A_i := \{\pi \in S_n| \pi(i)=i\}$. Následně vypozorujeme $A = \bigcap_i A_i$. Očividně $|A_i| = (n-1)!$, $|A_i \cup A_j| = (n-2)!$ ($i≠j$), …

            $$ |A| = \left| \bigcup_{i=1}^n \right| = \sum_{k=1}^n (-1)^{k+1}\sum_{I \in \binom{\{1, …, n\}}{k}} \left| \bigcap_{i \in I} A_i \right| = \sum_{k=1}^n (-1)^{k+1}\sum_{I \in \binom{\{1, …, n\}}{k}} (n-k)! = \sum_{k=1}^n (-1)^{k+1} \binom{n}{k}(n-k)! = \sum_{k=1}^n (-1)^{k+1} \frac{n!}{k!} $$
            $$ |A| = n! \sum_{k=1}^n \frac{(-1)^{k+1}}{k!} = n!(\frac{1}{1!} - \frac{1}{2!} + \frac{1}{3!} - … + \frac{(-1)^{n+1}}{n!}) $$
            $$ \text{Š}_n = |A| ≐ n! \frac{1}{e} $$ 
        \end{reseni}
    \end{priklad}

\section{Odhady}
    \begin{priklady}
        $$ 2^{n-1} ≤ n! ≤ n^n $$ 
        $$ n^{n/2} ≤ n! ≤ \(\frac{n+1}{2}\)^n $$
        $$ *\(\frac{n}{e}\)^n ≤ n! ≤ en·\(\frac{n}{e}\)^n  $$ 
        $$ ** n! \sim \(\frac{n}{e}\)^n·\sqrt{2\pi n} $$
        $$ \(\frac{n}{k}\)^k ≤ \binom{n}{k} ≤ n^k $$
        $$ *\binom{n}{k} ≤ \(\frac{en}{k}\)^k $$ 
        $$ \frac{4^n}{2n+1} ≤ \binom{2n}{n} ≤ 4^n $$ 
        $$ * \frac{4^n}{2\sqrt{n}} ≤ \binom{2n}{n} ≤ \frac{4^n}{\sqrt{2n}} $$ 
    \end{priklady}

% 9. 11. 2020

\section{Grafy}
    \begin{definice}[Graf, vrcholy, hrany]
        Graf je uspořádaná dvojice $(V, E)$, kde: $V$ je konečná neprázdná množina vrcholů (vertices) a $E \subseteq \binom{V}{2}$ je množina hran (edges).
    \end{definice}

    \begin{poznamka}[Rozšíření]
        Orientované, se smyčkami, multigrafy, nekonečné.
    \end{poznamka}

    \begin{priklady}
        Úplný graf ($K_n$): $V(K_n) := \{1, …, n\}$ a $E(K_n) := \binom{V(K_n)}{2}$.

        Prázdný graf ($E_n$): $V(E_n) := \{1, …, n\}$ a $E(E_n) := \O$.

        Cesta ($P_n$): $V(P_n) := \{0, 1, …, n\}$ a $E(P_n) := \{\{i, i+1\}|0≤i<n\}$.
        
        Kružnice ($C_n$): $V(C_n) := \{0, 1, …, n-1\}$ a $E(C_n) := \{\{i, i+1 \mod n\}|0≤i≤n\}$.

        Úplný bipartitní graf ($K_{m, n}$): $V(K_n) := \{a_1, …, a_n\} \cup \{b_1, …, b_n\}$ a $E(K_n) := \{\{a_i, b_j\}|1≤i≤m, 1≤j≤n\}$.
    \end{priklady}

    \begin{definice}[Bipartitní graf]
        Graf $G$ je bipartitní $≡ \exists$ rozklad množiny $V(G)$ na $X, Y$ (= partity) tak, že $E(G) \subseteq \{\{x, y\} | x\in X, y \in Y\}$. (Lze zapsat i jako $\forall e \in e(G): |e\cap X| = 1$.)
    \end{definice}

    \begin{definice}[Isomorfismus grafů]
        Grafy $G$ a $H$ jsou isomorfní (značme $G \cong H$) $≡ \exists f: V(G) \rightarrow V(H)$ bijekce tak, že $\forall u, v \in V(G):(\{u, v\} \in E(g) \Leftrightarrow \{f(u), f(v)\} \in E(H))$.
    \end{definice}

    \begin{poznamka}[K nahlédnutí]
        Na libovolné množině grafů je $\cong$ ekvivalence.
    \end{poznamka}

    \begin{definice}[Stupeň vrcholu]
        Stupeň vrcholu $v$ v grafu $G$ je $\deg_G(v) := \left|\{u \in V(G)|\{u, v\} \in E(G)\}\right|$.
    \end{definice}

    \begin{definice}[Regulární graf]
        Graf je $k$-regulární (pro $k \in ®N$) $≡ \forall u \in V(G): \deg_G(u) = k$.

        Graf $G$ je regulární $≡ \exists k: G$ je $k$-regulární.
    \end{definice}

    \begin{definice}[Skóre grafu]
        Skóre grafu $G$ je posloupnost stupňů všech vrcholů (až na uspořádání).
    \end{definice}

    \begin{veta}
        Pro každý graf $(V, E)$ platí:
        $$ \sum_{v\in V} \deg(v) = 2·|E| $$ 
    \end{veta}

    \begin{dusledek}[Princip sudosti]
        $\sum_v \deg(v)$ je sudé číslo $\implies (\#v\in V \text{ lichého stupně})$ je sudý.
    \end{dusledek}

    \begin{veta}[O skóre]
        Posloupnost $D = d_1 ≤ … ≤ d_n$ pro $n≥2$ je skóre grafu $\Leftrightarrow$ $D' = d'_1, …, d'_{n-1}$ je skóre grafui a $0≤d_n≤n-1$. ($d'_i = d_i$ pro $i < n-d_n$ a $d'_i = d_i - 1$ pro $i ≥ n-d_n$.)

        \begin{dukazin}
            $(\Leftarrow)$ nechť $G'$ je graf se skóre $D'$ a vrcholy $v_1, …, v_{n-1}$ tak, že $\forall i \deg_{G'}(v_i)=d'_i$. Vytvořím $G$ doplněním vrcholu $v_n$ a hran $\{v_i, v_n\}$ pro $i \in \{n-d_n, …, n-1\}$. $G$ má skóre $D$.

            $(\implies)$ Lemma: Nechť ©G je množina všech grafů se skóre $D$, $©G ≠ \O$. Potom $\exists G \in ©G: \{v_n, v_i\}\in E(G)$ pro všechna $i \in \{n-d_n, …, n-1\}$.

            Důkaz lemmatu: (Kdyby $d_n = n-1$, pak zřejmě každý $G \in ©G$ splňuje lemma.) Pro $G \in ©G$ definujeme $j(G) := \max\{j|\{v_j, v_n\} \notin E(G)\}$ (kdyby $j(g) = n-d_ni-1$, pak jsme vyhráli, jinak $G$ nesplňuje lemma). Najdeme $G \in ©G$, jehož $j(G)$ je minimální. Pokračujeme sporem: Kdyby $j(G) > n - d_n - 1$, musí $\exists i < j: \{v_i, v_n\} \in E(G)$. Následně chceme ukázat, že $\exists k: \{v_i, v_k\}\notin E(G) \land \{v_j, v_k\} \in E(G)$, to ukážeme na základě toho, že posloupnost je seřazena, tedy $d_i ≤ d_j$ a vrchol $v_i$ je spojen minimálně s jedním vrcholem, se kterým není spojené $v_j$ ($v_n$). Upravíme graf $G$ na $G_\lightning: V(G_\lightning) := V(G), E(G_\lightning) := E(G) \cup \{\{v_i, v_k\}, \{v_j, v_n\}\} \setminus \{\{v_i, v_n\}, \{v_j, v_k\}\}$. Ale jelikož jsme vrcholům odstranili stejný počet hran, jako přidali, $G_\lightning \in ©G$. Navíc zřejmě $j(G_\lightning) < j(G)$, $\lightning$.
        \end{dukazin}
    \end{veta}

% 16. 11. 2020

    \begin{priklad}[Kolik je grafů na $n$ vrcholech? Kolik je neizomorfních?]
        Grafů je tolik, kolik je podmnožin množiny všech hran, tedy $2^{\binom{n}{2}}$.

        Izomorfních grafů jednomu grafu nemůže být více než $n!$, tedy neizomorfních bude více jak
        $$ \frac{2^{\binom{n}{2}}}{n!} $$ 
    \end{priklad}

    \begin{definice}[Podgraf a indukovaný graf]
        Graf $G' = (V', E')$ je podgrafem (značíme $G' \subseteq G$) grafu $G = (V, E) ≡ V' \subseteq V \land E' \subseteq E$.

        Graf $G' = (V', E')$ je indukovaným (množinou $V'$, značíme $G\[V'\]$) podgrafem grafu $G = (V, E) ≡ V' \subseteq V \land E' = E \cap \binom{V'}{2}$.
    \end{definice}

    \begin{definice}[Cesta v grafu]
        Cesta v grafu $G$ je:

        1.) $G' \subseteq G: G' \cong P_n$ pro nějaké $n$.

        2.) $(v_0, e_1, v_1, e_2, v_2, …, e_n, v_n)$, kde $v_0, …, v_n$ jsou navzájem různé vrcholy, $e_1, …, e_n$ jsou hrany, $\forall i e_i = \{v_{i-1}, v_i\}$.
    \end{definice}

    \begin{definice}[Kružnice (cyklus) v grafu]
    

        1.) $G' \subseteq G: G' \cong C_n$ pro nějaké $n$.

        2.) $(v_0, e_0, v_1, e_1, v_2, …, v_{n-1}, e_{n-1}, v_0)$, kde $v_0, …, v_{n-1}$ jsou navzájem různé vrcholy, $e_1, …, e_{n-1}$ jsou hrany, $\forall i e_i = \{v_{i}, v_{(i+1) \mod n}\}$.
    \end{definice}

    \begin{definice}[Souvislý graf]
        Graf $G$ je souvislý $≡ \forall u, v \in V(G) \exists$ cesta v $G$ s krajními vrcholy $u, v$.
    \end{definice}

    \begin{definice}[Dosažitelnost]
        Dosažitelnost v $G$ je binární relace $\sim$ na $V(G)$ taková, že $u \sim v ≡ \exists$ cesta v $G$ s krajními vrcholy $u, v$.
    \end{definice}

    \begin{lemma}
        Relace $\sim$ je ekvivalence.
        \begin{dukazin}
            Reflexivita: $u \sim u$ (existuje triviální cesta).

            Symetrie: $u \sim v \Leftarrow v \sim u$ (koncové vrcholy cesty jsou neuspořádaná dvojice).

            Tranzitivita: $u \sim v \land v \sim w \implies u \sim w$ (definice a lemmátko viz dále, $\sim$ můžeme definovat i pomocí sledů, které už lze „slepovat“).
        \end{dukazin}
    \end{lemma}

    \begin{definice}[Komponenty souvislosti]
        Komponenty souvislosti jsou podgrafy indukované třídami ekvivalence.
    \end{definice}

    \begin{dusledek}
        Graf je souvislý $\Leftarrow$ má 1 komponentu.
    \end{dusledek}

    \begin{definice}[Sled, tah]
        Sled (walk) je $(v_0, e_1, v_1, e_2, v_2, …, e_n, v_n)$, kde $v_0, …, v_n$ jsou vrcholy, $e_1, …, e_n$ jsou hrany, $\forall i e_i = \{v_{i-1}, v_i\}$.

        Tah je $(v_0, e_1, v_1, e_2, v_2, …, e_n, v_n)$, kde $v_0, …, v_n$ jsou vrcholy, $e_1, …, e_n$ jsou navzájem různé hrany, $\forall i e_i = \{v_{i-1}, v_i\}$.
    \end{definice}

    \begin{lemma}[Lemmátko]
        $\exists$ cesta mezi $u, v$ $\Leftrightarrow \exists$ sled mezi $u, v$.
        \begin{dukazin}
            $(\implies)$ triviální. $(\Leftarrow)$ Uvažujme sled $S$. Kdyby se v $S$ neopakovaly vrcholy, je to cesta. Pokud $v_k = v_l$, potom $(v_0, e_1, v_1, …, e_k, (v_k = v_l), e_{l+1}, v_{l+1}, …, e_n, v_n)$ je kratší sled, označme ho $S$. Opakujeme dokud $S$ není cesta.
        \end{dukazin}
    \end{lemma}

    \begin{definice}[Matice sousednosti]
            Matice sousednosti $A(G)$ grafu $G$ při očíslování vrcholů $v_1, …, v_n \in V(G)$ je
            $$ A_{ij} := \[{v_i, v_j} \in E\] $$ 
    \end{definice}

    \begin{poznamka}[Značení výše]
        $[\phi]$ dává 1, pokud $\phi$ platí, a 0, pokud $\phi$ neplatí.
    \end{poznamka}

    \begin{poznamka}[Matice sousednosti]
        Je symetrická.

        Součty řádků / sloupců jsou stupně vrcholů.

        $t$-tá mocnina udává kolik sledů délky $t$ existuje mezi danými vrcholy. (Důkaz indukcí.)
    \end{poznamka}

    \begin{priklad}
        Počet trojúhelníků v grafu.

        \begin{reseni}
            Uzavřený sled délky 3 je trojúhelník. Tedy umocníme $A$ na třetí a podíváme se na diagonálu (sečteme a vydělíme 6).
        \end{reseni}
    \end{priklad}

    \begin{definice}[Vzdálenost (grafová metrika)]
        $d_G: V^2 \rightarrow ®N$. $d_G (u, v) :=$ minimum z délek všech cest mezi $u, v$.

        \begin{dukazin}[Metrika]
            $d(u, v) ≥ 0$ (velikosti nezáporné).

            $d(u, v) = 0 \Leftrightarrow u = v$ (když jsou totožné, tak cesta neobsahuje žádnou hranu, když nejsou, tak naopak musí obsahovat nějakou hranu).

            $d(u, v) = d(v, u)$ (cesta není orientovaná).

            $d(u, v) ≤ d(u, w) + d(w, v)$ (slepením dvou cest dostanu sled a ten lze \emph{zmenšit} na cestu)
        \end{dukazin}
    \end{definice}

    \begin{definice}[Grafové operace]
        $G+v$, $G+e$ je přidání vrcholu či hrany. $G - v$, $G-e$ je naopak smazání (v případě mazání vrcholu vytváříme indukovaný podgraf = mažeme i hrany z tohoto vrcholu). $G\% e$ je dělení hrany (vytvořím vrchol „uprostřed“ = hrana $e = \{u, v\} \rightarrow$ hrany $\{u, x\}$ a $\{x, v\}$ a vrchol $x$). $G.e$ je kontrakce hrany („slepíme“ vrcholy hrany).
    \end{definice}

    \begin{poznamka}[Pozorování]
        Cesty (resp. vrcholy) jde vyrábět postupným dělením $P_1$ (resp. $C_3$) a libovolnou cestu (kružnice) lze „zkontrahovat“ do $P_1$ ($C_3$).
    \end{poznamka}

% 23. 11. 2020

    \begin{definice}[Eulerovský tah]
        Eulerovský tah je takový tah, který obsahuje všechny vrcholy a hrany grafu.
    \end{definice}

    \begin{definice}[Uzavřený tah]
        Tah, ve kterém je první a poslední vrchol totožný.
    \end{definice}

    \begin{definice}[Eulerovský graf]
        Graf je eulerovský $≡$ existuje v něm uzavřený eulerovský tah.
    \end{definice}

    \begin{veta}[O eulerovských tazích]
        Graf $G$ je eulerovský $\Leftrightarrow G$ je souvislý $\land \forall v \in V(G): \deg_G(v)$ je sudý.
        \begin{dukazin}
            $(\implies)$ Zřejmé z toho, že mezi každými vrcholy vede tah a že musí do vrcholu „vstupovat“ a „vystupovat“ z něho.

            $(\Leftarrow)$ Uvažme $T := $ libovolný nejdelší tah. 1. $T$ je uzavřený (sporem: Krajní vrchol má „použito“ lichý počet hran, tedy existuje ještě jedna hrana jdoucí z tohoto vrcholu. Tu ale můžeme přidat do $T$, tedy nebyl nejdelší $\lightning$.) 2. $T$ je eulerovský: a) $\{u, v\} \in E(G),\ u, v \in T \implies \{u, v\} \in T$ (Sporem, kdyby ne, tak při některém „průchodu“ vrcholem $u$ tah $T$ „rozpojíme“ a na konec přidáme $\{u, v\}$, čímž dostaneme větší graf, $\lightning$.) b) $T$ obsahuje všechny vrcholy (Kdyby $\exists u \in V(E) \land u \notin T:$ zvolíme $v \in T$ libovolně a ze souvislosti $G$ víme, že existuje cesta $C$ mezi $u, v$. $\exists r, s \in C: r\in T, s \notin T, \{r, s\}\in E(C)$, tedy $T$ „rozpojíme“ v $R$ a prodloužíme o $\{r, s\}$, tedy $T$ není nejdelší $\lightning$.)
        \end{dukazin}
    \end{veta}

    \begin{priklad}
        $G$ obsahuje otevřený eulerovský tah $\Leftrightarrow$ $G$ je souvislý $\land$ právě dva vrcholy mají lichý stupeň.
    \end{priklad}

    \begin{poznamka}
        Věta o eulerovských tazích platí i pro multigrafy. (Smyčky musíme do stupně vrcholu počítat dvakrát. To už musíme pro paritu součtu stupňů.)
    \end{poznamka}

    \subsection{Orientované grafy}
        \begin{poznamka}[Co se změnilo]
            Sledy, tahy, cesty, kružnice jsou orientované. Hranám se říká šipky. Matice sousednosti není symetrická. Hlavně se změní souvislost.
        \end{poznamka}

        \begin{definice}[Podkladový graf, slabá a silná souvislost]
            Pro orientovaný graf $G=(V, E)$ nazveme $G^0 = (V, E^0)$, kde $\{u, v\} \in E^0 ≡ (u, v) \in E \lor (v, u) \in E$, podkladovým grafem.

            Graf je slabě souvislý právě tehdy, když jeho podkladový graf je souvislý. Slabě souvislá komponenta je slabě souvislý podgraf / komponenta souvislosti podkladového grafu.

            Graf je silně souvislý $≡ \forall u, v \in V(G)\ \exists$ orientovaná cesta z $u$ do $v$. Silně souvislá komponenta je silně souvislý podgraf.

            *Graf je polosouvislý $≡ \forall u, v \in V(G)\ \exists$ cesta z $u$ do $v$ nebo z $v$ do $u$.
        \end{definice}

        \begin{definice}[Stupně]
                $\deg^{in}(v) := \#u: (u, v) \in E$, $\deg^{out}(v) := \#v: (u, v) \in E$. (Občas se používá $\deg^+$ a $\deg^-$, tam se však nelze shodnout, co je co.)
        \end{definice}

        \begin{definice}
            Graf je vyvážený $≡ \forall v \in V: \deg^{in}(v) = \deg^{out}(v)$.
        \end{definice}

        \begin{veta}
            Následující vlastnosti orientovaného grafu $G$ jsou ekvivalentní: 1. $G$ je vyvážený a slabě souvislý, 2. $G$ je eulerovský, 3. $G$ je vyvážený a silně souvislý.
            \begin{dukazin}
                $(3\implies 1)$ je zřejmé, jelikož silně souvislý graf je i slabě souvislý. $(2 \implies 3)$ $\exists$ orientovaný tah $u\rightarrow v \implies \exists$ cesta $u \rightarrow v$. $(1 \implies 2)$ analogicky Větě o eulerovských tazích. 
            \end{dukazin}
        \end{veta}

    \subsection{Stromy}
        \begin{definice}[Strom]
            Strom je souvislý graf bez kružnic (tzv. acyklický graf).
        \end{definice}

        \begin{definice}[Les]
            Les je acyklický graf. (Jeho komponenty souvislosti jsou stromy.)
        \end{definice}

        \begin{definice}[List]
            List je vrchol stupně 1.
        \end{definice}

        \begin{upozorneni}
            Existuje právě jeden strom bez listů (jednovrcholový).
        \end{upozorneni}

        \begin{lemma}[O koncovém vrcholu]
            Každý strom s alespoň 2 vrcholy má alespoň 2 listy.
            \begin{dukazin}
                Uvažme nejdelší cestu ve stromu, potom krajní vrcholy jsou listy (Sporem: kdyby krajní vrchol nebyl list, pak z něj vede hrana, která neleží na cestě a jejíž druhý vrchol buď na této cestě už leží (spor s acykličností), nebo neleží (spor s maximalitou)).
            \end{dukazin}
        \end{lemma}

        \begin{lemma}[Vandalské (trháme listy) a pěstovatelské (necháváme vyrůst listy)]
            Nechť $v$ je list grafu $G$. Pak $G$ je strom $\Leftrightarrow G - v$ je strom.
            \begin{dukazin}
                $(\implies)$ Odebráním vrcholu nevznikne kružnice, přes list nemohla vést cesta (jelikož je stupně 1).

                $(\Leftarrow)$ Přidáním vrcholu stupně 1 nevznikne kružnice, z tranzitivity dosažitelnosti vede z libovolného vrcholu do listu cesta.
            \end{dukazin}
        \end{lemma}

% 30. 11. 2020

        \begin{veta}[O charakterizaci stromů]
            Pro graf $G$ jsou následující tvrzení ekvivalentní:
            \begin{enumerate}
                \item $G$ je souvislý a acyklický.
                \item $\forall u, v \in V(G) \exists!$ cesta mezi $u, v$ v $G$ (jednoznačně souvislý).
                \item $G$ je souvislý a $\forall e \in E(G): G-e$ není souvislý (minimální souvislý).
                \item $G$ je acyklický a $\forall e \in \binom{V(G)}{2} \setminus E(G): G + e$ obsahuje cyklus (maximální acyklický).
                \item $G$ je souvislý a $|E(G)| = |V(G)| - 1$ (speciální případ Eulerovy formule).
            \end{enumerate}
            \begin{dukazin}
                $G = (V, E)$

                $(1 \implies 2)$ Indukcí podle $|V|$ … pro $|V| = 1$ zřejmě (2) platí. Pro $|V| = n$: Buď $G$ graf s $n$ vrcholy. Platí li (1), $G$ je strom $\implies \exists l$ list v $G$, $s$ jediný soused $l$. $G-l$ je také strom, má $n-1$ vrcholů, tedy z indukčního předpokladu $G-l$ je jednoznačně souvislý. Nechť $u, v \in V$: a) $u, v ≠ l$: $G-l$ obsahuje právě jednu cestu přidáním listu nemohla vzniknout nová, b) $u, v = l$: Triviální, c) BÚNO $u=l, v≠l$ cesta $v…l$ jde přes $s$, mezi $v, s \exists!$ cesta (z IP) a ta se dá rozšířit do $l$ právě jedním způsobem.

                $(1 \implies 3)$ Jednoduchou indukcí podle $|V|$. $(1 \implies 4)$ Jednoduchou indukcí podle $|V|$. $(1 \implies 5)$ Jednoduchou indukcí podle $|V|$.

                $(2 \implies 1)$ Dokážeme jako $(\neg 1 \implies \neg 2)$, tedy není souvislý nebo není acyklický, pak počet cest mezi nějakými (existují $u, v$ tak, že) $u, v$ není 1. Pokud není souvislý, tak existují vrcholy, mezi kterými nevede hrana, pokud není acyklický, tak existují vrcholy na cyklu a ty mezi sebou mají minimálně 2 cesty.

                $(3 \implies 1)$ dokážeme jako není souvislý nebo obsahuje cyklus, pak není souvislý nebo $\exists e \in E: G-e$ je souvislý. $(4 \implies 1)$ dokážeme jako není souvislý nebo obsahuje cyklus, pak obsahuje cyklus nebo $\exists e: G+e$ je acyklický.

                $(5 \implies 1)$ Potřebujeme dokázat lemma $(5) \land |V|≥2 \implies \exists l$ list. Poté už dokážeme indukcí podle $|V|$ odebíráním listu. Důkaz lemmatu: $\sum_{v\in V} \deg(v) = 2·|E|=2n-2$. Kdyby neexistoval list $\forall v: \deg(v) ≥ 1$ a součet stupňů by byl $≥2n>2n-2 \lightning$.
            \end{dukazin}
        \end{veta}

        \begin{definice}[Kostra grafu]
            $T\subseteq G$ je kostra grafu $G ≡ T$ je strom $\land |V(T)| = |V(G)|$.
        \end{definice}

        \begin{veta}
            Graf má kostru $\Leftrightarrow$ je souvislý.
            \begin{dukazin}
                $(\implies)$ Zřejmé. $(\Leftarrow)$ Mažeme hrany na cyklech, dokud nebude strom.
            \end{dukazin}
        \end{veta}
    
    \subsection{Kreslení do roviny}
        \begin{definice}[Oblouk]
            Oblouk je prosté spojité zobrazení $f: \[0, 1\]$ do $®R^2$, $f(0), f(1):$ krajní body.

            Často budeme oblouk říkat obrazu tohoto zobrazení
        \end{definice}

        \begin{definice}[Topologická kružnice]
            Topologická kružnice je spojité zobrazení $f: \[0, 1\] \rightarrow ®R^2$, které je prosté vyjma $f(0) = f(1)$.
        \end{definice}

        \begin{definice}[Nakreslení grafu do roviny]
            Nakreslení grafu $G=(V, E)$ do roviny: a) Vrcholům $v \in V$ přiřadíme navzájem různé body $b(v) \in ®R^2$. b) Hranám $e \in E$ přiřadíme oblouky $o(e)$ tak, že je-li $e=\{u, v\}$, pak $b(u), b(v)$ jsou krajní body $o(e)$. c) $\forall v \in V\ \forall e \in E:$ pokud $b(v) \in o(e)$, pak $v \in e$. d) $\forall e, f \in E:$ pokud $o(e)$ a $o(f)$ mají společný bod, pak to je jejich krajní bod.
        \end{definice}

        \begin{poznamka}
            Nakreslením cesty je oblouk, nakreslením kružnice je topologická kružnice.
        \end{poznamka}

        \begin{definice}[Topologický graf]
            Topologický graf $:=$ graf + jeho nakreslení.
        \end{definice}

        \begin{definice}[Oblouková souvislost]
            $X \subseteq ®R^2$ je obloukově souvislá $≡ \forall x, y \in X\ \exists\text{oblouk}\subseteq X$ s krajními body $x, y$.

            Oblouková souvislost nám dává relaci dosažitelnosti (ekvivalence) a ekvivalenční třídy (komponenty obloukové souvislosti).
        \end{definice}

        \begin{definice}[Stěny nakreslení]
            Stěny nakreslení $≡$ komponenty obloukové souvislosti množiny $®R^2 \setminus \bigcup_{e \in E} o(e)$.
        \end{definice}

        \begin{upozorneni}
            Stěny nejsou vlastností grafu, ale konkrétního nakreslení (tedy též topologického grafu).
        \end{upozorneni}

\end{document}
