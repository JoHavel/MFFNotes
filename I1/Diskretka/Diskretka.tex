\documentclass[12pt]{article}					% Začátek dokumentu
\usepackage{../../MFFStyle}					    % Import stylu



\begin{document}
\section{Úvod}
    \begin{poznamka}[Co je diskrétní matematika]
        Protipól matematiky spojité. Souhrnný název pro matematické disciplíny, zabývající se diskrétními objekty.
    \end{poznamka}

    \begin{poznamka}[Co je potřeba]
        Cvičení + zkouška z věcí z přednášky.
    \end{poznamka}

    \begin{poznamka}[literatura]
        Kapitoly z diskrétní matematiky od Matouška.
    \end{poznamka}

    \begin{definice}[Důkaz (neformální)]
        Rozebírání tvrzení na tvrzení, která už jsou zřejmá.
    \end{definice}

    \begin{definice}[Definice (neformální)]
        Definujeme objekty pomocí jednodušších a jednodušších, až axiomů.
    \end{definice}
    
    \begin{definice}[Důkaz sporem]
        Dokážeme $\phi$ tím, že vyvrátíme $\phi$
    \end{definice}

    \begin{definice}[Důkaz matematickou indukcí]
            Dokážeme $\phi(n), \forall n \in ®N$ tak, že dokážeme $\phi(0)\land(\forall n \in ®N)(\phi(n)\implies\phi(n+1))$
    \end{definice}

    \begin{definice}[Dolní a horní celá část]
        $\left\lceil x\right\rceil$ je nejbližší nižší celé číslo k $x$

        $\left\lfloor x\right\rfloor$ je nejbližší vyšší celé číslo k $x$
    \end{definice}

    \begin{definice}[Sčítání mnoha čísel]
        $\sum_{i=13}^n x_i = x_{13} + x_{14} + … + x_n$ = Sčítání $x$ od indexu 13 do indexu $n$

        $\sum_\O = 0$
    \end{definice}

    \begin{definice}[Sčítání mnoha čísel]
        $\prod_{i=13}^n x_i = x_{13} \cdot x_{14} \cdot … \cdot x_n$ = Násobení $x$ od indexu 13 do indexu $n$

        $\prod_\O = 1$
    \end{definice}

    \begin{poznamka}[Klasické množiny]
        ®N ®Z ®Q ®R ®C
    \end{poznamka}

    \begin{poznamka}[Klasické množinové operace]
        $$ x \in ®A $$
        $$ ®A \subseteq ®B $$
        $$ ®A \cap ®B $$
        $$ ®A \cup ®B $$ 
        $$ ®A \setminus ®B $$
        $$ ®A\triangle®B = (®A\setminus ®B)\cup(®B\setminus ®A) = \text{disperze} $$
        $$ 2^®A = ©P(®A) $$
    \end{poznamka}

    \begin{definice}[Uspořádaná dvojice]
        Uspořádaná dvojice je $(x, y)$ nebo $\{\{x\},\{x, y\}\}$.

        Vytváří se např. kartézským součinem $®A\times ®B := \{(a,b)|a\in ®A, b \in ®B\}$.

        Uspořádaná trojice je $(x, y, z) = ((x, y), z) = (x, (y, z))$. Atd. pro n-tice.
    \end{definice}

    \begin{definice}[Relace]
        ®A je relace (binární) mezi množinami ®X a ®Y $≡ ®A \subseteq ®X \times ®Y.$

        ®A je relace (binární) na množině ®X $≡$ mezi ®X a ®X.

        Inverze je relace mezi ®Y a ®X: $R^{-1} := \{(y, x) | (x, y) \in R\}$.

        Skládání $T = R \circ S = \{(x, z)| \exists y: xRy \land ySz\}$

        Diagonála = diagonální relace: $\triangle x := \{(x, x) \in ®X\}$
    \end{definice}

    \begin{definice}[Funkce = zobrazení]
        Funkce z množiny ®X do množiny ®Y je relace $A$ mezi ®X a ®Y taková, že $\forall x \in ®X \exists! y\in ®Y: xAy$    
    \end{definice}

    \begin{definice}[Vlastnosti funkcí]
        Funkce $f: ®X \rightarrow ®Y$ je:
        \begin{itemize}
            \item prostá (injektivní) $≡ \not\exists x, x' \in ®X: x≠x' \land f(x) = f(x')$
            \item na ®Y (surjektivní) $≡ \forall y \in ®Y \exists x \in ®X: f(x) = y$
            \item vzájemně jednoznačná (bijektivní, 1-1 (jedna ku jedné)) $\forall y \in ®Y \exists! x \in ®X: f(x) = y$
        \end{itemize}
    \end{definice}

    \begin{definice}[Vlastnoti relací]
        Relace $R$ na ®X je:
        \begin{itemize}
            \item reflexivní $≡ \forall x \in ®X: xRx$
            \item symetrická $≡ \forall x, y \in ®X: xRy \implies yRx (\Leftrightarrow R = R^{-1})$
            \item antisymetrická $≡ \forall x, y \in ®X: xRy \land yRx \implies x = y$
            \item tranzitivní $≡ \forall x, y, z \in ®X: xRy \land yRz \implies xRz$
        \end{itemize}
    \end{definice}

    \begin{definice}[Ekvivalence]
        Relace se nazývá ekvivalence, pokud je tranzitivní, reflexivní a symetrická.
    \end{definice}

    \begin{definice}[Ekvivalenční třídy]
        $$ R[x] = \{y \in ®X | xRy\} $$
    \end{definice}

    \begin{veta}
        $$ 1) \forall x \in ®X R[x] ≠ \O $$
        $$ 2) \forall x, y \in ®X: R[x] = R[Y] XOR R[x]\cap R[y] = \O $$
        $$ 3) \{R[x]| x \in ®X\} \text{určuje ekvivalenci $R$ jednoznačně} $$ 
        \begin{dukazin}
            1) triviální

            2) Dokážeme: pokud $R[x] \cap R[y] ≠ \O$, pak $R[x] = R[y]$. (Tranzitivita).

            3)
        \end{dukazin}
    \end{veta}

    \begin{definice}[Rozklad množiny]
        Množinový systém $©S \subseteq 2^{®X}$ je rozklad množiny ®X tehdy, když\\
        (R1) $\forall ®A \in ©S: ®A ≠ \O$,\\
        (R2) $\forall ®A, ®B \in ©S: ®A ≠ ®B \implies ®A\cap ®B = \O$,\\
        (R3) $\bigcup_{®A \in ©S} = ®X$.
    \end{definice}

\end{document}
