\documentclass[12pt]{article}					% Začátek dokumentu
\usepackage{../../MFFStyle}					    % Import stylu



\begin{document}
\section{Úvod}
    \begin{poznamka}[Co je diskrétní matematika]
        Protipól matematiky spojité. Souhrnný název pro matematické disciplíny, zabývající se diskrétními objekty.
    \end{poznamka}

    \begin{poznamka}[Co je potřeba]
        Cvičení + zkouška z věcí z přednášky.
    \end{poznamka}

    \begin{poznamka}[literatura]
        Kapitoly z diskrétní matematiky od Matouška.
    \end{poznamka}

    \begin{definice}[Důkaz (neformální)]
        Rozebírání tvrzení na tvrzení, která už jsou zřejmá.
    \end{definice}

    \begin{definice}[Definice (neformální)]
        Definujeme objekty pomocí jednodušších a jednodušších, až axiomů.
    \end{definice}
    
    \begin{definice}[Důkaz sporem]
        Dokážeme $\phi$ tím, že vyvrátíme $\phi$
    \end{definice}

    \begin{definice}[Důkaz matematickou indukcí]
            Dokážeme $\phi(n), \forall n \in ®N$ tak, že dokážeme $\phi(0)\land(\forall n \in ®N)(\phi(n)\implies\phi(n+1))$
    \end{definice}

    \begin{definice}[Dolní a horní celá část]
        $\left\lceil x\right\rceil$ je nejbližší nižší celé číslo k $x$

        $\left\lfloor x\right\rfloor$ je nejbližší vyšší celé číslo k $x$
    \end{definice}

    \begin{definice}[Sčítání mnoha čísel]
        $\sum_{i=13}^n x_i = x_{13} + x_{14} + … + x_n$ = Sčítání $x$ od indexu 13 do indexu $n$

        $\sum_\O = 0$
    \end{definice}

    \begin{definice}[Sčítání mnoha čísel]
        $\prod_{i=13}^n x_i = x_{13} \cdot x_{14} \cdot … \cdot x_n$ = Násobení $x$ od indexu 13 do indexu $n$

        $\prod_\O = 1$
    \end{definice}

    \begin{poznamka}[Klasické množiny]
        ®N ®Z ®Q ®R ®C
    \end{poznamka}

    \begin{poznamka}[Klasické množinové operace]
        $$ x \in ®A $$
        $$ ®A \subseteq ®B $$
        $$ ®A \cap ®B $$
        $$ ®A \cup ®B $$ 
        $$ ®A \setminus ®B $$
        $$ ®A\triangle®B = (®A\setminus ®B)\cup(®B\setminus ®A) = \text{disperze} $$
        $$ 2^®A = ©P(®A) $$
    \end{poznamka}

    \begin{definice}[Uspořádaná dvojice]
        Uspořádaná dvojice je $(x, y)$ nebo $\{\{x\},\{x, y\}\}$.

        Vytváří se např. kartézským součinem $®A\times ®B := \{(a,b)|a\in ®A, b \in ®B\}$.

        Uspořádaná trojice je $(x, y, z) = ((x, y), z) = (x, (y, z))$. Atd. pro n-tice.
    \end{definice}

    \begin{definice}[Relace]
        ®A je relace (binární) mezi množinami ®X a ®Y $≡ ®A \subseteq ®X \times ®Y.$

        ®A je relace (binární) na množině ®X $≡$ mezi ®X a ®X.

        Inverze je relace mezi ®Y a ®X: $R^{-1} := \{(y, x) | (x, y) \in R\}$.

        Skládání $T = R \circ S = \{(x, z)| \exists y: xRy \land ySz\}$

        Diagonála = diagonální relace: $\triangle x := \{(x, x) \in ®X\}$
    \end{definice}

    \begin{definice}[Funkce = zobrazení]
        Funkce z množiny ®X do množiny ®Y je relace $A$ mezi ®X a ®Y taková, že $\forall x \in ®X \exists! y\in ®Y: xAy$    
    \end{definice}

    \begin{definice}[Vlastnosti funkcí]
        Funkce $f: ®X \rightarrow ®Y$ je:
        \begin{itemize}
            \item prostá (injektivní) $≡ \not\exists x, x' \in ®X: x≠x' \land f(x) = f(x')$
            \item na ®Y (surjektivní) $≡ \forall y \in ®Y \exists x \in ®X: f(x) = y$
            \item vzájemně jednoznačná (bijektivní, 1-1 (jedna ku jedné)) $\forall y \in ®Y \exists! x \in ®X: f(x) = y$
        \end{itemize}
    \end{definice}

    \begin{definice}[Vlastnoti relací]
        Relace $R$ na ®X je:
        \begin{itemize}
            \item reflexivní $≡ \forall x \in ®X: xRx$
            \item symetrická $≡ \forall x, y \in ®X: xRy \implies yRx (\Leftrightarrow R = R^{-1})$
            \item antisymetrická $≡ \forall x, y \in ®X: xRy \land yRx \implies x = y$
            \item tranzitivní $≡ \forall x, y, z \in ®X: xRy \land yRz \implies xRz$
        \end{itemize}
    \end{definice}

    \begin{definice}[Ekvivalence]
        Relace se nazývá ekvivalence, pokud je tranzitivní, reflexivní a symetrická.
    \end{definice}

    \begin{definice}[Ekvivalenční třídy]
        $$ R[x] = \{y \in ®X | xRy\} $$
    \end{definice}

    \begin{veta}
        $$ 1) \forall x \in ®X R[x] ≠ \O $$
        $$ 2) \forall x, y \in ®X: R[x] = R[Y] XOR R[x]\cap R[y] = \O $$
        $$ 3) \{R[x]| x \in ®X\} \text{určuje ekvivalenci $R$ jednoznačně} $$ 
        \begin{dukazin}
            1) triviální

            2) Dokážeme: pokud $R[x] \cap R[y] ≠ \O$, pak $R[x] = R[y]$. (Tranzitivita).

            3)
        \end{dukazin}
    \end{veta}

    \begin{definice}[Rozklad množiny]
        Množinový systém $©S \subseteq 2^{®X}$ je rozklad množiny ®X tehdy, když\\
        (R1) $\forall ®A \in ©S: ®A ≠ \O$,\\
        (R2) $\forall ®A, ®B \in ©S: ®A ≠ ®B \implies ®A\cap ®B = \O$,\\
        (R3) $\bigcup_{®A \in ©S} = ®X$.
    \end{definice}

    \begin{definice}[Uspořádání]
        Relace $R$ na množině $®X$ je uspořádání $≡ R$ je reflexivní, antisymetrická a tranzitivní.

        \begin{poznamkain}
            Někdy se říká částečné uspořádání a částečně uspořádaná množina (čum), aby se zdůraznilo, že nemusí být lineární.
        \end{poznamkain}
    \end{definice}

    \begin{definice}[Uspořádaná množina]
        Dvojice $(X, R)$, kde $X$ je množina a $R$ je uspořádání na ní.
    \end{definice}

    \begin{definice}[Porovnatelné prvky a lineární uspořádání]
        $xy \in X$ jsou porovnatelné $≡ xRy \lor yRx$

        Uspořádání $R$ je lineární $≡ \forall x,y \in X$ porovnatelné.
    \end{definice}

    \begin{definice}[Ostrá nerovnost]
        $(X, ≤)$ ČUM $\rightarrow (X, <): x<y≡x≤y \land x≠y$
    \end{definice}

    \begin{definice}[Hasseův diagram]
        \ 
        \begin{poznamkain}
            Splňuje následující:
            1. To, co je nahoře je větší než to, co je dole\\
            2. Nezakreslujeme tranzitivitu\\
        \end{poznamkain}
        \ 
        \begin{definicein}[Bezprostřední předchůdce ($x \triangleleft y$)]
                $x$ je bezprostřední předchůdce $y$ v uspořádání $≤ ≡ x<y \land (\not\exists z: x<z \land z<y)$
        \end{definicein}

        V hasseově diagramu jsou mezi vrcholy (prvky množiny) hrany pouze, pokud dolní vrchol je bezprostředním předchůdcem toho nahoře.
    \end{definice}
    
    \begin{definice}[Nejmenší, minimální, největší a maximální prvek]
        \ 
        \begin{itemize}
            \item $x \in ®X$ je nemenší $≡ \forall y \in ®X: x≤y$
            \item $x \in ®X$ je minimální $≡ \nexists y \in ®X: y<x$
            \item největší a maximální obdobně
        \end{itemize}
    \end{definice}

    \begin{lemma}
        Každá konečná neprázdná ČUM má minimální prvek.
        \begin{dukazin}[Důkazík]
            $x_1 \in ®X$ zvolíme libovolně, pokud $x_1$ není minimální $\exists x_2 < x_1$… $\exists k \in ®N x_k$ je minimální.
        \end{dukazin}
    \end{lemma}

    \begin{definice}[Řetězec]
        Pro $(X, ≤)$ ČUM $A \subseteq X$ je řetězec $≡ \forall a, b \in A: a,b$ jsou porovnatelné.

        Naopak $A \subseteq X$ je antiřetězec (nezávislá množina) $≡ \nexists a, b \in A$ různé a porovnatelné.
    \end{definice}

    \begin{definice}[Délka nejdelšího řetězce]
            $$ \omega(X, ≤) := \text{maximum z délek řetězců („výška uspořádání“)} $$
            $$ \alpha(X, ≤) := \text{maximum z „délek“ (velikostí) antiřetězců („šířka uspořádání“)} $$
    \end{definice}

    \begin{veta}[O dlouhém a Širokém]
        $$ \forall (X, ≤) \text{ČUM}: \alpha(X, ≤)·\omega(X, ≤) ≥ |X| $$ 
        (Neboli buď $\alpha ≥ \sqrt{|X|}$ nebo $\omega ≥ \sqrt{|X|}$.)

        \begin{dukazin}
            Sestrojíme $X_1:= \{x \in X|x \text{je minimální}\}$.

            Když máme $X_1, …, X_i$, $Z_i:= X\setminus \(\bigcup_{j=1}^i x_j\)$. Pokud $Z_i = \O$, tak jsme skončili, jinak $X_{i+1} := \{x\in Z_i | x \text{je minimální v} Z_i\}$.

            Přitom $\forall i$ $X_i$ je antiřetězec, $\{X_1, …, X_k\}$ tvoří rozklad $X$ a $\exists \{r_j \in X_j\}_{j = 1}^k$, $\{r_j\}_{j=1}^k$ je řetězec. ($r_k \in X_k$ zvolíme libovolně, $r_j \notin X_{j-1} \implies \exists r_{j-1} \in X_{j-1}: r_{j-1}<r_j$.)

            $$ |X| = \sum_{i=1}^k |X_i| ≤ k·\max_{1≤i≤k}|X_i| ≤ \omega·\alpha. $$ 

        \end{dukazin}
    \end{veta}

\end{document}
