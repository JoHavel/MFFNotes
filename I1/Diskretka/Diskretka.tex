\documentclass[12pt]{article}					% Začátek dokumentu
\usepackage{../../MFFStyle}					    % Import stylu



\begin{document}
\section{Úvod}
    \begin{poznamka}[Co je diskrétní matematika]
        Protipól matematiky spojité. Souhrnný název pro matematické disciplíny, zabývající se diskrétními objekty.
    \end{poznamka}

    \begin{poznamka}[Co je potřeba]
        Cvičení + zkouška z věcí z přednášky.
    \end{poznamka}

    \begin{poznamka}[literatura]
        Kapitoly z diskrétní matematiky od Matouška.
    \end{poznamka}

    \begin{definice}[Důkaz (neformální)]
        Rozebírání tvrzení na tvrzení, která už jsou zřejmá.
    \end{definice}

    \begin{definice}[Definice (neformální)]
        Definujeme objekty pomocí jednodušších a jednodušších, až axiomů.
    \end{definice}
    
    \begin{definice}[Důkaz sporem]
        Dokážeme $\phi$ tím, že vyvrátíme $\phi$
    \end{definice}

    \begin{definice}[Důkaz matematickou indukcí]
            Dokážeme $\phi(n), \forall n \in ®N$ tak, že dokážeme $\phi(0)\land(\forall n \in ®N)(\phi(n)\implies\phi(n+1))$
    \end{definice}

    \begin{definice}[Dolní a horní celá část]
        $\left\lceil x\right\rceil$ je nejbližší nižší celé číslo k $x$

        $\left\lfloor x\right\rfloor$ je nejbližší vyšší celé číslo k $x$
    \end{definice}

    \begin{definice}[Sčítání mnoha čísel]
        $\sum_{i=13}^n x_i = x_{13} + x_{14} + … + x_n$ = Sčítání $x$ od indexu 13 do indexu $n$

        $\sum_\O = 0$
    \end{definice}

    \begin{definice}[Sčítání mnoha čísel]
        $\prod_{i=13}^n x_i = x_{13} \cdot x_{14} \cdot … \cdot x_n$ = Násobení $x$ od indexu 13 do indexu $n$

        $\prod_\O = 1$
    \end{definice}

    \begin{poznamka}[Klasické množiny]
        ®N ®Z ®Q ®R ®C
    \end{poznamka}

    \begin{poznamka}[Klasické množinové operace]
        $$ x \in ®A $$
        $$ ®A \subseteq ®B $$
        $$ ®A \cap ®B $$
        $$ ®A \cup ®B $$ 
        $$ ®A \setminus ®B $$
        $$ ®A\triangle®B = (®A\setminus ®B)\cup(®B\setminus ®A) = \text{disperze} $$
        $$ 2^®A = ©P(®A) $$
    \end{poznamka}

    \begin{definice}[Uspořádaná dvojice]
        Uspořádaná dvojice je $(x, y)$ nebo $\{\{x\},\{x, y\}\}$.

        Vytváří se např. kartézským součinem $®A\times ®B := \{(a,b)|a\in ®A, b \in ®B\}$.

        Uspořádaná trojice je $(x, y, z) = ((x, y), z) = (x, (y, z))$. Atd. pro n-tice.
    \end{definice}

    \begin{definice}[Relace]
        ®A je relace (binární) mezi množinami ®X a ®Y $≡ ®A \subseteq ®X \times ®Y.$

        ®A je relace (binární) na množině ®X $≡$ mezi ®X a ®X.

        Inverze je relace mezi ®Y a ®X: $R^{-1} := \{(y, x) | (x, y) \in R\}$.

        Skládání $T = R \circ S = \{(x, z)| \exists y: xRy \land ySz\}$

        Diagonála = diagonální relace: $\triangle x := \{(x, x) \in ®X\}$
    \end{definice}

    \begin{definice}[Funkce = zobrazení]
        Funkce z množiny ®X do množiny ®Y je relace $A$ mezi ®X a ®Y taková, že $\forall x \in ®X \exists! y\in ®Y: xAy$    
    \end{definice}

    \begin{definice}[Vlastnosti funkcí]
        Funkce $f: ®X \rightarrow ®Y$ je:
        \begin{itemize}
            \item prostá (injektivní) $≡ \not\exists x, x' \in ®X: x≠x' \land f(x) = f(x')$
            \item na ®Y (surjektivní) $≡ \forall y \in ®Y \exists x \in ®X: f(x) = y$
            \item vzájemně jednoznačná (bijektivní, 1-1 (jedna ku jedné)) $\forall y \in ®Y \exists! x \in ®X: f(x) = y$
        \end{itemize}
    \end{definice}

    \begin{definice}[Vlastnoti relací]
        Relace $R$ na ®X je:
        \begin{itemize}
            \item reflexivní $≡ \forall x \in ®X: xRx$
            \item symetrická $≡ \forall x, y \in ®X: xRy \implies yRx (\Leftrightarrow R = R^{-1})$
            \item antisymetrická $≡ \forall x, y \in ®X: xRy \land yRx \implies x = y$
            \item tranzitivní $≡ \forall x, y, z \in ®X: xRy \land yRz \implies xRz$
        \end{itemize}
    \end{definice}

    \begin{definice}[Ekvivalence]
        Relace se nazývá ekvivalence, pokud je tranzitivní, reflexivní a symetrická.
    \end{definice}

    \begin{definice}[Ekvivalenční třídy]
        $$ R[x] = \{y \in ®X | xRy\} $$
    \end{definice}

    \begin{veta}
        $$ 1) \forall x \in ®X R[x] ≠ \O $$
        $$ 2) \forall x, y \in ®X: R[x] = R[Y] XOR R[x]\cap R[y] = \O $$
        $$ 3) \{R[x]| x \in ®X\} \text{určuje ekvivalenci $R$ jednoznačně} $$ 
        \begin{dukazin}
            1) triviální

            2) Dokážeme: pokud $R[x] \cap R[y] ≠ \O$, pak $R[x] = R[y]$. (Tranzitivita).

            3)
        \end{dukazin}
    \end{veta}

    \begin{definice}[Rozklad množiny]
        Množinový systém $©S \subseteq 2^{®X}$ je rozklad množiny ®X tehdy, když\\
        (R1) $\forall ®A \in ©S: ®A ≠ \O$,\\
        (R2) $\forall ®A, ®B \in ©S: ®A ≠ ®B \implies ®A\cap ®B = \O$,\\
        (R3) $\bigcup_{®A \in ©S} = ®X$.
    \end{definice}

    \begin{definice}[Uspořádání]
        Relace $R$ na množině $®X$ je uspořádání $≡ R$ je reflexivní, antisymetrická a tranzitivní.

        \begin{poznamkain}
            Někdy se říká částečné uspořádání a částečně uspořádaná množina (čum), aby se zdůraznilo, že nemusí být lineární.
        \end{poznamkain}
    \end{definice}

    \begin{definice}[Uspořádaná množina]
        Dvojice $(X, R)$, kde $X$ je množina a $R$ je uspořádání na ní.
    \end{definice}

    \begin{definice}[Porovnatelné prvky a lineární uspořádání]
        $xy \in X$ jsou porovnatelné $≡ xRy \lor yRx$

        Uspořádání $R$ je lineární $≡ \forall x,y \in X$ porovnatelné.
    \end{definice}

    \begin{definice}[Ostrá nerovnost]
        $(X, ≤)$ ČUM $\rightarrow (X, <): x<y≡x≤y \land x≠y$
    \end{definice}

    \begin{definice}[Hasseův diagram]
        \ 
        \begin{poznamkain}
            Splňuje následující:
            1. To, co je nahoře je větší než to, co je dole\\
            2. Nezakreslujeme tranzitivitu\\
        \end{poznamkain}
        \ 
        \begin{definicein}[Bezprostřední předchůdce ($x \triangleleft y$)]
                $x$ je bezprostřední předchůdce $y$ v uspořádání $≤ ≡ x<y \land (\not\exists z: x<z \land z<y)$
        \end{definicein}

        V hasseově diagramu jsou mezi vrcholy (prvky množiny) hrany pouze, pokud dolní vrchol je bezprostředním předchůdcem toho nahoře.
    \end{definice}
    
    \begin{definice}[Nejmenší, minimální, největší a maximální prvek]
        \ 
        \begin{itemize}
            \item $x \in ®X$ je nemenší $≡ \forall y \in ®X: x≤y$
            \item $x \in ®X$ je minimální $≡ \nexists y \in ®X: y<x$
            \item největší a maximální obdobně
        \end{itemize}
    \end{definice}

    \begin{lemma}
        Každá konečná neprázdná ČUM má minimální prvek.
        \begin{dukazin}[Důkazík]
            $x_1 \in ®X$ zvolíme libovolně, pokud $x_1$ není minimální $\exists x_2 < x_1$… $\exists k \in ®N x_k$ je minimální.
        \end{dukazin}
    \end{lemma}

    \begin{definice}[Řetězec]
        Pro $(X, ≤)$ ČUM $A \subseteq X$ je řetězec $≡ \forall a, b \in A: a,b$ jsou porovnatelné.

        Naopak $A \subseteq X$ je antiřetězec (nezávislá množina) $≡ \nexists a, b \in A$ různé a porovnatelné.
    \end{definice}

    \begin{definice}[Délka nejdelšího řetězce]
            $$ \omega(X, ≤) := \text{maximum z délek řetězců („výška uspořádání“)} $$
            $$ \alpha(X, ≤) := \text{maximum z „délek“ (velikostí) antiřetězců („šířka uspořádání“)} $$
    \end{definice}

    \begin{veta}[O dlouhém a Širokém]
        $$ \forall (X, ≤) \text{ČUM}: \alpha(X, ≤)·\omega(X, ≤) ≥ |X| $$ 
        (Neboli buď $\alpha ≥ \sqrt{|X|}$ nebo $\omega ≥ \sqrt{|X|}$.)

        \begin{dukazin}
            Sestrojíme $X_1:= \{x \in X|x \text{je minimální}\}$.

            Když máme $X_1, …, X_i$, $Z_i:= X\setminus \(\bigcup_{j=1}^i x_j\)$. Pokud $Z_i = \O$, tak jsme skončili, jinak $X_{i+1} := \{x\in Z_i | x \text{je minimální v} Z_i\}$.

            Přitom $\forall i$ $X_i$ je antiřetězec, $\{X_1, …, X_k\}$ tvoří rozklad $X$ a $\exists \{r_j \in X_j\}_{j = 1}^k$, $\{r_j\}_{j=1}^k$ je řetězec. ($r_k \in X_k$ zvolíme libovolně, $r_j \notin X_{j-1} \implies \exists r_{j-1} \in X_{j-1}: r_{j-1}<r_j$.)

            $$ |X| = \sum_{i=1}^k |X_i| ≤ k·\max_{1≤i≤k}|X_i| ≤ \omega·\alpha. $$ 

        \end{dukazin}
    \end{veta}

% 26. 10. 2020

    \begin{veta}
        $\# f: N \rightarrow M = m^n, |N| = n, |M| = m, m > 0, n > 0$

        \begin{dukazin}[Indukcí]
            $$ n=1: \# f = m = m^1 $$
            $$ n \rightarrow n+1: f \text{jednoznačně určena} f(x) \text{a} f':N\setminus \{x\} \rightarrow M \implies \# f = m · m^n = m^{n+1} $$ 
        \end{dukazin}
    \end{veta}

    \begin{veta}
        Je-li $N$ $n$-prvková množina, pak $|2^N| = 2^n$.

        \begin{dukazin}
            $$ \text{charakteristická funkce: } A \subseteq N \rightarrow C_A: N \rightarrow \{0, 1\} C_A(x) = 0, x \notin A, C_A(x) = 1, x \in A $$ 
        \end{dukazin}
    \end{veta}

    \begin{veta}
        Nechť $X ≠ \O$ je konečná množina, $©S := \{S \subseteq X| |S| \text{je sudá}\}$, $©L := \{L \subseteq X| |L| \text{je lichá}\}$. Potom $|©S| = |©L| = 2^{n-1}$.
        
        \begin{dukazin}
            Víme, že $©S \cup ©L = 2^X$. Stačí tedy $|©S| = |©L|$. Zvolíme si $a \in X$. Pak $f(S):= S \triangle \{a\}$ je bijekce z ©S do ©L.
        \end{dukazin}
    \end{veta}

    \begin{veta}
        Nechť $N$ je $n$ prvková, $M$ je $m$-prvková. Potom $\# f: N \rightarrow M$ prostých $= m·(m-1)·…·(m-n+1)$.

        \begin{poznamka}[Možná značení]
            $$ \[n\]:= \{0, 1, …, \} $$
            $$ m^{\underline{n}} = \frac{m!}{(m-n)!} \text{($m$ na $n$ klesající)} $$ 
        \end{poznamka}
    \end{veta}

    \begin{poznamka}[Kódování funkcemi]
        \ 
        \begin{itemize}
            \item $X \rightarrow \{0, 1\} … 2^X$
            \item $\{1, 2\} \rightarrow X … (x, y)\in X^2$
            \item $\{1, …, k\} \rightarrow X … \text{uspořádané $k$-tice … $X^k$}$
            \item $®N \rightarrow X$ … nekonečné posloupnosti prvků $X$
            \item permutace na $X$, tj. počet bijekcí nebo počet lineárních uspořádání na konečném $X$ … $|X|!$ ($0!=1$)
        \end{itemize}
    \end{poznamka}

    \begin{definice}[Kombinační číslo]
        Kombinační číslo / binomický koeficient ($n$ nad $k$) je $\binom{n}{k} := \frac{n^{\underline{k}}}{k!} = \frac{n!}{k!(n-k)!}$.
    \end{definice}

    \begin{definice}
        Pro množinu $X$ a $k≥0$ definujeme $\binom{X}{k}:= \{A \subseteq X: |A| = k\}$.
    \end{definice}

    \begin{veta}
        Pro každou množinu $X$ a $k≥0$: $\left|\binom{X}{k}\right| = \binom{|X|}{k}$.
    \end{veta}

    \begin{poznamka}[Vlastnosti kombinačních čísel]
        $$ \binom{n}{0} = \binom{n}{n} = 1 $$ 
        $$ \binom{n}{1} = \binom{n}{n-1} = n $$ 
        $$ \binom{n}{k} = \binom{n}{n-k} $$
        $$ \binom{n}{k} = \binom{n-1}{k} + \binom{n-1}{k-1} \text{ (Lze upočítat / nebo rozdělit na případ vybereme / nevybereme konkrétní prvek.}$$
        $$ \sum_{k=0}^n \binom{n}{k} = 2^n \text{ BV $A=1$, $B=1$} $$ 
        $$ \sum_{k=0}^n (-1)^k \binom{n}{k} = 0 \text{ BV $A=1$, $B=-11$} $$
    \end{poznamka}

    \begin{poznamka}
        Vlastnosti se dají vykoukat v tzv. Pascalově trojúhelníku.
    \end{poznamka}

    \begin{veta}[Binomická]
        $$ (A + B)^n = \sum_{k=0}^n A^k·B^{n-k}·\binom{n}{k} $$

        \begin{dukazin}
            Vybírá se $k$ z $n$ členů, ze kterých bude $A$…
        \end{dukazin}
    \end{veta}

% 2. 11. 2020
    
    \begin{veta}[Princip inkluze a exkluze]
        Pro konečné množiny $A_1-A_n$:
        $$ \left| \bigcup_{i=1}^n \right| = \sum_k^n (-1)^{k+1} \sum_{I \in \binom{\{1, 2, …, n\}}{k}} \left|\bigcap_{i \in I} A_i\right| $$

        Nebo alternativně:
        $$ \left| \bigcup_{i=1}^n A_i \right| = \sum_{\O ≠ I \subseteq \{1, …, n\}} (-1)^{|I| + 1} \left| \bigcap_{i \in I} A_i \right| $$ 

        \begin{dukazin}
            Pro každý prvek $x \in \bigcup_i A_i$ spočítáme příspěvky k levé (vždy 1) a k pravé straně. Nechť $x$ patří právě j množin z $A_1, …, A_n$. Průniky $k$-tic: (1) $k>j$ přispěje 0. (2) $k≤j$ přispěje $(-1)^{k+1} \binom{j}{k}$. Součet toho je alternující řada kombinačních čísel „bez 1“, tedy součet je 1.
        \end{dukazin}

        \begin{dukazin}[Druhý]
            Vyjdeme z
            $$ \prod_{i=1}^n (1+x_i) = \sum_{I \subseteq \{1, …, n\}} \prod_{i\in I} x_i. $$
            Definujeme si charakteristickou funkci a zjistíme, že ch. f. průniku je součin, doplňku je 1-ch. f. původního, sjednocení je doplněk průniku doplňků a velikost je součet ch. funkce. Tedy dosadíme za $x_i$ mínus charakteristické funkce (1 nám vypadla z prázdné podmnožiny):
            $$ 1-c_{\bigcup_i A_i} = \(\sum_{\O≠I\subseteq \{1, …, n\}} (-1)^{|I|}·c_{\bigcap_{i \in I} A_i}\) + 1 $$
            Následně ještě přeformulujeme do velikostí a získáme princip inkluze a exkluze.
        \end{dukazin}
    \end{veta}

    \begin{priklad}[Šatnářka]
        Šatnářka náhodně vydala klobouky gentlemanům. Jaká je pravděpodobnost, že se ani jeden klobouk nedostal k majiteli?

        Tj. $S_n := \{\pi | \pi \text{ permutace na } \{ 1, …, n \}\}$, $\pi(i) = i \implies i$ je pevný bod:
        $$ \text{Š}_n := \{ \pi \in S_n | \nexists i: \pi(i) = i \}. $$ 
        Příklad se tedy ptá na $\frac{\text{Š}_n}{n!}$.

        \begin{reseni}
            Lepší je počítat doplněk: $A := \{\pi \in S_n | \pi \text{ má pevný bod}\}$. Definujeme si $A_i := \{\pi \in S_n| \pi(i)=i\}$. Následně vypozorujeme $A = \bigcap_i A_i$. Očividně $|A_i| = (n-1)!$, $|A_i \cup A_j| = (n-2)!$ ($i≠j$), …

            $$ |A| = \left| \bigcup_{i=1}^n \right| = \sum_{k=1}^n (-1)^{k+1}\sum_{I \in \binom{\{1, …, n\}}{k}} \left| \bigcap_{i \in I} A_i \right| = \sum_{k=1}^n (-1)^{k+1}\sum_{I \in \binom{\{1, …, n\}}{k}} (n-k)! = \sum_{k=1}^n (-1)^{k+1} \binom{n}{k}(n-k)! = \sum_{k=1}^n (-1)^{k+1} \frac{n!}{k!} $$
            $$ |A| = n! \sum_{k=1}^n \frac{(-1)^{k+1}}{k!} = n!(\frac{1}{1!} - \frac{1}{2!} + \frac{1}{3!} - … + \frac{(-1)^{n+1}}{n!}) $$
            $$ \text{Š}_n = |A| ≐ n! \frac{1}{e} $$ 
        \end{reseni}
    \end{priklad}

\section{Odhady}
    \begin{priklady}
        $$ 2^{n-1} ≤ n! ≤ n^n $$ 
        $$ n^{n/2} ≤ n! ≤ \(\frac{n+1}{2}\)^n $$
        $$ *\(\frac{n}{e}\)^n ≤ n! ≤ en·\(\frac{n}{e}\)^n  $$ 
        $$ ** n! \sim \(\frac{n}{e}\)^n·\sqrt{2\pi n} $$
        $$ \(\frac{n}{k}\)^k ≤ \binom{n}{k} ≤ n^k $$
        $$ *\binom{n}{k} ≤ \(\frac{en}{k}\)^k $$ 
        $$ \frac{4^n}{2n+1} ≤ \binom{2n}{n} ≤ 4^n $$ 
        $$ * \frac{4^n}{2\sqrt{n}} ≤ \binom{2n}{n} ≤ \frac{4^n}{\sqrt{2n}} $$ 
    \end{priklady}

% 9. 11. 2020

\section{Grafy}
    \begin{definice}[Graf, vrcholy, hrany]
        Graf je uspořádaná dvojice $(V, E)$, kde: $V$ je konečná neprázdná množina vrcholů (vertices) a $E \subseteq \binom{V}{2}$ je množina hran (edges).
    \end{definice}

    \begin{poznamka}[Rozšíření]
        Orientované, se smyčkami, multigrafy, nekonečné.
    \end{poznamka}

    \begin{priklady}
        Úplný graf ($K_n$): $V(K_n) := \{1, …, n\}$ a $E(K_n) := \binom{V(K_n)}{2}$.

        Prázdný graf ($E_n$): $V(E_n) := \{1, …, n\}$ a $E(E_n) := \O$.

        Cesta ($P_n$): $V(P_n) := \{0, 1, …, n\}$ a $E(P_n) := \{\{i, i+1\}|0≤i<n\}$.
        
        Kružnice ($C_n$): $V(C_n) := \{0, 1, …, n-1\}$ a $E(C_n) := \{\{i, i+1 \mod n\}|0≤i≤n\}$.

        Úplný bipartitní graf ($K_{m, n}$): $V(K_n) := \{a_1, …, a_n\} \cup \{b_1, …, b_n\}$ a $E(K_n) := \{\{a_i, b_j\}|1≤i≤m, 1≤j≤n\}$.
    \end{priklady}

    \begin{definice}[Bipartitní graf]
        Graf $G$ je bipartitní $≡ \exists$ rozklad množiny $V(G)$ na $X, Y$ (= partity) tak, že $E(G) \subseteq \{\{x, y\} | x\in X, y \in Y\}$. (Lze zapsat i jako $\forall e \in e(G): |e\cap X| = 1$.)
    \end{definice}

    \begin{definice}[Isomorfismus grafů]
        Grafy $G$ a $H$ jsou isomorfní (značme $G \cong H$) $≡ \exists f: V(G) \rightarrow V(H)$ bijekce tak, že $\forall u, v \in V(G):(\{u, v\} \in E(g) \Leftrightarrow \{f(u), f(v)\} \in E(H))$.
    \end{definice}

    \begin{poznamka}[K nahlédnutí]
        Na libovolné množině grafů je $\cong$ ekvivalence.
    \end{poznamka}

    \begin{definice}[Stupeň vrcholu]
        Stupeň vrcholu $v$ v grafu $G$ je $\deg_G(v) := \left|\{u \in V(G)|\{u, v\} \in E(G)\}\right|$.
    \end{definice}

    \begin{definice}[Regulární graf]
        Graf je $k$-regulární (pro $k \in ®N$) $≡ \forall u \in V(G): \deg_G(u) = k$.

        Graf $G$ je regulární $≡ \exists k: G$ je $k$-regulární.
    \end{definice}

    \begin{definice}[Skóre grafu]
        Skóre grafu $G$ je posloupnost stupňů všech vrcholů (až na uspořádání).
    \end{definice}

    \begin{veta}
        Pro každý graf $(V, E)$ platí:
        $$ \sum_{v\in V} \deg(v) = 2·|E| $$ 
    \end{veta}

    \begin{dusledek}[Princip sudosti]
        $\sum_v \deg(v)$ je sudé číslo $\implies (\#v\in V \text{ lichého stupně})$ je sudý.
    \end{dusledek}

    \begin{veta}[O skóre]
        Posloupnost $D = d_1 ≤ … ≤ d_n$ pro $n≥2$ je skóre grafu $\Leftrightarrow$ $D' = d'_1, …, d'_{n-1}$ je skóre grafui a $0≤d_n≤n-1$. ($d'_i = d_i$ pro $i < n-d_n$ a $d'_i = d_i - 1$ pro $i ≥ n-d_n$.)

        \begin{dukazin}
            $(\Leftarrow)$ nechť $G'$ je graf se skóre $D'$ a vrcholy $v_1, …, v_{n-1}$ tak, že $\forall i \deg_{G'}(v_i)=d'_i$. Vytvořím $G$ doplněním vrcholu $v_n$ a hran $\{v_i, v_n\}$ pro $i \in \{n-d_n, …, n-1\}$. $G$ má skóre $D$.

            $(\implies)$ Lemma: Nechť ©G je množina všech grafů se skóre $D$, $©G ≠ \O$. Potom $\exists G \in ©G: \{v_n, v_i\}\in E(G)$ pro všechna $i \in \{n-d_n, …, n-1\}$.

            Důkaz lemmatu: (Kdyby $d_n = n-1$, pak zřejmě každý $G \in ©G$ splňuje lemma.) Pro $G \in ©G$ definujeme $j(G) := \max\{j|\{v_j, v_n\} \notin E(G)\}$ (kdyby $j(g) = n-d_ni-1$, pak jsme vyhráli, jinak $G$ nesplňuje lemma). Najdeme $G \in ©G$, jehož $j(G)$ je minimální. Pokračujeme sporem: Kdyby $j(G) > n - d_n - 1$, musí $\exists i < j: \{v_i, v_n\} \in E(G)$. Následně chceme ukázat, že $\exists k: \{v_i, v_k\}\notin E(G) \land \{v_j, v_k\} \in E(G)$, to ukážeme na základě toho, že posloupnost je seřazena, tedy $d_i ≤ d_j$ a vrchol $v_i$ je spojen minimálně s jedním vrcholem, se kterým není spojené $v_j$ ($v_n$). Upravíme graf $G$ na $G_\lightning: V(G_\lightning) := V(G), E(G_\lightning) := E(G) \cup \{\{v_i, v_k\}, \{v_j, v_n\}\} \setminus \{\{v_i, v_n\}, \{v_j, v_k\}\}$. Ale jelikož jsme vrcholům odstranili stejný počet hran, jako přidali, $G_\lightning \in ©G$. Navíc zřejmě $j(G_\lightning) < j(G)$, $\lightning$.
        \end{dukazin}
    \end{veta}

\end{document}
