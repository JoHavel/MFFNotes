\documentclass[12pt]{article}					% Začátek dokumentu
\usepackage{../../MFFStyle}					    % Import stylu



\begin{document}
\section{Úvod}
    \begin{definice}[Matice]
            Reálná matice typu $ m \times n$ je obdélníkové schéma (tabulka) reálných čísel. Prvek na pozici $(i,\ j)$ matice $A$ značíme $a_{ij}$ nebo $A_{ij}$. A $i$-tý řádek matice $A$ značíme $A_{i*}$ a $j$-tý řádek matice $A$ značíme $A_{*j}$.
    \end{definice}

    \begin{definice}[Vektor]
        Reálný $n$-rozměrný aritmetický sloupcový vektor (standardní) je matice typu $n \times 1$ a řádkový $1 \times n$.
    \end{definice}

    \begin{definice}[Soustava lineárních rovnic]
        Lineární = neznámé jsou v 1. mocnině.\\
        Soustava = více rovnic.\\
        Rovnice výraz z neznámých (bez absolutního členu) a koeficientů rovný konstantě.
        \begin{definicein}[Řešení]
            Řešením rozumíme každý vektor hodnot neznámých vyhovující všem rovnicím.
        \end{definicein}

        \begin{definicein}[Matice soustavy]
            Matice soustavy je matice koeficientů u neznámých.\\
            Rozšířená matice soustavy je matice soustavy „následována“ vektorem hodnot konstant jednotlivých rovnic.
        \end{definicein}

        \begin{poznamkain}[Geometrický význam]
            Průsečík $n$ „přímek“ v $n$ rozměrném prostoru
        \end{poznamkain}
    \end{definice}

    \begin{definice}[Elementární řádkové úpravy]
        \ 
        \begin{itemize}
            \item Vynásobení řádku nenulovým reálným číslem.
            \item Přičtení jednoho řádku k druhému.
            \item Výměna dvou řádků. (Není elementární, protože jde vytvořit pomocí prvních dvou.)
        \end{itemize}

        \begin{tvrzeniin}
            Elementární řádkové operace zachovávají množinu řešení soustavy.
            \begin{dukazin}
                Elementární úpravou neztratíme žádné řešení, protože pokud je $x$ řešením před úpravou, je i po úpravě. A naopak ho lze invertovat, takže žádné řešení ani nepřibude.
            \end{dukazin}
        \end{tvrzeniin}
    \end{definice}

    \begin{definice}[Odsupňovaný tvar matice REF]
        Matice $A \in \R^{m\times n}$ je v řádkově odstupňovaném tvaru, pokud existuje $r$ takové, že platí: řádky $1,…,r$ (tzv. bazické) jsou nenulové (obsahují alespoň 1 nenulový prvek), řádky $r+1,…,m$ jsou nulové, a navíc označíme-li jako $p_i=min{j; a_{ij}\neq 0}$ (tzv. pivot) pozici prvního nenulového prvku v $i$-tém řádku, tak platí: $p_1<p_2<\cdots<p_r$.
        \begin{prikladyin}
            Matice, které jsou, a matice, které nejsou.
        \end{prikladyin}
    \end{definice}

    \begin{definice}[Hodnost matice]
            Počet nenulových řádků po převodu do odstupňovaného tvaru (nebo libovolného s maximálním počtem nulových řádků) značený $\rank(A)$.
    \end{definice}

    Dále jsme dělali Gaussovu eliminaci (nemá řešení ($\rank(A) \neq \rank(A|b)$), má 1 řešení ($\rank(A|b) = n$), má mnoho řešení (pak bazické proměnné vyjádřím pomocí nebazických)).

    \begin{definice}[Redukovaný odstupňovaný tvar matice RREF]
        Matice v odstupňovaném tvaru je v redukovaném OT, jestliže $\forall 0≤i≤r, i \in \N: a_{ip_i} = 1 \land \forall i>x \in \R a_{xp_i} = 0$.
    \end{definice}

    \begin{poznamka}
        Tento tvar je jednoznačný.
    \end{poznamka}

    \begin{definice}[Rovnost matic]
        Dvě matice se rovnají, pokud mají stejné rozměry a stejné prvky na stejných souřadnicích.
    \end{definice}

    \begin{definice}[Součet matic]
        Pro součet musí mít matice stejné rozměry a poté sčítáme po složkách.

        \begin{poznamkain}[Vlastnosti]
            Komutativita (pokud jsou prvky matice komutativní).
        \end{poznamkain}
    \end{definice}

    \begin{definice}[Násobení skalárem]
        Násobíme po složkách.
    \end{definice}

    \begin{definice}[Součin matic]
        Nechť $A \in \R^{m\times n}$ a $B \in \R^{n \times o}$ jsou matice. Potom matice $C \in \R^{m \times o}$ definovaná jako $c_{ij} = a_{i*}\cdot b_{*j}$ je jejich součinem.

        \begin{poznamkain}[Vlastnosti]
            Komutativita neplatí.

            Asociativita, distributivita zleva a distributivita z prava platí. Stejně tak „asociativita“ násobení skalárem.
        \end{poznamkain}
    \end{definice}

    \begin{definice}[Transpozice]
        Buď $A\in\R^{m \times n}$. Potom $A^T\in \R^{n \times m}$ definována jako $(A^T)_{ij} = a_{ji}$ je transponovaná matice $A$.
        \begin{poznamka}[Vlastnosti]
            Je sama sobě inverzním zobrazením. Distributivita pro všechny operace (pozor u násobení je antisymetrická).

            $$ (A^T)^T = A $$
            $$ (A + B)^T = A^T + B^T $$
            $$ (\alpha A)^T = \alpha A^T $$
            $$ (AB)^T = B^TA^T $$
        \end{poznamka}
    \end{definice}

    \begin{definice}[Symetrická a antisymetrická matice]
        Matice $A$ je symetrická, pokud $A = A^T$, a antisymetrická $A = -A^T$.

        \begin{poznamka}[Vlastnosti]
            Symetrické matice jsou uzavřené na součet, ale na součin ne.
        \end{poznamka}
    \end{definice}

    \begin{definice}[Jednotkový vektor]
        $e_j$ definovaný jako $\(e_j\)_j = 1$ a $\forall i\neq j \(e_j\)_i = 0$ je $j$-tý jednotkový vektor.
        \begin{poznamkain}[Vlastnosti]
            $$  Ae_i = A_{*i} $$
            $$  e_i^T = A_{i*} $$
        \end{poznamkain}
    \end{definice}

    \begin{definice}[Skalární součin vektorů]
        $ u\cdot v = u^Tv $ je skalární součin vektorů $u$ a $v$.

        $ uv^T $ je ? součin vektorů $u$ a $v$
    \end{definice}

% 15. 10. 2020

    \begin{poznamka}[Zápis SLR jako maticové násovení]
        SLR lze zapsat jako $Ax = b$.
    \end{poznamka}

    \begin{poznamka}[Matice a lineární zobrazení $x \rightarrow Ax$]
        Je užitečné se na matici $A \in ®R^{m \times n}$ jako na určité zobrazení z $®R^n$ do $®R^m$ definované předpisem $x \rightarrow Ax$.

        Na řešení SLR se lze pak dívat jako na vzor $b$ v zobrazení dané $A$.

        Zároveň na maticový součin se lze dívat na skládání $(BA)x = B(Ax)$. (Základní motivace, aby se součin definoval tak, jak je.)
    \end{poznamka}

    TODO?

    \begin{definice}[Regulární matice]
        Buď $A \in ®R^{n\times n}$. Pak následující tvrzení jsou ekvivalentní:
        \begin{enumerate}
            \item $A$ je regulární,
            \item $\RREF(A) = ©I_n$,
            \item $\rank(A) = n$
            \item pro nějaké $b \in ®R^n$ má soustava $Ax = b$ jediné řešení,
            \item pro všechna $b \in ®R^n$ má soustava $Ax = b$ jediné řešení.
        \end{enumerate}

        Matice $A$ nesplňující tvrzení je singulární.
    \end{definice}

    \begin{tvrzeni}[Uzavřenost na součin]
        Buďte $A, B \in ®R^{n\times n}$ regulární matice. Pak $AB$ je také regulární.
        \begin{dukazin}
            Buď $x$ řešení soustavy $ABx = 0$. Chceme ukázat, že $x$ musí být nulový vektor. Z předchozího tvrzení $\forall y Ay=0$ má jediné řešení. Zároveň $\forall y Bx = y$ má jediné řešení.
        \end{dukazin}
    \end{tvrzeni}

    \begin{tvrzeni}
        Je-li alespoň jedna z matic $A, B \in ®R^{n \times n}$, pak $AB$ je také singulární.

        \begin{dukazin}
            Je-li matice $B$ singulární, pak $Bx = 0$ pro nějaké $x ≠ 0$. Z toho ale plyne $(AB)x = A(Bx) = A(0)=0$, tedy i $AB$ je singulární.

            Nyní předpokládejme, že matice $B$ je regulární, tedy matice $A$ singulární a existuje $y≠0$ takové, že $Ay = 0$. Z regularity matice $B$ existuje $x≠0$ takové, že $Bx=y$. Celkem dostáváme $(AB)x = A(Bx) = Ay = 0$, tedy $AB$ je singulární.
        \end{dukazin}
    \end{tvrzeni}

    \begin{definice}[Matice elementárních úprav]
        Elementární úpravy jdou reprezentovat násobením tzv. elementární maticí zleva: $E_i(\alpha)$ jako násobení řádku $i$ číslem $\alpha$ je jednotková matice s $\alpha$ místo $i$-té jedničky. $E_{ij}$ jako prohození řádků $i, j$ je jednotková matice s prohozeným $i$-tým a $j$-tým řádkem, …

        Tyto matice jsou regulární.
    \end{definice}

    \begin{veta}
        Buď $A \in ®R^{m\times n}$. Pak $\RREF(A) = QA$ pro nějakou regulární matici $Q \in ®R^{m \times m}$.
        \begin{dukazin}
            $\RREF(A)$ získáme aplikací konečně mnoha elementárních řádkových úprav a součin regulárních matic je regulární matice.
        \end{dukazin}
    \end{veta}

    \begin{tvrzeni}
        Každá regulární matice $A \in ®R^{n\times n}$ se dá vyjádřit jako součin konečně mnoha elementárních matic.
        \begin{dukazin}
            Elementární úpravy lze invertovat elementárními úpravami, tedy i inverze úprav regulární matice na ©I je regulární (a ©I je také regulární).
        \end{dukazin}
    \end{tvrzeni}

    \begin{definice}[Inverzní matice]
        Buď $A \in ®R^{n\times n}$. Pak $A^{-1}$ je inverzní maticí k A, pokud splňuje $AA^{-1} = A^{-1}A = ©I_n$.

        \begin{prikladyin}
            $©I_n^{-1} = ©I_n$, $0_n^{-1}$ neexistuje.
        \end{prikladyin}
    \end{definice}

    \begin{veta}[O existenci inverzní matice]
        Buď $A \in ®R^{n\times n}$. Je-li $A$ regulární, pak k ní existuje inverzní matice a je určená jednoznačně. Naopak, existuje-li $A^{-1}$, pak $A$ je regulární.

        \begin{dukazin}
                Existence: $A$ je regulární, tedy soustava $Ax = e_j$ má řešení $x_j$ pro každé $j$. Ukážeme, že $A^{-1} = (x_1 | x_2 | … | x_n)$ je hledaná inverze. (Porovnáním po sloupcích: $(AA^{-1})_{*j} = Ax_j = e_j = ©I_{*j}$. Komutativní výraz dokážeme $A(A^{-1}A - ©I) = AA^{-1}A - A = ©IA - A = 0$, $A^{-1}A - ©I$ je vektor, který je jednoznačně určen tím, že $A$ je regulární.)

                Jednoznačnost. Nechť pro nějakou matici $B$ platí $AB = BA = ©I$. Pak
                $$ B = B©I = B(AA^{-1}) = (BA)A^{-1} = ©IA^{-1} - A^{-1}. $$ 

                Naopak. Nechť pro $A$ existuje inverzní matice. Buď $x$ řešení soustavy $Ax = 0$. Pak
                $$ x = ©Ix = (A^{-1}A)x = A^{-1}(Ax) = A^{-1}0 = 0, $$ 
                tedy $A$ je regulární.
        \end{dukazin}
    \end{veta}

    \begin{tvrzeni}[Vlastnosti inverzní matice]
        Je-li $A$ regulární, pak $A^T$ je regulární.
        \begin{dukazin}
            Je-li $A$ regulární, pak existuje inverze a platí $AA^{-1} = A^{-1}A = ©I_n$. Po transponování všech stran dostaneme
            $$(AA^{-1})^T = (A^{-1}A)^T = ©I_n^T,$$
            neboli
            $$ (A^{-1})^TA^T = A^T(A^{-1})^T = ©I_n. $$ 
            Matice $A^T$ má inverzi a je tudíž regulární. (Navíc $(A^T)^{-1} = (A^{-1})^T$, občas se značí $A^{-T}$).
        \end{dukazin}
    \end{tvrzeni}

    \begin{veta}[Jedna rovnost stačí]
        Buďte $A, B \in ®R^{n\times n}$. Je-li $BA = I_n$, pak obě matice $A, B$ jsou regulární a navzájem k sobě inverzní, to jest $B = A^{-1}$ a $A = B^{-1}$.
        \begin{dukazin}
                Regularita vyplývá z dřívějšího tvrzení vzhledem k regularitě $©I_n$. Tudíž existují inverze $A^{-1}$, $B^{-1}$. Odvodíme
                $$ B = B©I_n = B(AA^{-1}) = (BA)A^{-1} = ©I_nA^{-1} = A^{-1}. $$
                Úplně stejně druhá rovnost.
        \end{dukazin}
    \end{veta}

    \begin{poznamka}[Výpočet inverzní matice]
        Důkaz věty ukázal návod: $j$-tý sloupec $A^{-1}$ je řešením soustavy $Ax = e_j$.
    \end{poznamka}

    \begin{veta}[Výpočet inverzní matice]
        Buď $A \in ®R^{n\times n}$. Je-li $\RREF(A|©I_n) = (I_n|B)$, pak $B = A^{-1}$, jinak je $A$ singulární.
        \begin{dukazin}
            Je-li $\RREF(A| ©I_n) = (©I_n | B)$, potom existuje regulární $Q$ tak, že $(©I_n|B) = Q(A|©I_n)$. Po roztržení na dvě části $©I_n = QA$ tj. $Q = A^{-1}$ a $B = Q©I_n = Q = A^{-1}$.

            Pokud není toho tvaru, pak z definice $A$ není regulární.
        \end{dukazin}
    \end{veta}

    \begin{tvrzeni}[Vlastnosti inverzní matice]
        Buďte $A, B \in ®R^{n\times n}$ regulární. Pak:
        \begin{enumerate}
            \item $(A^{-1})^{-1} = A$
            \item $(A^{-1})^T = (A^T)^{-1}$
            \item $(\alpha A)^{-1} = \frac{1}{\alpha} A^{-1}\ …\ (\alpha≠0)$
            \item $(AB)^{-1} = B^{-1}A^{-1}$
        \end{enumerate}
        \begin{dukazin}
            1., 2. triviální, 3. vynásobím $A\alpha$, 4. přezávorkuji.
        \end{dukazin}
        \begin{poznamkain}
            Pro $(A+B)^{-1}$ žádný jednoduchý vzoreček není.
        \end{poznamkain}
    \end{tvrzeni}
    
    \begin{poznamka}[Inverzní matice a soustava rovnic]
        Buď $Q$ regulární. Pak soustava $Ax = b$ je ekvivalentní s (QA)x = (Qb).
        \begin{dukazin}
            Žádné řešení neztratíme, zpět se dostaneme přednásobením $Q^{-1}$ zleva.
        \end{dukazin}
    \end{poznamka}

    \begin{veta}[Soustava rovnic a inverzní matice]
        Buď $A\in ®R^{n\times n}$ regulářní. Pak řešení soustavy $Ax = b$ je dáno vzorcem $A^{-1}b = x$.
    \end{veta}

    \begin{poznamka}[Inverzní matice - geometrie]
        Buď $A \in ®R^{n\times n}$ regulární matice. Pro každé $y \in ®R^n$ existuje právě jedno $x \in ®R^n$ takové, že $Ax = y$, zobrazení je tedy bijekcí.
    \end{poznamka}

    \begin{poznamka}
        Skládání zobrazení nám může dát i vhled do toho, jak funguje inverze součinu matic.
    \end{poznamka}

\end{document}
