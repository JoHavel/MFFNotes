\documentclass[12pt]{article}					% Začátek dokumentu
\usepackage{../../MFFStyle}					    % Import stylu



\begin{document}
\section{Úvod}
    \begin{definice}[Matice]
            Reálná matice typu $ m \times n$ je obdélníkové schéma (tabulka) reálných čísel. Prvek na pozici $(i,\ j)$ matice $A$ značíme $a_{ij}$ nebo $A_{ij}$. A $i$-tý řádek matice $A$ značíme $A_{i*}$ a $j$-tý řádek matice $A$ značíme $A_{*j}$.
    \end{definice}

    \begin{definice}[Vektor]
        Reálný $n$-rozměrný aritmetický sloupcový vektor (standardní) je matice typu $n \times 1$ a řádkový $1 \times n$.
    \end{definice}

    \begin{definice}[Soustava lineárních rovnic]
        Lineární = neznámé jsou v 1. mocnině.\\
        Soustava = více rovnic.\\
        Rovnice výraz z neznámých (bez absolutního členu) a koeficientů rovný konstantě.
        \begin{definicein}[Řešení]
            Řešením rozumíme každý vektor hodnot neznámých vyhovující všem rovnicím.
        \end{definicein}

        \begin{definicein}[Matice soustavy]
            Matice soustavy je matice koeficientů u neznámých.\\
            Rozšířená matice soustavy je matice soustavy „následována“ vektorem hodnot konstant jednotlivých rovnic.
        \end{definicein}

        \begin{poznamkain}[Geometrický význam]
            Průsečík $n$ „přímek“ v $n$ rozměrném prostoru
        \end{poznamkain}
    \end{definice}

    \begin{definice}[Elementární řádkové úpravy]
        \ 
        \begin{itemize}
            \item Vynásobení řádku nenulovým reálným číslem.
            \item Přičtení jednoho řádku k druhému.
            \item Výměna dvou řádků. (Není elementární, protože jde vytvořit pomocí prvních dvou.)
        \end{itemize}

        \begin{tvrzeniin}
            Elementární řádkové operace zachovávají množinu řešení soustavy.
            \begin{dukazin}
                Elementární úpravou neztratíme žádné řešení, protože pokud je $x$ řešením před úpravou, je i po úpravě. A naopak ho lze invertovat, takže žádné řešení ani nepřibude.
            \end{dukazin}
        \end{tvrzeniin}
    \end{definice}

    \begin{definice}[Odsupňovaný tvar matice REF]
        Matice $A \in \R^{m\times n}$ je v řádkově odstupňovaném tvaru, pokud existuje $r$ takové, že platí: řádky $1,…,r$ (tzv. bazické) jsou nenulové (obsahují alespoň 1 nenulový prvek), řádky $r+1,…,m$ jsou nulové, a navíc označíme-li jako $p_i=min{j; a_{ij}\neq 0}$ (tzv. pivot) pozici prvního nenulového prvku v $i$-tém řádku, tak platí: $p_1<p_2<\cdots<p_r$.
        \begin{prikladyin}
            Matice, které jsou, a matice, které nejsou.
        \end{prikladyin}
    \end{definice}

    \begin{definice}[Hodnost matice]
            Počet nenulových řádků po převodu do odstupňovaného tvaru (nebo libovolného s maximálním počtem nulových řádků) značený $\rank(A)$.
    \end{definice}

    Dále jsme dělali Gaussovu eliminaci (nemá řešení ($\rank(A) \neq \rank(A|b)$), má 1 řešení ($\rank(A|b) = n$), má mnoho řešení (pak bazické proměnné vyjádřím pomocí nebazických)).

    \begin{definice}[Redukovaný odstupňovaný tvar matice RREF]
        Matice v odstupňovaném tvaru je v redukovaném OT, jestliže $\forall 0≤i≤r, i \in \N: a_{ip_i} = 1 \land \forall i>x \in \R a_{xp_i} = 0$.
    \end{definice}

    \begin{poznamka}
        Tento tvar je jednoznačný.
    \end{poznamka}

    \begin{definice}[Rovnost matic]
        Dvě matice se rovnají, pokud mají stejné rozměry a stejné prvky na stejných souřadnicích.
    \end{definice}

    \begin{definice}[Součet matic]
        Pro součet musí mít matice stejné rozměry a poté sčítáme po složkách.

        \begin{poznamkain}[Vlastnosti]
            Komutativita (pokud jsou prvky matice komutativní).
        \end{poznamkain}
    \end{definice}

    \begin{definice}[Násobení skalárem]
        Násobíme po složkách.
    \end{definice}

    \begin{definice}[Součin matic]
        Nechť $A \in \R^{m\times n}$ a $B \in \R^{n \times o}$ jsou matice. Potom matice $C \in \R^{m \times o}$ definovaná jako $c_{ij} = a_{i*}\cdot b_{*j}$ je jejich součinem.

        \begin{poznamkain}[Vlastnosti]
            Komutativita neplatí.

            Asociativita, distributivita zleva a distributivita z prava platí. Stejně tak „asociativita“ násobení skalárem.
        \end{poznamkain}
    \end{definice}

    \begin{definice}[Transpozice]
        Buď $A\in\R^{m \times n}$. Potom $A^T\in \R^{n \times m}$ definována jako $(A^T)_{ij} = a_{ji}$ je transponovaná matice $A$.
        \begin{poznamka}[Vlastnosti]
            Je sama sobě inverzním zobrazením. Distributivita pro všechny operace (pozor u násobení je antisymetrická).

            $$ (A^T)^T = A $$
            $$ (A + B)^T = A^T + B^T $$
            $$ (\alpha A)^T = \alpha A^T $$
            $$ (AB)^T = B^TA^T $$
        \end{poznamka}
    \end{definice}

    \begin{definice}[Symetrická a antisymetrická matice]
        Matice $A$ je symetrická, pokud $A = A^T$, a antisymetrická $A = -A^T$.

        \begin{poznamka}[Vlastnosti]
            Symetrické matice jsou uzavřené na součet, ale na součin ne.
        \end{poznamka}
    \end{definice}

    \begin{definice}[Jednotkový vektor]
        $e_j$ definovaný jako $\(e_j\)_j = 1$ a $\forall i\neq j \(e_j\)_i = 0$ je $j$-tý jednotkový vektor.
        \begin{poznamkain}[Vlastnosti]
            $$  Ae_i = A_{*i} $$
            $$  e_i^T = A_{i*} $$
        \end{poznamkain}
    \end{definice}

    \begin{definice}[Skalární součin vektorů]
        $ u\cdot v = u^Tv $ je skalární součin vektorů $u$ a $v$.

        $ uv^T $ je ? součin vektorů $u$ a $v$
    \end{definice}


\end{document}
