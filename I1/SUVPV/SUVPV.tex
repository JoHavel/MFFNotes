\documentclass[12pt]{article}					% Začátek dokumentu
\usepackage{../../MFFStyle}					    % Import stylu



\begin{document}
\section{Úvod}
    Bude se pracovat v Matlabu, v moodle je skupina.

% 7. 10. 2020

    \begin{poznamka}[Úkoly počítačového vidění]
        Detekovat, najít a určit věci (tváře, jestli se smějí, nádory, znaky jako znaky na SPZ, biometrika jako oko, tvář, podpis, budovy, lidi, auta, …) na obrázku
    \end{poznamka}

    \begin{poznamka}[Rozpoznávání objektu tradičním náhledem]
        pixely -> určení feature (např. rozdělení oblastí podle tvarů, důležitých bodů, barev, umístění hledaného objektu) expertem -> učení klasifikátoru -> rozpoznávání
    \end{poznamka}
    
    \begin{definice}[Feature vektor]
        Univerzální převedení obrázku do důležitých věcí.    

        Měl by být invariantní (měl by být stejný při rotaci a škálování), diskriminační (dobře rozdělovat objekty), kompaktní (co nejmenší)
    \end{definice}

    \begin{definice}[Rozpoznávání]
        Feature vektory tvoří prostor, kde se algoritmus naučí najít hranici, která odděluje objekty, co jsou nějaké a co jsou jinaké.
    \end{definice}

    \begin{poznamka}[Klasifikace může být za pomocí]
        Statistiky -- Bayesova teorie rozhodování

        Pravidel -- Rozhodovací strom

        Metriky -- Technika nejbližšího souseda, diskriminační analýza?, podpůrné vektorové stroje?

        Biologické inspirace -- Neuronové sítě
    \end{poznamka}

    \begin{definice}[Učení s učitelem]
        Na training setu víme správné odpovědi.
    \end{definice}

    \begin{definice}[Naivní Bayesův klasifikátor]
        Vychází z podmíněné pravděpodobnosti na základě věcí, co víme.
        \begin{prikladyin}
            Rozeznávání falešného úsměvu. 91,3\%.
        \end{prikladyin}
    \end{definice}

    \begin{definice}[Rozhodovací stromy]
        Pravidly určíme, kterou větví se vydáme. Výhodou je, že nepotřebujeme koncept vzdálenosti.
        \begin{prikladyin}
            Rozpoznávání, co se děje na videu (např. vražda). 70\% - 100\%.
        \end{prikladyin}
    \end{definice}
    
    \begin{definice}[K nejbližších sousedů]
        Podíváme se na nejbližší známé objekty a rozhodneme se podle nich.
        \begin{prikladyin}
            Čtení znaků. 99\% čísla, 94\% velká a 89\% malá písmena.
        \end{prikladyin}
    \end{definice}

    \begin{definice}[Lineární klasifikace]
        Rozdělení prostoru nadrovinou. Zlepšením je tzv. podpůrné vektorové stroje? (support vector machines)
        \begin{prikladyin}
            Rozpoznávání lidí a věku. 66,9 - 80\% lidi, 63,8 - 75,7\% věk.
        \end{prikladyin}
    \end{definice}

    \begin{definice}[Umělé neuronové sítě]
        Sítě z neuronů, které jsou velmi jednoduše simulovány, viz moje maturitní práce. (Na GitHubu pod uživatelem JoHavel).
        \begin{prikladyin}
            Rozpoznávání tváře. 90\%. (80\% na portrétech.)
        \end{prikladyin}
    \end{definice}

    \begin{poznamka}[Hluboké učení]
        pixely -> učení se včetně feature -> rozpoznávání

        \begin{prikladyin}
            AlexNet (top 5 error cca. 16\%)

            Každoročně se pořádá ILSVRC (Imagenet Large Scale Visual Recognition Challenge), kde už se dosáhlo méně než 4\% chyby (152 vrstev NN)…
        \end{prikladyin}
    \end{poznamka}

    \begin{poznamka}[Kombinovaný přístup]
        pixely -> featury nalezené NN -> trénování klasifikátoru -> rozpoznávání

        Hluboké učení se ale zdá účinnější.
    \end{poznamka}

\end{document}
