\documentclass[12pt]{article}					% Začátek dokumentu
\usepackage{../../MFFStyle}					    % Import stylu



\begin{document}
\section*{Organizační úvod}
    \begin{poznamka}[Za co bude zápočet]
            Najít / vymyslet 2 úlohy (poslat je cca týden před cvičením (na e-mail ook@ucw.cz)). K tomu aktivně řešit zadaných soutěží / úloh.
    \end{poznamka}

    \begin{poznamka}[ICPC]
        Jedna z nejstarších programovacích soutěží ($\approx$ 1977).

        Předkolo CTU open (na ČVUTu, 27.-28. 11.?) a středoevropské kolo CERC (Praha, ???).
    \end{poznamka}

    \begin{poznamka}[Pravidla]
        Tříčlenné týmy, jeden počítač, 8-10 úloh, 5 hodin.

        Průběžné vyhodnocování a výsledky.

        Možnost používat písemné materiály.

        Jazyky: C++, C, Java, Python, Kotlin.

        Vstup ze stdin, výstup na stdout. Nesmí se používat soubory (ani stderr) a thready. A knihovny stylu síťové.

        Paměťový limit $\approx$ 1 GB.
    \end{poznamka}

    \begin{poznamka}[Hodnocení]
        Počet vyřešených úloh, poté kumulativní čas za každou vyřešenou úlohu (nejsou částečné body) plus 20 minut za každé neúspěšné odevzdání vyřešené úlohy.
    \end{poznamka}

    \begin{poznamka}[Strategie]
        Vyřeší se cca 8. úloh, je dobré se snažit řešit úlohy od nejlehčích (které je potřeba rychle najít).
    \end{poznamka}

    \begin{poznamka}[Využití počítače]
        Psát program bez chyb (maximalizovat programování, minimalizovat ladění), ale chce to rovnováhu mezi přehledností a rychlostí. Hledání chyb / ladění na výtisku.
    \end{poznamka}
    
\end{document}
