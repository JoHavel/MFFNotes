\documentclass[12pt]{article}                   % Začátek dokumentu
\usepackage{../../MFFStyle}                     % Import stylu

\begin{document}

\begin{priklad}[jehly]
	Spočtěte, kolikrát se dané jehly vyskytují v daném seně (podslova v textu). Zajímají mne počty každé jehly zvlášť.

	\begin{reseni}
		Použijeme algoritmus Aho-Corasick. Místo toho, abychom nálezy vypisovali, si v každém stavu automatu (vrcholu trie) budeme pamatovat, kolikrát jsme ho navštívili. Po projití sena si v každém stavu odpovídajícímu přímo výslednému slovu pořídíme počítadlo výskytů. Následně spočítáme výskyty tak, že projdeme automat (trii) a v každém stavu se podíváme na všechny zkratkové hrany a přičteme počet navštívení aktuálního stavu do počtu nálezu stavu, kam vedou. (Pak ještě přičteme počet navštívený těch daných stavů do počtu jejich nálezů).

		Tak jsme neudělali nic víc, než použili Aho-Corasick bez výpisu (tj. časová složitost součet délek jehel a sena) a pak prošli stavy + zkratkové hrany (tj. časová složitost součet délek jehel), nakonec vypíšeme počty jednotlivých jehel (tj. časová složitost počet jehel (případně i součet jejich délek, pokud projdeme)), tedy celková časová složitost je součet délek jehel a délky sena.
	\end{reseni}
\end{priklad}

\end{document}
