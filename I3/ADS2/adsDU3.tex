\documentclass[12pt]{article}                   % Začátek dokumentu
\usepackage{../../MFFStyle}                     % Import stylu

\begin{document}

\begin{priklad}[podposloupnost]
	Spočtěte, kolikrát se daná jehla vyskytne v daném seně (dva řetězce, seno mnohem delší) jako podposloupnost (možnost proložení jinými znaky). Například v \verb|BARBARAR| je \verb|BAR| 9krát. 

	\begin{reseni}[Dynamickým programováním]
		Pro každou pozici v jehle si budeme pamatovat, kolika způsoby jsme schopni se do ní dostat (v prvních $n$ písmenech sena). Tyto hodnoty pak pro $n+1$ písmen sena přepočítáme jednoduše tak, že k hodnotám na místech, kde je $n+1$-ní písmeno sena, přičteme hodnotu na předchozí pozici (-1. pozice má vždy hodnotu 1) (musíme přičítat od poslední pozice v jehle). Po průchodu senem už stačí jen vypsat hodnotu na konci jehly.

		Tedy máme časovou složitost součin délky jehly a délky sena (pokud si pro každé písmeno uchováme všechny pozice, kde se vyskytuje, tak se to dá srazit na součet délky jehly a součin počtu výskytů nejčastějšího písmene a délky sena).
	\end{reseni}
\end{priklad}

\end{document}
