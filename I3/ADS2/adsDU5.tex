\documentclass[12pt]{article}                   % Začátek dokumentu
\usepackage{../../MFFStyle}                     % Import stylu

\begin{document}

\begin{priklad}[kluby]
	Na kolejích jsou různé kluby, koleják může být součástí libovolného počtu klubů. Jste postaveni před úlohu vybrat z každého klubu předsedu a místopředsedu a to tak, aby každý koleják byl vybrán za nejvýš jeden klub a pro něj do právě jedné funkce.

	\begin{reseni}[Přes toky]
		Ze zdroje povedeme hrany o váze 2 do vrcholů odpovídajících klubům (každý klub má 2 „významné“ kolejáky). Z každého klubu povedeme hrany libovolné (tj. třeba jednotkové) váhy do vrcholů odpovídajícím kolejákům. Z každého kolejáka pak vedeme hranu o váze 1 (můžeme každého kolejáka „zvýznamnit“ pouze 1). Následně najdeme největší tok. Tím jsme přiřadili kolejáky funkcím (pokud po hraně z klubu do kolejáka, tento koleják je předsedou/místopředsedou tohoto klubu).

		Nechť klubů je $k$ a kolejáků $l$. Při hledání maximálního toku nám trvá $O(k·l)$ vyhledání cesty (hran je nejvýše $k + l + k·l$) a zlepšit tok můžeme nejvýše $2k$ krát, protože pak už je každá role obsazena. Složitost je tedy $O\(k^2·l\)$.
	\end{reseni}
\end{priklad}

\end{document}
